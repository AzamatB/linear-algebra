% Chapter 3, Section 1 _Linear Algebra_ Jim Hefferon
%  http://joshua.smcvt.edu/linearalgebra
%  2001-Jun-11
\chapter{Maps Between Spaces}

%\begin{comment}
\section{Isomorphisms}
\index{isomorphism|(}
In the examples following the definition of a vector space
we expressed the idea that some spaces are ``the same'' as others.
For instance, the space of two-tall column vectors and
the space of two-wide row vectors
are not equal because their 
elements\Dash column vectors and row vectors\Dash are 
not equal, but we have 
the idea that these spaces differ only in how their elements appear.
We will now make this intuition precise.

This section illustrates a common aspect of a mathematical investigation.
With the help of some examples, we've gotten an idea.
We will next give a formal definition and then
we will produce some results backing our contention 
that the definition captures the idea.
We've seen this happen already, 
for instance in the first section of the Vector Space chapter.
There, the study of linear systems
led us to consider collections closed under linear combinations.
We defined such a collection as a vector space and
we followed it with some supporting results.

Of course, that definition wasn't an end point, 
instead it led to new insights such as the idea of a basis.
Here too, after producing a definition and supporting it,
we will get two surprises (pleasant ones).
First, we will find that 
the definition applies to some unforeseen, and interesting, cases.
Second, the study of the definition will lead to new ideas.
In this way, our investigation will build momentum.










\subsection{Def{}inition and Examples}
We start with two examples
that suggest the right definition. 

\begin{example}  \label{exam:TwoWideIsoTwoTall}
The 
space of two-wide row vectors and
the space of two-tall column vectors
are ``the same'' in that if we associate the vectors 
that have the same components, e.g.,
\begin{equation*}
   \rowvec{1 &2} \quad\longleftrightarrow\quad \colvec[r]{1 \\ 2}
\end{equation*}
then this correspondence preserves the operations,
for instance this addition
\begin{equation*}
  \rowvec{1 &2}+\rowvec{3 &4}=\rowvec{4 &6}
  \quad\longleftrightarrow\quad
  \colvec[r]{1 \\ 2}+\colvec[r]{3 \\ 4}=\colvec[r]{4 \\ 6}
\end{equation*}
and this scalar multiplication.
\begin{equation*}
  5\cdot\rowvec{1  &2}=\rowvec{5 &10}
  \quad\longleftrightarrow\quad
  5\cdot\colvec[r]{1 \\ 2}=\colvec[r]{5 \\ 10}
\end{equation*}
More generally stated, under the correspondence
\begin{equation*}
  \rowvec{a_0  &a_1}
  \quad\longleftrightarrow\quad
  \colvec{a_0 \\ a_1}
\end{equation*} 
both operations are preserved:
\begin{equation*}
  \rowvec{a_0  &a_1}+\rowvec{b_0  &b_1}=\rowvec{a_0+b_0 &a_1+b_1}
  \hfill\longleftrightarrow\hfill
  \colvec{a_0 \\ a_1}+\colvec{b_0  \\ b_1}=\colvec{a_0+b_0  \\ a_1+b_1}
\end{equation*}
and
\begin{equation*}
 r\cdot\rowvec{a_0 &a_1}=\rowvec{ra_0 &ra_1}
  \quad\longleftrightarrow\quad
 r\cdot\colvec{a_0  \\ a_1}=\colvec{ra_0  \\ ra_1}
\end{equation*}
(all of the variables are real numbers).
\end{example}

\begin{example} \label{exam:PolyTwoIsoRThree}  
Another two spaces we can think of as ``the same'' are       
\( \polyspace_2 \), the space of quadratic polynomials,  and \( \Re^3 \). 
A natural correspondence is this.
\begin{equation*}
  a_0+a_1x+a_2x^2
  \quad\longleftrightarrow\quad
  \colvec{a_0 \\ a_1 \\ a_2}
%  \qquad
  \tag*{\mbox{(e.g., $1+2x+3x^2\,\longleftrightarrow\,\colvec[r]{1 \\ 2 \\ 3}$)}}
\end{equation*}
The structure is preserved:~corresponding elements add in a corresponding way
\begin{equation*}
    \mbox{\begin{tabular}{r}
        \( a_0+a_1x+a_2x^2 \)  \\
     +\,\,\( b_0+b_1x+b_2x^2 \)  \\ \hline
        \( (a_0+b_0)+(a_1+b_1)x+(a_2+b_2)x^2 \)
     \end{tabular} }
    \quad\longleftrightarrow\quad
    \colvec{a_0 \\ a_1 \\ a_2}
    +\colvec{b_0 \\ b_1 \\ b_2}
    =\colvec{a_0+b_0 \\ a_1+b_1 \\ a_2+b_2}
\end{equation*}
and scalar multiplication corresponds also.
\begin{equation*}
  r\cdot(a_0+a_1x+a_2x^2)=
    (ra_0)+(ra_1)x+(ra_2)x^2
  \quad\longleftrightarrow\quad
  r\cdot\colvec{a_0 \\ a_1 \\ a_2}
  =\colvec{ra_0 \\ ra_1 \\ ra_2}
\end{equation*}
\end{example}

\begin{definition}
\label{def:Isomorphism}
An \definend{isomorphism}\index{isomorphism!definition}%
\index{vector space!isomorphism}
between two vector spaces $V$ and $W$
is a map \( \map{f}{V}{W} \) that
\begin{enumerate}
  \item is a correspondence:\index{correspondence}
     \( f \) is one-to-one and onto;\footnotemark % keep fn in minipage
                   %\appendrefs{one-to-one and onto maps}
  \item \definend{preserves structure:}\index{function!structure preserving}
   if \( \vec{v}_1,\vec{v}_2\in V \) then
   \begin{equation*}
      f(\vec{v}_1+\vec{v}_2)=f(\vec{v}_1)+f(\vec{v}_2)
   \end{equation*}
   and if \( \vec{v}\in V \) and \( r\in\Re \) then
   \begin{equation*}
         f(r\vec{v})=r\,f(\vec{v})
   \end{equation*}
\end{enumerate}
(we write \( V\isomorphicto W \), read ``\( V \) is isomorphic to \( W \)'',
when such a map exists).
\end{definition}\footnotetext{More information on one-to-one and onto maps 
                              is in the appendix.}

\noindent (``Morphism''\index{morphism} means map, 
so ``isomorphism'' means a map expressing sameness.)

\begin{example}  \label{ex:CheckMapIsIso}
The vector space
\(  G=\set{c_1\cos\theta+c_2\sin\theta\suchthat c_1,c_2\in\Re} \)
of functions of \( \theta \) is isomorphic to the vector space
\( \Re^2 \) under this map.
\begin{equation*}
  c_1\cos\theta+c_2\sin\theta\mapsunder{f}\colvec{c_1 \\ c_2}
\end{equation*}
We will check this by going through the conditions in the definition.

We will first verify condition~(1), that the map is a 
correspondence between the sets underlying the spaces.

To establish that $f$ is one-to-one, 
we must prove that \( f(\vec{a})=f(\vec{b}) \) only when \( \vec{a}=\vec{b} \).
If
\begin{equation*}
   f(a_1\cos\theta+a_2\sin\theta)=f(b_1\cos\theta+b_2\sin\theta)
\end{equation*}
then, by the definition of $f$,
\begin{equation*}
  \colvec{a_1 \\ a_2}=\colvec{b_1 \\ b_2}
\end{equation*}
from which we can conclude that \( a_1=b_1 \) and \( a_2=b_2 \)
because column vectors are equal only when they have equal components.
We've proved that \( f(\vec{a})=f(\vec{b}) \) implies 
that \( \vec{a}=\vec{b} \),
which shows that \( f \) is one-to-one.

To check that $f$ is
onto we must check that any member of the codomain \( \Re^2 \)
is the image of some member of the domain $G$.
But that's clear\Dash any
\begin{equation*}
   \colvec{x \\ y}\in \Re^2
\end{equation*}
is the image under $f$ of \( x\cos\theta+y\sin\theta\in G \).

Next we will verify condition~(2), that $f$ preserves structure. 

This computation shows that \( f \) preserves addition.
\begin{multline*}
  f\bigl(\,(a_1\cos\theta+a_2\sin\theta)
      +(b_1\cos\theta+b_2\sin\theta)\,\bigr)                     \\
 \begin{aligned}
  &=f\bigl(\,(a_1+b_1)\cos\theta+(a_2+b_2)\sin\theta\,\bigr)   \\
  &=\colvec{a_1+b_1 \\ a_2+b_2}   \\
  &=\colvec{a_1 \\ a_2}+\colvec{b_1 \\ b_2}   \\
  &=f(a_1\cos\theta+a_2\sin\theta)+f(b_1\cos\theta+b_2\sin\theta)
 \end{aligned}
\end{multline*}
A similar computation shows that $f$ preserves scalar multiplication.
\begin{align*}
  f\bigl(\,r\cdot(a_1\cos\theta+a_2\sin\theta)\,\bigr)
  &=f(\,ra_1\cos\theta+ra_2\sin\theta\,)   \\
  &=\colvec{ra_1 \\ ra_2}   \\
  &=r\cdot\colvec{a_1 \\ a_2}   \\
  &=r\cdot\, f(a_1\cos\theta+a_2\sin\theta)
\end{align*}

With that,
conditions (1) and~(2) are verified, 
so we know that $f$ is an isomorphism
and we can say that the spaces are isomorphic $G\isomorphicto\Re^2$.
\end{example}

\begin{example} \label{ex:LinComboThreeIsoPTwo}
Let \( V \) be the space 
\( \set{c_1x+c_2y+c_3z\suchthat c_1,c_2,c_3\in\Re} \)
of linear combinations of three variables $x$, $y$, and $z$, 
under the natural addition and scalar multiplication operations. 
Then $V$ is isomorphic to $\polyspace_2$, the space of quadratic 
polynomials.

To show this we will produce an isomorphism map.
There is more than one possibility; for instance, here are four.
\begin{center}
  \begin{tabular}{c}
    $c_1x+c_2y+c_3z$
  \end{tabular}  \quad
  \begin{tabular}{rl}
      $\mapsunder{f_1}$    &$c_1+c_2x+c_3x^2$  \\
      $\mapsunder{f_2}$    &$c_2+c_3x+c_1x^2$  \\
      $\mapsunder{f_3}$    &$-c_1-c_2x-c_3x^2$  \\
      $\mapsunder{f_4}$    &$c_1+(c_1+c_2)x+(c_1+c_3)x^2$
  \end{tabular}
\end{center}
The first map is the more natural correspondence
in that it just carries the coefficients over. 
However, below
we shall verify that the second one is an isomorphism,
to underline that there are isomorphisms
other than just the obvious one 
(showing that $f_1$ is an isomorphism is 
\nearbyexercise{exer:NatMapAlsoIso}).

To show that $f_2$ is one-to-one, we will prove that if
$f_2(c_1x+c_2y+c_3z)=f_2(d_1x+d_2y+d_3z)$
then $c_1x+c_2y+c_3z=d_1x+d_2y+d_3z$.
The assumption that $f_2(c_1x+c_2y+c_3z)=f_2(d_1x+d_2y+d_3z)$
gives, by the definition of $f_2$, 
that $c_2+c_3x+c_1x^2=d_2+d_3x+d_1x^2$.
Equal polynomials have equal coefficients, so 
$c_2=d_2$, $c_3=d_3$, and $c_1=d_1$.
Thus $f_2(c_1x+c_2y+c_3z)=f_2(d_1x+d_2y+d_3z)$ implies that
$c_1x+c_2y+c_3z=d_1x+d_2y+d_3z$ and therefore $f_2$ is one-to-one.

The map $f_2$ is onto because any member $a+bx+cx^2$ of the codomain 
is the image of some member of the domain,
namely it is the image of $cx+ay+bz$.
For instance, $2+3x-4x^2$ is $f_2(-4x+2y+3z)$.

The computations for structure preservation are like those 
in the prior example.
This map preserves addition
\begin{multline*}
  f_2\bigl((c_1x+c_2y+c_3z)
        +(d_1x+d_2y+d_3z)\bigr)       \\
 \begin{aligned}
  &=f_2\bigl((c_1+d_1)x+(c_2+d_2)y+(c_3+d_3)z\bigr)   \\
  &=(c_2+d_2)+(c_3+d_3)x+(c_1+d_1)x^2   \\
  &=(c_2+c_3x+c_1x^2)+(d_2+d_3x+d_1x^2)   \\
  &=f_2(c_1x+c_2y+c_3z)+f_2(d_1x+d_2y+d_3z)
 \end{aligned}
\end{multline*}
and scalar multiplication.
\begin{align*}
  f_2\bigl(r\cdot(c_1x+c_2y+c_3z)\bigr)
  &=f_2(rc_1x+rc_2y+rc_3z)   \\
  &=rc_2+rc_3x+rc_1x^2   \\
  &=r\cdot(c_2+c_3x+c_1x^2)   \\
  &=r\cdot\, f_2(c_1x+c_2y+c_3z)
\end{align*}
Thus $f_2$ is an isomorphism and we write $V\isomorphicto\polyspace_2$.
\end{example}

We are sometimes interested in an isomorphism of a space with itself,
called an \definend{automorphism}\index{automorphism}%
\index{isomorphism!of a space with itself}.
An identity map is an automorphism.
The next two examples show that there are others.

\begin{example} \label{exam:RigidPlaneMapsAutos}
A \definend{dilation}\index{dilation}\index{automorphism!dilation}
map $\map{d_s}{\Re^2}{\Re^2}$ that multiplies all vectors by a nonzero
scalar $s$ is an automorphism of $\Re^2$.
\begin{center} \small
  \includegraphics{ch3.14}                                
\end{center}
A \definend{rotation\/}\index{rotation (or turning)}%
\index{automorphism!rotation}
or \definend{turning map}\index{turning map} $\map{t_{\theta}}{\Re^2}{\Re^2}$
that rotates all vectors through an angle $\theta$ is an automorphism.
\begin{center} \small
  \includegraphics{ch3.15}                                
\end{center}
A third type of automorphism of $\Re^2$ is a map 
$\map{f_\ell}{\Re^2}{\Re^2}$ that \definend{flips}
or \definend{reflects}\index{reflection (or flip) about a line}%
\index{automorphism!reflection}
all vectors over a line $\ell$ through the origin.
\begin{center} \small
  \includegraphics{ch3.16}                                
\end{center}
See \nearbyexercise{exer:RigidPlaneMapsAutos}.
\end{example}

\begin{example} \label{ex:OORThreeToFourLinCom}
Consider the space $\polyspace_5$ of polynomials of degree $5$
or less and the map $f$ that sends a polynomial $p(x)$ to $p(x-1)$.
For instance, under this map $x^2\mapsto (x-1)^2=x^2-2x+1$
and $x^3+2x\mapsto (x-1)^3+2(x-1)=x^3-3x^2+5x-3$.
This map is an automorphism of this space;
the check is \nearbyexercise{exer:PolyToPolyLinSubst}.

This isomorphism of $\polyspace_5$ with itself does more than just tell us
that the space is ``the same'' as itself.
It gives us some insight into the space's structure.
For instance, below is shown a family of parabolas, graphs of members of 
$\polyspace_5$.
Each has a vertex at $y=-1$, and  
the left-most one has zeroes at $-2.25$ and $-1.75$, 
the next one has zeroes at $-1.25$ and $-0.75$, etc.
\begin{center}
  \includegraphics{ch3.13}                                
\end{center}
Geometrically, the substitution of \( x-1 \) for \( x \) in any function's
argument shifts its graph to the right by one.
Thus, $f(p_0)=p_1$ and 
$f$'s action is to shift all of the parabolas to the right by one.
Notice that the picture before $f$ is applied is the same as the 
picture after $f$ is applied, because while each parabola moves to the 
right, another one comes in from the left to take its place.
This also holds true for cubics, etc.
So the automorphism $f$ gives us the insight that $P_5$  
has a certain
horizontal-homogeneity; this space looks the same near $x=1$ as near $x=0$. 
\end{example}


As described in the preamble to this section, we will next 
produce some results supporting the
contention that the definition of isomorphism above  
captures our intuition of vector spaces being the same.

Of course the definition itself is persuasive:~a vector space 
consists of two components, a set and some structure, 
and the definition simply requires
that the sets correspond and that the structures correspond also.
Also persuasive are the examples above.
In particular, \nearbyexample{exam:TwoWideIsoTwoTall}, which
gives an isomorphism between the space of two-wide row vectors and 
the space of two-tall column vectors,
dramatizes our intuition that isomorphic spaces are
the same in all relevant respects.
Sometimes people say, where \( V\isomorphicto W \), that ``\( W \) is just
\( V \) painted green''\Dash any differences are merely cosmetic.

Further support for the definition, in case it is needed, 
is provided by the following results that, taken together, suggest that
all the things of interest in a vector space
correspond under an isomorphism.
Since we studied vector spaces to study linear combinations,
``of interest'' means ``pertaining to linear combinations''.
Not of interest is the way that the vectors are presented
typographically (or their color!\spacefactor1000).

As an example, although the definition of isomorphism doesn't explicitly
say that the zero vectors must correspond, 
it is a consequence of that definition.

\begin{lemma}       \label{le:IsoSendsZeroToZero}
An isomorphism maps a zero vector to a zero vector.
\end{lemma}

\begin{proof}
Where \( \map{f}{V}{W} \) is an isomorphism, fix any \( \vec{v}\in V \).
Then \( f(\zero_V)=f(0\cdot\vec{v})=0\cdot f(\vec{v})=\zero_W \). 
\end{proof}

The definition of isomorphism requires that sums of two vectors correspond and
that so do scalar multiples.
We can extend that to say that all linear combinations correspond.

\begin{lemma}       \label{le:PresStructIffPresCombos}
For any map \( \map{f}{V}{W} \) between vector spaces
these statements are equivalent.
\begin{tfae}
   \item \( f \) preserves structure\index{function!structure preserving}
     \begin{equation*}
        f(\vec{v}_1+\vec{v}_2)=f(\vec{v}_1)+f(\vec{v}_2)
         \quad\text{and}\quad
        f(c\vec{v})=c\,f(\vec{v})
     \end{equation*}
   \item \( f \) preserves linear combinations of two vectors
     \begin{equation*}
        f(c_1\vec{v}_1+c_2\vec{v}_2)=c_1f(\vec{v}_1)+c_2f(\vec{v}_2)
     \end{equation*}
   \item \( f \) preserves linear combinations of any finite number of 
     vectors
     \begin{equation*}
        f(c_1\vec{v}_1+\dots+c_n\vec{v}_n)=
           c_1f(\vec{v}_1)+\dots+c_nf(\vec{v}_n)
     \end{equation*}
\end{tfae}
\end{lemma}

\begin{proof}
Since the implications \mbox{$\text{(3)}\!\implies\!\text{(2)}$} and 
\mbox{$\text{(2)}\!\implies\!\text{(1)}$}
are clear, we need only show that \mbox{$\text{(1)}\!\implies\!\text{(3)}$}.
Assume statement~(1).
We will prove statement~(3) by induction on the number of summands $n$.

The one-summand base case, that
\( f(c\vec{v}_1)=c\,f(\vec{v}_1) \), is covered by the assumption of
statement~(1).

For the inductive step assume that statement~(3) holds whenever 
there are \( k \) or fewer summands, that is, whenever
$n=1$, or $n=2$, \ldots, or $n=k$.
Consider the $k+1$-summand case.
The first half of (1) gives
\begin{equation*}
  f(c_1\vec{v}_1+\dots+c_k\vec{v}_k+c_{k+1}\vec{v}_{k+1})
  =f(c_1\vec{v}_1+\dots+c_k\vec{v}_k)+f(c_{k+1}\vec{v}_{k+1})
\end{equation*}
by breaking the sum along the final `$+$'.
Then the inductive hypothesis lets us break up the $k$-term sum.
\begin{equation*}
  =f(c_1\vec{v}_1)+\dots+f(c_k\vec{v}_k)+f(c_{k+1}\vec{v}_{k+1})
\end{equation*}
Finally, the second half of statement~(1) gives
\begin{equation*}
  =c_1\,f(\vec{v}_1)+\dots+c_k\,f(\vec{v}_k)+c_{k+1}\,f(\vec{v}_{k+1})
\end{equation*}
when applied $k+1$ times.
\end{proof}

In addition to adding to the intuition that the definition of isomorphism
does indeed preserve the things of interest in a vector space, 
that lemma's second item is an especially handy way of
checking that a map preserves structure.

We close with a summary.
The material in this section augments the chapter on Vector Spaces.
There, after giving the definition of a vector space, 
we informally looked at what different things can happen.
Here, we defined the relation 
`\( \isomorphicto \)' between vector spaces and 
we have argued that it
is the right way to split the collection of vector spaces
into cases because it preserves the features of interest in a vector
space\Dash in particular, it preserves linear combinations.
That is, we have now said precisely what we mean by `the same', 
and by `different', and so we have precisely classified the vector spaces.


\begin{exercises}
  \recommended \item \label{exer:PThreeIsoRFour}
   Verify, using \nearbyexample {ex:CheckMapIsIso} as a model,
   that the two correspondences given before the definition 
   are isomorphisms.
   \begin{exparts*}
      \partsitem \nearbyexample{exam:TwoWideIsoTwoTall}
      \partsitem \nearbyexample{exam:PolyTwoIsoRThree}
    \end{exparts*}
    \begin{answer}
     \begin{exparts}
     \partsitem Call the map $f$. 
       \begin{equation*}
         \rowvec{a &b} \mapsunder{f} \colvec{a \\ b}
       \end{equation*}
       It is one-to-one because if $f$ sends two members of the domain to 
       the same image, that is, if
       $f\left(\rowvec{a &b}\right)=f\left(\rowvec{c &d}\right)$,
       then the definition of $f$ gives that
       \begin{equation*}
          \colvec{a \\ b}=\colvec{c \\ d}
       \end{equation*}
       and since column vectors are equal only if they have equal components,
       we have that $a=c$ and that $b=d$. 
       Thus, if $f$ maps two row vectors from the domain to the same column
       vector then the two row vectors are equal:
       $\rowvec{a &b}=\rowvec{c &d}$.

       To show that $f$ is onto we must show that any member of the codomain
       $\Re^2$ is the image under $f$ of some row vector.
       That's easy;
       \begin{equation*}
          \colvec{x \\ y} 
       \end{equation*}
       is $f\left(\rowvec{x &y}\right)$.

       The computation for preservation of addition is this.
       \begin{equation*}
         f\left(\rowvec{a &b}+\rowvec{c &d}\right) 
           = f\left(\rowvec{a+c &b+d}\right)           
           = \colvec{a+c \\ b+d}           
           = \colvec{a \\ b}+\colvec{c \\ d}         
           = f\left(\rowvec{a &b}\right)+f\left(\rowvec{c &d}\right)
       \end{equation*}
       The computation for preservation of scalar multiplication is similar.
       \begin{equation*}
         f\left(r\cdot\rowvec{a &b}\right) 
           = f\left(\rowvec{ra &rb}\right)          
           = \colvec{ra \\ rb}          
           = r\cdot\colvec{a \\ b}          
           = r\cdot f\left(\rowvec{a &b}\right)
       \end{equation*}
     \partsitem Denote the map from \nearbyexample{exam:PolyTwoIsoRThree} by
      $f$. 
      To show that it is one-to-one,
      assume that \( f(a_0+a_1x+a_2x^2)=f(b_0+b_1x+b_2x^2) \).
      Then by the definition of the function, 
      \begin{equation*}
        \colvec{a_0 \\ a_1 \\ a_2}
        =\colvec{b_0 \\ b_1 \\ b_2}
      \end{equation*}
      and so \( a_0=b_0 \) and \( a_1=b_1 \) and \( a_2=b_2 \).
      Thus \( a_0+a_1x+a_2x^2=b_0+b_1x+b_2x^2 \), and consequently \( f \)
      is one-to-one.

      The function $f$ is onto because there is a polynomial sent to
      \begin{equation*}
        \colvec{a \\ b \\ c}
      \end{equation*}
      by \( f \), namely, \( a+bx+cx^2 \).

      As for structure,
      this shows that $f$ preserves addition
      \begin{align*}
        f\left(\,(a_0+a_1x+a_2x^2)+(b_0+b_1x+b_2x^2)\,\right)
        &=f\left(\,(a_0+b_0)+(a_1+b_1)x+(a_2+b_2)x^2\,\right)       \\
        &=\colvec{a_0+b_0 \\ a_1+b_1 \\ a_2+b_2}                    \\
        &=\colvec{a_0 \\ a_1 \\ a_2} +\colvec{b_0 \\ b_1 \\ b_2}   \\
        &=f(a_0+a_1x+a_2x^2)+f(b_0+b_1x+b_2x^2)
      \end{align*}
      and this shows 
      \begin{align*}
        f(\,r(a_0+a_1x+a_2x^2)\,)
        &=f(\,(ra_0)+(ra_1)x+(ra_2)x^2\,)    \\
        &=\colvec{ra_0 \\ ra_1 \\ ra_2}     \\
        &=r\cdot\colvec{a_0 \\ a_1 \\ a_2}     \\
        &=r\,f(a_0+a_1x+a_2x^2)
      \end{align*}   
      that it preserves scalar multiplication. 
    \end{exparts} 
   \end{answer}
  \recommended \item 
    For the map
    \( \map{f}{\polyspace_1}{\Re^2} \) given by
    \begin{equation*}
       a+bx\mapsunder{f}\colvec{a-b \\ b}
    \end{equation*}
    Find the image of each of these elements of the domain.
    \begin{exparts*}
      \partsitem \( 3-2x \)
      \partsitem \( 2+2x \)
      \partsitem \( x \)
    \end{exparts*}
    Show that this map is an isomorphism.
    \begin{answer}
       These are the images.
       \begin{exparts*}
        \partsitem \( \colvec[r]{5 \\ -2} \)
        \partsitem \( \colvec[r]{0 \\ 2} \)
        \partsitem \( \colvec[r]{-1 \\ 1} \)
      \end{exparts*}

      To prove that $f$ is one-to-one, assume that it maps two linear
      polynomials to the same image $f(a_1+b_1x)=f(a_2+b_2x)$.
      Then
      \begin{equation*}
        \colvec{a_1-b_1 \\ b_1}=\colvec{a_2-b_2 \\ b_2}
      \end{equation*}
      and so, since column vectors are equal only when their components are
      equal, $b_1=b_2$ and $a_1=a_2$.
      That shows that the two linear polynomials are equal, and so $f$ is
      one-to-one.

      To show that $f$ is onto, note that this member of the codomain 
      \begin{equation*}
         \colvec{s \\ t}
      \end{equation*}
      is the image of this member of the domain $(s+t)+tx$.

      To check that $f$ preserves structure,
      we can use item~(2) of \nearbylemma{le:PresStructIffPresCombos}.
      \begin{align*}
        f\left(c_1\cdot (a_1+b_1x)+c_2\cdot (a_2+b_2x)\right)
        &=f\left((c_1a_1+c_2a_2)+(c_1b_1+c_2b_2)x\right)                \\
        &=\colvec{(c_1a_1+c_2a_2)-(c_1b_1+c_2b_2) \\ c_1b_1+c_2b_2}       \\  
        &=c_1\cdot\colvec{a_1-b_1 \\ b_1}+c_2\cdot\colvec{a_2-b_2 \\ b_2}  \\
        &=c_1\cdot f(a_1+b_1x)+c_2\cdot f(a_2+b_2x)
      \end{align*}
     \end{answer}
  \item \label{exer:NatMapAlsoIso}
    Show that the natural map $f_1$ from 
    \nearbyexample{ex:LinComboThreeIsoPTwo} 
    is an isomorphism.
    \begin{answer}
      To verify it is one-to-one, assume that
       $f_1(c_1x+c_2y+c_3z)=f_1(d_1x+d_2y+d_3z)$.
      Then $c_1+c_2x+c_3x^2=d_1+d_2x+d_3x^2$
      by the definition of $f_1$.
      Members of $\polyspace_2$ are equal only when they have the same
      coefficients, so this implies that 
      $c_1=d_1$ and $c_2=d_2$ and $c_3=d_3$.
      Therefore $f_1(c_1x+c_2y+c_3z)=f_1(d_1x+d_2y+d_3z)$ implies that 
      $c_1x+c_2y+c_3z=d_1x+d_2y+d_3z$, and so $f_1$ is one-to-one.

      To verify that it is onto, consider an arbitrary member of the codomain
      $a_1+a_2x+a_3x^2$ and observe that 
      it is indeed the image of a member of the
      domain, namely, it is $f_1(a_1x+a_2y+a_3z)$.
      (For instance, $0+3x+6x^2=f_1(0x+3y+6z)$.)

      The computation checking that $f_1$ preserves addition is this.
      \begin{align*}
        f_1\left(\,(c_1x+c_2y+c_3z)+(d_1x+d_2y+d_3z)\,\right)
          &=f_1\left(\,(c_1+d_1)x+(c_2+d_2)y+(c_3+d_3)z\,\right)  \\
          &=(c_1+d_1)+(c_2+d_2)x+(c_3+d_3)x^2  \\
          &=(c_1+c_2x+c_3x^2)+(d_1+d_2x+d_3x^2)  \\
          &=f_1(c_1x+c_2y+c_3z)+f_1(d_1x+d_2y+d_3z)
      \end{align*}

      The check that $f_1$ preserves scalar multiplication is this.
      \begin{align*}
        f_1(\,r\cdot (c_1x+c_2y+c_3z)\,)
          &=f_1(\,(rc_1)x+(rc_2)y+(rc_3)z\,)  \\
          &=(rc_1)+(rc_2)x+(rc_3)x^2  \\
          &=r\cdot (c_1+c_2x+c_3x^2)  \\
          &=r\cdot f_1(c_1x+c_2y+c_3z)
      \end{align*}
    \end{answer}
  \recommended \item 
   Decide whether each map is an isomorphism
   (if it is an isomorphism then prove it and if it isn't then
   state a condition that it fails to satisfy).
    \begin{exparts}
      \partsitem \( \map{f}{\matspace_{\nbyn{2}}}{\Re} \) given by
        \begin{equation*}
           \begin{mat}
             a  &b  \\
             c  &d
           \end{mat}
           \mapsto
           ad-bc
        \end{equation*}
      \partsitem \( \map{f}{\matspace_{\nbyn{2}}}{\Re^4} \) given by
        \begin{equation*}
           \begin{mat}
             a  &b  \\
             c  &d
           \end{mat}
           \mapsto
           \colvec{a+b+c+d \\ a+b+c \\ a+b \\ a}
        \end{equation*}
      \partsitem \( \map{f}{\matspace_{\nbyn{2}}}{\polyspace_3} \) given by
        \begin{equation*}
           \begin{mat}
             a  &b  \\
             c  &d
           \end{mat}
           \mapsto
           c+(d+c)x+(b+a)x^2+ax^3
        \end{equation*}
      \partsitem \( \map{f}{\matspace_{\nbyn{2}}}{\polyspace_3} \) given by
        \begin{equation*}
           \begin{mat}
             a  &b  \\
             c  &d
           \end{mat}
           \mapsto
           c+(d+c)x+(b+a+1)x^2+ax^3
        \end{equation*}
    \end{exparts}
  \begin{answer}
    \begin{exparts}
      \partsitem No; this map is not one-to-one.
        In particular, the matrix of all zeroes is mapped to the same
        image as the matrix of all ones.
      \partsitem Yes, this is an isomorphism.

        It is one-to-one:
        \begin{equation*}
          \text{if }
          f(\begin{mat}
              a_1  &b_1  \\
              c_1  &d_1
            \end{mat})
         =f(\begin{mat}
              a_2  &b_2  \\
              c_2  &d_2
            \end{mat})
          \text{ then }
          \colvec{a_1+b_1+c_1+d_1 \\ a_1+b_1+c_1 \\ a_1+b_1 \\ a_1}
          =\colvec{a_2+b_2+c_2+d_2 \\ a_2+b_2+c_2 \\ a_2+b_2 \\ a_2}
        \end{equation*}
        gives that \( a_1=a_2 \), and that \( b_1=b_2 \), 
        and that \( c_1=c_2 \), and that \( d_1=d_2 \).

        It is onto, since this shows
        \begin{equation*}
          \colvec{x \\ y \\ z \\ w}
         =f(\begin{mat}
              w    &z-w  \\
              y-z  &x-y
            \end{mat})
        \end{equation*}
        that any four-tall vector is the image of a $\nbyn{2}$~matrix.

        Finally, it preserves combinations
        \begin{align*}
          f(\,r_1\cdot\begin{mat}
              a_1  &b_1  \\
              c_1  &d_1
            \end{mat}
            +r_2\cdot\begin{mat}
              a_2  &b_2  \\
              c_2  &d_2
            \end{mat}\,)
          &=f(\begin{mat}
              r_1a_1+r_2a_2  &r_1b_1+r_2b_2  \\
              r_1c_1+r_2c_2  &r_1d_1+r_2d_2
            \end{mat})                            \\
          &=\colvec{r_1a_1+\dots+r_2d_2 \\ r_1a_1+\dots+r_2c_2 \\ 
                       r_1a_1+\dots+r_2b_2 \\ r_1a_1+r_2a_2}          \\
          &=r_1\cdot\colvec{a_1+\dots+d_1 \\ a_1+\dots+c_1 \\ 
                       a_1+b_1 \\ a_1}
          +r_2\cdot\colvec{a_2+\dots+d_2 \\ a_2+\dots+c_2 \\ 
                       a_2+b_2 \\ a_2}                         \\
          &=r_1\cdot f(\begin{mat}
                  a_1  &b_1  \\
                  c_1  &d_1
            \end{mat})
          +r_2\cdot f(\begin{mat}
                  a_2  &b_2  \\
                  c_2  &d_2
            \end{mat})
        \end{align*}
        and so item~(2) of \nearbylemma{le:PresStructIffPresCombos} shows that
        it preserves structure. 
      \partsitem Yes, it is an isomorphism.

        To show that it is one-to-one, we suppose that two members of the
        domain have the same image under $f$.
        \begin{equation*}
          f(\begin{mat}
              a_1  &b_1  \\
              c_1  &d_1
            \end{mat})
          =f(\begin{mat}
              a_2  &b_2  \\
              c_2  &d_2
            \end{mat})
        \end{equation*}
        This gives, by the definition of $f$, that 
        $c_1+(d_1+c_1)x+(b_1+a_1)x^2+a_1x^3
          =c_2+(d_2+c_2)x+(b_2+a_2)x^2+a_2x^3$
        and then the fact that polynomials are equal only when their
        coefficients are equal gives a set of linear equations
        \begin{align*}
          c_1      &=  c_2      \\
          d_1+c_1  &=  d_2+c_2  \\
          b_1+a_1  &=  b_2+a_2  \\
          a_1      &=  a_2
        \end{align*}
        that has only the solution $a_1=a_2$, $b_1=b_2$, $c_1=c_2$, and 
        $d_1=d_2$.

        To show that $f$ is onto, we note that $p+qx+rx^2+sx^3$ is the image
        under $f$ of this matrix.
        \begin{equation*}
          \begin{mat}
            s  &r-s   \\
            p  &q-p
          \end{mat}
        \end{equation*}

        We can check that $f$ preserves structure by using 
        item~(2) of \nearbylemma{le:PresStructIffPresCombos}.
        \begin{align*}
          f(r_1\cdot\begin{mat}
              a_1  &b_1  \\
              c_1  &d_1
            \end{mat}
            +r_2\cdot\begin{mat}
              a_2  &b_2  \\
              c_2  &d_2
            \end{mat})
           &=
           f(\begin{mat}
               r_1a_1+r_2a_2  &r_1b_1+r_2b_2  \\
               r_1c_1+r_2c_2  &r_1d_1+r_2d_2
             \end{mat})                         \\
           &=\begin{aligned}[t]
               &(r_1c_1+r_2c_2)
               +(r_1d_1+r_2d_2+r_1c_1+r_2c_2)x                 \\
               &\mbox{}\quad +(r_1b_1+r_2b_2+r_1a_1+r_2a_2)x^2
                 +(r_1a_1+r_2a_2)x^3
             \end{aligned}                   \\
           &=\begin{aligned}[t]
                &r_1\cdot\left(c_1+(d_1+c_1)x+(b_1+a_1)x^2+a_1x^3\right) \\
                &\mbox{}\quad +r_2\cdot\left(c_2+(d_2+c_2)x
                            +(b_2+a_2)x^2+a_2x^3\right)
              \end{aligned} \\
           &=r_1\cdot f(\begin{mat}
              a_1  &b_1  \\
              c_1  &d_1
            \end{mat})
            +r_2\cdot f(\begin{mat}
              a_2  &b_2  \\
              c_2  &d_2
            \end{mat})
        \end{align*}
      \partsitem No, this map does not preserve structure.
        For instance, it does not send the matrix of all zeroes
        to the zero polynomial.
    \end{exparts}  
   \end{answer}
  \item 
    Show that the map \( \map{f}{\Re^1}{\Re^1} \) 
    given by \( f(x)=x^3 \) is  one-to-one and onto.
    Is it an isomorphism?
    \begin{answer}
      It is one-to-one and onto, a correspondence,
      because it has an inverse (namely, \( f^{-1}(x)=\sqrt[3]{x} \)).
      However, it is not an isomorphism.
      For instance, \( f(1)+f(1)\neq f(1+1) \).     
    \end{answer}
  \recommended \item 
    Refer to \nearbyexample{exam:TwoWideIsoTwoTall}.
    Produce two more isomorphisms 
    (of course, you must also verify that they satisfy the conditions in the 
    definition of isomorphism).
    \begin{answer}
     Many maps are possible.
     Here are two.
     \begin{equation*}
       \rowvec{a &b}\mapsto\colvec{b \\ a}
       \quad\text{and}\quad
       \rowvec{a &b}\mapsto\colvec{2a \\ b}
     \end{equation*} 
     The verifications are straightforward adaptations of the others above.
    \end{answer}
  \item 
    Refer to \nearbyexample{exam:PolyTwoIsoRThree}.
    Produce two more isomorphisms
    (and verify that they satisfy the conditions).
    \begin{answer}
      Here are two.
      \begin{equation*}
        a_0+a_1x+a_2x^2 \mapsto \colvec{a_1 \\ a_0 \\ a_2}
        \quad\text{and}\quad
        a_0+a_1x+a_2x^2 \mapsto \colvec{a_0+a_1 \\ a_1 \\ a_2}
      \end{equation*}
      Verification is straightforward (for the second, to show that it is onto,
      note that 
      \begin{equation*}
        \colvec{s \\ t \\ u}
      \end{equation*}
      is the image of $(s-t)+tx+ux^2$).
     \end{answer}
  \recommended \item 
     Show that, although \( \Re^2 \) is not itself a subspace of
     \( \Re^3 \),
     it is isomorphic to the \( xy \)-plane subspace of \( \Re^3 \).
     \begin{answer}
       The space $\Re^2$ is not a subspace of $\Re^3$ because it is not a 
       subset of $\Re^3$.
       The two-tall vectors in $\Re^2$ are not members of $\Re^3$.

       The natural isomorphism \( \map{\iota}{\Re^2}{\Re^3} \)
       (called the \definend{injection} map) is this.
       \begin{equation*}
         \colvec{x \\ y}
           \mapsunder{\iota}
         \colvec{x \\ y \\ 0}
       \end{equation*}

       This map is one-to-one because 
       \begin{equation*}
         f(\colvec{x_1 \\ y_1})=f(\colvec{x_2 \\ y_2})
         \quad\text{implies}\quad
         \colvec{x_1 \\ y_1 \\ 0}=\colvec{x_2 \\ y_2 \\ 0}
       \end{equation*}
       which in turn implies that $x_1=x_2$ and $y_1=y_2$, and therefore
       the initial two two-tall vectors are equal.

       Because
       \begin{equation*}
         \colvec{x \\ y \\ 0}
         =f(\colvec{x \\ y})
       \end{equation*}
       this map is onto the $xy$-plane.

       To show that this map preserves structure, we will use
       item~(2) of \nearbylemma{le:PresStructIffPresCombos} and show 
       \begin{multline*}
         f(c_1\cdot \colvec{x_1 \\ y_1}+c_2\cdot \colvec{x_2 \\ y_2})
          = f(\colvec{c_1x_1+c_2x_2 \\ c_1y_1+c_2y_2})  
          =\colvec{c_1x_1+c_2x_2 \\ c_1y_1+c_2y_2 \\ 0}                     \\
          =c_1\cdot \colvec{x_1 \\ y_1 \\ 0}+c_2\cdot \colvec{x_2 \\ y_2 \\ 0}
          =c_1\cdot f(\colvec{x_1 \\ y_1})+c_2\cdot f(\colvec{x_2 \\ y_2})
       \end{multline*}
       that it preserves combinations of two vectors.
     \end{answer}
  \item 
    Find two isomorphisms between \( \Re^{16} \) and
    \( \matspace_{\nbyn{4}} \).
     \begin{answer}
        Here are two:
        \begin{equation*}
           \colvec{r_1 \\ r_2 \\ \vdots \\ r_{16}}
              \mapsto
           \begin{mat}
              r_1  &r_2  &\ldots  \\
                   &              \\
                   &     &\ldots  &r_{16}
           \end{mat}
           \quad\text{and}\quad
           \colvec{r_1 \\ r_2 \\ \vdots \\ r_{16}}
              \mapsto
           \begin{mat}
              r_1    &                    \\
              r_2    &                    \\
              \vdots &   &        &\vdots \\
                   &     &        &r_{16}
           \end{mat}
        \end{equation*}     
        Verification that each is an isomorphism is easy.
    \end{answer}
  \recommended \item
    For what \( k \) is \( \matspace_{\nbym{m}{n}} \) isomorphic to
    \( \Re^{k} \)?
     \begin{answer}
        When $k$ is the product \( k=mn \), here is an isomorphism.
        \begin{equation*}
           \begin{mat}
              r_1  &r_2  &\ldots  \\
                   &\vdots        \\
                   &     &\ldots  &r_{m\cdot n}
           \end{mat}
           \mapsto
           \colvec{r_1 \\ r_2 \\ \vdots \\ r_{m\cdot n}}
        \end{equation*}
        Checking that this is an isomorphism is easy.
      \end{answer}
  \item 
     For what \( k \) is \( \polyspace_k \) isomorphic to \( \Re^n \)?
     \begin{answer}
        If \( n\geq 1 \) then \( \polyspace_{n-1}\isomorphicto\Re^n \).
        (If we take \( \polyspace_{-1} \) and \( \Re^0 \) to be trivial vector
        spaces, then the relationship extends one dimension lower.)  
        The natural isomorphism between them is this. 
        \begin{equation*}
          a_0+a_1x+\dots+a_{n-1}x^{n-1}
          \mapsto\colvec{a_0 \\ a_1 \\ \vdots \\ a_{n-1}}
        \end{equation*}
        Checking that it is an isomorphism is straightforward.
     \end{answer}
  \item \label{exer:PolyToPolyLinSubst}
    Prove that the map in \nearbyexample{ex:OORThreeToFourLinCom},
    from \( \polyspace_5 \) to \( \polyspace_5 \) given by
    \( p(x)\mapsto p(x-1) \), is a vector space isomorphism.
    \begin{answer}
      This is the map, expanded.
      \begin{align*}
        f(a_0+a_1x+a_2x^2+a_3x^3+a_4x^4+a_5x^5)
        &=\begin{aligned}[t]
             &a_0+a_1(x-1)+a_2(x-1)^2+a_3(x-1)^3 \\
             &\mbox{}\quad +a_4(x-1)^4+a_5(x-1)^5 
          \end{aligned}                                          \\
        &=\begin{aligned}[t]
             &a_0+a_1(x-1)+a_2(x^2-2x+1)         \\
             &\mbox{}\quad +a_3(x^3-3x^2+3x-1)   \\
             &\mbox{}\quad +a_4(x^4-4x^3+6x^2-4x+1) \\
             &\mbox{}\quad +a_5(x^5-5x^4+10x^3-10x^2+5x-1)
           \end{aligned}   \\
        &=\begin{aligned}[t]
           &(a_0-a_1+a_2-a_3+a_4-a_5)  \\
           &\mbox{}\quad +(a_1-2a_2+3a_3-4a_4+5a_5)x  \\
           &\mbox{}\quad +(a_2-3a_3+6a_4-10a_5)x^2
                        +(a_3-4a_4+10a_5)x^3         \\
           &\mbox{}\quad +(a_4-5a_5)x^4
                        +a_5x^5
          \end{aligned}
      \end{align*}
      This map is a correspondence because it has an inverse, the map
      \( p(x)\mapsto p(x+1) \).

      To finish checking that it is an isomorphism, we apply  
      item~(2) of \nearbylemma{le:PresStructIffPresCombos} and show 
      that it preserves linear combinations of two polynomials.
      Briefly, the check goes like this.
      \begin{multline*}
        f(c\cdot (a_0+a_1x+\dots +a_5x^5)+d\cdot (b_0+b_1x+\dots +b_5x^5))  \\
         =\dots=
         (ca_0-ca_1+ca_2-ca_3+ca_4-ca_5+db_0-db_1+db_2-db_3+db_4-db_5)
          +\dots+(ca_5+db_5)x^5                                 \\
         =\dots=
         c\cdot f(a_0+a_1x+\dots +a_5x^5)+d\cdot f(b_0+b_1x+\dots +b_5x^5)
      \end{multline*}
    \end{answer}
  \item  
    Why, in \nearbylemma{le:IsoSendsZeroToZero}, must there be a
    \( \vec{v}\in V \)?
    That is, why must $V$ be nonempty?
     \begin{answer}
       No vector space has the empty set underlying it.   
       We can take $\vec{v}$ to be the zero vector.      
     \end{answer}
  \item  
     Are any two trivial spaces isomorphic?
     \begin{answer}
        Yes; where the two spaces are \( \set{\vec{a}} \) and
        \( \set{\vec{b}} \),
        the map sending \( \vec{a} \) to \( \vec{b} \) is clearly one-to-one
        and onto, and also preserves what little structure there is.  
      \end{answer}
  \item 
    In the proof of \nearbylemma{le:PresStructIffPresCombos},
    what about the zero-summands case (that is, if $n$ is zero)?
    \begin{answer}
       A linear combination of $n=0$ vectors adds to the zero vector and so
       \nearbylemma{le:IsoSendsZeroToZero}
       shows that the three statements are equivalent in this case.
    \end{answer}
  \item 
    Show that any isomorphism \( \map{f}{\polyspace_0}{\Re^1} \) has the
    form \( a\mapsto ka \) for some nonzero real number \( k \).
    \begin{answer}
      Consider the basis \( \sequence{1} \) for \( \polyspace_0 \)
      and let \( f(1)\in\Re \) be \( k \).
      For any \( a\in\polyspace_0 \)
      we have that 
      \( f(a)=f(a\cdot 1)=af(1)=ak \) and so \( f \)'s action is multiplication
      by \( k \).
      Note that \( k\neq 0 \) or else the map is not one-to-one.
      (Incidentally, any such map $a\mapsto ka$ is an isomorphism,
      as is easy to check.)
     \end{answer}
  \recommended \item 
    These prove that isomorphism is an equivalence relation.
    \begin{exparts}
      \partsitem Show that the identity map $\map{\mbox{id}}{V}{V}$ is an 
        isomorphism.
        Thus, any vector space is isomorphic to itself.
      \partsitem Show that if $\map{f}{V}{W}$ is an isomorphism then
        so is its inverse $\map{f^{-1}}{W}{V}$. 
        Thus, if $V$ is isomorphic to $W$ then also $W$ is isomorphic to $V$.
      \partsitem Show that a composition of isomorphisms is an isomorphism:
        if $\map{f}{V}{W}$ is an isomorphism and $\map{g}{W}{U}$ is an
        isomorphism then so also is $\map{\composed{g}{f}}{V}{U}$.
        Thus, if $V$ is isomorphic to $W$ and $W$ is isomorphic to $U$,
        then also $V$ is isomorphic to $U$.
    \end{exparts}
    \begin{answer}
      In each item, following item~(2) of 
      \nearbylemma{le:PresStructIffPresCombos}, we show that the map preserves
      structure by showing that the it preserves linear combinations
      of two members of the domain.
      \begin{exparts}
        \partsitem 
          The identity map is clearly one-to-one and onto.
          For linear combinations the check is easy.
          \begin{equation*}
            \mbox{id}(c_1\cdot \vec{v}_1+c_2\cdot \vec{v}_2)
             =c_1\vec{v}_1+c_2\vec{v}_2
             =c_1\cdot \mbox{id}(\vec{v}_1)+c_2\cdot \mbox{id}(\vec{v}_2)
          \end{equation*}
        \partsitem The inverse of a correspondence is also a correspondence
          (as stated in the appendix), so we need only check that the 
          inverse preserves linear combinations.
          Assume that \( \vec{w}_1=f(\vec{v}_1) \) 
          (so \( f^{-1}(\vec{w}_1)=\vec{v}_1 \)) 
          and assume that \( \vec{w}_2=f(\vec{v}_2) \).
          \begin{align*}
            f^{-1}(c_1\cdot\vec{w}_1+c_2\cdot\vec{w}_2)
              &=f^{-1}\bigl(\,c_1\cdot f(\vec{v}_1)
                             +c_2\cdot f(\vec{v}_2)\,\bigr) \\
              &=f^{-1}(\,f\bigl(c_1\vec{v}_1+c_2\vec{v}_2)\,\bigr)    \\
              &=c_1\vec{v}_1+c_2\vec{v}_2                             \\ 
              &=c_1\cdot f^{-1}(\vec{w}_1)+c_2\cdot f^{-1}(\vec{w}_2)    
          \end{align*}
        \partsitem The composition of two correspondences is a correspondence
          (as stated in the appendix), so we need only check that the 
          composition map preserves linear combinations.
          \begin{align*}
           \composed{g}{f}\,\bigl(c_1\cdot\vec{v}_1+c_2\cdot\vec{v}_2\bigr)
             &=g\bigl(\,f(c_1\vec{v}_1+c_2\vec{v}_2)\,\bigr)        \\
             &=g\bigl(\,c_1\cdot f(\vec{v}_1)+c_2\cdot f(\vec{v}_2)\,\bigr) \\ 
             &=c_1\cdot g\bigl(f(\vec{v}_1))+c_2\cdot g(f(\vec{v}_2)\bigr)  \\
             &=c_1\cdot\composed{g}{f}\,(\vec{v}_1)
                  +c_2\cdot\composed{g}{f}\,(\vec{v}_2)
          \end{align*}
      \end{exparts}
     \end{answer}
  \item 
    Suppose that \( \map{f}{V}{W} \) preserves structure.
    Show that \( f \) is one-to-one if and only if the unique member of 
    \( V \) mapped by \( f \) to \( \zero_W \) is \( \zero_V \).
    \begin{answer}
      One direction is easy:~by definition, if \( f \) is one-to-one 
      then for any
      \( \vec{w}\in W \) at most one \( \vec{v}\in V \) has
      \( f(\vec{v}\,)=\vec{w} \), and so in particular, at most one member of
      \( V \) is mapped to \( \zero_W \).
      The proof of \nearbylemma{le:IsoSendsZeroToZero} does not use the
      fact that the map is a correspondence and therefore 
      shows that any structure-preserving map 
      \( f \) sends \( \zero_V \) to \( \zero_W \).

      For the other direction, assume that the only member of \( V \) 
      that is mapped to \( \zero_W \) is \( \zero_V \).
      To show that 
      \( f \) is one-to-one assume that \( f(\vec{v}_1)=f(\vec{v}_2) \).
      Then \( f(\vec{v}_1)-f(\vec{v}_2)=\zero_W \) and so
      \( f(\vec{v}_1-\vec{v}_2)=\zero_W \).
      Consequently \( \vec{v}_1-\vec{v}_2=\zero_V \), so
      $\vec{v}_1=\vec{v}_2$, and so \( f \) is one-to-one.    
    \end{answer}
  \item 
    Suppose that \( \map{f}{V}{W} \) is an isomorphism.
    Prove that the set 
    \( \set{\vec{v}_1,\dots,\vec{v}_k}\subseteq V \) is linearly
    dependent if and only if the set of images
    \( \set{f(\vec{v}_1),\dots,f(\vec{v}_k)}\subseteq W \)
    is linearly dependent.
    \begin{answer}
      We will prove something stronger\Dash not only is the existence of a
      dependence preserved by isomorphism, but each instance of a dependence is
      preserved, that is,
      \begin{multline*}
        \vec{v}_i=c_1\vec{v}_1+\cdots+c_{i-1}\vec{v}_{i-1}
                   +c_{i+1}\vec{v}_{i+1}+\cdots+c_k\vec{v}_k   \\
        \iff
        f(\vec{v}_i)=c_1f(\vec{v}_1)+\cdots+c_{i-1}f(\vec{v}_{i-1})
                     +c_{i+1}f(\vec{v}_{i+1})+\cdots+c_kf(\vec{v}_k).
      \end{multline*}
      The \( \implies \) direction of this statement holds by item~(3) of
      \nearbylemma{le:PresStructIffPresCombos}.
      The \( \isimpliedby \) direction holds by regrouping
      \begin{align*}
         f(\vec{v}_i)
           &=c_1f(\vec{v}_1)+\dots+c_{i-1}f(\vec{v}_{i-1})
                     +c_{i+1}f(\vec{v}_{i+1})+\dots+c_kf(\vec{v}_k)  \\
           &=f(c_1\vec{v}_1+\dots+c_{i-1}\vec{v}_{i-1}
                     +c_{i+1}\vec{v}_{i+1}+\dots+c_k \vec{v}_k) 
      \end{align*}
      and applying the fact that $f$ is one-to-one, and so for the two
      vectors $\vec{v}_i$ and 
      $c_1\vec{v}_1+\dots+c_{i-1}\vec{v}_{i-1}
              +c_{i+1}f\vec{v}_{i+1}+\dots+c_kf(\vec{v}_k$
      to be mapped to the same image by $f$, they must be equal.
    \end{answer}
  \recommended \item \label{exer:RigidPlaneMapsAutos} 
    Show that each type of map from \nearbyexample{exam:RigidPlaneMapsAutos}
    is an automorphism.
    \begin{exparts}
      \partsitem Dilation $d_s$ by a nonzero scalar \( s \).
      \partsitem Rotation $t_\theta$ through an angle \( \theta \).
      \partsitem Reflection $f_\ell$ over a line through the origin.
    \end{exparts}
    \textit{Hint.}
     For the second and third items, polar coordinates are useful.
    \begin{answer}
      \begin{exparts}
        \partsitem 
          This map is one-to-one because if \( d_s(\vec{v}_1)=d_s(\vec{v}_2) \)
          then by definition of the map,
          \( s\cdot\vec{v}_1=s\cdot\vec{v}_2 \) and so 
          \( \vec{v}_1=\vec{v}_2 \), as \( s \) is nonzero.
          This map is onto as any \( \vec{w}\in\Re^2 \) is the image of
          \( \vec{v}=(1/s)\cdot\vec{w} \) (again, note that $s$ is nonzero).
          (Another way to see that this map 
          is a correspondence is to observe that it has
          an inverse: the inverse of \( d_s \) is \( d_{1/s} \).)

          To finish, note that this map preserves linear combinations
          \begin{equation*}
            d_s(c_1\cdot\vec{v}_1+c_2\cdot\vec{v}_2)
            =s(c_1\vec{v}_1+c_2\vec{v}_2)
            =c_1s\vec{v}_1+c_2s\vec{v}_2
            =c_1\cdot d_s(\vec{v}_1)+c_2\cdot d_s(\vec{v}_2)
          \end{equation*}
          and therefore is an isomorphism.
        \partsitem As in the prior item, we can show that the map $t_\theta$
          is a correspondence by noting that it has an inverse, $t_{-\theta}$.

          That the map preserves structure is  geometrically easy to see. 
          For instance,
          adding two vectors and then rotating them has the same effect as
          rotating first and then adding. 
          For an algebraic argument, consider polar coordinates:
          the map \( t_\theta \) sends the vector with endpoint
          \( (r,\phi) \) to the vector with endpoint
          \( (r,\phi+\theta) \).
          Then the familiar trigonometric formulas
          $\cos(\phi+\theta)=\cos\phi\,\cos\theta-\sin\phi\,\sin\theta$
          and 
          $\sin(\phi+\theta)=\sin\phi\,\cos\theta+\cos\phi\,\sin\theta$
          show how to express the map's action in the usual
          rectangular coordinate system.
          \begin{equation*}
            \colvec{x \\ y}=\colvec{r\cos\phi \\ r\sin\phi}
              \mapsunder{t_\theta}
            \colvec{r\cos(\phi+\theta) \\ r\sin(\phi+\theta)}
            =\colvec{x\cos\theta-y\sin\theta \\ x\sin\theta+y\cos\theta}
          \end{equation*}
          Now the calculation for preservation of addition
          is routine.
          \begin{equation*}
            \colvec{x_1+x_2 \\ y_1+y_2}
              \mapsunder{t_\theta}
            \colvec{(x_1+x_2)\cos\theta-(y_1+y_2)\sin\theta \\ 
                       (x_1+x_2)\sin\theta+(y_1+y_2)\cos\theta}
            =\colvec{x_1\cos\theta-y_1\sin\theta \\ 
                       x_1\sin\theta+y_1\cos\theta}
            +\colvec{x_2\cos\theta-y_2\sin\theta \\ 
                       x_2\sin\theta+y_2\cos\theta}
          \end{equation*}
          The calculation for preservation of scalar multiplication is similar.
        \partsitem 
          This map is a correspondence because it has an inverse (namely,
          itself).

          As in the last item, that the reflection map preserves
          structure is geometrically easy to see:~adding vectors and then 
          reflecting gives the same result as reflecting first and then
          adding, for instance.
          For an algebraic proof, suppose that the line $\ell$ has slope $k$
          (the case of a line with undefined slope can be done as a separate,
          but easy, case).
          We can follow the hint and use
          polar coordinates:~where the line \( \ell \) forms an angle of
          \( \phi \) with the \( x \)-axis, the action of \( f_\ell \) is to
          send the vector with endpoint \( (r\cos\theta,r\sin\theta) \) to 
          the one with endpoint 
          \( (r\cos(2\phi-\theta),r\sin(2\phi-\theta)) \).          
          \begin{center}  \small
            \includegraphics{ch3.72}
          \end{center}
          To convert to rectangular coordinates, we will use some trigonometric
          formulas, as we  did in the prior item.
          First observe that $\cos\phi$ and $\sin\phi$ can be determined
          from the slope $k$ of the line.
          This picture
          \begin{center}  \small
            \includegraphics{ch3.73}
          \end{center}
          gives that $\cos\phi=1/\sqrt{1+k^2}$ and $\sin\phi=k/\sqrt{1+k^2}$.
          Now,
          \begin{align*}
            \cos(2\phi-\theta)
              &=\cos(2\phi)\,\cos\theta+\sin(2\phi)\,\sin\theta        \\
              &=\left(\cos^2\phi-\sin^2\phi\right)\,\cos\theta
                 +\left(2\sin\phi\cos\phi\right)\,\sin\theta        \\
              &=\left((\frac{1}{\sqrt{1+k^2}})^2
                      -(\frac{k}{\sqrt{1+k^2}})^2\right)
                   \,\cos\theta
                 +\left(2\frac{k}{\sqrt{1+k^2}}\frac{1}{\sqrt{1+k^2}}\right)
                   \,\sin\theta        \\
              &=\left(\frac{1-k^2}{1+k^2}\right)\,\cos\theta
                 +\left(\frac{2k}{1+k^2}\right)\,\sin\theta        
          \end{align*}
          and thus the first component of the image vector is this.
          \begin{equation*}
            r\cdot\cos(2\phi-\theta)=\frac{1-k^2}{1+k^2}\cdot x
                                     +\frac{2k}{1+k^2}\cdot y
          \end{equation*}
          A similar calculation shows that the second component of the image
          vector is this.
          \begin{equation*}
            r\cdot\sin(2\phi-\theta)=\frac{2k}{1+k^2}\cdot x
                                     -\frac{1-k^2}{1+k^2}\cdot y
          \end{equation*}
          With this algebraic description of the action of $f_\ell$
          \begin{equation*}
            \colvec{x \\ y}
             \mapsunder{f_\ell}
            \colvec{(1-k^2/1+k^2)\cdot x
                                     +(2k/1+k^2)\cdot y \\ 
                   (2k/1+k^2)\cdot x
                                     -(1-k^2/1+k^2)\cdot y}  
          \end{equation*}
          checking that it preserves structure is routine.
      \end{exparts}  
    \end{answer}
  \item 
    Produce an automorphism of \( \polyspace_2 \) other than the
    identity map, and other than a shift map $p(x)\mapsto p(x-k)$.
    \begin{answer}
       First, the map $p(x)\mapsto p(x+k)$ doesn't count
       because it is a version of $p(x)\mapsto p(x-k)$.
       Here is a correct answer (many others are also correct):
       \( a_0+a_1x+a_2x^2 \mapsto a_2+a_0x+a_1x^2 \).
       Verification that this is an isomorphism is straightforward.
    \end{answer}
  \item
    \begin{exparts}
      \partsitem Show that 
        a function \( \map{f}{\Re^1}{\Re^1} \) is an automorphism
        if and only if it has the form \( x\mapsto kx \) for some
        \( k\neq 0 \).
      \partsitem Let \( f \) be an automorphism of \( \Re^1 \) such that
        \( f(3)=7 \).
        Find \( f(-2) \).
      \partsitem Show that a function \( \map{f}{\Re^2}{\Re^2} \) is
        an automorphism if and only if it has the form
        \begin{equation*}
          \colvec{x \\ y}
            \mapsto
          \colvec{ax+by \\ cx+dy}
        \end{equation*}
        for some \( a,b,c,d\in\Re \) with \( ad-bc\neq 0 \).
        \textit{Hint.}
        Exercises in prior subsections have shown that 
        \begin{equation*}
          \colvec{b \\ d}\text{\ is not a multiple of\ }\colvec{a \\ c}
        \end{equation*}
        if and only if $ad-bc\neq 0$.
      \partsitem Let \( f \) be an automorphism of \( \Re^2 \) with
        \begin{equation*}
          f(\colvec[r]{1 \\ 3})=\colvec[r]{2 \\ -1}
           \quad\text{and}\quad
          f(\colvec[r]{1 \\ 4})=\colvec[r]{0 \\ 1}.
        \end{equation*}
        Find
        \begin{equation*}
          f(\colvec[r]{0 \\ -1}).
        \end{equation*}
    \end{exparts}
    \begin{answer}
      \begin{exparts}
        \partsitem For the `only if' half, let \( \map{f}{\Re^1}{\Re^1} \) 
          to be an isomorphism. 
          Consider the basis \( \sequence{1}\subseteq\Re^1 \).
          Designate \( f(1) \) by \( k \).
          Then for any \( x \) we have that
          \( f(x)=f(x\cdot 1)=x\cdot f(1)=xk \), and so $f$'s action is
          multiplication by $k$.
          To finish this half, just note that
          \( k\neq 0 \) or else $f$ would not be one-to-one.

          For the `if' half we only have to check that such a map is an 
          isomorphism when $k\neq 0$.
          To check that it is one-to-one, assume that \( f(x_1)=f(x_2) \) 
          so that \( kx_1=kx_2 \) and
          divide by the nonzero factor \( k \) to conclude that $x_1=x_2$.
          To check that it is onto, note that any \( y\in\Re^1 \) 
          is the image of \( x=y/k \) (again, $k\neq 0$).
          Finally, to check that such a map preserves combinations of
          two members of the domain,
          we have this.
          \begin{equation*}
            f(c_1x_1+c_2x_2)
            =k(c_1x_1+c_2x_2)
            =c_1kx_1+c_2kx_2
            =c_1f(x_1)+c_2f(x_2)
          \end{equation*}
        \partsitem By the prior item, $f$'s action is \( x\mapsto (7/3)x \).
          Thus \( f(-2)=-14/3 \).
        \partsitem For the `only if' half,
          assume that \( \map{f}{\Re^2}{\Re^2} \) is an automorphism. 
          Consider the standard basis \( \stdbasis_2 \) for \( \Re^2 \).
          Let
          \begin{equation*}
            f(\vec{e}_1)=\colvec{a \\ c}
            \quad\text{and}\quad
            f(\vec{e}_2)=\colvec{b \\ d}.
          \end{equation*}
          Then the action of $f$ on any vector is determined by
          by its action on the two basis vectors. 
          \begin{equation*}
            f(\colvec{x \\ y})
            =f(x\cdot\vec{e}_1+y\cdot\vec{e}_2)
            =x\cdot f(\vec{e}_1)+y\cdot f(\vec{e}_2)
            =x\cdot\colvec{a \\ c}+y\cdot\colvec{b \\ d}
            =\colvec{ax+by \\ cx+dy}
          \end{equation*}
          To finish this half,
          note that if \( ad-bc=0 \), that is, 
          if \( f(\vec{e}_2) \) is a multiple of \( f(\vec{e}_1) \),
          then \( f \) is not one-to-one.

          For `if' we must check that the map is an isomorphism,
          under the condition that $ad-bc\neq 0$.
          The structure-preservation check is easy; we will here show that 
          \( f \) is a correspondence.
          For the argument that the map is one-to-one, assume this.
          \begin{equation*}
            f(\colvec{x_1 \\ y_1})
            =f(\colvec{x_2 \\ y_2})
            \quad\text{and so}\quad
            \colvec{ax_1+by_1 \\ cx_1+dy_1}
            =\colvec{ax_2+by_2 \\ cx_2+dy_2}
          \end{equation*}
          Then, because \( ad-bc\neq 0 \), the resulting system
          \begin{equation*}
            \begin{linsys}{2}
              a(x_1-x_2)  &+  &b(y_1-y_2)  &=  &0  \\
              c(x_1-x_2)  &+  &d(y_1-y_2)  &=  &0  
            \end{linsys}
          \end{equation*}
          has a unique solution, namely the trivial one
          $x_1-x_2=0$ and $y_1-y_2=0$
          (this follows from the hint).

          The argument that this map is onto is closely related\Dash this system
          \begin{equation*}
            \begin{linsys}{2}
              ax_1  &+  &by_1  &=  &x  \\
              cx_1  &+  &dy_1  &=  &y  
            \end{linsys}
          \end{equation*}
          has a solution for any \( x \) and \( y \) if and only if
          this set
          \begin{equation*}
            \set{\colvec{a \\ c},
                 \colvec{b \\ d} }
          \end{equation*}
          spans \( \Re^2 \), i.e., if and only if this set is 
          a basis (because it is a two-element subset of $\Re^2$), 
          i.e., if and only if \( ad-bc\neq 0 \).
        \partsitem
          \begin{equation*}
            f(\colvec[r]{0 \\ -1})=f(\colvec[r]{1 \\ 3}-\colvec[r]{1 \\ 4})
            =f(\colvec[r]{1 \\ 3})-f(\colvec[r]{1 \\ 4})
            =\colvec[r]{2 \\ -1}-\colvec[r]{0 \\ 1}
            =\colvec[r]{2 \\ -2}
          \end{equation*}
      \end{exparts}    
    \end{answer}
  \item 
     Refer to \nearbylemma{le:IsoSendsZeroToZero} and
     \nearbylemma{le:PresStructIffPresCombos}. 
     Find two more things preserved by isomorphism.
     \begin{answer}
       There are many answers; two are linear independence and
       subspaces.  

       To show that if a set $\set{\vec{v}_1,\dots,\vec{v}_n}$ is linearly
       independent then its image $\set{f(\vec{v}_1),\dots,f(\vec{v}_n)}$
       is also linearly independent, consider a linear relationship
       among members of the image set.
       \begin{equation*}
         0=c_1f(\vec{v}_1)+\dots+c_nf(\vec{v_n})
          =f(c_1\vec{v}_1)+\dots+f(c_n\vec{v_n})
          =f(c_1\vec{v}_1+\dots+c_n\vec{v_n})
       \end{equation*}
       Because this map is an isomorphism, it is one-to-one.
       So $f$ maps only one vector from the domain to the zero vector in the
       range, that is, $c_1\vec{v}_1+\dots+c_n\vec{v}_n$ equals the 
       zero vector (in the domain, of course).
       But, if $\set{\vec{v}_1,\dots,\vec{v}_n}$ is linearly
       independent then all of the $c$'s are zero, and so 
       $\set{f(\vec{v}_1),\dots,f(\vec{v}_n)}$ is linearly independent also.
       (\textit{Remark.}
       There is a small point about this argument that is worth mention.
       In a set, repeats collapse, that is, strictly speaking, this is a
       one-element set: $\set{\vec{v},\vec{v}}$, because the things
       listed as in it are the same thing.
       Observe, however, the use of the subscript~$n$ in the above argument.
       In moving from the domain set $\set{\vec{v}_1,\dots,\vec{v}_n}$ to
       the image set 
       $\set{f(\vec{v}_1),\dots,f(\vec{v}_n)}$, there is no collapsing,
       because the image set does not have repeats,
       because the isomorphism $f$ is one-to-one.)

       To show that if $\map{f}{V}{W}$ is an isomorphism and if 
       $U$ is a subspace of the domain $V$ then the set of image vectors 
       $f(U)=\set{\vec{w}\in W
                 \suchthat\text{$\vec{w}=f(\vec{u})$ for some $\vec{u}\in U$}}$
       is a subspace of $W$, we need only show that it is closed under linear
       combinations of two of its members (it is nonempty because it contains
       the image of the zero vector).
       We have
       \begin{equation*}
         c_1\cdot f(\vec{u}_1)+c_2\cdot f(\vec{u}_2)
        =f(c_1\vec{u}_1)+f(c_2\vec{u}_2)
        =f(c_1\vec{u}_1+c_2\vec{u}_2)
       \end{equation*}
       and $c_1\vec{u}_1+c_2\vec{u}_2$ is a member of $U$ because of 
       the closure of a subspace under combinations. 
       Hence the combination of $f(\vec{u}_1)$ and $f(\vec{u}_2)$ is 
       a member of $f(U)$.
      \end{answer}
  \item 
    We show that isomorphisms can be tailored
    to fit in that, sometimes, given vectors in the domain and in
    the range we can produce an isomorphism associating those vectors.
    \begin{exparts}
      \partsitem Let \( B=\sequence{\vec{\beta}_1,\vec{\beta}_2,
                   \vec{\beta}_3} \)
        be a basis for \( \polyspace_2 \) so that
        any \( \vec{p}\in\polyspace_2 \) has a unique representation
        as \(
        \vec{p}=c_1\vec{\beta}_1+c_2\vec{\beta}_2+c_3\vec{\beta}_3 \),
        which we denote in this way.
        \begin{equation*}
          \rep{\vec{p}}{B}=\colvec{c_1 \\ c_2 \\ c_3}
        \end{equation*}
        Show that the $\rep{\cdot}{B}$ operation is a function from 
        \( \polyspace_2 \) to \( \Re^3 \)
        (this entails showing that with every domain vector
        \( \vec{v}\in\polyspace_2 \)
        there is an associated image vector in \( \Re^3 \), 
        and further, that with every domain vector 
        \( \vec{v}\in\polyspace_2 \) there is at most one associated 
        image vector).
      \partsitem Show that this $\rep{\cdot}{B}$ function 
        is one-to-one and onto.
      \partsitem Show that it preserves structure.
      \partsitem Produce an isomorphism from \( \polyspace_2 \) to 
        \( \Re^3 \) that fits these specifications.
        \begin{equation*}
          x+x^2\mapsto\colvec[r]{1 \\ 0 \\ 0}
          \quad\text{and}\quad
          1-x\mapsto\colvec[r]{0 \\ 1 \\ 0}
        \end{equation*}
    \end{exparts}
    \begin{answer}
      \begin{exparts}
        \partsitem The association
          \begin{equation*}
            \vec{p}=c_1\vec{\beta}_1+c_2\vec{\beta}_2+c_3\vec{\beta}_3
            \mapsunder{\rep{\cdot}{B}}
            \colvec{c_1 \\ c_2 \\ c_3}
          \end{equation*}
          is a function if every member $\vec{p}$ of the domain is associated
          with at least one member of the codomain, and if every member
          $\vec{p}$ of the
          domain is associated with at most one member of the codomain.
          The first condition
          holds because the basis $B$ spans the domain\Dash every
          $\vec{p}$ can be written as at least one linear combination of
          $\vec{\beta}$'s.
          The second condition holds because the basis $B$ is linearly
          independent\Dash every member $\vec{p}$ 
          of the domain can be written as
          at most one linear combination of the $\vec{\beta}$'s.
        \partsitem For the one-to-one argument,
          if $\rep{\vec{p}}{B}=\rep{\vec{q}}{B}$, that is, if
          \( \rep{p_1\vec{\beta}_1+p_2\vec{\beta}_2+p_3\vec{\beta}_3}{B}
             =\rep{q_1\vec{\beta}_1+q_2\vec{\beta}_2+q_3\vec{\beta}_3}{B} \)
          then
          \begin{equation*}
            \colvec{p_1 \\ p_2 \\ p_3}
            =\colvec{q_1 \\ q_2 \\ q_3}
          \end{equation*}
          and so \( p_1=q_1 \) and \( p_2=q_2 \) and \( p_3=q_3 \),
          which gives the conclusion that $\vec{p}=\vec{q}$.
          Therefore this map is one-to-one.

          For onto, we can just note that
          \begin{equation*}
            \colvec{a \\ b \\ c}
          \end{equation*}
          equals $\rep{a\vec{\beta}_1+b\vec{\beta}_2+c\vec{\beta}_3}{B}$, and
          so any member of the codomain $\Re^3$ is the image of some
          member of the domain $\polyspace_2$. 
        \partsitem This map respects addition and scalar multiplication
           because it respects combinations of two members of the domain
           (that is, we are using item~(2) of 
           \nearbylemma{le:PresStructIffPresCombos}):
           where $\vec{p}=p_1\vec{\beta}_1+p_2\vec{\beta}_2+p_3\vec{\beta}_3$
           and $\vec{q}=q_1\vec{\beta}_1+q_2\vec{\beta}_2+q_3\vec{\beta}_3$,
           we have this.
           \begin{align*}
             \rep{c\cdot\vec{p}+d\cdot\vec{q}}{B}
             &=\rep{\,(cp_1+dq_1)\vec{\beta}_1+(cp_2+dq_2)\vec{\beta}_2
                  +(cp_3+dq_3)\vec{\beta}_3\,}{B}  \\
             &=\colvec{cp_1+dq_1 \\ cp_2+dq_2 \\ cp_3+dq_3}  \\
             &=c\cdot\colvec{p_1 \\ p_2 \\ p_3}  
               +d\cdot\colvec{q_1 \\ q_2 \\ q_3}  \\
             &=\rep{\vec{p}}{B}+\rep{\vec{q}}{B}
           \end{align*}
        \partsitem Use any basis \( B \) for $\polyspace_2$
          whose first two members are \( x+x^2 \) and \( 1-x \), 
          say \( B=\sequence{x+x^2,1-x,1} \).
      \end{exparts}   
    \end{answer}
  \item 
    Prove that a space is \( n \)-dimensional if and only if it is
    isomorphic to \( \Re^n \).
    \textit{Hint.} 
    Fix a basis $B$ for the space and consider the 
    map sending a vector over to its representation with respect to $B$.
    \begin{answer}
      See the next subsection.  
    \end{answer}
  \item 
    \textit{(Requires the subsection on Combining Subspaces, 
    which is optional.)} 
    Let \( U \) and \( W \) be vector spaces.
    Define a new vector space, consisting of the set
    \( U\times W=\set{(\vec{u},\vec{w})  \suchthat  \vec{u}\in U\text{\ and\ }
                                                   \vec{w}\in W}  \)
    along with these operations.
    \begin{equation*}
      (\vec{u}_1,\vec{w}_1)+(\vec{u}_2,\vec{w}_2)=
          (\vec{u}_1+\vec{u}_2,\vec{w}_1+\vec{w}_2)
      \quad\text{and}\quad
      r\cdot (\vec{u},\vec{w})=(r\vec{u},r\vec{w})
    \end{equation*}
    This is a vector space,
    the \definend{external direct sum}\index{direct sum!external}%
    \index{external direct sum} of \( U \) and \( W \).
    \begin{exparts}
      \partsitem Check that it is a vector space. 
      \partsitem Find a basis for, and the dimension of,
        the external direct sum \( \polyspace_2\times\Re^2 \).
      \partsitem What is the relationship among
        \( \dim(U) \), \( \dim(W) \), and \( \dim(U\times W) \)?
      \partsitem Suppose that \( U \) and \( W \) are subspaces of a vector
        space \( V \) such that \( V=U\directsum W \)
        (in this case we say that $V$ is the 
        \definend{internal direct sum}\index{direct sum!internal}%
        \index{internal direct sum} of $U$ and $W$).
        Show that the map \( \map{f}{U\times W}{V} \) given by
        \begin{equation*}
          (\vec{u},\vec{w})\mapsunder{f} \vec{u}+\vec{w}
        \end{equation*}
        is an isomorphism. 
        Thus if the internal direct sum is defined
        then the internal and external direct sums are isomorphic.
    \end{exparts}
    \begin{answer}
      \begin{exparts}
        \partsitem Most of the conditions in the definition of a
          vector space are routine.
          We here sketch the verification of part~(1) of that definition.

          For closure of $U\times W$, note that because 
          \( U \) and \( W \) are closed, we have that
          $\vec{u}_1+\vec{u}_2\in U$ and $\vec{w}_1+\vec{w}_2\in W$
          and so \( (\vec{u}_1+\vec{u}_2,\vec{w}_1+\vec{w}_2)\in U\times W \).
          Commutativity of addition in \( U\times W \) follows from
          commutativity of addition in \( U \) and \( W \).
          \begin{equation*}
             (\vec{u}_1,\vec{w}_1)+(\vec{u}_2,\vec{w}_2)=
                 (\vec{u}_1+\vec{u}_2,\vec{w}_1+\vec{w}_2)
                 =(\vec{u}_2+\vec{u}_1,\vec{w}_2+\vec{w}_1)
             =(\vec{u}_2,\vec{w}_2)+(\vec{u}_1,\vec{w}_1)
          \end{equation*}
          The check for associativity of addition is similar.
          The zero element is \( (\zero_U,\zero_W)\in U\times W \) 
          and the additive inverse of
          \( (\vec{u},\vec{w}) \) is \( (-\vec{u},-\vec{w}) \).

          The checks for the second part of the definition of a vector space
          are also straightforward.
        \partsitem This is a basis
          \begin{equation*}
            \sequence{\, (1,\colvec[r]{0 \\ 0}), (x,\colvec[r]{0 \\ 0}),
              (x^2,\colvec[r]{0 \\ 0}), (1,\colvec[r]{1 \\ 0}),
              (1,\colvec[r]{0 \\ 1}) \,}
          \end{equation*}
          because there is one and only one way to represent any
          member of $\polyspace_2\times \Re^2$ with respect to this set; 
          here is an example.
          \begin{equation*}
            (3+2x+x^2,\colvec[r]{5 \\ 4})
             =  
              3\cdot(1,\colvec[r]{0 \\ 0})
              +2\cdot(x,\colvec[r]{0 \\ 0})
              +(x^2,\colvec[r]{0 \\ 0}) 
              +5\cdot(1,\colvec[r]{1 \\ 0})
              +4\cdot(1,\colvec[r]{0 \\ 1}) 
          \end{equation*}
          The dimension of this space is five.
        \partsitem We have \( \dim(U\times W)=\dim(U)+\dim(W) \) 
          as this is a basis.
          \begin{equation*}
            \sequence{ (\vec{\mu}_1,\zero_W), \dots,
              (\vec{\mu}_{\dim(U)},\zero_W), (\zero_U,\vec{\omega}_1),
              \ldots,(\zero_U,\vec{\omega}_{\dim(W)}) }
          \end{equation*}
        \partsitem We know that if \( V=U\directsum W \) then each 
          \( \vec{v}\in V \)
          can be written as \( \vec{v}=\vec{u}+\vec{w} \) in one and only one
          way.
          This is just what we need to prove that 
          the given function an isomorphism.

          First, to show that \( f \) is one-to-one we can show that if
          \( f\left((\vec{u}_1,\vec{w}_1)\right)
                 =\left((\vec{u}_2,\vec{w}_2)\right) \), that is, if 
          \( \vec{u}_1+\vec{w}_1=\vec{u}_2+\vec{w}_2 \) then
          \( \vec{u}_1=\vec{u}_2 \) and \( \vec{w}_1=\vec{w}_2 \).
          But the statement 
          `each $\vec{v}$ is such a sum in only one way' is exactly what is
          needed to make this conclusion.
          Similarly, the argument that \( f \) is onto is completed by the
          statement that 
          `each $\vec{v}$ is such a sum in at least one way'.

          This map also preserves linear combinations
          \begin{align*}
            f(\,c_1\cdot(\vec{u}_1,\vec{w}_1)+c_2\cdot (\vec{u}_2,\vec{w}_2)\,)
            &=f(\,(c_1\vec{u}_1+c_2\vec{u}_2,c_1\vec{w}_1+c_2\vec{w}_2)\,) \\
            &=c_1\vec{u}_1+c_2\vec{u}_2+c_1\vec{w}_1+c_2\vec{w}_2    \\
            &=c_1\vec{u}_1+c_1\vec{w}_1+c_2\vec{u}_2+c_2\vec{w}_2    \\
            &=c_1\cdot f(\,(\vec{u}_1,\vec{w}_1)\,)
               +c_2\cdot f(\,(\vec{u}_2,\vec{w}_2)\,)
          \end{align*}
          and so it is an isomorphism.
      \end{exparts}     
    \end{answer}
\end{exercises}












%\subsection{Real Spaces Represent\index{representative!of isomorphism classes}
%   \protect$n\protect$--Spaces}
\subsection{Dimension Characterizes Isomorphism}

In the prior subsection, after stating the definition of an isomorphism,
we gave some results supporting the intuition that such a map describes 
spaces as ``the same''. 
Here we will formalize this intuition.
While two spaces that are isomorphic are not equal, we think of them
as almost equal\Dash as equivalent.
In this subsection 
we shall show that the relationship `is isomorphic~to' is an equivalence
relation.\appendrefs{equivalence relations and equivalence classes}

\begin{theorem}
\label{th:IsoEquivRel}
Isomorphism\index{equivalence relation!isomorphism} is an equivalence
relation between vector spaces.
\end{theorem}

\begin{proof}
We must prove that this relation has the three properties of being
symmetric, reflexive, and transitive.
For each of the three we will use item~(2) of 
\nearbylemma{le:PresStructIffPresCombos} and show that the map preserves
structure by showing that it preserves linear combinations
of two members of the domain.

To check reflexivity, that any space is isomorphic to itself,
consider the identity map.
It is clearly one-to-one and onto.
The calculation showing that it preserves linear combinations is easy.
\begin{equation*}
   \mbox{id}(c_1\cdot \vec{v}_1+c_2\cdot \vec{v}_2)
   =c_1\vec{v}_1+c_2\vec{v}_2
   =c_1\cdot \mbox{id}(\vec{v}_1)+c_2\cdot \mbox{id}(\vec{v}_2)
\end{equation*}

To check symmetry, that if $V$ is isomorphic to $W$ via some 
map $\map{f}{V}{W}$ then there is an isomorphism going the other way, 
consider the inverse map $\map{f^{-1}}{W}{V}$. 
As stated in the appendix, such an inverse function exists and
it is also a correspondence.
Thus we have reduced the symmetry issue to checking that, because $f$ 
preserves linear combinations, so also does $f^{-1}$.
Assume that \( \vec{w}_1=f(\vec{v}_1) \) and \( \vec{w}_2=f(\vec{v}_2) \),
i.e., that \( f^{-1}(\vec{w}_1)=\vec{v}_1 \) 
and \( f^{-1}(\vec{w}_2)=\vec{v}_2 \).
\begin{align*}
  f^{-1}(c_1\cdot\vec{w}_1+c_2\cdot\vec{w}_2)
    &=f^{-1}\bigl(\,c_1\cdot f(\vec{v}_1)
                             +c_2\cdot f(\vec{v}_2)\,\bigr) \\
    &=f^{-1}(\,f\bigl(c_1\vec{v}_1+c_2\vec{v}_2)\,\bigr)    \\
    &=c_1\vec{v}_1+c_2\vec{v}_2                             \\ 
    &=c_1\cdot f^{-1}(\vec{w}_1)+c_2\cdot f^{-1}(\vec{w}_2)     
\end{align*}

Finally, we must check transitivity, that if $V$ is isomorphic to $W$ via some
map $f$ and if $W$ is isomorphic to $U$ via some map $g$ then also 
$V$ is isomorphic to $U$.
Consider the composition $\map{\composed{g}{f}}{V}{U}$.
The appendix notes that 
the composition of two correspondences is a correspondence, 
so we need only check that the composition preserves linear combinations.
\begin{align*}
  \composed{g}{f}\,\bigl(c_1\cdot\vec{v}_1+c_2\cdot\vec{v}_2\bigr)
     &=g\bigl(\,f(c_1\cdot \vec{v}_1+c_2\cdot\vec{v}_2)\,\bigr)        \\
     &=g\bigl(\,c_1\cdot f(\vec{v}_1)+c_2\cdot f(\vec{v}_2)\,\bigr) \\ 
     &=c_1\cdot g\bigl(f(\vec{v}_1))+c_2\cdot g(f(\vec{v}_2)\bigr)  \\
     &=c_1\cdot(\composed{g}{f})\,(\vec{v}_1)
                  +c_2\cdot(\composed{g}{f})\,(\vec{v}_2)
\end{align*}
Thus \( \map{\composed{g}{f}}{V}{U}\) is an isomorphism.
\end{proof}

As a consequence of that result, we know that the universe of vector spaces is
partitioned into classes:\index{partitions!into isomorphism classes}
every space is in one and only one isomorphism class.
\begin{center} \small
  \raisebox{.4in}{\begin{tabular}{l}
                    All finite dimensional \\
                    vector spaces:
                  \end{tabular}}
  \includegraphics{ch3.17}
  \raisebox{.4in}{\begin{tabular}{l}
                     $V\isomorphicto W$
                  \end{tabular}}
\end{center}

\begin{theorem} \label{th:NDimSpaceIsoRN}
Vector spaces are isomorphic if and only if they have the same dimension.
\end{theorem}

This follows from the next two lemmas.

\begin{lemma}   \label{lem:IsoImpliesSameDim}
If spaces are isomorphic then they have the same dimension.
\end{lemma}

\begin{proof}
We shall show that an isomorphism of two spaces gives a correspondence 
between their bases.
That is, where \( \map{f}{V}{W} \) is an isomorphism and
a basis for the domain $V$ is 
\( B=\sequence{\vec{\beta}_1,\dots,\vec{\beta}_n} \),
then the image set
\( D=\sequence{f(\vec{\beta}_1),\dots,f(\vec{\beta}_n)} \)
is a basis for the codomain \( W \).
(The other half of the correspondence\Dash that  
for any basis of $W$ the inverse image is a 
basis for $V$\Dash follows on recalling that if $f$ is an isomorphism 
then $f^{-1}$ is also an isomorphism, 
and applying the prior sentence to $f^{-1}$.)

To see that \( D \) spans \( W \), fix any \( \vec{w}\in W \),
note that \( f \) is onto and so there is a \( \vec{v}\in V \) with
\( \vec{w}=f(\vec{v}) \), and expand \( \vec{v} \) as a combination of
basis vectors.
\begin{equation*}
  \vec{w}=f(\vec{v})
  =f(v_1\vec{\beta}_1+\dots+v_n\vec{\beta}_n)  
  =v_1\cdot f(\vec{\beta}_1)+\dots+v_n\cdot f(\vec{\beta}_n)
\end{equation*}
For linear independence of $D$, if
\begin{equation*}
  \zero_W
  =c_1f(\vec{\beta}_1)+\dots+c_nf(\vec{\beta}_n)  
  =f(c_1\vec{\beta}_1+\dots+c_n\vec{\beta}_n)
\end{equation*}
then, since \( f \) is one-to-one and so the only vector sent to 
\( \zero_W \) is \( \zero_V \), we have that
\( \zero_V=c_1\vec{\beta}_1+\dots+c_n\vec{\beta}_n \),
implying that all of the \( c \)'s are zero.
\end{proof}

\begin{lemma} \label{lem:EqDimImpIso}
If spaces have the same dimension then they are isomorphic. 
\end{lemma}

\begin{proof}
To show that any two spaces of dimension~$n$ are isomorphic, 
we can simply show that any one is isomorphic to $\Re^n$.
Then we will have shown that they are isomorphic to each other, 
by the transitivity of isomorphism 
(which was established in \nearbytheorem{th:IsoEquivRel}).

Let $V$ be \( n \)-dimensional.
Fix a basis
\( B=\sequence{\vec{\beta}_1,\dots,\vec{\beta}_n} \) for the domain $V$.
Consider the representation of the 
members of that domain with respect to the basis as a function
from $V$ to $\Re^n$ 
\begin{equation*}
   \vec{v}=v_1\vec{\beta}_1+\dots+v_n\vec{\beta}_n
   \,\mapsunder{\text{\small Rep}_B}\,\colvec{v_1 \\ \vdots \\ v_n}
\end{equation*}
(it is well-defined\appendrefs{well-definedness}\spacefactor1000 since 
every \( \vec{v} \) 
has one and only one such representation\Dash see 
\nearbyremark{not:WellDefFcns} 
below).

This function is one-to-one because if
\begin{equation*}
   \text{Rep}_B(u_1\vec{\beta}_1+\dots+u_n\vec{\beta}_n)
     =\text{Rep}_B(v_1\vec{\beta}_1+\dots+v_n\vec{\beta}_n)
\end{equation*}
then
\begin{equation*}
  \colvec{u_1 \\ \vdots \\ u_n}
    =
  \colvec{v_1 \\ \vdots \\ v_n}
\end{equation*}
and so \( u_1=v_1 \), \ldots, $u_n=v_n$, and therefore the original arguments
\( u_1\vec{\beta}_1+\dots+u_n\vec{\beta}_n \) and 
\( v_1\vec{\beta}_1+\dots+v_n\vec{\beta}_n\) are equal.

This function is onto; any \( n \)-tall vector
\begin{equation*}
   \vec{w}=\colvec{w_1 \\ \vdots \\ w_n}
\end{equation*}
is the image of some \( \vec{v}\in V \), namely
\( \vec{w}=\rep{w_1\vec{\beta}_1+\dots+w_n\vec{\beta}_n}{B}  \).

Finally, this function preserves structure.
\begin{align*}
  \rep{r\cdot\vec{u}+s\cdot\vec{v}}{B}
  &=\rep{\,(ru_1+sv_1)\vec{\beta}_1+\dots+(ru_n+sv_n)\vec{\beta}_n\,}{B}  \\
  &=\colvec{ru_1+sv_1 \\ \vdots \\ ru_n+sv_n}  \\
  &=r\cdot\colvec{u_1 \\ \vdots \\ u_n}+s\cdot\colvec{v_1 \\ \vdots \\ v_n} \\
  &=r\cdot\rep{\vec{u}}{B}+s\cdot\rep{\vec{v}}{B}
\end{align*}
Thus the $\mbox{Rep}_B$~function is an isomorphism and thus
any $n$-dimensional space is isomorphic to the $n$-dimensional space $\Re^n$.
Consequently, 
any two spaces with the same dimension are isomorphic.
\end{proof}

\begin{remark} \label{not:WellDefFcns}
The parenthetical comment in that proof
about the role played by the `one and only one representation' result 
requires some explanation.
We need to show that (for a fixed $B$) 
each vector in the domain is associated by 
$\mbox{Rep}_B$ with one and only one vector in the codomain.

A contrasting example, where an association doesn't have this property, 
is illuminating.
Consider this subset of \( \polyspace_2 \), which is not a basis.
\begin{equation*}
 A=\set{1+0x+0x^2,
              0+1x+0x^2,
              0+0x+1x^2,
              1+1x+2x^2}
\end{equation*}
Call those four polynomials $\vec{\alpha}_1$, \ldots, $\vec{\alpha}_4$.
If, mimicing above proof,
we try to write the members of $\polyspace_2$ as 
\( \vec{p}=c_1\vec{\alpha}_1+c_2\vec{\alpha}_2+
             c_3\vec{\alpha}_3+c_4\vec{\alpha}_4 \),
and associate $\vec{p}$ with the four-tall vector with components
$c_1$, \ldots, $c_4$ then there is a problem. 
For, consider \( \vec{p}(x)=1+x+x^2 \).
The set $A$ spans the space $\polyspace_2$, 
so there is at least one four-tall vector associated with $\vec{p}$.
But $A$ is not linearly independent and so vectors do not have unique
decompositions.
In this case, both
\begin{equation*}
  \vec{p}(x)=1\vec{\alpha}_1+1\vec{\alpha}_2+1\vec{\alpha}_3+0\vec{\alpha}_4 
  \quad\text{and}\quad
  \vec{p}(x)=0\vec{\alpha}_1+0\vec{\alpha}_2-
                1\vec{\alpha}_3+1\vec{\alpha}_4 
\end{equation*}
and so there is more than one four-tall vector associated with $\vec{p}$.
\begin{equation*}
  \colvec[r]{1 \\ 1 \\ 1 \\ 0}
  \quad\text{and}\quad
  \colvec[r]{0 \\ 0 \\ -1 \\ 1}
\end{equation*}
That is,
with input $\vec{p}$ this association does not have a 
well-defined (i.e., single) output value.

Any map whose definition appears possibly ambiguous must be checked to 
see that it is well-defined.
For $\mbox{Rep}_B$ in the above proof 
that check is \nearbyexercise{exer:FcnWellDef}.
\end{remark}

That ends the proof of \nearbytheorem{th:NDimSpaceIsoRN}.
We say that the isomorphism classes are 
\definend{characterized}\index{characterized}%
\index{isomorphism!characterized by dimension} 
by dimension because we can
describe each class simply by giving the number that is the dimension of
all of the spaces in that class.

This subsection's results give us a collection of 
representatives of the isomorphism 
classes. % \appendrefs{equivalence class representatives}\spacefactor1000 %
 
\begin{corollary}
A finite-dimensional vector space is isomorphic to one and only one
of the $\Re^n$.
\end{corollary}

The proofs above pack many ideas into a small space.
Through the rest of this chapter
we'll consider these ideas again, and fill them out.
For a taste of this, we will expand here 
on the proof of \nearbylemma{lem:EqDimImpIso}.

\begin{example}
The space $\matspace_{\nbyn{2}}$ of $\nbyn{2}$ matrices is isomorphic to 
$\Re^4$.
With this basis for the domain
\begin{equation*}
  B=\sequence{\begin{mat}[r]
                1  &0  \\
                0  &0
              \end{mat},
              \begin{mat}[r]
                0  &1  \\
                0  &0
              \end{mat},
              \begin{mat}[r]
                0  &0  \\
                1  &0
              \end{mat},
              \begin{mat}[r]
                0  &0  \\
                0  &1
              \end{mat} }
\end{equation*}
the isomorphism given in the lemma, the representation map $f_1=\mbox{Rep}_B$,
simply carries the entries over.
\begin{equation*}
  \begin{mat}
    a  &b  \\
    c  &d
  \end{mat}
  \mapsunder{f_1}
  \colvec{a \\ b \\ c \\ d}
\end{equation*}
One way to think of the map $f_1$ is:~fix the basis $B$ for 
the domain and the basis $\stdbasis_4$ for the codomain, and 
associate $\vec{\beta}_1$ with $\vec{e}_1$, 
and $\vec{\beta}_2$ with $\vec{e}_2$, etc.
Then extend this association to all of the members of two spaces.
\begin{equation*}
  \begin{mat}
    a  &b  \\
    c  &d
  \end{mat}
  =a\vec{\beta}_1+b\vec{\beta}_2+c\vec{\beta}_3+d\vec{\beta}_4
  \;\;\mapsunder{f_1}\;\;
  a\vec{e}_1+b\vec{e}_2+c\vec{e}_3+d\vec{e}_4
  =\colvec{a \\ b \\ c \\ d}
\end{equation*}
We say that the map has been 
\definend{extended linearly}\index{map!extended linearly}
from the bases to the spaces.

We can do the same thing with different bases, 
for instance, taking this basis for the domain.
\begin{equation*}
  A=\sequence{
              \begin{mat}[r]
                2  &0  \\
                0  &0
              \end{mat},
              \begin{mat}[r]
                0  &2  \\
                0  &0
              \end{mat},
              \begin{mat}[r]
                0  &0  \\
                2  &0
              \end{mat},
              \begin{mat}[r]
                0  &0  \\
                0  &2
              \end{mat} }
\end{equation*}
Associating corresponding members of $A$ and $\stdbasis_4$ and extending
linearly
\begin{multline*}
  \begin{mat}
    a  &b  \\
    c  &d
  \end{mat}
  =(a/2)\vec{\alpha}_1+(b/2)\vec{\alpha}_2
      +(c/2)\vec{\alpha}_3+(d/2)\vec{\alpha}_4          \\
  \mapsunder{f_2}\;\;
  (a/2)\vec{e}_1+(b/2)\vec{e}_2+(c/2)\vec{e}_3+(d/2)\vec{e}_4
  =\colvec{a/2 \\ b/2 \\ c/2 \\ d/2}
\end{multline*}
gives rise to an isomorphism that is different than $f_1$.

The prior map arose by changing the basis for the domain.
We can also change the basis for the codomain.
Starting with
\begin{equation*}
  B
  \quad\text{and}\quad
  D=\sequence{\colvec[r]{1 \\ 0 \\ 0 \\ 0},
                \colvec[r]{0 \\ 1 \\ 0 \\ 0},
                \colvec[r]{0 \\ 0 \\ 0 \\ 1},
                \colvec[r]{0 \\ 0 \\ 1 \\ 0}}
\end{equation*}
associating $\vec{\beta}_1$ with $\vec{\delta}_1$, etc., and then
linearly extending that correspondence to all of the two spaces
\begin{equation*}
  \begin{mat}
     a  &b  \\
     c  &d  
   \end{mat}
  =a\vec{\beta}_1+b\vec{\beta}_2+c\vec{\beta}_3+d\vec{\beta}_4
  \;\;\mapsunder{f_3}\;\;
  a\vec{\delta}_1+b\vec{\delta}_2+c\vec{\delta}_3+d\vec{\delta}_4
  =\colvec{a \\ b \\ d \\ c}
\end{equation*}
gives still another isomorphism.

So there is a connection between the maps between spaces 
and bases for those  spaces. 
Later sections will explore that connection.
\end{example}

We will close this section with a summary.

Recall that in the first chapter 
we defined two matrices as row equivalent if they can be derived 
from each other by elementary row operations (this
was the meaning of same-ness that was of interest there).
We showed that is an equivalence relation and so 
the collection of matrices is partitioned into classes, 
where all the matrices that are 
row equivalent fall together into a single class.
Then, for insight into which matrices are 
in each class, we gave representatives for the 
classes,
the reduced echelon form matrices.

In this section, except that the appropriate notion of same-ness 
here is vector space isomorphism, we have followed much the same outline.
First we defined isomorphism, saw some examples,
and established some properties.
Then we showed that it is an equivalence relation, and 
now we have a set of class representatives, 
the real vector spaces $\Re^1$, $\Re^2$, etc.
\begin{center}
  \raisebox{.4in}{\begin{tabular}{l}
                    All finite dimensional \\
                    vector spaces:
                  \end{tabular}}
  \includegraphics{ch3.18}
  \raisebox{.4in}{\begin{tabular}{l}
                    One representative \\ per class
                  \end{tabular}}
\end{center}
As before, the list of representatives helps us to understand the partition.
It is simply a classification of spaces by dimension.

In the second chapter,
with the definition of vector spaces, we seemed to have opened up our studies
to many examples of new structures besides the familiar $\Re^n$'s.
We now know that isn't the case.
Any finite-dimensional vector space is actually
``the same'' as a real space.
We are thus considering exactly the structures that we need to consider.

The rest of the chapter fills out
the work in this section.
In particular, in the next section we
will consider maps that preserve structure, but are not necessarily
correspondences.




\begin{exercises}
  \recommended \item 
    Decide if the spaces are isomorphic.
    \begin{exparts*}
       \partsitem \( \Re^2 \),~\( \Re^4 \)
       \partsitem \( \polyspace_5 \),~\( \Re^5 \)
       \partsitem \( \matspace_{\nbym{2}{3}} \),~\( \Re^6 \)
       \partsitem \( \polyspace_5 \),~\( \matspace_{\nbym{2}{3}} \)
       \partsitem \( \matspace_{\nbym{2}{k}} \),~\( \C^k \)
    \end{exparts*}
    \begin{answer}
       Each pair of spaces is isomorphic if and only if the two have the 
       same dimension.
       We can, when there is an isomorphism, state
       a map, but it isn't strictly necessary.
       \begin{exparts}
         \partsitem No, they have different dimensions.
         \partsitem No, they have different dimensions.
         \partsitem Yes, they have the same dimension. 
           One isomorphism is this.
           \begin{equation*}
             \begin{mat}
               a  &b  &c  \\
               d  &e  &f
             \end{mat}
             \mapsto
             \colvec{a \\ \vdots \\ f}
           \end{equation*}
         \partsitem Yes, they have the same dimension.
           This is an isomorphism.
           \begin{equation*}
             a+bx+\cdots+fx^5
             \mapsto
             \begin{mat}
               a  &b  &c  \\
               d  &e  &f
             \end{mat}
           \end{equation*}
         \partsitem Yes, both have dimension \( 2k \).      
      \end{exparts}    
    \end{answer}
  \recommended \item 
    Consider the isomorphism
    \( \map{\rep{\cdot}{B}}{\polyspace_1}{\Re^2} \) 
    where \( B=\sequence{1,1+x} \).
    Find the image of each of these elements of the domain.
    \begin{exparts*}
      \partsitem \( 3-2x \);
      \partsitem \( 2+2x \);
      \partsitem \( x \)
    \end{exparts*}
    \begin{answer}
      \begin{exparts*}
        \partsitem \( \rep{3-2x}{B}=\colvec[r]{5 \\ -2} \)
        \partsitem \( \colvec[r]{0 \\ 2} \)
        \partsitem \( \colvec[r]{-1 \\ 1} \)
      \end{exparts*}  
     \end{answer}
  \recommended \item
    Show that if \( m\neq n \) then \( \Re^m\not\isomorphicto\Re^n \).
     \begin{answer}
       They have different dimensions.  
     \end{answer}
  \recommended \item
    Is \( \matspace_{\nbym{m}{n}}\isomorphicto\matspace_{\nbym{n}{m}} \)?
    \begin{answer}
      Yes, both are \( mn \)-dimensional.  
    \end{answer}
  \recommended \item
    Are any two planes through the origin in \( \Re^3 \) isomorphic?
    \begin{answer}
      Yes, any two (nondegenerate) planes are both two-dimensional 
      vector spaces.
    \end{answer}
  \item 
     Find a set of equivalence class representatives other than the
     set of \( \Re^n \)'s.
     \begin{answer}
        There are many answers, one is the set of \( \polyspace_k \)
        (taking \( \polyspace_{-1} \) to be the trivial vector space).  
     \end{answer}
  \item 
    True or false:~between any \( n \)-dimensional space and \( \Re^n \)
    there is exactly one isomorphism.
    \begin{answer}
      False  (except when \( n=0 \)).
      For instance,
      if \( \map{f}{V}{\Re^n} \) is an isomorphism then multiplying by any
      nonzero scalar, gives another, different, isomorphism.
      (Between trivial spaces the isomorphisms are unique; the only map
      possible is $\zero_V\mapsto 0_W$.)  
    \end{answer}
  \item 
    Can a vector space be isomorphic to one of its (proper) subspaces?
    \begin{answer}
      No.
      A proper subspace has a strictly lower dimension than it's superspace;
      if $U$ is a proper subspace of $V$ then any linearly independent subset 
      of $U$ must have fewer than $\dim(V)$ members or else that set would
      be a basis for $V$, and $U$ wouldn't be proper.  
    \end{answer}
  \recommended \item 
    This subsection shows that for any isomorphism, the inverse map is
    also an isomorphism.
    This subsection also shows that for a fixed basis \( B \) of an
    \( n \)-dimensional vector space \( V \), the map
    \( \map{\text{Rep}_B}{V}{\Re^n} \) is an isomorphism.
    Find the inverse of this map.
    \begin{answer}
      Where \( B=\sequence{\vec{\beta}_1,\ldots,\vec{\beta}_n} \), the
      inverse is this.
      \begin{equation*}
        \colvec{c_1 \\ \vdots \\ c_n}
        \mapsto c_1\vec{\beta}_1+\cdots+c_n\vec{\beta}_n
      \end{equation*}  
    \end{answer}
  \recommended \item 
    Prove these facts about matrices.
    \begin{exparts}
      \partsitem The row space of a matrix is isomorphic to the 
        column space of its transpose.
      \partsitem The row space of a matrix is isomorphic to its column space.
    \end{exparts}
    \begin{answer}
      All three spaces have dimension equal to the rank of the matrix.
    \end{answer}
  \item \label{exer:FcnWellDef}
    Show that the function from \nearbytheorem{th:NDimSpaceIsoRN}
    is well-defined.
     \begin{answer}
        We must show that if \( \vec{a}=\vec{b} \) then
        \( f(\vec{a})=f(\vec{b}) \).
        So suppose that 
        $a_1\vec{\beta}_1+\dots+a_n\vec{\beta}_n
           =b_1\vec{\beta}_1+\dots+b_n\vec{\beta}_n$.
        Each vector in a vector space (here, the domain space)
        has a unique representation as a linear combination
        of basis vectors, so we can conclude that \( a_1=b_1 \), \ldots,
        \(a_n=b_n \). 
        Thus,
        \begin{equation*}
          f(\vec{a})
          =\colvec{a_1 \\ \vdots \\ a_n}=\colvec{b_1 \\ \vdots \\ b_n}
          =f(\vec{b})          
        \end{equation*}
        and so the function is well-defined.
     \end{answer}
  \item 
     Is the proof of \nearbytheorem{th:NDimSpaceIsoRN} valid when \( n=0 \)?
     \begin{answer}
       Yes, because a zero-dimensional space is a trivial space.
     \end{answer}
  \item 
    For each, decide if it is a set of isomorphism class representatives.
    \begin{exparts*}
      \partsitem \( \set{\C^k \suchthat k\in\N} \)
      \partsitem $\set{\polyspace_k\suchthat k\in \set{-1,0,1,\ldots}}$
      \partsitem \( \set{\matspace_{\nbym{m}{n}}\suchthat 
                   m,n\in\N} \)
    \end{exparts*}
    \begin{answer}
      \begin{exparts}
        \partsitem No, this collection has no spaces of odd dimension.
        \partsitem Yes, because $\polyspace_{k}\isomorphicto\Re^{k+1}$.
        \partsitem No, for instance,
          \( \matspace_{\nbym{2}{3}}\isomorphicto\matspace_{\nbym{3}{2}} \).
      \end{exparts}  
    \end{answer}
  \item
    Let \( f \) be a correspondence between vector spaces \( V \) and \( W \)
    (that is, a map that is one-to-one and onto).
    Show that the spaces \( V \) and \( W \) are isomorphic via \( f \)
    if and only if there are bases \( B\subset V \)
    and \( D\subset W \) such that corresponding vectors have the same
    coordinates:
    \( \rep{\vec{v}}{B}=\rep{f(\vec{v})}{D} \).
    \begin{answer}
       One direction is easy:~if the two are isomorphic via \( f \)
       then for any basis \( B\subseteq V \),
       the set \( D=f(B) \) is also a basis (this is shown in
       \nearbylemma{lem:IsoImpliesSameDim}).
       The check that corresponding vectors have the same coordinates:
       \( f(c_1\vec{\beta}_1+\dots+c_n\vec{\beta}_n)
          =c_1f(\vec{\beta}_1)+\dots+c_nf(\vec{\beta}_n)
          =c_1\vec{\delta}_1+\dots+c_n\vec{\delta}_n   \)
       is routine.

       For the other half, assume that there are bases such that corresponding
       vectors have the same coordinates with respect to those bases.
       Because \( f \) is a correspondence, to show that it is an isomorphism,
       we need only show that it preserves structure.
       Because \( \rep{\vec{v}\,}{B}=\rep{f(\vec{v}\,)}{D} \), the
       map \( f \) preserves structure if and only if
       representations preserve addition:
       \( \rep{\vec{v}_1+\vec{v}_2}{B}=\rep{\vec{v}_1}{B}+\rep{\vec{v}_2}{B} \)
       and scalar multiplication: 
       \( \rep{r\cdot\vec{v}\,}{B}=r\cdot\rep{\vec{v}\,}{B} \)
       The addition calculation is this:
       \( (c_1+d_1)\vec{\beta}_1+\dots+(c_n+d_n)\vec{\beta}_n
          =c_1\vec{\beta}_1+\dots+c_n\vec{\beta}_n
          +d_1\vec{\beta}_1+\dots+d_n\vec{\beta}_n \),
       and the scalar multiplication calculation is similar.
     \end{answer}
  \item 
    Consider the isomorphism \( \map{\text{Rep}_B}{\polyspace_3}{\Re^4} \).
    \begin{exparts}
      \partsitem Vectors in a real space are orthogonal if and only if
        their dot product is zero.
        Give a definition of orthogonality for polynomials.
      \partsitem The derivative of a member of \( \polyspace_3 \) is in
        \( \polyspace_3 \).
        Give a definition of the derivative of a vector in \( \Re^4 \).
    \end{exparts}
    \begin{answer}
       \begin{exparts}
        \partsitem Pulling the definition back from 
          \( \Re^4 \) to \( \polyspace_3 \)
          gives that \( a_0+a_1x+a_2x^2+a_3x^3 \) is orthogonal to
          \( b_0+b_1x+b_2x^2+b_3x^3 \) if and only if
          \( a_0b_0+a_1b_1+a_2b_2+a_3b_3=0 \).
        \partsitem A natural definition is this.
          \begin{equation*}
            D(\colvec{a_0 \\ a_1 \\ a_2 \\ a_3})=
              \colvec{a_1 \\ 2a_2 \\ 3a_3 \\ 0}
          \end{equation*}
      \end{exparts}   
    \end{answer}
  \recommended \item
    Does every correspondence between bases, when extended to the 
    spaces, give an isomorphism?
    \begin{answer}
      Yes.

      Assume that \( V \) is a vector space with basis
      \( B=\sequence{\vec{\beta}_1,\ldots,\vec{\beta}_n} \) and that \( W \)
      is another vector space such that the map \( \map{f}{B}{W} \) is
      a correspondence.
      Consider the extension \( \map{\hat{f}}{V}{W} \) of \( f \).
      \begin{equation*}
        \hat{f}(c_1\vec{\beta}_1+\cdots+c_n\vec{\beta}_n)=
        c_1f(\vec{\beta}_1)+\cdots+c_nf(\vec{\beta}_n).
      \end{equation*}
      The map \( \hat{f} \) is an isomorphism.

      First, \( \hat{f} \) is well-defined because every member of \( V \)
      has one and only one representation as a linear combination of elements
      of \( B \).

      Second, \( \hat{f} \) is one-to-one because every member of \( W \) has
      only one representation as a linear combination of elements of
      \( \sequence{f(\vec{\beta}_1),\dots,f(\vec{\beta}_n)} \).
      That map \( \hat{f} \) is onto because every member of \( W \) has at
      least one representation as a linear combination of members of
      \( \sequence{f(\vec{\beta}_1),\dots,f(\vec{\beta}_n)} \).

      Finally, preservation of structure is routine to check.
      For instance, here is the preservation of addition calculation.
      \begin{align*}
        \hat{f}(\,(c_1\vec{\beta}_1+\dots+c_n\vec{\beta}_n)+
                  (d_1\vec{\beta}_1+\dots+d_n\vec{\beta}_n)\,)
        &=\hat{f}(\,(c_1+d_1)\vec{\beta}_1+\dots+(c_n+d_n)\vec{\beta}_n\,) \\
        &=(c_1+d_1)f(\vec{\beta}_1)+\dots+(c_n+d_n)f(\vec{\beta}_n) \\
        &=c_1f(\vec{\beta}_1)+\dots+c_nf(\vec{\beta}_n)
          +d_1f(\vec{\beta}_1)+\dots+d_nf(\vec{\beta}_n) \\
        &=\hat{f}(c_1\vec{\beta}_1+\dots+c_n\vec{\beta}_n)+
          +\hat{f}(d_1\vec{\beta}_1+\dots+d_n\vec{\beta}_n).
      \end{align*}
      Preservation of scalar multiplication is similar.   
    \end{answer}
  \item 
    \textit{(Requires the subsection on Combining Subspaces,
    which is optional.)}
    Suppose that \( V=V_1\directsum V_2 \) and that 
    \( V \) is isomorphic to the space \( U \) under the map \( f \).
    Show that \( U=f(V_1)\directsum f(U_2) \).
    \begin{answer}
      Because \( V_1\intersection V_2=\set{\zero_V} \) and \( f \) is
      one-to-one we have that \( f(V_1)\intersection f(V_2)=\set{\zero_U} \).
      To finish, count the dimensions:
      \( \dim(U)=\dim(V)=\dim(V_1)+\dim(V_2)=\dim(f(V_1))+\dim(f(V_2)) \),
      as required.  
    \end{answer}
  \item Show that this is not a well-defined function from the rational numbers
    to the integers: with each fraction, associate the value of its numerator.
    \begin{answer}
      Rational numbers have many representations, e.g.,
      \( 1/2=3/6 \), and the numerators can vary among
      representations.
    \end{answer}
\index{isomorphism|)}
\end{exercises}
