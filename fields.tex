% Chapter 2, Topic _Linear Algebra_ Jim Hefferon
%  http://joshua.smcvt.edu/linalg.html
%  2001-Jun-11
\topic{Fields}
\index{field|(}
Computations involving only integers or only rational numbers are much
easier than those with real numbers.
Could other algebraic structures, such as the integers or the rationals,
work in the
place of \( \Re \) in the definition of a vector space?

Yes and no.
If we take ``work'' to mean that the results of this chapter
remain true
then an analysis of the properties of the reals that we have
used in this chapter gives a list of conditions that a structure
needs in order to ``work'' in the place of $\Re$.

\begin{shadebox}
\begin{trivlist}
\item[\hskip \labelsep{\bf Definition.}]
A \definend{field\/}\index{field!definition} is a set 
\( \F \) with two operations
`\( + \)'
and `\( \cdot \)' such that
\begin{enumerate}
   \item for any \( a,b\in\F \) the result of \( a+b \) is in \( \F \) and
      \begin{itemize}
         \item \( a+b=b+a \)
         \item if \( c\in\F \) then \( a+(b+c)=(a+b)+c \)
      \end{itemize}
   \item for any \( a,b\in\F \) the result of \( a\cdot b \) is in \( \F \) and
      \begin{itemize}
         \item \( a\cdot b=b\cdot a \)
         \item if \( c\in\F \) then \( a\cdot (b\cdot c)=(a\cdot b)\cdot c \)
      \end{itemize}
   \item if \( a,b,c\in\F \) then \( a\cdot (b+c)=a\cdot b+a\cdot c \)
   \item there is an element \( 0\in\F \) such that
      \begin{itemize}
         \item if \( a\in\F \) then \( a+0=a \)
         \item for each \( a\in\F \) there is an element \( -a\in\F \)
                such that \( (-a)+a=0 \)
      \end{itemize}
   \item there is an element \( 1\in\F \) such that
      \begin{itemize}
         \item if \( a\in\F \) then \( a\cdot 1=a \)
         \item for each element \( a\neq 0 \) of \( \F \)
                there is an element \( a^{-1}\in\F \)
                such that \( a^{-1}\cdot a=1 \).
      \end{itemize}
\end{enumerate}
\end{trivlist}
\end{shadebox}

The algebraic structure 
consisting of the set of real numbers along with its usual
addition and multiplication operation is a field, naturally.
Another field is the set of rational numbers with its usual addition
and multiplication operations.
An example of an algebraic structure that is not a field is
the integers, because it fails the final condition.

Some examples are surprising.
The set \( \set{0,1} \) under these operations:
\begin{center}
  \begin{tabular}{c|cc}
    \( + \) &\( 0 \) &\( 1 \) \\
    \hline
    \( 0 \) &\( 0 \) &\( 1 \) \\
    \( 1 \) &\( 1 \) &\( 0 \)
  \end{tabular}
  \qquad
  \begin{tabular}{c|cc}
    \( \cdot \) &\( 0 \) &\( 1 \) \\
     \hline
       \( 0 \)  &\( 0 \) &\( 0 \)  \\
       \( 1 \)  &\( 0 \) &\( 1 \)
  \end{tabular}
\end{center}
is a field (see \nearbyexercise{exer:BinField}).

We could in this book develop Linear Algebra as the theory of
vector spaces with scalars from an arbitrary field.
In that case, 
almost all of the statements here would carry over by replacing
`\( \Re \)' with `\( \F \)', that is, by
taking coefficients, vector entries,
and matrix entries to be elements of \( \F \)
(the exceptions are statements involving distances or angles).
Here are some examples; each applies to a vector space \( V \)
over a field \( \F \).
\begin{itemize}
  %\makebox[\parindent]{\ \hfil\ }
  \item[$*$] For any \( \vec{v}\in V \) and \( a\in\F \),
  (i)~\( 0\cdot\vec{v}=\zero \),
  (ii)~\( -1\cdot\vec{v}+\vec{v}=\zero \),
  and (iii)~\( a\cdot\vec{0}=\vec{0} \).

  \item[$*$] The span, the set of linear combinations, of a subset of \( V \)
  is a subspace of \( V \).

  \item[$*$] Any subset of a linearly independent set is also 
     linearly independent.

  \item[$*$] In a finite-dimensional vector space, 
    any two bases have the same number of elements.
\end{itemize}
(Even statements that don't explicitly mention \( \F \) use
field properties in their proof.)

We will not develop vector spaces in this more general setting because
the additional abstraction can be a distraction.
The ideas we want to bring out already appear when we stick to the reals.

The only exception is Chapter Five.
There we must factor polynomials,
so we will switch to considering vector spaces over the
field of complex numbers.

\begin{exercises}
  \item 
    Show that the real numbers form a field.
    \begin{answer}
      These checks are all routine; most consist only of remarking that 
      property is so familiar that it does not need to be proved.
    \end{answer}
  \item 
    Prove that these are fields.
    \begin{exparts*}
       \partsitem The rational numbers $\Q$
       \partsitem The complex numbers  $\C$
    \end{exparts*}
    \begin{answer}
      For both of these structures, these checks are all routine.
      As with the prior question, most of the checks consist only of remarking
      that property is so familiar that it does not need to be proved.
    \end{answer}
  \item 
     Give an example that shows that the integer number system
     is not a field.
     \begin{answer}
       There is no multiplicative inverse for $2$ so the integers do not
       satisfy condition~(5).
     \end{answer}
  \item \label{exer:BinField} 
     Consider the set $\mathcal{B}=\set{0,1}$ subject to the operations 
     given above.
     Show that it is a field.
     \begin{answer}
       These checks can be done by listing all of the possibilities.
       For instance, to verify the commutativity of addition, that $a+b=b+a$,
       we can easily check it for all possible pairs $a$, $b$, because there
       are only four such pairs.  
       Similarly, for associativity, there are only eight triples $a$, $b$,
       $c$, and so the check is not too long.
       (There are other ways to do the checks, in particular, a reader may 
       recognize these operations as arithmetic `mod~$2$'.)
     \end{answer}
  \item 
     Give suitable operations to make the set $\set{0,1,2}$
     a field.     
     \begin{answer}
       These will do.
       \begin{center}
          \begin{tabular}{c|ccc}
             \( + \) &\( 0 \) &\( 1 \) &$2$ \\
             \hline
             \( 0 \) &\( 0 \) &\( 1 \) &$2$ \\
             \( 1 \) &\( 1 \) &\( 2 \) &$0$ \\
             $2$     &$2$     &$0$     &$1$
          \end{tabular}
          \qquad
          \begin{tabular}{c|ccc}
            \( \cdot \) &\( 0 \) &\( 1 \) &$2$ \\
            \hline
            \( 0 \)  &\( 0 \) &\( 0 \) &$0$ \\
            \( 1 \)  &\( 0 \) &\( 1 \) &$2$ \\
            $2$      &$0$     &$2$     &$1$
          \end{tabular}
       \end{center}
       As in the prior item, the check that they satisfy the conditions can 
       be done by listing all of the cases, although this way of checking is
       somewhat long (making use of commutativity is helpful in 
       shortening the work).
     \end{answer}
\end{exercises}
\index{field|)}
%\include{la}
%\end{document}
