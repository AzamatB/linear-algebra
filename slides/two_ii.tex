% see: https://groups.google.com/forum/?fromgroups#!topic/comp.text.tex/s6z9Ult_zds
\makeatletter\let\ifGm@compatii\relax\makeatother 
\documentclass[10pt,t,serif,professionalfont]{beamer}
\PassOptionsToPackage{pdfpagemode=FullScreen}{hyperref}
\PassOptionsToPackage{usenames,dvipsnames}{color}
% \DeclareGraphicsRule{*}{mps}{*}{}
\usepackage{../linalgjh}
\usepackage{present}
\usepackage{xr}\externaldocument{../vs2} % read refs from .aux file
\usepackage{xr}\externaldocument{../gr3} % read refs from .aux file
\usepackage{catchfilebetweentags}
\usepackage{etoolbox} % from http://tex.stackexchange.com/questions/40699/input-only-part-of-a-file-using-catchfilebetweentags-package
\makeatletter
\patchcmd{\CatchFBT@Fin@l}{\endlinechar\m@ne}{}
  {}{\typeout{Unsuccessful patch!}}
\makeatother

\mode<presentation>
{
  \usetheme{boxes}
  \setbeamercovered{invisible}
  \setbeamertemplate{navigation symbols}{} 
}
\addheadbox{filler}{\ }  % create extra space at top of slide 
\hypersetup{colorlinks=true,linkcolor=blue} 

\title[Linear Independence] % (optional, use only with long paper titles)
{Two.II Linear Independence}

\author{\textit{Linear Algebra} \\ {\small Jim Hef{}feron}}
\institute{
  \texttt{http://joshua.smcvt.edu/linearalgebra}
}
\date{}


\subject{Linear Independence}
% This is only inserted into the PDF information catalog. Can be left
% out. 

\begin{document}
\begin{frame}
  \titlepage
\end{frame}

% =============================================
% \begin{frame}{Reduced Echelon Form} 
% \end{frame}



% ..... Two.I.1 .....
\section{Definition and examples}
%..........
\begin{frame}{Linear independence}
\df[def:LinInd]
\ExecuteMetaData[../vs2.tex]{df:LinInd}

\pause\medskip
\ExecuteMetaData[../vs2.tex]{LinInd}
\end{frame}



%..........
\begin{frame}
\lm[le:LDIffANonTrivLinRel]
\ExecuteMetaData[../vs2.tex]{lm:LDIffANonTrivLinRel}

\pause
\pf
\ExecuteMetaData[../vs2.tex]{pf:LDIffANonTrivLinRel}
\qed
\end{frame}



%..........
\begin{frame}
\ex 
This set of vectors in the plane $\Re^2$ is linearly independent.
\begin{equation*}
  \set{\colvec{1 \\ 0},
       \colvec{0 \\ 1}}
\end{equation*}
The only solution to this equation
\begin{equation*}
  c_1\colvec{1 \\ 0}+c_2\colvec{0 \\ 1}=\colvec{0 \\ 0}
\end{equation*}
is the trivial solution $c_1=0$, $c_2=0$.
\pause
\ex In the vector space of cubic polynomials 
$\polyspace_3=\set{a_0+a_1x+a_2x^2+a_3x^3\suchthat a_i\in\Re}$ the set
$\set{1-x,1+x^2}$ is linearly independent.
The equation
$c_0(1-x)+c_1(1+x^2)=0$ leads to this linear system
\begin{equation*}
  \begin{linsys}{2}
    c_0 &- &c_1 &=  &0 \\
    c_0 &+ &c_1 &=  &0
  \end{linsys}
\end{equation*}
which has only the trivial solution.
\end{frame}



%..........
\begin{frame}
\ex
The nonzero rows of this matrix form a linearly independent set.
\begin{equation*}
  \begin{mat}[r]
    2 &0  &1   &-1  \\
    0 &1  &-3  &1/2  \\
    0 &0  &0   &5    \\
    0 &0  &0   &0
  \end{mat}
\end{equation*}
We showed in Lemma~One.III.\ref{le:EchFormNoLinCombo} that for any
echelon from matrix the nonzero
rows make a linearly independent set. 

\ex
This subset of $\Re^3$ is linearly dependent.
\begin{equation*}
  \set{\colvec{1  \\ 1 \\ 3}, 
       \colvec{-1 \\ 1 \\ 0},
       \colvec{1  \\ 3 \\ 6}}
\end{equation*}
One way to show that is to spot that the third vector is twice the first plus 
the second.
Another way is to solve the linear system
\begin{equation*}
  \begin{linsys}{3}
    c_1  &-  &c_2  &+  &c_3    &=  &0  \\
    c_1  &+  &c_2  &+  &3c_3   &=  &0  \\
    3c_1 &   &     &+  &6c_3   &=  &0
  \end{linsys}
\end{equation*}
and note that there are more solutions than just the trivial one.
\end{frame}



%..........
\begin{frame}
\lm[lm:ExpandSpanByAddingNonRepeat]
\ExecuteMetaData[../vs2.tex]{lm:ExpandSpanByAddingNonRepeat}

\pause
\pf
\ExecuteMetaData[../vs2.tex]{pf:ExpandSpanByAddingNonRepeat0}

\pause
\ExecuteMetaData[../vs2.tex]{pf:ExpandSpanByAddingNonRepeat1}
\end{frame}\begin{frame}
\ExecuteMetaData[../vs2.tex]{pf:ExpandSpanByAddingNonRepeat2}
\qed
\end{frame}



%..........
\begin{frame}
\ex 
In $\polyspace_2$ consider the set $S=\set{1-x,1+x}$.
The span $\spanof{S}$ is the subset of
linear polynomials $\set{a+bx\suchthat a,b\in\Re}$.
(The span is a subset of the linear polynomials because no member of
$S$ has a quadratic term.
To see that the span equals the linear polynomials, solve
the equation $a+bx=c_1(1-x)+c_2(1+x)$ to get $a=c_1-c_2$ and 
$b=-c_1+c_2$.)  

\pause
If we add a linear polynomial $S_1=S\union\set{2+2x}$
then the span is unchanged $\spanof{S}=\spanof{S_1}$.

\pause
We can enlarge the span by adding a quadratic polynomial
$S_2=S\union\set{2+x^2}$.
The span of $S_2$ is all of $\polyspace_2$.
\end{frame}



%..........
\begin{frame}
\co[th:AlwaysAnLDSubset]
\ExecuteMetaData[../vs2.tex]{th:AlwaysAnLDSubset}

\pause
\pf
\ExecuteMetaData[../vs2.tex]{pf:AlwaysAnLDSubset0}

\pause
\ExecuteMetaData[../vs2.tex]{pf:AlwaysAnLDSubset1}

\pause
\ExecuteMetaData[../vs2.tex]{pf:AlwaysAnLDSubset2}
\qed
\end{frame}



%..........
\begin{frame}
\co[cor:LDMeansLC]
\ExecuteMetaData[../vs2.tex]{co:LDMeansLC}

\pause
\pf
\ExecuteMetaData[../vs2.tex]{pf:LDMeansLC}
\qed
\end{frame}




%..................
\begin{frame}{Linear independence and subset}
\lm[le:SubsetPreserveLI]
\ExecuteMetaData[../vs2.tex]{lm:SubsetPreserveLI}

\pf
\ExecuteMetaData[../vs2.tex]{pf:SubsetPreserveLI}
\qed

\pause
\ExecuteMetaData[../vs2.tex]{IndependenceAndSubsetTable}
\end{frame}



%...........................
% \begin{frame}
% \ExecuteMetaData[../gr3.tex]{GaussJordanReduction}
% \df[def:RedEchForm]
% 
% \end{frame}
\end{document}
