% see: https://groups.google.com/forum/?fromgroups#!topic/comp.text.tex/s6z9Ult_zds
\makeatletter\let\ifGm@compote\relax\makeatother 
\documentclass[10pt,t]{beamer}
\usefonttheme{professionalfonts}
\usefonttheme{serif}
\PassOptionsToPackage{pdfpagemode=FullScreen}{hyperref}
\PassOptionsToPackage{usenames,dvipsnames}{color}
% \DeclareGraphicsRule{*}{mps}{*}{}
\usepackage{../linalgjh}
\usepackage{present}
\usepackage{xr}\externaldocument{../jc1} % read refs from .aux file
\usepackage{catchfilebetweentags}
\usepackage{etoolbox} % from http://tex.stackexchange.com/questions/40699/input-only-part-of-a-file-using-catchfilebetweentags-package
\makeatletter
\patchcmd{\CatchFBT@Fin@l}{\endlinechar\m@ne}{}
  {}{\typeout{Unsuccessful patch!}}
\makeatother

\usepackage{polynom}  % for polynomial long division

\mode<presentation>
{
  \usetheme{boxes}
  \setbeamercovered{invisible}
  \setbeamertemplate{navigation symbols}{} 
}
\addheadbox{filler}{\ }  % create extra space at top of slide 
\hypersetup{colorlinks=true,linkcolor=blue} 

\title[Complex Vector Spaces] % (optional, use only with long paper titles)
{Five.I Complex Vector Spaces}

\author{\textit{Linear Algebra} \\ {\small Jim Hef{}feron}}
\institute{
  \texttt{http://joshua.smcvt.edu/linearalgebra}
}
\date{}


\subject{Complex Vector Spaces}
% This is only inserted into the PDF information catalog. Can be left
% out. 

\begin{document}
\begin{frame}
  \titlepage
\end{frame}

% =============================================
% \begin{frame}{Reduced Echelon Form} 
% \end{frame}



% ..... Five.I .....
\section{Chapter Five.  Similarity}
%..........
\begin{frame}{Scalars will now be complex}
\ExecuteMetaData[../jc1.tex]{ReasonForShiftToComplexNumbers}

% We start with a review of polynomials, factoring, and complex numbers.
% Because it is a review, we leave off some proofs.  
\end{frame}




\section{Factoring and Complex Numbers}
\begin{frame}{Division Theorem for Polynomials}
\ExecuteMetaData[../jc1.tex]{PolynomialReview0}  

\pause
\ExecuteMetaData[../jc1.tex]{PolynomialReview1}  

\pause
\ex
\begin{equation*}
  \only<2>{\polylongdiv[stage=1]{3x^3+2x^2-x+4}{x^2+x}}
  \only<3>{\polylongdiv[stage=2]{3x^3+2x^2-x+4}{x^2+x}}
  \only<4>{\polylongdiv[stage=3]{3x^3+2x^2-x+4}{x^2+x}}
  \only<5>{\polylongdiv[stage=4]{3x^3+2x^2-x+4}{x^2+x}}
  \only<6>{\polylongdiv[stage=5]{3x^3+2x^2-x+4}{x^2+x}}
  \only<7>{\polylongdiv[stage=6]{3x^3+2x^2-x+4}{x^2+x}}
  \only<8->{\polylongdiv[stage=8]{3x^3+2x^2-x+4}{x^2+x}}
\end{equation*}
\only<8->{So, $x^2+x$ goes $3x-1$ times into $3x^3+2x^2-x+4$ with remainder~$4$.
In $n=dq+r$ form:
$3x^3+2x^2-x+4=(x^2+x)\cdot(3x-1)+4$.}
\end{frame}




\begin{frame}
\th[th:EuclidForPolys]
\ExecuteMetaData[../jc1.tex]{th:EuclidForPolys}  
\pause
\co[co:PolyDividedByLinearPolyIsConstant]
\ExecuteMetaData[../jc1.tex]{co:PolyDividedByLinearPolyIsConstant}  

\pause
\pf
\ExecuteMetaData[../jc1.tex]{co:PolyDividedByLinearPolyIsConstant}  
\qed

\pause
\ExecuteMetaData[../jc1.tex]{PolynomialFactor}  

\co[co:RootOfPolyIsAssocLinearFactor]
\ExecuteMetaData[../jc1.tex]{co:RootOfPolyIsAssocLinearFactor}

\pause
\pf  
\ExecuteMetaData[../jc1.tex]{pf:RootOfPolyIsAssocLinearFactor}  
\qed
\end{frame}




\begin{frame}{Factoring over the real numbers}
\th[th:CubicsAndHigherFactor]
\ExecuteMetaData[../jc1.tex]{th:CubicsAndHigherFactor}

\pause
\co[co:RealPolysFactorIntoLinearsAndQuads]  
\ExecuteMetaData[../jc1.tex]{co:RealPolysFactorIntoLinearsAndQuads}
\end{frame}




\begin{frame}{Factoring over the complex numbers}
\ExecuteMetaData[../jc1.tex]{ComplexNumbers}

\pause
\th[th:FundThmAlg]\textcolor{blue}{[Fundamental Theorem of Algebra]}  
\ExecuteMetaData[../jc1.tex]{th:FundThmAlg}
\end{frame}




\section{Complex Representations}
\begin{frame}
\ExecuteMetaData[../jc1.tex]{ComplexOperations}  

With those rules for scalars, all of
the operations that we've covered
for real vector spaces carry over unchanged.

\ex
\begin{equation*}
  \begin{mat}
    2-i  &1+i \\
    i    &4
  \end{mat}
  \begin{mat}
    0    &3+3i \\
    1-i  &2
  \end{mat}
  =
  \begin{mat}
    2    &9+3i \\
    4-4i &5+3i
  \end{mat}
\end{equation*}
\end{frame}



\begin{frame}
We shall carry over unchanged from the previous work 
everything else that we can.
For instance, this
\begin{equation*}
   \sequence{\colvec{1+0i \\ 0+0i \\ \vdots \\ 0+0i},
             \dots,
             \colvec{0+0i \\ 0+0i \\ \vdots \\ 1+0i}}
\end{equation*}
is the \definend{standard basis\/}\index{basis!standard}%
\index{basis!standard over the complex numbers} 
for \( \C^n \) as a vector space over $\C$ and
we denote it \( \stdbasis_n \).
\end{frame}


%...........................
% \begin{frame}g
% \ExecuteMetaData[../gr3.tex]{GaussJordanReduction}
% \df[def:RedEchForm]
% 
% \end{frame}
\end{document}
