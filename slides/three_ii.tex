% see: https://groups.google.com/forum/?fromgroups#!topic/comp.text.tex/s6z9Ult_zds
\makeatletter\let\ifGm@compatii\relax\makeatother 
\documentclass[10pt,t,serif,professionalfont]{beamer}
\PassOptionsToPackage{pdfpagemode=FullScreen}{hyperref}
\PassOptionsToPackage{usenames,dvipsnames}{color}
% \DeclareGraphicsRule{*}{mps}{*}{}
\usepackage{../linalgjh}
\usepackage{present}
\usepackage{xr}\externaldocument{../map2} % read refs from .aux file
\usepackage{xr}\externaldocument{../vs3} % read refs from .aux file
\usepackage{catchfilebetweentags}
\usepackage{etoolbox} % from http://tex.stackexchange.com/questions/40699/input-only-part-of-a-file-using-catchfilebetweentags-package
\makeatletter
\patchcmd{\CatchFBT@Fin@l}{\endlinechar\m@ne}{}
  {}{\typeout{Unsuccessful patch!}}
\makeatother

\mode<presentation>
{
  \usetheme{boxes}
  \setbeamercovered{invisible}
  \setbeamertemplate{navigation symbols}{} 
}
\addheadbox{filler}{\ }  % create extra space at top of slide 
\hypersetup{colorlinks=true,linkcolor=blue} 

\title[Homomorphisms] % (optional, use only with long paper titles)
{Three.II Homomorphisms}

\author{\textit{Linear Algebra} \\ {\small Jim Hef{}feron}}
\institute{
  \texttt{http://joshua.smcvt.edu/linearalgebra}
}
\date{}


\subject{Homomorphisms}
% This is only inserted into the PDF information catalog. Can be left
% out. 

\begin{document}
\begin{frame}
  \titlepage
\end{frame}

% =============================================
% \begin{frame}{Reduced Echelon Form} 
% \end{frame}



% ..... Three.II.1 .....
\section{Definition}
%..........
\begin{frame}{Homomorphism}
\df[def:Homo]
\ExecuteMetaData[../map2.tex]{df:Homo}
\end{frame}




%..........
\begin{frame}
\ex
The function $\map{h}{\polyspace_2}{\Re^2}$ given by
\begin{equation*}
  h(a+bx+cx^2)
  =
  \colvec{a+c \\ 0}
\end{equation*}
is a homomorphism.
To check that we need to verify that it respects the addition and
scalar multiplication operations.

\pause
Addition is routine.
\begin{gather*}
  h(\,(a_1+b_1x+c_1x^2)+(a_2+b_2x+c_2x^2)\,)      \\
  \begin{align*}
    \qquad
    &=h(\,(a_1+a_2)+(b_1+b_2)x+(c_1+c_2)x^2\,)     \\
    &=\colvec{a_1+a_2+c_1+c_2 \\ 0}                 \\
    &=\colvec{a_1+c_1 \\ 0}+\colvec{a_2+c_2 \\ 0}   \\
    &=h(a_1+b_1x+c_1x^2)+h(a_2+b_2x+c_2x^2)       
  \end{align*}
\end{gather*}
\pause
So is scalar multiplication.
\begin{equation*}
  r\cdot h(a+bx+cx^2)
  =r\cdot\colvec{a+c \\ 0}  
  =\colvec{ra+rc \\ 0}
  =h(\,r(a+bx+cx^2)\,)
\end{equation*}
\end{frame}



%..........
\begin{frame}
\ex
Of these two maps $\map{h,g}{\Re^2}{\Re}$
the first is linear while the second is not.
\begin{equation*}
  \colvec{x \\ y}\mapsunder{h} 2x-3y
  \qquad
  \colvec{x \\ y}\mapsunder{g} 2x-3y+1
\end{equation*}

\pause
The map $h$ respects addition
\begin{multline*}
  h(\colvec{x_1 \\ y_1}+\colvec{x_2 \\ y_2})
  =h(\colvec{x_1+x_2 \\ y_1+y_2})             
  =2(x_1+x_2)-3(y_1+y_2)                    \\
  =(2x_1-3y_1)+(2x_2-3y_2)
  =h(\colvec{x_1 \\ y_1})+h(\colvec{x_2 \\ y_2})
\end{multline*}
and scalar multiplication.
\begin{equation*}
  r\cdot h(\colvec{x \\ y})
  =r\cdot(2x-3y)
  =2rx-3ry
  =(2r)x-(3r)y
  =h(r\cdot\colvec{x \\ y})
\end{equation*}

\pause
This example shows that $g$ does not respect addition.
\begin{equation*} 
  g(\colvec{1 \\ 4}+\colvec{5 \\ 6})=-17
  \quad\text{while}\quad
  g(\colvec{1 \\ 4})+g(\colvec{5 \\ 6})=-16
\end{equation*}
\end{frame}




%..........
\begin{frame}
We proved these two in the context of studying isomorphisms.

\lm[le:HomoSendsZeroToZero]
\ExecuteMetaData[../map2.tex]{lm:HomoSendsZeroToZero}

\lm[le:HomoPreserveLinCombo]
\ExecuteMetaData[../map2.tex]{lm:HomoPreserveLinCombo}

\pause
\medskip
\ex
Between any two vector spaces the zero map $\map{Z}{V}{W}$, defined by
$Z(\vec{v})=\zero_W$ is a homomorphism.
The check is: 
$Z(c_1\vec{v}_1+c_2\vec{v}_2)=\zero_W
   =\zero_W+\zero_W=c_1Z(\vec{v}_1)+c_2Z(\vec{v}_2)$.
\end{frame}




%..........
\begin{frame}
\ex
The \definend{inclusion map} $\map{\iota}{\Re^2}{\Re^3}$
\begin{equation*}
  \iota(\colvec{x  \\ y})
  =\colvec{x \\ y \\ z}
\end{equation*}
is a homomorphism.
Here is the verification.
\begin{align*}
  \iota(c_1\cdot\colvec{x_1 \\ y_1}+c_2\cdot\colvec{x_2 \\ y_2})
  &=\iota(\colvec{c_1x_1+c_2x_2 \\ c_1y_1+c_2y_2})       \\
  &=\colvec{c_1x_1+c_2x_2 \\ c_1y_1+c_2y_2 \\ 0}      \\
  &=\colvec{c_1x_1 \\ c_1y_1 \\ 0}
   +\colvec{c_2x_2 \\ c_2y_2 \\ 0}                    \\
  &=c_1\cdot\iota(\colvec{x_1 \\ y_1})+c_2\cdot\iota(\colvec{x_2 \\ y_2})
\end{align*}
\end{frame}




%..........
\begin{frame}
\ex
Consider 
the function $\map{h}{\polyspace_1}{\polyspace_1}$ defined below.
\begin{equation*}
  h(a+bx)=b+bx
\end{equation*}
Here are two examples of the action of this function: $h(1+2x)=2+2x$
and $h(3-x)=-1-x$. 

\pause
This function is linear.
\begin{multline*}
  h(\,c_1\cdot (a_1+b_1x)+c_2\cdot(a_2+b_2x)\,)                  \\
  \begin{aligned}
    &=h(\,(c_1a_1+c_2a_2)+(c_1b_1+c_2b_2)x\,)       \\
    &=(c_1b_1+c_2b_2)+(c_1b_1+c_2b_2)x          \\
    &=(c_1b_1+c_1b_1x)+(c_2b_2+c_2b_2x)         \\
    &=c_1\cdot h(a_1+b_1x)+c_2\cdot h(a_2+b_2x)
  \end{aligned}
\end{multline*}
\end{frame}




%..........
\begin{frame}
\ex
The derivative map $\map{d/dx}{\polyspace_2}{\polyspace_1}$
is given by $d/dx\,(ax^2+bx+c)=2ax+b$.
For instance, $d/dx\,(3x^2-2x+4)=6x-2$
and $d/dx\,(x^2+1)=2x$.
\pause
It is a homomorphism.
\begin{multline*}
  d/dx\,\big(\,r_1(a_1x^2+b_1x+c_1)+r_2(a_2x^2+b_2x+c_2)\,\big)  \hspace*{5em}  \\
  \begin{aligned} 
    &=d/dx\,\big(\,(r_1a_1+r_2a_2)x^2+(r_1b_1+r_2b_2)x+(r_1c_1+r_2c_2)\,\big)   \\
    &=2(r_1a_1+r_2a_2)x+(r_1b_1+r_2b_2)   \\
    &=(2r_1a_1x+r_1b_1)+(2r_2a_2x+r_2b_2)   \\
    &=r_1\cdot d/dx\,(a_1x^2+b_1x+c_1)
      +r_2\cdot d/dx\,(a_2x^2+b_2x+c_2)
  \end{aligned}
\end{multline*}
\end{frame}



%..........
\begin{frame}
The \definend{trace} of a square matrix 
is the sum down the uppper-left to lower-right diagonal.
Thus
$\map{\trace}{\matspace_{\nbyn{2}}}{\Re}$
is this.
\begin{equation*}
  \trace(\begin{mat}
    a &b \\
    c &d
  \end{mat})
  =a+b
\end{equation*}
It is linear.
\begin{multline*}
  \trace(\,
  r_1\cdot\begin{mat}
    a_1 &b_1 \\
    c_1 &d_1
  \end{mat}
  +
  r_2\cdot\begin{mat}
    a_2 &b_2 \\
    c_2 &d_2
  \end{mat}
  \,)                                              \\
  \begin{aligned}
    &=\trace(
      \begin{mat}
        r_1a_1+r_2a_2 &r_1b_1+r_2b_2 \\
        r_1c_1+r_2c_2 &r_1d_1+r_2d_2
      \end{mat}
      )                                       \\
    &=(r_1a_1+r_2a_2)+(r_1d_1+r_2d_2)         \\
    &=r_1(a_1+d_1)+r_2(a_2+d_2)               \\
    &=r_1\cdot\trace(
      \begin{mat}
        a_1 &b_1 \\
        c_1 &d_1
      \end{mat}
      )
      +
      r_2\cdot\trace(
      \begin{mat}
        a_2 &b_2 \\
        c_2 &d_2
      \end{mat}
      )
  \end{aligned} 
\end{multline*}
\end{frame}



%..........
\begin{frame}
\th[th:HomoDetActOnBasis]
\ExecuteMetaData[../map2.tex]{th:HomoDetActOnBasis}

\pause
\pf
\ExecuteMetaData[../map2.tex]{pf:HomoDetActOnBasis0}

\pause
\ExecuteMetaData[../map2.tex]{pf:HomoDetActOnBasis1}
\end{frame}
\begin{frame}
\ExecuteMetaData[../map2.tex]{pf:HomoDetActOnBasis2}
\qed
\end{frame}




%..........
\begin{frame}
\ex 
One basis of the space of quadratic polynomials $\polyspace_2$
is $B=\sequence{x^2,x,1}$.
We can define a map $\map{\text{eval}_3}{\polyspace_2}{\Re}$ 
by specifying its action on that basis
\begin{equation*}
  x^2\mapsunder{\text{eval}_3}9
  \quad
  x\mapsunder{\text{eval}_3}3
  \quad
  1\mapsunder{\text{eval}_3}1
\end{equation*}
and then extending linearly.
\begin{equation*}
  \text{eval}_3(ax^2+bx+c)=a\cdot\text{eval}_3(x^2)
                          +b\cdot\text{eval}_3(x)
                          +c\cdot\text{eval}_3(1)
          =9a+3b+c
\end{equation*}
\pause
The action of this map on the basis elements is to plug the value~$3$ in for
$x$. 
That remains true when we extend linearly, so
$\text{eval}_3(\,p(x)\,)=p(3)$.
\end{frame}




%..........
\begin{frame}
\ex
Consider the standard basis $\stdbasis_2$ for the vector space $\Re^2$.
We can specify a rotation of the two basis vectors as here.
\begin{equation*}
  \colvec{1 \\ 0}\mapsto\colvec[r]{\cos\theta \\ \sin\theta}
  \quad
  \colvec{0 \\ 1}\mapsto\colvec[r]{-\sin\theta \\ \cos\theta}
  \qquad
  \vcenteredhbox{\includegraphics{asy/three_ii_rotate.pdf}}
\end{equation*}
\pause
Extend
that linearly
to get a homomorphism $\map{t_{\theta}}{\Re^2}{\Re^2}$. 
\begin{align*}
  t_{\theta}(\colvec{x \\ y})
  &=
  t_{\theta}(x\cdot\colvec{1 \\ 0}+y\cdot\colvec{0 \\ 1})  \\
  &=
  x\cdot t_{\theta}(\colvec{1 \\ 0})+y\cdot t_{\theta}(\colvec{0 \\ 1})  \\
  &=x\cdot\colvec[r]{\cos\theta \\ \sin\theta}
    +y\cdot\colvec[r]{-\sin\theta \\ \cos\theta}  \\
  &=\colvec{x\cos\theta-y\sin\theta \\ x\sin\theta+y\cos\theta}
\end{align*}
\pause
This is one-to-one and onto, so it is an automorphism.
\end{frame}




%..........
\begin{frame}
\df[df:LinearTransformation]
\ExecuteMetaData[../map2.tex]{df:LinearTransformation}

\pause
\ex
For any vector space $V$ the \definend{identity} map $\map{\textrm{id}}{V}{V}$
given by $\vec{v}\mapsto\vec{v}$ is a linear transformation.
The check is easy.

\pause
\ex
In $\Re^3$ the function $f_{yz}$  
\begin{equation*}
  \colvec{x \\ y \\ z}\mapsunder{f_{yz}}\colvec{-x \\ y \\ z}
\end{equation*}
that reflects vectors over the $yz$-plane is a linear
transformation.
\begin{multline*}
  f_{yz}(r_1\colvec{x_1 \\ y_1 \\ z_1}+r_2\colvec{x_2 \\ y_2 \\ z_2})
  =f_{yz}(\colvec{r_1x_1+r_2x_2 \\ r_1y_1+r_2y_2 \\ r_1z_1+r_2z_2})  
  =\colvec{-(r_1x_1+r_2x_2) \\ r_1y_1+r_2y_2 \\ r_1z_1+r_2z_2}    \\
  =r_1\colvec{-x_1 \\ y_1 \\ z_1}+r_2\colvec{-x_2 \\ y_2 \\ z_2}  
  =r_1f_{yz}(\colvec{x_1 \\ y_1 \\ z_1})+r_2f_{yz}(\colvec{x_2 \\ y_2 \\ z_2}) 
\end{multline*}
\end{frame}




%..........
\begin{frame}
\lm[le:SpLinFcns]
\ExecuteMetaData[../map2.tex]{lm:SpLinFcns}

\ExecuteMetaData[../map2.tex]{SpLinFcns}

\pause
\pf
\ExecuteMetaData[../map2.tex]{pf:SpLinFcns}
\qed
\end{frame}



% ..... Three.II.2 .....
\section{Range space and null space}
%..........
\begin{frame}
\lm[le:RangeIsSubSp]
\ExecuteMetaData[../map2.tex]{lm:RangeIsSubSp}

\pause
\pf
\ExecuteMetaData[../map2.tex]{pf:RangeIsSubSp}
\qed

\pause
\ex
For any $\theta$ the function $\map{t_{\theta}}{\Re^2}{\Re^2}$ that rotates 
vectors counterclockwise by $\theta$~radians is a homomorphism
(in fact, it is an automorphism).
In the domain~$\Re^2$ each line through the origin is a subspace.
The image of that line under this map is another line through the origin 
and thus is a subspace of the codomain~$\Re^2$. 
\end{frame}




%..........
\begin{frame}{Range space}
\df[df:RangeSpace]
\ExecuteMetaData[../map2.tex]{df:RangeSpace}

\pause
\ex
Projection $\map{\pi}{\Re^3}{\Re^2}$
\begin{equation*}
  \colvec{x \\ y \\ z}\mapsto\colvec{x \\ y}
\end{equation*}
is a linear map; the check is routine.
The range space
\begin{equation*}
  \rangespace{\pi}
  =\set{\colvec{x \\ y \\ 0}\suchthat x,y\in\Re}
\end{equation*}
is two-dimensional so the rank of $\pi$ is $2$.
\end{frame}
\begin{frame}
\ex
The derivative map
$\map{d/dx}{\Re^4}{\Re^4}$
is linear.
Its range is $\rangespace{d/dx}=\set{a_0+a_1x+a_2x^2+a_3x^3\suchthat a_i\in\Re}$.
(Verifying that every member of that space is the derivative of a fourth
degree polynomial is easy.)
Thus the rank of the derivative is $3$. 

\pause
\ex
This map from $\matspace_{\nbyn{2}}$ to $\Re^2$ is linear; the check is routine.
\begin{equation*}
  \begin{mat}
    a &b \\
    c &d
  \end{mat}
  \mapsunder{h}
  \colvec{a+b  \\ 2a+2b}
\end{equation*}
The rangespace is this line through the origin
\begin{equation*}
  \set{\colvec{t \\ 2t} \suchthat t\in\Re}
\end{equation*}
(every member of that set is the image 
\begin{equation*}
  \colvec{t \\ 2t}
  =
  h(
    \begin{mat}
      t  &0 \\
      0   &0
    \end{mat}\
   )
\end{equation*}
of a $\nbyn{2}$ matrix).
The rank of this map is $1$.
\end{frame}




%..........
\begin{frame}{Inverse image}
\centergraphic{../ch3.5}
\ExecuteMetaData[../map2.tex]{InverseImage}

\ex
Consider the projection map $\map{\pi}{\Re^2}{\Re}$.
\begin{equation*}
  \pi(\colvec{x \\ y})
  =x
\end{equation*}
The check that $\pi$ is a homomorphism is routine.
\end{frame}
\begin{frame}
Many vectors are mapped
to $2$;
think of these as ``$2$-vectors.''
\centergraphic{asy/three_ii_proj2.pdf}
\pause
These are ``$3$-vectors,'' the inverse images of $3$.
\centergraphic{asy/three_ii_proj3.pdf}
\pause
Where $\pi(\vec{u})=2$ and $\pi(\vec{v})=3$
then $\pi(\vec{u}+\vec{v})=\pi(\vec{u})+\pi(\vec{v})=5$
so $\vec{u}+\vec{v}$ is a ``$5$-vector.''
\centergraphic{asy/three_ii_proj5.pdf}
\end{frame}




%..........
\begin{frame}
\centergraphic{asy/three_ii_1vec.png}
\end{frame}




%..........
\begin{frame}
\ex
Consider $\map{h}{\polyspace_2}{\Re^2}$
\begin{equation*}
  ax^2+bx+c\mapsto\colvec{b \\ b}
\end{equation*}
and consider these three members of the range.
\begin{equation*}
  \vec{w}_1=\colvec{1 \\ 1},\;
  \vec{w}_2=\colvec{-1 \\ -1},\;   
  \vec{w}_3=\colvec{0 \\ 0}
\end{equation*}
\pause
The inverse image of $\vec{w}_1$ is 
$h^{-1}(\vec{w}_1)=\set{a_1x^2+x+c_1\suchthat a_1,c_1\in\Re^2}$.
Think of these as ``$\vec{w}_1$~vectors.''
Some examples are $3x^2+x+1$, $3x^2+x-4$, and $-2x^2+x$.
\pause
The inverse image of $\vec{w}_2$ is 
$h^{-1}(\vec{w}_2)=\set{a_2x^2-x+c_2\suchthat a_2,c_2\in\Re^2}$;
these are ``$\vec{w}_2$~vectors.''
The vectors from the domain associated with $\vec{w}_3$ are members of the set
$h^{-1}(\vec{w}_3)=\set{a_3x^2+c_3\suchthat a_3,c_3\in\Re^2}$.

\pause
As in the prior example, any $\vec{v}_1\in h^{-1}(\vec{w}_1)$
plus any $\vec{v}_2\in h^{-1}(\vec{w}_2)$
totals to a $\vec{v}_3\in h^{-1}(\vec{w}_3)$. 
This is
because $h$ is a homomorphism, so 
$h(\vec{v}_1+\vec{v}_2)=h(\vec{v}_1)+h(\vec{v}_2)=\vec{w}_1+\vec{w}_2=\vec{w}_3$.
That is, the sum of a ``$\vec{w}_1$~vector'' with a
``$\vec{w}_2$~vector'' is mapped by $h$ to $\vec{w}_3$. 
\end{frame}




%..........
\begin{frame}
\lm[le:NullspIsSubSp]
\ExecuteMetaData[../map2.tex]{lm:NullspIsSubSp}

\pause
\pf
\ExecuteMetaData[../map2.tex]{pf:NullspIsSubSp}
\qed
\end{frame}




%..........
\begin{frame}{Null space}
\df[df:NullSpace]
\ExecuteMetaData[../map2.tex]{df:NullSpace}

\pause
\medskip
\no 
Strictly, the nullspace of the codomain is not $\zero_{W}$, it is 
$\set{\zero_{W}}$.
Thus the nullspace should properly be
given as $h^{-1}(\set{\zero_{W}})$.
But authors often state it as above.
\end{frame}




%..........
\begin{frame}{Rank plus nullity}
\th[th:RankPlusNullEqDim]
\ExecuteMetaData[../map2.tex]{th:RankPlusNullEqDim}

\pause
\pf
\ExecuteMetaData[../map2.tex]{pf:RankPlusNullEqDim0}

\pause
\ExecuteMetaData[../map2.tex]{pf:RankPlusNullEqDim1}
\end{frame}
\begin{frame}
\ExecuteMetaData[../map2.tex]{pf:RankPlusNullEqDim2}
\qed
\end{frame}




%..........
\begin{frame}
\lm[lm:ImageLinearlyDependentIsLinearlyDependent]
\ExecuteMetaData[../map2.tex]{lm:ImageLinearlyDependentIsLinearlyDependent}

\pause
\pf
\ExecuteMetaData[../map2.tex]{pf:ImageLinearlyDependentIsLinearlyDependent}
\qed
\end{frame}




%..........
\begin{frame}
\th[th:OOHomoEquivalence]
\ExecuteMetaData[../map2.tex]{th:OOHomoEquivalence}

\pause
\pf
\ExecuteMetaData[../map2.tex]{pf:OOHomoEquivalence0}
\end{frame}
\begin{frame}
\ExecuteMetaData[../map2.tex]{pf:OOHomoEquivalence1}

\pause
\ExecuteMetaData[../map2.tex]{pf:OOHomoEquivalence2}
\end{frame}
\begin{frame}
\ExecuteMetaData[../map2.tex]{pf:OOHomoEquivalence3}
\qed
\end{frame}




%...........................
% \begin{frame}
% \ExecuteMetaData[../gr3.tex]{GaussJordanReduction}
% \df[def:RedEchForm]
% 
% \end{frame}
\end{document}
