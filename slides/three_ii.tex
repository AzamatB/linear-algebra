% see: https://groups.google.com/forum/?fromgroups#!topic/comp.text.tex/s6z9Ult_zds
\makeatletter\let\ifGm@compatii\relax\makeatother 
\documentclass[10pt,t,serif,professionalfont]{beamer}
\PassOptionsToPackage{pdfpagemode=FullScreen}{hyperref}
\PassOptionsToPackage{usenames,dvipsnames}{color}
% \DeclareGraphicsRule{*}{mps}{*}{}
\usepackage{../linalgjh}
\usepackage{present}
\usepackage{xr}\externaldocument{../map2} % read refs from .aux file
\usepackage{xr}\externaldocument{../vs3} % read refs from .aux file
\usepackage{catchfilebetweentags}
\usepackage{etoolbox} % from http://tex.stackexchange.com/questions/40699/input-only-part-of-a-file-using-catchfilebetweentags-package
\makeatletter
\patchcmd{\CatchFBT@Fin@l}{\endlinechar\m@ne}{}
  {}{\typeout{Unsuccessful patch!}}
\makeatother

\mode<presentation>
{
  \usetheme{boxes}
  \setbeamercovered{invisible}
  \setbeamertemplate{navigation symbols}{} 
}
\addheadbox{filler}{\ }  % create extra space at top of slide 
\hypersetup{colorlinks=true,linkcolor=blue} 

\title[Homomorphisms] % (optional, use only with long paper titles)
{Three.II Homomorphisms}

\author{\textit{Linear Algebra} \\ {\small Jim Hef{}feron}}
\institute{
  \texttt{http://joshua.smcvt.edu/linearalgebra}
}
\date{}


\subject{Homomorphisms}
% This is only inserted into the PDF information catalog. Can be left
% out. 

\begin{document}
\begin{frame}
  \titlepage
\end{frame}

% =============================================
% \begin{frame}{Reduced Echelon Form} 
% \end{frame}



% ..... Three.II.1 .....
\section{Definition}
%..........
\begin{frame}{Homomorphism}
\df[def:Homo]
\ExecuteMetaData[../map2.tex]{df:Homo}
\end{frame}




%..........
\begin{frame}
\ex
The function $\map{h}{\polyspace_2}{\Re^2}$ given by
\begin{equation*}
  h(a+bx+cx^2)
  =
  \colvec{a+c \\ 0}
\end{equation*}
is a homomorphism.
To check that we need to verify that it respects the addition and
scalar multiplication operations.

\pause
Addition is routine.
\begin{gather*}
  h(\,(a_1+b_1x+c_1x^2)+(a_2+b_2x+c_2x^2)\,)      \\
  \begin{align*}
    \qquad
    &=h(\,(a_1+a_2)+(b_1+b_2)x+(c_1+c_2)x^2\,)     \\
    &=\colvec{a_1+a_2+c_1+c_2 \\ 0}                 \\
    &=\colvec{a_1+c_1 \\ 0}+\colvec{a_2+c_2 \\ 0}   \\
    &=h(a_1+b_1x+c_1x^2)+h(a_2+b_2x+c_2x^2)       
  \end{align*}
\end{gather*}
\pause
So is scalar multiplication.
\begin{equation*}
  r\cdot h(a+bx+cx^2)
  =r\cdot\colvec{a+c \\ 0}  
  =\colvec{ra+rc \\ 0}
  =h(\,r(a+bx+cx^2)\,)
\end{equation*}
\end{frame}



%..........
\begin{frame}
\ex
Of these two maps $\map{h,g}{\Re^2}{\Re}$
the first is linear while the second is not.
\begin{equation*}
  \colvec{x \\ y}\mapsunder{h} 2x-3y
  \qquad
  \colvec{x \\ y}\mapsunder{g} 2x-3y+1
\end{equation*}

\pause
The map $h$ respects addition
\begin{multline*}
  h(\colvec{x_1 \\ y_1}+\colvec{x_2 \\ y_2})
  =h(\colvec{x_1+x_2 \\ y_1+y_2})             
  =2(x_1+x_2)-3(y_1+y_2)                    \\
  =(2x_1-3y_1)+(2x_2-3y_2)
  =h(\colvec{x_1 \\ y_1})+h(\colvec{x_2 \\ y_2})
\end{multline*}
and scalar multiplication.
\begin{equation*}
  r\cdot h(\colvec{x \\ y})
  =r\cdot(2x-3y)
  =2rx-3ry
  =(2r)x-(3r)y
  =h(r\cdot\colvec{x \\ y})
\end{equation*}

\pause
This example shows that $g$ does not respect addition.
\begin{equation*} 
  g(\colvec{1 \\ 4}+\colvec{5 \\ 6})=-17
  \quad\text{while}\quad
  g(\colvec{1 \\ 4})+g(\colvec{5 \\ 6})=-16
\end{equation*}
\end{frame}




%..........
\begin{frame}
We proved these two in the context of studying isomorphisms.

\lm[le:HomoSendsZeroToZero]
\ExecuteMetaData[../map2.tex]{lm:HomoSendsZeroToZero}

\lm[le:HomoPreserveLinCombo]
\ExecuteMetaData[../map2.tex]{lm:HomoPreserveLinCombo}

\pause
\medskip
\ex
Between any two vector spaces the zero map $\map{Z}{V}{W}$, defined by
$Z(\vec{v})=\zero_W$ is a homomorphism.
The check is: 
$Z(c_1\vec{v}_1+c_2\vec{v}_2)=\zero_W
   =\zero_W+\zero_W=c_1Z(\vec{v}_1)+c_2Z(\vec{v}_2)$.
\end{frame}




%..........
\begin{frame}
\ex
The \definend{inclusion map} $\map{\iota}{\Re^2}{\Re^3}$
\begin{equation*}
  \iota(\colvec{x  \\ y})
  =\colvec{x \\ y \\ z}
\end{equation*}
is a homomorphism.
Here is the check.
\begin{align*}
  \iota(c_1\colvec{x_1 \\ y_1}+c_2\colvec{x_2 \\ y_2})
  &=\iota(\colvec{c_1x_1+c_2x_2 \\ c_1y_1+c_2y_2})       \\
  &=\colvec{c_1x_1+c_2x_2 \\ c_1y_1+c_2y_2 \\ 0}      \\
  &=\colvec{c_1x_1 \\ c_1y_1 \\ 0}
   +\colvec{c_2x_2 \\ c_2y_2 \\ 0}                    \\
  &=c_1\iota(\colvec{x_1 \\ y_1})+c_2\iota(\colvec{x_2 \\ y_2})
\end{align*}
\end{frame}




%..........
\begin{frame}
\ex
Consider 
the function $\map{h}{\polyspace_1}{\polyspace_1}$ defined below.
\begin{equation*}
  h(a+bx)=b+bx
\end{equation*}
Here are two examples of the action of this function: $h(1+2x)=2+2x$
and $h(3-x)=-1-x$. 

\pause
This function is linear.
The verification is straightforward.
\begin{align*}
  h(\,c_1(a_1+b_1x)+c_2(a_2+b_2x)\,)
  &=h(\,(c_1a_1+c_2a_2)+(c_1b_1+c_2b_2)x\,)       \\
  &=(c_1b_1+c_2b_2)+(c_1b_1+c_2b_2)x          \\
  &=(c_1b_1+c_1b_1x)+(c_2b_2+c_2b_2x)         \\
  &=c_1h(a_1+b_1x)+c_2h(a_2+b_2x)
\end{align*}
\end{frame}




%..........
\begin{frame}
\ex
The derivative map $\map{d/dx}{\polyspace_2}{\polyspace_1}$
is given by $d/dx\,(ax^2+bx+c)=2ax+b$.
For instance, $d/dx\,(3x^2-2x+4)=6x-2$
and $d/dx\,(x^2+1)=2x$.

\pause
This is a homomorphism.
\begin{gather*}
  d/dx\,\big(\,r_1(a_1x^2+b_1x+c_1)+r_2(a_2x^2+b_2x+c_2)\,\big)  \hspace*{5em}         \\
  \begin{align*} 
    &=d/dx\,\big(\,(r_1a_1+r_2a_2)x^2+(r_1b_1+r_2b_2)x+(r_1c_1+r_2c_2)\,\big)   \\
    &=2(r_1a_1+r_2a_2)x+(r_1b_1+r_2b_2)   \\
    &=(2r_1a_1x+r_1b_1)+(2r_2a_2x+r_2b_2)   \\
    &=r_1\cdot d/dx\,(a_1x^2+b_1x+c_1)
      +r_2\cdot d/dx\,(a_2x^2+b_2x+c_2)
  \end{align*}
\end{gather*}
\end{frame}



%..........
\begin{frame}
The \definend{trace} of a square matrix 
is the sum down the uppper-left to lower-right diagonal
Thus
$\map{\trace}{\matspace_{\nbyn{2}}}{\Re}$
is this.
\begin{equation*}
  \begin{mat}
    a &b \\
    c &d
  \end{mat}
  =a+b
\end{equation*}
This map is linear.
\begin{align*}
  \trace(
  r_1\begin{mat}
    a_1 &b_1 \\
    c_1 &d_1
  \end{mat}
  +
  r_2\begin{mat}
    a_2 &b_2 \\
    c_2 &d_2
  \end{mat}
  )
  &=\trace(
    \begin{mat}
      r_1a_1+r_2a_2 &r_1b_1+r_2b_2 \\
      r_1c_1+r_2c_2 &r_1d_1+r_2d_2
    \end{mat}
    )                                       \\
  &=(r_1a_1+r_2a_2)+(r_1d_1+r_2d_2)         \\
  &=r_1(a_1+d_1)+r_2(a_2+d_2)               \\
  &=r_1\trace(
    \begin{mat}
      a_1 &b_1 \\
      c_1 &d_1
    \end{mat}
    )
    +
    r_2\trace(
    \begin{mat}
      a_2 &b_2 \\
      c_2 &d_2
    \end{mat}
    ) 
\end{align*}
\end{frame}



%..........
\begin{frame}
\th[th:HomoDetActOnBasis]
\ExecuteMetaData[../map2.tex]{th:HomoDetActOnBasis}

\pause
\pf
\ExecuteMetaData[../map2.tex]{pf:HomoDetActOnBasis0}

\pause
\ExecuteMetaData[../map2.tex]{pf:HomoDetActOnBasis1}
\end{frame}
\begin{frame}
\ExecuteMetaData[../map2.tex]{pf:HomoDetActOnBasis2}
\qed
\end{frame}




%..........
\begin{frame}
\df[df:LinearTransformation]
\ExecuteMetaData[../map2.tex]{df:LinearTransformation}
\end{frame}




%..........
\begin{frame}
\lm[le:SpLinFcns]
\ExecuteMetaData[../map2.tex]{lm:SpLinFcns}

\ExecuteMetaData[../map2.tex]{SpLinFcns}

\pause
\pf
\ExecuteMetaData[../map2.tex]{pf:SpLinFcns}
\qed
\end{frame}



% ..... Three.II.2 .....
\section{Range space and null space}
%..........
\begin{frame}
\lm[le:RangeIsSubSp]
\ExecuteMetaData[../map2.tex]{lm:RangeIsSubSp}

\pause
\pf
\ExecuteMetaData[../map2.tex]{pf:RangeIsSubSp}
\qed
\end{frame}




%..........
\begin{frame}{Range space}
\df[df:RangeSpace]
\ExecuteMetaData[../map2.tex]{df:RangeSpace}
\end{frame}




%..........
\begin{frame}
\lm[le:NullspIsSubSp]
\ExecuteMetaData[../map2.tex]{lm:NullspIsSubSp}

\pause
\pf
\ExecuteMetaData[../map2.tex]{pf:NullspIsSubSp}
\qed
\end{frame}




%..........
\begin{frame}{Null space}
\df[df:NullSpace]
\ExecuteMetaData[../map2.tex]{df:NullSpace}
\end{frame}




%..........
\begin{frame}{Rank plus nullity}
\th[th:RankPlusNullEqDim]
\ExecuteMetaData[../map2.tex]{th:RankPlusNullEqDim}

\pause
\pf
\ExecuteMetaData[../map2.tex]{pf:RankPlusNullEqDim0}

\pause
\ExecuteMetaData[../map2.tex]{pf:RankPlusNullEqDim1}
\end{frame}
\begin{frame}
\ExecuteMetaData[../map2.tex]{pf:RankPlusNullEqDim2}
\qed
\end{frame}




%..........
\begin{frame}
\lm[lm:ImageLinearlyDependentIsLinearlyDependent]
\ExecuteMetaData[../map2.tex]{lm:ImageLinearlyDependentIsLinearlyDependent}

\pause
\pf
\ExecuteMetaData[../map2.tex]{pf:ImageLinearlyDependentIsLinearlyDependent}
\qed
\end{frame}




%..........
\begin{frame}
\th[th:OOHomoEquivalence]
\ExecuteMetaData[../map2.tex]{th:OOHomoEquivalence}

\pause
\pf
\ExecuteMetaData[../map2.tex]{pf:OOHomoEquivalence0}
\end{frame}
\begin{frame}
\ExecuteMetaData[../map2.tex]{pf:OOHomoEquivalence1}

\pause
\ExecuteMetaData[../map2.tex]{pf:OOHomoEquivalence2}
\end{frame}
\begin{frame}
\ExecuteMetaData[../map2.tex]{pf:OOHomoEquivalence3}
\qed
\end{frame}




%...........................
% \begin{frame}
% \ExecuteMetaData[../gr3.tex]{GaussJordanReduction}
% \df[def:RedEchForm]
% 
% \end{frame}
\end{document}
