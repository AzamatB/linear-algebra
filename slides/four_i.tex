% see: https://groups.google.com/forum/?fromgroups#!topic/comp.text.tex/s6z9Ult_zds
\makeatletter\let\ifGm@compote\relax\makeatother 
\documentclass[10pt,t,serif,professionalfont]{beamer}
\PassOptionsToPackage{pdfpagemode=FullScreen}{hyperref}
\PassOptionsToPackage{usenames,dvipsnames}{color}
% \DeclareGraphicsRule{*}{mps}{*}{}
\usepackage{../linalgjh}
\usepackage{present}
\usepackage{xr}\externaldocument{../det1} % read refs from .aux file
\usepackage{xr}\externaldocument{../map4} % read refs from .aux file
\usepackage{catchfilebetweentags}
\usepackage{etoolbox} % from http://tex.stackexchange.com/questions/40699/input-only-part-of-a-file-using-catchfilebetweentags-package
\makeatletter
\patchcmd{\CatchFBT@Fin@l}{\endlinechar\m@ne}{}
  {}{\typeout{Unsuccessful patch!}}
\makeatother

\mode<presentation>
{
  \usetheme{boxes}
  \setbeamercovered{invisible}
  \setbeamertemplate{navigation symbols}{} 
}
\addheadbox{filler}{\ }  % create extra space at top of slide 
\hypersetup{colorlinks=true,linkcolor=blue} 

\title[Determinants] % (optional, use only with long paper titles)
{Four.I Determinants; Definition}

\author{\textit{Linear Algebra} \\ {\small Jim Hef{}feron}}
\institute{
  \texttt{http://joshua.smcvt.edu/linearalgebra}
}
\date{}


\subject{Determinants}
% This is only inserted into the PDF information catalog. Can be left
% out. 

\begin{document}
\begin{frame}
  \titlepage
\end{frame}

% =============================================
% \begin{frame}{Reduced Echelon Form} 
% \end{frame}



% ..... Four.I.1,2 .....
\section{Properties of Determinants}
%..........
\begin{frame}{Nonsingular matrices}
An \( \nbyn{n} \) matrix \( T \) is nonsingular if and only if
each of these holds:%
\ExecuteMetaData[../det1.tex]{EquivalentOfNonsingular}
In this chapter we will give a formula that determines whether a
matrix is nonsingular.
\end{frame}




\begin{frame}
\ExecuteMetaData[../det1.tex]{DeterminantIntro}  
\end{frame}




\begin{frame}
We will define the determinant function by listing 
some of its properties.
We are interested in these properties 
because they are convenient for computing the value of the determinant
on an input square matrix. 
Then we will show that only one function with those properties exists.

\pause
\medskip
\no
The first section of the text algebraically motivates the 
determinant definition. 
The section after this will give a geometric motivation.
In this section, beyond defining the determinant, 
we will show how to compute it
and give some key results.
\end{frame}




\begin{frame}{Definition of determinant}
\df[def:Det]
\ExecuteMetaData[../det1.tex]{df:Det}

\pause 
\re[rem:SwapRowsRedun] 
\ExecuteMetaData[../det1.tex]{re:SwapRowsRedun}
\end{frame}




\begin{frame}{Consequences of the definition}
\lm[le:IdenRowsDetZero]
\ExecuteMetaData[../det1.tex]{lm:IdenRowsDetZero}

\pause 
\pf 
\ExecuteMetaData[../det1.tex]{pf:IdenRowsDetZero0}

\pause
\ExecuteMetaData[../det1.tex]{pf:IdenRowsDetZero1}
\end{frame}
\begin{frame}
\ExecuteMetaData[../det1.tex]{pf:IdenRowsDetZero2}

\pause
\ExecuteMetaData[../det1.tex]{pf:IdenRowsDetZero3}  
\end{frame}
\begin{frame}
\ExecuteMetaData[../det1.tex]{pf:IdenRowsDetZero4}
\qed
\end{frame}




\begin{frame}
We can compute the determinant of a matrix using Gauss's Method
(presuming that the determinant function exists, which we will
cover later).

\ex  On this matrix the Gauss's Method reduces the first column  with 
$-2\rho_1+\rho_2$ and $-3\rho_1+\rho_3$.
Property~(1) says that these row operations
leave the determinant unchanged.
\begin{equation*}
  \begin{vmat}
    1  &3  &-2 \\
    2  &0  &4  \\
    3  &-1 &5
  \end{vmat}
  =
  \begin{vmat}
    1  &3   &-2 \\
    0  &-6  &-8  \\
    0  &-10 &-11
  \end{vmat}
\end{equation*}
\pause
Reduce the second column with $-(5/3)\rho_2+\rho_3$.
Again, by property~(1) the determinant stays the same.
\begin{equation*}
  =
  \begin{vmat}
    1  &3   &-2 \\
    0  &-6  &-8  \\
    0  &0   &-7/3
  \end{vmat}
\end{equation*}
\pause
By the prior lemma we can now 
find the determinant by taking the product down the
diagonal.
\begin{equation*}
  =1\cdot(-6)\cdot(-7/3)=14
\end{equation*}
\end{frame}
\begin{frame}
\ex
This matrix requires a row swap, which changes the sign of the determinant.
\begin{equation*}
  \begin{vmat}
    0  &3  &1 \\
    1  &2  &0 \\
    1  &5  &2
  \end{vmat}
  =
  -\begin{vmat}
    1  &2  &0 \\
    0  &3  &1 \\
    1  &5  &2
  \end{vmat}
\end{equation*}
Performing $-\rho_1+\rho_3$
\begin{equation*}
  =-\begin{vmat}
    1  &2  &0 \\
    0  &3  &1 \\
    0  &3  &2
  \end{vmat}
\end{equation*}
and $-\rho_2+\rho_3$
\begin{equation*}
  =-\begin{vmat}
    1  &2  &0 \\
    0  &3  &1 \\
    0  &0  &1
  \end{vmat}
\end{equation*}
and then multiplying down the diagonal gives that the determinant
of the original matrix is $-3$.
\end{frame}



\begin{frame}{The $\nbyn{n}$ determinant is unique}
\lm[lm:DetFcnIsUnique]
\ExecuteMetaData[../det1.tex]{lm:DetFcnIsUnique}

\pause 
\pf 
\ExecuteMetaData[../det1.tex]{pf:DetFcnIsUnique}
\qed

\pause
\medskip
So if there is a function mapping $\matspace_{\nbyn{n}}$ to $\Re$ with 
the four properties of the definition then there is only one such
function.
The next two subsections show that for each~$n$ a determinant function 
exists.
\end{frame}



% ..... Four.I.3 .....
\section{The Permutation Expansion}
\begin{frame}
\ExecuteMetaData[../det1.tex]{DifferentGaussMethodReductions}
\end{frame}
\begin{frame}
That the above computation gives a consistent result for these two
ways to do a reduction on one matrix does not ensure that determinants
always give a well-defined value.
Our algorithm for computing determinant values does not
plainly eliminate the possibility that there might be, say,
two reductions of some $\nbyn{7}$~matrix that lead to different 
determinant outputs.
In that case there would exist no determinant function, 
since functions must have that for each input there is exactly one output. 

\pause
To show that determinants are well-defined 
we will give an alternative way to compute
the value of a determinant. 
This new way is less useful in practice since it 
makes the computations awkward and slow, which is why we didn't start with it.
But it is useful for theory since it makes the proof that we need easier.
\end{frame}



\begin{frame}{The determinant function is not linear}
\ex
The second matrix is twice the first but the determinant does not double.
\begin{equation*}
  \begin{vmat}
    3  &-3  &9 \\
    1  &-1   &7 \\
    2  &4   &0
  \end{vmat}
  =-72
  \qquad
  \begin{vmat}
    6  &-6  &18 \\
    2  &-2   &14 \\
    4  &8   &0
  \end{vmat}
  =-576
\end{equation*}
Instead, by property~(3) of \nearbydefinition{def:Det}
the determinant scales one row at a time:   
\begin{align*}
  \begin{vmat}
    3  &-3  &9 \\
    1  &-1   &7 \\
    2  &4   &0
  \end{vmat}
  &=3\cdot
  \begin{vmat}
    1  &-1  &3 \\
    1  &-1   &7 \\
    2  &4   &0
  \end{vmat}           \\
  &=6\cdot
  \begin{vmat}
    1  &-1  &3 \\
    1  &-1   &7 \\
    1  &2   &0
  \end{vmat}           
  % &=6\cdot(-12)
\end{align*}
\end{frame}



%..........
\begin{frame}{Multilinear}
\df[def:multilinear]
\ExecuteMetaData[../det1.tex]{df:Multilinear}


\lm[lem:DetsMultilinear]
\ExecuteMetaData[../det1.tex]{lm:DetsMultilinear}

\pause
\pf
\ExecuteMetaData[../det1.tex]{pf:DetsMultilinear0}

\pause
\ExecuteMetaData[../det1.tex]{pf:DetsMultilinear1}
\end{frame}
\begin{frame}
\ExecuteMetaData[../det1.tex]{pf:DetsMultilinear2}
\end{frame}
\begin{frame}
\ExecuteMetaData[../det1.tex]{pf:DetsMultilinear3}
\qed
\end{frame}



\begin{frame}
Use multilinearity to break a determinant into a sum of 
simple determinants.

\ex
We can expand this determinant
\begin{equation*}
  \begin{vmat}
    1 &2 \\
    3 &4
  \end{vmat}
\end{equation*}
along the first row
\begin{equation*}
  =\begin{vmat}
    1 &0 \\
    3 &4
  \end{vmat}
  +\begin{vmat}
    0 &2 \\
    3 &4
  \end{vmat}
\end{equation*}
and the second row.
\begin{equation*}
  =\begin{vmat}
    1 &0 \\
    3 &0
  \end{vmat}
  +\begin{vmat}
    1 &0 \\
    0 &4
  \end{vmat}
  +\begin{vmat}
    0 &2 \\
    3 &0
  \end{vmat}
  +\begin{vmat}
    0 &2 \\
    0 &4
  \end{vmat}
\end{equation*}
We have four matrices, each with a single nonzero entry in each row.

\pause 
The first and last determinants are $0$ because the matrices are
nonsingular (since one row is a multiple of the other).
We are left with the two matrices in which there is 
one entry from each row and column from the starting matrix.
\end{frame}




\begin{frame}
\ex
Similarly we can start to evaluate this determinant 
\begin{equation*}
  \begin{vmat}
    1 &2 &3 \\
    4 &5 &6 \\
    7 &8 &9
  \end{vmat}
\end{equation*}
by breaking it into
a sum of determinants of matrices having one entry in each row from the 
starting matrix.   
\begin{equation*}
  =\begin{vmat}
    1 &0 &0 \\
    4 &0 &0 \\
    7 &0 &0
  \end{vmat}
  +\begin{vmat}
    1 &0 &0 \\
    4 &0 &0 \\
    0 &8 &0
  \end{vmat}
  +\cdots+
  \begin{vmat}
    0 &0 &3 \\
    0 &0 &6 \\
    0 &8 &0
  \end{vmat}
  +\begin{vmat}
    0 &0 &3 \\
    0 &0 &6 \\
    0 &0 &9
  \end{vmat}
\end{equation*}
This gives a number of matrices, 
each all $0$'s except that each row has a single
entry from the original matrix.

\pause
For any of these determinants, if two rows have their original matrix entry 
in the same column then the determinant is~$0$,
since if either entry is $0$ then the matrix has a zero row while if neither
is~$0$ then each row is a multiple of the other.
\pause
Therefore, the above reduces to a sum of determinants, each all $0$'s but
for a single entry in each row and column from the original matrix. 
\end{frame}
\begin{frame}
\vspace*{-3ex}
\begin{align*}
  \begin{vmat}
    1 &2 &3 \\
    4 &5 &6 \\
    7 &8 &9
  \end{vmat}
  &=\begin{vmat}
    1 &0 &0 \\
    0 &5 &0 \\
    0 &0 &9
  \end{vmat}
  +\begin{vmat}
    1 &0 &0 \\
    0 &0 &6 \\
    0 &8 &0
  \end{vmat}            \\
  &\quad+\begin{vmat}
    0 &2 &0 \\
    4 &0 &0 \\
    0 &0 &9
  \end{vmat}
  +\begin{vmat}
    0 &2 &0 \\
    0 &0 &6 \\
    7 &0 &0
  \end{vmat}         \\
  &\quad+\begin{vmat}
    0 &0 &3 \\
    4 &0 &0 \\
    0 &8 &0
  \end{vmat}
  +\begin{vmat}
    0 &0 &3 \\
    0 &5 &0 \\
    7 &0 &0
  \end{vmat}            \\
  &=45\cdot\begin{vmat}
    1 &0 &0 \\
    0 &1 &0 \\
    0 &0 &1
  \end{vmat}
  +48\cdot\begin{vmat}
    1 &0 &0 \\
    0 &0 &1 \\
    0 &1 &0
  \end{vmat}            \\
  &\quad+72\cdot\begin{vmat}
    0 &1 &0 \\
    1 &0 &0 \\
    0 &0 &1
  \end{vmat}
  +84\cdot\begin{vmat}
    0 &1 &0 \\
    0 &0 &1 \\
    1 &0 &0
  \end{vmat}         \\
  &\quad+96\cdot\begin{vmat}
    0 &0 &1 \\
    1 &0 &0 \\
    0 &1 &0
  \end{vmat}
  +105\cdot\begin{vmat}
    0 &0 &1 \\
    0 &1 &0 \\
    1 &0 &0
  \end{vmat}
\end{align*}
\end{frame}




\begin{frame}{Permutation matrices}
\ExecuteMetaData[../det1.tex]{NotationForPermutationMatrices}

\pause
\df[df:permutation]
\ExecuteMetaData[../det1.tex]{df:permutation}

\pause
\ex[ex:AllTwoThreePerms]
\ExecuteMetaData[../det1.tex]{ex:AllTwoThreePerms}
\end{frame}
\begin{frame}
\ExecuteMetaData[../det1.tex]{AssociatePermutationWithPermutationMatrix}

\ex
Associated with the $4$-permutation $\psi=\sequence{2,4,3,1}$ is the
matrix whose rows are the matching $\iota$'s.
\begin{equation*}
  P_{\psi}
  =
  \begin{mat}
    \iota_2 \\
    \iota_4 \\
    \iota_3 \\
    \iota_1
  \end{mat}
  =
  \begin{mat}
    0 &1 &0 &0 \\
    0 &0 &0 &1 \\
    0 &0 &1 &0 \\
    1 &0 &0 &0
  \end{mat}
\end{equation*}
\end{frame}




\begin{frame}{Permutation expansion}
\df[df:PermutationExpansion]
\ExecuteMetaData[../det1.tex]{df:PermutationExpansion}

\pause
\medskip
\ExecuteMetaData[../det1.tex]{SummationForPermutationExpansion}
\end{frame}
\begin{frame}
\ex
Recall that there are two $2$-permutations
\( \phi_1=\sequence{1,2} \) and \( \phi_2=\sequence{2,1} \).
So for the $\nbyn{2}$ case, the sum over all permutations has two terms.

\pause
These are the associated permutation matrices
\begin{equation*}
  P_{\phi_1}=
  \begin{mat}
    1 &0 \\
    0 &1
  \end{mat}
  \qquad
  P_{\phi_2}=
  \begin{mat}
    0 &1 \\
    1 &0
  \end{mat}
\end{equation*}
\pause
giving this expansion.
\begin{align*}
  \begin{vmat}
    t_{1,1}  &t_{1,2} \\
    t_{2,1}  &t_{2,2}
  \end{vmat}
  &=
  t_{1,1}t_{2,2}
  \begin{vmat}
    1  &0 \\
    0  &1
  \end{vmat}               
  +
  t_{1,2}t_{2,1}
  \begin{vmat}
    0  &1 \\
    1  &0
  \end{vmat}               \\
  &=
  t_{1,1}t_{2,2}\cdot 1
  +
  t_{1,2}t_{2,1}\cdot (-1)
\end{align*}
The determinant $\deter{P_{\phi_2}}$ equals $-1$ because we can bring that to 
the identitiy matrix with one row swap.
\pause
Renaming the matrix entries gives the familiar $\nbyn{2}$ formula.
\begin{equation*}
  \begin{vmat}
    a  &b  \\
    c  &d
  \end{vmat}
  =ad-bc
\end{equation*}
\end{frame}




\begin{frame}
Proofs for these two theorems are in the next subsection.

\th[th:DetsExist]
\ExecuteMetaData[../det1.tex]{th:DetsExist}

\th[th:DeterminantOfAMatrixEqualsDeterminantOfTranspose]
\ExecuteMetaData[../det1.tex]{th:DeterminantOfAMatrixEqualsDeterminantOfTranspose}

\pause
\medskip
\co[cor:ColSwapChgSign]
\ExecuteMetaData[../det1.tex]{co:ColSwapChgSign}
\pause
\pf
\ExecuteMetaData[../det1.tex]{pf:ColSwapChgSign}
\qed
\end{frame}




% ..... Four.I.4 .....
\section{Determinants Exist}
%..........
\begin{frame}{Inversion}
\df[df:Inversion]
\ExecuteMetaData[../det1.tex]{df:Inversion}
\pause
\ex The permutation $\phi=\sequence{3,2,1}$ has three inversions:
$3$ is before $2$, $3$ is before $1$, and 
$2$ is before $1$.
\end{frame}
\begin{frame}
\ex
Here there are two inversions:
\begin{equation*}
  \begin{mat}
    0 &1 &0 &0 \\
    1 &0 &0 &0 \\
    0 &0 &0 &1 \\
    0 &0 &1 &0
  \end{mat}
\end{equation*}
row one is inverted with respect to row~two and row~three is inverted with 
respect to row~four.
\end{frame}




%..........
\begin{frame}
\lm[le:SwapsChangeSgn]
\ExecuteMetaData[../det1.tex]{lm:SwapsChangeSgn}

\pause
\pf
\ExecuteMetaData[../det1.tex]{pf:SwapsChangeSgn0}
\end{frame}
\begin{frame}
\ExecuteMetaData[../det1.tex]{pf:SwapsChangeSgn1}
\end{frame}
\begin{frame}
\ExecuteMetaData[../det1.tex]{pf:SwapsChangeSgn2}
\qed
\end{frame}




%..........
\begin{frame}{Signum}
\df[df:Signum]
\ExecuteMetaData[../det1.tex]{df:Signum}

\pause
\ex The permutation
$\phi=\sequence{3,2,1}$
associated with this matrix 
\begin{equation*}
  P_{\phi}=
  \begin{mat}
    0 &0 &1 \\
    0 &1 &0 \\
    1 &0 &0
  \end{mat}
\end{equation*}
has three inversions
$3$ is before $2$, $3$ is before $1$, and 
$2$ is before $1$.
So the signum is $\sgn(\phi)=-1$.

\pause
\ex
The permutation $\psi=\sequence{3,2,4,1}$
has four inversions: $3$ is before $2$ and~$1$,
$2$ is before~$1$, and $4$ is before~$1$.
So $\sgn(\psi)=+1$.
\end{frame}




%..........
\begin{frame}
\co[cor:ParityInversEqParitySwaps]
\ExecuteMetaData[../det1.tex]{co:ParityInversEqParitySwaps}

\pause
\pf
\ExecuteMetaData[../det1.tex]{pf:ParityInversEqParitySwaps}
\qed
\end{frame}




%..........
\begin{frame}{Determinants exist}
\ExecuteMetaData[../det1.tex]{DefiningDFunction}
\end{frame}
\begin{frame}
\lm[lm:DeterminantsExist]
\ExecuteMetaData[../det1.tex]{lm:DeterminantsExist}

\pf
\ExecuteMetaData[../det1.tex]{pf:DeterminantsExist0}

\pause
\ExecuteMetaData[../det1.tex]{pf:DeterminantsExist1}
% \end{frame}
% \begin{frame}

\pause
\ExecuteMetaData[../det1.tex]{pf:DeterminantsExist2}
\end{frame}
\begin{frame}
\ExecuteMetaData[../det1.tex]{pf:DeterminantsExist3}
\end{frame}
\begin{frame}
\ExecuteMetaData[../det1.tex]{pf:DeterminantsExist4}
\end{frame}
\begin{frame}
\ExecuteMetaData[../det1.tex]{pf:DeterminantsExist5}
\end{frame}
\begin{frame}
\ExecuteMetaData[../det1.tex]{pf:DeterminantsExist6}
\end{frame}
\begin{frame}
\ExecuteMetaData[../det1.tex]{pf:DeterminantsExist7}
\qed
\end{frame}




%..........
\begin{frame}{The determinant of the transpose}
\th[th:DetMatrixEqualsDetTrans]
\ExecuteMetaData[../det1.tex]{th:DetMatrixEqualsDetTrans}

\pf
\ExecuteMetaData[../det1.tex]{pf:DetMatrixEqualsDetTrans0}
\end{frame}
\begin{frame}
\ExecuteMetaData[../det1.tex]{pf:DetMatrixEqualsDetTrans1}
\qed

\pause
\ex
We know the formula for $\nbyn{2}$~matrices.
\begin{equation*}
  \begin{vmat}
    a  &b  \\
    c  &d
  \end{vmat}
  =ad-bc
  \qquad
  \begin{vmat}
    a  &c  \\
    b  &d
  \end{vmat}
  =ad-cb
\end{equation*}
\end{frame}



% % ..... Four.I.2 .....
% \section{}
% %..........
% \begin{frame}
% \end{frame}




%...........................
% \begin{frame}
% \ExecuteMetaData[../gr3.tex]{GaussJordanReduction}
% \df[def:RedEchForm]
% 
% \end{frame}
\end{document}
