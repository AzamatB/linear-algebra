% see: https://groups.google.com/forum/?fromgroups#!topic/comp.text.tex/s6z9Ult_zds
\makeatletter\let\ifGm@compatii\relax\makeatother 
\documentclass[10pt,t,serif,professionalfont]{beamer}
\PassOptionsToPackage{pdfpagemode=FullScreen}{hyperref}
\PassOptionsToPackage{usenames,dvipsnames}{color}
% \DeclareGraphicsRule{*}{mps}{*}{}
\usepackage{../linalgjh}
\usepackage{present}
\usepackage{xr}\externaldocument{../det1} % read refs from .aux file
\usepackage{xr}\externaldocument{../map4} % read refs from .aux file
\usepackage{catchfilebetweentags}
\usepackage{etoolbox} % from http://tex.stackexchange.com/questions/40699/input-only-part-of-a-file-using-catchfilebetweentags-package
\makeatletter
\patchcmd{\CatchFBT@Fin@l}{\endlinechar\m@ne}{}
  {}{\typeout{Unsuccessful patch!}}
\makeatother

\mode<presentation>
{
  \usetheme{boxes}
  \setbeamercovered{invisible}
  \setbeamertemplate{navigation symbols}{} 
}
\addheadbox{filler}{\ }  % create extra space at top of slide 
\hypersetup{colorlinks=true,linkcolor=blue} 

\title[Determinants] % (optional, use only with long paper titles)
{Four.I Determinants; Definition}

\author{\textit{Linear Algebra} \\ {\small Jim Hef{}feron}}
\institute{
  \texttt{http://joshua.smcvt.edu/linearalgebra}
}
\date{}


\subject{Determinants}
% This is only inserted into the PDF information catalog. Can be left
% out. 

\begin{document}
\begin{frame}
  \titlepage
\end{frame}

% =============================================
% \begin{frame}{Reduced Echelon Form} 
% \end{frame}



% ..... Four.I.1,2 .....
\section{Properties of Determinants}
%..........
\begin{frame}{Nonsingular matrices}
An \( \nbyn{n} \) matrix \( T \) is nonsingular if and only if
each of these holds:%
\ExecuteMetaData[../det1.tex]{EquivalentOfNonsingular}
In this chapter we will give a formula that determines whether a
matrix is nonsingular.
\end{frame}




\begin{frame}
\ExecuteMetaData[../det1.tex]{DeterminantIntro}  
\end{frame}




\begin{frame}
We will define the determinant function not by listing how 
to compute it (because that turns out to have 
no immediately clear connection to its purpose)
but instead by some of its properties.
Then we will show that only one function with those properties exists.
\end{frame}




\begin{frame}{Definition of determinant}
\df[def:Det]
\ExecuteMetaData[../det1.tex]{df:Det}

\pause 
\re[rem:SwapRowsRedun] 
\ExecuteMetaData[../det1.tex]{re:SwapRowsRedun}
\end{frame}




\begin{frame}{Consequences of the definition}
\lm[le:IdenRowsDetZero]
\ExecuteMetaData[../det1.tex]{lm:IdenRowsDetZero}

\pause 
\pf 
\ExecuteMetaData[../det1.tex]{pf:IdenRowsDetZero0}

\pause
\ExecuteMetaData[../det1.tex]{pf:IdenRowsDetZero1}
\end{frame}
\begin{frame}
\ExecuteMetaData[../det1.tex]{pf:IdenRowsDetZero2}

\pause
\ExecuteMetaData[../det1.tex]{pf:IdenRowsDetZero3}  
\end{frame}
\begin{frame}
\ExecuteMetaData[../det1.tex]{pf:IdenRowsDetZero4}
\qed
\end{frame}




\begin{frame}
We can compute the determinant of a matrix using Gauss's Method
(presuming that the determinant function exists, which we will
cover later).

\ex  On this matrix the Gauss's Method steps on the first column are 
is $-2\rho_1+\rho_2$ and $-3\rho_1+\rho_3$.
These leave the determinant unchanged by property~(1).
\begin{equation*}
  \begin{vmat}
    1  &3  &-2 \\
    2  &0  &4  \\
    3  &-1 &5
  \end{vmat}
  =
  \begin{vmat}
    1  &3   &-2 \\
    0  &-6  &-8  \\
    0  &-10 &-11
  \end{vmat}
\end{equation*}
\pause
On the second column we do $-(5/3)\rho_2+\rho_3$.
Again, property~(1) says the determinant is unchanged.
\begin{equation*}
  =
  \begin{vmat}
    1  &3   &-2 \\
    0  &-6  &-8  \\
    0  &0   &-7/3
  \end{vmat}
\end{equation*}
\pause
By the prior lemma we find the determinant by taking the product down the
diagonal.
\begin{equation*}
  =1\cdot(-6)\cdot(-7/3)=14
\end{equation*}
\end{frame}
\begin{frame}
\ex
This matrix requires a row swap, which changes the sign of the determinant.
\begin{equation*}
  \begin{vmat}
    0  &3  &1 \\
    1  &2  &0 \\
    1  &5  &2
  \end{vmat}
  =
  -\begin{vmat}
    1  &2  &0 \\
    0  &3  &1 \\
    1  &5  &2
  \end{vmat}
\end{equation*}
Performing $-\rho_1+\rho_3$
\begin{equation*}
  =-\begin{vmat}
    1  &2  &0 \\
    0  &3  &1 \\
    0  &3  &2
  \end{vmat}
\end{equation*}
and $-\rho_2+\rho_3$
\begin{equation*}
  =-\begin{vmat}
    1  &2  &0 \\
    0  &3  &1 \\
    0  &0  &1
  \end{vmat}
\end{equation*}
and then multiplying down the diagonal gives that the determinant
of the original matrix is $-3$.
\end{frame}



\begin{frame}{The $\nbyn{n}$ determinant is unique}
\lm[lm:DetFcnIsUnique]
\ExecuteMetaData[../det1.tex]{lm:DetFcnIsUnique}

\pause 
\pf 
\ExecuteMetaData[../det1.tex]{pf:DetFcnIsUnique}
\qed
\end{frame}



% ..... Four.I.3 .....
\section{The Permutation Expansion}
\begin{frame}
\ExecuteMetaData[../det1.tex]{DifferentGaussMethodReductions}
\end{frame}
\begin{frame}
That the above computation gives a consistent result for these two
ways to do a reduction on one matrix does not ensure that determinants
always give a well-defined value.
The way that we have given to compute determinant values does not
plainly eliminate the possibility that there might be, say,
two reductions of some $\nbyn{7}$~matrix that lead to different 
determinant value outputs.
In that case we would not have a function, since the definition of a
function is that for each input there must be exactly one output. 

\pause
To show that determinants are well-defined 
we will give an alternative way to compute
the value of a determinant. 
This new way is less useful in practice since it 
makes the computations awkward and slow, which is why we didn't start with it.
But it is useful for theory since it makes the proof that we need easier.
\end{frame}



\begin{frame}{The determinant function is not linear}
\ex
The determinant function is not linear.
Here,
the second matrix is twice the first but the determinant does not double.
\begin{equation*}
  \begin{vmat}
    3  &-3  &9 \\
    1  &-1   &7 \\
    2  &4   &0
  \end{vmat}
  =-72
  \quad
  \begin{vmat}
    6  &-6  &18 \\
    2  &-2   &14 \\
    4  &8   &0
  \end{vmat}
  =-576
\end{equation*}
Instead, the determinant scales one row at a time:   
\begin{equation*}
  \begin{vmat}
    3  &-3  &9 \\
    1  &-1   &7 \\
    2  &4   &0
  \end{vmat}
  =3\cdot
  \begin{vmat}
    1  &-1  &3 \\
    1  &-1   &7 \\
    2  &4   &0
  \end{vmat}
\end{equation*}
by property~(3) of \nearbydefinition{def:Det}
\begin{equation*}
  =6\cdot
  \begin{vmat}
    1  &-1  &3 \\
    1  &-1   &7 \\
    1  &2   &0
  \end{vmat}
  =6\cdot(-12)
\end{equation*}
\end{frame}



%..........
\begin{frame}{Multilinear}
\lm[lem:DetsMultilinear]
\ExecuteMetaData[../det1.tex]{lm:DetsMultilinear}

\pause
\pf
\ExecuteMetaData[../det1.tex]{pf:DetsMultilinear0}

\pause
\ExecuteMetaData[../det1.tex]{pf:DetsMultilinear1}
\end{frame}
\begin{frame}
\ExecuteMetaData[../det1.tex]{pf:DetsMultilinear2}
\end{frame}
\begin{frame}
\ExecuteMetaData[../det1.tex]{pf:DetsMultilinear3}
\qed
\end{frame}




\begin{frame}{Permutation matrices}
\df[df:permutation]
\ExecuteMetaData[../det1.tex]{df:permutation}

\pause
Recall Definition~Three.IV.\ref{df:PermutationMatrix},
that a \definend{permutation matrix}
is square and all of its entries are~$0$'s except for
a single~$1$ in each row and column.

\ExecuteMetaData[../det1.tex]{NotationForPermutationMatrices}
\end{frame}




\begin{frame}{Permutation expansion}
\df[df:PermutationExpansion]
\ExecuteMetaData[../det1.tex]{df:PermutationExpansion}

\pause
\medskip
\ExecuteMetaData[../det1.tex]{SummationForPermutationExpansion}
\end{frame}




\begin{frame}
\th[th:DetsExist]
\ExecuteMetaData[../det1.tex]{th:DetsExist}

\pause
\th[th:DeterminantOfAMatrixEqualsDeterminantOfTranspose]
\ExecuteMetaData[../det1.tex]{th:DeterminantOfAMatrixEqualsDeterminantOfTranspose}

\pause
\co[cor:ColSwapChgSign]
\ExecuteMetaData[../det1.tex]{co:ColSwapChgSign}
\pause
\pf
\ExecuteMetaData[../det1.tex]{pf:ColSwapChgSign}
\qed
\end{frame}




% ..... Four.I.4 .....
\section{Determinants Exist}
%..........
\begin{frame}{Inversion}
\df[df:Inversion]
\ExecuteMetaData[../det1.tex]{df:Inversion}
\end{frame}




%..........
\begin{frame}
\lm[le:SwapsChangeSgn]
\ExecuteMetaData[../det1.tex]{lm:SwapsChangeSgn}

\pause
\pf
\ExecuteMetaData[../det1.tex]{pf:SwapsChangeSgn0}
\end{frame}
\begin{frame}
\ExecuteMetaData[../det1.tex]{pf:SwapsChangeSgn1}
\end{frame}
\begin{frame}
\ExecuteMetaData[../det1.tex]{pf:SwapsChangeSgn2}
\qed
\end{frame}




%..........
\begin{frame}{Signum}
\df[df:Signum]
\ExecuteMetaData[../det1.tex]{df:Signum}
\end{frame}




%..........
\begin{frame}
\co[cor:ParityInversEqParitySwaps]
\ExecuteMetaData[../det1.tex]{co:ParityInversEqParitySwaps}

\pause
\pf
\ExecuteMetaData[../det1.tex]{pf:ParityInversEqParitySwaps}
\qed
\end{frame}




%..........
\begin{frame}{Determinants exist}
\ExecuteMetaData[../det1.tex]{DefiningDFunction}
\end{frame}
\begin{frame}
\lm[lm:DeterminantsExist]
\ExecuteMetaData[../det1.tex]{lm:DeterminantsExist}

\pf
\ExecuteMetaData[../det1.tex]{pf:DeterminantsExist0}

\pause
\ExecuteMetaData[../det1.tex]{pf:DeterminantsExist1}
% \end{frame}
% \begin{frame}

\pause
\ExecuteMetaData[../det1.tex]{pf:DeterminantsExist2}
\end{frame}
\begin{frame}
\ExecuteMetaData[../det1.tex]{pf:DeterminantsExist3}
\end{frame}
\begin{frame}
\ExecuteMetaData[../det1.tex]{pf:DeterminantsExist4}
\end{frame}
\begin{frame}
\ExecuteMetaData[../det1.tex]{pf:DeterminantsExist5}
\end{frame}
\begin{frame}
\ExecuteMetaData[../det1.tex]{pf:DeterminantsExist6}
\end{frame}
\begin{frame}
\ExecuteMetaData[../det1.tex]{pf:DeterminantsExist7}
\qed
\end{frame}




%..........
\begin{frame}{The determinant of the transpose}
\th[th:DetMatrixEqualsDetTrans]
\ExecuteMetaData[../det1.tex]{th:DetMatrixEqualsDetTrans}

\pf
\ExecuteMetaData[../det1.tex]{pf:DetMatrixEqualsDetTrans0}
\end{frame}
\begin{frame}
\ExecuteMetaData[../det1.tex]{pf:DetMatrixEqualsDetTrans1}
\qed
\end{frame}



% % ..... Four.I.2 .....
% \section{}
% %..........
% \begin{frame}
% \end{frame}




%...........................
% \begin{frame}
% \ExecuteMetaData[../gr3.tex]{GaussJordanReduction}
% \df[def:RedEchForm]
% 
% \end{frame}
\end{document}
