% see: https://groups.google.com/forum/?fromgroups#!topic/comp.text.tex/s6z9Ult_zds
\makeatletter\let\ifGm@compote\relax\makeatother 
\documentclass[10pt,t]{beamer}
\usefonttheme{professionalfonts}
\usefonttheme{serif}
\PassOptionsToPackage{pdfpagemode=FullScreen}{hyperref}
\PassOptionsToPackage{usenames,dvipsnames}{color}
% \DeclareGraphicsRule{*}{mps}{*}{}
\usepackage{../linalgjh}
\usepackage{present}
\usepackage{xr}\externaldocument{../det2} % read refs from .aux file
\usepackage{xr}\externaldocument{../cramer} % read refs from .aux file
\usepackage{catchfilebetweentags}
\usepackage{etoolbox} % from http://tex.stackexchange.com/questions/40699/input-only-part-of-a-file-using-catchfilebetweentags-package
\makeatletter
\patchcmd{\CatchFBT@Fin@l}{\endlinechar\m@ne}{}
  {}{\typeout{Unsuccessful patch!}}
\makeatother

\mode<presentation>
{
  \usetheme{boxes}
  \setbeamercovered{invisible}
  \setbeamertemplate{navigation symbols}{} 
}
\addheadbox{filler}{\ }  % create extra space at top of slide 
\hypersetup{colorlinks=true,linkcolor=blue} 

\title[Geometry of Determinants] % (optional, use only with long paper titles)
{Four.II Geometry of Determinants}

\author{\textit{Linear Algebra} \\ {\small Jim Hef{}feron}}
\institute{
  \texttt{http://joshua.smcvt.edu/linearalgebra}
}
\date{}


\subject{Geometry of Determinants}
% This is only inserted into the PDF information catalog. Can be left
% out. 

\begin{document}
\begin{frame}
  \titlepage
\end{frame}

% =============================================
% \begin{frame}{Reduced Echelon Form} 
% \end{frame}



% ..... Four.II .....
\section{Determinants as size functions}
%..........
\begin{frame}{Box}
This parallelogram is defined by the two vectors.
\centergraphic{../ch4.30}

\df[df:Box]
\ExecuteMetaData[../det2.tex]{df:Box}

\medskip
This is the same as the definition of the span,
but with the scalars limited to the unit 
interval. 
\end{frame}
\begin{frame}{Area of a box}
\centergraphic{../ch4.31}
\begin{align*}
  \text{box area}
  &=\text{rectangle area}-\text{area of $A$}-\cdots-\text{area of F} \\
  &=(x_1+x_2)(y_1+y_2)-x_2y_1-x_1y_1/2        \\
    &\quad-x_2y_2/2-x_2y_2/2-x_1y_1/2-x_2y_1         \\
  &=x_1y_2-x_2y_1        
\end{align*}
We will argue that
the determinant of a square matrix   
gives the size of the box formed by the matrix's columns.  
\begin{equation*}
  \begin{vmat}
    x_1  &x_2  \\
    y_1  &y_2
  \end{vmat}
  =x_1y_2-x_2y_1
\end{equation*}
\end{frame}
\begin{frame}{Restatement for columns}
We switch from considering the determinant as a function of the 
rows to considering it as a function of the columns.
Observe that the row operation conditions in the definition 
translate over to column operation conditions.

The first condition is that a
determinant is unchanged by a row combination:~where 
$A$ is square and  
$A\!\!\raisebox{-.5ex}{\smash{\grstep{k\rho_i+\rho_j}}}\!\! \hat{A}$
(with~$i\neq j$)
then $\det(A)=\det{\hat{A}}$.
To see that the same holds about column combinations, recall that 
the determinant of a matrix equals 
the determinant of its transpose.
We can do a column combination on a square matrix
by transposing, doing a row combination,
and transposing back;
none of these changes the determinant.

\pause
The second condition in the definition is that a row swap 
changes the determinant's sign. 
The third condition is that multiplying a row by a
scalar multiplies the entire determinant by that scalar.
Here also transposing, performing the row operation, and transposing back
shows that the column translations hold.

The final condition is that the determinant of the identity matrix is~$1$.
This isn't about row operations so we don't need to
translate. 
\end{frame}




\begin{frame}\vspace*{-1ex}
  \frametitle{Geometric interpretation of the definition}
Condition~(3) from the definition of determinant is 
that rescaling a column rescales the entire determinant
$\det(\ldots,k\vec{v}_i,\ldots)=k\cdot\det(\ldots,\vec{v}_i,\ldots)$.
This fits :
if we scale a column by a factor~$k$ then the size of the box
scales by that factor. 
\begin{center}
  \includegraphics{../ch4.32}
  \qquad
  \includegraphics{../ch4.33}
\end{center}
% \end{frame}
% \begin{frame}

\pause
Condition~(1) is that the determinant is unaffected by 
combining columns.
The picture 
\begin{center}
  \includegraphics{../ch4.34}
  \quad
  \includegraphics{../ch4.35}
\end{center}   
shows that the box
formed by $\vec{v}$ and~$k\vec{v}+\vec{w}$ 
is slanted at a different angle than the original box but the two have
the same base and height, and hence the same area.
\end{frame}
\begin{frame}
As we noted after the determinamt's definition, 
condition~(2) is a consequence of the 
others so we leave it aside for the moment.  

Condition~(4) is that the determinant of the identity matrix is~$1$.
This fits also
with the proposal that the determinant
gives the size of the box formed by the columns of the matrix.
\centergraphic{../ch4.36}
\end{frame}




\begin{frame}{Orientation}
\re[re:PropertyTwoGivesSign] 
Although condition~(2) is redundant, what it says is notable.
Consider these two.
Swapping changes the sign of the determinant.
\begin{center} \small
  \begin{tabular}{c@{\hspace*{8em}}c}
    \includegraphics{../ch4.37}  
      &\includegraphics{../ch4.38}  \\[.25ex]
    \ $\begin{vmat}[r]
        4  &1   \\
        2  &3
      \end{vmat}=10$
      &\ $\begin{vmat}[r]
          1  &4   \\
          3  &2
        \end{vmat}=-10$
  \end{tabular}
\end{center}
\pause
On the left $\vec{u}$ is first in the matrix and then we follow the
counterclockwise arc to $\vec{v}$,
and get a positive size.
On the right following the clockwise arc gives a negative size.
The sign returned by the size function, the determinant, reflects the 
\definend{orientation}\index{box!orientation}
or \definend{sense} of the box.

(The same effect appears if we use Condition~(3) with a negative scalar.)
\end{frame}

\begin{frame}{More on orientation: three space} 
Starting with these two vectors we want to form a three-space box 
with positive size.
Those two vectors span a plane dividing three-space in two,
the part above the plane and the part below.
\begin{equation*}
  \vec{v}_1=\colvec{1 \\ 4 \\ 1},\,
  \vec{v}_2=\colvec{-2 \\ 3 \\ 1}
  %\vec{v}_3=\colvec{0 \\ -1 \\ 2}
  \qquad
  \vcenteredhbox{\only<1>{\includegraphics{asy/four_ii_orientation.pdf}}%
                 \only<2->{\includegraphics{asy/four_ii_orientation_pos.pdf}}}
\end{equation*}
\pause
The $\vec{v}_3$ shown is on the side of the plane containing vectors
with this property: 
a person at the tip of $\vec{v}_3$ 
looking at the box and tracing
out the parallelogram starting from the first vector~$\vec{v}_1$
finds that the trace looks counterclockwise.
Any such vector~$\vec{v}_3$ makes a box that is positive-sized.
\end{frame}
\begin{frame}
Shown is this vector~$\vec{v}_3$.
\begin{equation*}
  \vec{v}_3=\colvec{0 \\ -1 \\ 2}
  \qquad
  \begin{vmat}
    1 &-2 &0 \\
    4 &3 &-1 \\
    1 &1 &2
  \end{vmat}=25
\end{equation*}
\pause
A vector on the other side of the plane,
such as $-\vec{v}_3$, will have the same trace look clockwise, and will
give a negative determinant.
\begin{equation*}
  \begin{vmat}
    1 &-2 &0 \\
    4 &3 &1 \\
    1 &1 &-2
  \end{vmat}=-25
\end{equation*}
\only<1-6>{These pictures start above the plane, and the box looks to be oriented clockwise.}
\only<7->{But from below, the box looks to be oriented counterclockwise.}
\only<1>{\centergraphic{asy/four_ii_orientation_neg0.pdf}}%
\only<2>{\centergraphic{asy/four_ii_orientation_neg1.pdf}}%
\only<3>{\centergraphic{asy/four_ii_orientation_neg2.pdf}}%
\only<4>{\centergraphic{asy/four_ii_orientation_neg3.pdf}}%
\only<5>{\centergraphic{asy/four_ii_orientation_neg4.pdf}}%
\only<6>{\centergraphic{asy/four_ii_orientation_neg5.pdf}}%
\only<7>{\centergraphic{asy/four_ii_orientation_neg6.pdf}}%
\only<8>{\centergraphic{asy/four_ii_orientation_neg7.pdf}}%
\only<9>{\centergraphic{asy/four_ii_orientation_neg8.pdf}}%
\only<10>{\centergraphic{asy/four_ii_orientation_neg9.pdf}}%
\only<11->{\centergraphic{asy/four_ii_orientation_neg10.pdf}}%
\end{frame}




\begin{frame}{Determinants are multiplicative}
\th[th:MatChVolByDetMat]  
\ExecuteMetaData[../det2.tex]{th:MatChVolByDetMat}

\pause
\pf
\ExecuteMetaData[../det2.tex]{pf:MatChVolByDetMat0}

\pause
\ExecuteMetaData[../det2.tex]{pf:MatChVolByDetMat1}
\end{frame}
\begin{frame}
\ExecuteMetaData[../det2.tex]{pf:MatChVolByDetMat2}
\qed
\end{frame}
\begin{frame}
\ex
The transformation $\map{t_{\theta}}{\Re^2}{\Re^2}$ 
that rotates all vectors through a counterclockwise
angle~$\theta$ is represented 
by this matrix.
\begin{equation*}
  T_\theta=
  \rep{t_\theta}{\stdbasis_2,\stdbasis_2}
  =
  \begin{mat}
    \cos\theta  &-\sin\theta \\
    \sin\theta  &cos\theta
  \end{mat}
\end{equation*}
Observe that $t_\theta$ doesn't change the size of any boxes, it just rotates 
the entire box as a rigid whole.
Note that $\deter{T_\theta}=1$.

\pause
\ex The linear transformation $\map{s}{\Re^2}{\Re^2}$
represented with respect to the standard basis by this matrix
\begin{equation*}
  S=
  \begin{mat}
    1 &2 \\
    3 &4
  \end{mat}
\end{equation*}
will, by the theorem, change the size of a box by a factor of $\deter{S}=-2$.
Here is $s$ acting on a typical box. 
\end{frame}
\begin{frame} 
The box defined by the two vectors $\vec{v}_1=\binom{1}{0}$ and
$\vec{v}_2=\binom{1}{1}$ is transformed by~$s$ to the box defined by the 
two vectors $s(\vec{v}_1)=\binom{1}{3}$ and 
$s(\vec{v}_2)=\binom{3}{7}$. 
\begin{center}
  \vcenteredhbox{\includegraphics{asy/four_ii_2dtransedsize0.pdf}}
  \qquad$\underrightarrow{s}$\qquad
  \vcenteredhbox{\includegraphics{asy/four_ii_2dtransedsize1.pdf}}
\end{center}
Note the change in orientation, matching that the determinant is 
negative.

\pause
The two sizes are easy.
\begin{equation*}
  \begin{vmat}
    1 &1 \\ 
    0 &1  
  \end{vmat}
  =1
  \qquad
  \begin{vmat}
    1 &3 \\ 
    3 &7  
  \end{vmat}
  =-2
\end{equation*}
\end{frame}


\begin{frame}{Determinant of the inverse}
\co[co:DeterminantOfInverseIsInverseOfDeterminant]  
\ExecuteMetaData[../det2.tex]{co:DeterminantOfInverseIsInverseOfDeterminant}

\pause
\pf
\ExecuteMetaData[../det2.tex]{pf:DeterminantOfInverseIsInverseOfDeterminant}
\qed
\end{frame}




\begin{frame}{Volume}
\df[df:Volume]  
\ExecuteMetaData[../det2.tex]{df:Volume}

\ex 
The box formed by the vectors 
\begin{equation*}
  \sequence{\vec{v}_1,\vec{v}_2}
  =\sequence{\colvec[r]{-1 \\ 1},
             \colvec{1 \\ 1}}
  \hspace*{4em}
  \vcenteredhbox{\includegraphics{asy/four_ii_negvolbox.pdf}}
\end{equation*}
gives this determinant
\begin{equation*}
  \begin{vmat}[r]
    -1 &1 \\
     1 &1
  \end{vmat}
  =-2
\end{equation*}
so its volume is~$2$.
\end{frame}




% ..... Four.Topic .....
\section{Cramer's Rule}
%..........
\begin{frame}{Geometric interpretation of linear systems}
\ExecuteMetaData[../cramer.tex]{CramersRuleExample0}
\begin{center}
 \includegraphics{../ch4.1}
\end{center}
\end{frame}
\begin{frame}
\ExecuteMetaData[../cramer.tex]{CramersRuleExample1}
\begin{center}
 \includegraphics{../ch4.1}
\end{center}
\end{frame}
\begin{frame}
Consider expanding only one side of the parallelogram.
Compare the sizes of these shaded boxes.
\begin{center}
   \includegraphics{../ch4.2}
   \hfil
   \includegraphics{../ch4.3}
   \hfil
   \includegraphics{../ch4.4}
\end{center}
\pause
\ExecuteMetaData[../cramer.tex]{CramersRuleExample2}
\end{frame}
\begin{frame}
\ExecuteMetaData[../cramer.tex]{CramersRuleExample3}
\pause
\ExecuteMetaData[../cramer.tex]{CramersRuleExample4}
\pause
The symmetric argument for the other side gives this.
\begin{equation*}
  x_2=
  \frac{
  \begin{vmat}
    1  &6  \\
    3  &8
  \end{vmat}}{
  \begin{vmat}
    1  &2  \\
    3  &1
  \end{vmat}}
  =2
\end{equation*}
\end{frame}


\begin{frame}{Cramer's Rule}
Let $A$ be an $\nbyn{n}$ matrix, let $\vec{b}$ be an $n$-tall column vector,
and consider the linear system $A\vec{x}=\vec{b}$.
For any~$i\in[1,\ldots,n]$ let $B_i$ be the matrix obtained by
substituting $\vec{b}$ for column~$i$ of $A$.
Then the value of the $i$-th unknown is $x_i=\deter{B_i}/\deter{A}$.

\pause
\medskip
\nearbyexercise{ex:CramerRule} gives the proof.

\end{frame}
\begin{frame}
\ex
Given this system
\begin{equation*}
  \begin{linsys}{3}
    2x_1 &+ &x_2 &- &x_3 &= &4 \\
     x_1 &+ &3x_2 &  &   &= &2 \\
         &  &x_2 &- &5x_3 &= &0 \\
  \end{linsys}
\end{equation*}
we can rewrite it as 
\begin{equation*}
  \begin{mat}[r]
    2 &1 &-1 \\
    1 &3 &0  \\
    0 &1 &-5
  \end{mat}
  \colvec{x_1 \\ x_2 \\ x_3}
  =
  \colvec{4 \\ 2 \\ 0}
\end{equation*}
and
\begin{equation*}
  \deter{A}=
  \begin{vmat}[r]
    2 &1 &-1 \\
    1 &3 &0  \\
    0 &1 &-5
  \end{vmat}
  =-26
  \qquad
  \deter{B_2}=
  \begin{vmat}[r]
    2 &4 &-1 \\
    1 &2 &0  \\
    0 &0 &-5
  \end{vmat}
  =0
\end{equation*}
so $x_2=0/-26=0$.
\end{frame}
\begin{frame}{A caution}
Cramer's Rule is an interesting application of the geometry 
that we have developed.
And it allows us to mentally solve systems with
two or three variables that use simple numbers.
But don't use it for systems having many variables.
Taking a determinant of a general large matrix is very slow.   
\end{frame}




%...........................
% \begin{frame}g
% \ExecuteMetaData[../gr3.tex]{GaussJordanReduction}
% \df[def:RedEchForm]
% 
% \end{frame}
\end{document}
