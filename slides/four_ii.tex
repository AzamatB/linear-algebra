% see: https://groups.google.com/forum/?fromgroups#!topic/comp.text.tex/s6z9Ult_zds
\makeatletter\let\ifGm@compote\relax\makeatother 
\documentclass[10pt,t,serif,professionalfont]{beamer}
\PassOptionsToPackage{pdfpagemode=FullScreen}{hyperref}
\PassOptionsToPackage{usenames,dvipsnames}{color}
% \DeclareGraphicsRule{*}{mps}{*}{}
\usepackage{../linalgjh}
\usepackage{present}
\usepackage{xr}\externaldocument{../det2} % read refs from .aux file
\usepackage{catchfilebetweentags}
\usepackage{etoolbox} % from http://tex.stackexchange.com/questions/40699/input-only-part-of-a-file-using-catchfilebetweentags-package
\makeatletter
\patchcmd{\CatchFBT@Fin@l}{\endlinechar\m@ne}{}
  {}{\typeout{Unsuccessful patch!}}
\makeatother

\mode<presentation>
{
  \usetheme{boxes}
  \setbeamercovered{invisible}
  \setbeamertemplate{navigation symbols}{} 
}
\addheadbox{filler}{\ }  % create extra space at top of slide 
\hypersetup{colorlinks=true,linkcolor=blue} 

\title[Geometry of Determinants] % (optional, use only with long paper titles)
{Four.II Geometry of Determinants}

\author{\textit{Linear Algebra} \\ {\small Jim Hef{}feron}}
\institute{
  \texttt{http://joshua.smcvt.edu/linearalgebra}
}
\date{}


\subject{Geometry of Determinants}
% This is only inserted into the PDF information catalog. Can be left
% out. 

\begin{document}
\begin{frame}
  \titlepage
\end{frame}

% =============================================
% \begin{frame}{Reduced Echelon Form} 
% \end{frame}



% ..... Four.II .....
\section{Determinants as size functions}
%..........
\begin{frame}{Box}
This parallelogram is defined by the two vectors.
\centergraphic{../ch4.30}

\df[df:Box]
\ExecuteMetaData[../det2.tex]{df:Box}

\pause
The box like the span, except that the scalars are limited to the unit 
interval. 

\pause
\no
In this section we consider the determinant as a function of the columns
of the matrix.
\end{frame}
\begin{frame}{Area of a two dimensional box}
\centergraphic{../ch4.31}
\begin{align*}
  \text{box area}
  &=\text{rectangle area}-\text{area of $A$}-\cdots-\text{area of F} \\
  &=(x_1+x_2)(y_1+y_2)-x_2y_1-x_1y_1/2        \\
    &\quad-x_2y_2/2-x_2y_2/2-x_1y_1/2-x_2y_1         \\
  &=x_1y_2-x_2y_1        
\end{align*}
\pause
That area equals the value of the determinant.
\begin{equation*}
  \begin{vmat}
    x_1  &x_2  \\
    y_1  &y_2
  \end{vmat}
  =x_1y_2-x_2y_1
\end{equation*}
This is no coincidence.
The properties from the definition of determinants 
make good postulates for a function 
that measures the size of the box
defined by the columns of the matrix.  
\end{frame}



\begin{frame}{Definition of determinant reinterpreted}
Consider property~(3) from the definition of determinant,
that $\det{\ldots,k\vec{v}_i,\ldots}=k\det{\ldots,\vec{v}_i,\ldots}$.
This meshes perfectly with the intuition that the determinant of
a matrix gives the size of the box formed by the columns of the matrix.
If we scale a column by a factor~$k$ then the size of the box
scales by that factor. 
\begin{center}
  \includegraphics{../ch4.32}
  \qquad
  \includegraphics{../ch4.33}
\end{center}
\end{frame}








%...........................
% \begin{frame}
% \ExecuteMetaData[../gr3.tex]{GaussJordanReduction}
% \df[def:RedEchForm]
% 
% \end{frame}
\end{document}
