% see: https://groups.google.com/forum/?fromgroups#!topic/comp.text.tex/s6z9Ult_zds
\makeatletter\let\ifGm@compote\relax\makeatother 
\documentclass[10pt,t]{beamer}
\usefonttheme{professionalfonts}
\usefonttheme{serif}
\PassOptionsToPackage{pdfpagemode=FullScreen}{hyperref}
\PassOptionsToPackage{usenames,dvipsnames}{color}
% \DeclareGraphicsRule{*}{mps}{*}{}
\usepackage{../linalgjh}
\usepackage{present}
\usepackage{xr}\externaldocument{../det2} % read refs from .aux file
\usepackage{xr}\externaldocument{../cramer} % read refs from .aux file
\usepackage{catchfilebetweentags}
\usepackage{etoolbox} % from http://tex.stackexchange.com/questions/40699/input-only-part-of-a-file-using-catchfilebetweentags-package
\makeatletter
\patchcmd{\CatchFBT@Fin@l}{\endlinechar\m@ne}{}
  {}{\typeout{Unsuccessful patch!}}
\makeatother

\mode<presentation>
{
  \usetheme{boxes}
  \setbeamercovered{invisible}
  \setbeamertemplate{navigation symbols}{} 
}
\addheadbox{filler}{\ }  % create extra space at top of slide 
\hypersetup{colorlinks=true,linkcolor=blue} 

\title[Geometry of Determinants] % (optional, use only with long paper titles)
{Four.II Geometry of Determinants}

\author{\textit{Linear Algebra} \\ {\small Jim Hef{}feron}}
\institute{
  \texttt{http://joshua.smcvt.edu/linearalgebra}
}
\date{}


\subject{Geometry of Determinants}
% This is only inserted into the PDF information catalog. Can be left
% out. 

\begin{document}
\begin{frame}
  \titlepage
\end{frame}

% =============================================
% \begin{frame}{Reduced Echelon Form} 
% \end{frame}



% ..... Four.II .....
\section{Determinants as size functions}
%..........
\begin{frame}{Box}
This parallelogram is defined by the two vectors.
\centergraphic{../ch4.30}

\df[df:Box]
\ExecuteMetaData[../det2.tex]{df:Box}

\medskip
The box is like the span, except that the scalars are from the unit 
interval. 
\end{frame}
\begin{frame}{Area of a two dimensional box}
\centergraphic{../ch4.31}
\begin{align*}
  \text{box area}
  &=\text{rectangle area}-\text{area of $A$}-\cdots-\text{area of F} \\
  &=(x_1+x_2)(y_1+y_2)-x_2y_1-x_1y_1/2        \\
    &\quad-x_2y_2/2-x_2y_2/2-x_1y_1/2-x_2y_1         \\
  &=x_1y_2-x_2y_1        
\end{align*}
\pause
That area equals the value of the determinant.
\begin{equation*}
  \begin{vmat}
    x_1  &x_2  \\
    y_1  &y_2
  \end{vmat}
  =x_1y_2-x_2y_1
\end{equation*}
\end{frame}
\begin{frame}
We will argue here that
the properties of determinants 
make good postulates for a function 
that gives the size of the box formed by the columns of the matrix.  

\pause
In line with the above use of column vectors,
in this section we consider the determinant as a function of the columns
of the matrix.
Note that because the determinant of a matrix equals 
the determinant of its transpose, we can change the determinant properties
from statements about rows to statements about columns.
For instance, the first property as given earlier says that a
determinant is unaffected by a row operation $k\rho_i+\rho_j$ (with $i\neq j$).
By transposing we have the property that a determinant
does not change under combinations of columns. 
\end{frame}




\begin{frame}{Definition of determinant reinterpreted}
Recall property~(3) from the definition of determinant,
that rescaling a column rescales the entire determinant
$\det(\ldots,k\vec{v}_i,\ldots)=k\det(\ldots,\vec{v}_i,\ldots)$.
This fits with the idea that the determinant
gives the size of the box formed by the columns of the matrix:
if we scale a column by a factor~$k$ then the size of the box
scales by that factor. 
\begin{center}
  \includegraphics{../ch4.32}
  \qquad
  \includegraphics{../ch4.33}
\end{center}
% \end{frame}
% \begin{frame}

\pause
Property~(1) says that the determinant is unaffected by 
combining columns.
The picture 
\begin{center}
  \includegraphics{../ch4.34}
  \quad
  \includegraphics{../ch4.35}
\end{center}   
shows that the box
formed by $\vec{v}$ and~$k\vec{v}+\vec{w}$ 
is more slanted than the original box but the two have
the same base and height, and hence the same area.
\end{frame}
\begin{frame}
As we noted after the definition, property~(2) is a consequence of the 
other properties so we leave it aside for the moment.  

Property~(4) says that the determinant of the identity matrix is~$1$.
\centergraphic{../ch4.36}
\end{frame}




\begin{frame}{Aside: Orientation in two space}
\re[re:PropertyTwoGivesSign] 
Although property~(2) is redundant, it says something notable.
Consider these two.
\begin{center} \small
  \begin{tabular}{c@{\hspace*{8em}}c}
    \includegraphics{../ch4.37}  
      &\includegraphics{../ch4.38}  \\[.25ex]
    \ $\begin{vmat}[r]
        4  &1   \\
        2  &3
      \end{vmat}=10$
      &\ $\begin{vmat}[r]
          1  &4   \\
          3  &2
        \end{vmat}=-10$
  \end{tabular}
\end{center}
Swapping changes the sign;
on the left we take $\vec{u}$ first in the matrix and then follow the
counterclockwise arc to $\vec{v}$,
following the counterclockwise arc and
and get a positive size, while
on the right following the clockwise arc gives a negative size.
The sign returned by the size function, the determinant, reflects the 
\definend{orientation}\index{box!orientation}\index{orientation} 
or \definend{sense}\index{box!sense}\index{sense} of the box.
\pause
We see the same thing if we use Property~(3) with a negative scalar.
\end{frame}
\begin{frame}{Aside: Orientation in three space} 
Starting with these two vectors we want to form a box 
with positive size.
\begin{equation*}
  \vec{v}_1=\colvec{1 \\ 4 \\ 1},\,
  \vec{v}_2=\colvec{-2 \\ 3 \\ 1}
  %\vec{v}_3=\colvec{0 \\ -1 \\ 2}
  \qquad
  \vcenteredhbox{\only<1>{\includegraphics{asy/four_ii_orientation.png}}%
                 \only<2->{\includegraphics{asy/four_ii_orientation_pos.png}}}
\end{equation*}
Those two vectors span a plane, which divides three-space in two.
The $\vec{v}_3$ shown is on the side of the plane containing vectors
with this property: 
\pause
if a person at the tip of $\vec{v}_3$ 
looks down at the box and traces
out the parallelogram from $\zero$, to $\vec{v}_1$, to $\vec{v}_1+\vec{v}_2$,
to $\vec{v}_2$, and back to $\zero$ then their trace looks to them
to be counterclockwise.
Any such vector will make a positive-sized box.
\end{frame}
\begin{frame}
The vector shown is this.
\begin{equation*}
  \vec{v}_3=\colvec{0 \\ -1 \\ 2}
  \qquad
  \begin{vmat}
    1 &-2 &0 \\
    4 &3 &-1 \\
    1 &1 &2
  \end{vmat}=25
\end{equation*}
\pause
Any vector on the other side of the plane,
such as $-\vec{v}_3$, will have the same trace look clockwise and will
give a negative determinant.
\begin{equation*}
  \begin{vmat}
    1 &-2 &0 \\
    4 &3 &1 \\
    1 &1 &-2
  \end{vmat}=-25
\end{equation*}
\only<1>{\centergraphic{asy/four_ii_orientation_neg0.png}}%
\only<2>{\centergraphic{asy/four_ii_orientation_neg1.png}}%
\only<3>{\centergraphic{asy/four_ii_orientation_neg2.png}}%
\only<4>{\centergraphic{asy/four_ii_orientation_neg3.png}}%
\only<5>{\centergraphic{asy/four_ii_orientation_neg4.png}}%
\only<6>{\centergraphic{asy/four_ii_orientation_neg5.png}}%
\only<7>{\centergraphic{asy/four_ii_orientation_neg6.png}}%
\only<8->{\centergraphic{asy/four_ii_orientation_neg7.png}}%
\end{frame}




\begin{frame}{Determinants are multiplicative}
\th[th:MatChVolByDetMat]  
\ExecuteMetaData[../det2.tex]{th:MatChVolByDetMat}

\pause
\pf
\ExecuteMetaData[../det2.tex]{pf:MatChVolByDetMat0}

\pause
\ExecuteMetaData[../det2.tex]{pf:MatChVolByDetMat1}
\end{frame}
\begin{frame}
\ExecuteMetaData[../det2.tex]{pf:MatChVolByDetMat2}
\qed
\end{frame}
\begin{frame}
\ex
The transformation $\map{t_{\theta}}{\Re^2}{\Re^2}$ 
that rotates all vectors through a counterclockwise
angle~$\theta$ is represented 
by this matrix.
\begin{equation*}
  T_\theta=
  \rep{t_\theta}{\stdbasis_2}
  =
  \begin{mat}
    \cos\theta  &-\sin\theta \\
    \sin\theta  &cos\theta
  \end{mat}
\end{equation*}
Observe that $t_\theta$ doesn't change the size of any boxes, it just rotates 
the entire box as a rigid whole.
Note that $\deter{T_\theta}=1$.

\pause
\ex The linear transformation $\map{s}{\Re^2}{\Re^2}$
represented with respect to the standard basis by this matrix
\begin{equation*}
  S=
  \begin{mat}
    1 &2 \\
    3 &4
  \end{mat}
\end{equation*}
will, by the theorem, change the size of a box by a factor of $\deter{S}=-2$.
Here is $s$ acting on a typical box. 
\end{frame}
\begin{frame} 
The box defined by the two vectors $\vec{v}_1=\binom{1}{0}$ and
$\vec{v}_2=\binom{1}{1}$ is transformed by~$s$ to the box defined by the 
two vectors $s(\vec{v}_1)=\binom{1}{3}$ and 
$s(\vec{v}_2)=\binom{3}{7}$. 
\begin{center}
  \vcenteredhbox{\includegraphics{asy/four_ii_2dtransedsize0.pdf}}
  \quad$\mapsunder{s}$\quad
  \vcenteredhbox{\includegraphics{asy/four_ii_2dtransedsize1.pdf}}
\end{center}
Note the change in orientation.

\pause
The two sizes are easy.
\begin{equation*}
  \begin{vmat}
    1 &1 \\ 
    0 &1  
  \end{vmat}
  =1
  \qquad
  \begin{vmat}
    1 &3 \\ 
    3 &7  
  \end{vmat}
  =-2
\end{equation*}
\end{frame}


\begin{frame}{Determinant of the inverse is the inverse of the determinant}
\co[co:DeterminantOfInverseIsInverseOfDeterminant]  
\ExecuteMetaData[../det2.tex]{co:DeterminantOfInverseIsInverseOfDeterminant}

\pause
\pf
\ExecuteMetaData[../det2.tex]{pf:DeterminantOfInverseIsInverseOfDeterminant}
\qed
\end{frame}




\begin{frame}{Volume}
\df[df:Volume]  
\ExecuteMetaData[../det2.tex]{df:Volume}

\ex 
The box formed by thevectors 
\begin{equation*}
  \vec{v}_1=\colvec[r]{-1 \\ 1}
  \quad
  \vec{v}_2=\colvec{1 \\ 1}
\end{equation*}
gives this determinant
\begin{equation*}
  \begin{vmat}[r]
    -1 &1 \\
     1 &1
  \end{vmat}
  =-2
\end{equation*}
so its volume is~$2$.
\end{frame}




% ..... Four.Topic .....
\section{Cramer's Rule}
%..........
\begin{frame}{Geometric interpretation of linear systems}
\ExecuteMetaData[../cramer.tex]{CramersRuleExample0}
\centergraphic{../ch4.1}
\end{frame}
\begin{frame}
\ExecuteMetaData[../cramer.tex]{CramersRuleExample1}
\centergraphic{../ch4.1}

\pause
Next we consider expanding only one side of the parallelogram.
\end{frame}
\begin{frame}
Compare the sizes of these shaded boxes.
\begin{center}
   \includegraphics{../ch4.2}
   \hfil
   \includegraphics{../ch4.3}
   \hfil
   \includegraphics{../ch4.4}
\end{center}
\pause
\ExecuteMetaData[../cramer.tex]{CramersRuleExample2}
\end{frame}
\begin{frame}
\ExecuteMetaData[../cramer.tex]{CramersRuleExample3}
\pause
The symmetric argument for the other side gives this.
\begin{equation*}
  x_2=
  \frac{
  \begin{vmat}
    1  &6  \\
    3  &8
  \end{vmat}}{
  \begin{vmat}
    1  &2  \\
    3  &1
  \end{vmat}}
  =2
\end{equation*}
\end{frame}


\begin{frame}{Cramer's Rule}
Let $A$ be an $\nbyn{n}$ matrix, let $\vec{b}$ be an $n$-tall column vector,
and consider the linear system $A\vec{x}=\vec{b}$.
For any~$i\in[1,\ldots,n]$ let $B_i$ be the matrix obtained by
substituting $\vec{b}$ for column~$i$ of $A$.
Then the value of the $i$-th unknown is $x_i=\deter{B_i}/\deter{A}$.

\pause
\medskip
\nearbyexercise{ex:CramerRule} gives the proof.

\end{frame}
\begin{frame}
\ex
Given this system
\begin{equation*}
  \begin{linsys}{3}
    2x_1 &+ &x_2 &- &x_3 &= &4 \\
     x_1 &+ &3x_2 &  &   &= &2 \\
         &  &x_2 &- &5x_3 &= &0 \\
  \end{linsys}
\end{equation*}
we can rewrite it as 
\begin{equation*}
  \begin{mat}[r]
    2 &1 &-1 \\
    1 &3 &0  \\
    0 &1 &-5
  \end{mat}
  \colvec{x_1 \\ x_2 \\ x_3}
  =
  \colvec{4 \\ 2 \\ 0}
\end{equation*}
and
\begin{equation*}
  \deter{A}=
  \begin{vmat}[r]
    2 &1 &-1 \\
    1 &3 &0  \\
    0 &1 &-5
  \end{vmat}
  =-26
  \qquad
  \deter{B_2}=
  \begin{vmat}[r]
    2 &4 &-1 \\
    1 &2 &0  \\
    0 &0 &-5
  \end{vmat}
  =0
\end{equation*}
so $x_2=0/-26=0$.
\end{frame}
\begin{frame}{A caution}
Cramer's Rule is an interesting application of the geometry 
that we have developed.
And it allows us to mentally solve small systems, those with
two or three variables.

\pause
But it is a poor choice for systems having many variables.
Taking a determinant of a general large matrix is very slow.   
\end{frame}




%...........................
% \begin{frame}g
% \ExecuteMetaData[../gr3.tex]{GaussJordanReduction}
% \df[def:RedEchForm]
% 
% \end{frame}
\end{document}
