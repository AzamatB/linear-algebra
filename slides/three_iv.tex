% see: https://groups.google.com/forum/?fromgroups#!topic/comp.text.tex/s6z9Ult_zds
\makeatletter\let\ifGm@compatii\relax\makeatother 
\documentclass[10pt,t,serif,professionalfont]{beamer}
\PassOptionsToPackage{pdfpagemode=FullScreen}{hyperref}
\PassOptionsToPackage{usenames,dvipsnames}{color}
% \DeclareGraphicsRule{*}{mps}{*}{}
\usepackage{../linalgjh}
\usepackage{present}
\usepackage{xr}\externaldocument{../map4} % read refs from .aux file
\usepackage{catchfilebetweentags}
\usepackage{etoolbox} % from http://tex.stackexchange.com/questions/40699/input-only-part-of-a-file-using-catchfilebetweentags-package
\makeatletter
\patchcmd{\CatchFBT@Fin@l}{\endlinechar\m@ne}{}
  {}{\typeout{Unsuccessful patch!}}
\makeatother

\mode<presentation>
{
  \usetheme{boxes}
  \setbeamercovered{invisible}
  \setbeamertemplate{navigation symbols}{} 
}
\addheadbox{filler}{\ }  % create extra space at top of slide 
\hypersetup{colorlinks=true,linkcolor=blue} 

\title[Matrix Operations] % (optional, use only with long paper titles)
{Three.IV Matrix Operations}

\author{\textit{Linear Algebra} \\ {\small Jim Hef{}feron}}
\institute{
  \texttt{http://joshua.smcvt.edu/linearalgebra}
}
\date{}


\subject{Matrix Operations}
% This is only inserted into the PDF information catalog. Can be left
% out. 

\begin{document}
\begin{frame}
  \titlepage
\end{frame}

% =============================================
% \begin{frame}{Reduced Echelon Form} 
% \end{frame}



% ..... Three.IV.1 .....
\section{Sums and Scalar Products}
%..........
\begin{frame}{Representing operations on linear functions}
Recall that for two functions with the same domain and codomain 
$\map{g,h}{V}{W}$ the sum $g+h$ is defined by 
$(g+h)\,(\vec{v}_1+\vec{v}_2)=g(\vec{v}_1+\vec{v}_2)+h(\vec{v}_1+\vec{v}_2)$.
The scalar multiple of a function $r\cdot f$ is also defined in the natural
way by $r\cdot g\,(\vec{v})=r\cdot(g(\vec{v}))$.
We want to see how the function representations\Dash the matrices\Dash
combine.

\pause
\ex
Suppose that $\map{f}{V}{W}$ is linear maps represented with respect to 
some bases~$B$ by this matrix.
\begin{equation*}
  F=
  \rep{f}{B,D}
  \begin{mat}
    2  &1  \\
    3  &4  
  \end{mat}
\end{equation*}
\pause
This is the difference between the actions of the functions
$f$ and $5f$.
\begin{equation*}
  \vec{v}\mapsunder{f}\vec{w}
  \qquad
  \vec{v}\mapsunder{5f}5\cdot\vec{w}
\end{equation*}
We will find the matrix representing the function $5f$.
That is, we will show how to produce the matrix representation of $5f$
from the matrix representation~F of~$f$. 
\end{frame}
\begin{frame}
Consider the representatives.
\begin{equation*}
  \rep{\vec{v}}{B}=\colvec{v_1 \\ v_2}
  \quad
  \rep{\vec{w}}{D}=\colvec{w_1 \\ w_2}=\colvec{2v_1+v_2 \\ 3v_1+4v_2} 
\end{equation*}
\pause
Since $5\vec{w}=5\cdot(w_1\vec{\delta}_1+w_2\vec{\delta}_2)
=5w_1\vec{\delta}_1+5w_2\vec{\delta}_2$,
application of the function $\map{5f}{V}{W}$
transforms the representation to this.
\begin{equation*}
  \rep{5f(\vec{v})}{D}
  =\rep{5\vec{w}}{D}=\colvec{5w_1 \\ 5w_2} 
\end{equation*}
In short, it multiplies each entry by $5$.
\end{frame}
\begin{frame}
We want $\rep{5f}{B,D}$, the matrix that produces this transformation.
So we want the entries $a,b,c,d\in\Re$ that make this 
true for all $v_1,v_2\in\Re$.
\begin{equation*}
  \begin{mat}
    2  &1 \\
    3  &4
  \end{mat}
  \colvec{v_1 \\ v_2}
  =\colvec{2v_1+v_2 \\ 3v_1+4v_2}
  \qquad
  \begin{mat}
    a  &b \\
    c  &c
  \end{mat}
  \colvec{v_1 \\ v_2}
  =\colvec{5(2v_1+v_2) \\ 5(3v_1+4v_2)}
\end{equation*}
\pause
If it holds for all pairs $v_1,v_2$ then it holds for the particular
pair $v_1=1$ and $v_2=0$,
which gives the conclusion that $a=10$ and $c=15$.
It also holds for $v_1=0$ and $v_2=1$, which gives
$b=15$ and $c=20$.

\pause
Therefore, to represent $5f$
\begin{equation*}
  \rep{5f}{B,D}
  =
  \begin{mat}
    10 &5 \\
    15 &20
  \end{mat}
\end{equation*}
we multiply all the entries in~$F$ by $5$.
\end{frame}


\begin{frame}
\ex
Suppose that $\map{f,g}{V}{W}$ are linear maps represented with respect to 
some bases by these matrices.
\begin{equation*}
  F=
  \rep{f}{B,D}=
  \begin{mat}
    2  &1  \\
    3  &4  
  \end{mat}
  \qquad
  G=
  \rep{g}{B,D}=
  \begin{mat}
    5  &8  \\
    7  &6  
  \end{mat}
\end{equation*}
We want the matrix representing the function $f+g$.

\pause
If $f(\vec{v})=\vec{u}$ and $g(\vec{v})=\vec{w}$ then
$f+g$ does this.
\begin{align*}
  v_1\vec{\beta}_1+v_2\vec{\beta}_2
  &\mapsunder{f+g}
  u_1\vec{\delta}_1+u_2\vec{\delta}_2
  +
  w_1\vec{\delta}_1+w_2\vec{\delta}_2   \\
  &\qquad=(u_1+w_1)\vec{\delta}_1+(u_2+w_2)\vec{\delta}_2
\end{align*}
That is, the effect of the addition of the functions on the representations
\begin{equation*}
  \rep{f(\vec{v})}{D}=\colvec{u_1 \\ u_2}
  \quad
  \rep{g(\vec{v})}{D}=\colvec{w_1 \\ w_2}
\end{equation*}
is to add the column vectors.
\begin{equation*}
  \rep{(f+g)\,(\vec{v})}{D}=\colvec{u_1+w_1 \\ u_2+w_2}
\end{equation*}
\end{frame}
\begin{frame}
So we have this action on the representatives.
\begin{align*}
  \rep{f(\vec{v})}{D}=
  \begin{mat}
    2  &1  \\
    3  &4  
  \end{mat}
  \colvec{v_1 \\ v_2}
  =\colvec{2v_1+v_2 \\ 3v_1+4v_2}
  \\
  \rep{g(\vec{v})}{D}=
  \begin{mat}
    5  &8  \\
    7  &6  
  \end{mat}
  \colvec{v_1 \\ v_2}
  =\colvec{5v_1+8v_2 \\ 7v_1+6v_2}
\end{align*}
and we want the matrix with this effect.
\begin{equation*}
  \rep{(f+g)\,(\vec{v})}{D}=
  \begin{mat}
    a  &b  \\
    c  &d  
  \end{mat}
  \colvec{v_1 \\ v_2}
  =\colvec{7v_1+9v_2 \\ 10v_1+10v_2}
\end{equation*}
\pause
Since that equation must hold for all pairs $v_1,v_2$, we can specialize it
to $v_1=1$, $v_2=0$ to get $a=7$ and $c=10$.
Similarly, setting $v_1=0$, $v_2=1$ gives
$b=9$, $d=10$.

\pause
Therefore,
in this example at least,
the representation of $f+g$ is the entry-by-entry sum of the
representation of $f$ and the representation of~$g$.  
\end{frame}

\begin{frame}{Definition of matrix sum and scalar multiple}
\df[def:SumScalarProdMats]
\ExecuteMetaData[../map4.tex]{df:SumScalarProdMats}

\pause
\ex
Where
\begin{equation*}
  A=
  \begin{mat}
    1  &-1 \\
    2  &3
  \end{mat}
  \quad
  B=
  \begin{mat}
    0  &0     &2  \\
    9  &-1/2  &5
  \end{mat}
  \quad
  C=
  \begin{mat}
    1  &0 \\
    8  &-1
  \end{mat}
\end{equation*}
Then
\begin{equation*}
  A+C=
  \begin{mat}
    2  &-1  \\
    10 &2
  \end{mat}
  \qquad
  5B=
  \begin{mat}
    0  &0    &10 \\
    45 &-5/2 &25 
  \end{mat}
\end{equation*}
Because the sizes don't match, none of these is defined: $A+B$, $B+A$, 
$B+C$, $C+B$.

\end{frame}




%..........
\begin{frame}
\th[th:MatOpsRepMapOps]
\ExecuteMetaData[../map4.tex]{th:MatOpsRepMapOps}

\pause
\pf
The proof just generalizes the two examples that started this section.
See \nearbyexercise{exer:CorrspMapMatOps}.
\qed

\end{frame}




%..........
\begin{frame}
\df[df:ZeroMatrix]
\ExecuteMetaData[../map4.tex]{df:ZeroMatrix}
\end{frame}



% ..... Three.IV.2 .....
\section{Matrix Multiplication}
%..........
\begin{frame}
\lm[lm:CompositionOfLinearMapsIsLinear]
\ExecuteMetaData[../map4.tex]{lm:CompositionOfLinearMapsIsLinear}

\pause
\pf
\ExecuteMetaData[../map4.tex]{pf:CompositionOfLinearMapsIsLinear}
\qed
\end{frame}




%..........
\begin{frame}{Definition of matrix multiplication}
\df[def:MatMult]
\ExecuteMetaData[../map4.tex]{df:MatMult}
\end{frame}




%..........
\begin{frame}{Matrix multiplication represents composition}
\th[th:MatMultRepComp]
\ExecuteMetaData[../map4.tex]{th:MatMultRepComp}
\pause
\pf
\ExecuteMetaData[../map4.tex]{th:MatMultRepComp}
\qed
\end{frame}




%..........
\begin{frame}{Arrow diagrams}
% \ExecuteMetaData[../map4.tex]{MatMultArrowDiag0}
This pictures the relationship between maps and matrices.
\centergraphic{../ch3.20}
\pause
\ExecuteMetaData[../map4.tex]{MatMultArrowDiag1}
\end{frame}




%..........
\begin{frame}
\th[th:MatMultWellBehaved]
\ExecuteMetaData[../map4.tex]{th:MatMultWellBehaved}
\pause
\pf
\ExecuteMetaData[../map4.tex]{pf:MatMultWellBehaved0}

\pause
\ExecuteMetaData[../map4.tex]{pf:MatMultWellBehaved1}
\qed
\end{frame}




%..........
\begin{frame}
\df[df:ZeroMatrix]
\ExecuteMetaData[../map4.tex]{df:ZeroMatrix}
\end{frame}



% ..... Three.IV.2 .....
\section{Mechanics of Matrix Multiplication}
%..........
\begin{frame}
\df[df:UnitMatrix]
\ExecuteMetaData[../map4.tex]{df:UnitMatrix}
\end{frame}




%..........
\begin{frame}
\lm[lm:ColsAndRowsInMatrixMult]
\ExecuteMetaData[../map4.tex]{lm:ColsAndRowsInMatrixMult}
\end{frame}
\begin{frame}
\pf
\ExecuteMetaData[../map4.tex]{pf:ColsAndRowsInMatrixMult}
\qed
\end{frame}




%..........
\begin{frame}
\df[df:MainDiagonal]
\ExecuteMetaData[../map4.tex]{df:MainDiagonal}
\pause
\df[df:IdentityMatrix]
\ExecuteMetaData[../map4.tex]{df:IdentityMatrix}
\end{frame}




%..........
\begin{frame}
\df[df:DiagonalMatrix]
\ExecuteMetaData[../map4.tex]{df:DiagonalMatrix}
\end{frame}




%..........
\begin{frame}
\df[df:PermutationMatrix]
\ExecuteMetaData[../map4.tex]{df:PermutationMatrix}
\end{frame}




%..........
\begin{frame}
\df[df:ElementaryReductionMatrices]
\ExecuteMetaData[../map4.tex]{df:ElementaryReductionMatrices}
\end{frame}




\begin{frame}
\lm[GrByMatMult]
\ExecuteMetaData[../map4.tex]{lm:GrByMatMult}
\pf
Clear.
\qed
\co[cor:ReducViaMatrices]
\ExecuteMetaData[../map4.tex]{co:ReducViaMatrices}
\end{frame}



% ..... Three.IV.3 .....
\section{Inverses}
%..........
\begin{frame}{Definition of matrix inverse}
\df[df:MatrixInverse]
\ExecuteMetaData[../map4.tex]{df:MatrixInverse}
\end{frame}




%..........
\begin{frame}
\lm[le:LeftAndRightInvEqual]
\ExecuteMetaData[../map4.tex]{lm:LeftAndRightInvEqual}
\pause
\th[th:MatrixInvertibleIffNonsingular]
\ExecuteMetaData[../map4.tex]{th:MatrixInvertibleIffNonsingular}
\pause
\pf
\ExecuteMetaData[../map4.tex]{pf:MatrixInvertibleIffNonsingular}
\qed
\end{frame}




%..........
\begin{frame}
\lm[lem:ProdInvIsInv]
\ExecuteMetaData[../map4.tex]{lm:ProdInvIsInv}
\pause
\pf
\ExecuteMetaData[../map4.tex]{pf:ProdInvIsInv0}

\pause
\ExecuteMetaData[../map4.tex]{pf:ProdInvIsInv1}
\qed

\pause
Here is the arrow diagram for matrix inverses.
\centergraphic{../ch3.21}
\end{frame}




\begin{frame}
\lm[lem:ComputeInvMat]
\ExecuteMetaData[../map4.tex]{lm:ComputeInvMat}
\pause
\pf
\ExecuteMetaData[../map4.tex]{pf:ComputeInvMat0}

\pause
\ExecuteMetaData[../map4.tex]{pf:ComputeInvMat1}

\pause
\ExecuteMetaData[../map4.tex]{pf:ComputeInvMat2}
\qed  
\end{frame}




\begin{frame}
\co[cor:TwoByTwoInv]
\ExecuteMetaData[../map4.tex]{co:TwoByTwoInv}
\pause
\pf
\ExecuteMetaData[../map4.tex]{pf:TwoByTwoInv}
\qed  
\end{frame}


%...........................
% \begin{frame}
% \ExecuteMetaData[../gr3.tex]{GaussJordanReduction}
% \df[def:RedEchForm]
% 
% \end{frame}
\end{document}
