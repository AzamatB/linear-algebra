% see: https://groups.google.com/forum/?fromgroups#!topic/comp.text.tex/s6z9Ult_zds
\makeatletter\let\ifGm@compatii\relax\makeatother 
\documentclass[10pt,t,serif,professionalfont]{beamer}
\PassOptionsToPackage{pdfpagemode=FullScreen}{hyperref}
\PassOptionsToPackage{usenames,dvipsnames}{color}
% \DeclareGraphicsRule{*}{mps}{*}{}
\usepackage{../linalgjh}
\usepackage{present}
\usepackage{xr}\externaldocument{../map4} % read refs from .aux file
\usepackage{catchfilebetweentags}
\usepackage{etoolbox} % from http://tex.stackexchange.com/questions/40699/input-only-part-of-a-file-using-catchfilebetweentags-package
\makeatletter
\patchcmd{\CatchFBT@Fin@l}{\endlinechar\m@ne}{}
  {}{\typeout{Unsuccessful patch!}}
\makeatother

\mode<presentation>
{
  \usetheme{boxes}
  \setbeamercovered{invisible}
  \setbeamertemplate{navigation symbols}{} 
}
\addheadbox{filler}{\ }  % create extra space at top of slide 
\hypersetup{colorlinks=true,linkcolor=blue} 

\title[Matrix Operations] % (optional, use only with long paper titles)
{Three.IV Matrix Operations}

\author{\textit{Linear Algebra} \\ {\small Jim Hef{}feron}}
\institute{
  \texttt{http://joshua.smcvt.edu/linearalgebra}
}
\date{}


\subject{Matrix Operations}
% This is only inserted into the PDF information catalog. Can be left
% out. 

\begin{document}
\begin{frame}
  \titlepage
\end{frame}

% =============================================
% \begin{frame}{Reduced Echelon Form} 
% \end{frame}



% ..... Three.IV.1 .....
\section{Sums and Scalar Products}
%..........
\begin{frame}{Representing operations on linear functions}
Recall the operations on linear functions of scalar multiplication
and addition defined by: for  
$\map{f,g}{V}{W}$, where $r\in\Re$
the scalar multiple $r\cdot f$ function does this
\begin{equation*}
  \vec{v}\mapsunder{r\cdot f} r\cdot (f(\vec{v}))
\end{equation*}
and the sum of the two functions does this.
\begin{equation*}
  \vec{v}\mapsunder{f+g} f(\vec{v})+g(\vec{v})
\end{equation*}
We will see how the matrix representations $\rep{f}{B,D}$ and 
$\rep{g}{B,D}$
combine to give $\rep{r\cdot f}{B,D}$
and $\rep{f+g}{B,D}$.

\pause
\ex
Suppose that $\map{f}{V}{W}$ is the linear map represented with respect to 
some bases~$B,D$ by this matrix.
\begin{equation*}
  F=
  \rep{f}{B,D}=
  \begin{mat}
    2  &1  \\
    3  &4  
  \end{mat}
\end{equation*}
We will find the matrix representing the function $6f$.
That is, we will show how to produce the matrix representation of $6f$
from the matrix representation~F of~$f$. 
\end{frame}
\begin{frame}
Consider representations of vectors from the domain and codomain.
\begin{equation*}
  \rep{\vec{v}}{B}=\colvec{v_1 \\ v_2}
  \quad
  \rep{\vec{w}}{D}=\colvec{w_1 \\ w_2}
  % =\colvec{2v_1+v_2 \\ 3v_1+4v_2} 
\end{equation*}
\pause
Since $6\vec{w}=6\cdot(w_1\vec{\delta}_1+w_2\vec{\delta}_2)
=6w_1\vec{\delta}_1+6w_2\vec{\delta}_2$,
application of the function $\map{6f}{V}{W}$
yields this representation of the codomain vector.
\begin{equation*}
  \rep{6f(\vec{v})}{D}
  =\rep{6\vec{w}}{D}=\colvec{6w_1 \\ 6w_2} 
\end{equation*}
So moving from the function $f$ to the function~$6f$
has the effect on the representations of vectors that it
transforms the codomain vector by multipling each entry by $6$.
\end{frame}
\begin{frame}
In this example the action of the function is represented as here.  
\begin{equation*}\
  \rep{f}{B,D}\cdot \rep{\vec{v}}{B}=
  \begin{mat}
    2  &1 \\
    3  &4
  \end{mat}
  \colvec{v_1 \\ v_2}
  =\colvec{2v_1+v_2 \\ 3v_1+4v_2}
\end{equation*}
\pause
We want the matrix 
that makes this 
true.
\begin{equation*}
  \rep{6f}{B,D}\cdot
  \colvec{v_1 \\ v_2}
  =\colvec{6(2v_1+v_2) \\ 6(3v_1+4v_2)}
  =\colvec{12v_1+6v_2 \\  18v_1+24v_2}
\end{equation*}
Therefore, at least in this example
\begin{equation*}
  \rep{4f}{B,D}
  =
  \begin{mat}
    12 &6 \\
    18 &24
  \end{mat}
\end{equation*}
so going from the function $f$ to the 
function~$6f$ has the effect on the representation~$F$ of  
multiplying all the entries by $6$.
\end{frame}


\begin{frame}
\ex
Suppose that $\map{f,g}{V}{W}$ are linear maps represented with respect to 
some bases by these matrices.
\begin{equation*}
  F=
  \rep{f}{B,D}=
  \begin{mat}
    2  &1  \\
    3  &4  
  \end{mat}
  \qquad
  G=
  \rep{g}{B,D}=
  \begin{mat}
    5  &8  \\
    7  &6  
  \end{mat}
\end{equation*}
We want the matrix representing the function $f+g$.

\pause
If $f(\vec{v})=\vec{u}$ and $g(\vec{v})=\vec{w}$ then
$f+g$ does this.
\begin{align*}
  \vec{v}
  &\mapsunder{f+g}
  \vec{u}+\vec{v}                      \\
  &\qquad =u_1\vec{\delta}_1+u_2\vec{\delta}_2
  +
  w_1\vec{\delta}_1+w_2\vec{\delta}_2   \\
  &\qquad=(u_1+w_1)\vec{\delta}_1+(u_2+w_2)\vec{\delta}_2
\end{align*}
Thus for any $\vec{v}$, the effect on the representatives 
\begin{equation*}
  \rep{f(\vec{v})}{D}=\colvec{u_1 \\ u_2}
  \quad
  \rep{g(\vec{v})}{D}=\colvec{w_1 \\ w_2}
\end{equation*}
of adding the functions is to add the column vectors.
\begin{equation*}
  \rep{(f+g)\,(\vec{v})}{D}=\colvec{u_1+w_1 \\ u_2+w_2}
\end{equation*}
\end{frame}
\begin{frame}
The particular functions in this example have 
this action on the representations.
\begin{align*}
  \rep{f(\vec{v})}{D}=
  \begin{mat}
    2  &1  \\
    3  &4  
  \end{mat}
  \colvec{v_1 \\ v_2}
  =\colvec{2v_1+v_2 \\ 3v_1+4v_2}
  \\
  \rep{g(\vec{v})}{D}=
  \begin{mat}
    5  &8  \\
    7  &6  
  \end{mat}
  \colvec{v_1 \\ v_2}
  =\colvec{5v_1+8v_2 \\ 7v_1+6v_2}
\end{align*}
So we want the matrix with this effect.
\begin{equation*}
  \rep{(f+g)\,(\vec{v})}{D}
  \colvec{v_1 \\ v_2}
  =\colvec{7v_1+9v_2 \\ 10v_1+10v_2}
\end{equation*}
\pause
Thus
\begin{equation*}
  \rep{f+g}{B,D}=
  \begin{mat}
    7  &9  \\
    10  &10  
  \end{mat}
\end{equation*}
\pause
and at least in this example
the representation of the sum function is the entry-by-entry sum of the
representations of the two functions.  
\end{frame}

\begin{frame}{Definition of matrix sum and scalar multiple}
\df[def:SumScalarProdMats]
\ExecuteMetaData[../map4.tex]{df:SumScalarProdMats}

\pause
\ex
Where
\begin{equation*}
  A=
  \begin{mat}
    1  &-1 \\
    2  &3
  \end{mat}
  \quad
  B=
  \begin{mat}
    0  &0     &2  \\
    9  &-1/2  &5
  \end{mat}
  \quad
  C=
  \begin{mat}
    1  &0 \\
    8  &-1
  \end{mat}
\end{equation*}
Then
\begin{equation*}
  A+C=
  \begin{mat}
    2  &-1  \\
    10 &2
  \end{mat}
  \qquad
  5B=
  \begin{mat}
    0  &0    &10 \\
    45 &-5/2 &25 
  \end{mat}
\end{equation*}
Because the sizes don't match, none of these is defined: $A+B$, $B+A$, 
$B+C$, $C+B$.

\end{frame}




%..........
\begin{frame}
\th[th:MatOpsRepMapOps]
\ExecuteMetaData[../map4.tex]{th:MatOpsRepMapOps}

\pause
\pf
We can generalize the two examples that started this section.
See \nearbyexercise{exer:CorrspMapMatOps}.
\qed

\pause
\medskip
\df[df:ZeroMatrix]
\ExecuteMetaData[../map4.tex]{df:ZeroMatrix}

\pause
\medskip
\ex
The zero matrix is the identity element for matrix addition.
\begin{equation*}
  \begin{mat}
    3 &1 &2 \\
    5 &0 &9
  \end{mat}
  +
  \begin{mat}
    0 &0 &0 \\
    0 &0 &0
  \end{mat}
  =
  \begin{mat}
    3 &1 &2 \\
    5 &0 &9
  \end{mat}
\end{equation*}
This reflects the fact that a zero matrix represents a zero function
$\map{Z}{V}{W}$, which is the identity elements for
function addition. 
\end{frame}




% ..... Three.IV.2 .....
\section{Matrix Multiplication}
%..........
\begin{frame}
Besides scalar multiplication and addition, another operation that 
we can perform on functions is composition.

\pause
\lm[lm:CompositionOfLinearMapsIsLinear]
\ExecuteMetaData[../map4.tex]{lm:CompositionOfLinearMapsIsLinear}

\pause
\pf
\ExecuteMetaData[../map4.tex]{pf:CompositionOfLinearMapsIsLinear}
\qed

\pause
\medskip
We will see how the matrix
representations of the two functions combine to make
the matrix representation of their composition.
\end{frame}




%..........
\begin{frame}
\ex
Consider two linear functions $\map{h}{V}{W}$ and $\map{g}{W}{X}$
represented as here.
\begin{equation*}
  \rep{h}{B,C}=
  \begin{mat}
    3 &1 \\
    2 &5 \\
    4 &6
  \end{mat}
  \qquad
  \rep{g}{C,D}=
  \begin{mat}
    8 &7 &11 \\
    9 &10 &12 
  \end{mat}
\end{equation*}
(The sizes of the matrices show that 
$V$ has dimension~$2$, 
$W$ has dimension~$3$, 
and $X$ has dimension~$2$.)

We want to see how these two matrices combine to represent the map
$\map{\composed{g}{h}}{V}{X}$.

\pause
We will compute the action of $\composed{g}{h}$ by first computing the action
of~$h$ on a vector~$\vec{v}\in V$.
\begin{equation*}
  \rep{h(\vec{v})}{C}
  =\rep{h}{B,C}\cdot\rep{\vec{v}}{B}
  =
  \begin{mat}
    3 &1 \\
    2 &5 \\
    4 &6
  \end{mat}
  \colvec{v_1 \\ v_2}  
  =
  \colvec{3v_1+v_2 \\ 2v_1+5v_2 \\ 4v_1+6v_2}
\end{equation*}
\end{frame}
\begin{frame}
Apply $g$ to that.
\begin{multline*}
  \rep{g}{C,D}\cdot \rep{h(\vec{v})}{C}
  =
  \begin{mat}
    8 &7 &11 \\
    9 &10 &12     
  \end{mat}
  \colvec{3v_1+v_2 \\ 2v_1+5v_2 \\ 4v_1+6v_2}                \\
  =
  \colvec{8(3v_1+v_2)+7(2v_1+5v_2)+11(4v_1+6v_2)  \\
          9(3v_1+v_2)+10(2v_1+5v_2)+12(4v_1+6v_2)}
\end{multline*}
Gather terms.
\begin{equation*}
  =\colvec{(8\cdot 3+7\cdot 2+11\cdot 4)v_1+(8\cdot 1+7\cdot 5+11\cdot 6)v_2 \\
           (9\cdot 3+10\cdot 2+12\cdot 4)v_1+(9\cdot 1+10\cdot 5+12\cdot 6)v_2}
\end{equation*}
Rewrite that as a matrix-vector multiplication.
\begin{equation*}
  \begin{mat}
    8\cdot 3+7\cdot 2+11\cdot 4 &8\cdot 1+7\cdot 5+11\cdot 6 \\
    9\cdot 3+10\cdot 2+12\cdot 4 &9\cdot 1+10\cdot 5+12\cdot 6
  \end{mat}
  \colvec{v_1 \\ v_2}
\end{equation*}
Here is the combination.
\begin{equation*}
  \begin{mat}
    8 &7 &11 \\
    9 &10 &12 
  \end{mat}
  \begin{mat}
    3 &1 \\
    2 &5 \\
    4 &6
  \end{mat}
  =
  \begin{mat}
    8\cdot 3+7\cdot 2+11\cdot 4 &8\cdot 1+7\cdot 5+11\cdot 6 \\
    9\cdot 3+10\cdot 2+12\cdot 4 &9\cdot 1+10\cdot 5+12\cdot 6
  \end{mat}
\end{equation*}
\end{frame}




%..........
\begin{frame}{Definition of matrix multiplication}
\df[def:MatMult]
\ExecuteMetaData[../map4.tex]{df:MatMult}
\ex
\begin{equation*}
  \begin{mat}
    3 &1 &6 \\
    2 &5 &9
  \end{mat}
  \begin{mat}
    2 &0  &4 \\
    1 &-3 &5 \\
    4 &2  &7
  \end{mat}
  =
  \begin{mat}
    31  &9  &59  \\
    45  &3  &96
  \end{mat}
\end{equation*}
\end{frame}
\begin{frame}
\ex
This product is not defined.
\begin{equation*}
  \begin{mat}
    1  &3  &-1 \\
    0  &0  &0  \\
    2  &0  &0
  \end{mat}
  \begin{mat}
    5  &7  &1 \\
    2  &2  &0 
  \end{mat}
\end{equation*}

\pause
\ex
Square matrices of the same size have a defined product.
\begin{equation*}
  \begin{mat}
    1  &3  &-1 \\
    0  &0  &0  \\
    2  &0  &0
  \end{mat}
  \begin{mat}
    5  &7  &1 \\
    2  &2  &0 \\
    1  &-1 &2 
  \end{mat}
  =
  \begin{mat}
    11  &14  &-5  \\
     0  &0   &0   \\
    10  &14  &2
  \end{mat}
\end{equation*}
This reflects the fact that we can compose two
functions from a space to itself $\map{f,g}{V}{V}$.
\end{frame}




%..........
\begin{frame}{Matrix multiplication represents composition}
\th[th:MatMultRepComp]
\ExecuteMetaData[../map4.tex]{th:MatMultRepComp}
\pause
\pf
\ExecuteMetaData[../map4.tex]{pf:MatMultRepComp0}
\end{frame}
\begin{frame}
\ExecuteMetaData[../map4.tex]{pf:MatMultRepComp1}
\qed
\end{frame}




%..........
\begin{frame}{Arrow diagrams}
% \ExecuteMetaData[../map4.tex]{MatMultArrowDiag0}
This pictures the relationship between maps and matrices.
\centergraphic{../ch3.20}
\pause
\ExecuteMetaData[../map4.tex]{MatMultArrowDiag1}
\end{frame}



\begin{frame}{Order, dimensions, and sizes}
Consider the composition of two function.

First consider the order in which we write the composition.
In $\composed{g}{h}$ the function written first, $g$, is the 
function applied second.
\begin{equation*}
  \vec{v}\mapsunder{h} h(\vec{v}) \mapsunder{g} g( h(\vec{v}))
\end{equation*}
We write $g$ first to match the definition 
$\composed{g}{h}(\vec{v})=g(\,h(\vec{v})\,)$.
(Some people try to lessen confusion by reading 
`$\composed{g}{h}$' aloud as ``$g$ following $h$.'')
That order carries over to matrices:~$\composed{g}{h}$
is represented by $GH$.

\pause
Now consider the dimensions of the domain and codomain spaces.
\begin{equation*}
  \text{dimension \( n \) space}
  \;\stackrel{h}{\longrightarrow}\;
  \text{dimension \( r \) space}
  \;\stackrel{g}{\longrightarrow}\;
  \text{dimension \( m \) space}
\end{equation*}
The representing $\nbym{m}{n}$ matrix~$GH$ is the product of
an $\nbym{m}{r}$ matrix~$G$
and a $\nbym{r}{n}$ matrix~$H$.
Briefly,
`$\nbym{m}{r}\text{\ times\ }\nbym{r}{n}\text{\ equals\ }\nbym{m}{n}$'.
\end{frame}



\begin{frame}{Matrix multiplication is not commutative}
Function composition is not a commutative operation\Dash $\cos(x^2)$
is different than $\cos^2(x)$.
This also holds in the special case of 
composition of linear functions.
\pause
\ex
Changing the order in which we multiply these matrices
\begin{equation*}
  \begin{mat}
    3  &3  \\
    0  &4
  \end{mat}
  \begin{mat}
    -2  &6  \\
    6   &5
  \end{mat}
  =
  \begin{mat}
    12  &33 \\
    24  &20
  \end{mat}
\end{equation*}
changes the result.
\begin{equation*}
  \begin{mat}
    -2  &6  \\
    6   &5
  \end{mat}
  \begin{mat}
    3  &3  \\
    0  &4
  \end{mat}
  =
  \begin{mat}
     -6  &18   \\
     18  &38  
  \end{mat}
\end{equation*}
\pause
\ex
The product of these matrices
\begin{equation*}
  \begin{mat}
    3  &4  \\
    0  &2
  \end{mat}
  \qquad
  \begin{mat}
    8  &12  &0 \\
   -4  &0  &1/2
  \end{mat}
\end{equation*}
is defined in one order and not defined in the other.
\end{frame}




%..........
\begin{frame}
\th[th:MatMultWellBehaved]
\ExecuteMetaData[../map4.tex]{th:MatMultWellBehaved}
\pause
\pf
\ExecuteMetaData[../map4.tex]{pf:MatMultWellBehaved0}

\pause
\ExecuteMetaData[../map4.tex]{pf:MatMultWellBehaved1}
\qed
\end{frame}




% ..... Three.IV.3 .....
\section{Mechanics of Matrix Multiplication}
%..........
\begin{frame}{Combinatorics of multiplication}
The striking thing about matrix multiplication is the
way rows and columns combine.
Here a second row and a third column combine to make a $2,3$~entry.
\begin{equation*}
\setlength{\fboxsep}{1.5pt}
    \begin{mat}
       \begin{array}{@{}cc@{}} 1  &1 \end{array}                         \\ 
       \text{\highlight{\( \begin{array}{@{}cc@{}}  0  &1  \end{array} \)}}   \\  
       \begin{array}{@{}cc@{}} 1  &0 \end{array}
    \end{mat}
    \begin{mat}
      \begin{array}{@{}c@{}}  4  \\  5  \end{array}
      &\begin{array}{@{}c@{}}  6  \\  7  \end{array}
      &\text{\highlight{ \(\begin{array}{@{}c@{}}  8  \\  9  \end{array}\) }}
    \end{mat}
  =
    \begin{mat}[r]
      9  &13   &17                           \\
      5  &7    &\text{\highlight{ \(9\) }}   \\  
      4  &6    &8
    \end{mat}
\end{equation*}
\pause
The \( i,j \) entry of the matrix product $GH$ is the dot product of
row~$i$ of the left matrix $G$ with column~$j$
of the right one~$H$.
\begin{equation*} % \setlength{\fboxsep}{.15em}
  p_{i,j}
  =
  g_{i,\text{\highlight{$1 $}}}h_{\text{\highlight{$1 $}},j}
   +g_{i,\text{\highlight{$2 $}}}h_{\text{\highlight{$2 $}},j}
   +\dots+g_{i,\text{\highlight{$r $}}}h_{\text{\highlight{$r $}},j}
\end{equation*}
\pause
We can view this as the left matrix acting
by multiplying its rows into the columns of the right matrix.
Or we could see it as 
the right matrix using its columns to
act on the left matrix's rows.
\end{frame}




%..........
\begin{frame}
\df[df:UnitMatrix]
\ExecuteMetaData[../map4.tex]{df:UnitMatrix}
\ex
The $2,1$ unit $\nbym{2}{3}$ matrix multiplies from the left 
\begin{equation*}
  \begin{mat}
    0 &0 &0 \\
    1 &0 &0
  \end{mat}
  \begin{mat}
    2 &1 &3 \\
    5 &6 &4 \\
    7 &8 &9
  \end{mat}
  =
  \begin{mat}
    0 &0 &0 \\
    2 &1 &3 \\
    0 &0 &0
  \end{mat}
\end{equation*}
to copy row~$1$ of the multiplicand into row~$2$ of the result. 

\pause
\ex 
From the right the $2,1$ unit $\nbym{2}{3}$ matrix
\begin{equation*}
  \begin{mat}
    3 &4 \\
    6 &5
  \end{mat}
  \begin{mat}
    0 &0 &0 \\
    1 &0 &0
  \end{mat}
  =
  \begin{mat}
    4 &0 &0 \\
    5 &0 &0
  \end{mat}
\end{equation*}
copies column~$2$ of the first matrix into column~$1$ of the result.

\pause
\ex 
Rescaling the unit matrix rescales the result.
\begin{equation*}
  \begin{mat}
    3 &4 \\
    6 &5
  \end{mat}
  \begin{mat}
    0 &0 &0 \\
    3 &0 &0
  \end{mat}
  =
  \begin{mat}
    12 &0 &0 \\
    15 &0 &0
  \end{mat}
\end{equation*}
\end{frame}




%..........
\begin{frame}
\lm[lm:ColsAndRowsInMatrixMult]
\ExecuteMetaData[../map4.tex]{lm:ColsAndRowsInMatrixMult}
\end{frame}
\begin{frame}
\pf
\ExecuteMetaData[../map4.tex]{pf:ColsAndRowsInMatrixMult}
\qed
\end{frame}




%..........
\begin{frame}
\df[df:MainDiagonal]
\ExecuteMetaData[../map4.tex]{df:MainDiagonal}
\pause
\df[df:IdentityMatrix]
\ExecuteMetaData[../map4.tex]{df:IdentityMatrix}
\pause
Taking the product with an identity matrix returns the multiplicand. 
\ex
Multipication by an identity from the left
\begin{equation*}
  \begin{mat}
    1  &0 \\
    0  &1
  \end{mat}
  \begin{mat}
    3  &2  \\
   -1  &5
  \end{mat}
  =
  \begin{mat}
    3  &2  \\
   -1  &5
  \end{mat}
\end{equation*}
or from the right leaves the matrix unchanged.
\begin{equation*}
  \begin{mat}
    3  &2  \\
   -1  &5
  \end{mat}
  \begin{mat}
    1  &0 \\
    0  &1
  \end{mat}
  =
  \begin{mat}
    3  &2  \\
   -1  &5
  \end{mat}
\end{equation*}
\end{frame}




%..........
\begin{frame}
\df[df:DiagonalMatrix]
\ExecuteMetaData[../map4.tex]{df:DiagonalMatrix}
\pause
\ex
Multiplication from the left by a diagonal matrix rescales the rows.
\begin{equation*}
  \begin{mat}
    2  &0 \\
    0  &3
  \end{mat}
  \begin{mat}
    3  &2  \\
   -1  &5
  \end{mat}
  =
  \begin{mat}
    6  &4  \\
   -3  &15
  \end{mat}
\end{equation*}
From the right it rescales the columns.
\begin{equation*}
  \begin{mat}
    3  &2  \\
   -1  &5
  \end{mat}
  \begin{mat}
    2  &0 \\
    0  &3
  \end{mat}
  =
  \begin{mat}
    6  &6  \\
   -2  &15
  \end{mat}
\end{equation*}
\end{frame}




%..........
\begin{frame}
\df[df:PermutationMatrix]
\ExecuteMetaData[../map4.tex]{df:PermutationMatrix}
\pause
\ex
Multiplication by a permutation matrix from the left will swap rows.
\begin{equation*}
  \begin{mat}
    0 &1 &0 \\
    1 &0 &0 \\
    0 &0 &1
  \end{mat}
  \begin{mat}
    1 &2 &3 \\
    4 &5 &6 \\
    7 &8 &9
  \end{mat}\
  =
  \begin{mat}
    4 &5 &6 \\
    1 &2 &3 \\
    7 &8 &9
  \end{mat}
\end{equation*}
From the right it swaps columns.
\begin{equation*}
  \begin{mat}
    1 &2 &3 \\
    4 &5 &6 \\
    7 &8 &9
  \end{mat}\
  \begin{mat}
    0 &1 &0 \\
    1 &0 &0 \\
    0 &0 &1
  \end{mat}
  =
  \begin{mat}
    2 &1 &3 \\
    5 &4 &6 \\
    8 &7 &9
  \end{mat}
\end{equation*}
\end{frame}




%..........
\begin{frame}
\df[df:ElementaryReductionMatrices]
\ExecuteMetaData[../map4.tex]{df:ElementaryReductionMatrices}
\pause
\ex
Here are some $\nbyn{2}$ examples.
\begin{equation*}
  M_{2}(3)=
  \begin{mat}[r]
    1  &0 \\
    0  &3
  \end{mat},
  \quad
  P_{1,2}=
  \begin{mat}[r]
    0  &1 \\
    1  &0
  \end{mat},
  \quad
  C_{1,2}(-3)=
  \begin{mat}[r]
    1  &0 \\
    -3 &1
  \end{mat}
\end{equation*}
\pause
\ex
Some $\nbyn{3}$ examples.
\begin{equation*}
  % M_{2}(1/2)=
  % \begin{mat}
  %   1 &0   &0 \\
  %   0 &1/2 &0 \\
  %   0 &0   &1
  % \end{mat},
  % \quad
  P_{2,3}=
  \begin{mat}
    1 &0 &0 \\
    0 &0 &1 \\
    0 &1 &0
  \end{mat},
  \quad
  C_{2,3}(-4)=
  \begin{mat}[r]
    1 &0  &0 \\
    0 &1  &0 \\
    0 &-4 &0
  \end{mat}
\end{equation*}
\end{frame}




\begin{frame}
\ex Multiplying on the left by the $\nbyn{3}$ matrix
$M_{2}(1/2)$
has the effect of the row operation $(1/2)\rho_2$. 
\begin{equation*}
  \begin{mat}
    1 &0   &0 \\
    0 &1/2 &0 \\
    0 &0   &1
  \end{mat}
  \begin{mat}
    1  &3  &0  \\
    0  &2  &5  \\
    0  &0  &0
  \end{mat}
  =
  \begin{mat}
    1  &3  &0  \\
    0  &1  &5/2  \\
    0  &0  &0
  \end{mat}
\end{equation*}
\pause
\ex
Left multiplication by $C_{1,3}(-2)$ performs the
row operation $-2\rho_1+\rho_3$.
\begin{equation*}
  \begin{mat}[r]
    1 &0   &0 \\
    0 &1   &0 \\
    -2 &0   &1
  \end{mat}
  \begin{mat}[r]
    1  &3  &0  \\
    0  &2  &5  \\
    2  &1  &1
  \end{mat}
  =
  \begin{mat}[r]
    1  &3  &0  \\
    0  &2  &5  \\
    0  &-5 &1
  \end{mat}
\end{equation*}
\end{frame}




\begin{frame}
\lm[GrByMatMult]
\ExecuteMetaData[../map4.tex]{lm:GrByMatMult}
\pf
Clear.
\qed
\co[cor:ReducViaMatrices]
\ExecuteMetaData[../map4.tex]{co:ReducViaMatrices}

\pause
\ex
We can bring this augmented matrix to echelon form with matrix multiplication.
\begin{equation*}
  \begin{amat}{3}
    1  &-1  &2  &4 \\
    2  &-2  &-1 &6 \\
    0  &3   &1  &5 
  \end{amat}
\end{equation*}
\end{frame}
\begin{frame}
We first perform $-2\rho_1+\rho_2$ via left multiplication by $C_{1,2}(-2)$.
\begin{equation*}
  \begin{mat}
    1  &0  &0  \\
    0  &1  &0  \\
   -2  &0  &1
  \end{mat}
  \begin{amat}{3}
    1  &-1  &2  &4 \\
    2  &-2  &-1 &6  \\
    0  &3   &1  &5 
  \end{amat}
  =
  \begin{amat}{3}
    1  &-1  &2  &4 \\
    0  &0   &-5 &-2 \\
    0  &3   &1  &5 
  \end{amat}
\end{equation*}
\pause
Now we swap rows $2$ and $3$ with $P_{2,3}$
\begin{equation*}
  \begin{mat}
    1  &0  &0  \\
    0  &0  &1  \\
    0  &1  &0
  \end{mat}
  \begin{amat}{3}
    1  &-1  &2  &4 \\
    0  &0   &-5 &-2 \\
    0  &3   &1  &5 
  \end{amat}
  =
  \begin{amat}{3}
    1  &-1  &2  &4 \\
    0  &3   &1  &5 \\
    0  &0   &-5 &-2 
  \end{amat}
\end{equation*}
\pause
When writing them out in full, remember that the matrix used first appears
to the right of the matrix used second.
\begin{equation*}
  \begin{mat}
    1  &0  &0  \\
    0  &0  &1  \\
    0  &1  &0
  \end{mat}
  \begin{mat}
    1  &0  &0  \\
    0  &1  &0  \\
   -2  &0  &1
  \end{mat}
  \begin{amat}{3}
    1  &-1  &2  &4 \\
    2  &-2  &-1 &6  \\
    0  &3   &1  &5 
  \end{amat}
  =
  \begin{amat}{3}
    1  &-1  &2  &4 \\
    0  &3   &1  &5 \\
    0  &0   &-5 &-2 
  \end{amat}
\end{equation*}
\end{frame}



% ..... Three.IV.3 .....
\section{Inverses}
%..........
\begin{frame}{Definition of matrix inverse}
\df[df:MatrixInverse]
\ExecuteMetaData[../map4.tex]{df:MatrixInverse}
\end{frame}




%..........
\begin{frame}
\lm[le:LeftAndRightInvEqual]
\ExecuteMetaData[../map4.tex]{lm:LeftAndRightInvEqual}
\pause
\th[th:MatrixInvertibleIffNonsingular]
\ExecuteMetaData[../map4.tex]{th:MatrixInvertibleIffNonsingular}
\pause
\pf
\ExecuteMetaData[../map4.tex]{pf:MatrixInvertibleIffNonsingular}
\qed
\end{frame}




%..........
\begin{frame}
\lm[lem:ProdInvIsInv]
\ExecuteMetaData[../map4.tex]{lm:ProdInvIsInv}
\pause
\pf
\ExecuteMetaData[../map4.tex]{pf:ProdInvIsInv0}

\pause
\ExecuteMetaData[../map4.tex]{pf:ProdInvIsInv1}
\qed

\pause
Here is the arrow diagram for matrix inverses.
\centergraphic{../ch3.21}
\end{frame}




\begin{frame}
\lm[lem:ComputeInvMat]
\ExecuteMetaData[../map4.tex]{lm:ComputeInvMat}
\pause
\pf
\ExecuteMetaData[../map4.tex]{pf:ComputeInvMat0}

\pause
\ExecuteMetaData[../map4.tex]{pf:ComputeInvMat1}

\pause
\ExecuteMetaData[../map4.tex]{pf:ComputeInvMat2}
\qed  
\end{frame}




\begin{frame}
\co[cor:TwoByTwoInv]
\ExecuteMetaData[../map4.tex]{co:TwoByTwoInv}
\pause
\pf
\ExecuteMetaData[../map4.tex]{pf:TwoByTwoInv}
\qed  
\end{frame}


%...........................
% \begin{frame}
% \ExecuteMetaData[../gr3.tex]{GaussJordanReduction}
% \df[def:RedEchForm]
% 
% \end{frame}
\end{document}
