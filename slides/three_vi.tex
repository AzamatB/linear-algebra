% see: https://groups.google.com/forum/?fromgroups#!topic/comp.text.tex/s6z9Ult_zds
\makeatletter\let\ifGm@compatii\relax\makeatother 
\documentclass[10pt,t,serif,professionalfont]{beamer}
\PassOptionsToPackage{pdfpagemode=FullScreen}{hyperref}
\PassOptionsToPackage{usenames,dvipsnames}{color}
% \DeclareGraphicsRule{*}{mps}{*}{}
\usepackage{../linalgjh}
\usepackage{present}
\usepackage{xr}\externaldocument{../map6} % read refs from .aux file
\usepackage{catchfilebetweentags}
\usepackage{etoolbox} % from http://tex.stackexchange.com/questions/40699/input-only-part-of-a-file-using-catchfilebetweentags-package
\makeatletter
\patchcmd{\CatchFBT@Fin@l}{\endlinechar\m@ne}{}
  {}{\typeout{Unsuccessful patch!}}
\makeatother

\mode<presentation>
{
  \usetheme{boxes}
  \setbeamercovered{invisible}
  \setbeamertemplate{navigation symbols}{} 
}
\addheadbox{filler}{\ }  % create extra space at top of slide 
\hypersetup{colorlinks=true,linkcolor=blue} 

\title[Projection] % (optional, use only with long paper titles)
{Three.VI Projection}

\author{\textit{Linear Algebra} \\ {\small Jim Hef{}feron}}
\institute{
  \texttt{http://joshua.smcvt.edu/linearalgebra}
}
\date{}


\subject{Projection}
% This is only inserted into the PDF information catalog. Can be left
% out. 

\begin{document}
\begin{frame}
  \titlepage
\end{frame}

% =============================================
% \begin{frame}{Reduced Echelon Form} 
% \end{frame}



% ..... Three.VI.1 .....
\section{Orthogonal Projection Into a Line}
%..........
\begin{frame}
\ExecuteMetaData[../map6.tex]{PictureOrthogonalProjectionIntoALine0}
\begin{center}
  \includegraphics{../ch3.28}
  \hspace*{0.6in}
  \includegraphics{../ch3.29}      
\end{center}
\pause
\ExecuteMetaData[../map6.tex]{PictureOrthogonalProjectionIntoALine1}

\pause
\ExecuteMetaData[../map6.tex]{PictureOrthogonalProjectionIntoALine2}
\end{frame}
\begin{frame}
\df[df:ProjIntoLine]
\ExecuteMetaData[../map6.tex]{df:ProjIntoLine}
\end{frame}



% ..... Three.VI.2 .....
\section{Gram-Schmidt Orthogonalization}
%..........
\begin{frame}{Mutually orthogonal vectors}
The prior subsection suggests that
projecting a vector $\vec{v}$ into the line spanned by \( \vec{s} \)
decomposes $\vec{v}$ into two parts
\begin{center}  \small
  \vcenteredhbox{\includegraphics{../ch3.35}}
   \qquad
   $\displaystyle \vec{v}=\proj{\vec{v}}{\spanof{\vec{s}\,}}
             \,+\,\left(\vec{v}-\proj{\vec{v}}{\spanof{\vec{s}\,}}\right)$
\end{center}
that are orthogonal and so are not-interacting.
We will now develop that suggestion.
\pause
\df[df:MutuallyOrthogonal]
\ExecuteMetaData[../map6.tex]{df:MutuallyOrthogonal}
\end{frame}




\begin{frame}
\th[th:OrthoIsInd]
\ExecuteMetaData[../map6.tex]{th:OrthoIsInd}
\pause
\pf
\ExecuteMetaData[../map6.tex]{pf:OrthoIsInd}
\qed
\end{frame}




\begin{frame}
\co[cor:OrthAndBigEnoughIsBasis]
\ExecuteMetaData[../map6.tex]{co:OrthAndBigEnoughIsBasis}

\pause
\pf
\ExecuteMetaData[../map6.tex]{pf:OrthAndBigEnoughIsBasis}
\qed
\end{frame}




\begin{frame}
\df[df:OrthogonalBasis]
\ExecuteMetaData[../map6.tex]{df:OrthogonalBasis}
\end{frame}




\begin{frame}
\th[th:GramSchmidt]
\ExecuteMetaData[../map6.tex]{th:GramSchmidt}

\pause
\pf
\ExecuteMetaData[../map6.tex]{pf:GramSchmidt0}

\pause
\ExecuteMetaData[../map6.tex]{pf:GramSchmidt1}
\end{frame}
\begin{frame}
\ExecuteMetaData[../map6.tex]{pf:GramSchmidt2}

\pause
\ExecuteMetaData[../map6.tex]{pf:GramSchmidt3}
\end{frame}
\begin{frame}
\ExecuteMetaData[../map6.tex]{pf:GramSchmidt4}
\end{frame}
\begin{frame}
\ExecuteMetaData[../map6.tex]{pf:GramSchmidt5}
\qed
\end{frame}





% % ..... Three.VI.3 .....
% \section{Projection Into a Subspace}
% %..........
% \begin{frame}
% \end{frame}




%...........................
% \begin{frame}
% \ExecuteMetaData[../gr3.tex]{GaussJordanReduction}
% \df[def:RedEchForm]
% 
% \end{frame}
\end{document}
