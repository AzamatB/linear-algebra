% see: https://groups.google.com/forum/?fromgroups#!topic/comp.text.tex/s6z9Ult_zds
\makeatletter\let\ifGm@compote\relax\makeatother 
\documentclass[10pt,t]{beamer}
\usefonttheme{professionalfonts}
\usefonttheme{serif}
\PassOptionsToPackage{pdfpagemode=FullScreen}{hyperref}
\PassOptionsToPackage{usenames,dvipsnames}{color}
% \DeclareGraphicsRule{*}{mps}{*}{}
\usepackage{../linalgjh}
\usepackage{present}
\usepackage{xr}\externaldocument{../jc2} % read refs from .aux file
\usepackage{catchfilebetweentags}
\usepackage{etoolbox} % from http://tex.stackexchange.com/questions/40699/input-only-part-of-a-file-using-catchfilebetweentags-package
\makeatletter
\patchcmd{\CatchFBT@Fin@l}{\endlinechar\m@ne}{}
  {}{\typeout{Unsuccessful patch!}}
\makeatother

\usepackage{polynom}  % for polynomial long division

\mode<presentation>
{
  \usetheme{boxes}
  \setbeamercovered{invisible}
  \setbeamertemplate{navigation symbols}{} 
}
\addheadbox{filler}{\ }  % create extra space at top of slide 
\hypersetup{colorlinks=true,linkcolor=blue} 

\title[Angles and Eigenvectors] % (optional, use only with long paper titles)
{Angles and Eigenvectors}

\author{\textit{Linear Algebra} \\ {\small Jim Hef{}feron}}
\institute{
  \texttt{http://joshua.smcvt.edu/linearalgebra}
}
\date{}


\subject{Eigenvectors}
% This is only inserted into the PDF information catalog. Can be left
% out. 

\begin{document}
\begin{frame}
  \titlepage
\end{frame}


\section{Lines transform to lines}
%..........
\begin{frame}{Lines go to lines}
Consider a real space transformation
$\map{t}{\Re^n}{\Re^n}$.
A defining property of linear maps is that 
$t(r\cdot\vec{v})=r\cdot t(\vec{v})$.

In a real space $\Re^n$ a line through the origin is a set 
$\set{r\cdot \vec{v}\suchthat r\in\Re}$. 
So $t$'s action 
\begin{equation*}
  r\cdot\vec{v}\mapsunder{t} r\cdot t(\vec{v})
\end{equation*}
is to send members of the line $\set{r\cdot \vec{v}\suchthat r\in\Re}$
in the domain to members of the line
$\set{s\cdot t(\vec{v})\suchthat s\in\Re}$
in the codomain. 

Thus, lines through the origin 
transform to lines through the origin.
Further, the action of~$t$ is determined by its effect $t(\vec{v})$
on any
nonzero vector element of the domain line.
\end{frame}
\begin{frame}
\ex
The line~$y=2x$ in the plane is
\begin{equation*}
  \set{r\cdot\colvec{1 \\ 2}\suchthat r\in\Re}
\end{equation*}
and suppose that $\map{t}{\Re^2}{\Re^2}$ is this.
\begin{equation*}
  \colvec{x \\ y}
  \mapsto
  \colvec{x+3y \\ 2x+4y}
\end{equation*}
Then for instance
\begin{equation*}
  \vec{v}=\colvec{1 \\ 2}\mapsunder{t}\colvec{7 \\ 10}
\end{equation*}
The 
scalar multiplication property in the definition of linear map 
imposes a simple
uniformity on $t$'s action on lines through the origin:~it 
has twice the effect on $2\vec{v}$, three times the
effect on $3\vec{v}$, etc.
\begin{equation*}
  \colvec{2 \\ 4}\mapsunder{t}\colvec{14 \\ 20}
  \qquad
  \colvec{-3 \\ -6}\mapsunder{t}\colvec{-21 \\ -30}
  \qquad
  \colvec{r \\ 2r}\mapsunder{t}\colvec{7r \\ 10r}
\end{equation*}
The action of $t$ on any $r\vec{v}$
is determined by its action
on the nonzero $\vec{v}$.
\end{frame}


\begin{frame}
  \frametitle{Pick one, any one}
Every plane vector is in some line through the origin
so to understand what $\map{t}{\Re^2}{\Re^2}$ does to plane elements it suffices to 
understand what it does to lines through the origin. 
By the prior slide, to understand what $t$ does to a line through the 
origin it suffices to understand what it does to a single nonzero
vector in that line.

So one way to understand a transformation's action is to take
a set containing one nonzero vector from each line through the origin,
and describe where the transformation maps the elements of that set.

\pause
A natural set with one nonzero element from each line through the
origin is the upper half unit circle (we will explain the colors below).
\begin{equation*}
  \set{\colvec{x \\ y}
       =\colvec{\cos(t) \\ \sin(t)}
         \suchthat 
         0\leq t<\pi}
  \qquad
  \vcenteredhbox{\includegraphics[scale=.75]{graphics/five_ii_a_unithalfcircle.pdf}}  
\end{equation*}  
\end{frame}




\section{Angles in plane transformations}
% =============================================
\begin{frame}{Angles}
\ex
This plane transformation.
\begin{equation*} 
  \colvec{x \\ y} \mapsto \colvec{2x \\ 2x+2y}
\end{equation*}
is a skew
\begin{equation*}
  \vcenteredhbox{\includegraphics[scale=.75]{graphics/five_ii_a_skew1.pdf}}
  \quad\rightarrow\quad
  \vcenteredhbox{\includegraphics[scale=.75]{graphics/five_ii_a_skew2.pdf}}
\end{equation*}
As we move through the unit half circle the vectors have varying dilations,
and are turned through varying angles.  
\end{frame}
\begin{frame}
The red vector
is dilated by a factor of $2\sqrt{2}$ and rotated counterclockwise through
an angle of $\pi/4\approx0.78$~radians.
\begin{equation*}
  \colvec{1 \\ 0}\mapsto \colvec{2 \\ 2}
  \qquad
  \vcenteredhbox{\includegraphics[scale=.75]{graphics/five_ii_a_skew3.pdf}}
\end{equation*}
The orange vector
is dilated by a factor of $2\sqrt{\cos^2(\pi/6)+1}=\sqrt{7}$ and rotated 
by about $0.48$~radians.
\begin{equation*}
  \colvec{\cos(\pi/6) \\ \sin(\pi/6)}\mapsto \colvec{2\cos(\pi/6) \\ 2\cos(\pi/6)+2\sin(\pi/6)}
  \qquad
  \vcenteredhbox{\includegraphics[scale=.75]{graphics/five_ii_a_skew4.pdf}}
\end{equation*}
\end{frame}

\begin{frame}
Here the horizontal axis is the angle of 
a vectors in the upper half unit circle,
and the vertical axis is
the angle that vector is rotated through.  
\begin{equation*}
  \colvec{x \\ y}\mapsto \colvec{2x \\ 2x+2y}
  \qquad
  \vcenteredhbox{\includegraphics{graphics/five_ii_a_skew5.pdf}}
\end{equation*}
The rotation angle of interest is~$0$~radians, here achieved 
by some green vector.
\end{frame}


\begin{frame}{Definition}
A vector that is rotated through an angle of $0$~radians, or of $\pi$~radians,
is an \alert{eigenvector}.
The factor by which it is dilated is the \alert{eigenvalue}.    
\end{frame}



%................
% diagonal
\begin{frame}
\ex
This diagonal plane transformation
\begin{equation*} 
  \colvec{x \\ y} \mapsto \colvec{-x \\ 2y}
\end{equation*}
has this simple action on the upper half unit circle.
\begin{equation*}
  \vcenteredhbox{\includegraphics[scale=.75]{graphics/five_ii_a_diagonal1.pdf}}
  \quad\rightarrow\quad
  \vcenteredhbox{\includegraphics[scale=.75]{graphics/five_ii_a_diagonal2.pdf}}
\end{equation*}
\end{frame}
\begin{frame}
Here is the
plot of the vectors in the upper half unit circle
against the angle they are rotated through.  
\begin{equation*}
  \colvec{x \\ y}\mapsto \colvec{-x \\ 2y}
  \qquad
  \vcenteredhbox{\includegraphics{graphics/five_ii_a_diagonal3.pdf}}
\end{equation*}
\end{frame}



%................
% generic
\begin{frame}
\ex
This generic plane transformation
\begin{equation*} 
  \colvec{x \\ y} \mapsto \colvec{x+2y \\ 3x+4y}
\end{equation*}
has this action on the upper half unit circle.
\begin{equation*}
  \vcenteredhbox{\includegraphics[scale=.75]{graphics/five_ii_a_generic1.pdf}}
  \quad\rightarrow\quad
  \vcenteredhbox{\includegraphics[scale=.75]{graphics/five_ii_a_generic2.pdf}}
\end{equation*}
\end{frame}
\begin{frame}
Again we
plot the vectors in the upper half unit circle
against the angle they are rotated through.  
\begin{equation*}
  \colvec{x \\ y}\mapsto \colvec{x+2y \\ 3x+4y}
  \qquad
  \vcenteredhbox{\includegraphics{graphics/five_ii_a_generic3.pdf}}
\end{equation*}
\end{frame}



%...........................
% \begin{frame}g
% \ExecuteMetaData[../gr3.tex]{GaussJordanReduction}
% \df[def:RedEchForm]
% 
% \end{frame}
\end{document}

Orange vector calculations
sage: 2*((cos(pi/6))^2+1)^(1/2)
sqrt(7)
sage: v=vector(RR,[cos(pi/6), sin(pi/6)])
sage: v
(0.866025403784439, 0.500000000000000)
sage: M=matrix(RR, [[2,0], [2,2]])
sage: M
[ 2.00000000000000 0.000000000000000]
[ 2.00000000000000  2.00000000000000]
sage: M=matrix(RR, [[2,2], [0,2]])
sage: M
[ 2.00000000000000  2.00000000000000]
[0.000000000000000  2.00000000000000]
sage: w=v*M
sage: w
(1.73205080756888, 2.73205080756888)
sage: arccos((w*v)/(w.norm()*v.norm()))
0.482170608511459
