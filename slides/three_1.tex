% see: https://groups.google.com/forum/?fromgroups#!topic/comp.text.tex/s6z9Ult_zds
\makeatletter\let\ifGm@compatii\relax\makeatother 
\documentclass[10pt,t,serif,professionalfont]{beamer}
\PassOptionsToPackage{pdfpagemode=FullScreen}{hyperref}
\PassOptionsToPackage{usenames,dvipsnames}{color}
% \DeclareGraphicsRule{*}{mps}{*}{}
\usepackage{../linalgjh}
\usepackage{present}
\usepackage{xr}\externaldocument{../map1} % read refs from .aux file
\usepackage{catchfilebetweentags}
\usepackage{etoolbox} % from http://tex.stackexchange.com/questions/40699/input-only-part-of-a-file-using-catchfilebetweentags-package
\makeatletter
\patchcmd{\CatchFBT@Fin@l}{\endlinechar\m@ne}{}
  {}{\typeout{Unsuccessful patch!}}
\makeatother

\mode<presentation>
{
  \usetheme{boxes}
  \setbeamercovered{invisible}
  \setbeamertemplate{navigation symbols}{} 
}
\addheadbox{filler}{\ }  % create extra space at top of slide 
\hypersetup{colorlinks=true,linkcolor=blue} 

\title[Isomorphisms] % (optional, use only with long paper titles)
{Three.I Isomorphisms}

\author{\textit{Linear Algebra} \\ {\small Jim Hef{}feron}}
\institute{
  \texttt{http://joshua.smcvt.edu/linearalgebra}
}
\date{}


\subject{Isomorphisms}
% This is only inserted into the PDF information catalog. Can be left
% out. 

\begin{document}
\begin{frame}
  \titlepage
\end{frame}

% =============================================
% \begin{frame}{Reduced Echelon Form} 
% \end{frame}



% ..... Three.I.1 .....
\section{Definition and examples}
%..........
\begin{frame}{Isomorphism}
\df[def:Isomorphism]
\ExecuteMetaData[../map1.tex]{df:Isomorphism}
\end{frame}




%..........
\begin{frame}
\lm[le:IsoSendsZeroToZero]
\ExecuteMetaData[../map1.tex]{lm:IsoSendsZeroToZero}

\pause
\pf
\ExecuteMetaData[../map1.tex]{pf:IsoSendsZeroToZero}
\qed
\end{frame}




%..........
\begin{frame}
\lm[le:PresStructIffPresCombos]
\ExecuteMetaData[../map1.tex]{lm:PresStructIffPresCombos}

\pause
\pf
\ExecuteMetaData[../map1.tex]{pf:PresStructIffPresCombos0}

\pause
\ExecuteMetaData[../map1.tex]{pf:PresStructIffPresCombos1}

\pause
\ExecuteMetaData[../map1.tex]{pf:PresStructIffPresCombos2}
\qed
\end{frame}



% ..... Three.I.2 .....
\section{Dimension characterizes isomorphism}
%..........
\begin{frame}
\lm[le:IsoInvAlsoIso]
\ExecuteMetaData[../map1.tex]{le:IsoInvAlsoIso}

\pause
\pf
\ExecuteMetaData[../map1.tex]{pf:IsoInvAlsoIso0}

\pause
\ExecuteMetaData[../map1.tex]{pf:IsoInvAlsoIso1}
\qed
\end{frame}




%..........
\begin{frame}
\th[th:IsoEquivRel]
\ExecuteMetaData[../map1.tex]{th:IsoEquivRel}

\pause
\pf
\ExecuteMetaData[../map1.tex]{pf:IsoEquivRel0}

\pause
\ExecuteMetaData[../map1.tex]{pf:IsoEquivRel1}
\end{frame}
\begin{frame}
\ExecuteMetaData[../map1.tex]{pf:IsoEquivRel2}
\qed
\end{frame}




%..........
\begin{frame}
\th[th:NDimSpaceIsoRN]
\ExecuteMetaData[../map1.tex]{th:NDimSpaceIsoRN}

The proof is the next two lemmas.

% \pause
% \pf
% \ExecuteMetaData[../map1.tex]{pf:NDimSpaceIsoRN0}

% \pause
% \ExecuteMetaData[../map1.tex]{pf:NDimSpaceIsoRN1}
% \qed
\end{frame}




%..........
\begin{frame}
\lm[lem:EqDimImpIso]
\ExecuteMetaData[../map1.tex]{lm:EqDimImpIso}

\pause
\pf
\ExecuteMetaData[../map1.tex]{pf:EqDimImpIso0}

\pause
\ExecuteMetaData[../map1.tex]{pf:EqDimImpIso1}
\end{frame}
\begin{frame}
\ExecuteMetaData[../map1.tex]{pf:EqDimImpIso2}

\pause
\ExecuteMetaData[../map1.tex]{pf:EqDimImpIso3}

\pause
\ExecuteMetaData[../map1.tex]{pf:EqDimImpIso4}
\qed
\end{frame}




%..........
\begin{frame}
\co[co:FiniteDimensionalIsoToReN]
\ExecuteMetaData[../map1.tex]{co:FiniteDimensionalIsoToReN}
\end{frame}



%...........................
% \begin{frame}
% \ExecuteMetaData[../gr3.tex]{GaussJordanReduction}
% \df[def:RedEchForm]
% 
% \end{frame}
\end{document}
