% see: https://groups.google.com/forum/?fromgroups#!topic/comp.text.tex/s6z9Ult_zds
\makeatletter\let\ifGm@compatii\relax\makeatother 
\documentclass[10pt,t,serif,professionalfont]{beamer}
\PassOptionsToPackage{pdfpagemode=FullScreen}{hyperref}
\PassOptionsToPackage{usenames,dvipsnames}{color}
% \DeclareGraphicsRule{*}{mps}{*}{}
\usepackage{../linalgjh}
\usepackage{present}
\usepackage{xr}\externaldocument{../gr1} % read refs from .aux file
\usepackage{xr}\externaldocument{../gr3} % read refs from .aux file
\usepackage{catchfilebetweentags}
\usepackage{etoolbox} % from http://tex.stackexchange.com/questions/40699/input-only-part-of-a-file-using-catchfilebetweentags-package
\makeatletter
\patchcmd{\CatchFBT@Fin@l}{\endlinechar\m@ne}{}
  {}{\typeout{Unsuccessful patch!}}
\makeatother

\mode<presentation>
{
  \usetheme{boxes}
  \setbeamercovered{invisible}
  \setbeamertemplate{navigation symbols}{} 
}
\addheadbox{filler}{\ }  % create extra space at top of slide 
\hypersetup{colorlinks=true,linkcolor=blue} 

\title[Reduced Echelon Form] % (optional, use only with long paper titles)
{One.III Reduced Echelon Form}

\author{\textit{Linear Algebra} \\ {\small Jim Hef{}feron}}
\institute{
  \texttt{http://joshua.smcvt.edu/linearalgebra}
}
\date{}


\subject{Reduced Echelon Form}
% This is only inserted into the PDF information catalog. Can be left
% out. 

\begin{document}
\begin{frame}
  \titlepage
\end{frame}

% =============================================
% \begin{frame}{Reduced Echelon Form} 
% \end{frame}



% ..... One.III.1 .....
\section{Gauss-Jordan reduction}
\begin{frame}{Pivoting}
\ex
Here is an extension of Gauss's Method with some advantages.
\begin{eqnarray*}
  \begin{linsys}{3}
    x  &+  &y  &-  &z  &=  &2   \\
   2x  &-  &y  &   &   &=  &-1  \\
    x  &-  &2y &+  &2z &=  &-1 
  \end{linsys}
  &\grstep[-1\rho_1+\rho_3]{-2\rho_1+\rho_2}
  &\begin{linsys}{3}
    x  &+  &y  &-  &z  &=  &2   \\
       &-  &3y &   &2z &=  &-5  \\
       &-  &3y &+  &3z &=  &-3 
  \end{linsys}                         \\
  &\grstep{-1\rho_2+\rho_3}
  &\begin{linsys}{3}
    x  &+  &y  &-  &z  &=  &2   \\
       &-  &3y &   &2z &=  &-5  \\
       &   &   &   &z  &=  &2 
  \end{linsys}
\end{eqnarray*}
Instead of doing back substitution, we continue using the row operations.
\pause
First make all the leading entries one.
\begin{eqnarray*}
  % \begin{linsys}{3}
  %   x  &+  &y  &-  &z  &=  &2   \\
  %      &-  &3y &   &2z &=  &-5  \\
  %      &   &   &   &z  &=  &2 
  % \end{linsys}
  &\grstep{(-1/3)\rho_2}
  &\begin{linsys}{3}
    x  &+  &y  &-  &z      &=  &2   \\
       &   &y  &-  &(2/3)z &=  &5/3  \\
       &   &   &   &z      &=  &2 
  \end{linsys}
\end{eqnarray*}
\end{frame}\begin{frame}
\noindent Finish by eliminating upwards with the leading entries.
\begin{equation*}
  \begin{linsys}{3}
    x  &+  &y  &-  &z      &=  &2   \\
       &   &y  &-  &(2/3)z &=  &5/3  \\
       &   &   &   &z      &=  &2 
  \end{linsys} \;
  \grstep[(2/3)\rho_3+\rho_2]{\rho_3+\rho_1}\;
  \begin{linsys}{3}
    x  &+  &y  &   &       &=  &4   \\
       &   &y  &   &       &=  &3  \\
       &   &   &   &z      &=  &2 
  \end{linsys}
  \;\grstep{-\rho_2+\rho_1}\;
  \begin{linsys}{3}
    x  &   &   &   &       &=  &1   \\
       &   &y  &   &       &=  &3  \\
       &   &   &   &z      &=  &2 
  \end{linsys}
\end{equation*}
Now we can read off the solution.

\pause\medskip
\ExecuteMetaData[../gr3.tex]{Pivoting}
\end{frame}



% ..........
\begin{frame}
\ex
With this system
\begin{equation*}
  \begin{linsys}{4}
    x  &-  &y  &   &   &-  &2w  &=  &2   \\
    x  &+  &y  &+  &3z &+  &w   &=  &1  \\
       &-  &y  &+  &z  &-  &w   &=  &0 
  \end{linsys}
\end{equation*}
we can rewrite in matrix notation
and do Gauss's Method.
\begin{equation*}
  \grstep{-1\rho_1+\rho_2}\,
  \begin{amat}{4}
    1  &-1  &0  &-2  &2  \\
    0  &2   &3  &3   &1   \\
    0  &-1  &1  &-1  &0  
  \end{amat}                         
  \,\grstep{(1/2)\rho_2+\rho_3}\,
  \begin{amat}{4}
    1  &-1  &0    &-2   &2  \\
    0  &2   &3    &3    &1   \\
    0  &0   &5/2  &1/2  &-1/2  
  \end{amat}                         
\end{equation*}
We can combine the operations making the leading entries to one. 
\begin{equation*}
  \grstep[(2/5)\rho_3]{(1/2)\rho_2}\,
  \begin{amat}{4}
    1  &-1  &0    &-2    &2  \\
    0  &1   &3/2  &3/2   &-1/2   \\
    0  &0   &1    &1/5   &-1/5  
  \end{amat}                         
\end{equation*}
Now eliminate upwards.
\begin{equation*}
  \grstep{-(3/2)\rho_3+\rho_2}\,
  \begin{amat}{4}
    1  &-1  &0    &-2    &2  \\
    0  &1   &0    &6/5   &-1/5   \\
    0  &0   &1    &1/5   &-1/5  
  \end{amat}                           
  \,\grstep{\rho_2+\rho_1}\,
  \begin{amat}{4}
    1  &0   &0    &-4/5  &9/5  \\
    0  &1   &0    &6/5   &-1/5   \\
    0  &0   &1    &1/5   &-1/5  
  \end{amat}                         
\end{equation*}
\end{frame}\begin{frame}
\noindent The final augmented matrix
\begin{equation*}
  \begin{amat}{4}
    1  &0   &0    &-4/5  &9/5  \\
    0  &1   &0    &6/5   &-1/5   \\
    0  &0   &1    &1/5   &-1/5  
  \end{amat}                         
\end{equation*}
leads to this description of the solution set.
\begin{equation*}
  \set{\colvec{9/5 \\ -1/5 \\ -1/5 \\ 0} 
       +\colvec{4/5  \\ -6/5 \\ -1/5 \\ 1}w
       \suchthat w\in\Re}
\end{equation*}
\end{frame}




% ..........
\begin{frame}{Gauss-Jordan reduction}
\ExecuteMetaData[../gr3.tex]{GaussJordanReduction}

\df[def:RedEchForm]
\ExecuteMetaData[../gr3.tex]{df:RedEchForm}

\pause\medskip
\ExecuteMetaData[../gr3.tex]{CostRedEchForm}
\end{frame}




% ..........
\begin{frame}{Reduces to is an equivalence}
\lm[le:RowOpsRev]
\ExecuteMetaData[../gr3.tex]{lm:RowOpsRev}
\pause
\pf
\ExecuteMetaData[../gr3.tex]{pf:RowOpsRev}
(The third clause is false if $i=j$.)
\qed

\pause\medskip
\ExecuteMetaData[../gr3.tex]{EquivMatrices}
\end{frame}




% ..........
\begin{frame}
\lm[lm:ReducesToIsEqRel]
\ExecuteMetaData[../gr3.tex]{lm:ReducesToIsEqRel}

\pause
\pf
\ExecuteMetaData[../gr3.tex]{pf:ReducesToIsEqRel0}

\pause
\ExecuteMetaData[../gr3.tex]{pf:ReducesToIsEqRel1}

\pause
\ExecuteMetaData[../gr3.tex]{pf:ReducesToIsEqRel2}
\qed
\end{frame}




% ..........
\begin{frame}
\df[df:RowEquivalence]
\ExecuteMetaData[../gr3.tex]{df:RowEquivalence}

\pause\medskip
\ExecuteMetaData[../gr3.tex]{RowEquivalanceClasses}
\centergraphic{../ch1.27}
\end{frame}




% ..... One.III.2 .....
\section{Linear Combination Lemma}
\begin{frame}{How Gauss's method acts}
\ex Consider this reduction.
\begin{multline*}
  \begin{amat}{2}
    1  &3  &5  \\
    2  &4  &8  
  \end{amat}
  \,\grstep{-2\rho_1+\rho_2}\,
  \begin{amat}{2}
    1  &3   &5  \\
    0  &-2  &-2  
  \end{amat}
  \,\grstep{-(1/2)\rho_2}\,
  \begin{amat}{2}
    1  &3   &5  \\
    0  &1  &1  
  \end{amat}                         \\
  \grstep{-3\rho_2+\rho_1}\,
  \begin{amat}{2}
    1  &0  &2  \\
    0  &1  &1  
  \end{amat}
\end{multline*}
Denote the matrices as $A\rightarrow D\rightarrow G\rightarrow B$
The steps take us through these combinations.
\begin{eqnarray*}
  \left(\begin{array}{l}
    \alpha_1 \\
    \alpha_2
  \end{array}\right)
  &\grstep{-2\rho_1+\rho_2}
  &\left(\begin{array}{l}
    \delta_1=\alpha_1  \\
    \delta_2=-2\alpha_1+\alpha_2
  \end{array}\right)                                \\
  &\grstep{-(1/2)\rho_2}
  &\left(\begin{array}{l}
    \gamma_1=\alpha_1  \\ 
    \gamma_2=-1\alpha_1+(1/2)\alpha_2
  \end{array}\right)                                 \\
  &\grstep{-3\rho_2+\rho_1}
  &\left(\begin{array}{l}
    \beta_1= 4\alpha_1-(3/2)\alpha_2  \\
    \beta_2=-1\alpha_1+(1/2)\alpha_2
  \end{array}\right)                      
\end{eqnarray*}
\pause
Gauss's method systemmatically develops linear combinations of rows.
\end{frame}




% ..........
\begin{frame}{Linear Combination Lemma} 
\lm[lm:LinearCombinationLemma]
\ExecuteMetaData[../gr3.tex]{lm:LinearCombinationLemma}
\pause
\pf
\ExecuteMetaData[../gr3.tex]{pf:LinearCombinationLemma}
\qed
\end{frame}




% ..........
\begin{frame}
\co[cor:RowsOfEqMatsLinCombos]
\ExecuteMetaData[../gr3.tex]{co:RowsOfEqMatsLinCombos}
\pause

\pf 
\ExecuteMetaData[../gr3.tex]{pf:RowsOfEqMatsLinCombos0}

\pause
\ExecuteMetaData[../gr3.tex]{pf:RowsOfEqMatsLinCombos1}
\end{frame}\begin{frame}
\ExecuteMetaData[../gr3.tex]{pf:RowsOfEqMatsLinCombos2}
\qed
\end{frame}




% ..........
\begin{frame}
\lm[le:EchFormNoLinCombo]
\ExecuteMetaData[../gr3.tex]{lm:EchFormNoLinCombo}
\pause
\pf 
\ExecuteMetaData[../gr3.tex]{pf:EchFormNoLinCombo0}
\end{frame}\begin{frame}
\ExecuteMetaData[../gr3.tex]{pf:EchFormNoLinCombo1}

\pause
\ExecuteMetaData[../gr3.tex]{pf:EchFormNoLinCombo2}
\qed
\end{frame}




% ..........
\begin{frame}
\th[th:ReducedEchelonFormIsUnique]
\ExecuteMetaData[../gr3.tex]{th:ReducedEchelonFormIsUnique}
\pause
\pf 
\ExecuteMetaData[../gr3.tex]{pf:ReducedEchelonFormIsUnique0}

\pause
\ExecuteMetaData[../gr3.tex]{pf:ReducedEchelonFormIsUnique1}

\pause
\ExecuteMetaData[../gr3.tex]{pf:ReducedEchelonFormIsUnique2}
\end{frame}\begin{frame}
\ExecuteMetaData[../gr3.tex]{pf:ReducedEchelonFormIsUnique3}
\end{frame}\begin{frame}
\ExecuteMetaData[../gr3.tex]{pf:ReducedEchelonFormIsUnique4}
\qed

\pause\medskip
\ExecuteMetaData[../gr3.tex]{ReducedEchelonFormIsCanonicalForm}
\centergraphic{../ch1.31}
\end{frame}




%...........................
% \begin{frame}
% 
% \end{frame}
\end{document}
