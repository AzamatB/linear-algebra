% see: https://groups.google.com/forum/?fromgroups#!topic/comp.text.tex/s6z9Ult_zds
\makeatletter\let\ifGm@compatii\relax\makeatother 
\documentclass[10pt,t]{beamer}
\usefonttheme{professionalfonts}
\usefonttheme{serif}
\PassOptionsToPackage{pdfpagemode=FullScreen}{hyperref}
\PassOptionsToPackage{usenames,dvipsnames}{color}
% \DeclareGraphicsRule{*}{mps}{*}{}
\usepackage{../linalgjh}
\usepackage{present}
\usepackage{xr}\externaldocument{../map3} % read refs from .aux file
\usepackage{xr}\externaldocument{../map1} % read refs from .aux file
\usepackage{xr}\externaldocument{../vs3} % read refs from .aux file
\usepackage{catchfilebetweentags}
\usepackage{etoolbox} % from http://tex.stackexchange.com/questions/40699/input-only-part-of-a-file-using-catchfilebetweentags-package
\makeatletter
\patchcmd{\CatchFBT@Fin@l}{\endlinechar\m@ne}{}
  {}{\typeout{Unsuccessful patch!}}
\makeatother

\mode<presentation>
{
  \usetheme{boxes}
  \setbeamercovered{invisible}
  \setbeamertemplate{navigation symbols}{} 
}
\addheadbox{filler}{\ }  % create extra space at top of slide 
\hypersetup{colorlinks=true,linkcolor=blue} 

\title[Computing Linear Maps] % (optional, use only with long paper titles)
{Three.III Computing Linear Maps}

\author{\textit{Linear Algebra} \\ {\small Jim Hef{}feron}}
\institute{
  \texttt{http://joshua.smcvt.edu/linearalgebra}
}
\date{}


\subject{Computing Linear Maps}
% This is only inserted into the PDF information catalog. Can be left
% out. 

\begin{document}
\begin{frame}
  \titlepage
\end{frame}

% =============================================
% \begin{frame}{Reduced Echelon Form} 
% \end{frame}



% ..... Three.III.1 .....
\section{Representing Linear Maps with Matrices}
%..........
\begin{frame}{\parbox[t]{\paperwidth}{Linear maps are determined by the action on a basis}}
Consider a domain space~$V$ and a codomain space~$W$,
and fix a basis~$B_{V}=\sequence{\vec{\beta}_1,\ldots,\vec{\beta}_n}$ for
the domain.
Now for any linear map $\map{h}{V}{W}$ consider the 
action of $h$ on the basis elements
$h(\vec{\beta}_1)$, \ldots, $h(\vec{\beta}_n)$. 
\pause
Using those, we can calculate 
the action of $h$ on any $\vec{v}\in V$.
\begin{equation*}
  h(\vec{v})=h(c_1\cdot\vec{\beta}_1+\cdots+c_n\cdot\vec{\beta}_n)
            =c_1\cdot h(\vec{\beta}_1)+\cdots+c_n\cdot h(\vec{\beta}_n)
  \tag{*}
\end{equation*}
This section develops a scheme to easily do those calculations.

\pause\medskip
\ex
Let the domain be $V=\polyspace_2$ with the basis
$B_{V}=\sequence{1,1+x,1+x+x^2}$.
Let the codomain be $\Re^2$ with this basis.
\begin{equation*}
  B_{W}
  =\sequence{\colvec{2 \\ 0}, \colvec{-1 \\ 1}}
\end{equation*}

\pause
We are developing a scheme to compute the action of this map $h$
on any member 
$\vec{v}=c_1\cdot\vec{\beta}_1+c_2\cdot\vec{\beta}_2+c_3\cdot\vec{\beta}_3$
of the domain.
We want to detail the mechanics of calculating
$\rep{h(\vec{v})}{B_{w}}$ from the 
$\rep{h(\vec{\beta}_i)}{B_{V}}$.
\end{frame}
\begin{frame}
Let the map $h$ have this action on the domain basis.
\begin{equation*}
  h(1)=\colvec{0 \\ 1}
  \quad
  h(1+x)=\colvec{3 \\ 2}
  \quad
  h(1+x+x^2)=\colvec{-2 \\ -1}
\end{equation*}

\pause
We first find the representation, with respect to the codomain's basis $B_{W}$,
of the action of $h$ on $B_{V}$.
\begin{align*}
  &\rep{h(\vec{\beta}_1)}{B_{W}}
  =\colvec{1/2 \\ 1}
  \quad\text{since }
  (1/2)\cdot\colvec{2 \\ 0}+1\cdot\colvec{-1 \\ 1}=\colvec{0 \\ 1}     \\
  &\rep{h(\vec{\beta}_2)}{B_{W}}
  =\colvec{5/2 \\ 2}
  \quad\text{since }
  (5/2)\cdot\colvec{2 \\ 0}+2\cdot\colvec{-1 \\ 1}=\colvec{3 \\ 2}     \\
  &\rep{h(\vec{\beta}_3)}{B_{W}}
  =\colvec{-3/2 \\ -1}
  \quad\text{since }
  (-3/2)\cdot\colvec{2 \\ 0}-1\cdot\colvec{-1 \\ 1}=\colvec{-2 \\ -1}     
\end{align*}
\pause
We summarize by writing those three vectors side-by-side to make a matrix.
\begin{equation*}
  \begin{mat}
    1/2 &5/2 &-3/2 \\
    1   &2   &-1
  \end{mat}
\end{equation*}
\end{frame}
\begin{frame}
Consider the domain vector
$\vec{v}=c_1\cdot\vec{\beta}_1+c_2\cdot\vec{\beta}_2+c_3\cdot\vec{\beta}_3$.
\begin{equation*}
  \rep{\vec{v}}{B_{V}}=\colvec{c_1 \\ c_2 \\ c_3}
\end{equation*}
Apply equation~($*$).
\begin{align*}
  c_1h(\vec{\beta}_1)+c_2h(\vec{\beta}_2)+c_3h(\vec{\beta}_3) 
    &=c_1(\,(1/2)\cdot\colvec{2 \\ 0}+1\cdot\colvec{-1 \\ 1}\,)  \\
      &\quad +c_2(\,(5/2)\cdot\colvec{2 \\ 0}+2\cdot\colvec{-1 \\ 1}\,)  \\
      &\quad +c_3(\,(-3/2)\cdot\colvec{2 \\ 0}-1\cdot\colvec{-1 \\ 1}\,)
\end{align*}
Regroup.
\begin{gather*}
  ((1/2)c_1+(5/2)c_2-(3/2)c_3)\cdot\colvec{2 \\ 0}
  +(1c_1+2c_2-1c_3)\cdot\colvec{-1 \\ 1}          \\
  \rep{h(\vec{v})}{B_{W}}
  =\colvec{(1/2)c_1+(5/2)c_2-(3/2)c_3 \\ 1c_1+2c_2-1c_3}
\end{gather*}
\end{frame}
\begin{frame}
\noindent
So, the effect of the linear map summarized by this matrix
\begin{equation*}
  \begin{mat}
    1/2 &5/2 &-3/2 \\
    1   &2   &-1
  \end{mat}
\end{equation*}
on the domain vector summarized by this vector 
\begin{equation*}
  \colvec{c_1 \\ c_2 \\ c_3}
\end{equation*}
is to send it to this member of the codomain. 
\begin{equation*}
  \colvec{(1/2)c_1+(5/2)c_2-(3/2)c_3 \\ 1c_1+2c_2-1c_3}
\end{equation*}
\pause
This is the computational scheme:
we get the output vector by taking 
the dot product of each row of the matrix with the single column of the 
input vector.
\end{frame}




%..........
\begin{frame}{Matrix representation of a linear map}
\df[def:MatRepMap]
\ExecuteMetaData[../map3.tex]{df:MatRepMap}

\pause
\medskip
We often omit the subscript on the matrix.
\end{frame}
\begin{frame}
\ex
Consider projection $\map{\pi}{\Re^2}{\Re^2}$ onto the $x$-axis.
\begin{equation*}
  \colvec{a \\ b}\mapsunder{\pi}\colvec{a \\ 0}
\end{equation*}
If we take the input and output bases to be 
\begin{equation*}
  B=\sequence{\colvec{1 \\ 1}, \colvec{-1 \\ 1}}
  \text{ and }
  D=\sequence{\colvec{0 \\ 1}, \colvec{2 \\ 2}}
\end{equation*}
then we can compute 
\begin{equation*}
  \colvec{1 \\ 1}\mapsunder{\pi}\colvec{1 \\ 0}
  \text{ so }
  \rep{\pi(\vec{\beta}_1)}{D}=\colvec{-1 \\ 1/2} 
\end{equation*}
and
\begin{equation*}
  \colvec{-1 \\ 1}\mapsunder{\pi}\colvec{-1 \\ 0}
  \text{ so }
  \rep{\pi(\vec{\beta}_2)}{D}=\colvec{1 \\ -1/2}
\end{equation*}
and the matrix representing $\pi$ is this.
\begin{equation*}
  \rep{\pi}{B,D}=
  \begin{mat}
    -1  &1 \\
   1/2  &-1/2
  \end{mat}
\end{equation*}
\end{frame}
\begin{frame}
\ex
Again consider projection onto the $x$-axis
\begin{equation*}
  \colvec{a \\ b}\mapsunder{\pi}\colvec{a \\ 0}
\end{equation*}
and this time we take the input and output bases to be the standard one.
\begin{equation*}
  B=D=\stdbasis_2
  =\sequence{\colvec{1 \\ 0}, \colvec{0 \\ 1}}
\end{equation*}
We have
\begin{equation*}
  \colvec{1 \\ 0}\mapsunder{\pi}\colvec{1 \\ 0}
  \text{ so }
  \rep{\pi(\vec{\beta}_1)}{D}=\colvec{1 \\ 0}
\end{equation*}
and
\begin{equation*}
  \colvec{0 \\ 1}\mapsunder{\pi}\colvec{0 \\ 0}
  \text{ so }
  \rep{\pi(\vec{\beta}_2)}{D}=\colvec{0 \\ 0}
\end{equation*}
so $\rep{\pi}{\stdbasis_2,\stdbasis_2}$ is this.
\begin{equation*}
  \begin{mat}
    1  &0  \\
    0  &0
  \end{mat}
\end{equation*}
\end{frame}




%..........
\begin{frame}
\th[th:MatMultRepsFuncAppl]
\ExecuteMetaData[../map3.tex]{th:MatMultRepsFuncAppl}
\end{frame}
\begin{frame}
\pf
This formalizes the example that began this subsection.
See \nearbyexercise{exer:MatVecMultRepLinMap}.
\qed

\pause
\medskip
\df[def:MatrixVecProd]
\ExecuteMetaData[../map3.tex]{df:MatrixVecProd}

\pause
\ex
We don't need to describe spaces and bases to perform the operation.
\begin{equation*}
  \begin{mat}[r]
    3  &1  &2  \\
    0  &-2 &5
  \end{mat}
  \colvec[r]{4  \\ -1 \\ -3}
  =\colvec{3\cdot 4+1\cdot(-1)+2\cdot(-3) \\ 0\cdot 4-2\cdot(-1)+5\cdot(-3)}
  =\colvec[r]{5 \\ -13}
\end{equation*}
\end{frame}
\begin{frame}
\ex
Recall the two matrices 
\begin{equation*}  
  \rep{\pi}{B,D}=
  \begin{mat}
    -1  &1 \\
   1/2  &-1/2
  \end{mat}
  \quad\text{and}\quad
  \rep{\pi}{\stdbasis_2,\stdbasis_2}
  \begin{mat}
    1  &0  \\
    0  &0
  \end{mat}
\end{equation*}
representing 
projection $\map{\pi}{\Re^2}{\Re^2}$ onto the $x$-axis
% \begin{equation*}
%   \colvec{a \\ b}\mapsunder{\pi}\colvec{a \\ 0}
% \end{equation*}
with respect to 
\begin{equation*}
  B=\sequence{\colvec{1 \\ 1}, \colvec{-1 \\ 1}},\,
  D=\sequence{\colvec{0 \\ 1}, \colvec{2 \\ 2}}
\end{equation*}
and  with respect to the standard bases $\stdbasis_{2},\stdbasis_{2}$.
\pause
This domain vector has the given representations.
\begin{equation*}
  \vec{v}=\colvec{-1 \\ 5}
  \qquad
  \rep{\vec{v}}{B}=\colvec{2 \\ 3}
  \text{ and }
  \rep{\vec{v}}{\stdbasis_{2}}=\colvec{-1 \\ 5}
\end{equation*}
and the matrix-vector products
\begin{equation*}
  \begin{mat}
    -1  &1 \\
    1/2  &-1/2
  \end{mat}
  \colvec{2 \\ 3}
  =\colvec{1 \\ -1/2}
  \qquad
  \begin{mat}
    1  &0  \\
    0  &0
  \end{mat}
  \colvec{-1 \\ 5}
  =
  \colvec{-1 \\ 0}
\end{equation*}
give the representations
$\rep{\pi(\vec{v})}{D}$ and $\rep{\pi(\vec{v})}{\stdbasis_{2}}$.
\begin{equation*}
  1\cdot\colvec{0 \\ 1}-(1/2)\cdot\colvec{2 \\ 2}
  =\colvec{-1 \\ 0}
  =-1\cdot\vec{e}_1+0\cdot\vec{e}_2
\end{equation*}
\end{frame}







% ..... Three.III.2 .....
\section{Any Matrix Represents a Linear Map}
%..........
\begin{frame}
\ex
The prior subsection shows how to start with a linear map and produce its matrix
representation.
Suppose instead that we start with a matrix.
\begin{equation*}
  M=\begin{mat}
    1 &2 \\
    3 &4
  \end{mat}
\end{equation*}
We want to check that its application represents a linear map.
\pause
Fix a domain and codomain, and bases.
\begin{equation*}
  \stdbasis_2\subset\Re^2
  \quad
  \sequence{1-x,1+x}\subset\polyspace_1
\end{equation*}
We will produce a function $\map{m}{\Re^2}{\polyspace_1}$
that is represented by the matrix $M$ with respect to these bases.
Then we will verify that $m$ is linear.

\pause
We define $m$ in this way: 
first represent $\vec{u}, \vec{v}\in\Re^2$ with respect to the domain basis,
then multiply by $M$, and then take the resulting representations
with respect to the codomain basis and covert over to members of the codomain.
\pause
Note that there is one and only one function~$m$ defined in this way, because
the representation of a vector with respect to a basis can be done in
one and only one way.
\end{frame}
\begin{frame}
Now we will show that the map is linear 
$m(c\cdot\vec{u}+d\cdot\vec{v})=c\cdot m(\vec{u})+d\cdot m(\vec{v})$.
Let $\vec{u},\vec{v}\in\Re^2$ be elements of the domain. 
Because the domain basis is $\stdbasis_2$ 
the vectors are represented by themselves.
We multiply by $M$.
\begin{align*}
  \begin{mat}
    1 &2 \\
    3 &4
  \end{mat}
  \left(c\cdot\colvec{u_1 \\ u_2}+d\cdot\colvec{v_1 \\ v_2}\right)
  &=
  \begin{mat}
    1 &2 \\
    3 &4
  \end{mat}                              
  \colvec{cu_1+dv_1 \\ cu_2+dv_2}    \\
  &=
  \colvec{1(cu_1+dv_1)+2(cu_2+dv_2) \\ 3(cu_1+dv_1)+4(cu_2+dv_2)}   \\ 
  &=
  \colvec{1cu_1+2cu_2 \\ 3cu_1+4cu_2}  
  +
  \colvec{1dv_1+2dv_2 \\ 3dv_1+4dv_2}     \\              
  &=
  c\cdot\begin{mat}
    1 &2 \\
    3 &4
  \end{mat}
  \colvec{u_1 \\ u_2}
  +
  d\cdot\begin{mat}
    1 &2 \\
    3 &4
  \end{mat}
  \colvec{v_1 \\ v_2}
\end{align*}
The result is $c\cdot\rep{m(\vec{u})}{D}+d\rep{m(\vec{v})}{D}$.
We finish by using the basis $D$ for the codomain to convert the representations
to vectors in~$\polyspace_1$.
\end{frame}

\begin{frame}
\th[th:MatIsLinMap]
\ExecuteMetaData[../map3.tex]{th:MatIsLinMap}
\pause
\pf
\ExecuteMetaData[../map3.tex]{pf:MatIsLinMap}
\qed
\end{frame}




%..........
\begin{frame}
We will close this subsection by connecting properties of a linear map
with properties of any associated matrix.

\ex
We claim that a map $\map{h}{V}{W}$ is the zero map if and only
if it is represented, with respect to any bases, by the zero matrix.

\pause
First assume it is the zero map.
Then for any bases $B,D$ we have $h(\vec{\beta}_i)=\zero_W$, which is 
represented with respect to $D$ by the zero vector.
Thus $h$ is represented by the zero matrix.

\pause
Now assume that there are bases $B,D$ such that $\rep{h}{B,D}$ is the zero
matrix.
Then for each $\vec{\beta}_i$ we have that $\rep{h(\vec{\beta}_1)}{D}$ is a
column vector of zeros, and so $h(\vec{\beta}_i)$ is $\zero_W$.
Extending linearly, we have that $h$ maps each $\vec{v}\in V$ to $\zero_W$,
and $h$ is the zero map.  

\pause
\medskip
One thing this example does not illustrate is that typically a linear map
will have many different matrices representing it, with respect to 
the many different pairs of bases~$B,D$.
A matrix property that derives from the map will be shared across
all these representing matrices. 
\end{frame}




%..........
\begin{frame}
\th[th:RankMatEqRankMap]
\ExecuteMetaData[../map3.tex]{th:RankMatEqRankMap}
\pause
\pf
\ExecuteMetaData[../map3.tex]{pf:RankMatEqRankMap0}

\pause
\ExecuteMetaData[../map3.tex]{pf:RankMatEqRankMap1}
\end{frame}
\begin{frame}
\ExecuteMetaData[../map3.tex]{pf:RankMatEqRankMap2}
\qed
\end{frame}




%..........
\begin{frame}
\ex 
The range of the linear transformation $\map{t}{\Re^2}{\Re^2}$
given by
\begin{equation*}
  \colvec{a \\ b}\mapsto\colvec{2a-b \\ 2a-b}
\end{equation*}
is the line~$x=y$.
Thus the map $t$'s rank is~$1$.

Represent $t$ first with respect the standard bases $\stdbasis_2,\stdbasis_2$
and then with respect to these.
\begin{equation*}
  B=\sequence{\colvec{1 \\ 1}, \colvec{-1 \\ 1}},
  \,
  D=\sequence{\colvec{1/2 \\ 0}, \colvec{0 \\ 1/3}}
\end{equation*}
The standard basis case is easy.  This is the other calculation.
\begin{equation*}
  \rep{\colvec{1 \\ 1}}{D}=\colvec{2 \\ 3}
  \quad
  \rep{\colvec{-3 \\ -3}}{D}=\colvec{-6 \\ -9}
\end{equation*}
\pause
We end with these two representations, matrices of rank~$1$.
\begin{equation*}
  \rep{t}{\stdbasis_2,\stdbasis_2}
  =
  \begin{mat}
    2  &-1  \\
    2  &-1  
  \end{mat}
  \qquad
  \rep{t}{}
  =
  \begin{mat}
    2  &-6  \\
    3  &-9  
  \end{mat}
\end{equation*}
\end{frame}




%..........
\begin{frame}
\co[cor:MatDescsMap]
\ExecuteMetaData[../map3.tex]{co:MatDescsMap}
\pause
\pf
\ExecuteMetaData[../map3.tex]{pf:MatDescsMap0}

\pause
\ExecuteMetaData[../map3.tex]{pf:MatDescsMap1}
\qed
\end{frame}




%..........
\begin{frame}
\df[df:NonsingularMap]
\ExecuteMetaData[../map3.tex]{df:NonsingularMap}
\end{frame}




%..........
\begin{frame}
\lm[le:NonsingMatIffNonsingMap]
\ExecuteMetaData[../map3.tex]{le:NonsingMatIffNonsingMap}
\pause
\pf
\ExecuteMetaData[../map3.tex]{pf:NonsingMatIffNonsingMap0}

\pause
\ExecuteMetaData[../map3.tex]{pf:NonsingMatIffNonsingMap1}
\qed
\end{frame}




%..........
\begin{frame}
\ex
This matrix
\begin{equation*}
  \begin{mat}
    0  &3  \\
   -1  &2
  \end{mat}
\end{equation*}
is nonsingular since by inspection its two rows form a linearly independent
set.
So any map, with any domain and codomain, and represented by this matrix  
with respect to any bases,
is an isomorphism.

\pause
\ex
Gauss's method will show that this matrix
\begin{equation*}
  \begin{mat}
    2  &1  &-2  \\
    3  &2  &1   \\
   -1  &0  &5
  \end{mat}
\end{equation*}
is singular so any map that it represents will be a homomorphism that
is not an isomorphism.
\end{frame}




%...........................
% \begin{frame}
% \ExecuteMetaData[../gr3.tex]{GaussJordanReduction}
% \df[def:RedEchForm]
% 
% \end{frame}
\end{document}
