% see: https://groups.google.com/forum/?fromgroups#!topic/comp.text.tex/s6z9Ult_zds
\makeatletter\let\ifGm@compatii\relax\makeatother 
\documentclass[10pt,t,serif]{beamer}
\usefonttheme{professionalfonts}
\usefonttheme{serif}
\PassOptionsToPackage{pdfpagemode=FullScreen}{hyperref}
\PassOptionsToPackage{usenames,dvipsnames}{color}
% \usepackage{linear_algebra}
\usepackage{../linalgjh}
\usepackage{present}
\usepackage{xr}\externaldocument{../gr1}
\usepackage{catchfilebetweentags}
\usepackage{etoolbox} % from http://tex.stackexchange.com/questions/40699/input-only-part-of-a-file-using-catchfilebetweentags-package
\makeatletter
\patchcmd{\CatchFBT@Fin@l}{\endlinechar\m@ne}{}
  {}{\typeout{Unsuccessful patch!}}
\makeatother

\mode<presentation>
{
  \usetheme{boxes}
  \setbeamercovered{invisible}
  \setbeamertemplate{navigation symbols}{} 
}
\addheadbox{filler}{\ }  % create extra space at top of slide 
\hypersetup{colorlinks=true,linkcolor=blue} 

\title[Solving Linear Systems] % (optional, use only with long paper titles)
{Solving Linear Systems}

\author{\textit{Linear Algebra} \\ {\small Jim Hef{}feron}}
\institute{
  \texttt{http://joshua.smcvt.edu/linearalgebra}
}
\date{}


\subject{Solving Linear Systems}
% This is only inserted into the PDF information catalog. Can be left
% out. 
% \AtBeginDocument{\makeatletter
% \let\ref\@refstar
% \makeatother}
% \usepackage{xparse}% http://ctan.org/pkg/xparse

% \makeatletter
% \NewDocumentCommand{\reff}{s m}{%
%   \IfBooleanTF{#1}% Check for starred variant
%     {\beamer@origref{#2}}% \reff*
%     {\hyperlink{#2}{\beamer@origref{#2}}}% \reff
% }
% \makeatother

\begin{document}
\begin{frame}
  \titlepage
\end{frame}

% =============================================



% ..... One.I.1 .....
\section{Gauss's Method}
\begin{frame}{Linear systems} 
\df[def:linearcombination]
\ExecuteMetaData[../gr1.tex]{df:linearcombination}

\ex This is a linear combination of $x$, $y$, and $z$.
\begin{equation*}
   (1/4)x+y-z
\end{equation*}
\end{frame}


%...........................
\begin{frame}
\df[def:linearcombination]
\ExecuteMetaData[../gr1.tex]{df:linearequations}
\ex 
There are three linear equations in this linear system.
\begin{equation*}
  \begin{linsys}{3}
   (1/4)x  &+  &y  &-  &z  &=  &0  \\
        x  &+  &4y &+  &2z &=  &12  \\
       2x  &-  &3y &-  &z  &=  &3  
  \end{linsys}  
\end{equation*}
\end{frame}



%...........................
\begin{frame}{Solving a linear system}
\ex 
To find the solution of this system
\begin{equation*}
  \begin{linsys}{3}
   (1/4)x  &+  &y  &-  &z  &=  &0  \\
        x  &+  &4y &+  &2z &=  &12  \\
       2x  &-  &3y &-  &z  &=  &3  
  \end{linsys}  
\end{equation*}
we transform it to one whose solution is easy.
\pause Start by clearing the fraction.
\begin{eqnarray*}
  &\grstep{4\rho_1}
  &\begin{linsys}{3}
        x  &+  &4y &-  &4z  &=  &0  \\
        x  &+  &4y &+  &2z &=  &12  \\
       2x  &-  &3y &-  &z  &=  &3  
  \end{linsys}  
\end{eqnarray*}
\pause Next use the first row to act on the rows below, 
eliminating their $x$~terms.
\begin{eqnarray*}
  &\grstep[-2\rho_1+\rho_3]{-1\rho_1+\rho_2}
  &\begin{linsys}{3}
        x  &+  &4y   &-  &4z  &=  &0  \\
           &   &     &+  &6z  &=  &12  \\
           &   &-11y &+  &7z  &=  &3  
  \end{linsys} 
\end{eqnarray*} 
\end{frame}\begin{frame}
Then swap to bring a $y$ term to the second row.
\begin{eqnarray*}
  &\grstep{\rho_2\leftrightarrow\rho_3}
  &\begin{linsys}{3}
        x  &+  &4y   &-  &4z  &=  &0  \\
           &   &-11y &+  &7z  &=  &3  \\
           &   &     &   &6z  &=  &12  
  \end{linsys}  
\end{eqnarray*}
\pause Now solve the bottom row:
$z=2$.
\pause
With that, the shape of the transformed system 
lets us solve for $y$ by substituting into the second row:
$-11y+7(2)=3$ shows $y=1$.
\pause
The shape also lets us solve for $x$ by substituting into the
first row: $x+4(1)-4(2)=0$, so that $x=4$. 

\pause
\df[df:EchelonForm]
\ExecuteMetaData[../gr1.tex]{df:EchelonForm}
\end{frame}



%...........................
\begin{frame}
\ex 
\begin{eqnarray*}
  \begin{linsys}{4}
    2x  &-  &3y  &-  &z  &+ &2w &=  &-2  \\
     x  &   &    &+  &3z &+ &1w &=  &6  \\
    2x  &-  &3y  &-  &z  &+ &3w &=  &-3  \\
        &   &y   &+  &z  &- &2w &=  &4  
  \end{linsys}  
  &\grstep[-\rho_1+\rho_3]{(-1/2)\rho_1+\rho_2}
  &\begin{linsys}{4}
    2x  &-  &3y      &-  &z      &+  &2w &=  &-2  \\
        &   &(3/2)y  &+  &(7/2)z &   &   &=  &7  \\
        &   &        &   &       &   &w  &=   &-1  \\
        &   &y       &+  &z      &-  &2w  &=  &4  
  \end{linsys}                                       \\[1.5ex]  \pause
  &\grstep{(-2/3)\rho_2+\rho_4}
  &\begin{linsys}{4}
    2x  &-  &3y      &-  &z      &+  &2w &=  &-2  \\
        &   &(3/2)y  &+  &(7/2)z &   &   &=  &7  \\
        &   &        &   &       &   &w  &=   &-1  \\
        &   &        &-  &(4/3)z &- &2w  &=  &-2/3  
  \end{linsys}                                         \\[1.5ex]  \pause
  &\grstep{\rho_3\leftrightarrow\rho_4}
  &\begin{linsys}{4}
    2x  &-  &3y      &-  &z      &+  &2w &=  &-2  \\
        &   &(3/2)y  &+  &(7/2)z &   &   &=  &7  \\
        &   &        &-  &(4/3)z &- &2w  &=  &-2/3 \\ 
        &   &        &   &       &   &w  &=   &-1  
  \end{linsys}  
\end{eqnarray*}
\pause
The fourth equation says $w=-1$.
Substituting back into the third equation gives $z=2$.
Then back substitution into the second and first rows gives
$y=0$ and $x=1$.
The unique solution is $(1,0,2,-1)$.
\end{frame}



%...........................
\begin{frame}{Gauss's Method}
\th[th:GaussMethod]
\ExecuteMetaData[../gr1.tex]{th:GaussMethod}

\df[df:GaussMethod]
\ExecuteMetaData[../gr1.tex]{df:GaussMethod}
\end{frame}



%...........................
\begin{frame}
\pf[th:GaussMethod]
We verify the result for operation~(1).
The other two are similar.

\ExecuteMetaData[../gr1.tex]{pf:GaussMethod0}
\end{frame}
\begin{frame}
\ExecuteMetaData[../gr1.tex]{pf:GaussMethod1}\qed
\end{frame}



%...........................
\begin{frame}{Systems without a unique solution}
\ex
This system has no solution.
\begin{equation*}
  \begin{linsys}{3}
        x  &+  &y  &+  &z  &=  &6  \\
        x  &+  &2y &+  &z  &=  &8  \\
       2x  &+  &3y &+  &2z &=  &13  
  \end{linsys}    
\end{equation*}
On the left side of the equals sign
the sum of the first two rows equals the third row,
while on the right that is not so.
So there is no triple of reals that makes all three 
equations true.

\pause
Gauss' Method makes the inconsistency clear.
\begin{equation*}
  \grstep[-2\rho_1+\rho_3]{-\rho_1+\rho_2}\;
  \begin{linsys}{3}
        x  &+  &y  &+  &z  &=  &6  \\
           &   &y  &   &   &=  &2  \\
           &   &y  &   &   &=  &1  
  \end{linsys}    
  \;\grstep{-\rho_2+\rho_3}\;
  \begin{linsys}{3}
        x  &+  &y  &+  &z  &=  &6  \\
           &   &y  &   &   &=  &2  \\
           &   &   &   &0  &=  &-1  
  \end{linsys}    
\end{equation*}
\end{frame}



%...........................
\begin{frame}
\ex
This system has infinitely many solutions.
\begin{equation*}
  \begin{linsys}{3}
        x  &-  &y  &+  &z  &=  &4  \\
        x  &+  &y  &-  &2z &=  &-1   
  \end{linsys} 
  \;\grstep{-\rho_1+\rho_2}\;   
  \begin{linsys}{3}
        x  &-  &y  &+  &z  &=  &4  \\
           &   &2y &-  &3z &=  &-5   
  \end{linsys} 
\end{equation*}
Taking $z=0$ gives the solution $(3/2,-5/2,0)$.
Taking $z=-1$ gives $(1,-4,-1)$.

\pause
\ex
Another system with infinitely many solutions.
\begin{eqnarray*}
  \begin{linsys}{3}
        -x   &-  &y  &+  &3z  &=  &3  \\
         x   &   &   &+  &z   &=  &3  \\
        3x   &-  &y  &+  &7z  &=  &15   
  \end{linsys} 
  &\grstep[3\rho_1+\rho_3]{-\rho_1+\rho_2}   
  &\begin{linsys}{3}
        -x   &-  &y   &+  &3z  &=  &3  \\
             &   &-y  &+  &4z  &=  &6  \\
             &   &-4y &+  &16z &=  &24   
  \end{linsys}                                \\
  &\grstep{-4\rho_2+\rho_3}
  &\begin{linsys}{3}
        -x   &-  &y  &+  &3z  &=  &3  \\
             &   &-y  &+  &4z  &=  &6  \\
             &   &   &   &0    &=  &0   
  \end{linsys} 
\end{eqnarray*}
Taking $z=0$ gives $(3,-6,0)$ while
taking $z=1$ gives $(2,-2,1)$.
\end{frame}










% ..... One.I.2 .....
\section{Describing the solution set}
\begin{frame}{Parametrizing} 
We've seen that this system has infinitely many solutions.
\begin{equation*}
  \begin{linsys}{3}
        -x   &-  &y  &+  &3z  &=  &3  \\
         x   &   &   &+  &z   &=  &3  \\
        3x   &-  &y  &+  &7z  &=  &15   
  \end{linsys} 
  \;\grstep[3\rho_1+\rho_3]{-\rho_1+\rho_2}   
  \;\grstep{-4\rho_2+\rho_3}\;
  \begin{linsys}{3}
        -x   &-  &y  &+  &3z  &=  &3  \\
             &   &-y  &+  &4z  &=  &6  \\
             &   &   &   &0    &=  &0   
  \end{linsys} 
\end{equation*}
We want to describe the solution set.

\pause
Use the second row to express $y$ in terms of~$z$ as 
$y=-6+4z$. 
\pause 
Now substitute into the first row $-x-(-6+4z)+3z=3$
to 
express $x$ also in terms of~$z$ with
$x=3-z$.

\pause
\df[df:FreeVars]
\ExecuteMetaData[../gr1.tex]{df:FreeVars}

\ExecuteMetaData[../gr1.tex]{df:Parameter}

We shall routinely parametrize linear systems using the free variables.
\end{frame}




% ..........
\begin{frame}
\ex
\begin{eqnarray*}
  \begin{linsys}{4}
         x   &-  &y  &+  &2z  &+  &3w &=  &14  \\
        2x   &-  &2y &-  &z   &+  &2w &=  &6  \\
             &   &   &-  &3z  &+  &2w &=  &0   
  \end{linsys} 
  &\grstep{-2\rho_1+\rho_2}
  &\begin{linsys}{4}
         x   &-  &y  &+  &2z  &+  &3w &=  &14  \\
             &   &   &-  &5z  &-  &4w &=  &-22  \\
             &   &   &-  &3z  &+  &2w &=  &0   
  \end{linsys}                                   \\
  &\grstep{-(3/5)\rho_2+\rho_3}
  &\begin{linsys}{4}
         x   &-  &y  &+  &2z  &+  &3w      &=  &14  \\
             &   &   &-  &5z  &-  &4w      &=  &-22  \\
             &   &   &   &    &   &(22/5)w &=  &66/5   
  \end{linsys}                                  
\end{eqnarray*}
The leading variables are $x$, $z$, and $w$. 
We will parametrize with the free variable $y$.

\pause
The bottom row gives $w=3$ and substituting that into the next
row up gives $z=2$.
The top equation is $x-y+2\cdot 2+3\cdot 3=14$ so 
we have $x=1-y$.
\end{frame}




% ..........
\begin{frame}
\ex
This system is already in echelon form.
\begin{equation*}
  \begin{linsys}{4}
   -2x  &+  &y  &-  &z   &+   &w  &= &3/2  \\
        &   &   &   &2z  &-   &w  &= &1/2 
  \end{linsys} 
\end{equation*}
The leading variables are $x$ and $z$ so we will 
parametrize the solution set with $y$ and $w$.

\pause
The second row gives $z=1/4+(1/2)w$.
\pause
Substituting back into the first row gives
$-2x+y-((1/4)+(1/2)w)+w=3/2$,
and solving for $x$ leaves
$x=-(7/8)+(1/2)y+(1/4)w$.
\end{frame}



% ..........
\begin{frame}{Matrices and vectors}
\df[df:matrix]
\ExecuteMetaData[../gr1.tex]{df:matrix}

\pause
\ex
This is a $\nbym{2}{3}$ matrix
\begin{equation*}
  B=
  \begin{mat}[r]
    1  &-2  &3  \\
    4  &-5  &6
  \end{mat}
\end{equation*}
because it has $2$~rows and $3$~columns.
The entry in row~$2$ and column~$1$ is
\( b_{2,1}=4 \).

\pause
\df[df:vector]
\ExecuteMetaData[../gr1.tex]{df:vector}

\pause
We denote a vector with an over-arrow 
(many authors use boldface).
\ex
This column vector has three components.
\begin{equation*}
  \vec{v}=\colvec{-1  \\ -0.5  \\ 0}
\end{equation*}
\end{frame}

\begin{frame}
\ex
This row vector has three components
\begin{equation*}
  \vec{w}=\rowvec{-1  & -0.5  & 0}
\end{equation*}

\ex
This is
the two-component zero vector.
\begin{equation*}
  \zero=\colvec{0 \\ 0}
\end{equation*}
\end{frame}



% ..........
\begin{frame}{Vector operations}
\df[df:VectorSum]
\ExecuteMetaData[../gr1.tex]{df:VectorSum}
\df[df:VectorScalarMultiplication]
\ExecuteMetaData[../gr1.tex]{df:VectorScalarMultiplication}
\ex
\begin{equation*}
  3\colvec{1 \\2}-2\colvec{0 \\ 1}=\colvec{3 \\ 4}
\end{equation*}
\end{frame}


% ..........
\begin{frame}{Matrix notation for linear systems}
\ex
We can simplify the clerical load in reducing this system
\begin{equation*}
  \begin{linsys}{3}
       -3x   &   &   &+  &2z  &=  &-1  \\
         x   &-  &2y &+  &2z  &=  &-5/3  \\
        -x   &-  &4y &+  &6z  &=  &-13/3   
  \end{linsys} 
\end{equation*}
by writing it as an \definend{augmented matrix}.
\begin{eqnarray*}
    \begin{amat}[r]{3}
     -3  &0  &2  &-1  \\
      1  &-2 &2  &-5/3  \\
     -1  &-4 &6  &-13/3
    \end{amat}
  &\grstep[-(1/3)\rho_1+\rho_3]{(1/3)\rho_1+\rho_2}
  &\begin{amat}[r]{3}
     -3  &0  &2    &-1  \\
      0  &-2 &8/3  &-2  \\
      0  &-4 &16/3 &-4
    \end{amat}                      \\
  &\grstep{-2\rho_2+\rho_3}
  &\begin{amat}[r]{3}
     -3  &0  &2    &-1  \\
      0  &-2 &8/3  &-2  \\
      0  &0  &0    &0
    \end{amat}  
\end{eqnarray*}
The two nonzero rows give
$-3x+2z=-1$ and $-2y+(8/3)z=-2$.
\end{frame}



% ..........
\begin{frame}
Parametrizing
$-3x+2z=-1$ and $-2y+(8/3)z=-2$
gives 
$y=1+(4/3)z$ and $x=(1/3)+(2/3)z$.

\pause
We can write the solution set in vector notation.
\begin{equation*}
  \set{\colvec{x \\ y \\ z}=\colvec{1/3 \\ 1 \\ 0}+\colvec{2/3 \\ 4/3 \\ 1}z 
                 \suchthat z\in\Re}
\end{equation*}
\end{frame}



% ..........
\begin{frame}
\ex
Reducing this system
\begin{equation*}
  \begin{linsys}{4}
    x &+  &2y  &- &z  &  &  &= &2 \\
   2x &-  &y   &- &2z &+ &w &= &5
  \end{linsys}
\end{equation*}
using the augmented matrix notation
\begin{eqnarray*}
    \begin{amat}[r]{4}
      1  &2  &-1  &0  &2  \\
      2  &-1 &-2  &1  &5  
    \end{amat}
  &\grstep{-2\rho_1+\rho_2}
  &\begin{amat}[r]{4}
      1  &2  &-1  &0  &2  \\
      0  &-5 &0   &1  &1  
    \end{amat}
\end{eqnarray*}
gives this vector description of the solution set.
\begin{equation*}
  \set{\colvec{12/5 \\ -1/5 \\ 0 \\ 0}
       +\colvec{1 \\ 0 \\ 1 \\ 0}z
       +\colvec{-2/5 \\ 1/5 \\ 0 \\ 1}w
      \suchthat z,w\in\Re}
\end{equation*}
\end{frame}




% ..... One.I.3 .....
\section{\texorpdfstring{$\text{General}=\text{Particular}+\text{Homogeneous}$}{General=Particular+Homogeneous}}

\begin{frame}
\frametitle{Form of solution sets} 
\ex
This system
\begin{equation*}
  \begin{linsys}{4}
    x &+  &2y  &- &z  &  &  &= &2 \\
   2x &-  &y   &- &2z &+ &w &= &5
  \end{linsys}
\end{equation*}
has solutions of this form. 
\begin{equation*}
     \colvec{x  \\  y  \\  z  \\  w}
     =
     \colvec{12/5 \\ -1/5 \\ 0 \\ 0}
       +\colvec{1 \\ 0 \\ 1 \\ 0}z
       +\colvec{-2/5 \\ 1/5 \\ 0 \\ 1}w
   \qquad
   z,w\in\Re
\end{equation*}
Taking $z=w=0$ shows that the first vector is a particular solution of the
system.
% \pause
% We will show that every solution has this form.
% \begin{equation*}
% \underbracket[.7pt]{
%   \colvec{12/5 \\ -1/5 \\ 0 \\ 0}
%   }_{\text{\shortstack{\rule{0pt}{2ex}particular \\  solution}}}
%   +\underbracket[.7pt]{
%        \colvec{1 \\ 0 \\ 1 \\ 0}z
%        +\colvec{-2/5 \\ 1/5 \\ 0 \\ 1}w
%   }_{\text{\shortstack{\rule{0pt}{2ex}homogeneous \\  solution}}}
% \end{equation*}
\end{frame}



% ..........
\begin{frame}
\df[df:HomogeneousEquation]
\ExecuteMetaData[../gr1.tex]{df:HomogeneousEquation}

\ex
Consider the system of homogeneous equations derived from the above system
by changing the constants to $0$'s.
\begin{equation*}
  \begin{linsys}{4}
    x &+  &2y  &- &z  &  &  &= &0 \\
   2x &-  &y   &- &2z &+ &w &= &0
  \end{linsys}
\end{equation*}
The same Gauss's Method steps reduce it to echelon form.
\begin{equation*}
    \begin{amat}[r]{4}
      1  &2  &-1  &0  &0  \\
      2  &-1 &-2  &1  &0  
    \end{amat}
  \;\grstep{-2\rho_1+\rho_2}\;
  \begin{amat}[r]{4}
      1  &2  &-1  &0  &0  \\
      0  &-5 &0   &1  &0  
    \end{amat}
\end{equation*}
The vector description of the solution set
\begin{equation*}
  \set{
       \colvec{1 \\ 0 \\ 1 \\ 0}z
       +\colvec{-2/5 \\ 1/5 \\ 0 \\ 1}w
      \suchthat z,w\in\Re}
\end{equation*}
is the same as earlier but with a particular solution that is the
zero vector.
\end{frame}



% ..........
\begin{frame}
\lm[le:HomoSltnSpanVecs]
\ExecuteMetaData[../gr1.tex]{le:HomoSltnSpanVecs}
\ex
Consider this system of homogeneous equations.
\begin{equation*}
  \begin{linsys}{5}
     x  &+  &y   &+  &z  &+  &w  &=  &0  \\
        &   &y   &-  &z  &+  &w  &=  &0  
  \end{linsys}
\end{equation*}
With the bottom equation express the leading variable $y$ in terms of the
free variables 
$y=z-w$.
Move up to the equation above, substitute 
$x+(z-w)+z+w=0$, and solve for the leading variable
$x=-2z$.
\pause
To finish write the solution in vector notation.
\begin{equation*}
  \colvec{x \\ y \\ z \\ w}
    =\colvec{-2 \\ 1 \\ 1 \\ 0}z
     +\colvec{0 \\ -1 \\ 0 \\ 1}w
    \qquad z,w\in\Re
\end{equation*}
and recognize $\vec{\beta}_1$ and $\vec{\beta}_2$ as the vectors associated
with $z$ and~$w$.
\end{frame}



% ..........
\begin{frame}
\pf[le:HomoSltnSpanVecs]
\ExecuteMetaData[../gr1.tex]{pf:HomoSltnSpanVecs0}

\pause
\ExecuteMetaData[../gr1.tex]{pf:HomoSltnSpanVecs1}
\pause
\ExecuteMetaData[../gr1.tex]{pf:HomoSltnSpanVecs2}
\end{frame}\begin{frame}
\ExecuteMetaData[../gr1.tex]{pf:HomoSltnSpanVecs3}\qed
\end{frame}



% ..........
\begin{frame}
\lm[th:GenEqPartHomo]
\ExecuteMetaData[../gr1.tex]{th:GenEqPartHomo}
\pause
\pf[th:GenEqPartHomo]
\ExecuteMetaData[../gr1.tex]{pf:GenEqPartHomo0}
\end{frame}\begin{frame}
\ExecuteMetaData[../gr1.tex]{pf:GenEqPartHomo1}
\qed
\end{frame}



% ..........
\begin{frame}
\th[th:GenEqPartPlusHomo]
\ExecuteMetaData[../gr1.tex]{th:GenEqPartPlusHomo}
\pf
This restates the prior two lemmas.\qed
\end{frame}



% ..........
\begin{frame}
\co[co:ThreeKindsSolutionSets]
\ExecuteMetaData[../gr1.tex]{co:ThreeKindsSolutionSets}
\pause
\pf[co:ThreeKindsSolutionSets]
\ExecuteMetaData[../gr1.tex]{pf:ThreeKindsSolutionSets0}

\pause
\ExecuteMetaData[../gr1.tex]{pf:ThreeKindsSolutionSets1}
\qed
\end{frame}



% ..........
\begin{frame}{Summary: Kinds of Solution Sets}
\ExecuteMetaData[../gr1.tex]{table:KindsSolutionSets}
\pause
An important special case is when there are the same number of
equations as unknowns.

\df[df:Nonsingular]
\ExecuteMetaData[../gr1.tex]{df:Nonsingular}
\end{frame}





%...........................
% \begin{frame}
% 
% \end{frame}
\end{document}
