% see: https://groups.google.com/forum/?fromgroups#!topic/comp.text.tex/s6z9Ult_zds
\makeatletter\let\ifGm@compatii\relax\makeatother 
\documentclass[10pt,t,serif,professionalfont]{beamer}
\PassOptionsToPackage{pdfpagemode=FullScreen}{hyperref}
\PassOptionsToPackage{usenames,dvipsnames}{color}
% \DeclareGraphicsRule{*}{mps}{*}{}
\usepackage{../linalgjh}
\usepackage{present}
\usepackage{xr}\externaldocument{../vs3} % read refs from .aux file
\usepackage{catchfilebetweentags}
\usepackage{etoolbox} % from http://tex.stackexchange.com/questions/40699/input-only-part-of-a-file-using-catchfilebetweentags-package
\makeatletter
\patchcmd{\CatchFBT@Fin@l}{\endlinechar\m@ne}{}
  {}{\typeout{Unsuccessful patch!}}
\makeatother

\mode<presentation>
{
  \usetheme{boxes}
  \setbeamercovered{invisible}
  \setbeamertemplate{navigation symbols}{} 
}
\addheadbox{filler}{\ }  % create extra space at top of slide 
\hypersetup{colorlinks=true,linkcolor=blue} 

\title[Basis and Dimension] % (optional, use only with long paper titles)
{Two.III Basis and Dimension}

\author{\textit{Linear Algebra} \\ {\small Jim Hef{}feron}}
\institute{
  \texttt{http://joshua.smcvt.edu/linearalgebra}
}
\date{}


\subject{Basis and Dimension}
% This is only inserted into the PDF information catalog. Can be left
% out. 

\begin{document}
\begin{frame}
  \titlepage
\end{frame}

% =============================================
% \begin{frame}{Reduced Echelon Form} 
% \end{frame}



% ..... Two.III.1 .....
\section{Basis}
%..........
\begin{frame}{Definition of basis}
\df[def:basis]
\ExecuteMetaData[../vs3.tex]{df:basis}

\pause\medskip
\ExecuteMetaData[../vs3.tex]{BasisNotation}

\pause
\ex
This is a basis for $\Re^2$.
\begin{equation*}
  \sequence{\colvec{1 \\ -1},
            \colvec{1 \\ 1}
           }
\end{equation*}
It is linearly independent.
\begin{equation*}
  c_1\colvec{1 \\ -1}+c_2\colvec{1 \\ 1}=\colvec{0 \\ 0}
  \implies
  \begin{linsys}{2}
    c_1 &+ &c_2 &= &0  \\
   -c_1 &+ &c_2 &= &0
  \end{linsys}
  \implies
  c_1=0,\,c_2=0
\end{equation*}
And it spans $\Re^2$ since
\begin{equation*}
  c_1\colvec{1 \\ -1}+c_2\colvec{1 \\ 1}=\colvec{x \\ y}
  \implies
  \begin{linsys}{2}
    c_1 &+ &c_2 &= &x  \\
   -c_1 &+ &c_2 &= &y
  \end{linsys}
\end{equation*}
has the solution $c_1=(1/2)x-(1/2)y$
and $c_2=(1/2)x+(1/2)y$.
\end{frame}



%..........
\begin{frame}
\th[th:BasisIffUniqueRepWRT]
\ExecuteMetaData[../vs3.tex]{th:BasisIffUniqueRepWRT}

\pause
\pf
\ExecuteMetaData[../vs3.tex]{pf:BasisIffUniqueRepWRT0}
\end{frame}
\begin{frame}
\ExecuteMetaData[../vs3.tex]{pf:BasisIffUniqueRepWRT1}
\qed
\end{frame}



%..........
\begin{frame}
\df[def:RepresentingVectors]
\ExecuteMetaData[../vs3.tex]{df:RepresentingVectors}
\end{frame}



\section{Dimension}
%..........
\begin{frame}{Definition of dimension}
\df[df:FiniteDimensional]
\ExecuteMetaData[../vs3.tex]{df:FiniteDimensional}
\end{frame}



%..........
\begin{frame}{Exchange Lemma}
\lm[lm:ExchangeLemma]
\ExecuteMetaData[../vs3.tex]{lm:ExchangeLemma}

\pause
\pf
\ExecuteMetaData[../vs3.tex]{pf:ExchangeLemma0}
\end{frame}
\begin{frame}
\ExecuteMetaData[../vs3.tex]{pf:ExchangeLemma1}
\qed
\end{frame}



%..........
\begin{frame}{All of a spaces bases are the same size}
\th[th:AllBasesSameSize]
\ExecuteMetaData[../vs3.tex]{th:AllBasesSameSize}

\pause
\pf
\ExecuteMetaData[../vs3.tex]{pf:AllBasesSameSize0}

\pause
\ExecuteMetaData[../vs3.tex]{pf:AllBasesSameSize1}
\end{frame}
\begin{frame}
\ExecuteMetaData[../vs3.tex]{pf:AllBasesSameSize2}

\pause
\ExecuteMetaData[../vs3.tex]{pf:AllBasesSameSize3}
\qed
\end{frame}



%..........
\begin{frame}{Definition of dimension}
\df[df:Dimension]
\ExecuteMetaData[../vs3.tex]{df:Dimension}
\end{frame}



%..........
\begin{frame}
\co[cor:NoLiSetGreatDim]
\ExecuteMetaData[../vs3.tex]{co:NoLiSetGreatDim}

\pause
\pf
\ExecuteMetaData[../vs3.tex]{pf:NoLiSetGreatDim}
\qed

\pause
\medskip
\co[cor:LIExpBas]
\ExecuteMetaData[../vs3.tex]{co:LIExpBas}

\pause
\pf
\ExecuteMetaData[../vs3.tex]{pf:LIExpBas}
\qed
\end{frame}



%..........
\begin{frame}
\co[co:SpanningSetShrinksToABasis]
\ExecuteMetaData[../vs3.tex]{co:SpanningSetShrinksToABasis}

\pause
\pf
\ExecuteMetaData[../vs3.tex]{pf:SpanningSetShrinksToABasis}
\qed
\end{frame}



%..........
\begin{frame}
\co[cor:NVecsRNSpanIffLI]
\ExecuteMetaData[../vs3.tex]{co:NVecsRNSpanIffLI}

\pause
\pf
\ExecuteMetaData[../vs3.tex]{pf:NVecsRNSpanIffLI}
\qed
\end{frame}



%...........................
% \begin{frame}
% \ExecuteMetaData[../gr3.tex]{GaussJordanReduction}
% \df[def:RedEchForm]
% 
% \end{frame}
\end{document}
