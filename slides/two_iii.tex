% see: https://groups.google.com/forum/?fromgroups#!topic/comp.text.tex/s6z9Ult_zds
\makeatletter\let\ifGm@compatii\relax\makeatother 
\documentclass[10pt,t]{beamer}
\usefonttheme{professionalfonts}
\usefonttheme{serif}
\PassOptionsToPackage{pdfpagemode=FullScreen}{hyperref}
\PassOptionsToPackage{usenames,dvipsnames}{color}
% \DeclareGraphicsRule{*}{mps}{*}{}
\usepackage{../linalgjh}
\usepackage{present}
\usepackage{xr}\externaldocument{../vs3} % read refs from .aux file
\usepackage{xr}\externaldocument{../vs2} % read refs from .aux file
\usepackage{xr}\externaldocument{../gr1} % read refs from .aux file
\usepackage{xr}\externaldocument{../gr3} % read refs from .aux file
\usepackage{catchfilebetweentags}
\usepackage{etoolbox} % from http://tex.stackexchange.com/questions/40699/input-only-part-of-a-file-using-catchfilebetweentags-package
\makeatletter
\patchcmd{\CatchFBT@Fin@l}{\endlinechar\m@ne}{}
  {}{\typeout{Unsuccessful patch!}}
\makeatother

\mode<presentation>
{
  \usetheme{boxes}
  \setbeamercovered{invisible}
  \setbeamertemplate{navigation symbols}{} 
}
\addheadbox{filler}{\ }  % create extra space at top of slide 
\hypersetup{colorlinks=true,linkcolor=blue} 

\title[Basis and Dimension] % (optional, use only with long paper titles)
{Two.III Basis and Dimension}

\author[Jim Hefferon]{\textit{Linear Algebra} \\ {\small Jim Hef{}feron}}
\institute{
  \texttt{http://joshua.smcvt.edu/linearalgebra}
}
\date{}


\subject{Basis and Dimension}
% This is only inserted into the PDF information catalog. Can be left
% out. 

\begin{document}
\begin{frame}
  \titlepage
\end{frame}

% =============================================
% \begin{frame}{Reduced Echelon Form} 
% \end{frame}



% ..... Two.III.1 .....
\section{Basis}
%..........
\begin{frame}{Definition of basis}
\df[def:basis]
\ExecuteMetaData[../vs3.tex]{df:basis}

\pause\medskip
\ExecuteMetaData[../vs3.tex]{BasisNotation}

\pause
\ex
This is a basis for $\Re^2$.
\begin{equation*}
  \sequence{\colvec{1 \\ -1},
            \colvec{1 \\ 1}
           }
\end{equation*}
It is linearly independent.
\begin{equation*}
  c_1\colvec{1 \\ -1}+c_2\colvec{1 \\ 1}=\colvec{0 \\ 0}
  \implies
  \begin{linsys}{2}
    c_1 &+ &c_2 &= &0  \\
   -c_1 &+ &c_2 &= &0
  \end{linsys}
  \implies
  c_1=0,\,c_2=0
\end{equation*}
And it spans $\Re^2$ since
\begin{equation*}
  c_1\colvec{1 \\ -1}+c_2\colvec{1 \\ 1}=\colvec{x \\ y}
  \implies
  \begin{linsys}{2}
    c_1 &+ &c_2 &= &x  \\
   -c_1 &+ &c_2 &= &y
  \end{linsys}
\end{equation*}
has the solution $c_1=(1/2)x-(1/2)y$
and $c_2=(1/2)x+(1/2)y$.
\end{frame}


%..........
\begin{frame}
\ex
In the vector space of linear polynomials 
$\polyspace_1=\set{a+bx\suchthat a,b\in\Re}$
one basis is $B=\sequence{1+x,1-x}$.

Check that is a basis by verifying that it is
linearly independent
\begin{equation*}
  0=c_1(1+x)+c_2(1-x)
  \implies
  0=c_1+c_2,\;0=c_1-c_2
  \implies 
  c_1=c_2=0
\end{equation*}
and that it spans the space.
\begin{equation*}
  a+bx=c_1(1+x)+c_2(1-x)
  % \implies
  % a=c_1+c_2,\;b=c_1-c_2
  \implies 
  c_1=(a+b)/2,\;c_2=(a-b)/2
\end{equation*}
\pause
\ex
This is a basis for $\matspace_{\nbyn{2}}$.
\begin{equation*}
  \sequence{
    \begin{mat}
      1  &0  \\
      0  &0 
    \end{mat},
    \begin{mat}
      0  &2  \\
      0  &0 
    \end{mat},
    \begin{mat}
      0  &0  \\
      3  &0 
    \end{mat},
    \begin{mat}
      0  &0  \\
      0  &4 
    \end{mat}
  }
\end{equation*}
This is another one.
\begin{equation*}
  \sequence{
    \begin{mat}
      1  &0  \\
      0  &0 
    \end{mat},
    \begin{mat}
      1  &2  \\
      0  &0 
    \end{mat},
    \begin{mat}
      1  &2  \\
      3  &0 
    \end{mat},
    \begin{mat}
      1  &2  \\
      3  &4 
    \end{mat}
  }
\end{equation*}
\end{frame}
\begin{frame}
\ex
This is a basis for $\Re^3$.
\begin{equation*}
  \stdbasis_3=
  \sequence{
            \colvec{1 \\ 0 \\ 0},
            \colvec{0 \\ 1 \\ 0},
            \colvec{0 \\ 0 \\ 1}
            }
\end{equation*}
Calculus books sometimes call those 
$\vec{\imath}$, $\vec{\jmath}$, and $\vec{k}$.

\pause
\df[df:StandardBasis]
\ExecuteMetaData[../vs3.tex]{df:StandardBasis}

\medskip\noindent
Checking that $\stdbasis_n$ is a basis for $\Re^n$ is routine.
\end{frame}



%..........
\begin{frame}
Although a basis is a sequence we will follow the  
common practice 
and refer to it as a set.

\th[th:BasisIffUniqueRepWRT]
\ExecuteMetaData[../vs3.tex]{th:BasisIffUniqueRepWRT}

\pause
\pf
\ExecuteMetaData[../vs3.tex]{pf:BasisIffUniqueRepWRT0}
\end{frame}
\begin{frame}
\ExecuteMetaData[../vs3.tex]{pf:BasisIffUniqueRepWRT1}
\qed
\end{frame}



%..........
\begin{frame}
\df[def:RepresentingVectors]
\ExecuteMetaData[../vs3.tex]{df:RepresentingVectors}
\end{frame}




%..........
\begin{frame}
\ex
Above we saw that in  
$\polyspace_1=\set{a+bx\suchthat a,b\in\Re}$
one basis is $B=\sequence{1+x,1-x}$.
As part of that we computed the coefficients needed to 
express a member of $\polyspace_1$ as a combination of
basis vectors.
\begin{equation*}
  a+bx=c_1(1+x)+c_2(1-x)
  % \implies
  % a=c_1+c_2,\;b=c_1-c_2
  \implies 
  c_1=(a+b)/2,\;c_2=(a-b)/2
\end{equation*}
\pause
For instance, the polynomial $3+4x$ has this expression
\begin{equation*}
  3+4x=(7/2)\cdot(1+x)+(-1/2)\cdot(1-x)
\end{equation*}
so its representation is this.
\begin{equation*}
  \rep{3+4x}{B}=\colvec{7/2 \\ -1/2}
\end{equation*}
\end{frame}
\begin{frame}
\ex 
With respect to $\Re^3$'s standard basis $\stdbasis_3$ the vector
\begin{equation*}
  \vec{v}
  =\colvec{2 \\ -3 \\ 1/2}
\end{equation*}
has this representation.
\begin{equation*}
  \rep{\vec{v}}{\stdbasis_3}=\colvec{2 \\ -3 \\ 1/2}
\end{equation*}
In general, any $\vec{w}\in\Re^n$ 
has $\rep{\vec{w}}{\stdbasis_n}=\vec{w}$.
\end{frame}




\section{Dimension}
%..........
\begin{frame}{Definition of dimension}
\df[df:FiniteDimensional]
\ExecuteMetaData[../vs3.tex]{df:FiniteDimensional}

\ex
The space 
$\Re^3$
is finite-dimensional since it has a basis with three elements $\stdbasis_3$.

\ex
The space of quadratic polynomials $\polyspace_2$ has at least one
basis with finitely many elements, $\sequence{1,x,x^2}$, so it is
finite-dimensional. 

\ex
The space $\matspace_{\nbyn{2}}$ of $\nbyn{2}$ matrices is finite-dimensional.
Here is one basis with finitely many members.
\begin{equation*}
  \sequence{
    \begin{mat}
      1 &0 \\
      0 &0
    \end{mat},
    \begin{mat}
      1 &1 \\
      0 &0
    \end{mat},
    \begin{mat}
      1 &1 \\
      1 &0
    \end{mat},
    \begin{mat}
      1 &1 \\
      1 &1
    \end{mat}
        }
\end{equation*}

\pause
\no
From this point on we will restrict our attention to 
vector spaces that are finite-dimensional.
All the later examples, definitions, and theorems
assume this of the spaces.
\end{frame}




%..........
\begin{frame}
We will show that for any finite-dimensional space, all of its bases
have the same number of elements.

\ex
Each of these is a basis for $\polyspace_2$.
\begin{align*}
  &B_0=\sequence{1,1+x,1+x+x^2}       \\
  &B_1=\sequence{1+x+x^2,1+x,1}       \\
  &B_2=\sequence{x^2,1+x,1-x}         \\
  &B_3=\sequence{1,x,x^2}
\end{align*}
Each has two elements.

\pause
\ex
Here are two different bases for $\matspace_{\nbyn{2}}$. 
\begin{align*}
  &B_0=\sequence{
    \begin{mat}[r]
      1 &0 \\
      0 &0
    \end{mat},
    \begin{mat}[r]
      1 &1 \\
      0 &0
    \end{mat},
    \begin{mat}[r]
      1 &1 \\
      1 &0
    \end{mat},
    \begin{mat}[r]
      1 &1 \\
      1 &1
    \end{mat}
          }                 \\
  &B_1=\sequence{
    \begin{mat}[r]
      0 &0 \\
      0 &1
    \end{mat},
    \begin{mat}[r]
      0 &0 \\
      1 &0
    \end{mat},
    \begin{mat}[r]
      0 &1 \\
      0 &0
    \end{mat},
    \begin{mat}[r]
      1 &0 \\
      0 &0
    \end{mat}
          }                 
\end{align*}
Both have four elements.
\end{frame}



%..........
\begin{frame}{Exchange Lemma}
\ex
This is a basis for $\Re^3$.
\begin{equation*}
  B=\sequence{\colvec{1 \\ 1 \\ 0},
              \colvec{1 \\ -1 \\0},
              \colvec{0 \\ 0 \\ 1}}
\end{equation*}
Here are the representations of two vectors with respect to $B$.
\begin{align*}
  \vec{v}_1
  &=\colvec{3 \\ 2 \\ 0}
  =(5/2)\cdot\colvec{1 \\ 1 \\ 0}
    +(1/2)\cdot\colvec{1 \\ -1 \\ 0}
    +0\cdot\colvec{0 \\ 0 \\ 1}               \\
  \vec{v}_2
  &=\colvec{3 \\ 0 \\ 2}
  =(3/2)\cdot\colvec{1 \\ 1 \\ 0}
    +(3/2)\cdot\colvec{1 \\ -1 \\ 0}
    +2\cdot\colvec{0 \\ 0 \\ 1}              
\end{align*}
For the first line, the coefficient of the third vector is $0$
so $\vec{v}_1$ has no part in the direction of the third vector.
In the second line, the coefficient of the third vector is nonzero
so $\vec{v}_2$ has some part in the direction of the third vector of $B$.
\end{frame}
\begin{frame}
Consider what results when we exchange this third vector from $B$ for
$\vec{v}_1$ and~$\vec{v}_2$.
\begin{equation*}
  B_1=\sequence{\colvec{1 \\ 1 \\ 0},
                \colvec{1 \\ -1 \\0},
                \colvec{3 \\ 2 \\ 0}}
  \qquad
  B_2=\sequence{\colvec{1 \\ 1 \\ 0},
                \colvec{1 \\ -1 \\0},
                \colvec{3 \\ 0 \\ 2}}
\end{equation*}
\pause
Because $\vec{v}_1$ does not involve the third element of~$B$\Dash
that is, because in the reprresentation the coefficient of the 
thrid vector is $0$\Dash
the set~$B_1$ is linearly dependent.
But the second set $B_2$ is linearly independent
because in the representation of $\vec{v}_2$ 
the coefficient of the third vector of~$B$ is not zero,
so $\vec{v}_2$ is not redundant on the other two members of $B_2$.
% See Lemma~II.\ref{lm:AddVecLiSetIsLiIffVecNotInSpan}.
\end{frame}

\begin{frame}
\lm[lm:ExchangeLemma]
\ExecuteMetaData[../vs3.tex]{lm:ExchangeLemma}

\pause
\pf
\ExecuteMetaData[../vs3.tex]{pf:ExchangeLemma0}
\end{frame}
\begin{frame}
\ExecuteMetaData[../vs3.tex]{pf:ExchangeLemma1}
\qed
\end{frame}



%..........
\begin{frame}{All of a space's bases are the same size}
\th[th:AllBasesSameSize]
\ExecuteMetaData[../vs3.tex]{th:AllBasesSameSize}

\ex
The idea of the proof is that, given two bases, exchange
members of the second for members of the first, which shows that they
have the same number of elements.
For an illustration, these are bases for $\polyspace_2$.
\begin{equation*}
  B=\sequence{1+x+x^2,1+x,1}
  \qquad
  D=\sequence{2,2x,2x^2}
\end{equation*}
\pause
For the first element of~$D$,
represent it with respect to~$B$, look for a nonzero coefficient, and exchange.
\begin{equation*}
  2=0\cdot(1+x+x^2)+0\cdot(1+x)+2\cdot(1)
  \quad B_1=\sequence{1+x+x^2,1+x,2}
\end{equation*}
\pause
Iterate (only exchanging for elements of $B$).
\begin{equation*}
  2x=0\cdot(1+x+x^2)+2\cdot(1+x)-1\cdot(2)
  \quad B_2=\sequence{1+x+x^2,2x,2}
\end{equation*}
The third exchange finishes it off.
\begin{equation*}
  2x^2=2\cdot(1+x+x^2)-1\cdot(2x)-1\cdot(2)
  \quad B_3=D
\end{equation*}
\end{frame}

\begin{frame}
\pf
\ExecuteMetaData[../vs3.tex]{pf:AllBasesSameSize0}

\pause
\ExecuteMetaData[../vs3.tex]{pf:AllBasesSameSize1}

\pause
\ExecuteMetaData[../vs3.tex]{pf:AllBasesSameSize2}
\end{frame}
\begin{frame}
\ExecuteMetaData[../vs3.tex]{pf:AllBasesSameSize3}
\qed
\end{frame}



%..........
\begin{frame}{Definition of dimension}
\df[df:Dimension]
\ExecuteMetaData[../vs3.tex]{df:Dimension}

\pause
\ex
The vector space $\Re^n$ has dimension~$n$ because that is how many members
are in $\stdbasis_n$.

\pause
\ex
The vector space $\polyspace_2$ has dimension~$3$ because one of 
its bases is $\sequence{1,x,x^2}$.
\pause
More generally, $\polyspace_n$ has dimension~$n+1$.

\pause
\ex
The vector space $\matspace_{\nbym{n}{m}}$ has dimension $n\cdot m$.
A natural basis consists of matrices with a single~$1$ and the other
entries~$0$'s.
\end{frame}
\begin{frame}
\ex
The solution set $S$ of this system
\begin{equation*}
  \begin{linsys}{4}
    x  &-  &y  &+  &z  &  &   &=  &0  \\
   -x  &+  &2y &-  &z  &+ &2w &=  &0  \\
   -x  &+  &3y &-  &z  &+ &4w &=  &0  
  \end{linsys}
\end{equation*}
is a vector space (this is easy to check for any homogeneous system).
Solving the system
\begin{equation*}
  \begin{amat}[r]{4}
    1  &-1  &1  &0  &0  \\
   -1  &2   &-1 &2  &0  \\
    1  &3   &-1 &4  &0  
  \end{amat}
  \grstep[\rho_1+\rho_3]{\rho_1+\rho_2}
  \grstep{-2\rho_2+\rho_3}
  \begin{amat}[r]{4}
    1  &-1  &1  &0  &0  \\
    0  &1   &0  &2  &0  \\
    0  &0   &0  &0  &0  
  \end{amat}
\end{equation*}
and parametrizing gives a basis of two vectors.
\begin{equation*}
  \set{\colvec{x \\ y \\ z \\ w}
      =\colvec{-1 \\ 0 \\ 1 \\ 0}\cdot z
       +\colvec{-2 \\ -2 \\ 0 \\ 1}\cdot w
      \suchthat z,w\in\Re}
  \qquad
  B=\sequence{\colvec{-1 \\ 0 \\ 1 \\ 0},
               \colvec{-2 \\ -2 \\ 0 \\ 1}}
\end{equation*}
So $S$ is a vector space of dimension two. 
% \begin{equation*}
%   B=\sequence{\colvec{-1 \\ 0 \\ 1 \\ 0},
%               \colvec{-2 \\ 1 \\ 0 \\ 1}
%              }
% \end{equation*}
\end{frame}


%..........
\begin{frame}
\co[cor:NoLiSetGreatDim]
\ExecuteMetaData[../vs3.tex]{co:NoLiSetGreatDim}

\pause
\pf
\ExecuteMetaData[../vs3.tex]{pf:NoLiSetGreatDim}
\qed

\pause\medskip
\re
This is an example of a result that assumes the vector spaces are
finite-dimensional without specifically saying so.
\end{frame}



%..........
\begin{frame}
\co[cor:LIExpBas]
\ExecuteMetaData[../vs3.tex]{co:LIExpBas}

\pf
\ExecuteMetaData[../vs3.tex]{pf:LIExpBas}
\qed

\pause
\co[co:SpanningSetShrinksToABasis]
\ExecuteMetaData[../vs3.tex]{co:SpanningSetShrinksToABasis}

\pf
\ExecuteMetaData[../vs3.tex]{pf:SpanningSetShrinksToABasis}
\qed
\end{frame}



%..........
\begin{frame}
\co[cor:NVecsRNSpanIffLI]
\ExecuteMetaData[../vs3.tex]{co:NVecsRNSpanIffLI}

\pause
\pf
\ExecuteMetaData[../vs3.tex]{pf:NVecsRNSpanIffLI}
\qed

\pause
\ex
This clearly spans the space.
\begin{equation*}
  \sequence{\colvec{1 \\ 1 \\ 1},
            \colvec{1 \\ 1 \\ 0},
            \colvec{1 \\ 0 \\ 0}}\subseteq\Re^3
\end{equation*}
Because it has same number of elements as the dimension of the space,
it is therefore a basis.
\end{frame}






% ..... Two.III.3 .....
\section{Vector Spaces and Linear Systems}
%..........
\begin{frame}{Row space}
\df[df:RowSpace]
\ExecuteMetaData[../vs3.tex]{df:RowSpace}

\pause
\lm[le:RowSpUnchByGR]
\ExecuteMetaData[../vs3.tex]{lm:RowSpUnchByGR}

\pause
\pf
\ExecuteMetaData[../vs3.tex]{pf:RowSpUnchByGR0}
\end{frame}
\begin{frame}
\ExecuteMetaData[../vs3.tex]{pf:RowSpUnchByGR1}
\qed

\pause
\lm[le:RowSpUnchByGR]
\ExecuteMetaData[../vs3.tex]{lm:RowsEchMatLI}

\pause
\pf
\ExecuteMetaData[../vs3.tex]{pf:RowsEchMatLI}
\qed
\end{frame}




%..........
\begin{frame}
\ex
The matrix before Gauss's Method and the matrix after have equal row
spaces.
\begin{equation*}
  M=
  \begin{mat}[r]
    1 &2  &1 &0 &3 \\
   -1 &-2 &2 &2 &0 \\
    2 &4  &5 &2 &9 
  \end{mat}
  \grstep[-2\rho_1+\rho_3]{\rho_1+\rho_2}
  \grstep{-\rho_2+\rho_3}
  \begin{mat}[r]
    1 &2  &1 &0 &3 \\
    0 &0  &3 &2 &3 \\
    0 &0  &0 &0 &0 
  \end{mat}
\end{equation*}
The nonzero rows of the latter matrix form a basis for $\rowspace{M}$.
\begin{equation*}
  B=\sequence{\rowvec{1 &2  &1 &0 &3},\,
          \rowvec{0 &0  &3 &2 &3}
          }
\end{equation*}
The row rank is $2$.

So Gauss's Method produces a basis for the row space of a matrix.
It has found the ``repeat'' information, that $M$'s third
row is three times the first plus the second, and eliminated that extra row.
\end{frame}




%..........
\begin{frame}{Column space}
\df[df:ColumnSpace]
\ExecuteMetaData[../vs3.tex]{df:ColumnSpace}

\pause
\ex
This system
\begin{equation*}
  \begin{linsys}{2}
  2x &+ &3y     &= &d_1  \\
  -x &+ &(1/2)y &= &d_2
  \end{linsys}
\end{equation*}
has a solution for those $d_1,d_2\in\Re$ that we can find to satisfy
this vector equation.
\begin{equation*}
  x\cdot\colvec{2 \\ -1}+y\cdot\colvec{3 \\ 1/2}
  =\colvec{d_1 \\ d_2}
  \qquad x,y\in\Re
\end{equation*}
That is, the system has a solution if and only if the vector on the right
is in the column space of this matrix.
\begin{equation*}
  \begin{mat}
    2  &3  \\
    -1 &1/2
  \end{mat}
\end{equation*}
\end{frame}




%..........
\begin{frame}{Transpose}
\df[df:Transpose]
\ExecuteMetaData[../vs3.tex]{df:Transpose}

\pause
\ex
To find a basis for the column space of a matrix,
\begin{equation*}
  \begin{mat}
    2  &3  \\
    -1 &1/2
  \end{mat}
\end{equation*}
transpose,
\begin{equation*}
  \trans{\begin{mat}
    2  &3  \\
    -1 &1/2
  \end{mat}}
  =
  \begin{mat}
    2  &-1  \\
    3 &1/2
  \end{mat}
\end{equation*}
reduce, 
\begin{equation*}
  \begin{mat}
    2  &-1  \\
    3 &1/2
  \end{mat}
  \grstep{(-3/2)\rho_1+\rho_2}
  \begin{mat}
    2  &-1  \\
    0  &2
  \end{mat}
\end{equation*}
and transpose back.
\begin{equation*}
  \trans{\begin{mat}
    2  &-1  \\
    0  &2
  \end{mat}}
  =
  \begin{mat}
    2  &0  \\
    -1 &2
  \end{mat}
\end{equation*}
\end{frame}
\begin{frame}
\noindent
This basis
\begin{equation*}
  B=\sequence{\colvec{2 \\ -1},
              \colvec{0 \\ 2}
            }
\end{equation*}
shows that the column space is the entire vector space $\Re^2$. 
\end{frame}




%..........
\begin{frame}
\lm[le:RowOpsNoChngColRnk]
\ExecuteMetaData[../vs3.tex]{lm:RowOpsNoChngColRnk}

\pause
\pf
\ExecuteMetaData[../vs3.tex]{pf:RowOpsNoChngColRnk}
\qed
\end{frame}




%..........
\begin{frame}
\th[th:RowRankEqualsColumnRank]
\ExecuteMetaData[../vs3.tex]{th:RowRankEqualsColumnRank}

\pause
\pf
\ExecuteMetaData[../vs3.tex]{pf:RowRankEqualsColumnRank0}

\pause
\ExecuteMetaData[../vs3.tex]{pf:RowRankEqualsColumnRank1}
\qed

\pause
\df[df:Rank]
\ExecuteMetaData[../vs3.tex]{df:Rank}
\end{frame}




%..........
\begin{frame}
\ex
The column rank of this matrix 
\begin{equation*}
  \begin{mat}
    2 &-1 &3 &1 &0 &1  \\
    3 &0  &1 &1 &4 &-1 \\
    4 &-2 &6 &2 &0 &2  \\
    1 &0  &3 &0 &0 &2 
  \end{mat}
\end{equation*}
is $3$.  
Its largest set of linearly independent columns is size~$3$
because that's the size of its largest set of linearly independent rows.
\begin{equation*}\hspace*{-2em}
  \grstep[-2\rho_1+\rho_3 \\ -(1/2)\rho_1+\rho_4]{-(3/2)\rho_1+\rho_2}
  \grstep{-(1/3)\rho_2+\rho_4}
  \grstep{\rho_3\leftrightarrow\rho_4}
  \begin{mat}
    2 &-1   &3    &1    &0     &1    \\
    0 &3/2  &-7/2 &-1/2 &4     &-5/2 \\
    0 &0    &8/3  &-1/3 &-4/3  &7/3   \\
    0 &0    &0    &0    &0     &0 
  \end{mat}
\end{equation*}
\end{frame}



%..........
\begin{frame}
\th[th:RankVsSoltnSp]
\ExecuteMetaData[../vs3.tex]{tm:RankVsSoltnSp}

\pause
\pf
\ExecuteMetaData[../vs3.tex]{pf:RankVsSoltnSp}
\qed
\end{frame}




%..........
\begin{frame}
\co[co:EquivToNonsingular]
\ExecuteMetaData[../vs3.tex]{co:EquivToNonsingular}

\pause
\pf
\ExecuteMetaData[../vs3.tex]{pf:EquivToNonsingular}
\qed
\end{frame}




%...........................
% \begin{frame}
% \ExecuteMetaData[../gr3.tex]{GaussJordanReduction}
% \df[def:RedEchForm]
% 
% \end{frame}
\end{document}
