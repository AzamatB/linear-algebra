% see: https://groups.google.com/forum/?fromgroups#!topic/comp.text.tex/s6z9Ult_zds
\makeatletter\let\ifGm@compote\relax\makeatother 
\documentclass[10pt,t,serif,professionalfont]{beamer}
\PassOptionsToPackage{pdfpagemode=FullScreen}{hyperref}
\PassOptionsToPackage{usenames,dvipsnames}{color}
% \DeclareGraphicsRule{*}{mps}{*}{}
\usepackage{../linalgjh}
\usepackage{present}
\usepackage{xr}\externaldocument{../det3} % read refs from .aux file
\usepackage{catchfilebetweentags}
\usepackage{etoolbox} % from http://tex.stackexchange.com/questions/40699/input-only-part-of-a-file-using-catchfilebetweentags-package
\makeatletter
\patchcmd{\CatchFBT@Fin@l}{\endlinechar\m@ne}{}
  {}{\typeout{Unsuccessful patch!}}
\makeatother

\mode<presentation>
{
  \usetheme{boxes}
  \setbeamercovered{invisible}
  \setbeamertemplate{navigation symbols}{} 
}
\addheadbox{filler}{\ }  % create extra space at top of slide 
\hypersetup{colorlinks=true,linkcolor=blue} 

\title[Laplace's Expansion] % (optional, use only with long paper titles)
{Four.III Laplace's Expansion}

\author{\textit{Linear Algebra} \\ {\small Jim Hef{}feron}}
\institute{
  \texttt{http://joshua.smcvt.edu/linearalgebra}
}
\date{}


\subject{Laplace's Expansion}
% This is only inserted into the PDF information catalog. Can be left
% out. 

\begin{document}
\begin{frame}
  \titlepage
\end{frame}

% =============================================
% \begin{frame}{Reduced Echelon Form} 
% \end{frame}



% ..... Four.III.1 .....
\section{Laplace's formula for the determinant}
%..........
\begin{frame}
\ex[ex:ExpThreeFirstRow]
\ExecuteMetaData[../det3.tex]{LaplaceExample0}
\end{frame}
\begin{frame}
\ExecuteMetaData[../det3.tex]{LaplaceExample1}
\end{frame}
\begin{frame}
\ExecuteMetaData[../det3.tex]{LaplaceExample2}
\pause
\ExecuteMetaData[../det3.tex]{LaplaceExample3}
\end{frame}




\begin{frame}{Minor}
\df[df:Minor]
\ExecuteMetaData[../det3.tex]{df:Minor}

\pause
\ex For this matrix
\begin{equation*}
  S=
  \begin{mat}
    3 &1 &2 \\
    5 &4 &-1 \\
    7 &0 &-3
  \end{mat}
\end{equation*}
the $2,3$~minor is 
\begin{equation*}
  \begin{mat}
    3 &1 \\
    7 &0
  \end{mat}
\end{equation*}
so the associated cofactor is $S_{2,3}=(-1)^{5}\cdot (-7)=7$.
\end{frame}




\begin{frame}
\th[th:LaPlaceExp]
\ExecuteMetaData[../det3.tex]{th:LaPlaceExp}

\pause
\pf
\nearbyexercise{exer:LaplaceProof}.
\qed

\pause
We can find this determinant  
\begin{equation*}
  \begin{vmat}
    3 &1 &2 \\
    5 &4 &-1 \\
    7 &0 &-3
  \end{vmat}
\end{equation*}
by expanding along the second row.
Besides $S_{2,3}=7$, the other two cofactors are here.
\begin{equation*}
  S_{2,1}=
  (-1)^{3}\cdot
  \begin{vmat}
    1 &2 \\
    0 &-3
  \end{vmat}
  =3
  \quad
  S_{2,2}=
  (-1)^{4}\cdot
  \begin{vmat}
    3 &2 \\
    7 &-3
  \end{vmat}
  =-23
\end{equation*}
The Laplace expansion gives $5\cdot 3+4\cdot(-23)-1\cdot 7=-84$.
\end{frame}




\begin{frame}{Adjoint}
\df[df:Adjoint]
\ExecuteMetaData[../det3.tex]{df:Adjoint}

\medskip
Note that the order of the subscripts in this matrix
is opposite to the order that you might expect.
\end{frame}
\begin{frame}
\ex  The matrix adjoint to this
\begin{equation*}
  S=
  \begin{mat}[r]
    3 &1 &2 \\
    5 &4 &-1 \\
    7 &0 &-3
  \end{mat}
\end{equation*}
is this (some of these cofactors we have calculated above).
\begin{equation*}
  \begin{mat}
    S_{1,1} &S_{2,1} &S_{3,1} \\
    S_{1,2} &S_{2,2} &S_{3,2} \\
    S_{1,3} &S_{2,3} &S_{3,3} 
  \end{mat}
  =
  \begin{mat}[r]
    -12 &3   &-9 \\
      8 &-23 &13 \\
    -28 &7   &7    
  \end{mat}
\end{equation*}
\end{frame}



\begin{frame}
\th[th:MatTimesAdjEqDiagDets]
\ExecuteMetaData[../det3.tex]{th:MatTimesAdjEqDiagDets}

\pause
\medskip
This summarizes.
\begin{multline*}
 \generalmatrix{t}{n}{n}
 \begin{mat}
   T_{1,1}  &T_{2,1}  &\ldots  &T_{n,1}  \\
   T_{1,2}  &T_{2,2}  &\ldots  &T_{n,2}  \\
            &\vdots   &        &         \\
   T_{1,n}  &T_{2,n}  &\ldots  &T_{n,n}
 \end{mat}                                      \\                             
 =\begin{mat}
     |T|      &0        &\ldots  &0        \\
     0        &|T|      &\ldots  &0        \\
              &\vdots   &        &         \\
     0        &0        &\ldots  &|T|
   \end{mat} 
\end{multline*}
\end{frame}
\begin{frame}
\pf[th:MatTimesAdjEqDiagDets]
\nearbytheorem{th:LaPlaceExp} says we can calculate 
the determinant of an \( \nbyn{n} \) matrix \( T \)
by taking linear combinations of entries from a row and
their associated cofactors.
\begin{equation*}
  t_{i,1}\cdot T_{i,1}+t_{i,2}\cdot T_{i,2}+\dots+t_{i,n}\cdot T_{i,n}
   =\deter{T}  
% \tag*{($*$)}
\end{equation*}
This immediately gives the diagonal entries of the matrix result
of $T\adj(T)$.

\pause
For the off-diagonal entries,
recall that a matrix with two identical rows has a determinant of~$0$.
Thus, for any matrix \( T \),
weighing the cofactors by entries from 
row~$k$ with $k\neq i$ gives $0$
\begin{equation*}
  t_{i,1}\cdot T_{k,1}+t_{i,2}\cdot T_{k,2}+\dots+t_{i,n}\cdot T_{k,n}=0
%\tag*{($**$)}
\end{equation*}
because it represents the expansion along the row~$k$ of a matrix with 
row~\( i \) equal to row~\( k \).
\qed
\end{frame}




\begin{frame}
\co[cor:InvFromAdj]
\ExecuteMetaData[../det3.tex]{co:InvFromAdj}

\pause
\ex  
The inverse of this matrix 
\begin{equation*}
  S=
  \begin{mat}[r]
    3 &1 &2 \\
    5 &4 &-1 \\
    7 &0 &-3
  \end{mat}
\end{equation*}
is this.
\begin{equation*}
  \frac{1}{\deter{S}}\cdot\adj(S)
  =
  \frac{1}{84}
  \cdot
  \begin{mat}[r]
    -12 &3   &-9 \\
      8 &-23 &13 \\
    -28 &7   &7    
  \end{mat}
\end{equation*}

\pause
\medskip
\no
The formulas from this section are 
not the best choice for computations with arbitrary matrices
because they typically require more arithmetic than the 
Gauss-Jordan method.
\end{frame}



%...........................
% \begin{frame}g
% \ExecuteMetaData[../gr3.tex]{GaussJordanReduction}
% \df[def:RedEchForm]
% 
% \end{frame}
\end{document}
