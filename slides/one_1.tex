% see: https://groups.google.com/forum/?fromgroups#!topic/comp.text.tex/s6z9Ult_zds
\makeatletter\let\ifGm@compatii\relax\makeatother 
\documentclass[10pt,t,serif,professionalfont]{beamer}
\PassOptionsToPackage{pdfpagemode=FullScreen}{hyperref}
\PassOptionsToPackage{usenames,dvipsnames}{color}
% \usepackage{linear_algebra}
\usepackage{../linalgjh}
\usepackage{present}
\usepackage{catchfilebetweentags}
\usepackage{etoolbox} % from http://tex.stackexchange.com/questions/40699/input-only-part-of-a-file-using-catchfilebetweentags-package
\makeatletter
\patchcmd{\CatchFBT@Fin@l}{\endlinechar\m@ne}{}
  {}{\typeout{Unsuccessful patch!}}
\makeatother

\mode<presentation>
{
  \usetheme{boxes}
  \setbeamercovered{invisible}
  \setbeamertemplate{navigation symbols}{} 
}
\addheadbox{filler}{\ }  % create extra space at top of slide 
\hypersetup{colorlinks=true,linkcolor=blue} 

\title[Solving Linear Systems] % (optional, use only with long paper titles)
{Solving Linear Systems}

\author{\textit{Linear Algebra} \\ {\small Jim Hef{}feron}}
\institute{
  \texttt{http://joshua.smcvt.edu/linearalgebra}
}
\date{}


\subject{Solving Linear Systems}
% This is only inserted into the PDF information catalog. Can be left
% out. 
% \AtBeginDocument{\makeatletter
% \let\ref\@refstar
% \makeatother}
% \usepackage{xparse}% http://ctan.org/pkg/xparse

% \makeatletter
% \NewDocumentCommand{\reff}{s m}{%
%   \IfBooleanTF{#1}% Check for starred variant
%     {\beamer@origref{#2}}% \reff*
%     {\hyperlink{#2}{\beamer@origref{#2}}}% \reff
% }
% \makeatother

\begin{document}
\begin{frame}
  \titlepage
\end{frame}

% =============================================



% ..... One.I.1 .....
\section{Gauss' method}
\begin{frame}{Linear systems} 
\df[def:linearcombination]
\ExecuteMetaData[../gr1.tex]{df:linearcombination}

\ex This is a linear combination of $x$, $y$, and $z$.
\begin{equation*}
   (1/4)x+y-z
\end{equation*}
\end{frame}


%...........................
\begin{frame}
\df[def:linearcombination]
\ExecuteMetaData[../gr1.tex]{df:linearequations}
\ex 
There are three linear equations in this linear system.
\begin{equation*}
  \begin{linsys}{3}
   (1/4)x  &+  &y  &-  &z  &=  &0  \\
        x  &+  &4y &+  &2z &=  &12  \\
       2x  &-  &3y &-  &z  &=  &3  
  \end{linsys}  
\end{equation*}
\end{frame}



%...........................
\begin{frame}{Solving a linear system}
\ex 
To find the solution of this system
\begin{equation*}
  \begin{linsys}{3}
   (1/4)x  &+  &y  &-  &z  &=  &0  \\
        x  &+  &4y &+  &2z &=  &12  \\
       2x  &-  &3y &-  &z  &=  &3  
  \end{linsys}  
\end{equation*}
we transform it to one whose solution is easy.
\pause We can start by clearing the fractions.
\begin{eqnarray*}
  &\grstep{4\rho_1}
  &\begin{linsys}{3}
        x  &+  &4y &-  &4z  &=  &0  \\
        x  &+  &4y &+  &2z &=  &12  \\
       2x  &-  &3y &-  &z  &=  &3  
  \end{linsys}  
\end{eqnarray*}
\pause Use the first row to act on the rows below, eliminating the $x$~terms.
\begin{eqnarray*}
  &\grstep[-2\rho_1+\rho_3]{-1\rho_1+\rho_2}
  &\begin{linsys}{3}
        x  &+  &4y   &-  &4z  &=  &0  \\
           &   &     &+  &6z  &=  &12  \\
           &   &-11y &+  &7z  &=  &3  
  \end{linsys} 
\end{eqnarray*} 
\end{frame}\begin{frame}
Swap to bring a $y$ term to the second row.
\begin{eqnarray*}
  &\grstep{\rho_2\leftrightarrow\rho_3}
  &\begin{linsys}{3}
        x  &+  &4y   &-  &4z  &=  &0  \\
           &   &-11y &+  &7z  &=  &3  \\
           &   &     &   &6z  &=  &12  
  \end{linsys}  
\end{eqnarray*}
\pause We can solve the bottom row:
$z=2$.
\pause
The shape of the transformed system 
lets us solve for $y$ by substituting into the second row:
$-11y+7(2)=3$ shows that $y=1$.
\pause
The shape also lets us solve for $x$ by substituting into the
first row: $x+4(1)-4(2)=0$ gives that $x=4$. 

\df[df:EchelonForm]
\ExecuteMetaData[../gr1.tex]{df:EchelonForm}
\end{frame}



%...........................
\begin{frame}
\ex 
\begin{eqnarray*}
  \begin{linsys}{4}
    2x  &-  &3y  &-  &z  &+ &2w &=  &-2  \\
     x  &   &    &+  &3z &+ &1w &=  &6  \\
    2x  &-  &3y  &-  &z  &+ &3w &=  &-3  \\
        &   &y   &+  &z  &- &2w &=  &4  
  \end{linsys}  
  &\grstep[-\rho_1+\rho_3]{(-1/2)\rho_1+\rho_2}
  &\begin{linsys}{4}
    2x  &-  &3y      &-  &z      &+  &2w &=  &-2  \\
        &   &(3/2)y  &+  &(7/2)z &   &   &=  &7  \\
        &   &        &   &       &   &w  &=   &-1  \\
        &   &y       &+  &z      &-  &2w  &=  &4  
  \end{linsys}                                       \\
  &\grstep{(-2/3)\rho_2+\rho_4}
  &\begin{linsys}{4}
    2x  &-  &3y      &-  &z      &+  &2w &=  &-2  \\
        &   &(3/2)y  &+  &(7/2)z &   &   &=  &7  \\
        &   &        &   &       &   &w  &=   &-1  \\
        &   &        &-  &(4/3)z &- &2w  &=  &-2/3  
  \end{linsys}                                         \\
  &\grstep{\rho_3\leftrightarrow\rho_4}
  &\begin{linsys}{4}
    2x  &-  &3y      &-  &z      &+  &2w &=  &-2  \\
        &   &(3/2)y  &+  &(7/2)z &   &   &=  &7  \\
        &   &        &-  &(4/3)z &- &2w  &=  &-2/3 \\ 
        &   &        &   &       &   &w  &=   &-1  
  \end{linsys}  
\end{eqnarray*}
The fourth equation says $w=-1$.
Substituting back into the third equation gives $z=2$.
Then back substitution into the second and first rows gives
$y=0$ and $x=1$.
The unique solution is $(1,0,2,-1)$.
\end{frame}



%...........................
\begin{frame}{Gauss' method}
\th[th:GaussMethod]
\ExecuteMetaData[../gr1.tex]{th:GaussMethod}

\df[df:GaussMethod]
\ExecuteMetaData[../gr1.tex]{df:GaussMethod}
\end{frame}



%...........................
\begin{frame}
\pf[th:GaussMethod]
We cover one of the three.
The other two are similar.

\ExecuteMetaData[../gr1.tex]{pf:GaussMethod}
\end{frame}



%...........................
\begin{frame}{Systems without a unique solution}
\ex
This system has no solution.
\begin{equation*}
  \begin{linsys}{3}
        x  &+  &y  &+  &z  &=  &6  \\
        x  &+  &2y &+  &z  &=  &8  \\
       2x  &+  &3y &+  &2z &=  &13  
  \end{linsys}    
\end{equation*}
On the left side the sum of the first two equals the third,
while on the right that is not so.
So there is no triple of reals that makes all three 
equations true.
\pause
Gauss' Method makes the inconsistency clear.
\begin{equation*}
  \grstep[-2\rho_1+\rho_3]{-\rho_1+\rho_2}\;
  \begin{linsys}{3}
        x  &+  &y  &+  &z  &=  &6  \\
           &   &y  &   &   &=  &2  \\
           &   &y  &   &   &=  &1  
  \end{linsys}    
  \;\grstep{-\rho_2+\rho_3}\;
  \begin{linsys}{3}
        x  &+  &y  &+  &z  &=  &6  \\
           &   &y  &   &   &=  &2  \\
           &   &   &   &0  &=  &-1  
  \end{linsys}    
\end{equation*}
\end{frame}



%...........................
\begin{frame}
\ex
This system has infinitely many solutions.
\begin{equation*}
  \begin{linsys}{3}
        x  &-  &y  &+  &z  &=  &4  \\
        x  &+  &y  &-  &2z &=  &-1   
  \end{linsys} 
  \;\grstep{-\rho_1+\rho_2}\;   
  \begin{linsys}{3}
        x  &-  &y  &+  &z  &=  &4  \\
           &   &2y &-  &3z &=  &-5   
  \end{linsys} 
\end{equation*}
Taking $z=0$ gives the solution $(3/2,-5/2,0)$.
Taking $z=-1$ gives $(1,-4,-1)$.

\pause
\ex
Another system with infinitely many solutions.
\begin{eqnarray*}
  \begin{linsys}{3}
        -x   &-  &y  &+  &3z  &=  &3  \\
         x   &   &   &+  &z   &=  &3  \\
        3x   &-  &y  &+  &7z  &=  &15   
  \end{linsys} 
  &\grstep[3\rho_1+\rho_3]{-\rho_1+\rho_2}   
  &\begin{linsys}{3}
        -x   &-  &y   &+  &3z  &=  &3  \\
             &   &-y  &+  &4z  &=  &6  \\
             &   &-4y &+  &16z &=  &24   
  \end{linsys}                                \\
  &\grstep{-4\rho_2+\rho_3}
  &\begin{linsys}{3}
        -x   &-  &y  &+  &3z  &=  &3  \\
             &   &-y  &+  &4z  &=  &6  \\
             &   &   &   &0    &=  &0   
  \end{linsys} 
\end{eqnarray*}
Taking $z=0$ gives $(3,-6,0)$ while
taking $z=1$ gives $(2,-2,1)$.
\end{frame}










% ..... One.I.2 .....
\section{Describing the solution set}
\begin{frame}{Parametrizing} 
We've seen that this system has infinitely many solutions
\begin{equation*}
  \begin{linsys}{3}
        -x   &-  &y  &+  &3z  &=  &3  \\
         x   &   &   &+  &z   &=  &3  \\
        3x   &-  &y  &+  &7z  &=  &15   
  \end{linsys} 
  \;\grstep[3\rho_1+\rho_3]{-\rho_1+\rho_2}   
  \;\grstep{-4\rho_2+\rho_3}\;
  \begin{linsys}{3}
        -x   &-  &y  &+  &3z  &=  &3  \\
             &   &-y  &+  &4z  &=  &6  \\
             &   &   &   &0    &=  &0   
  \end{linsys} 
\end{equation*}
Use the second row to express $y$ in terms of~$z$ as 
$y=-6+4z$. 
\pause 
Now substitute into the first row $-x-(-6+4z)+3z=3$
to 
express $x$ also in terms of~$z$ with
$x=3-z$.

\pause
\df[df:FreeVars]
\ExecuteMetaData[../gr1.tex]{df:FreeVars}

\ExecuteMetaData[../gr1.tex]{df:Parameter}
We shall routinely parametrize linear systems using the free variables.
\end{frame}




% ..........
\begin{frame}
\ex
\begin{eqnarray*}
  \begin{linsys}{4}
         x   &-  &y  &+  &2z  &+  &3w &=  &14  \\
        2x   &-  &2y &-  &z   &+  &2w &=  &6  \\
             &   &   &-  &3z  &+  &2w &=  &0   
  \end{linsys} 
  &\grstep{-2\rho_1+\rho_2}
  &\begin{linsys}{4}
         x   &-  &y  &+  &2z  &+  &3w &=  &14  \\
             &   &   &-  &5z  &-  &4w &=  &-22  \\
             &   &   &-  &3z  &+  &2w &=  &0   
  \end{linsys}                                   \\
  &\grstep{-(3/5)\rho_2+\rho_3}
  &\begin{linsys}{4}
         x   &-  &y  &+  &2z  &+  &3w      &=  &14  \\
             &   &   &-  &5z  &-  &4w      &=  &-22  \\
             &   &   &   &    &   &(22/5)w &=  &66/5   
  \end{linsys}                                  
\end{eqnarray*}
The leading variables are $x$, $w$, and $z$, so we parametrize with $y$.
The bottom row gives $w=3$ and substituting that into the next
row up gives $z=2$.
\pause
The top equation is $x-y+2\cdot 2+3\cdot 3=14$, so 
we have $x=1-y$.
\end{frame}




% ..........
\begin{frame}
\ex
This system has already been brought to echelon form.
\begin{equation*}
  \begin{linsys}{4}
   -2x  &+  &y  &-  &z   &+   &w  &= &3/2  \\
        &   &   &   &2z  &-   &w  &= &1/2 
  \end{linsys} 
\end{equation*}
The leading variables are $x$ and $z$ so we will 
parametrize the solution set with $y$ and $w$.
\pause
The second row gives $z=1/4+(1/2)w$.
\pause
Substituting back into the first row gives
$-2x+y-((1/4)+(1/2)w)+w=3/2$,
and solving for $x$ leaves
$x=-(7/8)+(1/2)y+(1/4)w$.
\end{frame}



% ..........
\begin{frame}{Matrices and vectors}
\df[df:matrix]
\ExecuteMetaData[../gr1.tex]{df:matrix}

\pause
\ex
This is a $\nbym{2}{3}$ matrix
\begin{equation*}
  B=
  \begin{mat}[r]
    1  &-2  &3  \\
    4  &-5  &6
  \end{mat}
\end{equation*}
because it has $2$~rows and $3$~columns.
The entry in row~$2$ and column~$1$ is
\( b_{2,1}=4 \).

\pause
\df[df:vector]
\ExecuteMetaData[../gr1.tex]{df:vector}

\pause
\ex
This column vector has three components.
\begin{equation*}
  \vec{v}=\colvec{-1  \\ -0.5  \\ 0}
\end{equation*}
\end{frame}



% ..........
\begin{frame}{Vector operations}
\df[df:VectorSum]
\ExecuteMetaData[../gr1.tex]{df:VectorSum}
\df[df:VectorScalarMultiplication]
\ExecuteMetaData[../gr1.tex]{df:VectorScalarMultiplication}
\ex
\begin{equation*}
  3\colvec{1 \\2}-2\colvec{0 \\ 1}=\colvec{3 \\ 4}
\end{equation*}
\end{frame}


% ..........
\begin{frame}{Matrix notation for linear systems}
\ex
We can simplify the clerical load in reducing this system
\begin{equation*}
  \begin{linsys}{3}
       -3x   &   &   &+  &2z  &=  &-1  \\
         x   &-  &2y &+  &2z  &=  &-5/3  \\
        -x   &-  &4y &+  &6z  &=  &-13/3   
  \end{linsys} 
\end{equation*}
by writing it as a \definend{augmented matrix}.
\begin{eqnarray*}
    \begin{amat}[r]{3}
     -3  &0  &2  &-1  \\
      1  &-2 &2  &-5/3  \\
     -1  &-4 &6  &-13/3
    \end{amat}
  &\grstep[-(1/3)\rho_1+\rho_3]{(1/3)\rho_1+\rho_2}
  &\begin{amat}[r]{3}
     -3  &0  &2    &-1  \\
      0  &-2 &8/3  &-2  \\
      0  &-4 &16/3 &-4
    \end{amat}                      \\
  &\grstep{-2\rho_2+\rho_3}
  &\begin{amat}[r]{3}
     -3  &0  &2    &-1  \\
      0  &-2 &8/3  &-2  \\
      0  &0  &0    &0
    \end{amat}  
\end{eqnarray*}
The two nonzero rows give
$-3x+2z=-1$ and $-2y+(8/3)z=-2$.
\end{frame}



% ..........
\begin{frame}
Solving
$-3x+2z=-1$ and $-2y+(8/3)z=-2$
for the leading variables
gives 
$y=1+(4/3)z$ and $x=(1/3)+(2/3)z$.
\pause
We can write the solution set in vector notation.
\begin{equation*}
  \set{\colvec{x \\ y \\ z}=\colvec{1/3 \\ 1 \\ 0}+\colvec{2/3 \\ 4/3 \\ 1}z 
                 \suchthat z\in\Re}
\end{equation*}
\end{frame}



% ..........
\begin{frame}
\ex
We can reduce this system
\begin{equation*}
  \begin{linsys}{4}
    x &+  &2y  &- &z  &  &  &= &2 \\
   2x &-  &y   &- &2z &+ &w &= &5
  \end{linsys}
\end{equation*}
using the augmented matrix notation
\begin{eqnarray*}
    \begin{amat}[r]{4}
      1  &2  &-1  &0  &2  \\
      2  &-1 &-2  &1  &5  
    \end{amat}
  &\grstep{-2\rho_1+\rho_2}
  &\begin{amat}[r]{4}
      1  &2  &-1  &0  &2  \\
      0  &-5 &0   &1  &1  
    \end{amat}
\end{eqnarray*}
to get this vector description of the solution set.
\begin{equation*}
  \set{\colvec{12/5 \\ -1/5 \\ 0 \\ 0}
       +\colvec{1 \\ 0 \\ 1 \\ 0}z
       +\colvec{-2/5 \\ 1/5 \\ 0 \\ 1}w
      \suchthat z,w\in\Re}
\end{equation*}
\end{frame}




% ..... One.I.3 .....
\section{$\text{General}=\text{Particular}+\text{Homogeneous}$}
\begin{frame}
\frametitle{Form of solution sets} 

\end{frame}








%...........................
% \begin{frame}
% 
% \end{frame}
\end{document}
