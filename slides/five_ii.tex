% see: https://groups.google.com/forum/?fromgroups#!topic/comp.text.tex/s6z9Ult_zds
\makeatletter\let\ifGm@compote\relax\makeatother 
\documentclass[10pt,t]{beamer}
\usefonttheme{professionalfonts}
\usefonttheme{serif}
\PassOptionsToPackage{pdfpagemode=FullScreen}{hyperref}
\PassOptionsToPackage{usenames,dvipsnames}{color}
% \DeclareGraphicsRule{*}{mps}{*}{}
\usepackage{../linalgjh}
\usepackage{present}
\usepackage{xr}\externaldocument{../jc2} % read refs from .aux file
\usepackage{catchfilebetweentags}
\usepackage{etoolbox} % from http://tex.stackexchange.com/questions/40699/input-only-part-of-a-file-using-catchfilebetweentags-package
\makeatletter
\patchcmd{\CatchFBT@Fin@l}{\endlinechar\m@ne}{}
  {}{\typeout{Unsuccessful patch!}}
\makeatother

\usepackage{polynom}  % for polynomial long division

\mode<presentation>
{
  \usetheme{boxes}
  \setbeamercovered{invisible}
  \setbeamertemplate{navigation symbols}{} 
}
\addheadbox{filler}{\ }  % create extra space at top of slide 
\hypersetup{colorlinks=true,linkcolor=blue} 

\title[Similarity] % (optional, use only with long paper titles)
{Five.II Similarity}

\author{\textit{Linear Algebra} \\ {\small Jim Hef{}feron}}
\institute{
  \texttt{http://joshua.smcvt.edu/linearalgebra}
}
\date{}


\subject{Similarity}
% This is only inserted into the PDF information catalog. Can be left
% out. 

\begin{document}
\begin{frame}
  \titlepage
\end{frame}



\section{Definition and Examples}
% =============================================
\begin{frame}
\vspace*{-2ex}
\ExecuteMetaData[../jc2.tex]{SimilarityMotiviation0}  
\pause  
\ExecuteMetaData[../jc2.tex]{SimilarityMotiviation1}  
\end{frame}




% ..... Five.II.1 .....
\begin{frame}{Similar matrices}
\df[df:Similar]
\ExecuteMetaData[../jc2.tex]{df:Similar}  

\ex
Consider the derivative map $\map{d/dx}{\polyspace_2}{\polyspace_2}$.
Fix the basis $B=\sequence{1,x,x^2}$ 
and the basis $D=\sequence{1,1+x,1+x+x^2}$.
In this arrow diagram we will first get $T$, and then calculate $\hat{T}$ 
from it.
\begin{equation*}
  \begin{CD}
    V_{\wrt{B}}                   @>t>T>        V_{\wrt{B}}       \\
    @V{\scriptstyle\identity} VV              @V{\scriptstyle\identity} VV \\
    V_{\wrt{D}}                   @>t>\hat{T}>        V_{\wrt{D}}
  \end{CD}
\end{equation*}
\pause
The action of $d/dx$ on the 
elements of the basis $B$ is $1\mapsto 0$, $x\mapsto 1$, and $x^2\mapsto 2x$.
\begin{equation*}
  \rep{\frac{d}{dx}(1)}{B}=\colvec{0 \\ 0 \\ 0}
  \hspace{0.8em}
  \rep{\frac{d}{dx}(x)}{B}=\colvec{1 \\ 0 \\ 0}
  \hspace{0.8em}
  \rep{\frac{d}{dx}(x^2)}{B}=\colvec{0 \\ 2 \\ 0}
\end{equation*}
\end{frame}
\begin{frame}
So we have this matrix representation of the map.
\begin{equation*}
  T=\rep{\frac{d}{dx}}{B,B}=
  \begin{mat}
    0 &1 &0 \\
    0 &0 &2 \\
    0 &0 &0
  \end{mat}
\end{equation*}
\pause
The matrix changing bases from $B$ to $D$ is $\rep{\identity}{B,D}$.
We find these by eye
\begin{equation*}
  \rep{1}{D}=\colvec{1 \\ 0 \\ 0}
  \quad
  \rep{x}{D}=\colvec{-1 \\ 1 \\ 0}
  \quad
  \rep{x^2}{D}=\colvec{0 \\ -1 \\ 1}
\end{equation*}
to get this.
\begin{equation*}
  P=
  \begin{mat}
    1 &-1 &0  \\
    0 &1  &-1 \\
    0 &0  &1
  \end{mat}
  \qquad
  P^{-1}=
  \begin{mat}
    1 &1  &1  \\
    0 &1  &1 \\
    0 &0  &1
  \end{mat}
\end{equation*}
Now, by following the arrow diagram we have $\hat{T}=PTP^{-1}$.
\begin{equation*}
  \hat{T}=
  \begin{mat}
    0 &1 &-1 \\
    0 &0 &2  \\
    0 &0 &0
  \end{mat}
\end{equation*}
\end{frame}
\begin{frame}
To check that, and to underline what the arrow diagram says 
\begin{equation*}
  \begin{CD}
    V_{\wrt{B}}                   @>t>T>        V_{\wrt{B}}       \\
    @V{\scriptstyle\identity} VV              @V{\scriptstyle\identity} VV \\
    V_{\wrt{D}}                   @>t>\hat{T}>        V_{\wrt{D}}
  \end{CD}
\end{equation*}
we calculate $T$ directly.
The effect of the map on the basis elements is 
$d/dx(1)=0$, $d/dx(1+x)=1$, and $d/dx(1+x+x^2)=1+2x$.
Representing of those with respect to $D$
\begin{equation*}
  \rep{0}{D}=\colvec{0 \\ 0 \\ 0}
  \quad
  \rep{1}{D}=\colvec{1 \\ 0 \\ 0}
  \quad
  \rep{1+2x}{D}=\colvec{-1 \\ 2 \\ 0}
\end{equation*}
gives the same matrix $\hat{T}=\rep{d/dx}{D,D}$ as we found above.
\end{frame}
\begin{frame}
The definition doesn't require that we consider the underlying maps.
We can just multiply matrices.  

\ex
Where 
\begin{equation*}
  T=
  \begin{mat}
    0 &-1 &-2 \\
    2 &3 &2   \\
    4 &5 &2
  \end{mat}
  \qquad
  P=
  \begin{mat}
    1 &1 &0 \\
   -1 &1 &0   \\
    0 &0 &3
  \end{mat}
\end{equation*}
(note that $P$ is nonsingular) we can compute this $\hat{T}=PTP^{-1}$.
\begin{equation*}
  \hat{T}=
  \begin{mat}
    2   &0   &0 \\
    3   &1   &4/3 \\
   27/2 &3/2 &2
  \end{mat}
\end{equation*}

\pause
\ex[ex:OnlyZeroSimToZero]
\ExecuteMetaData[../jc2.tex]{ex:OnlyZeroSimToZero}  
\end{frame}



\begin{frame}{Similarity is an equivalence}
\nearbyexercise{exer:SimIsEquivRel} checks that
similarity is an equivalence relation.

\pause
Since matrix similarity is a special case of matrix equivalence, 
if two matrices are similar then they are matrix equivalent.
What about the converse:~must any two matrix equivalent square matrices be 
similar?
No; the matrix equivalence class
of an identity consists of all nonsingular matrices of that size while the 
prior example shows that an identity matrix is alone in its similarity
class. 

\pause
So some matrix equivalence classes
split into two or more similarity classes\Dash similarity gives a finer
partition than does equivalence.
This pictures some matrix equivalence classes subdivided into
similarity classes.
\centergraphic{../ch5.4} 

\pause
We naturally want a canonical form to represent the similarity classes.
Some classes, but not all,
are represented by a diagonal form.
\end{frame}




% ..... Five.II.2 .....
\section{Diagonalizability}
\begin{frame}
\df[df:Diagonalizable]
\ExecuteMetaData[../jc2.tex]{df:Diagonalizable}  

\ex
This matrix
\begin{equation*}
  \begin{mat}
    6 &-1  &-1 \\
    2 &11  &-1 \\
   -6 &-5  &7
  \end{mat}
\end{equation*}
is diagonalizable by using this
\begin{equation*}
  P=
  \begin{mat}
    1/2 &1/4  &1/4 \\
   -1/2 &1/4  &1/4 \\
   -1/2 &-3/4 &1/4
  \end{mat}
  \quad
  P^{-1}=
  \begin{mat}
    1 &-1 &0 \\
    0 &1 &-1 \\
    2 &1 &1
  \end{mat}
\end{equation*}
to get this $D=PSP^{-1}$.
\begin{equation*}
  D=
  \begin{mat}
    4 &0 &0 \\
    0 &8 &0 \\
    0 &0 &12
  \end{mat}
\end{equation*}
\end{frame}
\begin{frame}
\ex  
\ExecuteMetaData[../jc2.tex]{ex:NotDiagonalizable}  
\end{frame}




\begin{frame}
\lm[lm:DiagIffBasisOfEigens]
\ExecuteMetaData[../jc2.tex]{lm:DiagIffBasisOfEigens}  

\pause
\pf
\ExecuteMetaData[../jc2.tex]{pf:DiagIffBasisOfEigens}  
\qed
\end{frame}

\begin{frame}
\ex
This matrix is not diagonal
\begin{equation*}
  T=
  \begin{mat}
    4  &1  \\
    0  &-1
  \end{mat}
\end{equation*}
but we can
find a diagonal matrix similar to it, by finding an appropriate basis.

\pause
Suppose that $T=\rep{t}{\stdbasis_2,\stdbasis_2}$ for $\map{t}{\Re^2}{\Re^2}$.
We will find a basis $B=\sequence{\vec{\beta}_1,\vec{\beta}_2}$ giving a 
diagonal representation.
\begin{equation*}
  D=\rep{t}{B,B}=
  \begin{mat}
    \lambda_1  &0 \\
    0    &\lambda_2
  \end{mat}
\end{equation*}
Here is the arrow diagram.
\begin{equation*}
  \begin{CD}
    V_{\wrt{\stdbasis_2}}            @>t>T>        V_{\wrt{\stdbasis_2}}       \\
    @V{\scriptstyle\identity} VV              @V{\scriptstyle\identity} VV \\
    V_{\wrt{B}}                   @>t>D>        V_{\wrt{B}}
  \end{CD}
\end{equation*}
\end{frame}
\begin{frame}
We want $\lambda_1$ and $\lambda_2$ making these true.
\begin{equation*}
  \begin{mat}
    4  &1  \\
    0  &-1    
  \end{mat}
  \vec{\beta}_1
  =\lambda_1\cdot\vec{\beta}_1
  \qquad
  \begin{mat}
    4  &1  \\
    0  &-1    
  \end{mat}
  \vec{\beta}_2
  =\lambda_2\cdot\vec{\beta}_2
\end{equation*}
More precisely, 
we want all scalars $x\in\C$ such that this system
\begin{equation*}
  \begin{mat}
    4  &1  \\
    0  &-1    
  \end{mat}
  \colvec{b_1 \\ b_2}
  =x\cdot\colvec{b_1 \\ b_2}
\end{equation*}
has solutions $b_1,b_2\in\C$ that are not both zero
(the zero vector is not an element of any basis).
% \end{frame}
% \begin{frame}

\pause
Rewrite that as a linear system.
\begin{equation*}
  \begin{linsys}{2}
    (4-x)\cdot b_1 &+ &b_2             &= &0 \\
                   &  &(-1-x)\cdot b_2 &= &0 
  \end{linsys}
  % \tag{$*$}
\end{equation*}
One solution is $\lambda_1=-1$, associated with those
$(b_1,b_2)$ such that $b_1=(-1/5)b_2$.
The other solution is $\lambda_2=4$, associated with the
$(b_1,b_2)$ such that
$b_2=0$.
\end{frame}
\begin{frame}
Thus the original matrix  
\begin{equation*}
  T=
  \begin{mat}[r]
    4  &1  \\
    0  &-1
  \end{mat}
\end{equation*}
is diagonalizable to
\begin{equation*}
  D=
  \begin{mat}[r]
    -1  &0  \\
    0  &4
  \end{mat}
\end{equation*}
where this is a basis.
\begin{equation*}
  B=\sequence{\colvec[r]{-1 \\ 5},
              \colvec{1 \\ 0}}
\end{equation*}
\end{frame}




% ..... Five.II.3 .....
\section{Eigenvalues and Eigenvectors}
\begin{frame}{Eigenvalues and eigenvectors}
\df[def:Eigen]
\ExecuteMetaData[../jc2.tex]{df:Eigen}  

\pause
\df[df:EigenOfMatrix]
\ExecuteMetaData[../jc2.tex]{df:EigenOfMatrix}  

\ex
The matrix
\begin{equation*}
  D=
  \begin{mat}
    4  &0 \\
    0  &2  
  \end{mat}
\end{equation*}
has an eigenvalue $\lambda_1=4$ and a second eigenvalue $\lambda_2=2$.
The first is true because an associated eigenvector is
$\vec{e_1}$
\begin{equation*}
  \begin{mat}
    4  &0 \\
    0  &2  
  \end{mat}
  \colvec{1 \\ 0}
  =
  4\cdot\colvec{1 \\ 0}  
\end{equation*}
and similarly for the second an associated eigenvector
is $e_2$.
\begin{equation*}
  \begin{mat}
    4  &0 \\
    0  &2  
  \end{mat}
  \colvec{0 \\ 1}
  =
  2\cdot\colvec{0 \\ 1}  
\end{equation*}
\end{frame}
\begin{frame}
Thinking of the matrix as representing a transformation of the plane,
the transformation acts on those vectors in a particularly simple way, 
by rescaling.

Not every vector is simply rescaled.
\begin{equation*}
  \begin{mat}
    4  &0 \\
    0  &2  
  \end{mat}
  \colvec{1 \\ 1}
  =\colvec{4 \\ 2}
  \neq
  x\cdot\colvec{1 \\ 1}  
\end{equation*}
\end{frame}



\begin{frame}
Matrices that are similar have the same eigenvalues, but
needn't have the same eigenvectors.

\ex
These two are similar 
\begin{equation*}
  T=
  \begin{mat}
    4 &0 &0 \\
    0 &8 &0 \\
    0 &0 &12
  \end{mat}
  \qquad
  S=
  \begin{mat}[r]
    6 &-1  &-1 \\
    2 &11  &-1 \\
   -6 &-5  &7
  \end{mat}
\end{equation*}
since $S=PTP^{-1}$ for this $P$.
\begin{equation*}
  P=
  \begin{mat}[r]
    1 &-1 &0 \\
    0 &1 &-1 \\
    2 &1 &1
  \end{mat}
  \qquad
  P^{-1}=
  \begin{mat}[r]
    1/2 &1/4  &1/4 \\
   -1/2 &1/4  &1/4 \\
   -1/2 &-3/4 &1/4
  \end{mat}
\end{equation*}
\end{frame}
\begin{frame}
\noindent Suppose that $\map{t}{\C^3}{\C^3}$ is
represented by $T$ with respect to the standard basis.
Then this is the action of $t$.
\begin{equation*}
  \colvec{x \\ y \\ z}\mapsunder{t}\colvec{4x \\ 8y  \\ 12z}
\end{equation*}
\pause
By eye we see that three 
eigenvalues of~$t$ are $\lambda_1=4$, $\lambda_2=8$, and~$\lambda_3=12$.
For instance this holds.
\begin{equation*}
  T\cdot\colvec{1 \\ 0 \\ 0}
  =\begin{mat}
    4 &0 &0 \\
    0 &8 &0 \\
    0 &0 &12
  \end{mat}\colvec{1 \\ 0 \\ 0}
  =4\cdot\colvec{1 \\ 0 \\ 0}
\end{equation*}
\end{frame}
\begin{frame}
Contrast that with $S=PTP^{-1}$, which represents the same function, but 
with respect to a different basis.  
\begin{equation*}
  \begin{CD}
    V_{\wrt{\stdbasis_3}}            @>t>T>        V_{\wrt{\stdbasis_3}}       \\
    @V{\scriptstyle\identity} VV              @V{\scriptstyle\identity} VV \\
    V_{\wrt{B}}                   @>t>S>        V_{\wrt{B}}       
  \end{CD}
\end{equation*}
We can easily find the basis~$B$.
Since $P^{-1}=\rep{\identity}{B,\stdbasis_3}$, its first column is 
$\rep{\identity(\vec{\beta}_1)}{\stdbasis_3}=\rep{\vec{\beta}_1}{\stdbasis_3}$.
With respect to the standard basis any vector is represented by itself 
so the first basis element $\vec{\beta}_1$ is the first column of $P^{-1}$.
The same goes for the other two columns.
\begin{equation*}
  B=\sequence{\colvec[r]{1/2 \\ -1/2 \\ -1/2},
              \colvec[r]{1/4 \\ 1/4 \\ -3/4},
              \colvec[r]{1/4 \\ 1/4 \\ 1/4}}
\end{equation*}
\end{frame}
\begin{frame}
% We know that the transformation~$t$ has eigenvalues of $4$, $8$, and~$12$.
% For instance $t(\vec{e}_1)=4\vec{e}_1$.
Now, since each represents the transformation~$t$, the matrices~$T$ and $S$
reflect the same action $\vec{e}_1\mapsto4\vec{e}_1$.
\begin{align*}
  &\rep{t}{\stdbasis_3,\stdbasis_3}\cdot\rep{\vec{e}_1}{\stdbasis_3}
    =T\cdot\rep{\vec{e}_1}{\stdbasis_3}                
    =4\cdot\rep{\vec{e}_1}{\stdbasis_3}                  \\               
    &\rep{t}{B,B}\cdot\rep{\vec{e}_1}{B}
    =S\cdot\rep{\vec{e}_1}{B}                                 
    =4\cdot\rep{\vec{e}_1}{B}
\end{align*}
But, while in the two equations the $4$'s are the same, the vectors
representations are not. 
\begin{align*}
    T\cdot\rep{\vec{e}_1}{\stdbasis_3}
    =T\colvec{1 \\ 0 \\ 0}
    &=4\cdot\colvec{1 \\ 0 \\ 0}                \\
    S\cdot\rep{\vec{e}_1}{B}  
    =S\cdot\colvec{1 \\ 0 \\ 2}
    &=4\cdot\colvec{1 \\ 0 \\ 2}
\end{align*}
So the two matrices have the same eigenvalues but different eigenvectors.
\end{frame}




\begin{frame}{Computing eigenvalues and eigenvectors}
\ex
We will find the eigenvalues and associated eigenvectors of this matrix.
\begin{equation*}
  T=
  \begin{mat}
    0 &5 &7 \\
   -2 &7 &7 \\
   -1 &1 &4
  \end{mat}
\end{equation*}
We want to find scalars~$x$ such that $T\vec{\zeta}=x\vec{\zeta}$ for 
some nonzero $\vec{\zeta}$.
Bring the terms to the left side.
\begin{equation*}
  \begin{mat}
    0 &5 &7 \\
   -2 &7 &7 \\
   -1 &1 &4
  \end{mat}
  \colvec{z_1 \\ z_2 \\ z_3}
  -x\colvec{z_1 \\ z_2 \\ z_3}
  =\colvec{0 \\ 0 \\ 0}
\end{equation*}
and factor.
\begin{equation*}
  \begin{mat}
    0-x &5   &7 \\
   -2   &7-x &7 \\
   -1   &1   &4-x
  \end{mat}
  \colvec{z_1 \\ z_2 \\ z_3}
  =
  \colvec{0 \\ 0 \\ 0}
  \tag{$*$}
\end{equation*}
This homogeneous system has nonzero solutions if and only if the 
matrix is singular, that is, has a determinant of zero.
\end{frame}
\begin{frame}
Some computation gives the determinant and its factors.
\begin{align*}
  0&=
  \begin{vmat}
    0-x &5   &7 \\
   -2   &7-x &7 \\
   -1   &1   &4-x
  \end{vmat}          \\
  &=
  x^3 - 11x^2 + 38x - 40
  =(x - 5)(x - 4)(x - 2)
\end{align*}
So the eigenvalues are $\lambda_1=5$, $\lambda_2=4$, and $\lambda_3=2$.

\pause
To find the eigenvectors associated with the eigenvalue of $5$ 
specialize equation~($*$) for $x=5$.
\begin{equation*}
  \begin{mat}
   -5   &5   &7 \\
   -2   &2   &7 \\
   -1   &1   &-1
  \end{mat}
  \colvec{z_1 \\ z_2 \\ z_3}
  =
  \colvec{0 \\ 0 \\ 0}
\end{equation*}
Gauss's Method gives this solution set; its nonzero elements are the 
eigenvectors.
\begin{equation*}
  V_5=\set{\colvec{1 \\ 1 \\ 0}z_2\suchthat z_2\in\C}
\end{equation*}
\end{frame}
\begin{frame}
Similarly, to find the eigenvectors associated with the eigenvalue of~$4$ 
specialize equation~($*$) for $x=4$.
\begin{equation*}
  \begin{mat}
   -4   &5   &7 \\
   -2   &3   &7 \\
   -1   &1   &0
  \end{mat}
  \colvec{z_1 \\ z_2 \\ z_3}
  =
  \colvec{0 \\ 0 \\ 0}
\end{equation*}
Gauss's Method gives this.
\begin{equation*}
  V_4=\set{\colvec{-7 \\ -7 \\ 1}z_3\suchthat z_3\in\C}
\end{equation*}
\pause
Specializing~($*$) for~$x=2$
\begin{equation*}
  \begin{mat}
   -2   &5   &7 \\
   -2   &5   &7 \\
   -1   &1   &2
  \end{mat}
  \colvec{z_1 \\ z_2 \\ z_3}
  =
  \colvec{0 \\ 0 \\ 0}
\end{equation*}
gives this.
\begin{equation*}
  V_2=\set{\colvec{1 \\ -1 \\ 1}z_3\suchthat z_3\in\C}
\end{equation*}
\end{frame}




% --

\begin{frame}
\ex
To find the eigenvalues and associated eigenvectors for the matrix
\begin{equation*}
  T=
  \begin{mat}
    3  &1  \\
    1  &3
  \end{mat}
\end{equation*}
start with this equation.
\begin{equation*}
  \begin{mat}
    3  &1  \\
    1  &3    
  \end{mat}
  \colvec{b_1 \\ b_2}
  =x\colvec{b_1 \\ b_2}
  \quad\Longrightarrow\quad
  \begin{mat}
    3-x  &1  \\
    1    &3-x    
  \end{mat}
  \colvec{b_1 \\ b_2}
  =\colvec{0 \\ 0}
  \tag{$*$}
\end{equation*}
\pause
That system
has a nontrivial solution if this determinant is nonzero.
\begin{equation*}
  \begin{vmat}
    3-x  &1  \\
    1    &3-x
  \end{vmat}
  =x^2-6x+8
  =(x-2)(x-4)
\end{equation*}
\pause
First take the $x=2$ version of~($*$).
\begin{equation*}
  \begin{linsys}{2}
    1\cdot b_1 &+ &b_2        &= &0 \\
    b_1        &+ &1\cdot b_2 &= &0 
  \end{linsys}
  \quad\Longrightarrow\quad
  V_2=\set{\colvec{b_1 \\ b_2}\suchthat \text{$b_1=-b_2$ where $b_2\in\C$}}
\end{equation*}
Solving the second system is just as easy.
\begin{equation*}
  \begin{linsys}{2}
    -1\cdot b_1 &+ &b_2        &= &0 \\
    b_1        &- &1\cdot b_2 &= &0 
  \end{linsys}
  \quad\Longrightarrow\quad
  V_4=\set{\colvec{b_1 \\ b_2}\suchthat \text{$b_1=b_2$ where $b_2\in\C$}}
\end{equation*}
\end{frame}




\begin{frame}
\ex 
If the matrix is upper diagonal or lower diagonal
\begin{equation*}
  T=
  \begin{mat}
    2 &1 &0 \\
    0 &3 &1 \\
    0 &0 &2
  \end{mat}
\end{equation*}
then the polynomial is easy to factor.
\begin{equation*}
  0=
  \begin{vmat}
    2-x &1   &0 \\
    0   &3-x &1 \\
    0   &0   &2-x
  \end{vmat}
  =(3-x)(2-x)^2
\end{equation*}
\pause
These are the solutions for $\lambda_1=3$.
\begin{equation*}
  \begin{mat}
    -1  &1   &0 \\
    0   &0   &1 \\
    0   &0   &-1
  \end{mat}
  \colvec{z_1 \\ z_2 \\ z_3}
  =
  \colvec{0 \\ 0 \\ 0}
  \quad\Longrightarrow\quad
  V_3=\set{\colvec{1 \\ 1 \\ 0}z_2\suchthat z_2\in\C}
\end{equation*}
These are for $\lambda_2=2$.
\begin{equation*}
  \begin{mat}
    0  &1   &0 \\
    0   &1   &1 \\
    0   &0   &0
  \end{mat}
  \colvec{z_1 \\ z_2 \\ z_3}
  =
  \colvec{0 \\ 0 \\ 0}
  \quad\Longrightarrow\quad
  V_2=\set{\colvec{1 \\ 0 \\ 0}z_1\suchthat z_1\in\C}
\end{equation*}  
\end{frame}




\begin{frame}{Characteristic polynomial}
\df[df:CharacteristicPoly]
\ExecuteMetaData[../jc2.tex]{df:CharacteristicPoly}

\no
\ExecuteMetaData[../jc2.tex]{df:CharacteristicPolyExer}

\pause
\lm[le:MapNonTrivSpHasEigen] 
\ExecuteMetaData[../jc2.tex]{lm:MapNonTrivSpHasEigen}

\pause
\pf
\ExecuteMetaData[../jc2.tex]{pf:MapNonTrivSpHasEigen}
\qed

\pause
\re
This result is why we switched in this chapter 
from working with real number scalars
to complex number scalars.
\end{frame}




\begin{frame}{Eigenspace}
\df[df:Eigenspace]
\ExecuteMetaData[../jc2.tex]{df:Eigenspace}

\ex
Recall that this matrix has three eigenvalues, $5$, $4$, and~$2$.
\begin{equation*}
  T=
  \begin{mat}[r]
    0 &5 &7 \\
   -2 &7 &7 \\
   -1 &1 &4
  \end{mat}
\end{equation*}
Earlier, we found that these are the eigenspaces.
\begin{equation*}
    V_5=\set{\colvec[r]{1 \\ 1 \\ 0}c\suchthat c\in\C}  
    \quad
    V_4=\set{\colvec[r]{-7 \\ -7 \\ 1}c\suchthat c\in\C} 
    \quad
    V_2=\set{\colvec[r]{1 \\ -1 \\ 1}c\suchthat c\in\C}    
\end{equation*}
\end{frame}
\begin{frame}
\lm[le:EigSpaceIsSubSp] 
\ExecuteMetaData[../jc2.tex]{lm:EigSpaceIsSubSp}

\pause
\pf
\ExecuteMetaData[../jc2.tex]{pf:EigSpaceIsSubSp}
\qed
\end{frame}




\begin{frame}
\th[th:DistEValueGivesLIEvecs]
\ExecuteMetaData[../jc2.tex]{th:DistEValueGivesLIEvecs}

\pause
\pf
\ExecuteMetaData[../jc2.tex]{pf:DistEValueGivesLIEvecs0}
\end{frame}
\begin{frame}
\ExecuteMetaData[../jc2.tex]{pf:DistEValueGivesLIEvecs1}
\qed
\end{frame}




\begin{frame}
\ex
This matrix from above has three eigenvalues, $5$, $4$, and~$2$.
\begin{equation*}
  T=
  \begin{mat}
    0 &5 &7 \\
   -2 &7 &7 \\
   -1 &1 &4
  \end{mat}
\end{equation*}
Picking a nonzero vector from each eigenspace we get this linearly independent
set (which is a basis because it has three elements).
\begin{equation*}
    \set{\colvec{1 \\ 1 \\ 0},
         \colvec{-14 \\ -14 \\ 2},
         \colvec{-1/2 \\ 1/2 \\ -1/2}}    
\end{equation*}

\pause
\ex 
This upper-triangular matrix has the eigenvalues $2$ and~$3$
\begin{equation*}
  \begin{mat}
    2 &1 &0 \\
    0 &3 &1 \\
    0 &0 &2
  \end{mat}
\end{equation*}
Picking a vector from each of $V_3$ and~$V_2$ gives this linearly independent set.
\begin{equation*}
  \set{\colvec{1 \\ 1 \\ 0},
       \colvec{2 \\ 0 \\ 0}}
\end{equation*}
\end{frame}



\begin{frame}{A criteria for diagonalizability}
\co[co:DistinctEivenvaluesImpliesDiagonal]
\ExecuteMetaData[../jc2.tex]{co:DistinctEivenvaluesImpliesDiagonal}
\pf
\ExecuteMetaData[../jc2.tex]{pf:DistinctEivenvaluesImpliesDiagonal}
\qed
\end{frame}


%...........................
% \begin{frame}g
% \ExecuteMetaData[../gr3.tex]{GaussJordanReduction}
% \df[def:RedEchForm]
% 
% \end{frame}
\end{document}
