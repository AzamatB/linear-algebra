% Chapter 3, Section 3 _Linear Algebra_ Jim Hefferon
%  http://joshua.smcvt.edu/linalg.html
%  2001-Jun-12
\section{Computing Linear Maps}
The prior section shows that
a linear map is determined by its action on a basis.
In fact, the equation
\begin{equation*}
  h(\vec{v})
  =h(c_1\cdot\vec{\beta}_1+\dots+c_n\cdot\vec{\beta}_n)
  =c_1\cdot h(\vec{\beta}_1)+\dots +c_n\cdot h(\vec{\beta}_n)
\tag*{}\end{equation*}
shows that, if we know the value of the map on the vectors in a basis, then we
can compute the value of the map on any vector $\vec{v}$ at all.
We just need to  
find the $c$'s to express $\vec{v}$ with respect to the basis.

This section gives the scheme
that computes, from the representation of a vector in the domain 
$\rep{\vec{v}}{B}$,
the representation of that vector's image in the codomain 
$\rep{h(\vec{v})}{D}$,
using the representations of 
\( h(\vec{\beta}_1) \), \ldots, \( h(\vec{\beta}_n) \).














\subsection{Representing Linear Maps with Matrices}
\index{homomorphism!matrix representing|(}
\begin{example}  \label{ex:TypLinMapRepByMat}
Consider a map $h$ with domain $\Re^2$ and codomain $\Re^3$, fixing 
\begin{equation*}
   B=\sequence{
               \colvec{2 \\ 0},
               \colvec{1 \\ 4}}
   \quad\text{and}\quad
   D=\sequence{
               \colvec{1 \\ 0 \\ 0},
               \colvec{0 \\ -2 \\ 0},
               \colvec{1 \\ 0 \\ 1}}
\end{equation*}
as the bases for these spaces, that is determined by this action
on the vectors in the domain's basis.
\begin{equation*}
  \colvec{2 \\ 0}
    \mapsunder{h}
  \colvec{1 \\ 1 \\ 1}
  \qquad
  \colvec{1 \\ 4}
    \mapsunder{h}
  \colvec{1 \\ 2 \\ 0}
\end{equation*}
To compute the action of this map on any vector at all from the domain,
we first express $h(\vec{\beta}_1)$ and $h(\vec{\beta}_2)$
with respect to the codomain's basis:
\begin{equation*}
  \colvec{1 \\ 1 \\ 1}=
         0\colvec{1 \\ 0 \\ 0}
         -\frac{1}{2}\colvec{0 \\ -2 \\ 0}
         +1\colvec{1 \\ 0 \\ 1}
  \quad\text{so}\quad
   \rep{ h(\vec{\beta}_1) }{D}=\colvec{0 \\ -1/2 \\ 1}_D           
\end{equation*}
and
\begin{equation*}
  \colvec{1 \\ 2 \\ 0}=
        1\colvec{1 \\ 0 \\ 0}
        -1\colvec{0 \\ -2 \\ 0}
        +0\colvec{1 \\ 0 \\ 1}
  \quad\text{so}\quad
  \rep{ h(\vec{\beta}_2) }{D}=\colvec{1 \\ -1 \\ 0}_D
\end{equation*}
(these are easy to check).
Then, as described in the preamble, for any member $\vec{v}$ of the domain,
we can express the image $h(\vec{v})$ in terms of the $h(\vec{\beta})$'s.
\begin{align*}
  h(\vec{v})
  &=h(c_1\cdot \colvec{2 \\ 0}+c_2\cdot \colvec{1 \\ 4})   \\
  &=c_1\cdot h(\colvec{2 \\ 0})+c_2\cdot h(\colvec{1 \\ 4}) \\
  &=c_1\cdot (
        0\colvec{1 \\ 0 \\ 0}
        \!-\frac{1}{2}\colvec{0 \\ -2 \\ 0}
        \!+1\colvec{1 \\ 0 \\ 1}\!  )
  +c_2\cdot (
        1\colvec{1 \\ 0 \\ 0}
        \!-1\colvec{0 \\ -2 \\ 0}
        \!+0\colvec{1 \\ 0 \\ 1}\!  )      \\
  &=(0c_1+1c_2)\cdot \colvec{1 \\ 0 \\ 0}
   +(-\frac{1}{2}c_1-1c_2)\cdot \colvec{0 \\ -2 \\ 0}
   +(1c_1+0c_2)\cdot \colvec{1 \\ 0 \\ 1}
\end{align*}
Thus,
\begin{center}
  with $\rep{\vec{v}}{B}=\colvec{c_1 \\ c_2}$
  then $\rep{\,h(\vec{v})\,}{D}
  =\colvec{0c_1+1c_2 \\ -(1/2)c_1-1c_2 \\ 1c_1+0c_2}$.
\end{center}
For instance, 
\begin{center}
  with $\rep{\colvec{4 \\ 8}}{B}=\colvec{1 \\ 2}_B$
  then 
  $\rep{\,h(\colvec{4 \\ 8})\,}{D}
   =\colvec{2 \\ -5/2 \\ 1}$.
\end{center}
\end{example}

We will express computations like the one above with a matrix notation.
\begin{equation*}
    \begin{pmatrix}
      0             &1  \\
      -1/2          &-1  \\
      1             &0
    \end{pmatrix}_{B,D}
  \colvec{c_1 \\ c_2}_B
  =
  \colvec{0c_1+1c_2 \\ (-1/2)c_1-1c_2 \\ 1c_1+0c_2}_D
\end{equation*}
In the middle is the argument $\vec{v}$ to the map, 
represented with respect to the domain's basis $B$
by a column vector with components $c_1$ and $c_2$.
On the right is the value $h(\vec{v})$ of the map on that argument,
represented with respect to the codomain's basis $D$
by a column vector with components $0c_1+1c_2$, etc.
The matrix on the left is the new thing.
It consists of the coefficients from the vector on the right,
$0$ and $1$ from the first row, $-1/2$ and $-1$ from the
second row, and $1$ and $0$ from the third row.

This notation simply breaks the parts from the right, 
the coefficients and the $c$'s, out separately on the left, into a vector that
represents the map's argument and a matrix that we will take to
represent the map itself.

\begin{definition} \label{def:MatRepMap}
Suppose that \( V \) and \( W \) are vector spaces of dimensions \( n \) and
\( m \) with bases \( B \) and \( D \),
and that \( \map{h}{V}{W} \) is a linear map.
If
\begin{equation*}
  \rep{h( \vec{\beta}_1 )}{D}
  =
  \colvec{h_{1,1} \\ h_{2,1} \\ \vdots \\ h_{m,1}}_D
  \;\ldots\;
  \rep{h( \vec{\beta}_n )}{D}
  =
  \colvec{h_{1,n} \\ h_{2,n} \\ \vdots \\ h_{m,n}}_D
\end{equation*}
then 
\begin{equation*}
  \rep{h}{B,D}=\generalmatrix{h}{n}{m}_{B,D}
\end{equation*}
is the \definend{matrix representation of \( h \) with respect to \( B, D \)}.%
\index{representation!of a matrix}\index{matrix!representation}%
\index{homomorphism!matrix representing}
\end{definition}

Briefly, the vectors
representing the $h(\vec{\beta})$'s are adjoined to 
make the matrix representing the map. 
\begin{equation*}
  \rep{h}{B,D}=
  \left(\begin{array}{c|@{\hspace{1.5em}}c@{\hspace{1.5em}}|c}
     \vdots                         &       &\vdots    \\
     \rep{\,h(\vec{\beta}_1)\,}{D}  &\cdots &\rep{\,h(\vec{\beta}_n)\,}{D}  \\
     \vdots                         &       &\vdots
  \end{array}\right)
\end{equation*}
Observe that the number of columns~$n$ of the matrix is the 
dimension of the domain of the map 
and the number of rows~$m$ is the dimension of the codomain.

\begin{example}  \label{ex:PolyOneToRThree}
If \( \map{h}{\Re^3}{\polyspace_1} \) is given by
\begin{equation*}
  \colvec{a_1 \\ a_2 \\ a_3}
     \mapsunder{h}
   (2a_1+a_2)+(-a_3)x
\end{equation*}
then where
\begin{equation*}
  B=
  \sequence{\colvec{0 \\ 0 \\ 1},
            \colvec{0 \\ 2 \\ 0},
            \colvec{2 \\ 0 \\ 0} }
  \quad\text{and}\quad
  D=
  \sequence{1+x,-1+x}
\end{equation*}
the action of \( h \) on \( B \) is given by
\begin{equation*}
  \colvec{0 \\ 0 \\ 1}\mapsunder{h}-x
  \qquad \colvec{0 \\ 2 \\ 0}\mapsunder{h}2
  \qquad \colvec{2 \\ 0 \\ 0}\mapsunder{h}4
\end{equation*}
and a simple calculation gives
\begin{equation*}
  \rep{-x}{D}=\colvec{-1/2 \\ -1/2}_D
  \quad \rep{2}{D}=\colvec{1 \\ -1}_D 
  \quad \rep{4}{D}=\colvec{2 \\ -2}_D 
\end{equation*}
showing that this is the matrix representing $h$ with respect to the bases.
\begin{equation*}
  \rep{h}{B,D}
   =
   \begin{pmatrix}
     -1/2  &1   &2  \\
     -1/2  &-1  &-2
   \end{pmatrix}_{B,D}
\end{equation*}
\end{example}

We will use lower case letters for a map, 
upper case for the matrix,
and lower case again for the entries of the matrix.
Thus for the map \( h \), the matrix representing it is \( H \), with
entries \( h_{i,j} \).

\begin{theorem} \label{th:MatMultRepsFuncAppl}
Assume that \( V \) and \( W \) are vector spaces
of dimensions \( n \) and \( m \)
with bases \( B \) and \( D \),
and that \( \map{h}{V}{W} \) is a linear map.
If \( h \) is represented by
\begin{equation*}
  \rep{h}{B,D}=\generalmatrix{h}{n}{m}_{B,D}
\end{equation*}
and \( \vec{v}\in V \) is represented by
\begin{equation*}
  \rep{\vec{v}}{B}=\colvec{c_1 \\ c_2 \\ \vdots \\ c_n}_B
\end{equation*}
then the representation of the image of $\vec{v}$ is this.
\begin{equation*}
  \rep{\, h(\vec{v}) \,}{D}
  =
  \colvec{h_{1,1}c_1+h_{1,2}c_2+\dots+h_{1,n}c_n \\
          h_{2,1}c_1+h_{2,2}c_2+\dots+h_{2,n}c_n \\
          \vdots \\
          h_{m,1}c_1+h_{m,2}c_2+\dots+h_{m,n}c_n}_D
\end{equation*}
\end{theorem}

\begin{proof}
This formalizes \nearbyexample{ex:TypLinMapRepByMat}; 
see \nearbyexercise{exer:MatVecMultRepLinMap}.
\end{proof}

\begin{definition}
\label{def:MatrixVecProd}
The \definend{matrix-vector product}\index{multiplication!matrix-vector}%
\index{matrix!matrix-vector product}
of a \( \nbym{m}{n} \) matrix and a
\( \nbym{n}{1} \) vector is this.
\begin{equation*}
  \generalmatrix{a}{n}{m}
  \colvec{c_1 \\ \vdots \\ c_n}
  =
  \colvec{a_{1,1}c_1+a_{1,2}c_2+\dots+a_{1,n}c_n \\
             a_{2,1}c_1+a_{2,2}c_2+\dots+a_{2,n}c_n \\
             \vdots \\ 
             a_{m,1}c_1+a_{m,2}c_2+\dots+a_{m,n}c_n}
\end{equation*}
\end{definition}

The point of \nearbydefinition{def:MatRepMap} is to generalize
\nearbyexample{ex:TypLinMapRepByMat}.
That is, the point of the definition is 
\nearbytheorem{th:MatMultRepsFuncAppl}: 
the product of the matrix $\rep{h}{B,D}$ and the vector $\rep{\vec{v}}{B}$ 
is the vector $\rep{h(\vec{v})}{D}$.
Briefly, 
application of a linear map is represented by the matrix-vector product 
of the map's representative and the vector's representative.

\begin{example}
With the matrix from \nearbyexample{ex:PolyOneToRThree}
we can calculate where that map sends this vector.
\begin{equation*}
  \vec{v}=\colvec{4 \\ 1 \\ 0}
\end{equation*}
This vector is represented, with respect to the domain basis $B$, by
\begin{equation*}
  \rep{\vec{v}}{B}=\colvec{0 \\ 1/2 \\ 2}_B
\end{equation*}
and so this is the representation of the value $h(\vec{v})$ with respect to
the codomain basis $D$.
\begin{align*}
  \rep{h(\vec{v})}{D}
  &=\begin{pmatrix}
      -1/2  &1   &2  \\
      -1/2  &-1  &-2
    \end{pmatrix}_{B,D}
    \colvec{0 \\ 1/2 \\ 2}_B                            \\
  &=\colvec{(-1/2)\cdot 0+1\cdot (1/2) + 2\cdot 2 \\ 
          (-1/2)\cdot 0-1\cdot (1/2) - 2\cdot 2}_D
  =\colvec{9/2 \\ -9/2}_D
\end{align*}
To find $h(\vec{v})$ itself, not its representation,
take $(9/2)(1+x)-(9/2)(-1+x)=9$.
\end{example}

\begin{example}
Let \( \map{\pi}{\Re^3}{\Re^2} \) be projection onto the \( xy \)-plane.
To give a matrix representing this map, we first fix bases.
\begin{equation*}
  B=\sequence{
              \colvec{1 \\ 0 \\ 0},
              \colvec{1 \\ 1 \\ 0},
              \colvec{-1 \\ 0 \\ 1} }
  \qquad
  D=\sequence{
              \colvec{2 \\ 1},
              \colvec{1 \\ 1} }
\end{equation*}
For each vector in the domain's basis, we find its image under the map. 
\begin{equation*}
  \colvec{1 \\ 0 \\ 0}\mapsunder{\pi}\colvec{1 \\ 0}
  \quad
  \colvec{1 \\ 1 \\ 0}\mapsunder{\pi}\colvec{1 \\ 1}
  \quad
  \colvec{-1 \\ 0 \\ 1}\mapsunder{\pi}\colvec{-1 \\ 0}
\end{equation*}
Then we find the representation of each image with respect to the codomain's
basis 
\begin{equation*}
  \rep{\colvec{1 \\ 0}}{D}=\colvec{1 \\ -1}
  \quad
  \rep{\colvec{1 \\ 1}}{D}=\colvec{0 \\ 1}
  \quad
  \rep{\colvec{-1 \\ 0}}{D}=\colvec{-1 \\ 1}
\end{equation*}
(these are easily checked).
Finally, adjoining these representations gives the matrix representing 
\( \pi \) with respect to \( B,D \).
\begin{equation*}
    \rep{\pi}{B,D}
    =\begin{pmatrix}
      1  &0  &-1  \\
      -1 &1  &1
    \end{pmatrix}_{B,D}
\end{equation*}
We can illustrate \nearbytheorem{th:MatMultRepsFuncAppl} by computing
the matrix-vector product representing the following statement 
about the projection map.
\begin{equation*}
  \pi(
     \colvec{2 \\ 2 \\ 1}
     )=\colvec{2 \\ 2}
\end{equation*}
Representing this vector from the domain
with respect to the domain's basis
\begin{equation*}
   \rep{\colvec{2 \\ 2 \\ 1}}{B}=
        \colvec{1 \\ 2 \\ 1}_B
\end{equation*}
gives this matrix-vector product.
\begin{equation*}
   \rep{ \,\pi(\colvec{2 \\ 1 \\ 1})\,}{D}=
      \begin{pmatrix}
        1  &0  &-1  \\
        -1 &1  &1
      \end{pmatrix}_{B,D}
   \colvec{1 \\ 2 \\ 1}_B
   =
   \colvec{0 \\ 2}_D
\end{equation*}
Expanding this representation into a linear combination of vectors from
\( D \) 
\begin{equation*}
   0\cdot\colvec{2 \\ 1}
   +2\cdot\colvec{1 \\ 1}
   =
   \colvec{2 \\ 2}
\end{equation*}
checks that the map's action is indeed
reflected in the operation of the matrix.
(We will sometimes compress these three displayed equations into one 
\begin{equation*}
  \colvec{2 \\ 2 \\ 1}=\colvec{1 \\ 2 \\ 1}_B
    \;\overset{h}{\underset{H}{\longmapsto}}\;
  \colvec{0 \\ 2}_D=\colvec{2 \\ 2}    
\end{equation*}
in the course of a calculation.)
\end{example}

We now have two ways to compute the effect of projection,
the straightforward formula that drops each three-tall vector's third component
to make a two-tall vector, 
and the above formula that uses representations and matrix-vector 
multiplication.
Compared to the first way, the second way might seem complicated.
However, it has advantages.
The next example shows that giving a formula for some maps is 
simplified by this new scheme.  

\begin{example} \label{exam:RepsOfRigidPlaneMaps}
To represent a  \definend{rotation}\index{rotation (or turning)!represented}
map $\map{t_{\theta}}{\Re^2}{\Re^2}$ that 
turns all vectors in the plane counterclockwise through an angle $\theta$
\begin{center}  \small
  \includegraphics{ch3.15}
\end{center}
we start by fixing bases.
Using $\stdbasis_2$ both as a domain basis and as a codomain basis is natural,
Now, we find the image under the map of each
vector in the domain's basis.
\begin{equation*}
  \colvec{1 \\ 0}\mapsunder{t_\theta}\colvec{\cos\theta \\ \sin\theta}
  \qquad
  \colvec{0 \\ 1}\mapsunder{t_\theta}\colvec{-\sin\theta \\ \cos\theta}
\end{equation*}
Then we represent these images with respect to the codomain's basis.
Because this basis is $\stdbasis_2$, vectors are represented by themselves.
Finally, adjoining the representations gives the matrix representing the map.
\begin{equation*}
  \rep{t_\theta}{\stdbasis_2,\stdbasis_2}
  =
  \begin{pmatrix}
    \cos\theta  &-\sin\theta \\
    \sin\theta  &\cos\theta
  \end{pmatrix}
\end{equation*}
The advantage of this scheme is that just by knowing how to
represent the image of the two basis vectors, 
we get a formula that tells us the image of any vector at 
all; here a vector rotated by $\theta=\pi/6$.
\begin{equation*}
  \colvec{3  \\ -2}\;\mapsunder{t_{\pi/6}}\;
  \begin{pmatrix}
    \sqrt{3}/2  &-1/2  \\
     1/2        &\sqrt{3}/2
  \end{pmatrix}
  \colvec{3  \\ -2}
  \approx
  \colvec{3.598 \\ -0.232}  
\end{equation*}
(Again, we are using the fact that, with respect to $\stdbasis_2$, 
vectors represent themselves.)
\end{example}

We have already seen the addition and scalar multiplication
operations of matrices and  
the dot product operation of vectors. 
Matrix-vector multiplication is a new operation in the arithmetic of 
vectors and matrices.
Nothing in \nearbydefinition{def:MatrixVecProd} requires us to view
it in terms of representations.
We can get some insight into this operation 
by turning away from  what is being represented, and instead focusing on
how the entries combine.

\begin{example}
In the definition
the width of the matrix equals the height of the vector.
Hence, the first product below is defined while the second is not. 
\begin{equation*}
    \begin{pmatrix}
      1  &0  &0  \\
      4  &3  &1
    \end{pmatrix}
  \colvec{1 \\ 0 \\ 2}
  =
  \colvec{1 \\ 6}
  \qquad
    \begin{pmatrix}
      1  &0  &0  \\
      4  &3  &1
    \end{pmatrix}
  \colvec{1 \\ 0}
\end{equation*}
One reason that this product is not defined is purely formal:~the
definition requires that the sizes match, and these sizes don't match.
Behind the formality, though, 
is a reason why we will leave it undefined\Dash the
matrix represents a map with a three-dimensional domain while the 
vector represents a member of a two-dimensional space.
\end{example}

A good way to view a matrix-vector product is
as the dot products of the rows of the matrix with the column vector.
\begin{equation*}
    \begin{pmatrix}
               &\vdots                         \\
      a_{i,1}  &a_{i,2}  &\ldots   &a_{i,n}    \\
               &\vdots
    \end{pmatrix}
  \colvec{c_1 \\ c_2 \\ \vdots \\ c_n}
  =
  \colvec{\vdots \\ a_{i,1}c_1+a_{i,2}c_2+\ldots+a_{i,n}c_n \\ \vdots}
\end{equation*}
Looked at in this row-by-row way,
this new operation generalizes dot product.

Matrix-vector product can also be viewed column-by-column.
\begin{align*}
           \generalmatrix{h}{n}{m}
           \colvec{c_1 \\ c_2 \\ \vdots \\ c_n}
  &=\colvec{h_{1,1}c_1+h_{1,2}c_2+\dots+h_{1,n}c_n \\
               h_{2,1}c_1+h_{2,2}c_2+\dots+h_{2,n}c_n \\
               \vdots \\
               h_{m,1}c_1+h_{m,2}c_2+\dots+h_{m,n}c_n}    \\
  &=c_1\colvec{h_{1,1} \\ h_{2,1} \\ \vdots \\ h_{m,1}}
%   +c_2\colvec{h_{1,2} \\ h_{2,2} \\ \vdots \\ h_{m,2}}
   +\dots
   +c_n\colvec{h_{1,n} \\ h_{2,n} \\ \vdots \\ h_{m,n}}
\end{align*}

\begin{example}
\begin{equation*}
    \begin{pmatrix}
      1  &0  &-1  \\
      2  &0  &3
    \end{pmatrix}
  \colvec{2 \\ -1 \\ 1}
  =
  2\colvec{1 \\ 2}
  -1\colvec{0 \\ 0}
  +1\colvec{-1 \\ 3}
  =
  \colvec{1 \\ 7}
\end{equation*}
\end{example}

The result has the
columns of the matrix weighted by the entries of the vector.
This way of looking at it
brings us back to the objective stated at the start of this section, to compute
\( h(c_1\vec{\beta}_1+\dots+c_n\vec{\beta}_n) \)
as
\( c_1h(\vec{\beta}_1)+\dots+c_nh(\vec{\beta}_n) \).

We began this section
by noting that the equality of these two enables us to compute the action 
of $h$ on any
argument knowing only $h(\vec{\beta}_1)$, \ldots, $h(\vec{\beta}_n)$.
We have developed this into a scheme to
compute the action of the map by taking 
the matrix-vector product of the matrix representing the 
map and the vector representing the argument.
In this way, any linear map is represented with respect to some bases
by a matrix.
In the next subsection, we will show the converse, that any matrix represents
a linear map.


\begin{exercises}
  \recommended \item  
    Multiply the matrix
    \begin{equation*}
      \begin{pmatrix}
        1  &3  &1  \\
        0  &-1 &2  \\
        1  &1  &0
      \end{pmatrix}
    \end{equation*}
    by each vector (or state ``not defined'').
    \begin{exparts*}
      \partsitem \( \colvec{2 \\ 1 \\ 0} \)
      \partsitem \( \colvec{-2 \\ -2} \)
      \partsitem \( \colvec{0 \\ 0 \\ 0} \)
    \end{exparts*}
    \begin{answer}
      \begin{exparts*}
        \partsitem \(
           \colvec{1\cdot 2+3\cdot 1+1\cdot 0         \\
                        0\cdot 2+(-1)\cdot 1+2\cdot 0 \\
                        1\cdot 2+1\cdot 1+0\cdot 0}
           =\colvec{5 \\ -1 \\ 3}   \)
        \partsitem Not defined.
        \partsitem \(  \colvec{0 \\ 0 \\ 0}  \)
      \end{exparts*}  
    \end{answer}
  \item 
    Perform, if possible, each matrix-vector multiplication.
    \begin{exparts*}
      \partsitem $\begin{pmatrix}
                    2  &1  \\
                    3  &-1/2
                  \end{pmatrix}
                  \colvec{4  \\ 2}$
      \partsitem $\begin{pmatrix}
                    1  &1  &0 \\
                    -2 &1  &0
                  \end{pmatrix}
                  \colvec{1 \\ 3 \\ 1}$
      \partsitem $\begin{pmatrix}
                    1  &1  \\
                    -2  &1
                  \end{pmatrix}
                  \colvec{1 \\ 3 \\ 1}$
    \end{exparts*}
    \begin{answer}
      \begin{exparts*}
        \partsitem $\colvec{2\cdot 4 +1\cdot 2 \\
                            3\cdot 4-(1/2)\cdot 2}
                   =\colvec{10 \\ 11}$
        \partsitem $\colvec{4 \\ 1}$
        \partsitem Not defined.
      \end{exparts*}
    \end{answer}
  \recommended \item  
    Solve this matrix equation.
    \begin{equation*}
      \begin{pmatrix}
        2  &1  &1  \\
        0  &1  &3  \\
        1  &-1 &2
      \end{pmatrix}
      \colvec{x \\ y \\ z}
      =\colvec{8 \\ 4 \\ 4}
    \end{equation*}
    \begin{answer}
      Matrix-vector multiplication gives rise to a linear system.
      \begin{equation*}
        \begin{linsys}{3}
          2x  &+  &y  &+  &z  &=  &8  \\
              &   &y  &+  &3z &=  &4  \\
           x  &-  &y  &+  &2z &=  &4 
        \end{linsys}
      \end{equation*}
      Gaussian reduction shows that \( z=1 \), \( y=1 \), and \( x=3 \).  
    \end{answer}
  \recommended \item 
    For a homomorphism from \( \polyspace_2 \) to \( \polyspace_3 \) that sends
    \begin{equation*}
      1\mapsto 1+x,
      \quad
      x\mapsto 1+2x,
      \quad\text{and}\quad
      x^2\mapsto x-x^3
    \end{equation*}
    where does \( 1-3x+2x^2 \) go?
    \begin{answer}
      Here are two ways to get the answer.

      First, obviously $1-3x+2x^2=1\cdot 1-3\cdot x+2\cdot x^2$, and so we can
      apply the general property of preservation of combinations to get
      $h(1-3x+2x^2)
       =h(1\cdot 1-3\cdot x+2\cdot x^2) 
       =1\cdot h(1)-3\cdot h(x)+2\cdot h(x^2) 
       =1\cdot (1+x)-3\cdot (1+2x)+2\cdot (x-x^3) 
       =-2-3x-2x^3$.

      The other way uses the computation scheme developed in this subsection.
      Because we know where these elements of the space go, we consider
      this basis \( B=\sequence{1,x,x^2} \) for the domain.
      Arbitrarily, we can take \( D=\sequence{1,x,x^2,x^3} \)
      as a basis for the codomain.
      With those choices, we have that
      \begin{equation*}
        \rep{h}{B,D}
        =\begin{pmatrix}
           1   &1  &0  \\
           1   &2  &1  \\
           0   &0  &0  \\
           0   &0  &-1
         \end{pmatrix}_{B,D}
      \end{equation*}
      and, as
      \begin{equation*}
        \rep{1-3x+2x^2}{B}=\colvec{1 \\ -3 \\ 2}_B
      \end{equation*}
      the matrix-vector multiplication calculation gives this.
      \begin{equation*}
        \rep{h(1-3x+2x^2)}{D}=
         \begin{pmatrix}
           1   &1  &0  \\
           1   &2  &1  \\
           0   &0  &0  \\
           0   &0  &-1
         \end{pmatrix}_{B,D}
         \colvec{1 \\ -3 \\ 2}_B
         =\colvec{-2 \\ -3 \\ 0 \\ -2}_D
      \end{equation*}
      Thus, \( h(1-3x+2x^2)
              =-2\cdot 1-3\cdot x+0\cdot x^2-2\cdot x^3
              =-2-3x-2x^3 \),
      as above.  
    \end{answer}
  \recommended \item  
    Assume that \( \map{h}{\Re^2}{\Re^3} \) is determined by 
    this action.
    \begin{equation*}
      \colvec{1 \\ 0}\mapsto\colvec{2 \\ 2 \\ 0}
      \qquad
      \colvec{0 \\ 1}\mapsto\colvec{0 \\ 1 \\ -1}
    \end{equation*}
    Using the standard bases, find
    \begin{exparts}
      \partsitem the matrix representing this map;
      \partsitem a general formula for \( h(\vec{v}) \).
    \end{exparts}
    \begin{answer}
      Again, as recalled in the subsection, 
      with respect to $\stdbasis_i$, a column vector represents itself.
      \begin{exparts}
        \partsitem To represent \( h \) with respect 
          to \( \stdbasis_2,\stdbasis_3 \) we take 
          the images of the basis vectors from the domain,
          and represent them with respect to the basis for the codomain.
          \begin{equation*}
            \rep{\,h(\vec{e}_1)\,}{\stdbasis_3}
            =\rep{\colvec{2 \\ 2 \\ 0}}{\stdbasis_3}
            =\colvec{2 \\ 2 \\ 0}
            \qquad
            \rep{\,h(\vec{e}_2)\,}{\stdbasis_3}
            =\rep{\colvec{0 \\ 1 \\ -1}}{\stdbasis_3}
            =\colvec{0 \\ 1 \\ -1}
          \end{equation*}
          These are adjoined to make the matrix.
          \begin{equation*}
            \rep{h}{\stdbasis_2,\stdbasis_3}=
            \begin{pmatrix}
              2  &0  \\
              2  &1  \\
              0  &-1
            \end{pmatrix}
          \end{equation*}
        \partsitem For any \( \vec{v} \) in the domain \( \Re^2 \),
          \begin{equation*}
            \rep{\vec{v}}{\stdbasis_2}
            =\rep{\colvec{v_1 \\ v_2}}{\stdbasis_2}
            =\colvec{v_1 \\ v_2}
          \end{equation*}
          and so
          \begin{equation*}
            \rep{\,h(\vec{v})\,}{\stdbasis_3}
            =\begin{pmatrix}
              2  &0  \\
              2  &1  \\
              0  &-1
            \end{pmatrix}
            \colvec{v_1 \\ v_2}
            =\colvec{2v_1 \\ 2v_1+v_2 \\ -v_2}
          \end{equation*}
          is the desired representation.
      \end{exparts}  
    \end{answer}
  \recommended \item  
    Let \( \map{d/dx}{\polyspace_3}{\polyspace_3} \) be the derivative
    transformation.
    \begin{exparts}
      \partsitem Represent \( d/dx \) with respect to \( B,B \) where
        \( B=\sequence{1,x,x^2,x^3} \).
      \partsitem Represent \( d/dx \) with respect to \( B,D \) where
        \( D=\sequence{1,2x,3x^2,4x^3} \).
    \end{exparts}
    \begin{answer}
      \begin{exparts*}
        \partsitem 
        We must first find the image of each vector from the domain's basis,
        and then represent that image with respect to the codomain's basis.
        \begin{equation*}
          \rep{\frac{d\,1}{dx}}{B}=\colvec{0 \\ 0 \\ 0 \\ 0}
          \quad
          \rep{\frac{d\,x}{dx}}{B}=\colvec{1 \\ 0 \\ 0 \\ 0}
          \quad
          \rep{\frac{d\,x^2}{dx}}{B}=\colvec{0 \\ 2 \\ 0 \\ 0}
          \quad
          \rep{\frac{d\,x^3}{dx}}{B}=\colvec{0 \\ 0 \\ 3 \\ 0}
        \end{equation*}
        Those representations are then adjoined to make the matrix
        representing the map.
        \begin{equation*}
          \rep{\frac{d}{dx}}{B,B}=
          \begin{pmatrix}
            0  &1  &0  &0  \\
            0  &0  &2  &0  \\
            0  &0  &0  &3  \\
            0  &0  &0  &0 
          \end{pmatrix}  
        \end{equation*}
        \partsitem Proceeding as in the prior item, we represent the images
          of the domain's basis vectors
        \begin{equation*}
          \rep{\frac{d\,1}{dx}}{B}=\colvec{0 \\ 0 \\ 0 \\ 0}
          \quad
          \rep{\frac{d\,x}{dx}}{B}=\colvec{1 \\ 0 \\ 0 \\ 0}
          \quad
          \rep{\frac{d\,x^2}{dx}}{B}=\colvec{0 \\ 1 \\ 0 \\ 0}
          \quad
          \rep{\frac{d\,x^3}{dx}}{B}=\colvec{0 \\ 0 \\ 1 \\ 0}
        \end{equation*}
        and adjoin to make the matrix.
        \begin{equation*}
          \rep{\frac{d}{dx}}{B,D}=
          \begin{pmatrix}
            0  &1  &0  &0  \\
            0  &0  &1  &0  \\
            0  &0  &0  &1  \\
            0  &0  &0  &0
          \end{pmatrix}  
        \end{equation*}
      \end{exparts*}  
    \end{answer}
  \recommended \item 
    Represent each linear map with respect to each pair of bases.
    \begin{exparts}
      \partsitem \( \map{d/dx}{\polyspace_n}{\polyspace_n} \) with respect to
        \( B,B \) where \( B=\sequence{1,x,\dots,x^n} \), given by
        \begin{equation*}
            a_0+a_1x+a_2x^2+\dots+a_nx^n
            \mapsto
            a_1+2a_2x+\dots+na_nx^{n-1}
        \end{equation*}
      \partsitem \( \map{\int}{\polyspace_n}{\polyspace_{n+1}} \) 
        with respect to
        \( B_n,B_{n+1} \) where \( B_i=\sequence{1,x,\dots,x^i} \), given by
        \begin{equation*}
            a_0+a_1x+a_2x^2+\dots+a_nx^n
            \mapsto
            a_0x+\frac{a_1}{2}x^2+\dots+\frac{a_n}{n+1}x^{n+1}
        \end{equation*}
      \partsitem \( \map{\int^1_0}{\polyspace_n}{\Re} \) with respect to
        \( B,\stdbasis_1 \) where \( B=\sequence{1,x,\dots,x^n} \)
        and \( \stdbasis_1=\sequence{1} \), given by
        \begin{equation*}
            a_0+a_1x+a_2x^2+\dots+a_nx^n
            \mapsto
            a_0+\frac{a_1}{2}+\dots+\frac{a_n}{n+1}
        \end{equation*}
      \partsitem \( \map{\text{eval}_3}{\polyspace_n}{\Re} \) with respect to
        \( B,\stdbasis_1 \) where \( B=\sequence{1,x,\dots,x^n} \)
        and \( \stdbasis_1=\sequence{1} \), given by
        \begin{equation*}
            a_0+a_1x+a_2x^2+\dots+a_nx^n
            \mapsto
            a_0+a_1\cdot 3+a_2\cdot 3^2+\dots+a_n\cdot 3^n
        \end{equation*}
      \partsitem \( \map{\text{slide}_{-1}}{\polyspace_n}{\polyspace_n} \) 
        with respect
        to \( B,B \) where \( B=\sequence{1,x,\ldots,x^n} \), given by
        \begin{equation*}
            a_0+a_1x+a_2x^2+\dots+a_nx^n
            \mapsto
            a_0+a_1\cdot (x+1)+\dots+a_n\cdot (x+1)^n
        \end{equation*}
    \end{exparts}
    \begin{answer}
      For each, we must find the image of each of the domain's basis vectors,
      represent each image with respect to the codomain's basis,
      and then adjoin those representations to get the matrix.
      \begin{exparts}
        \partsitem The basis vectors from the domain have these images
          \begin{equation*}
            1\mapsto 0  
            \quad x\mapsto 1  
            \quad x^2\mapsto 2x  
            \quad \ldots
          \end{equation*}
          and these images are represented with respect to the codomain's
          basis in this way.
          \begin{equation*}
            \rep{0}{B}=\colvec{0 \\ 0 \\ 0 \\ \vdots \\ \  \\  \ }
            \quad
            \rep{1}{B}=\colvec{1 \\ 0 \\ 0 \\ \vdots \\ \  \\ \ }
            \quad
            \rep{2x}{B}=\colvec{0 \\ 2 \\ 0 \\ \vdots \\ \  \\ \ }
            \quad\ldots\quad
            \rep{nx^{n-1}}{B}=\colvec{0 \\ 0 \\ 0 \\ \vdots \\ n \\ 0}
          \end{equation*}
          The matrix
          \begin{equation*}
            \rep{\frac{d}{dx}}{B,B}
            =\begin{pmatrix}
              0  &1  &0  &\ldots  &0  \\
              0  &0  &2  &\ldots  &0  \\
                 &\vdots             \\
              0  &0  &0  &\ldots  &n  \\
              0  &0  &0  &\ldots  &0
            \end{pmatrix}
          \end{equation*}
          has $n+1$ rows and columns.
       \partsitem Once the images under this map of the domain's basis
          vectors are determined
          \begin{equation*}
            1\mapsto x 
            \quad x\mapsto x^2/2  
            \quad x^2\mapsto x^3/3 
            \quad \ldots
          \end{equation*}
          then they can be represented with respect to the codomain's basis
          \begin{equation*}
            \rep{x}{B_{n+1}}=\colvec{0 \\ 1 \\ 0 \\ \vdots \\ \ }
            \quad
            \rep{x^2/2}{B_{n+1}}=\colvec{0 \\ 0 \\ 1/2 \\ \vdots \\ \ }
            \quad\ldots\quad
            \rep{x^{n+1}/(n+1)}{B_{n+1}}
                     =\colvec{0 \\ 0 \\ 0 \\ \vdots \\ 1/(n+1)}
          \end{equation*}
          and put together to make the matrix.
          \begin{equation*}
            \rep{\int}{B_{n},B_{n+1}}
            =\begin{pmatrix}
              0  &0  &\ldots  &0      &0  \\
              1  &0  &\ldots  &0      &0  \\
              0  &1/2&\ldots  &0      &0  \\
                 &\vdots                  \\
              0  &0  &\ldots  &0      &1/(n+1)
            \end{pmatrix}
          \end{equation*}
        \partsitem The images of the basis vectors of the domain are
          \begin{equation*} 
            1\mapsto 1 
            \quad x\mapsto 1/2 
            \quad x^2\mapsto 1/3 
            \quad \ldots
          \end{equation*}
          and they are represented with respect to the codomain's basis as
          \begin{equation*}
            \rep{1}{\stdbasis_1}=1
            \quad \rep{1/2}{\stdbasis_1}=1/2
            \quad \ldots
          \end{equation*}
          so the matrix is 
          \begin{equation*}
            \rep{\int}{B,\stdbasis_1}
            =\begin{pmatrix}
              1  &1/2 &\cdots  &1/n    &1/(n+1)
            \end{pmatrix}
          \end{equation*}
          (this is an $\nbym{1}{(n+1)}$ matrix).
        \partsitem Here, the images of the domain's basis vectors are
          \begin{equation*}  
            1\mapsto 1 
            \quad x\mapsto 3 
            \quad x^2\mapsto 9
            \quad \ldots 
          \end{equation*}
          and they are represented in the codomain as
          \begin{equation*}
            \rep{1}{\stdbasis_1}=1
            \quad\rep{3}{\stdbasis_1}=3
            \quad\rep{9}{\stdbasis_1}=9
            \quad \ldots 
          \end{equation*}
          and so the matrix is this.
          \begin{equation*}
            \rep{\int_0^1}{B,\stdbasis_1}
            =\begin{pmatrix}
              1  &3   &9  &\cdots  &3^n
            \end{pmatrix}
          \end{equation*}
        \partsitem The images of the basis vectors from the domain are
          \begin{equation*} 
          1\mapsto 1 
          \quad x\mapsto x+1=1+x  
          \quad x^2\mapsto (x+1)^2=1+2x+x^2  
          \quad x^3\mapsto (x+1)^3=1+3x+3x^2+x^3  
          \quad \ldots 
          \end{equation*}
          which are represented as 
          \begin{equation*}
            \rep{1}{B}=\colvec{1 \\ 0 \\ 0 \\ 0 \\ \vdots \\ 0}
            \quad
            \rep{1+x}{B}=\colvec{1 \\ 1 \\ 0 \\ 0 \\ \vdots \\ 0}
            \quad
            \rep{1+2x+x^2}{B}=\colvec{1 \\ 2 \\ 1 \\ 0 \\ \vdots \\ 0}
            \quad\ldots
          \end{equation*}
          The resulting matrix
          \begin{equation*}
            \renewcommand{\arraystretch}{1.2}
            \rep{\int}{B,B}
            =\begin{pmatrix}
              1  &1   &1  &1  &\ldots  &1           \\
              0  &1   &2  &3  &\ldots  &\binom{n}{2}  \\
              0  &0   &1  &3  &\ldots  &\binom{n}{3}  \\
                 &\vdots                        \\
              0  &0   &0  &   &\ldots  &1
            \end{pmatrix}
          \end{equation*}
          is \definend{Pascal's triangle}\index{Pascal's triangle}
          (recall that $\binom{n}{r}$ is the number of ways to choose $r$
          things, without order and without repetition,
          from a set of size $n$). 
      \end{exparts}  
    \end{answer}
  \item 
    Represent the identity map on any nontrivial
    space with respect to \( B,B \), where \( B \) is any basis.
    \begin{answer}
      Where the space is \( n \)-dimensional,
      \begin{equation*}
        \rep{\text{id}}{B,B}=
        \begin{pmatrix}
          1  &0  \ldots  &0  \\
          0  &1  \ldots  &0  \\
             &\vdots         \\
          0  &0  \ldots  &1
        \end{pmatrix}_{B,B}
      \end{equation*}
      is the $\nbyn{n}$ identity matrix.  
    \end{answer}
  \item 
    Represent, with respect to the natural basis, 
    the transpose transformation on the space 
    \( \matspace_{\nbyn{2}} \) of $\nbyn{2}$ matrices.
    \begin{answer}
      Taking this as the natural basis
      \begin{equation*}
        B=\sequence{\vec{\beta}_1,\vec{\beta}_2,\vec{\beta}_3,\vec{\beta}_4}
         =\sequence{
            \begin{pmatrix}
              1  &0  \\
              0  &0
            \end{pmatrix},
            \begin{pmatrix}
              0  &1  \\
              0  &0
            \end{pmatrix},
            \begin{pmatrix}
              0  &0  \\
              1  &0
            \end{pmatrix},
            \begin{pmatrix}
              0  &0  \\
              0  &1
            \end{pmatrix}   }
      \end{equation*}
      the transpose map acts in this way
      \begin{equation*}
        \vec{\beta}_1\mapsto\vec{\beta}_1 
        \quad \vec{\beta}_2\mapsto\vec{\beta}_3 
        \quad \vec{\beta}_3\mapsto\vec{\beta}_2 
        \quad \vec{\beta}_4\mapsto\vec{\beta}_4  
      \end{equation*}
      so that representing the images with respect to the codomain's
      basis and adjoining those column vectors together gives this.
      \begin{equation*}
        \rep{\text{trans}}{B,B}=
        \begin{pmatrix}
          1  &0  &0  &0  \\
          0  &0  &1  &0  \\
          0  &1  &0  &0  \\
          0  &0  &0  &1
        \end{pmatrix}_{B,B}
      \end{equation*}   
     \end{answer}
  \item 
     Assume that 
     \( B=\sequence{\vec{\beta}_1,\vec{\beta}_2,\vec{\beta}_3,\vec{\beta}_4} \)
     is a basis for a vector space.
     Represent with respect to \( B,B \) the transformation that is determined
     by each.
     \begin{exparts}
       \partsitem \( \vec{\beta}_1\mapsto\vec{\beta}_2 \),
         \( \vec{\beta}_2\mapsto\vec{\beta}_3 \),
         \( \vec{\beta}_3\mapsto\vec{\beta}_4 \),
         \( \vec{\beta}_4\mapsto\zero \)
       \partsitem \( \vec{\beta}_1\mapsto\vec{\beta}_2 \),
         \( \vec{\beta}_2\mapsto\zero \),
         \( \vec{\beta}_3\mapsto\vec{\beta}_4 \),
         \( \vec{\beta}_4\mapsto\zero \)
       \partsitem \( \vec{\beta}_1\mapsto\vec{\beta}_2 \),
         \( \vec{\beta}_2\mapsto\vec{\beta}_3 \),
         \( \vec{\beta}_3\mapsto\zero \),
         \( \vec{\beta}_4\mapsto\zero \)
     \end{exparts}
     \begin{answer}
       \begin{exparts}
         \partsitem With respect to the basis of the codomain, the images of
           the members of the basis of the domain are represented as
           \begin{equation*}
             \rep{\vec{\beta}_2}{B}=\colvec{0 \\ 1 \\ 0 \\ 0}
             \quad
             \rep{\vec{\beta}_3}{B}=\colvec{0 \\ 0 \\ 1 \\ 0}
             \quad
             \rep{\vec{\beta}_4}{B}=\colvec{0 \\ 0 \\ 0 \\ 1}
             \quad
             \rep{\zero}{B}=\colvec{0 \\ 0 \\ 0 \\ 0}
           \end{equation*}
           and consequently, the matrix representing the transformation is 
           this.
           \begin{equation*}
             \begin{pmatrix}
                   0  &0  &0  &0  \\
                   1  &0  &0  &0  \\
                   0  &1  &0  &0  \\
                   0  &0  &1  &0
                 \end{pmatrix}  
         \end{equation*}
         \partsitem 
              $\begin{pmatrix}
                   0  &0  &0  &0  \\
                   1  &0  &0  &0  \\
                   0  &0  &0  &0  \\
                   0  &0  &1  &0
                 \end{pmatrix}$
         \partsitem 
               $\begin{pmatrix}
                   0  &0  &0  &0  \\
                   1  &0  &0  &0  \\
                   0  &1  &0  &0  \\
                   0  &0  &0  &0
                 \end{pmatrix}$
       \end{exparts}  
     \end{answer}
  \item
    \nearbyexample{exam:RepsOfRigidPlaneMaps} shows how to represent
    the rotation transformation of the plane with respect to the 
    standard basis.
    Express these other transformations also with respect to the standard
    basis.
    \begin{exparts}
      \partsitem the \definend{dilation} map $d_s$, which multiplies 
        all vectors by the same scalar $s$\index{dilation!representing}
      \partsitem the \definend{reflection} map $f_\ell$, which reflects all
        all vectors across a line $\ell$ through the origin
    \end{exparts}
    \begin{answer}
      \begin{exparts}
        \partsitem The picture of $\map{d_s}{\Re^2}{\Re^2}$ is this.
          \begin{center}  \small
            \includegraphics{ch3.75}
          \end{center}
         This map's effect on the vectors in the standard basis for the domain 
         is 
         \begin{equation*} 
            \colvec{1 \\ 0}\mapsunder{d_s}\colvec{s \\ 0}
            \qquad
            \colvec{0 \\ 1}\mapsunder{d_s}\colvec{0 \\ s}
         \end{equation*}
         and those images are represented with respect to the 
         codomain's basis (again, the standard basis) by themselves.
         \begin{equation*} 
           \rep{\colvec{s \\ 0}}{\stdbasis_2}=\colvec{s \\ 0}
           \qquad
           \rep{\colvec{0 \\ s}}{\stdbasis_2}=\colvec{0 \\ s}
         \end{equation*}
         Thus the representation of the dilation map is this.
         \begin{equation*}
           \rep{d_s}{\stdbasis_2,\stdbasis_2}
           =\begin{pmatrix}
              s  &0  \\
              0  &s
           \end{pmatrix}
        \end{equation*}
      \partsitem The picture of $\map{f_\ell}{\Re^2}{\Re^2}$ is this.
         \begin{center}  \small
           \includegraphics{ch3.76}
        \end{center}
       Some calculation (see Exercise~I.\ref{exer:RigidPlaneMapsAutos})
       shows that when the line has slope $k$ 
       \begin{equation*}
         \colvec{1 \\ 0}
             \mapsunder{f_\ell}\colvec{(1-k^2)/(1+k^2) \\ 2k/(1+k^2)}
         \qquad
         \colvec{0 \\ 1}
             \mapsunder{f_\ell}\colvec{2k/(1+k^2) \\ -(1-k^2)/(1+k^2)}
       \end{equation*}
       (the case of a line with undefined slope is separate but easy) 
       and so the matrix representing reflection is this.
       \begin{equation*}
         \rep{f_\ell}{\stdbasis_2,\stdbasis_2}
         =\frac{1}{1+k^2}\cdot\begin{pmatrix}
            1-k^2  &2k  \\
            2k     &-(1-k^2)
         \end{pmatrix}
       \end{equation*}
      \end{exparts}  
    \end{answer}
  \recommended \item  
    Consider a linear transformation of \( \Re^2 \) determined
    by these two.
    \begin{equation*}
      \colvec{1 \\ 1}\mapsto\colvec{2 \\ 0}
      \qquad
      \colvec{1 \\ 0}\mapsto\colvec{-1 \\ 0}
    \end{equation*}
    \begin{exparts}
      \partsitem Represent this transformation with respect to the standard
        bases.
      \partsitem Where does the transformation send this vector?
        \begin{equation*}
          \colvec{0 \\ 5}
        \end{equation*}
      \partsitem Represent this transformation with respect to these bases.
        \begin{equation*}
          B=\sequence{\colvec{1 \\ -1},\colvec{1 \\ 1}}
          \qquad
          D=\sequence{\colvec{2 \\ 2},\colvec{-1 \\ 1}}
        \end{equation*}
      \partsitem Using \( B \) from the prior item, 
        represent the transformation with respect to \( B,B \).
    \end{exparts}
    \begin{answer}
      Call the map \( \map{t}{\Re^2}{\Re^2} \).
      \begin{exparts}
        \partsitem To represent this map with respect to the standard bases, we
          must find, and then represent, the images of the vectors $\vec{e}_1$
          and $\vec{e}_2$ from the domain's basis.
          The image of $\vec{e}_1$ is given.

          One way to find the image of $\vec{e}_2$ is by 
          eye\Dash we can see this.
          \begin{equation*}
            \colvec{1 \\ 1}-\colvec{1 \\ 0}=\colvec{0 \\ 1}
            \;\mapsunder{t}\;
            \colvec{2 \\ 0}-\colvec{-1 \\ 0}=\colvec{3 \\ 0}
          \end{equation*}

          A more systemmatic way to find the image of $\vec{e}_2$ is to
          use the given information to represent the transformation, and then 
          use that representation to determine the image.  
          Taking this for a basis,
          \begin{equation*}
            C=\sequence{\colvec{1 \\ 1},\colvec{1 \\ 0}}
          \end{equation*}
          the given information says this.
          \begin{equation*}
            \rep{t}{C,\stdbasis_2}
            \begin{pmatrix}
              2  &-1  \\
              0  &0
            \end{pmatrix}
          \end{equation*}
          As
          \begin{equation*}
            \rep{\vec{e}_2}{C}=\colvec{1 \\ -1}_C
          \end{equation*}
          we have that
          \begin{equation*}
            \rep{t(\vec{e}_2)}{\stdbasis_2}
            =\begin{pmatrix}
               2  &-1  \\
               0  &0
             \end{pmatrix}_{C,\stdbasis_2}
            \colvec{1 \\ -1}_C
            =\colvec{3 \\ 0}_{\stdbasis_2}
          \end{equation*}
          and consequently we know that $t(\vec{e}_2)=3\cdot\vec{e}_1$
          (since, with respect to the standard basis, this vector is 
          represented by itself).
          Therefore, this is the representation of $t$ with respect to
          $\stdbasis_2,\stdbasis_2$.
          \begin{equation*}
            \rep{t}{\stdbasis_2,\stdbasis_2}
            =\begin{pmatrix}
               -1 &3   \\
               0  &0
             \end{pmatrix}_{\stdbasis_2,\stdbasis_2}
          \end{equation*}
        \partsitem To use the matrix developed in the prior item, note that
          \begin{equation*}
            \rep{\colvec{0 \\ 5}}{\stdbasis_2}=\colvec{0 \\ 5}_{\stdbasis_2}
          \end{equation*}
          and so we have this is the representation, with respect to the 
          codomain's basis, of the image of the given vector.
          \begin{equation*}
            \rep{t(\colvec{0 \\ 5})}{\stdbasis_2}
            =\begin{pmatrix}
              -1  &3   \\
               0  &0
             \end{pmatrix}_{\stdbasis_2,\stdbasis_2}
            \colvec{0 \\ 5}_{\stdbasis_2}
            =\colvec{15 \\ 0}_{\stdbasis_2}
          \end{equation*}
          Because the codomain's basis is the standard one, and so vectors
          in the codomain are represented by themselves, we have this.
          \begin{equation*}
            t(\colvec{0 \\ 5})
            =\colvec{15 \\ 0}
          \end{equation*}
        \partsitem We first find the image of each member of \( B \), and then
          represent those images with respect to \( D \).
          For the first step, we can use the matrix developed earlier.
          \begin{equation*}
            \rep{\colvec{1 \\-1}}{\stdbasis_2}
            =\begin{pmatrix}
              -1  &3   \\
               0  &0
             \end{pmatrix}_{\stdbasis_2,\stdbasis_2}
            \colvec{1 \\ -1}_{\stdbasis_2}
            =\colvec{-4 \\ 0}_{\stdbasis_2}
            \quad\text{so}\quad
            t(\colvec{1 \\ -1})=\colvec{-4 \\ 0}
          \end{equation*}
          Actually, for the second member of $B$ there is no need to apply the
          matrix because the problem statement gives its image.
          \begin{equation*}
            t(\colvec{1 \\ 1})=\colvec{2 \\ 0}
          \end{equation*}
          Now representing those images with respect to $D$ is routine.
          \begin{equation*}
            \rep{\colvec{-4 \\ 0}}{D}=\colvec{-1 \\ 2}_D
            \quad\text{and}\quad
            \rep{\colvec{2 \\ 0}}{D}=\colvec{1/2 \\ -1}_D
          \end{equation*}
          Thus, the matrix is this.
          \begin{equation*}
            \rep{t}{B,D}=
            \begin{pmatrix}
              -1  &1/2  \\
               2  &-1
            \end{pmatrix}_{B,D}
          \end{equation*}
        \partsitem We know the images of the members of the domain's basis
          from the prior item.
          \begin{equation*}
             t(\colvec{1 \\ -1})=\colvec{-4 \\ 0}
             \qquad
             t(\colvec{1 \\ 1})=\colvec{2 \\ 0}
          \end{equation*}
          We can  compute the representation of those images with respect to
          the codomain's basis.
          \begin{equation*}
             \rep{\colvec{-4 \\ 0}}{B}=\colvec{-2 \\ -2}_B
             \quad\text{and}\quad
             \rep{\colvec{2 \\ 0}}{B}=\colvec{1 \\ 1}_B
          \end{equation*}
          Thus this is the matrix.
          \begin{equation*}
            \rep{t}{B,B}=
            \begin{pmatrix}
              -2  &1  \\
              -2  &1
            \end{pmatrix}_{B,B}
          \end{equation*}
      \end{exparts}  
    \end{answer}
  \item 
    Suppose that \( \map{h}{V}{W} \) is one-to-one so that 
    by \nearbytheorem{th:OOHomoEquivalence}, for any basis
    \( B=\sequence{\vec{\beta}_1,\dots,\vec{\beta}_n}\subset V \) the image
    \( h(B)=\sequence{h(\vec{\beta}_1),\dots,h(\vec{\beta}_n)} \)
    is a basis for \( W \).
    \begin{exparts}
      \partsitem Represent the map $h$ with respect to $B,h(B)$.
      \partsitem For a member $\vec{v}$ of the domain, where
        the representation of $\vec{v}$ has components $c_1$, \ldots, $c_n$,
        represent the image vector \( h(\vec{v}) \) with respect to 
        the image basis $h(B)$.
    \end{exparts}
    \begin{answer}
      \begin{exparts}
        \partsitem The images of the members of the domain's basis are
          \begin{equation*}
            \vec{\beta}_1\mapsto h(\vec{\beta}_1)
            \quad 
            \vec{\beta}_2\mapsto h(\vec{\beta}_2)
            \quad\ldots\quad 
            \vec{\beta}_n\mapsto h(\vec{\beta}_n)
          \end{equation*}
          and those images are represented with respect to the codomain's
          basis in this way.
          \begin{equation*}
            \rep{\,h(\vec{\beta}_1)\,}{h(B)}=\colvec{1 \\ 0 \\ \vdots \\ 0}
            \quad
            \rep{\,h(\vec{\beta}_2)\,}{h(B)}=\colvec{0 \\ 1 \\ \vdots \\ 0}
            \quad\ldots\quad
            \rep{\,h(\vec{\beta}_n)\,}{h(B)}=\colvec{0 \\ 0 \\ \vdots \\ 1}
          \end{equation*}
          Hence, the matrix is the identity.
          \begin{equation*}
            \rep{h}{B,h(B)}
            =\begin{pmatrix}
               1  &0  &\ldots  &0  \\
               0  &1  &        &0  \\
                  &   &\ddots      \\
               0  &0  &        &1 
            \end{pmatrix}
          \end{equation*}
        \partsitem Using the matrix in the prior item, 
          the representation is this.
          \begin{equation*}
            \rep{\,h(\vec{v})\,}{h(B)}
             =\colvec{c_1 \\ \vdots \\ c_n}_{h(B)}
          \end{equation*}
        \end{exparts}  
     \end{answer}
  \item 
    Give a formula for the product of a matrix and \( \vec{e}_i \), the
    column vector that is all zeroes except for a single one in the \( i \)-th
    position.
    \begin{answer}
      The product
      \begin{equation*}
        \begin{pmatrix}
          h_{1,1} &\ldots  &h_{1,i} &\ldots &h_{1,n}  \\
          h_{2,1} &\ldots  &h_{2,i} &\ldots &h_{2,n}  \\
                  &\vdots                             \\
          h_{m,1} &\ldots  &h_{m,i} &\ldots &h_{1,n}
        \end{pmatrix}
        \colvec{0 \\ \vdots \\ 1 \\ \vdots \\ 0}
        =
        \colvec{h_{1,i} \\ h_{2,i} \\ \vdots \\ h_{m,i}}
      \end{equation*}
      gives the \( i \)-th column of the matrix.  
    \end{answer}
  \recommended \item 
    For each vector space of functions of one real variable,
    represent the derivative transformation with respect to \( B,B \).
    \begin{exparts}
      \partsitem \( \set{a\cos x+b\sin x \suchthat a,b\in\Re} \),
         \( B=\sequence{\cos x,\sin x} \)
      \partsitem \( \set{ae^x+be^{2x} \suchthat a,b\in\Re} \),
         \( B=\sequence{e^x,e^{2x}} \)
      \partsitem \( \set{a+bx+ce^x+dxe^{x} \suchthat a,b,c,d\in\Re} \),
         \( B=\sequence{1,x,e^x,xe^{x}} \)
    \end{exparts}
    \begin{answer}
      \begin{exparts}
        \partsitem The images of the basis vectors for the domain are
         \( \cos x\mapsunder{d/dx}-\sin x \) and
          \( \sin x\mapsunder{d/dx}\cos x \).
          Representing those with respect to the codomain's basis (again, $B$)
          and adjoining the representations gives this matrix.
          \begin{equation*}
            \rep{\frac{d}{dx}}{B,B}=
            \begin{pmatrix}
                0  &1  \\
               -1  &0
            \end{pmatrix}_{B,B}
          \end{equation*}
        \partsitem The images of the vectors in the domain's basis are 
          \( e^x\mapsunder{d/dx}e^x \) and
          \( e^{2x}\mapsunder{d/dx}2e^{2x} \).
          Representing with respect to the codomain's basis and adjoining
          gives this matrix.
          \begin{equation*}
            \rep{\frac{d}{dx}}{B,B}=
            \begin{pmatrix}
                1  &0  \\
                0  &2
            \end{pmatrix}_{B,B}
          \end{equation*}
        \partsitem The images of the members of the domain's basis are 
          \( 1\mapsunder{d/dx}0 \), 
          \( x\mapsunder{d/dx}1 \),
          \( e^{x}\mapsunder{d/dx}e^{x} \), and
          \( xe^{x}\mapsunder{d/dx}e^x+xe^x \).
          Representing these images with respect to $B$ and adjoining
          gives this matrix.
          \begin{equation*}
            \rep{\frac{d}{dx}}{B,B}=
            \begin{pmatrix}
                0  &1  &0  &0 \\
                0  &0  &0  &0 \\
                0  &0  &1  &1 \\
                0  &0  &0  &1
            \end{pmatrix}_{B,B}
          \end{equation*}
      \end{exparts}  
    \end{answer}
  \item 
    Find the range of the linear transformation of \( \Re^2 \) represented
    with respect to the standard bases by each matrix.
    \begin{exparts*}
      \partsitem $\begin{pmatrix}
            1  &0 \\
            0  &0
          \end{pmatrix}$
      \partsitem $\begin{pmatrix}
          0  &0  \\
          3  &2  
        \end{pmatrix}$
      \partsitem a matrix of the form 
        $\begin{pmatrix}
            a   &b  \\
            2a  &2b
          \end{pmatrix}$
    \end{exparts*}
    \begin{answer}
      \begin{exparts}
        \partsitem It is the set of vectors of the codomain represented with
          respect to the codomain's basis in this way.
          \begin{equation*}
            \set{
              \begin{pmatrix}
                1  &0  \\
                0  &0
              \end{pmatrix}
              \colvec{x  \\ y}
              \suchthat x,y\in\Re}
            =\set{\colvec{x  \\ 0}
                  \suchthat x,y\in\Re}
          \end{equation*}
          As the codomain's basis is $\stdbasis_2$, 
          and so each vector is represented
          by itself, the range of this transformation is the $x$-axis.
        \partsitem It is the set of vectors of the codomain represented
          in this way.
          \begin{equation*}
            \set{
              \begin{pmatrix}
                0  &0  \\
                3  &2
              \end{pmatrix}
              \colvec{x  \\ y}
              \suchthat x,y\in\Re}
            =\set{\colvec{0  \\ 3x+2y}
                  \suchthat x,y\in\Re}
          \end{equation*}
          With respect to $\stdbasis_2$ vectors represent 
          themselves so this range
          is the $y$~axis.
        \partsitem The set of vectors represented with 
          respect to $\stdbasis_2$ as
          \begin{equation*}
            \set{
              \begin{pmatrix}
                a   &b  \\
                2a  &2b
              \end{pmatrix}
              \colvec{x  \\ y}
              \suchthat x,y\in\Re}
            =\set{\colvec{ax+by  \\ 2ax+2by}
                  \suchthat x,y\in\Re}
            =\set{(ax+by)\cdot\colvec{1  \\ 2}
                  \suchthat x,y\in\Re}
          \end{equation*}
          is the line $y=2x$, provided either $a$ or $b$ is not zero, and
          is the set consisting of just the origin if both are zero.
      \end{exparts}  
    \end{answer}
  \recommended \item  
    Can one matrix represent two different linear maps?
    That is, can \( \rep{h}{B,D}=\rep{\hat{h}}{\hat{B},\hat{D}} \)?
    \begin{answer}
      Yes, for two reasons.

      First, the two maps $h$ and $\hat{h}$ need not have the same domain
      and codomain.
      For instance,
      \begin{equation*}
        \begin{pmatrix}
          1  &2  \\
          3  &4
        \end{pmatrix}
      \end{equation*}
      represents a map \( \map{h}{\Re^2}{\Re^2} \) with respect to the standard
      bases that sends
      \begin{equation*}
        \colvec{1 \\ 0}\mapsto\colvec{1 \\ 3}
        \quad\text{and}\quad
        \colvec{0 \\ 1}\mapsto\colvec{2 \\ 4}
      \end{equation*}
      and also represents a
      \( \map{\hat{h}}{\polyspace_1}{\Re^2} \) with respect to
      \( \sequence{1,x} \) and \( \stdbasis_2 \) that acts in this way.
      \begin{equation*}
        1\mapsto\colvec{1 \\ 3}
        \quad\text{and}\quad
        x\mapsto\colvec{2 \\ 4}
      \end{equation*}

      The second reason is that, even if the domain and
      codomain of \( h \) and \( \hat{h} \) coincide, different bases produce
      different maps.
      An example is the $\nbyn{2}$ identity matrix
      \begin{equation*}
        I=\begin{pmatrix}
          1  &0  \\
          0  &1
        \end{pmatrix}
      \end{equation*}
      which represents the identity map on $\Re^2$ with respect to
      $\stdbasis_2,\stdbasis_2$.
      However, with respect to $\stdbasis_2$ for the domain but the basis 
      $D=\sequence{\vec{e}_2,\vec{e}_1}$ for the codomain,
      the same matrix $I$ represents the map that swaps the first and second
      components 
      \begin{equation*}
        \colvec{x \\ y}\mapsto\colvec{y \\ x}
      \end{equation*}
      (that is, reflection about the line $y=x$).
    \end{answer}
  \item \label{exer:MatVecMultRepLinMap} 
    Prove \nearbytheorem{th:MatMultRepsFuncAppl}.
    \begin{answer}
      We mimic \nearbyexample{ex:TypLinMapRepByMat}, just replacing the 
      numbers with letters.

      Write \( B \) as \( \sequence{\vec{\beta}_1,\ldots,\vec{\beta}_n} \)
      and  \( D \) as \( \sequence{\vec{\delta}_1,\dots,\vec{\delta}_m} \).
      By definition of representation of a map with respect to bases,
      the assumption that
      \begin{equation*}
        \rep{h}{B,D}
        =\begin{pmatrix}
           h_{1,1} &\ldots  &h_{1,n}  \\
           \vdots  &        &\vdots   \\
           h_{m,1} &\ldots  &h_{m,n}
         \end{pmatrix}
      \end{equation*}
      means that
      $h(\vec{\beta}_i)=h_{i,1}\vec{\delta}_1+\dots+h_{i,n}\vec{\delta}_n$.
      And, by the definition of the representation of a vector with respect to
      a basis, the assumption that
      \begin{equation*}
        \rep{\vec{v}}{B}=\colvec{c_1 \\ \vdots \\ c_n}
      \end{equation*}
      means that \( \vec{v}=c_1\vec{\beta}_1+\cdots+c_n\vec{\beta}_n \).
      Substituting gives
      \begin{align*}
        h(\vec{v})
        &=h(c_1\cdot\vec{\beta}_1+\dots+c_n\cdot\vec{\beta}_n)      \\
        &=c_1\cdot h(\vec{\beta}_1)+\dots+c_n\cdot \vec{\beta}_n    \\
        &=c_1\cdot (h_{1,1}\vec{\delta}_1+\dots+h_{m,1}\vec{\delta}_m) 
        +\dots                                         
        +c_n\cdot (h_{1,n}\vec{\delta}_1+\dots+h_{m,n}\vec{\delta}_m) \\
        &=(h_{1,1}c_1+\dots+h_{1,n}c_n)\cdot\vec{\delta}_1    
        +\cdots                                
        +(h_{m,1}c_1+\dots+h_{m,n}c_n)\cdot\vec{\delta}_m
      \end{align*}
      and so $h(\vec{v})$ is represented as required.   
    \end{answer}
  \recommended \item  
    \nearbyexample{exam:RepsOfRigidPlaneMaps} shows how to represent 
    rotation of all vectors in the plane through an angle
    \( \theta \) about the origin,
    with respect to the standard bases.
    \begin{exparts}
      \partsitem Rotation of all vectors in three-space through an angle
        \( \theta \) about the \( x \)-axis is a transformation of $\Re^3$.
        Represent it with respect to the standard bases.
        Arrange the rotation so that 
        to someone whose feet are at the origin and
        whose head is at \( (1,0,0) \), the movement appears clockwise.
      \partsitem Repeat the prior item, only rotate about the \( y \)-axis 
        instead.
        (Put the person's head at $\vec{e}_2$.)
      \partsitem Repeat, about the \( z \)-axis.
      \partsitem Extend the prior item to \( \Re^4 \).
        (\textit{Hint:} 
        `rotate about the \( z \)-axis' can be restated as `rotate parallel
        to the \( xy \)-plane'.)
    \end{exparts}
    \begin{answer}
      \begin{exparts}
        \partsitem The picture is this.
          \begin{center}  \small
            \includegraphics{ch3.77}
         \end{center}
         The images of the vectors from the domain's basis 
         \begin{equation*}
           \colvec{1 \\ 0 \\ 0}\mapsto\colvec{1 \\ 0 \\ 0}
           \qquad
           \colvec{0 \\ 1 \\ 0}\mapsto\colvec{0 \\ \cos\theta \\ -\sin\theta}
           \qquad
           \colvec{0 \\ 0 \\ 1}\mapsto\colvec{0 \\ \sin\theta \\ \cos\theta}
         \end{equation*}
         are represented with respect to the codomain's basis
         (again, $\stdbasis_3$) by themselves, 
         so adjoining the representations to
         make the matrix gives this.
         \begin{equation*}
            \rep{r_\theta}{\stdbasis_3,\stdbasis_3}=
            \begin{pmatrix}
              1  &0       &0                \\
              0  &\cos\theta   &\sin\theta   \\
              0  &-\sin\theta  &\cos\theta
            \end{pmatrix}                  
         \end{equation*}
       \partsitem The picture is similar to the one in the prior answer. 
         The images of the vectors from the domain's basis 
         \begin{equation*}
           \colvec{1 \\ 0 \\ 0}\mapsto\colvec{\cos\theta \\ 0 \\ \sin\theta}
           \qquad
           \colvec{0 \\ 1 \\ 0}\mapsto\colvec{0 \\ 1 \\ 0}
           \qquad
           \colvec{0 \\ 0 \\ 1}\mapsto\colvec{-\sin\theta \\ 0 \\ \cos\theta}
         \end{equation*}
         are represented with respect to the codomain's basis $\stdbasis_3$
         by themselves, so this is the matrix.
         \begin{equation*}
            \begin{pmatrix}
              \cos\theta  &0       &-\sin\theta   \\
              0           &1       &0             \\
              \sin\theta  &0       &\cos\theta
            \end{pmatrix}                  
         \end{equation*}
       \partsitem To a person standing up, with the vertical $z$-axis,
         a rotation of the $xy$-plane that is clockwise proceeds from
         the positive $y$-axis to the positive $x$-axis.
         That is, it rotates opposite to the direction in  
         \nearbyexample{exam:RepsOfRigidPlaneMaps}.
         The images of the vectors from the domain's basis 
         \begin{equation*}
           \colvec{1 \\ 0 \\ 0}\mapsto\colvec{\cos\theta \\ -\sin\theta \\ 0}
           \qquad
           \colvec{0 \\ 1 \\ 0}\mapsto\colvec{\sin\theta \\ \cos\theta \\ 0}
           \qquad
           \colvec{0 \\ 0 \\ 1}\mapsto\colvec{0 \\ 0 \\ 1}
         \end{equation*}
         are represented with respect to $\stdbasis_3$
         by themselves, so the matrix is this.
         \begin{equation*}
            \begin{pmatrix}
              \cos\theta    &\sin\theta  &0   \\
              -\sin\theta   &\cos\theta  &0    \\
              0             &0           &1
            \end{pmatrix}                  
         \end{equation*}
        \partsitem 
            $\begin{pmatrix}
              \cos\theta  &\sin\theta &0 &0 \\
              -\sin\theta  &\cos\theta  &0 &0 \\
              0           &0           &1 &0 \\
              0           &0           &0 &1
            \end{pmatrix}$
      \end{exparts}   
    \end{answer}
  \item (Schur's Triangularization Lemma)\index{Triangularization}%
    \index{matrix!triangular}
    \begin{exparts}
      \partsitem Let \( U \) be a subspace of \( V \) and fix bases
        \( B_U\subseteq B_V \).
        What is the relationship between the representation of a vector
        from \( U \) with
        respect to \( B_U \) and the representation of that vector
        (viewed as a member of \( V \)) with
        respect to \( B_V \)?
      \partsitem What about maps?
      \partsitem Fix a basis 
        \( B=\sequence{\vec{\beta}_1,\dots,\vec{\beta}_n} \)
        for \( V \) and observe that the spans
        \begin{equation*}
          \spanof{\set{\zero}}=\set{\zero}\subset\spanof{\set{\vec{\beta}_1}}
                        \subset\spanof{\set{\vec{\beta}_1,\vec{\beta}_2}}
               \subset \quad\cdots\quad 
               \subset\spanof{B}=V
        \end{equation*}
        form a strictly increasing chain of subspaces.
        Show that for any linear map \( \map{h}{V}{W} \) there is a chain
        \( W_0=\set{\zero}\subseteq W_1\subseteq \dots \subseteq W_m =W \) of
        subspaces of \( W \) such that 
        \begin{equation*}
          h(\spanof{\set{\vec{\beta}_1,\dots,\vec{\beta}_i}})\subset W_i
        \end{equation*}
        for each \( i \).
      \partsitem Conclude that for every linear map 
        \( \map{h}{V}{W} \) there are
        bases \( B,D \) so the matrix representing \( h \) with respect to
        \( B,D \) is upper-triangular
        (that is, each entry \( h_{i,j} \) with \( i>j \) is zero).
      \item Is an upper-triangular representation unique?
    \end{exparts}
    \begin{answer}
      \begin{exparts}
        \partsitem 
          Write \( B_U \) as 
          \( \sequence{\vec{\beta}_1,\dots,\vec{\beta}_k} \) and
          then $B_V$ as \( \sequence{\vec{\beta}_1,\dots,\vec{\beta}_k,
                            \vec{\beta}_{k+1},\dots,\vec{\beta}_n} \).
          If 
          \begin{equation*}
            \rep{\vec{v}}{B_U}=\colvec{c_1 \\ \vdots \\ c_k}
            \qquad\text{so that\ }
            \vec{v}=c_1\cdot\vec{\beta}_1+\cdots+c_k\cdot\vec{\beta}_k
          \end{equation*}
          then, 
          \begin{equation*}
            \rep{\vec{v}}{B_V}=\colvec{c_1 \\ \vdots\\ c_k \\ 0 \\ \vdots \\ 0}
          \end{equation*}
          because $\vec{v}=c_1\cdot\vec{\beta}_1+\dots+c_k\cdot\vec{\beta}_k
                    +0\cdot\vec{\beta}_{k+1}+\dots+0\cdot\vec{\beta}_n$.
        \partsitem
          We must first decide what the question means.
          Compare \( \map{h}{V}{W} \) with its restriction to the subspace
          \( \map{\restrictionmap{h}{U}}{U}{W} \).
          The rangespace of the restriction is a subspace of \( W \), so fix a
          basis \( D_{h(U)} \) for this rangespace and extend it to a basis 
          \( D_V \) for \( W \).
          We want the relationship between these two.
          \begin{equation*}
            \rep{h}{B_V,D_V}
            \quad\text{and}\quad
            \rep{\restrictionmap{h}{U}}{B_U,D_{h(U)}}
          \end{equation*}
          The answer falls right out of the prior item:~if
          \begin{equation*}
            \rep{\restrictionmap{h}{U}}{B_U,D_{h(U)}}
             =\begin{pmatrix}
                h_{1,1}  &\ldots  &h_{1,k}  \\
                \vdots   &        &\vdots   \\
                h_{p,1}  &\ldots  &h_{p,k}
              \end{pmatrix}
          \end{equation*}
          then the extension is represented in this way.
          \begin{equation*}
            \rep{h}{B_V,D_V}
             =\begin{pmatrix}
                h_{1,1}  &\ldots  &h_{1,k}  &h_{1,k+1}  &\ldots  &h_{1,n}  \\
                \vdots   &        &         &           &        &\vdots   \\
                h_{p,1}  &\ldots  &h_{p,k}  &h_{p,k+1}  &\ldots  &h_{p,n}  \\
                0        &\ldots  &0        &h_{p+1,k+1}&\ldots  &h_{p+1,n}  \\
                \vdots   &        &         &           &        &\vdots   \\
                0        &\ldots  &0        &h_{m,k+1}  &\ldots  &h_{m,n}
              \end{pmatrix}
          \end{equation*}
        \partsitem Take \( W_i \) to be the span of
          \( \set{h(\vec{\beta}_1),\dots,h(\vec{\beta}_i)} \).
        \partsitem Apply the answer from the second item to the third item.
        \partsitem No.
          For instance \( \map{\pi_x}{\Re^2}{\Re^2} \), projection onto
          the \( x \)~axis, is represented by these two upper-triangular
          matrices 
          \begin{equation*}
             \rep{\pi_x}{\stdbasis_2,\stdbasis_2}=
             \begin{pmatrix}
               1  &0  \\
               0  &0
             \end{pmatrix}
             \quad\text{and}\quad
             \rep{\pi_x}{C,\stdbasis_2}=
             \begin{pmatrix}
               0  &1  \\
               0  &0
             \end{pmatrix}
          \end{equation*}
          where \( C=\sequence{\vec{e}_2,\vec{e}_1} \).
      \end{exparts}  
    \end{answer}
\index{homomorphism!matrix representing|)}
\end{exercises}















\subsectionoptional{Any Matrix Represents a Linear Map}

The prior subsection shows that
the action of a linear map $h$ is described by a matrix $H$,
with respect to appropriate bases, in this way.
\begin{equation*}
 \vec{v}=\colvec{v_1 \\ \vdots \\ v_n}_B
  \;\overset{h}{\underset{H}{\longmapsto}}\;
  \colvec{h_{1,1}v_1+\dots+h_{1,n}v_n \\ 
                     \vdots                      \\
                     h_{m,1}v_1+\dots+h_{m,n}v_n}_D
  =h(\vec{v})
\end{equation*}
In this subsection, we will show the converse, that
each matrix represents a linear map.

Recall that, in the definition of the matrix representation of a linear map,
the number of columns of the matrix is the dimension of the map's
domain and the number of rows of the matrix is the dimension of the 
map's codomain.
Thus, for instance,
a \( \nbym{2}{3} \) matrix cannot represent a map from \( \Re^5 \) to
\( \Re^4 \).
The next result says that, beyond this restriction on the dimensions, 
there are no other limitations:~the
\( \nbym{2}{3} \) matrix represents a map from any
three-dimensional space to any two-dimensional space. 

\begin{theorem}
\label{th:MatIsLinMap}\index{homomorphism!matrix representing}
Any matrix represents a homomorphism between vector spaces of
appropriate dimensions, with respect to any pair of bases.
\end{theorem}

\begin{proof}
For the matrix
\begin{equation*}
  H=\generalmatrix{h}{n}{m}
\end{equation*}
fix any \( n \)-dimensional domain space $V$ and  any
\( m \)-dimensional codomain space $W$.
Also fix bases
\( B=\sequence{\vec{\beta}_1,\dots,\vec{\beta}_n} \) and
\( D=\sequence{\vec{\delta}_1,\dots,\vec{\delta}_m} \) for those spaces.
Define a function \( \map{h}{V}{W} \) by:~where $\vec{v}$ in the domain
is represented as 
\begin{equation*}
  \rep{\vec{v}}{B}
    =\colvec{v_1 \\ \vdots \\ v_n}_B
\end{equation*}
then its image \( h(\vec{v}) \) is the member the codomain represented by
\begin{equation*}
  \rep{\,h(\vec{v})\,}{D}
    =\colvec{h_{1,1}v_1+\dots+h_{1,n}v_n \\ \vdots \\ 
                   h_{m,1}v_1+\dots+h_{m,n}v_n}_D
\end{equation*}
that is,
$h(\vec{v})=h(v_1\vec{\beta}_1+\dots+v_n\vec{\beta}_n)$ is defined to be 
$(h_{1,1}v_1+\dots+h_{1,n}v_n)\cdot\vec{\delta}_1
  +\dots+
  (h_{m,1}v_1+\dots+h_{m,n}v_n)\cdot\vec{\delta}_m$.
(This is well-defined by the uniqueness of the representation  
\( \rep{\vec{v}}{B} \).)

Observe that \( h \) has simply been defined to make it the map that is
represented with respect to \( B,D \) by the matrix \( H \).
So to finish, we need only check that \( h \) is linear.
If $\vec{v}, \vec{u}\in V$ are such that
\begin{equation*}
  \rep{\vec{v}}{B}=\colvec{v_1 \\ \vdots \\ v_n}
    \quad\text{and}\quad
  \rep{\vec{u}}{B}=\colvec{u_1 \\ \vdots \\ u_n}
\end{equation*}
and \( c,d\in\Re \) then the calculation
\begin{align*}
  h(c\vec{v}+d\vec{u})
  &=\bigl(h_{1,1}(cv_1+du_1)+\dots+
          h_{1,n}(cv_n+du_n)\bigr)\cdot\vec{\delta}_1+  \\
  & \hbox{}\quad\cdots+\bigl(h_{m,1}(cv_1+du_1)+\dots
         +h_{m,n}(cv_n+du_n)\bigr)\cdot\vec{\delta}_m  \\
  &=c\cdot h(\vec{v})+d\cdot h(\vec{u})
\end{align*}
provides this verification.
\end{proof}

\begin{example} \label{ex:CngBasesChgMap}
Which map the matrix represents depends on which bases are used.
If
\begin{equation*}
  H=
  \begin{pmatrix}
    1  &0  \\
    0  &0
  \end{pmatrix},
  \quad
  B_1=D_1=\sequence{\colvec{1 \\ 0},\colvec{0 \\ 1} },
  \quad\text{and}\quad
  B_2=D_2=\sequence{\colvec{0 \\ 1},\colvec{1 \\ 0} },
\end{equation*}
then \( \map{h_1}{\Re^2}{\Re^2} \) represented by \( H \)
with respect to \( B_1,D_1 \) maps
\begin{equation*}
  \colvec{c_1 \\ c_2}
  =\colvec{c_1 \\ c_2}_{B_1}
  \quad
  \mapsto
  \quad
  \colvec{c_1 \\ 0}_{D_1}
  =
  \colvec{c_1 \\ 0}
\end{equation*}
while \( \map{h_2}{\Re^2}{\Re^2} \) represented by \( H \)
with respect to \( B_2,D_2 \) is this map.
\begin{equation*}
  \colvec{c_1 \\ c_2}
  =\colvec{c_2 \\ c_1}_{B_2}
  \quad
  \mapsto
  \quad
  \colvec{c_2 \\ 0}_{D_2}
  =
  \colvec{0 \\ c_2}
\end{equation*}
These two are different.
The first is projection onto the \( x \)~axis, while the second 
is projection onto the $y$~axis.
\end{example}

So not only is any linear map described by a
matrix but any matrix describes a linear map.
This means that we can, when convenient, 
handle linear maps entirely as matrices,
simply doing the computations, without
have to worry that a matrix of interest does not represent 
a linear map on some pair of spaces of interest.
(In practice, when we are working with a matrix but no spaces or bases have
been specified, we will often take the
domain and codomain to be $\Re^n$ and $\Re^m$ and use the standard
bases.
In this case, because the
representation is transparent\Dash the representation with respect to the
standard basis of $\vec{v}$ is $\vec{v}$\Dash the
column space of the matrix equals the range of the map.
Consequently,
the column space of \( H \) is often denoted by \( \rangespace{H} \).) 

With the theorem, we have characterized linear maps as
those maps that act in this matrix way.
Each linear map is described by a matrix and each matrix describes a 
linear map.
We finish this section by illustrating how a matrix can be used to tell things
about its maps.

\begin{theorem} \label{th:RankMatEqRankMap}
\index{rank}\index{homomorphism!rank}\index{matrix!rank}
The rank of a matrix equals the rank of any map that it represents.
\end{theorem}

\begin{proof}
Suppose that the matrix \( H \) is \( \nbym{m}{n} \). 
Fix domain and codomain spaces $V$ and $W$ of dimension $n$ and~$m$, with
bases \( B=\sequence{\vec{\beta}_1,\dots,\vec{\beta}_n} \) and \( D \).
Then \( H \) represents some linear map $h$ between those spaces with respect 
to these bases whose rangespace
\begin{align*}
  \set{h(\vec{v})\suchthat \vec{v}\in V}
  &=\set{h(c_1\vec{\beta}_1+\dots+c_n\vec{\beta}_n)  
           \suchthat c_1,\dots,c_n\in\Re}                    \\
  &=\set{c_1h(\vec{\beta}_1)+\dots+c_nh(\vec{\beta}_n)
            \suchthat c_1,\dots,c_n\in\Re}
\end{align*}
is the span $\spanof{\set{h(\vec{\beta}_1),\dots,h(\vec{\beta}_n)}}$.
The rank of $h$ is the dimension of this rangespace.

The rank of the matrix is its column rank (or its row rank; the two are equal).
This is the dimension of the column space of the matrix, which is the span of
the set of column vectors 
$\spanof{\set{\rep{h(\vec{\beta}_1)}{D},\dots,\rep{h(\vec{\beta}_n)}{D}}}$.

To see that the two spans have the same dimension, recall that a
representation with respect to a basis gives an isomorphism 
$\map{\mbox{Rep}_D}{W}{\Re^m}$.
Under this isomorphism, there is a linear relationship among members of the
rangespace if and only if the same relationship holds in the
column space, e.g,
$\zero=c_1h(\vec{\beta}_1)+\dots+c_nh(\vec{\beta}_n)$ if and only if
$\zero=c_1\rep{h(\vec{\beta}_1)}{D}+\dots+c_n\rep{h(\vec{\beta}_n)}{D}$.
Hence, a subset of the rangespace is linearly independent if and only if the
corresponding subset of the column space is linearly independent.
This means that the size of the largest linearly independent subset of the
rangespace equals the size of the largest linearly independent subset of the
column space, and so the two spaces have the same dimension.
\end{proof}

\begin{example}
Any map represented by
\begin{equation*}
    \begin{pmatrix}
      1  &2  &2  \\
      1  &2  &1  \\
      0  &0  &3  \\
      0  &0  &2
    \end{pmatrix}
\end{equation*}
must, by definition, be from a three-dimensional domain
to a four-dimensional codomain.
In addition, because the rank of this matrix is two 
(we can spot this by eye or get it with Gauss' method),
any map represented by this matrix has a two-dimensional rangespace.
\end{example}

\begin{corollary} \label{cor:MatDescsMap}
Let $h$ be a linear map represented by a matrix $H$.
Then $h$ is onto if and only if the rank of $H$ equals the number 
of its rows, 
and $h$ is one-to-one if and only if the rank of $H$ equals the number of
its columns.
\end{corollary}

\begin{proof}
For the first half, the dimension of the rangespace of $h$ is the rank of $h$,
which equals the rank of $H$ by the theorem.
Since the dimension of the codomain of $h$ is the number of rows in $H$,
if the rank of $H$ equals the number of rows, then the dimension of the
rangespace equals the dimension of the codomain.
But a subspace with the same dimension as its superspace must equal
that superspace
(a basis for the rangespace is a linearly independent subset of the codomain,
whose size is equal to the dimension of the codomain, and so this set is 
a basis for the codomain).

For the second half, 
a linear map is one-to-one if and only if it is an isomorphism
between its domain and its range, that is, if and only if its domain has the
same dimension as its range.
But the number of columns in $h$ is the dimension of $h$'s domain, and
by the theorem the rank of $H$ equals the dimension of 
$h$'s range.
\end{proof}

The above results end any confusion caused by our use of the
word `rank' to mean apparently different things when applied to matrices and
when applied to maps.

\begin{definition}
A linear map that is one-to-one and onto is 
\definend{nonsingular}\index{homomorphism!nonsingular}%
\index{nonsingular!homomorphism}, otherwise it is 
\definend{singular}\index{homomorphism!singular}%
\index{singular!homomorphism}.
\end{definition}

\begin{remark}
Some authors use `nonsingular' as a synonym for `one-to-one'  
while others use it the way that we have here.
The difference is slight because a one-to-one map is onto its
rangespace.
\end{remark}

In the first chapter we defined a matrix to be nonsingular 
if it is square and is
the matrix of coefficients of a linear system with a unique solution.
The next result justifies our dual use of the term.

\begin{corollary} \label{cor:NonsingMatIffNonsingMap}
\index{homomorphism!nonsingular}\index{matrix!nonsingular}
\index{nonsingular}
A square matrix represents nonsingular maps if and only if it is a nonsingular
matrix.
Thus, a matrix represents isomorphisms if and only if it is square and
nonsingular. 
\end{corollary}

\begin{proof}
Immediate from the prior result.
\end{proof}

\begin{example}
Any map from \( \Re^2 \) to \( \polyspace_1 \) represented 
with respect to any pair of bases by
\begin{equation*}
  \begin{pmatrix}
     1  &2  \\
     0  &3  
  \end{pmatrix}
\end{equation*}
is nonsingular because this matrix has rank two.
\end{example}

\begin{example} \label{ex:NonSMatHasNonSMap}
Any map \( \map{g}{V}{W} \) represented by
\begin{equation*}
  \begin{pmatrix}
    1  &2  \\
    3  &6
  \end{pmatrix}
\end{equation*}
is singular because this matrix is singular.
\end{example}

We've now seen that the relationship between maps and 
matrices goes both ways:~for a particular pair of bases, 
any linear map is represented by a
matrix and any matrix describes a linear map.
That is, by fixing spaces and bases we get
a correspondence between maps and matrices.
In the rest of this chapter we will explore this correspondence.
For instance, we've defined for linear maps the operations of addition 
and scalar multiplication and we shall see what the corresponding matrix 
operations are.
We shall also see the matrix operation that represent the map operation
of composition.
And, we shall see how to find the matrix that represents a map's inverse.


\begin{exercises}
  \recommended \item  
    Decide if the vector is in the column space of the matrix.
    \begin{exparts*}
      \partsitem \( \begin{pmatrix}
                   2  &1  \\
                   2  &5
                 \end{pmatrix} \),~\( \colvec{1 \\ -3}  \)
      \partsitem \( \begin{pmatrix}
                   4  &-8 \\
                   2  &-4
                 \end{pmatrix} \),~\( \colvec{0 \\ 1}  \)
      \partsitem \( \begin{pmatrix}
                   1  &-1  &1  \\
                   1  &1   &-1 \\
                  -1  &-1  &1
                \end{pmatrix} \),~\( \colvec{2 \\ 0 \\ 0}  \)
    \end{exparts*}
    \begin{answer}
      \begin{exparts}
        \partsitem Yes;
          we are asking if there are scalars \( c_1 \) and \( c_2 \) such that
          \begin{equation*}
            c_1\colvec{2 \\ 2}+c_2\colvec{1 \\ 5}=\colvec{1 \\ -3}
          \end{equation*}
          which gives rise to a linear system
          \begin{equation*}
            \begin{linsys}{2}
              2c_1  &+  &c_2  &=  &1  \\
              2c_1  &+  &5c_2 &=  &-3
            \end{linsys}
            \;\grstep{-\rho_1+\rho_2}\;
            \begin{linsys}{2}
              2c_1  &+  &c_2  &=  &1  \\
                    &   &4c_2 &=  &-4
            \end{linsys}
          \end{equation*}
          and Gauss' method produces \( c_2=-1 \) and \( c_1=1 \). 
          That is, there is indeed such a pair of scalars and so the vector
          is indeed in the column space of the matrix.
        \partsitem No;
          we are asking if there are scalars $c_1$ and $c_2$
          such that 
          \begin{equation*}
            c_1\colvec{4 \\ 2}+c_2\colvec{-8 \\ -4}=\colvec{0 \\ 1}
          \end{equation*}
          and one way to proceed is to consider the resulting linear system
          \begin{equation*}
            \begin{linsys}{2}
              4c_1  &-  &8c_2  &=  &0  \\
              2c_1  &-  &4c_2  &=  &1
            \end{linsys}
          \end{equation*}
          that is easily seen to have no solution.
          Another way to proceed is to note
          that any linear combination of the columns on the left
          has a second component half as big as its first component, 
          but the  vector on the right does not meet that criterion.
        \partsitem Yes; we can simply observe that the vector
          is the first column minus the second.
          Or, failing that, setting up the relationship among the columns 
          \begin{equation*}
            c_1\colvec{1 \\ 1 \\ -1}
             +c_2\colvec{-1 \\ 1 \\ -1}
             +c_3\colvec{1 \\ -1 \\ 1}
             =\colvec{2 \\ 0 \\ 0}
          \end{equation*}
          and considering the resulting linear system
          \begin{equation*}
            \begin{linsys}{3}
              c_1  &-  &c_2  &+  &c_3  &=  &2  \\
              c_1  &+  &c_2  &-  &c_3  &=  &0  \\
             -c_1  &-  &c_2  &+  &c_3  &=  &0    
            \end{linsys}
            \;\grstep[\rho_1+\rho_3]{-\rho_1+\rho_2}\;
            \begin{linsys}{3}
              c_1  &-  &c_2  &+  &c_3  &=  &2  \\
                   &   &2c_2 &-  &2c_3 &=  &-2  \\
                   &   &-2c_2&+  &2c_3 &=  &2    
            \end{linsys}
            \;\grstep{\rho_2+\rho_3}\;
            \begin{linsys}{3}
              c_1  &-  &c_2  &+  &c_3  &=  &2  \\
                   &   &2c_2 &-  &2c_3 &=  &-2  \\
                   &   &     &   &0    &=  &0    
            \end{linsys}
          \end{equation*}
          gives the additional information (beyond that there is at least one
          solution) that there are infinitely many solutions.
          Parametizing gives $c_2=-1+c_3$ and $c_1=1$, and so taking $c_3$ to 
          be zero gives a particular solution of $c_1=1$, $c_2=-1$, and
          $c_3=0$ (which is, of course, the observation made at the start).
      \end{exparts}  
    \end{answer}
 \recommended \item 
   Decide if each vector lies in the range of the map from \( \Re^3 \)
   to \( \Re^2 \) represented with respect to the standard bases by the matrix.
   \begin{exparts*}
     \partsitem \( \begin{pmatrix}
                1  &1  &3  \\
                0  &1  &4
              \end{pmatrix}  \),~\( \colvec{1 \\ 3} \)
     \partsitem \( \begin{pmatrix}
                2  &0  &3  \\
                4  &0  &6
              \end{pmatrix}  \),~\( \colvec{1 \\ 1} \)
   \end{exparts*}
   \begin{answer} 
     As described in the subsection, with respect to the standard bases,
     representations are transparent,
     and so, for instance, the first matrix describes this map.
     \begin{equation*}
       \colvec{1 \\ 0 \\ 0}=\colvec{1 \\ 0 \\ 0}_{\stdbasis_3}
          \!\mapsto\colvec{1 \\ 0}_{\stdbasis_2}=\colvec{1 \\ 0}
       \qquad
       %\colvec{0 \\ 1 \\ 0}=\colvec{0 \\ 1 \\ 0}_{\stdbasis_3}
       %   \!\mapsto\colvec{1 \\ 1}_{\stdbasis_2}=\colvec{1 \\ 1}
       \colvec{0 \\ 1 \\ 0}
          \!\mapsto\colvec{1 \\ 1}
       \qquad
       %\colvec{0 \\ 0 \\ 1}=\colvec{0 \\ 0 \\ 1}_{\stdbasis_3}
       %   \!\mapsto\colvec{3 \\ 4}_{\stdbasis_2}=\colvec{3 \\ 4}
       \colvec{0 \\ 0 \\ 1}
          \!\mapsto\colvec{3 \\ 4}
     \end{equation*}
     So, for this first one, we are asking whether thare are scalars such that
     \begin{equation*}
       c_1\colvec{1 \\ 0}+c_2\colvec{1 \\ 1}+c_3\colvec{3 \\ 4}=\colvec{1 \\ 3}
     \end{equation*}
     that is, whether the vector is in the column space of the matrix.
     \begin{exparts}
       \partsitem Yes.
         We can get this conclusion by setting up the resulting linear system
         and applying Gauss' method, as usual.
         Another way to get it is to note by inspection of the equation of
         columns that taking $c_3=3/4$, and $c_1=-5/4$, and $c_2=0$ will do.  
         Still a third way to get this conclusion is to note that the rank
         of the matrix is two, which equals the dimension of the
         codomain, and so the map is onto\Dash the range is all of $\Re^2$ and
         in particular includes the given vector.
       \partsitem No; note that all of the columns in the matrix have a second
         component that is twice the first, while the vector does not.
         Alternatively, the column space of the matrix is 
         \begin{equation*}
           \set{c_1\colvec{2 \\ 4}
                +c_2\colvec{0 \\ 0}
                +c_3\colvec{3 \\ 6} \suchthat c_1,c_2,c_3\in\Re}
           =\set{c\colvec{1 \\ 2}\suchthat c\in\Re}
         \end{equation*}
         (which is the fact already noted, but was arrived at by calculation 
         rather than inspiration), and the given vector is not in this set.
     \end{exparts}  
    \end{answer}
  \recommended \item  
    Consider this matrix, representing a transformation of $\Re^2$, 
    and these bases for that space.
    \begin{equation*}
      \frac{1}{2}\cdot
      \begin{pmatrix}
        1  &1  \\
        -1 &1
      \end{pmatrix}
      \qquad
      B=\sequence{\colvec{0 \\ 1},\colvec{1 \\ 0}}
      \quad 
      D=\sequence{\colvec{1 \\ 1},\colvec{1 \\ -1}}
    \end{equation*}
    \begin{exparts}
      \partsitem To what vector in the codomain 
        is the first member of $B$ mapped? 
      \partsitem The second member?
      \partsitem Where is a general vector from the domain (a vector with 
        components $x$ and $y$) mapped? 
        That is, what transformation of \( \Re^2 \) is represented with 
        respect to \( B,D \) by this matrix?
    \end{exparts}
    \begin{answer}
      \begin{exparts}
        \partsitem The first member of the basis
          \begin{equation*}
            \colvec{0 \\ 1}=\colvec{1 \\ 0}_B
          \end{equation*}
          is mapped to 
          \begin{equation*}
            \colvec{1/2 \\ -1/2}_D
          \end{equation*}
          which is this member of the codomain.
          \begin{equation*}
            \frac{1}{2}\cdot\colvec{1 \\ 1}
              -\frac{1}{2}\cdot\colvec{1 \\ -1}
              =\colvec{0 \\ 1}
          \end{equation*}
        \partsitem The second member of the basis is mapped
          \begin{equation*}
            \colvec{1 \\ 0}=\colvec{0 \\ 1}_B
            \mapsto
            \colvec{(1/2 \\ 1/2}_D 
          \end{equation*}
          to this member of the codomain.
          \begin{equation*}
            \frac{1}{2}\cdot\colvec{1 \\ 1}
              +\frac{1}{2}\cdot\colvec{1 \\ -1}
              =\colvec{1 \\ 0}
          \end{equation*}
        \partsitem Because the map that the matrix represents is the identity
          map on the basis, it must be the identity on all members of the 
          domain.
          We can come to the same conclusion in another way by considering
          \begin{equation*}
            \colvec{x \\ y}=\colvec{y \\ x}_B
          \end{equation*}
          which is mapped to
          \begin{equation*}
            \colvec{(x+y)/2 \\ (x-y)/2}_D
          \end{equation*}
          which represents this member of $\Re^2$. 
          \begin{equation*}
            \frac{x+y}{2}\cdot\colvec{1 \\ 1}
              +\frac{x-y}{2}\cdot\colvec{1 \\ -1}
            =\colvec{x \\ y}
          \end{equation*}
      \end{exparts}
    \end{answer}
  \item 
    What transformation of
    \( F=\set{a\cos\theta+b\sin\theta\suchthat a,b\in\Re} \)
    is represented with respect to
    \( B=\sequence{\cos\theta-\sin\theta,\sin\theta} \) and
    \( D=\sequence{\cos\theta+\sin\theta,\cos\theta} \) by this matrix?
    \begin{equation*}
      \begin{pmatrix}
          0  &0  \\
          1  &0
      \end{pmatrix}
    \end{equation*}
    \begin{answer}
      A general member of the domain, represented with respect to the
      domain's basis as
      \begin{equation*}
        a\cos\theta+b\sin\theta=\colvec{a \\ a+b}_B
      \end{equation*}
      is mapped to 
      \begin{equation*}
        \colvec{0 \\ a}_D
          \quad\text{representing}\quad
        0\cdot(\cos\theta+\sin\theta)+a\cdot(\cos\theta)
      \end{equation*}
      and so the linear map represented by the matrix with respect to these
      bases 
      \begin{equation*}
        a\cos\theta+b\sin\theta  
            \mapsto
        a\cos\theta
      \end{equation*}
      is projection onto the first component.  
    \end{answer}
  \recommended \item  
     Decide whether $1+2x$ is in the range of the map from $\Re^3$ to 
     $\polyspace_2$ represented with respect to $\stdbasis_3$ and 
     $\sequence{1,1+x^2,x}$ by this matrix.
     \begin{equation*}
       \begin{pmatrix}
         1  &3  &0  \\
         0  &1  &0  \\
         1  &0  &1
       \end{pmatrix}
     \end{equation*}
     \begin{answer}
       Denote the given basis of
       $\polyspace_2$
       by $B$.
       Then application of the linear map is represented by matrix-vector 
       multiplication.
       Thus, the first vector in $\stdbasis_3$ is mapped to the element
       of $\polyspace_2$ represented with respect to $B$ by
       \begin{equation*}
       \begin{pmatrix}
         1  &3  &0  \\
         0  &1  &0  \\
         1  &0  &1
       \end{pmatrix}
       \colvec{1 \\ 0 \\ 0}
       =
       \colvec{1 \\ 0 \\ 1}  
       \end{equation*}
       and that element is $1+x$.
       The other two images of basis vectors are calculated similarly.
       \begin{equation*}
         \begin{pmatrix}
           1  &3  &0  \\
           0  &1  &0  \\
           1  &0  &1
         \end{pmatrix}
         \colvec{0 \\ 1 \\ 0}
         =
         \colvec{3 \\ 1 \\ 0}=\rep{4+x^2}{B}
         \quad
         \begin{pmatrix}
           1  &3  &0  \\
           0  &1  &0  \\
           1  &0  &1
         \end{pmatrix}
         \colvec{0 \\ 0 \\ 1}
         =
         \colvec{0 \\ 0 \\ 1}=\rep{x}{B}
       \end{equation*}
       So the range of $h$ is the span of three polynomials 
       $1+x$, $4+x^2$, and $x$. 
       We can thus decide if $1+2x$ is in the range of the map by 
       looking for scalars $c_1$, $c_2$, and $c_3$ such that
       \begin{equation*}
         c_1\cdot(1)+c_2\cdot(1+x^2)+c_3\cdot(x)=1+2x
       \end{equation*}
       and obviously $c_1=1$, $c_2=0$, and $c_3=1$ suffice.
       Thus it is in the range, and in fact it is the image of
       this vector. 
       \begin{equation*}
         1\cdot\colvec{1 \\ 0 \\ 0}+0\cdot\colvec{0 \\ 1 \\ 0}
             +1\cdot\colvec{0 \\ 0 \\ 1}
       \end{equation*}

       \textit{Comment.}
       A more slick argument is to note that the matrix is nonsingular,
       so it has rank~$3$, so the codomain has dimension~$4$,
       and thus every polynomial is the image of some vector.
     \end{answer}
  \item 
    \nearbyexample{ex:NonSMatHasNonSMap} gives a matrix that is
    nonsingular and is therefore associated with maps that are nonsingular.
    \begin{exparts}
      \partsitem Find the set of column vectors representing the members of
        the nullspace of any map represented by this matrix.
      \partsitem Find the nullity of any such map.
      \partsitem Find the set of column vectors representing the members of
        the rangespace of any map represented by this matrix.
      \partsitem Find the rank of any such map.
      \partsitem Check that rank plus nullity equals the dimension of the
        domain.
    \end{exparts}
    \begin{answer}
      Let the matrix be $G$, and suppose that it rperesents $\map{g}{V}{W}$ 
      with respect to bases $B$ and $D$.
      Because $G$ has two columns, $V$ is two-dimensional.
      Because $G$ has two rows, $W$ is two-dimensional.
      The action of $g$ on a general member of the domain is this.
      \begin{equation*}
        \colvec{x \\ y}_B 
         \;\mapsto\; 
        \colvec{x+2y \\ 3x+6y}_D
      \end{equation*}
      \begin{exparts}
        \partsitem The only representation of the zero vector in the codomain
           is 
           \begin{equation*}
             \rep{\zero}{D}=\colvec{0 \\ 0}_D
           \end{equation*}
           and so the set of representations of members of the nullspace is
           this.
           \begin{equation*}
             \set{\colvec{x \\ y}_B\suchthat \text{$x+2y=0$ and $3x+6y=0$}}
             =\set{y\cdot\colvec{-1/2 \\ 1}_D\suchthat y\in\Re}
           \end{equation*}
         \partsitem The representation map $\map{\mbox{Rep}_D}{W}{\Re^2}$
           and its inverse
           are isomorphisms, and so preserve the dimension of subspaces.
           The subspace of $\Re^2$ that is in the prior item is
           one-dimensional.
           Therefore, the image of that subspace under the inverse of the
           representation map\Dash the nullspace of $G$, 
           is also one-dimensional.
         \partsitem The set of representations of members of the rangespace is
           this.
           \begin{equation*}
             \set{\colvec{x+2y \\ 3x+6y}_D\suchthat x,y\in\Re}
             =\set{k\cdot\colvec{1 \\ 3}_D\suchthat k\in\Re}
           \end{equation*}
         \partsitem Of course, \nearbytheorem{th:RankMatEqRankMap} gives that
           the rank of the map equals the rank of the matrix, which is one.
           Alternatively, the same argument that was used above for the 
           nullspace gives here that the dimension of the rangespace is one.
         \partsitem One plus one equals two. 
      \end{exparts}
    \end{answer}
  \recommended \item  
    Because
    the rank of a matrix equals the rank of any map it represents, if
    one matrix represents two different maps 
    \( H=\rep{h}{B,D}=\rep{\hat{h}}{\hat{B},\hat{D}} \) 
    (where \( \map{h,\hat{h}}{V}{W} \))
    then the dimension of the rangespace of
    \( h \) equals the dimension of the rangespace of \( \hat{h} \).
    Must these equal-dimensioned rangespaces actually be the same?
    \begin{answer}
      No, the rangespaces may differ.
      \nearbyexample{ex:CngBasesChgMap} shows this.
    \end{answer}
  \recommended \item 
    Let \( V \) be an \( n \)-dimensional space with bases \( B \) and
    \( D \).
    Consider a map that sends, for \( \vec{v}\in V\), 
    the column vector representing \( \vec{v} \) with
    respect to \( B \) to the column vector representing \( \vec{v} \) with
    respect to \( D \).
    Show that map is a linear transformation of \( \Re^n \).
    \begin{answer}
      Recall that the represention map
      \begin{equation*}
        V\mapsunder{\text{Rep}_{B}}\Re^n
      \end{equation*}
      is an isomorphism.
      Thus, its inverse map $\map{\mbox{Rep}_B^{-1}}{\Re^n}{V}$
      is also an isomorphism.
      The desired transformation of $\Re^n$ is then this composition.
      \begin{equation*}
        \Re^n\mapsunder{\text{Rep}_{B}^{-1}}
        V\mapsunder{\text{Rep}_{D}}\Re^n
      \end{equation*}
      Because a composition of isomorphisms is also an isomorphism, 
      this map $\composed{\mbox{Rep}_{D}}{\mbox{Rep}_{B}^{-1}}$
      is an isomorphism.
    \end{answer}
  \item 
    \nearbyexample{ex:CngBasesChgMap} shows that changing the pair of
    bases can change the map that a matrix
    represents, even though the domain and codomain remain the same.
    Could the map ever not change?
    Is there a matrix \( H \), vector spaces \( V \) and \( W \), and
    associated pairs of bases \( B_1,D_1 \) and \( B_2,D_2 \) (with
    \( B_1\neq B_2 \) or \( D_1\neq D_2 \) or both) 
    such that the map represented
    by \( H \) with respect to \( B_1,D_1 \) equals the map represented
    by \( H \) with respect to \( B_2,D_2 \)?
    \begin{answer}
      Yes.
      Consider
      \begin{equation*}
        H=\begin{pmatrix}
            1  &0  \\
            0  &1
          \end{pmatrix}
      \end{equation*}
      representing a map from \( \Re^2 \) to \( \Re^2 \).
      With respect to the standard bases 
      \( B_1=\stdbasis_2, D_1=\stdbasis_2 \) this matrix
      represents the identity map.
      With respect to
      \begin{equation*}
        B_2=D_2=\sequence{\colvec{1 \\ 1},\colvec{1 \\ -1}}
      \end{equation*}
      this matrix again represents the identity.
      In fact, as long as the starting and ending bases 
      are equal\Dash as long as
      \( B_i=D_i \)\Dash then the map represented by $H$ is the identity.
    \end{answer}
  \recommended \item 
    A square matrix is a 
    \definend{diagonal}\index{matrix!diagonal}\index{diagonal matrix} %
    matrix if it is all zeroes
    except possibly for the entries on its upper-left to lower-right
    diagonal\Dash its \( 1,1 \) entry, its \( 2,2 \) entry, etc.
    Show that a linear map is an isomorphism if there are bases such that,
    with respect to those bases, the map is represented by a diagonal matrix 
    with no zeroes on the diagonal.
    \begin{answer}
      This is immediate from \nearbycorollary{cor:NonsingMatIffNonsingMap}.
    \end{answer}
  \item 
    Describe geometrically the action on \( \Re^2 \) of
    the map represented with respect to the standard 
    bases $\stdbasis_2,\stdbasis_2$ by this matrix.
    \begin{equation*}
      \begin{pmatrix}
        3  &0  \\
        0  &2
      \end{pmatrix}
    \end{equation*}
    Do the same for these.
    \begin{equation*}
      \begin{pmatrix}
        1  &0  \\
        0  &0
      \end{pmatrix}
      \quad
      \begin{pmatrix}
        0  &1  \\
        1  &0
      \end{pmatrix}
      \quad
      \begin{pmatrix}
        1  &3  \\
        0  &1
      \end{pmatrix}
    \end{equation*}
    \begin{answer}
      The first map 
      \begin{equation*}
        \colvec{x \\ y}=\colvec{x \\ y}_{\stdbasis_2}
        \mapsto
        \colvec{3x \\ 2y}_{\stdbasis_2}=\colvec{3x \\ 2y}
      \end{equation*}
      stretches vectors by a factor of three in the
      \( x \)~direction and by a factor of two in the \( y \)~direction.
      The second map
      \begin{equation*}
        \colvec{x \\ y}=\colvec{x \\ y}_{\stdbasis_2}
        \mapsto
        \colvec{x \\ 0}_{\stdbasis_2}=\colvec{x \\ 0}
      \end{equation*}
      projects vectors onto the \( x \)~axis.
      The third 
      \begin{equation*}
        \colvec{x \\ y}=\colvec{x \\ y}_{\stdbasis_2}
        \mapsto
        \colvec{y \\ x}_{\stdbasis_2}=\colvec{y \\ x}
      \end{equation*}
      interchanges first and second components
      (that is, it is a reflection about the line \( y=x \)).
      The last 
      \begin{equation*}
        \colvec{x \\ y}=\colvec{x \\ y}_{\stdbasis_2}
        \mapsto
        \colvec{x+3y \\ y}_{\stdbasis_2}=\colvec{x+3y \\ y}
      \end{equation*}
      stretches vectors parallel to the \( y \)~axis, by an amount
      equal to three times their distance from that axis 
      (this is a \definend{skew}.)  
     \end{answer}
  \item 
     The fact that for any linear map the rank plus the nullity
     equals the dimension of the domain shows that a necessary
     condition for the existence of a homomorphism between two spaces, onto
     the second space, is that there be no gain in dimension.
     That is, where $\map{h}{V}{W}$ is onto, the dimension of $W$ must
     be less than or equal to the dimension of $V$.
     \begin{exparts}
       \partsitem Show that this (strong) converse holds: 
          no gain in dimension implies that 
          there is a homomorphism and, further, 
          any matrix with the correct size and correct rank 
          represents such a map.
       \partsitem Are there bases for $\Re^3$ such that
          this matrix
          \begin{equation*}
            H=\begin{pmatrix}
                1  &0  &0 \\
                2  &0  &0 \\
                0  &1  &0 
              \end{pmatrix}
          \end{equation*}
          represents a map from $\Re^3$ to $\Re^3$ whose range is
          the $xy$~plane subspace of $\Re^3$?
     \end{exparts}
     \begin{answer}
       \begin{exparts}
         \partsitem  This is immediate from
           \nearbytheorem{th:RankMatEqRankMap}.
         \partsitem Yes.
            This is immediate from the prior item.

            To give a specific example, we can 
            start with $\stdbasis_3$ as the basis for the domain, and then
            we require a basis $D$ for the codomain $\Re^3$.
            The matrix $H$ gives the action of the map as this
            \begin{equation*}
              \colvec{1 \\ 0 \\ 0}=\colvec{1 \\ 0 \\ 0}_{\stdbasis_3}
                 \mapsto\colvec{1 \\ 2 \\ 0}_D
              \quad       
              \colvec{0 \\ 1 \\ 0}=\colvec{0 \\ 1 \\ 0}_{\stdbasis_3}
                 \mapsto\colvec{0 \\ 0 \\ 1}_D
              \quad       
              \colvec{0 \\ 0 \\ 1}=\colvec{0 \\ 0 \\ 1}_{\stdbasis_3}
                 \mapsto\colvec{0 \\ 0 \\ 0}_D
            \end{equation*}
            and there is no harm in finding a basis $D$ so that
            \begin{equation*}
              \rep{\colvec{1 \\ 0 \\ 0}}{D}=\colvec{1 \\ 2 \\ 0}_D
              \quad\mbox{and}\quad       
              \rep{\colvec{0 \\ 1 \\ 0}}{D}=\colvec{0 \\ 0 \\ 1}_D
            \end{equation*}
            that is, so that the map represented by $H$ with respect to
            $\stdbasis_3,D$ is projection down onto the $xy$~plane.
            The second condition gives that the third member of $D$
            is $\vec{e}_2$.
            The first condition gives that the first member of $D$ plus twice
            the second equals $\vec{e}_1$, and so this basis will do.
            \begin{equation*}
              D=\sequence{\colvec{0 \\ -1 \\ 0},
                          \colvec{1/2 \\ 1/2 \\ 0},
                          \colvec{0 \\ 1 \\ 0}}
            \end{equation*}
       \end{exparts} 
     \end{answer}
  \item 
    Let \( V \) be an \( n \)-dimensional space and suppose
    that \( \vec{x}\in\Re^n \).
    Fix a basis \( B \) for \( V \) and consider the map
    \( \map{h_{\vec{x}}}{V}{\Re} \) given 
    $\vec{v}\mapsto\vec{x}\dotprod\rep{\vec{v}}{B}$ by the dot product.
    \begin{exparts}
      \partsitem Show that this map is linear.
      \partsitem Show that for any linear map \( \map{g}{V}{\Re} \) there is 
        an \( \vec{x}\in\Re^n \) such that \( g=h_{\vec{x}} \).
      \partsitem In the prior item we fixed the basis and varied the 
        \( \vec{x} \) to get all possible linear maps.
        Can we get all possible linear maps by fixing an \( \vec{x} \) and
        varying the basis?
    \end{exparts}
    \begin{answer}
      \begin{exparts}
        \partsitem Recall that the representation map
          $\map{\mbox{Rep}_{B}}{V}{\Re^n}$ is linear (it is actually
          an isomorphism, but we do not need that it is one-to-one or onto
          here).
          Considering the column vector $x$ to be a $\nbym{n}{1}$ matrix
          gives that the map from $\Re^n$ to $\Re$ that takes a column vector
          to its dot product with $\vec{x}$ is linear (this is a matrix-vector
          product and so \nearbytheorem{th:MatIsLinMap} applies).
          Thus the map under consideration $h_{\vec{x}}$ is linear because 
          it is the composistion of two linear maps.
          \begin{equation*}
            \vec{v}\mapsto \rep{\vec{v}}{B}
                   \mapsto \vec{x}\cdot\rep{\vec{v}}{B}     
          \end{equation*}
       \partsitem Any linear map $\map{g}{V}{\Re}$ is represented by some
          matrix
          \begin{equation*}
            \begin{pmatrix}
              g_1  &g_2 &\cdots &g_n
            \end{pmatrix}
          \end{equation*}
          (the matrix has $n$ columns because $V$ is $n$-dimensional and it
          has only one row because $\Re$ is one-dimensional).
          Then taking $\vec{x}$ to be the column vector that is the transpose
          of this matrix
          \begin{equation*}
            \vec{x}=\colvec{g_1 \\ \vdots \\ g_n}
          \end{equation*}
          has the desired action.
          \begin{equation*}
            \vec{v}=\colvec{v_1 \\ \vdots \\ v_n}
             \mapsto
            \colvec{g_1 \\ \vdots \\ g_n}\dotprod\colvec{v_1 \\ \vdots \\ v_n}
            =g_1v_1+\dots+g_nv_n
          \end{equation*}
        \partsitem No.
          If \( \vec{x} \) has any nonzero entries then \( h_{\vec{x}} \)
          cannot be the zero map (and if \( \vec{x} \) is the zero vector
          then \( h_{\vec{x}} \) can only be the zero map).
      \end{exparts}  
    \end{answer}
  \item
    Let \( V,W,X \) be vector spaces with bases \( B,C,D \).
    \begin{exparts}
      \partsitem Suppose that \( \map{h}{V}{W} \) 
        is represented with respect to \( B,C \) by the matrix \( H \).
        Give the matrix representing the scalar multiple
        \( rh \) (where \( r\in\Re \)) with
        respect to \( B,C \) by expressing it in terms of \( H \).
      \partsitem Suppose that \( \map{h,g}{V}{W} \) are represented with 
        respect to \( B,C \) by \( H \) and \( G \).
        Give the matrix representing \( h+g \) with
        respect to \( B,C \) by expressing it in terms of \( H \) and \( G \).
      \partsitem Suppose that \( \map{h}{V}{W} \) is represented 
        with respect to \( B,C \) by \( H \) and
        \( \map{g}{W}{X} \) is represented with respect to
        \( C,D \) by \( G \).
        Give the matrix representing \( \composed{g}{h} \) with
        respect to \( B,D \) by expressing it in terms of \( H \) and \( G \).
    \end{exparts}
    \begin{answer}
       See the following section.
    \end{answer}
\end{exercises}
