% Chapter 1, Topic from _Linear Algebra_ Jim Hefferon
%  http://joshua.smcvt.edu/linalg.html
%  2001-Jun-09
\topic{Computer Algebra Systems}
\index{computer algebra systems|(}
The linear systems in this chapter are small enough that their
solution by hand is easy.
But large systems are easiest, and safest, to do on a computer.
There are special purpose programs such as LINPACK\index{LINPACK} for this job.
Another popular tool is a general purpose computer algebra system,
including both commercial packages such as Maple,\index{Maple} 
Mathematica,\index{Mathematica} or MATLAB,\index{MATLAB} 
or free packages such as Sage\index{Sage}.

For example, in the Topic on Networks, we need to solve this.
\begin{equation*}
  \begin{linsys}{7}
    i_0  &-  &i_1  &-  &i_2  &   &    &  &    &   &    &  &    &=  &0  \\
         &   &i_1  &   &     &-  &i_3 &  &    &-  &i_5 &  &    &=  &0  \\
         &   &     &   &i_2  &   &    &- &i_4 &+  &i_5 &  &    &=  &0  \\
         &   &     &   &     &   &i_3 &+ &i_4 &   &    &- &i_6 &=  &0  \\
         &   &5i_1 &   &     &+  &10i_3  &  & &   &    &  &    &=  &10  \\
         &   &     &   &2i_2 &   &    &+ &4i_4 &  &    &  &    &=  &10  \\
         &   &5i_1 &-  &2i_2 &   &    &  &    &+  &50i_5 &&    &=  &0   
  \end{linsys}
\end{equation*}
It can be done by hand, but it would take a while and be error-prone.
Using a computer is better.

We illustrate by solving that system under Sage.
\begin{lstlisting}
sage: var('i0,i1,i2,i3,i4,i5,i6')
(i0, i1, i2, i3, i4, i5, i6)
sage: network_system=[i0-i1-i2==0, i1-i3-i5==0, 
....:       i2-i4+i5==0,, i3+i4-i6==0, 5*i1+10*i3==10,
....:       2*i2+4*i4==10, 5*i1-2*i2+50*i5==0]
sage: solve(network_system, i0,i1,i2,i3,i4,i5,i6)     
[[i0 == (7/3), i1 == (2/3), i2 == (5/3), i3 == (2/3), 
      i4 == (5/3), i5 == 0, i6 == (7/3)]] 
\end{lstlisting}
Magic.

Here is the same system solved under Maple.
We enter the array of coefficients 
and the vector of constants,
and then we get the solution.
\begin{lstlisting}
> A:=array( [[1,-1,-1,0,0,0,0],
             [0,1,0,-1,0,-1,0],
             [0,0,1,0,-1,1,0],
             [0,0,0,1,1,0,-1],
             [0,5,0,10,0,0,0],
             [0,0,2,0,4,0,0],
             [0,5,-2,0,0,50,0]] );
> u:=array( [0,0,0,0,10,10,0] );
> linsolve(A,u);
      7  2  5  2  5     7
    [ -, -, -, -, -, 0, - ]
      3  3  3  3  3     3
\end{lstlisting}

Systems with infinitely many solutions are entered in the same
way but for the solution the computer will return a parametrization.




\begin{exercises}
  % \item[]\textit{Answers for this Topic use Maple as
  %                               the computer algebra system.
  %                               In particular, all of these were tested
  %                               on Maple~$V$ running under MS-DOS
  %                               NT version~$4.0$.
  %                               (On all of them, the preliminary command
  %                               to load the linear algebra package
  %                               %\protect\mbox{\texttt{with(linalg);}},
  %                               along with Maple's responses to the 
  %                               Enter key, 
  %                               have been omitted.)
  %                               Other systems have similar
  %                               commands.}
  \item 
    Use the computer to solve the two problems that opened this
    chapter.
    \begin{exparts}
      \partsitem This is the Statics problem.
         \begin{align*}
            40h+15c  &= 100  \\
            25c      &= 50+50h
         \end{align*}
      \partsitem This is the Chemistry problem.
         \begin{align*} 
             7h      &= 7j  \\
             8h +1i  &= 5j+2k  \\
             1i      &= 3j  \\
             3i      &= 6j+1k
         \end{align*}
    \end{exparts}
    \begin{answer}
      \begin{exparts}
        \partsitem The commands
\begin{indented}{\small
\begin{verbatim}
> A:=array( [[40,15],
             [-50,25]] );
> u:=array([100,50]);
> linsolve(A,u);
\end{verbatim}
}\end{indented}
           yield the answer $[1,4]$.
        \partsitem Here there is a free variable:
\begin{indented}{\small
\begin{verbatim}
> A:=array( [[7,0,-7,0],
             [8,1,-5,2],
             [0,1,-3,0],
             [0,3,-6,-1]] );
> u:=array([0,0,0,0]);
> linsolve(A,u);
\end{verbatim}
}\end{indented}
         prompts the reply $[\_t_1,3\_t_1,\_t_1,3\_t_1]$.
      \end{exparts}
    \end{answer}
\item 
    Use the computer to solve these systems from the
    first subsection,
    or conclude `many solutions' or `no solutions'.
    \begin{exparts*}
      \partsitem \(
               \begin{linsys}[t]{2}
                  2x  &+  &2y  &=  &5  \\
                   x  &-  &4y  &=  &0  
               \end{linsys}
             \)
      \partsitem \(
               \begin{linsys}[t]{2}
                  -x  &+  &y   &=  &1  \\
                   x  &+  &y   &=  &2  
               \end{linsys}
             \) 
      \partsitem  \(
               \begin{linsys}[t]{3}
                   x  &-  &3y  &+  &z  &=  &1  \\
                   x  &+  &y   &+  &2z &=  &14 
                \end{linsys}
             \) 
      \partsitem  \(
               \begin{linsys}[t]{2}
                  -x  &-  &y   &=  &1  \\
                 -3x  &-  &3y  &=  &2  
               \end{linsys}
             \) 
      \partsitem  \(
               \begin{linsys}[t]{3}
                      &   &4y  &+  &z  &=  &20 \\
                  2x  &-  &2y  &+  &z  &=  &0  \\
                   x  &   &    &+  &z  &=  &5  \\
                   x  &+  &y   &-  &z  &=  &10 
                \end{linsys}
             \)
      \partsitem \( \begin{linsys}[t]{4}
                 2x  &   &   &+  &z  &+  &w  &=  &5  \\
                     &   &y  &   &   &-  &w  &=  &-1 \\
                 3x  &   &   &-  &z  &-  &w  &=  &0  \\
                 4x  &+  &y  &+  &2z &+  &w  &=  &9  
               \end{linsys}
            \)
    \end{exparts*}
    \begin{answer}
      These are easy to type in.
      For instance, the first  
\begin{indented}{\small
\begin{verbatim}
> A:=array( [[2,2],
             [1,-4]] );
> u:=array([5,0]);
> linsolve(A,u);
\end{verbatim}
}\end{indented}
      gives the expected answer of $[2,1/2]$.
      The others are entered similarly.
      \begin{exparts}
        \partsitem The answer is \( x=2 \) and \( y=1/2 \).
        \partsitem The answer is \( x=1/2 \) and \( y=3/2 \). 
        \partsitem This system has infinitely many solutions.
           In the first subsection, with $z$ as a parameter, 
           we got $x=(43-7z)/4$ and $y=(13-z)/4$.
           Maple responds with $[-12+7\_t_1,\_t_1,13-4\_t_1]$,
           for some reason preferring $y$ as a parameter.
        \partsitem There is no solution to this system.
           When the array $A$ and vector $u$ are given to Maple
           and it is asked to \texttt{linsolve(A,u)}, 
           it returns no result at all, that is, it responds with
           no solutions.
        \partsitem The solutions is \( (x,y,z)=(5,5,0) \).
        \partsitem There are many solutions.
           Maple gives $[1,-1+\_t_1,3-\_t_1,\_t_1]$.
      \end{exparts}
    \end{answer}
  \item 
    Use the computer to solve these systems from the second subsection.
    \begin{exparts*}
      \partsitem \( \begin{linsys}[t]{2}
                  3x  &+  &6y  &=  &18  \\
                   x  &+  &2y  &=  &6   
                   \end{linsys}  \)
      \partsitem \( \begin{linsys}[t]{2}
                   x  &+  &y   &=  &1  \\
                   x  &-  &y   &=  &-1   
                    \end{linsys}  \)
      \partsitem \( \begin{linsys}[t]{3}
                   x_1  &   &     &+  &x_3   &=  &4  \\
                   x_1  &-  &x_2  &+  &2x_3  &=  &5  \\
                  4x_1  &-  &x_2  &+  &5x_3  &=  &17  
                   \end{linsys}  \)
      \partsitem \( \begin{linsys}[t]{3}
                   2a   &+  &b    &-  &c     &=  &2  \\
                   2a   &   &     &+  &c     &=  &3  \\
                    a   &-  &b    &   &      &=  &0   
                    \end{linsys}  \)
      \partsitem \( \begin{linsys}[t]{4}
                     x  &+  &2y   &-   &z   &    &    &=  &3  \\
                    2x  &+  &y    &    &    &+   &w   &=  &4  \\
                     x  &-  &y    &+   &z   &+   &w   &=  &1  
                    \end{linsys}  \)
      \partsitem \( \begin{linsys}[t]{4}
                     x  &   &     &+   &z   &+   &w   &=  &4  \\
                    2x  &+  &y    &    &    &-   &w   &=  &2  \\
                    3x  &+  &y    &+   &z   &    &    &=  &7  
                     \end{linsys}  \)
    \end{exparts*}
    \begin{answer}
      As with the prior question, entering these is easy.
      \begin{exparts}
        \partsitem This system has infinitely many solutions. 
              In the second subsection we gave the solution set as
              \begin{equation*}
              \set{\colvec{6 \\ 0}+\colvec{-2 \\ 1}y
                      \suchthat y\in\Re}
              \end{equation*}
              and Maple responds with $[6-2\_t_1,\_t_1]$.
        \partsitem The solution set has only one member
          \begin{equation*}
             \set{\colvec{0 \\ 1} }
          \end{equation*}
          and Maple has no trouble finding it $[0,1]$.
        \partsitem This system's solution set is infinite
          \begin{equation*}
            \set{\colvec{4 \\ -1 \\ 0}+\colvec{-1 \\ 1 \\ 1}x_3
                             \suchthat x_3\in\Re}
          \end{equation*}
          and Maple gives $[\_t_1,-\_t_1+3,-\_t_1+4]$.
        \partsitem There is a unique solution
           \begin{equation*}
             \set{\colvec{1 \\ 1 \\ 1}}
           \end{equation*}
           and Maple gives $[1,1,1]$.
        \partsitem This system has infinitely many solutions; in the 
           second subsection we described the solution set with
           two parameters
           \begin{equation*}
             \set{\colvec{5/3 \\ 2/3 \\ 0 \\ 0}
                  +\colvec{-1/3 \\ 2/3 \\ 1 \\ 0}z
                  +\colvec{-2/3 \\ 1/3 \\ 0 \\ 1}w
                  \suchthat z,w\in\Re}
           \end{equation*}
           as does Maple $[3-2\_t_1+\_t_2,\_t_1,\_t_2,-2+3\_t_1-2\_t_2]$.
        \partsitem The solution set is empty and Maple replies to the
           \texttt{linsolve(A,u)} command with no returned solutions.
      \end{exparts}
    \end{answer}
  \item 
    What does the computer give for the solution of the general
    $\nbyn{2}$  system?
    \begin{equation*}
      \begin{linsys}{2}
        ax  &+  &cy  &=  &p  \\
        bx  &+  &dy  &=  &q
      \end{linsys}
    \end{equation*}
    \begin{answer}
       In response to this prompting
\begin{indented}{\small
\begin{verbatim}
> A:=array( [[a,c],
             [b,d]] );
> u:=array([p,q]);
> linsolve(A,u);
\end{verbatim}
}\end{indented}
      Maple thought for perhaps twenty seconds and gave this reply.
      \begin{equation*}
        \bigl[-\frac{-d\,p+q\,c}{-b\,c+a\,d},
          \frac{-b\,p+a\,q}{-b\,c+a\,d}\bigr]
      \end{equation*}
    \end{answer}
\end{exercises}
\index{computer algebra systems|)}
\endinput


