% Chapter 4, Section 2 _Linear Algebra_ Jim Hefferon
%  http://joshua.smcvt.edu/linalg.html
%  2001-Jun-12
\section{Similarity}
\index{similarity|(}

\subsection{Definition and Examples}
We've defined \( H \) and \( \hat{H} \) to be
matrix-equivalent if there are nonsingular matrices \( P \) and \( Q \)
such that \( \hat{H}=PHQ \).
That definition is motivated by this diagram
\begin{equation*}
  \begin{CD}
    V_\wrt{B}                   @>h>H>        W_\wrt{D}       \\
    @V{\scriptstyle\identity} VV              @V{\scriptstyle\identity} VV \\
    V_\wrt{\hat{B}}             @>h>\hat{H}>  W_\wrt{\hat{D}}
  \end{CD}
\end{equation*}
showing that $H$ and $\hat{H}$ both represent $h$ but with respect to different
pairs of bases.
We now specialize that setup to the case
where the codomain equals the domain, and
where the codomain's basis equals the domain's basis.\index{arrow diagram}
\begin{equation*}
  \begin{CD}
    V_\wrt{B}                   @>t>>        V_\wrt{B}       \\
    @V{\scriptstyle\identity} VV              @V{\scriptstyle\identity} VV \\
    V_\wrt{D}                   @>t>>        V_\wrt{D}
  \end{CD}
\end{equation*}
To move from the lower left to the lower right we can either go straight
over, or up, over, and then down. 
In matrix terms, 
\begin{equation*}
\rep{t}{D,D}
  =\rep{\identity}{B,D}\;\rep{t}{B,B}\;\bigl(\rep{\identity}{B,D}\bigr)^{-1}
\end{equation*}
(recall that a representation of composition like this one reads right to
left).

\begin{definition}
The matrices  \( T \) and $S$ are 
\definend{similar}\index{matrix!similarity}%
\index{equivalence relation!matrix similarity}\index{similar matrices}
if there is a nonsingular \( P \) such that
$
  T=PSP^{-1}
$.
\end{definition}

%\noindent (\nearbyexercise{exer:SimIsEquivRel} checks that
%similarity is an equivalence relation.)

\noindent Since nonsingular matrices are square, 
the similar matrices $T$ and $S$ must
be square and of the same size.

\begin{example}
%One way to produce similar matrices is to start with an arbitrary nonsingular
%matrix $P$, and an arbitrary $T$.
With these two,
\begin{equation*}
  P=
  \begin{pmatrix}
    2  &1  \\
    1  &1
  \end{pmatrix}
  \qquad
  S=
  \begin{pmatrix}
    2  &-3  \\
    1  &-1
  \end{pmatrix}
\end{equation*}
calculation gives that $S$ is similar to
this matrix.
\begin{equation*}
  T=
  \begin{pmatrix}
    0  &-1  \\
    1  &1
  \end{pmatrix}
\end{equation*}
\end{example}

\begin{example}  \label{ex:OnlyZeroSimToZero}
The only matrix similar to the zero matrix is itself:~$PZP^{-1}=PZ=Z$.
The only matrix similar to the identity matrix is 
itself:~$PIP^{-1}=PP^{-1}=I$.
\end{example}

Since matrix similarity is a special case of matrix equivalence, 
if two matrices are similar then they are equivalent.
What about the converse:~must matrix equivalent square matrices be similar?
The answer is no.
The prior example shows that the similarity classes are different
from the matrix equivalence classes, because the matrix equivalence class
of the identity consists of all nonsingular matrices of that size. 
Thus, for instance, these two are
matrix equivalent but not similar.
\begin{equation*}
  T=
  \begin{pmatrix}
    1  &0  \\
    0  &1
  \end{pmatrix}
  \qquad
  S=
  \begin{pmatrix}
    1  &2  \\
    0  &3
  \end{pmatrix}
\end{equation*}

So some matrix equivalence classes
split into two or more similarity classes\Dash similarity gives a finer
partition than does equivalence.
This picture shows some matrix equivalence classes subdivided into
similarity classes.
\begin{center}
  \includegraphics{ch5.4}
\end{center}

To understand the similarity relation we shall study the similarity classes.
We approach this question in the same way that we've studied both the
row equivalence and matrix equivalence relations, by finding
a canonical form for
representatives\appendrefs{representatives}\spacefactor=1000 %
of the similarity classes, called Jordan form.
With this canonical form, we can decide if two matrices are similar by checking
whether they reduce to the same representative.
We've also seen with both row equivalence and matrix equivalence that a
canonical form gives us insight into the ways in which members of
the same class are alike
(e.g., two identically-sized matrices are matrix equivalent
if and only if they have the same rank).
%(Along the way we shall see ideas that are interesting and important
%in their own right, not just as stepping stones to Jordan form.)

\begin{exercises}
  \item 
    For
    \begin{equation*}
      S=
      \begin{pmatrix}
        1  &3  \\
       -2  &-6
      \end{pmatrix}
      \quad
      T=
      \begin{pmatrix}
        0    &0  \\
       -11/2 &-5
      \end{pmatrix}
      \quad
      P=
      \begin{pmatrix}
        4  &2  \\
       -3  &2
      \end{pmatrix}
    \end{equation*}
    check that $T=PSP^{-1}$.
    \begin{answer}
      One way to proceed is left to right.
      \begin{equation*}
        PSP^{-1}=
        \begin{pmatrix}
          4  &2  \\
         -3  &2
        \end{pmatrix}
        \begin{pmatrix}
          1  &3  \\
         -2  &-6
        \end{pmatrix}
        \begin{pmatrix}
          2/14  &-2/14  \\
          3/14  &4/14
        \end{pmatrix}
        =
        \begin{pmatrix}
          0  &0  \\
         -7  &-21
        \end{pmatrix}  
        \begin{pmatrix}
          2/14  &-2/14  \\
          3/14  &4/14
        \end{pmatrix}
        =
        \begin{pmatrix}
          0    &0  \\
         -11/2 &-5
        \end{pmatrix}
      \end{equation*}
    \end{answer}
  \recommended \item
    \nearbyexample{ex:OnlyZeroSimToZero} shows that the only matrix similar
    to a zero matrix is itself and that 
    the only matrix similar to the identity
    is itself.
    \begin{exparts}
      \partsitem Show that the $\nbyn{1}$ matrix $(2)$,
         also, is similar only to itself.
      \partsitem Is a matrix of the form $cI$ for some scalar $c$
         similar only to itself?
     \partsitem Is a diagonal matrix similar only to itself?
     % \partsitem Is a square block partial-identity matrix similar only to 
     % itself?
    \end{exparts}
    \begin{answer}
     \begin{exparts}
      \partsitem Because the matrix $(2)$ is $\nbyn{1}$, the matrices
         $P$ and $P^{-1}$ are also $\nbyn{1}$ and so where
         $P=(p)$ the inverse is $P^{-1}=(1/p)$.  
         Thus $P(2)P^{-1}=(p)(2)(1/p)=(2)$.
      \partsitem Yes:~recall that scalar multiples can be brought out 
        of a matrix \( P(cI)P^{-1}=cPIP^{-1}=cI \).
        By the way, the zero and identity matrices are the special cases
        $c=0$ and $c=1$.
      \partsitem No, as this example shows.
        \begin{equation*}
           \begin{pmatrix}
              1  &-2  \\
             -1  &1
            \end{pmatrix}
           \begin{pmatrix}
             -1  &0   \\
              0  &-3
           \end{pmatrix}
           \begin{pmatrix}
              -1  &-2   \\
              -1  &-1
           \end{pmatrix}
           =
           \begin{pmatrix}
              -5  &-4   \\
              2   &1
           \end{pmatrix}
        \end{equation*} 
    \end{exparts}  
   \end{answer}
  \item 
    Show that these matrices are not similar.
    \begin{equation*}
       \begin{pmatrix}
          1  &0  &4  \\
          1  &1  &3  \\
          2  &1  &7
       \end{pmatrix}
       \qquad
       \begin{pmatrix}
          1  &0  &1  \\
          0  &1  &1  \\
          3  &1  &2
       \end{pmatrix}
    \end{equation*}
    \begin{answer}
      Gauss' method shows that
      the first matrix represents maps of rank two while the second
      matrix represents maps of rank three.
    \end{answer}
  \item 
    Consider the transformation $\map{t}{\polyspace_2}{\polyspace_2}$
    described by
    $x^2\mapsto x+1$, $x\mapsto x^2-1$, and $1\mapsto 3$.
    \begin{exparts}
      \partsitem Find $T=\rep{t}{B,B}$ where $B=\sequence{x^2,x,1}$. 
      \partsitem Find $S=\rep{t}{D,D}$ where $D=\sequence{1,1+x,1+x+x^2}$. 
      \partsitem Find the matrix $P$ such that $T=PSP^{-1}$. 
    \end{exparts}
    \begin{answer}
      \begin{exparts}
        \partsitem Because $t$ is described with the members of $B$,
          finding the matrix representation is easy:
          \begin{equation*}
            \rep{t(x^2)}{B}=\colvec{0 \\ 1 \\ 1}_B
            \quad
            \rep{t(x)}{B}=\colvec{1 \\ 0 \\ -1}_B
            \quad
            \rep{t(1)}{B}=\colvec{0 \\ 0 \\ 3}_B
          \end{equation*}
          gives this.
          \begin{equation*}
            \rep{t}{B,B}
            \begin{pmatrix}
              0  &1  &0  \\
              1  &0  &0  \\
              1  &-1 &3  
            \end{pmatrix}
          \end{equation*}
        \partsitem We will find $t(1)$, $t(1+x)$, and $t(1+x+x^2$,
          to find how each is represented with respect to $D$.
          We are given that $t(1)=3$, and the other two are easy to see:
          $t(1+x)=x^2+2$ and $t(1+x+x^2)=x^2+x+3$.
          By eye, we get the representation of each vector
          \begin{equation*}
            \rep{t(1)}{D}=\colvec{3 \\ 0 \\ 0}_D
            \quad
            \rep{t(1+x)}{D}=\colvec{2  \\ -1 \\  1}_D
            \quad
            \rep{t(1+x+x^2)}{D}=\colvec{2 \\ 0 \\ 1}_D
          \end{equation*}
          and thus the representation of the map.
          \begin{equation*}
            \rep{t}{D,D}
            =
            \begin{pmatrix}
              3  &2  &2  \\
              0  &-1 &0  \\
              0  &1  &1
            \end{pmatrix}
          \end{equation*}
         \partsitem The diagram, adapted for this $T$ and $S$,
           \begin{equation*}
             \begin{CD}
               V_\wrt{D}                  @>t>S>  V_\wrt{D}       \\
               @V\scriptstyle\identity VPV      @V\scriptstyle\identity VPV \\
               V_\wrt{B}                  @>t>T>  V_\wrt{B}
             \end{CD}
           \end{equation*}
           shows that $P=\rep{\identity}{D,B}$.
           \begin{equation*}
             P=
             \begin{pmatrix}
               0  &0  &1  \\ 
               0  &1  &1  \\
               1  &1  &1 
             \end{pmatrix}
           \end{equation*}
      \end{exparts}
    \end{answer}
  \recommended \item
     Exhibit an nontrivial similarity relationship in this way:~let
     \( \map{t}{\C^2}{\C^2} \) act by
     \begin{equation*}
        \colvec{1 \\ 2}\mapsto\colvec{3 \\ 0}
        \qquad
        \colvec{-1 \\ 1}\mapsto\colvec{-1 \\ 2}
     \end{equation*}
     and pick two bases,
     and represent \( t \) with respect to then
     \( T=\rep{t}{B,B} \) and \( S=\rep{t}{D,D} \).
     Then compute 
     the \( P \) and \( P^{-1} \) to change bases from \( B \) to \( D \) and
     back again.
     \begin{answer}
       One possible choice of the bases is 
       \begin{equation*}
         B=\sequence{\colvec{1 \\ 2},\colvec{-1 \\ 1}}
         \qquad
         D=\stdbasis_2=\sequence{\colvec{1 \\ 0},\colvec{0 \\ 1}}
       \end{equation*}
       (this $B$ is suggested by the map description).
       To find the matrix $T=\rep{t}{B,B}$, solve the relations 
       \begin{equation*}
          c_1\colvec{1 \\ 2}+c_2\colvec{-1 \\ 1}=\colvec{3 \\ 0}
          \qquad
         \hat{c}_1\colvec{1 \\ 2}+\hat{c}_2\colvec{-1 \\ 1}=\colvec{-1 \\ 2}
       \end{equation*}
       to get \( c_1=1 \), \( c_2=-2 \), \( \hat{c}_1=1/3 \) and
       \( \hat{c}_2=4/3 \).
       \begin{equation*}
          \rep{t}{B,B}=
          \begin{pmatrix}
             1  &1/3 \\
            -2  &4/3
          \end{pmatrix}
       \end{equation*}

       Finding \( \rep{t}{D,D} \) involves a bit more computation.
       We first find \( t(\vec{e}_1) \).
       The relation
       \begin{equation*}
         c_1\colvec{1 \\ 2}+c_2\colvec{-1 \\ 1}=\colvec{1 \\ 0}
       \end{equation*}
       gives \( c_1=1/3 \) and \( c_2=-2/3 \), and so 
       \begin{equation*}
         \rep{\vec{e}_1}{B}=\colvec{1/3 \\ -2/3}_B
       \end{equation*}
       making
       \begin{equation*}
          \rep{t(\vec{e}_1)}{B}=
                 \begin{pmatrix}
                    1  &1/3  \\
                   -2  &4/3
                 \end{pmatrix}_{B,B}
                 \colvec{1/3 \\ -2/3}_B
                 =
                 \colvec{1/9 \\ -14/9}_B
       \end{equation*}
       and hence $t$ acts on the first basis vector $\vec{e}_1$ in this way.
       \begin{equation*}
         t(\vec{e}_1)
         =(1/9)\cdot\colvec{1 \\ 2}-(14/9)\cdot\colvec{-1 \\ 1}
         =\colvec{5/3 \\ -4/3}      
       \end{equation*}
       The computation for \( t(\vec{e}_2) \) is similar.
       The relation
       \begin{equation*}
         c_1\colvec{1 \\ 2}+c_2\colvec{-1 \\ 1}=\colvec{0 \\ 1}
       \end{equation*}
       gives \( c_1=1/3 \) and \( c_2=1/3 \), so
       \begin{equation*}
         \rep{\vec{e}_1}{B}=\colvec{1/3 \\ 1/3}_B      
       \end{equation*}
       making
       \begin{equation*}
         \rep{t(\vec{e}_1)}{B}=
                 \begin{pmatrix}
                    1  &1/3  \\
                   -2  &4/3
                 \end{pmatrix}_{B,B}
                 \colvec{1/3 \\ 1/3}_B
                 =
                 \colvec{4/9 \\ -2/9}_B
       \end{equation*}
       and hence $t$ acts on the second basis vector $\vec{e}_2$ in this way.
       \begin{equation*}
         t(\vec{e}_2)
         =(4/9)\cdot\colvec{1 \\ 2}-(2/9)\cdot\colvec{-1 \\ 1}
         =\colvec{2/3 \\ 2/3}
       \end{equation*}
       Therefore
       \begin{equation*}
          \rep{t}{D,D}=
          \begin{pmatrix}
             5/3  &2/3  \\
            -4/3  &2/3
          \end{pmatrix}
       \end{equation*}
       and these are the change of basis matrices.
       \begin{equation*}
         P=\rep{\identity}{B,D}
         =\begin{pmatrix}
           1  &-1  \\
           2  &1 
         \end{pmatrix}
         \qquad
         P^{-1}=\bigl(\rep{\identity}{B,D}\bigr)^{-1}
         =\begin{pmatrix}
            1  &-1 \\
            2  &1
         \end{pmatrix}^{-1}
         =
         \begin{pmatrix}
           1/3  &1/3  \\
           -2/3 &1/3
         \end{pmatrix}
      \end{equation*}
      The check of these computations is routine.
      \begin{equation*}
         \begin{pmatrix}
            1  &-1 \\
            2  &1
         \end{pmatrix}
         \begin{pmatrix}
            1  &1/3 \\
           -2  &4/3
         \end{pmatrix}
         \begin{pmatrix}
           1/3 &1/3 \\
          -2/3 &1/3
         \end{pmatrix}
         =
         \begin{pmatrix}
           5/3 &2/3 \\
          -4/3 &2/3
         \end{pmatrix}
      \end{equation*}  
    \end{answer}
  \item 
    Explain \nearbyexample{ex:OnlyZeroSimToZero} in terms of maps.
    \begin{answer}
      The only representation of a zero map is a zero matrix,
      no matter what the pair of bases $\rep{z}{B,D}=Z$,
      and so in particular for any single basis $B$ we have $\rep{z}{B,B}=Z$.  
      The case of the identity is related, but slightly different:~the 
      only representation of the identity map, with respect to any $B,B$, 
      is the identity $\rep{\identity}{B,B}=I$.
      (\textit{Remark:}~of course, we have seen examples where $B\neq D$ and 
      $\rep{\identity}{B,D}\neq I$\Dash in fact, we have seen that any 
      nonsingular matrix is a representation of the identity map with
      respect to some $B,D$.)
    \end{answer}
  \recommended \item 
    Are there two matrices \( A \) and \( B \) that are
    similar while \( A^2 \) and \( B^2 \) are not similar?
    \cite{Halmos}
    \begin{answer}
      No.
      If \( A=PBP^{-1} \) then \( A^2=(PBP^{-1})(PBP^{-1})=PB^2P^{-1} \).
    \end{answer}
  \recommended \item
    Prove that if two matrices are similar and one is invertible then
    so is the other.
    \begin{answer}
       Matrix similarity is a special case of matrix equivalence 
       (if matrices are similar then they are matrix equivalent)
       and matrix equivalence preserves nonsingularity.
    \end{answer}
  \recommended \item \label{exer:SimIsEquivRel}
    Show that similarity is an equivalence relation.
    \begin{answer}
       A matrix is similar to itself; take \( P \) to be the identity
       matrix:~$IPI^{-1}=IPI=P$.

       If \( T \) is similar to \( S \) then \( T=PSP^{-1} \)
       and so \( P^{-1}TP=S \).
       Rewrite this as \( S=(P^{-1})T(P^{-1})^{-1} \) to conclude that
       $S$ is similar to \( T \).

       If \( T \) is similar to \( S \) and \( S \) is similar to \( U \)
       then \( T=PSP^{-1} \) and \( S=QUQ^{-1} \).
       Then \( T=PQUQ^{-1}P^{-1}=(PQ)U(PQ)^{-1} \), showing that \( T \)
       is similar to \( U \).  
     \end{answer}
  \item 
     Consider a 
     matrix representing, with respect to some $B,B$, 
     reflection across the \( x \)-axis in \( \Re^2 \).
     Consider also  
     a matrix representing, with respect to some $D,D$,
     reflection across the \( y \)-axis.
     Must they be similar?
     \begin{answer}
        Let $f_x$ and $f_y$ be the reflection maps (sometimes called `flip's).
        For any bases
        \( B \) and \( D \), the matrices \( \rep{f_x}{B,B}  \) and
        \( \rep{f_y}{D,D} \) are similar.
        First note that
        \begin{equation*}
           S=\rep{f_x}{\stdbasis_2,\stdbasis_2}=
           \begin{pmatrix}
              1  &0  \\
              0  &-1
           \end{pmatrix}
           \qquad
           T=\rep{f_y}{\stdbasis_2,\stdbasis_2}=
           \begin{pmatrix}
             -1  &0  \\
              0  &1
           \end{pmatrix}
        \end{equation*}
        are similar because the second matrix is the representation of $f_x$
        with respect to the basis \( A=\sequence{\vec{e}_2,\vec{e}_1} \):
        \begin{equation*}
           \begin{pmatrix}
              1  &0  \\
              0  &-1
           \end{pmatrix}
           =    
           P
           \begin{pmatrix}
             -1  &0  \\
              0  &1
           \end{pmatrix}
           P^{-1}
        \end{equation*}
        where $P=\rep{\identity}{A,\stdbasis_2}$.
        \begin{equation*}
          \begin{CD}
            \Re^2_\wrt{A}                   
               @>f_x>T>        
               V\Re^2_\wrt{A}       \\
            @V\scriptstyle\identity VPV                
               @V\scriptstyle\identity VPV \\
            \Re^2_\wrt{\stdbasis_2}         
               @>f_x>S>        
               \Re^2_\wrt{\stdbasis_2}
          \end{CD}
        \end{equation*}
        Now the conclusion follows from the transitivity part of
        \nearbyexercise{exer:SimIsEquivRel}.
        
        To finish without relying on that exercise, write
        $\rep{f_x}{B,B}=QTQ^{-1}=Q\rep{f_x}{\stdbasis_2,\stdbasis_2}Q^{-1}$
        and
        $\rep{f_y}{D,D}=RSR^{-1}=R\rep{f_y}{\stdbasis_2,\stdbasis_2}R^{-1}$.
        Using the equation in the first paragraph, 
        the first of these two becomes
        $\rep{f_x}{B,B}=QP\rep{f_y}{\stdbasis_2,\stdbasis_2}P^{-1}Q^{-1}$
        and  
        rewriting the second of these two as
        $R^{-1}\cdot\rep{f_y}{D,D}\cdot R=\rep{f_y}{\stdbasis_2,\stdbasis_2}$
        and substituting gives the desired relationship
        \begin{multline*}
          \rep{f_x}{B,B}
          =QP\rep{f_y}{\stdbasis_2,\stdbasis_2}P^{-1}Q^{-1}  \\
          =QPR^{-1}\cdot\rep{f_y}{D,D}\cdot RP^{-1}Q^{-1}
          =(QPR^{-1})\cdot\rep{f_y}{D,D}\cdot (QPR^{-1})^{-1}
        \end{multline*}
        Thus the matrices \( \rep{f_x}{B,B}  \) and \( \rep{f_y}{D,D} \) are
        similar. 
      \end{answer}
  \item 
     Prove that similarity preserves determinants and rank.
     Does the converse hold?
     \begin{answer}
       We must show that if two matrices are similar then they have the same
       determinant and the same rank.
       Both determinant and rank are properties of matrices that we 
       have already shown to be preserved by matrix equivalence. 
       They are therefore preserved by similarity (which is a 
       special case of matrix equivalence:~if two matrices
       are similar then they are matrix equivalent).

       To prove the statement without quoting the results about
       matrix equivalence, note first that
       rank is a property of the map (it is the dimension of the rangespace)
       and since we've shown that 
       the rank of a map is the rank of a representation,
       it must be the same for all representations.
       As for determinants,
       \( \deter{PSP^{-1}}=\deter{P}\cdot\deter{S}\cdot\deter{P^{-1}}
          =\deter{P}\cdot\deter{S}\cdot\deter{P}^{-1}=\deter{S} \).

       The converse of the statement does not hold; 
       for instance,
       there are matrices with the same determinant that are not similar.
       To check this, consider a nonzero matrix with a
       determinant of zero.
       It is not similar to the zero matrix, the zero matrix is similar
       only to itself, but they have they same determinant. 
       The argument for rank is much the same. 
     \end{answer}
  \item 
    Is there a matrix equivalence class with only one matrix similarity
    class inside?
    One with infinitely many similarity classes?
    \begin{answer}
      The matrix equivalence class containing all \( \nbyn{n} \) rank
      zero matrices contains only a single matrix, the zero matrix.
      Therefore it has as a subset only one similarity class.

      In contrast, the matrix equivalence class of \( \nbyn{1} \) matrices
      of rank one consists of those 
      $\nbyn{1}$ matrices \( (k) \) where \( k\neq 0 \).
      For any basis \( B \), the representation
      of multiplication by the scalar \( k \)
      is \( \rep{t_k}{B,B}=(k) \),
      so each such matrix is alone in its similarity class.
      So this is a case where a matrix equivalence class splits into
      infinitely many similarity classes.  
     \end{answer}
  \item 
    Can two different diagonal matrices be in the same similarity class?
    \begin{answer}
      Yes, these are similar
      \begin{equation*}
         \begin{pmatrix}
           1  &0  \\
           0  &3
         \end{pmatrix}
         \qquad
         \begin{pmatrix}
           3  &0  \\
           0  &1
         \end{pmatrix}
      \end{equation*}
      since, where the first matrix is $\rep{t}{B,B}$ for 
      $B=\sequence{\vec{\beta}_1,\vec{\beta}_2}$, 
      the second matrix is $\rep{t}{D,D}$ for 
      $D=\sequence{\vec{\beta}_2,\vec{\beta}_1}$.
     \end{answer}
  \recommended \item
    Prove that if two matrices are similar then their \( k \)-th powers
    are similar when \( k>0 \).
    What if \( k\leq 0 \)?
    \begin{answer}
      The \( k \)-th powers are similar because, where each matrix represents
      the map $t$, the $k$-th powers represent
      \( t^k \), the composition of $k$-many $t$'s.
      (For instance, if $T=rep{t}{B,B}$ then $T^2=\rep{\composed{t}{t}}{B,B}$.)

      Restated more computationally, if \( T=PSP^{-1} \) then
      \( T^2=(PSP^{-1})(PSP^{-1})=PS^2P^{-1} \).
      Induction extends that to all powers.

      For the $k\leq 0$ case, 
      suppose that \( S \) is invertible and that \( T=PSP^{-1} \).
      Note that \( T \) is invertible:
      \( T^{-1}=(PSP^{-1})^{-1}=PS^{-1}P^{-1} \),
      and that same equation shows that 
      \( T^{-1} \) is similar to \( S^{-1} \).
      Other negative powers are now given by the first paragraph.  
    \end{answer}
  \recommended \item
    Let \( p(x) \) be the polynomial \( c_nx^n+\cdots+c_1x+c_0 \).
    Show that if \( T \) is similar to \( S \) then
    \( p(T)=c_nT^n+\cdots+c_1T+c_0I \) is similar to
    \( p(S)=c_nS^n+\cdots+c_1S+c_0I \).
    \begin{answer}
       In conceptual terms, both represent \( p(t) \) for some
       transformation \( t \).
       In computational terms, we have this.
       \begin{align*}
          p(T)
          &=c_n(PSP^{-1})^n+\dots+c_1(PSP^{-1})+c_0I   \\
          &=c_nPS^nP^{-1}+\dots+c_1PSP^{-1}+c_0I   \\
          &=Pc_nS^nP^{-1}+\dots+Pc_1SP^{-1}+Pc_0P^{-1}   \\
          &=P(c_nS^n+\dots+c_1S+c_0)P^{-1}
       \end{align*}  
     \end{answer}
  \item 
    List all of the matrix equivalence classes of \( \nbyn{1} \) matrices.
    Also list the similarity classes, and describe which similarity classes are
    contained inside of each matrix equivalence class.
    \begin{answer}
      There are two equivalence classes, (i)~the class of rank~zero matrices, 
      of which there is one:
      $\mathscr{C}_1=\set{(0)}$,
      and (2)~the class of rank~one matrices,
      of which there are infinitely many: 
      \( \mathscr{C}_2=\set{(k)\suchthat k\neq0} \).

      Each \( \nbyn{1} \) matrix is alone in its similarity class.
      That's because any transformation of a one-dimensional space
      is multiplication by a scalar $\map{t_k}{V}{V}$ given by
      $\vec{v}\mapsto k\cdot\vec{v}$. 
      Thus, for any basis \( B=\sequence{\vec{\beta}} \),
      the matrix representing a transformation \( t_k \) 
      with respect to \( B,B \) is
      \( (\rep{t_k(\vec{\beta})}{B})=(k) \). 
      
      So, contained in the matrix equivalence class
      $\mathscr{C}_1$ is (obviously) the single 
      similarity class consisting of the matrix $(0)$.
      And, contained in the matrix equivalence class $\mathscr{C}_2$ are the
      infinitely many, one-member-each, similarity classes consisting of
      $(k)$ for $k\neq0$.  
    \end{answer}
  \item 
    Does similarity preserve sums?
    \begin{answer}
      No.
      Here is an example that has two pairs, each of two similar matrices:
      \begin{equation*}
         \begin{pmatrix}
            1  &-1  \\
            1  &2
         \end{pmatrix}
         \begin{pmatrix}
            1  &0   \\
            0  &3
         \end{pmatrix}
         \begin{pmatrix}
            2/3  &1/3   \\
           -1/3  &1/3
         \end{pmatrix}
         =
         \begin{pmatrix}
            5/3  &-2/3  \\
           -4/3  &7/3
         \end{pmatrix}
      \end{equation*}
      and
      \begin{equation*}
         \begin{pmatrix}
            1  &-2  \\
           -1  &1
         \end{pmatrix}
         \begin{pmatrix}
           -1  &0   \\
            0  &-3
         \end{pmatrix}
         \begin{pmatrix}
            -1  &-2   \\
            -1  &-1
         \end{pmatrix}
         =
         \begin{pmatrix}
            -5  &-4   \\
             2  &1
         \end{pmatrix}
      \end{equation*}
      (this example is mostly arbitrary, but not entirely, because the
      the center matrices on the two left sides add to the zero matrix).
      Note that the sums of these similar matrices are not similar
      \begin{equation*}
         \begin{pmatrix}
            1  &0   \\
            0  &3
         \end{pmatrix}
         +
         \begin{pmatrix}
           -1  &0   \\
            0  &-3
         \end{pmatrix}
         =
         \begin{pmatrix}
           0  &0  \\
           0  &0
         \end{pmatrix}
         \qquad
         \begin{pmatrix}
            5/3  &-2/3   \\
            -4/3 &7/3
         \end{pmatrix}
         +
         \begin{pmatrix}
            -5  &-4   \\
             2  &1
         \end{pmatrix}
         \neq
         \begin{pmatrix}
           0  &0  \\
           0  &0
         \end{pmatrix}
      \end{equation*}
      since the zero matrix is similar only to itself.
    \end{answer}
  \item 
    Show that if \( T-\lambda I \) and \( N \) are similar matrices then
    \( T \) and \( N+\lambda I \) are also similar.
    \begin{answer}
    If \( N=P(T-\lambda I)P^{-1} \) then
    \( N=PTP^{-1}-P(\lambda I)P^{-1} \).
    The diagonal matrix \( \lambda I \) commutes with anything, so
    \( P(\lambda I)P^{-1}=PP^{-1}(\lambda I)=\lambda I \).
    Thus \( N=PTP^{-1}-\lambda I \) and
    consequently \( N+\lambda I=PTP^{-1} \).
    (So not only are they similar, in fact they are similar via 
    the same \( P \).)
    \end{answer}
\end{exercises}

















\subsection{Diagonalizability}
The prior subsection defines the relation of similarity and shows that,
although similar matrices are necessarily matrix equivalent, the converse
does not hold.
Some matrix-equivalence classes break into two or more similarity 
classes (the nonsingular $\nbyn{n}$ matrices, for instance).
This means that the canonical form for matrix equivalence, 
a block partial-identity, cannot be used as a canonical form 
for matrix similarity because
the partial-identities cannot be in more than one
similarity class, so there are similarity classes without one.
This picture illustrates.
As earlier in this book, class representatives are shown with stars.
\begin{center}
  \includegraphics{ch5.5}
\end{center}
We are developing a canonical form for representatives of
the similarity classes.
We naturally try to build on our previous work, meaning 
first that the partial identity matrices should represent the similarity
classes into which they fall, 
and beyond that, that the representatives should be as simple as possible. 
The simplest extension of the partial-identity form is a diagonal form.

\begin{definition}
A transformation is \definend{diagonalizable}\index{diagonalizable|(}%
\index{transformation!diagonalizable}
if it has a diagonal representation
with respect to the same basis for the codomain as for the domain.
A \definend{diagonalizable matrix}\index{matrix!diagonalizable}
is one that is similar to a diagonal matrix:~\( T \) is diagonalizable
if there is a nonsingular \( P \) such that \( PTP^{-1} \) is diagonal.
\end{definition}

\begin{example} \label{ex:DiagTwoByTwo}
The matrix 
\begin{equation*}
  \begin{pmatrix} 
     4 &-2 \\ 
     1 &1 
  \end{pmatrix}
\end{equation*} 
is diagonalizable.
\begin{equation*}
  \begin{pmatrix}
     2  &0   \\
     0  &3
  \end{pmatrix}
  =
  \begin{pmatrix}
    -1  &2  \\
     1  &-1
  \end{pmatrix}
  \begin{pmatrix}
     4  &-2 \\
     1  &1
  \end{pmatrix}
  \begin{pmatrix}
    -1  &2  \\
     1  &-1
  \end{pmatrix}^{-1}
\end{equation*}
\end{example}

\begin{example}
Not every matrix is diagonalizable.
The square of
\begin{equation*}
  N=\begin{pmatrix}
       0  &0  \\
       1  &0
    \end{pmatrix}
\end{equation*}
is the zero matrix.
Thus, for any map~$n$ that \( N \) represents 
(with respect to the same basis for the domain as for the codomain), 
the composition \( \composed{n}{n} \) is the zero map.
This implies that no such map \( n \) can be diagonally represented (with
respect to any $B,B$) because no power
of a nonzero diagonal matrix is zero.
That is, there is no diagonal matrix in $N$'s similarity class.
\end{example}

That example shows that a diagonal form will not do for a 
canonical form\Dash we cannot
find a diagonal matrix in each matrix similarity class.
However, the canonical form that we are developing has the property that if
a matrix can be diagonalized then the diagonal matrix is the canonical
representative of the similarity class. 
The next result characterizes which maps can be diagonalized.

\begin{corollary}
\label{cor:DiagIffBasisOfEigens}
A transformation \( t \) is diagonalizable if and only if
there is a basis
\( B=\sequence{\vec{\beta}_1,\ldots,\vec{\beta}_n } \)
and scalars \( \lambda_1,\ldots,\lambda_n \) such that
\( t(\vec{\beta}_i)=\lambda_i\vec{\beta}_i \)
for each \( i \).
\end{corollary}

\begin{proof}
This follows from the definition by
considering a diagonal representation matrix.
\begin{equation*}
   \rep{t}{B,B}=
   \begin{pmat}{c|@{\hspace*{1em}}c@{\hspace*{1em}}|c}
      \vdots                    &       &\vdots                     \\
      \rep{t(\vec{\beta}_1)}{B} &\cdots &\rep{t(\vec{\beta}_n)}{B}  \\
      \vdots                    &       &\vdots
   \end{pmat}
   =
   \begin{pmat}{c|@{\hspace*{1em}}c@{\hspace*{1em}}|c}
      \lambda_1   &       &0         \\
      \vdots      &\ddots &\vdots    \\
      0           &       &\lambda_n
   \end{pmat}
\end{equation*}
This representation is equivalent to the existence of a basis satisfying the
stated conditions simply by the definition of matrix representation.
\end{proof}

\begin{example}     \label{ex:DiagUpperTrian}
To diagonalize
\begin{equation*}
   T=\begin{pmatrix}
      3  &2  \\
      0  &1
   \end{pmatrix}
\end{equation*}
we take it as the representation of a transformation with respect to the
standard basis $T=\rep{t}{\stdbasis_2,\stdbasis_2}$ and we look for a basis
\( B=\sequence{\vec{\beta}_1,\vec{\beta}_2} \) such that
\begin{equation*}
  \rep{t}{B,B}
  =
  \begin{pmatrix}
    \lambda_1  &0          \\
    0          &\lambda_2
  \end{pmatrix}
\end{equation*}
that is, such that 
$t(\vec{\beta}_1)=\lambda_1\vec{\beta}_1$ 
and $t(\vec{\beta}_2)=\lambda_2\vec{\beta}_2$.
\begin{equation*}
  \begin{pmatrix}
     3  &2  \\
     0  &1
  \end{pmatrix}
  \vec{\beta}_1=\lambda_1\cdot\vec{\beta}_1
  \qquad
  \begin{pmatrix}
     3  &2  \\
     0  &1
  \end{pmatrix}
  \vec{\beta}_2=\lambda_2\cdot\vec{\beta}_2
\end{equation*} 
We are looking for scalars \( x \) such that this equation
\begin{equation*}
 \begin{pmatrix}
    3  &2  \\
    0  &1
 \end{pmatrix}
 \colvec{b_1 \\ b_2}=x\cdot\colvec{b_1 \\ b_2}
\end{equation*}
has solutions $b_1$ and $b_2$, which are not both zero.
Rewrite that as a linear system.
\begin{equation*}
  \begin{linsys}{2}
     (3-x)\cdot b_1  &+  &2\cdot b_2       &=  &0  \\
                     &   &(1-x)\cdot b_2   &=  &0 
  \end{linsys}
\tag*{($*$)}\end{equation*}
In the bottom equation
the two numbers multiply to give zero only if
at least one of them is zero so there are two possibilities,
$b_2=0$ and $x=1$. 
In the \( b_2=0 \) possibility, 
the first equation gives that either $b_1=0$ or \( x=3 \).
Since the case of both $b_1=0$ and $b_2=0$ is disallowed,
we are left looking at the possibility of $x=3$. 
With it, the first equation in ($*$) is $0\cdot b_1+2\cdot b_2=0$
and so associated with $3$ are vectors
with a second component of zero and a first component that is free. 
\begin{equation*}
     \begin{pmatrix}
        3  &2  \\
        0  &1
     \end{pmatrix}
     \colvec{b_1 \\ 0}=3\cdot\colvec{b_1 \\ 0} 
\end{equation*}
That is, one solution to ($*$) is $\lambda_1=3$, and we have a 
first basis vector.
\begin{equation*}
   \vec{\beta}_1=\colvec{1 \\ 0}
\end{equation*}
In the $x=1$ possibility, 
the first equation in ($*$) is $2\cdot b_1+2\cdot b_2=0$, and so
associated with $1$ are vectors whose 
second component is the negative of their first component.
\begin{equation*}
     \begin{pmatrix}
        3  &2  \\
        0  &1
     \end{pmatrix}
     \colvec{b_1 \\ -b_1}=1\cdot\colvec{b_1 \\ -b_1} 
\end{equation*}
Thus, another solution is $\lambda_2=1$ and 
a second basis vector is this.
\begin{equation*}
   \vec{\beta}_2=\colvec{1 \\ -1}
\end{equation*}
To finish, drawing the similarity diagram        
\begin{equation*}
     \begin{CD}
            \Re^2_\wrt{\stdbasis_2}                   
               @>t>T>        
               \Re^2_\wrt{\stdbasis_2}       \\
            @V{\scriptstyle\identity} VV                
               @V{\scriptstyle\identity} VV \\
            \Re^2_\wrt{B}         
               @>t>D>        
               \Re^2_\wrt{B}
      \end{CD}
\end{equation*}
and noting that the matrix $\rep{\identity}{B,\stdbasis_2}$ is easy
leads to this diagonalization.
\begin{equation*}
   \begin{pmatrix}
     3  &0  \\
     0  &1
   \end{pmatrix}
   =
   \begin{pmatrix}
     1  &1  \\
     0  &-1
   \end{pmatrix}^{-1}
   \begin{pmatrix}
     3  &2  \\
     0  &1
   \end{pmatrix}
   \begin{pmatrix}
     1  &1  \\
     0  &-1
   \end{pmatrix}
\end{equation*}
\end{example}

In the next subsection, we will expand on that example by considering 
more closely the property of \nearbycorollary{cor:DiagIffBasisOfEigens}.
This includes seeing another way, 
the way that we will routinely use, to find the $\lambda$'s.


\begin{exercises}
  \recommended \item 
    Repeat \nearbyexample{ex:DiagUpperTrian} for the matrix from 
    \nearbyexample{ex:DiagTwoByTwo}.
    \begin{answer}
      Because the basis vectors are chosen arbitrarily, many different answers
      are possible.
      However, here is one way to go;
      to diagonalize
      \begin{equation*}
         T=\begin{pmatrix}
            4  &-2  \\
            1  &1
         \end{pmatrix}
      \end{equation*}
      take it as the representation of a transformation with respect to the
      standard basis $T=\rep{t}{\stdbasis_2,\stdbasis_2}$ and look for
      \( B=\sequence{\vec{\beta}_1,\vec{\beta}_2} \) such that
      \begin{equation*}
        \rep{t}{B,B}
        =
        \begin{pmatrix}
          \lambda_1  &0          \\
          0          &\lambda_2
        \end{pmatrix}
      \end{equation*}
      that is, such that 
      $t(\vec{\beta}_1)=\lambda_1$ and $t(\vec{\beta}_2)=\lambda_2$.
      \begin{equation*}
        \begin{pmatrix}
           4  &-2  \\
           1  &1
        \end{pmatrix}
        \vec{\beta}_1=\lambda_1\cdot\vec{\beta}_1
        \qquad
        \begin{pmatrix}
           4  &-2  \\
           1  &1
        \end{pmatrix}
        \vec{\beta}_2=\lambda_2\cdot\vec{\beta}_2
      \end{equation*} 
      We are looking for scalars \( x \) such that this equation
      \begin{equation*}
       \begin{pmatrix}
          4  &-2  \\
          1  &1
       \end{pmatrix}
       \colvec{b_1 \\ b_2}=x\cdot\colvec{b_1 \\ b_2}
      \end{equation*}
      has solutions $b_1$ and $b_2$, which are not both zero.
      Rewrite that as a linear system
      \begin{equation*}
        \begin{linsys}{2}
           (4-x)\cdot b_1  &+  &-2\cdot b_2       &=  &0  \\
           1\cdot b_1      &+   &(1-x)\cdot b_2   &=  &0 
        \end{linsys}
      \end{equation*}
      If $x=4$ then the first equation gives that $b_2=0$, and then
      the second equation gives that $b_1=0$.
      The case where both $b$'s are zero is disallowed
      so we can assume that $x\neq 4$.
      \begin{equation*}
        \grstep{(-1/(4-x))\rho_1+\rho_2}\;
        \begin{linsys}{2}
           (4-x)\cdot b_1  &+   &-2\cdot b_2                   &=  &0  \\
                           &    &((x^2-5x+6)/(4-x))\cdot b_2   &=  &0 
        \end{linsys} 
      \end{equation*}
      Consider the bottom equation.
      If \( b_2=0 \) then the first equation gives $b_1=0$ or $x=4$.
      The $b_1=b_2=0$ case is disallowed.
      The other possibility for the bottom equation is that the numerator 
      of the fraction $x^2-5x+6=(x-2)(x-3)$ is zero.
      The $x=2$ case gives a first equation of $2b_1-2b_2=0$, and so 
      associated with $x=2$ we have
      vectors whose first and second components are equal:
      \begin{equation*}
         \vec{\beta}_1=\colvec{1 \\ 1}
         \qquad\text{(so \(
           \begin{pmatrix}
              4  &-2  \\
              1  &1
           \end{pmatrix}
           \colvec{1 \\ 1}=2\cdot\colvec{1 \\ 1} \), and $\lambda_1=2$).}
      \end{equation*}
      If \( x=3 \) then the first equation is
      $b_1-2b_2=0$ and so the associated vectors 
      are those whose first component is 
      twice their second:
      \begin{equation*}
         \vec{\beta}_2=\colvec{2 \\ 1}
         \qquad\text{(so \(
           \begin{pmatrix}
              4  &-2  \\
              1  &1
           \end{pmatrix}
           \colvec{2 \\ 1}=3\cdot\colvec{2 \\ 1} \), and so $\lambda_2=3$).}
      \end{equation*}
      This picture 
        \begin{equation*}
          \begin{CD}
            \Re^2_\wrt{\stdbasis_2}                   
               @>t>T>        
               \Re^2_\wrt{\stdbasis_2}       \\
            @V{\scriptstyle\identity} VV                
               @V{\scriptstyle\identity} VV \\
            \Re^2_\wrt{B}         
               @>t>D>        
               \Re^2_\wrt{B}
          \end{CD}
        \end{equation*}
      shows how to get the diagonalization.
      \begin{equation*}
         \begin{pmatrix}
           2  &0  \\
           0  &3
         \end{pmatrix}
         =
         \begin{pmatrix}
           1  &2  \\
           1  &1
         \end{pmatrix}^{-1}
         \begin{pmatrix}
           4  &-2  \\
           1  &1
         \end{pmatrix}
         \begin{pmatrix}
           1  &2  \\
           1  &1
         \end{pmatrix}
      \end{equation*}
      \textit{Comment}.
      This equation matches the $T=PSP^{-1}$ definition under this renaming.
      \begin{equation*}
        T=
         \begin{pmatrix}
           2  &0  \\
           0  &3
         \end{pmatrix}
         \quad
         P=
         \begin{pmatrix}
           1  &2  \\
           1  &1
         \end{pmatrix}^{-1}
         \quad
         P^{-1}=
         \begin{pmatrix}
           1  &2  \\
           1  &1
         \end{pmatrix}
         \quad
         S=
         \begin{pmatrix}
           4  &-2  \\
           1  &1
         \end{pmatrix}
      \end{equation*}
    \end{answer}
  \item 
    Diagonalize these upper triangular matrices.
    \begin{exparts*}
      \partsitem 
        $\begin{pmatrix}
          -2  &1  \\
           0  &2
        \end{pmatrix}$
      \partsitem 
        $\begin{pmatrix}
          5  &4  \\
          0  &1
        \end{pmatrix}$
    \end{exparts*}
    \begin{answer}
      \begin{exparts}
        \partsitem
          Setting up
          \begin{equation*}
            \begin{pmatrix}
              -2  &1  \\
               0  &2
            \end{pmatrix}
            \colvec{b_1 \\ b_2}
            =x\cdot\colvec{b_1 \\ b_2}
            \qquad\Longrightarrow\qquad 
            \begin{linsys}{2}
               (-2-x)\cdot b_1  &+  &b_2            &=  &0  \\
                                &   &(2-x)\cdot b_2 &= &0
            \end{linsys}
          \end{equation*}
          gives the two possibilities that $b_2=0$ and $x=2$.
          Following the $b_2=0$ possibility leads to the first equation
          $(-2-x)b_1=0$ with the two cases that $b_1=0$ and that
          $x=-2$.
          Thus, under this first possibility, we find $x=-2$ and the 
          associated vectors whose second component is zero, and whose
          first component is free.  
          \begin{equation*}
            \begin{pmatrix}
              -2  &1  \\
               0  &2
            \end{pmatrix}
            \colvec{b_1 \\ 0}
            =-2\cdot\colvec{b_1 \\ 0}
            \qquad
            \vec{\beta}_1=\colvec{1 \\ 0}
          \end{equation*}
          Following the other possibility leads to a first equation of
          $-4b_1+b_2=0$ and so the vectors associated with this 
          solution have a second component that is four times their first
          component.
          \begin{equation*}
            \begin{pmatrix}
              -2  &1  \\
               0  &2
            \end{pmatrix}
            \colvec{b_1 \\ 4b_1}
            =2\cdot\colvec{b_1 \\ 4b_1}
            \qquad
            \vec{\beta}_2=\colvec{1 \\ 4}
          \end{equation*}
          The diagonalization is this.
          \begin{equation*}
            \begin{pmatrix}
              1  &1  \\
              0  &4
            \end{pmatrix}
            \begin{pmatrix}
              -2  &1  \\
               0  &2
            \end{pmatrix}
            \begin{pmatrix}
              1  &1  \\
              0  &4
            \end{pmatrix}^{-1}
            =
            \begin{pmatrix}
              -2  &0  \\
               0  &2
            \end{pmatrix}
          \end{equation*}
      \partsitem The calculations are like those in the prior part.
          \begin{equation*}
            \begin{pmatrix}
               5  &4  \\
               0  &1
            \end{pmatrix}
            \colvec{b_1 \\ b_2}
            =x\cdot\colvec{b_1 \\ b_2}
            \qquad\Longrightarrow\qquad 
            \begin{linsys}{2}
               (5-x)\cdot b_1  &+  &4\cdot b_2     &=  &0  \\
                               &   &(1-x)\cdot b_2 &=  &0
            \end{linsys}
          \end{equation*}
          The bottom equation
          gives the two possibilities that $b_2=0$ and $x=1$.
          Following the $b_2=0$ possibility, and discarding the
          case where both $b_2$ and $b_1$ are zero, gives
          that $x=5$, associated with vectors whose second component
          is zero and whose first component is free.
          \begin{equation*}
            \vec{\beta}_1=\colvec{1 \\ 0}
          \end{equation*}
          The $x=1$ possibility gives a first equation of
          $4b_1+4b_2=0$ and so the associated vectors have a 
          second component that is the negative of their first component.
          \begin{equation*}
            \vec{\beta}_1=\colvec{1 \\ -1}
          \end{equation*}
          We thus have this diagonalization.
          \begin{equation*}
            \begin{pmatrix}
              1  &1  \\
              0  &-1
            \end{pmatrix}
            \begin{pmatrix}
               5  &4  \\
               0  &1
            \end{pmatrix}
            \begin{pmatrix}^{-1}
              1  &1  \\
              0  &-1
            \end{pmatrix}
            =
            \begin{pmatrix}
               5  &0  \\
               0  &1
            \end{pmatrix}
          \end{equation*}
      \end{exparts}
    \end{answer}
  \recommended \item \label{exer:PowersOfDiags}
    What form do the powers of a diagonal matrix have?
    \begin{answer}
      For any integer \( p \),
      \begin{equation*}
        \begin{pmatrix}
          d_1  &0      &   \\
          0    &\ddots &   \\
               &       &d_n
        \end{pmatrix}^p=
        \begin{pmatrix}
          d_1^p  &0      &   \\
          0      &\ddots &   \\
                 &       &d_n^p
        \end{pmatrix}.
     \end{equation*} 
    \end{answer}
  \item 
     Give two same-sized diagonal matrices that are not similar.
     Must any two different diagonal matrices come from different similarity
     classes?
     \begin{answer}
       These two are not similar 
       \begin{equation*}
          \begin{pmatrix}
             0  &0  \\
             0  &0
          \end{pmatrix}
          \qquad
          \begin{pmatrix}
             1  &0  \\
             0  &1
          \end{pmatrix}
       \end{equation*}
       because each is alone in its similarity class.

       For the second half, these
       \begin{equation*}
          \begin{pmatrix}
             2  &0  \\
             0  &3
          \end{pmatrix}
          \qquad
          \begin{pmatrix}
             3  &0  \\
             0  &2
          \end{pmatrix}
       \end{equation*}
       are similar via the matrix that changes bases from
       \( \sequence{\vec{\beta}_1,\vec{\beta}_2} \) to
       \( \sequence{\vec{\beta}_2,\vec{\beta}_1} \).
       (\textit{Question.}
        Are two diagonal matrices similar if and only if their diagonal
        entries are permutations of each other's?)  
    \end{answer}
  \item 
    Give a nonsingular diagonal matrix.
    Can a diagonal matrix ever be singular?
    \begin{answer}
      Contrast these two.
      \begin{equation*}
         \begin{pmatrix}
           2  &0  \\
           0  &1
         \end{pmatrix}
         \qquad
         \begin{pmatrix}
           2  &0  \\
           0  &0
         \end{pmatrix}
      \end{equation*}
      The first is nonsingular, the second is singular.  
     \end{answer}
  \recommended \item
    Show that the inverse of a diagonal matrix is the diagonal of the
    the inverses, if no element on that diagonal is zero.
    What happens when a diagonal entry is zero?
    \begin{answer}  
       To check that the inverse of a diagonal matrix is the diagonal
       matrix of the inverses, just multiply.
       \begin{equation*}
          \begin{pmatrix}
             a_{1,1}  &0                \\
             0        &a_{2,2}          \\
                      &       &\ddots    \\
                      &       &      &a_{n,n}
          \end{pmatrix}
          \begin{pmatrix}
            1/a_{1,1}  &0                \\
             0        &1/a_{2,2}          \\
                      &       &\ddots    \\
                      &       &      &1/a_{n,n}
          \end{pmatrix}
      \end{equation*}
      (Showing that it is a left inverse is just as easy.)

      If a diagonal entry is zero then the diagonal matrix is
      singular; it has a zero determinant.  
    \end{answer}
  \item 
    The equation ending \nearbyexample{ex:DiagUpperTrian}
    \begin{equation*}
       \begin{pmatrix}
         1  &1  \\
         0  &-1
       \end{pmatrix}^{-1}
       \begin{pmatrix}
         3  &2  \\
         0  &1
       \end{pmatrix}
       \begin{pmatrix}
         1  &1  \\
         0  &-1
       \end{pmatrix}
       =
       \begin{pmatrix}
         3  &0  \\
         0  &1
       \end{pmatrix}
    \end{equation*}
    is a bit jarring because for $P$ we must take the first matrix,
    which is shown as an inverse, and for $P^{-1}$ we take the inverse of the
    first matrix, so that the two $-1$ powers cancel and this matrix is 
    shown without a superscript $-1$.
    \begin{exparts}
      \partsitem Check that this nicer-appearing equation holds.
        \begin{equation*}
           \begin{pmatrix}
             3  &0  \\
             0  &1
           \end{pmatrix}
           =
           \begin{pmatrix}
             1  &1  \\
             0  &-1
           \end{pmatrix}
           \begin{pmatrix}
             3  &2  \\
             0  &1
           \end{pmatrix}
           \begin{pmatrix}
             1  &1  \\
             0  &-1
           \end{pmatrix}^{-1}
        \end{equation*}
      \partsitem Is the previous item a coincidence?
        Or can we always switch the $P$ and the $P^{-1}$?
   \end{exparts}
   \begin{answer}
     \begin{exparts}
       \partsitem The check is easy.
         \begin{equation*}
           \begin{pmatrix}
             1  &1  \\
             0  &-1
           \end{pmatrix}
           \begin{pmatrix}
             3  &2  \\
             0  &1
           \end{pmatrix}
           =
           \begin{pmatrix}
             3  &3  \\
             0  &-1
           \end{pmatrix}
           \qquad
           \begin{pmatrix}
             3  &3  \\
             0  &-1
           \end{pmatrix}
           \begin{pmatrix}
             1  &1  \\
             0  &-1
           \end{pmatrix}^{-1}
           =
           \begin{pmatrix}
             3  &0  \\
             0  &1
           \end{pmatrix}
         \end{equation*}
        \partsitem It is a coincidence, in the sense that if $T=PSP^{-1}$
          then $T$ need not equal $P^{-1}SP$.
          Even in the case of a diagonal matrix~$D$, the condition that
          $D=PTP^{-1}$ does not imply that $D$ equals $P^{-1}TP$.
          The matrices from \nearbyexample{ex:DiagTwoByTwo} show this.
          \begin{equation*}
            \begin{pmatrix}
              1  &2  \\
              1  &1
            \end{pmatrix}
            \begin{pmatrix}
              4  &-2  \\
              1  &1
            \end{pmatrix}
            =
            \begin{pmatrix}
              6  &0  \\
              5  &-1
            \end{pmatrix}
            \qquad
            \begin{pmatrix}
              6  &0  \\
              5  &-1
            \end{pmatrix}
            \begin{pmatrix}
              1  &2  \\
              1  &1
            \end{pmatrix}^{-1}
            =
            \begin{pmatrix}
              -6  &12  \\
              -6  &11
            \end{pmatrix}
          \end{equation*}
     \end{exparts}
   \end{answer}
  \item 
    Show that the $P$ used to diagonalize in  
    \nearbyexample{ex:DiagUpperTrian} is not unique.
    \begin{answer}
      The columns of the matrix are chosen as the vectors associated with
      the $x$'s.
      The exact choice, and the order of the choice was
      arbitrary.
      We could, for instance, get a different matrix by swapping 
      the two columns.
    \end{answer}
  \item 
    Find a formula for the powers of this matrix
    \textit{Hint}:~see \nearbyexercise{exer:PowersOfDiags}.
    \begin{equation*}
      \begin{pmatrix}
        -3  &1  \\
        -4  &2
      \end{pmatrix}
    \end{equation*}
    \begin{answer}
      Diagonalizing and then taking powers of the diagonal matrix shows that
      \begin{equation*}
        \begin{pmatrix}
          -3  &1  \\
          -4  &2
        \end{pmatrix}^k
        =
        \frac{1}{3}
        \begin{pmatrix}
          -1  &1  \\
          -4  &4
        \end{pmatrix}
        +(\frac{-2}{3})^k
        \begin{pmatrix}
           4  &-1 \\
           4  &-1
        \end{pmatrix}.
      \end{equation*}  
     \end{answer}
  \recommended \item \label{exer:DiagThese} 
    Diagonalize these.
    \begin{exparts*}
      \partsitem \( \begin{pmatrix}
                  1  &1  \\
                  0  &0
               \end{pmatrix} \)
      \partsitem \( \begin{pmatrix}
                  0  &1  \\
                  1  &0
               \end{pmatrix} \)
    \end{exparts*}
    \begin{answer}
     \begin{exparts}
        \partsitem \( \begin{pmatrix}
                    1  &1  \\
                    0  &-1
                 \end{pmatrix}^{-1}
                 \begin{pmatrix}
                    1  &1  \\
                    0  &0
                 \end{pmatrix}
                 \begin{pmatrix}
                    1  &1  \\
                    0  &-1
                 \end{pmatrix}
                =\begin{pmatrix}
                    1  &0  \\
                    0  &0
                 \end{pmatrix} \)
        \partsitem \( \begin{pmatrix}
                    1  &1  \\
                    0  &-1
                 \end{pmatrix}^{-1}
                 \begin{pmatrix}
                    0  &1  \\
                    1  &0
                 \end{pmatrix}
                 \begin{pmatrix}
                    1  &1  \\
                    0  &-1
                 \end{pmatrix}
                =\begin{pmatrix}
                    1  &0  \\
                    0  &-1
                 \end{pmatrix} \)
      \end{exparts}  
     \end{answer}
  \item 
    We can ask how diagonalization interacts with the matrix operations.
    Assume that \( \map{t,s}{V}{V} \) are each diagonalizable.
    Is \( ct \) diagonalizable for all scalars \( c \)?
    What about \( t+s \)?
    \( \composed{t}{s} \)?
    \begin{answer}
      Yes, \( ct \) is diagonalizable by the final theorem of this
      subsection.

      No, \( t+s \) need not be diagonalizable.
      Intuitively, the problem arises when the two maps diagonalize with
      respect to different bases (that is, when they are not
      \definend{simultaneously diagonalizable}).
      Specifically, these two are diagonalizable but their sum is not:
      \begin{equation*}
         \begin{pmatrix}
            1  &1  \\
            0  &0
         \end{pmatrix}
         \qquad
         \begin{pmatrix}
           -1  &0  \\
            0  &0
         \end{pmatrix}
      \end{equation*}
      (the second is already diagonal; for the first, see 
      \nearbyexercise{exer:DiagThese}).
      The sum is not diagonalizable because its square is the zero matrix. 

     The same intuition suggests that \( \composed{t}{s} \) is not
     be diagonalizable.
     These two are diagonalizable but their product is not:
     \begin{equation*}
        \begin{pmatrix}
           1  &0  \\
           0  &0
        \end{pmatrix}
        \qquad
        \begin{pmatrix}
           0  &1  \\
           1  &0
        \end{pmatrix}
     \end{equation*}
     (for the second, see \nearbyexercise{exer:DiagThese}).   
    \end{answer}
  \recommended \item 
    Show that matrices of this form are not diagonalizable.
    \begin{equation*}
       \begin{pmatrix}
          1  &c  \\
          0  &1
       \end{pmatrix}
       \qquad c\neq 0
    \end{equation*}
    \begin{answer}
      If
      \begin{equation*}
         P
         \begin{pmatrix}
            1  &c  \\
            0  &1
         \end{pmatrix}
         P^{-1}
         =
         \begin{pmatrix}
            a  &0  \\
            0  &b
         \end{pmatrix}
      \end{equation*}
      then
      \begin{equation*}
         P
         \begin{pmatrix}
            1  &c  \\
            0  &1
         \end{pmatrix}
         =
         \begin{pmatrix}
            a  &0  \\
            0  &b
         \end{pmatrix}
         P
      \end{equation*}
      so
      \begin{align*}
         \begin{pmatrix}
            p  &q  \\
            r  &s
         \end{pmatrix}
         \begin{pmatrix}
            1  &c  \\
            0  &1
         \end{pmatrix}
         &=
         \begin{pmatrix}
            a  &0  \\
            0  &b
         \end{pmatrix}
         \begin{pmatrix}
            p  &q  \\
            r  &s
         \end{pmatrix}        \\
         \begin{pmatrix}
            p  &cp+q  \\
            r  &cr+s
         \end{pmatrix}
         &=
         \begin{pmatrix}
            ap  &aq  \\
            br  &bs
         \end{pmatrix}
      \end{align*}
      The \( 1,1 \) entries show that \( a=1 \) and the \( 1,2 \) entries
      then show that \( pc=0 \).
      Since \( c\neq 0 \) this means that \( p=0 \).
      The \( 2,1 \) entries show that 
      \( b=1 \) and the \( 2,2 \) entries then show that
      \( rc=0 \).
      Since \( c\neq 0 \) this means that \( r=0 \).
      But if both \( p \) and \( r \) are \( 0 \) then \( P \) is not
      invertible.  
     \end{answer}
  \item 
    Show that each of these is diagonalizable.
    \begin{exparts*}
      \partsitem
       \( \begin{pmatrix}
             1  &2  \\
             2  &1
          \end{pmatrix}  \)
      \partsitem
       \( \begin{pmatrix}
             x  &y  \\
             y  &z
          \end{pmatrix}
          \qquad \text{$x,y,z$ scalars}  \)
    \end{exparts*}
    \begin{answer}
      \begin{exparts}
      \partsitem Using the formula for the inverse of a $\nbyn{2}$
        matrix gives this.
        \begin{align*}
           \begin{pmatrix}
              a  &b  \\
              c  &d
           \end{pmatrix}
           \begin{pmatrix}
              1  &2  \\
              2  &1
           \end{pmatrix}
           \cdot\frac{1}{ad-bc}\cdot
           \begin{pmatrix}
              d  &-b \\
             -c  &a
           \end{pmatrix}
           &=\frac{1}{ad-bc}
           \begin{pmatrix}
              ad+2bd-2ac-bc    &-ab-2b^2+2a^2+ab \\
              cd+2d^2-2c^2-cd  &-bc-2bd+2ac+ad
           \end{pmatrix}
        \end{align*}
        Now pick scalars \( a,\ldots,d \) so that
        \( ad-bc\neq 0 \) and \( 2d^2-2c^2=0 \) and \( 2a^2-2b^2=0 \).
        For example, these will do.
        \begin{equation*}
          \begin{pmatrix}
            1  &1  \\
            1  &-1
          \end{pmatrix}
          \begin{pmatrix}
            1  &2  \\
            2  &1
          \end{pmatrix}
          \cdot\frac{1}{-2}\cdot
          \begin{pmatrix}
            -1  &-1  \\
            -1  &1
          \end{pmatrix}
          =
          \frac{1}{-2}
          \begin{pmatrix}
            -6  &0   \\
             0  &2
          \end{pmatrix}
        \end{equation*}
      \partsitem As above,
        \begin{align*}
           \begin{pmatrix}
              a  &b  \\
              c  &d
           \end{pmatrix}
           \begin{pmatrix}
              x  &y  \\
              y  &z
           \end{pmatrix}
           \cdot\frac{1}{ad-bc}\cdot
           \begin{pmatrix}
              d  &-b \\
             -c  &a
           \end{pmatrix}
           &=\frac{1}{ad-bc}
           \begin{pmatrix}
              adx+bdy-acy-bcz    &-abx-b^2y+a^2y+abz \\
              cdx+d^2y-c^2y-cdz  &-bcx-bdy+acy+adz
           \end{pmatrix}
        \end{align*}
        we are looking for scalars \( a,\ldots,d \) so that
        \( ad-bc\neq 0 \) and
        \( -abx-b^2y+a^2y+abz=0 \)
        and \( cdx+d^2y-c^2y-cdz=0 \), no matter what values
        \( x \), \( y \), and \( z \) have.

        For starters, we assume that \( y\neq 0 \), else the given matrix is
        already diagonal.
        We shall use that assumption because if we (arbitrarily) let
        \( a=1 \) then we get
        \begin{align*}
           -bx-b^2y+y+bz
           &=0              \\
           (-y)b^2+(z-x)b+y
           &=0
        \end{align*}
        and the quadratic formula gives
        \begin{equation*}
           b=\frac{-(z-x)\pm\sqrt{(z-x)^2-4(-y)(y)} }{-2y}
           \qquad
           y\neq 0
        \end{equation*}
        (note that if \( x \), \( y \), and \( z \) are real then these two
         \( b \)'s are real as the discriminant is positive).
        By the same token, if we (arbitrarily) let \( c=1 \) then
        \begin{align*}
           dx+d^2y-y-dz
           &=0              \\
           (y)d^2+(x-z)d-y
           &=0
        \end{align*}
        and we get here
        \begin{equation*}
           d=\frac{-(x-z)\pm\sqrt{(x-z)^2-4(y)(-y)} }{2y}
           \qquad
           y\neq 0
        \end{equation*}
        (as above, if \( x,y,z\in\Re \) then this discriminant is positive
        so a symmetric, real, \( \nbyn{2} \) matrix is similar to a real
        diagonal matrix).

        For a check we try \( x=1 \), \( y=2 \), \( z=1 \).
        \begin{equation*}
           b=\frac{0\pm\sqrt{0+16} }{-4}=\mp 1
           \qquad
           d=\frac{0\pm\sqrt{0+16} }{4}=\pm 1
        \end{equation*}
        Note that not all four choices \( (b,d)=(+1,+1),\dots,(-1,-1) \)
        satisfy \( ad-bc\neq 0 \).
      \end{exparts} 
    \end{answer}
\index{diagonalizable|)}
\end{exercises}































\subsection{Eigenvalues and Eigenvectors}
In this subsection we will focus on the
property of \nearbycorollary{cor:DiagIffBasisOfEigens}.

\begin{definition} \label{def:Eigen}
A transformation \( \map{t}{V}{V} \) has a scalar
\definend{eigenvalue}\index{eigenvalue, eigenvector!of a transformation}%
\index{transformation!eigenvalue, eigenvector}
\( \lambda \)
if there is a nonzero \definend{eigenvector} \( \vec{\zeta}\in V \)
such that
$
  t(\vec{\zeta})=\lambda\cdot\vec{\zeta}
$.
\end{definition}

\noindent (``Eigen'' is German for ``characteristic of'' or ``peculiar to''; 
some authors call these \definend{characteristic}%
\index{characteristic!vectors, values} values and vectors.
No authors call them ``peculiar''.)

\begin{example}
The projection map
\begin{equation*}
  \colvec{x \\ y \\ z}
     \mapsunder{\pi}
  \colvec{x \\ y \\ 0}
   \qquad x,y,z\in\C
\end{equation*}
has an eigenvalue of \( 1 \) associated with any eigenvector of the form
\begin{equation*}
   \colvec{x \\ y \\ 0}
\end{equation*}
where \( x \) and \( y \) are non-\( 0 \) scalars.
On the other hand, \( 2 \) is not an eigenvalue of \( \pi \) 
since no non-\( \zero \) vector is doubled.
\end{example}

That example shows why the `non-$\zero$' appears in the definition.
Disallowing \( \zero \) as an eigenvector eliminates trivial eigenvalues.
(Note, however, that a matrix can
have an eigenvalue $\lambda=0$.)

\begin{example}  \label{ex:NoEigenOnTrivSp}
The only transformation on the trivial space \( \set{\zero\,} \) is
%\begin{equation*}
$\zero\mapsto\zero$.
%\end{equation*}
This map has no eigenvalues because there are no non-\( \zero \) vectors
$\vec{v}$ mapped to a scalar multiple $\lambda\cdot\vec{v}$ of themselves.
\end{example}        

\begin{example}
Consider the homomorphism \( \map{t}{\polyspace_1}{\polyspace_1} \)
given by \( c_0+c_1x\mapsto(c_0+c_1)+(c_0+c_1)x \).
The range of \( t \) is one-dimensional.
Thus an application of
\( t \) to a vector in the range will simply rescale that vector:
\( c+cx\mapsto (2c)+(2c)x \).
That is, \( t \) has an eigenvalue of \( 2 \) associated with eigenvectors of
the form \( c+cx \) where \( c\neq 0 \).

This map also has an eigenvalue of \( 0 \) associated with eigenvectors of
the form \( c-cx \) where \( c\neq 0 \).
\end{example}

%We can extend \nearbydefinition{def:Eigen}, which applies to transformations,
%to also cover square matrices.

\begin{definition}
A square matrix \( T \) has a scalar
\definend{eigenvalue}\index{eigenvalue, eigenvector!of a matrix}
\( \lambda \) associated with the non-\( \zero \)
\definend{eigenvector} \( \vec{\zeta} \) if
\( T\vec{\zeta}=\lambda\cdot\vec{\zeta} \).
\end{definition}

\begin{remark}
Although this extension from maps to matrices is obvious, 
there is a point that must be made.
Eigenvalues of a map are also the eigenvalues of matrices representing
that map, and so similar matrices have the same eigenvalues.
But the eigenvectors are different\Dash similar matrices need not have the 
same eigenvectors.

For instance, consider again the transformation 
\( \map{t}{\polyspace_1}{\polyspace_1} \) given by
\( c_0+c_1x\mapsto (c_0+c_1)+(c_0+c_1)x \).
It has an eigenvalue of \( 2 \) associated with eigenvectors of the form
\( c+cx \) where \( c\neq 0 \).
If we represent \( t \) with respect to \( B=\sequence{1+1x,1-1x} \)
\begin{equation*}
   T=\rep{t}{B,B}=
   \begin{pmatrix}
      2  &0  \\
      0  &0
   \end{pmatrix}
\end{equation*}
then \( 2 \) is an eigenvalue of \( T \), associated with these eigenvectors.
\begin{equation*}
   \set{\colvec{c_0 \\ c_1}\suchthat \begin{pmatrix}
                                         2  &0  \\
                                         0  &0
                                      \end{pmatrix}\colvec{c_0 \\ c_1}
                                      =\colvec{2c_0 \\ 2c_1}  }
  =\set{\colvec{c_0 \\ 0}\suchthat c_0\in\C,\, c_0\neq 0 }
\end{equation*}
On the other hand, representing $t$ with respect to
\( D=\sequence{2+1x,1+0x} \) gives 
\begin{equation*}
   S=\rep{t}{D,D}=
   \begin{pmatrix}
      3  &1  \\
     -3  &-1
   \end{pmatrix}
\end{equation*}
and the eigenvectors of \( S \) associated with the eigenvalue \( 2 \) are
these.
\begin{equation*}
   \set{\colvec{c_0 \\ c_1}\suchthat \begin{pmatrix}
                                         3  &1  \\
                                        -3  &-1
                                      \end{pmatrix}\colvec{c_0 \\ c_1}
                                      =\colvec{2c_0 \\ 2c_1}  }
  =\set{\colvec{0 \\ c_1}\suchthat c_1\in\C,\, c_1\neq 0 }
\end{equation*}
Thus similar matrices can have different eigenvectors.

Here is an informal description of what's happening.  
The underlying transformation doubles the 
eigenvectors $\vec{v}\mapsto 2\cdot\vec{v}$.
But when the matrix representing the transformation is
\( T=\rep{t}{B,B} \) then it ``assumes'' that column vectors are 
representations with respect to \( B \).
In contrast, \( S=\rep{t}{D,D} \) ``assumes'' that column vectors 
are representations with respect to \( D \).
So the vectors that get doubled by each matrix look different.
\end{remark}

The next example illustrates the basic tool for
finding eigenvectors and eigenvalues.

\begin{example}  \label{ex:IntroCharEqn}
What are the eigenvalues and eigenvectors of this matrix?
\begin{equation*}
  T=
  \begin{pmatrix}
     1    &2    &1    \\
     2    &0    &-2   \\
    -1    &2    &3
  \end{pmatrix}
\end{equation*}
To find the scalars \( x \) such that
\( T\vec{\zeta}=x\vec{\zeta} \) for non-\( \zero \) eigenvectors
\( \vec{\zeta} \), bring everything to the left-hand side
\begin{equation*}
  \begin{pmatrix}
     1    &2    &1    \\
     2    &0    &-2   \\
    -1    &2    &3
  \end{pmatrix}
  \colvec{z_1 \\ z_2 \\ z_3}
  -x\colvec{z_1 \\ z_2 \\ z_3}
  =\zero
\end{equation*}
and factor
\( (T-x I)\vec{\zeta}=\zero \).
(Note that it says $T-xI$; the expression \( T-x \) doesn't make sense 
because \( T \) is a matrix while \( x \) is a scalar.)
This homogeneous linear system
\begin{equation*}
  \begin{pmatrix}
   1-x           &2            &1            \\
     2           &0-x          &-2           \\
    -1           &2            &3-x
  \end{pmatrix}
  \colvec{z_1 \\ z_2 \\ z_3}
  =
  \colvec{0 \\ 0 \\ 0}
\end{equation*}
has a non-\( \zero \) solution if and only if the
matrix is singular.
We can determine when that happens.
\begin{align*}
  0
  &=\deter{T-x I}                                               \\
  &=\begin{vmatrix}
     1-x          &2            &1            \\
     2           &0-x          &-2           \\
    -1           &2            &3-x
   \end{vmatrix}                                       \\
  &=x^3-4x^2+4x  \\
  &=x(x-2)^2
\end{align*}
The eigenvalues are \( \lambda_1=0 \) and \( \lambda_2=2 \).
To find the associated eigenvectors, plug in each eigenvalue.
Plugging in $\lambda_1=0$ gives
\begin{equation*}
  \begin{pmatrix}
     1-0         &2            &1            \\
     2           &0-0          &-2           \\
    -1           &2            &3-0
  \end{pmatrix}
  \colvec{z_1 \\ z_2 \\ z_3}
  =
  \colvec{0 \\ 0 \\ 0}
  \qquad\Longrightarrow\qquad
  \colvec{z_1 \\ z_2 \\ z_3}
  =
  \colvec{a \\ -a \\ a}
\end{equation*}
for a scalar parameter \( a\neq 0 \)
(\( a \) is non-\( 0 \) because eigenvectors must be non-\( \zero \)).
In the same way, plugging in $\lambda_2=2$ gives 
\begin{equation*}
  \begin{pmatrix}
     1-2         &2            &1            \\
     2           &0-2          &-2           \\
    -1           &2            &3-2
  \end{pmatrix}
  \colvec{z_1 \\ z_2 \\ z_3}
  =
  \colvec{0 \\ 0 \\ 0}
  \qquad\Longrightarrow\qquad
  \colvec{z_1 \\ z_2 \\ z_3}
  =
  \colvec{b \\ 0 \\ b}
\end{equation*}
with \( b\neq 0 \).
\end{example}

\begin{example} \label{ex:AnotherCharPoly}
If
\begin{equation*}
  S=
  \begin{pmatrix}
    \pi      &1      \\
    0        &3
  \end{pmatrix}
\end{equation*}
(here \( \pi \) is not a projection map, it is the number
\( 3.14\ldots \)) then
\begin{equation*}
  \deter{
    \begin{pmatrix}
      \pi-x &1         \\
      0     &3-x
    \end{pmatrix} }
  =
  (x-\pi)(x-3)
\end{equation*}
so \( S \) has eigenvalues of \( \lambda_1=\pi \) and \( \lambda_2=3 \).
To find associated eigenvectors, first plug in $\lambda_1$ for $x$:
\begin{equation*}
  \begin{pmatrix}
    \pi-\pi     &1         \\
    0           &3-\pi
  \end{pmatrix}
  \colvec{z_1 \\ z_2}
  =
  \colvec{0 \\ 0}
  \qquad\Longrightarrow\qquad
  \colvec{z_1 \\ z_2}
  =
  \colvec{a \\ 0}
\end{equation*}
for a scalar \( a\neq 0 \),
and then plug in $\lambda_2$:
\begin{equation*}
  \begin{pmatrix}
    \pi-3       &1         \\
    0           &3-3
  \end{pmatrix}
  \colvec{z_1 \\ z_2}
  =
  \colvec{0 \\ 0}
  \qquad\Longrightarrow\qquad
  \colvec{z_1 \\ z_2}
  =
  \colvec{-b/(\pi-3) \\ b}
\end{equation*}
where \( b\neq 0 \).
\end{example}

\begin{definition}
The \definend{characteristic polynomial}\index{characteristic polynomial}%
\index{matrix!characteristic polynomial}
of a square matrix \( T \) is the
determinant of the matrix \( T-x I \), where \( x \) is a variable.
The \definend{characteristic equation}\index{characteristic equation}%
\index{matrix!characteristic polynomial}
is $\deter{T-xI}=0$.
The characteristic polynomial of a transformation \( t \) is the polynomial
of any \( \rep{t}{B,B} \).\index{transformation!characteristic polynomial}
\end{definition}

\noindent \nearbyexercise{exer:CharPolyTransWellDefed} checks that the 
characteristic polynomial of a transformation is 
well-defined, that is, any choice of basis yields the same polynomial.

\begin{lemma} \label{le:MapNonTrivSpHasEigen}
A linear transformation on a nontrivial vector space has at least one
eigenvalue.
\end{lemma}

\begin{proof}
Any root of the characteristic polynomial is an eigenvalue.
Over the complex numbers, any polynomial of degree one or greater
has a root.
(This is the reason that in this chapter 
we've gone to scalars that are complex.)
\end{proof}

Notice the familiar form of the sets of eigenvectors in the above examples.

\begin{definition}
The \definend{eigenspace}\index{eigenspace}\index{transformation!eigenspace}
of a transformation \( t \) associated with the
eigenvalue \( \lambda \) is
$
  V_\lambda=\set{\vec{\zeta}\suchthat t(\vec{\zeta}\,)=\lambda\vec{\zeta}\,}
              % \union\set{\zero\,}
$.
The eigenspace of a matrix is defined analogously.
\end{definition}

\begin{lemma}  \label{le:EigSpaceIsSubSp}
An eigenspace is a subspace.
\end{lemma}
\begin{proof}
An eigenspace must be nonempty\Dash for one thing it contains the zero
vector since a. linear transformation maps the zero vector to the 
zero vector. 
Thus we need only check closure.
Take vectors \( \vec{\zeta}_1,\ldots,\vec{\zeta}_n \) from \( V_\lambda \),
to show that any linear combination is in \( V_\lambda \)
\begin{align*}
  t(\lincombo{c}{\vec{\zeta}})
  &=c_1t(\vec{\zeta}_1)+\dots+c_nt(\vec{\zeta}_n)               \\
  &=c_1\lambda\vec{\zeta}_1+\dots+c_n\lambda\vec{\zeta}_n          \\
  &=\lambda(c_1\vec{\zeta}_1+\dots+c_n\vec{\zeta}_n)
\end{align*}
(the second equality holds even if any \( \vec{\zeta}_i \) is \( \zero \) since
\( t(\zero)=\lambda\cdot\zero=\zero \)).
\end{proof}

\begin{example}
In \nearbyexample{ex:AnotherCharPoly} the eigenspace associated with the
eigenvalue \( \pi \) 
and the eigenspace associated with the eigenvalue \( 3 \) are these.
\begin{equation*}
  V_{\pi}=\set{\colvec{a \\ 0}\suchthat a\in\Re}
  \qquad
  V_3=\set{\colvec{-b/\pi-3 \\ b}\suchthat b\in\Re}
\end{equation*}
\end{example}

\begin{example}
In \nearbyexample{ex:IntroCharEqn},
these are the eigenspaces associated with the eigenvalues \( 0 \) 
and \( 2 \).
\begin{equation*}
  V_0=\set{\colvec{a \\ -a \\ a}\suchthat a\in\Re},
  \qquad
  V_2=\set{\colvec{b \\ 0 \\ b}\suchthat b\in\Re}.
\end{equation*}
\end{example}

\begin{remark}
The characteristic equation is \( 0=x(x-2)^2 \) so in some sense
\( 2 \) is an eigenvalue ``twice''.
However there are not ``twice'' as many eigenvectors, in that the dimension
of the eigenspace is one, not two.
The next example shows a case where a number, \( 1 \), is a double root of
the characteristic equation and the dimension of the associated eigenspace
is two.
\end{remark}

\begin{example}
With respect to the standard bases, this matrix
\begin{equation*}
  \begin{pmatrix}
     1  &0  &0  \\
     0  &1  &0  \\
     0  &0  &0
  \end{pmatrix}
\end{equation*}
represents projection.
\begin{equation*}
  \colvec{x \\ y \\ z}
     \mapsunder{\pi}
  \colvec{x \\ y \\ 0}
   \qquad x,y,z\in\C
\end{equation*}
Its eigenspace associated with the eigenvalue \( 0 \) and
its eigenspace associated with the eigenvalue \( 1 \)
are easy to find.
\begin{equation*}
   V_0=\set{\colvec{0 \\ 0 \\ c_3}\suchthat c_3\in\C}
   \qquad
   V_1=\set{\colvec{c_1 \\ c_2 \\ 0}\suchthat c_1,c_2\in\C}
\end{equation*}
\end{example}

By the lemma, if two eigenvectors $\vec{v}_1$ and $\vec{v}_2$ are 
associated with the same eigenvalue then any linear combination of those
two is also an eigenvector associated with that same eigenvalue.
But, if two eigenvectors \( \vec{v}_1 \) and \( \vec{v}_2 \) 
are associated with different eigenvalues
then the sum \( \vec{v}_1+\vec{v}_2 \) need not be related 
to the eigenvalue of either one.
In fact, just the opposite.
If the eigenvalues are different then the eigenvectors are not 
linearly related.

\begin{theorem}
\label{th:DistEValueGivesLIEvecs}
For any set of distinct eigenvalues of a map or matrix, a set of associated
eigenvectors, one per eigenvalue, is linearly independent.
\end{theorem}

\begin{proof}
We will use induction on the number of eigenvalues.
If there is no eigenvalue or
only one eigenvalue then the set of associated eigenvectors is
empty or is a singleton set with a non-$\zero$ member, 
and in either case is linearly independent.

For induction, assume that the theorem
is true for any set of \( k \)
distinct eigenvalues, suppose that 
\( \lambda_1,\dots,\lambda_{k+1} \)
are distinct eigenvalues,
and let \( \vec{v}_1,\dots,\vec{v}_{k+1} \)
be associated eigenvectors.
If
\( c_1\vec{v}_1+\dots+c_k\vec{v}_k+c_{k+1}\vec{v}_{k+1}=\zero \)
then after multiplying both sides of the displayed equation by
\( \lambda_{k+1} \), applying the map or matrix to both sides
of the displayed equation,
and subtracting the first result from the second, we have this.
\begin{equation*}
  c_1(\lambda_{k+1}-\lambda_1)\vec{v}_1+\dots
  +c_k(\lambda_{k+1}-\lambda_k)\vec{v}_k
      +c_{k+1}(\lambda_{k+1}-\lambda_{k+1})\vec{v}_{k+1}=\zero
\end{equation*}
The induction hypothesis now applies:
\( c_1(\lambda_{k+1}-\lambda_1)=0,\dots,c_k(\lambda_{k+1}-\lambda_k)=0 \).
Thus, as all the eigenvalues are distinct,
\( c_1,\,\dots,\,c_k \) are all \( 0 \).
Finally, now \( c_{k+1} \) must be \( 0 \) because
we are left with the equation \( \vec{v}_{k+1}\neq\zero \).
\end{proof}

\begin{example}
The eigenvalues of
\begin{equation*}
     \begin{pmatrix}
        2   &-2   &2   \\
        0   &1    &1   \\
       -4   &8    &3
     \end{pmatrix}
\end{equation*}
are distinct: \( \lambda_1=1 \), \( \lambda_2=2 \), and~\( \lambda_3=3 \).
A set of associated eigenvectors like
\begin{equation*}
  \set{
       \colvec{2 \\ 1 \\ 0},
       \colvec{9 \\ 4 \\ 4},
       \colvec{2 \\ 1 \\ 2}  }
\end{equation*}
is linearly independent.
\end{example}

\begin{corollary}
An \( \nbyn{n} \) matrix with \( n \) distinct eigenvalues is diagonalizable.
\end{corollary}

\begin{proof}
   Form a basis of eigenvectors.
   Apply \nearbycorollary{cor:DiagIffBasisOfEigens}.
\end{proof}




\begin{exercises}
  \item 
    For each, find the characteristic polynomial and the eigenvalues.
    \begin{exparts*}
      \partsitem \( \begin{pmatrix}
                 10  &-9 \\
                  4  &-2
            \end{pmatrix}  \)
      \partsitem $\begin{pmatrix}
                    1  &2  \\
                    4  &3  
                 \end{pmatrix}$
      \partsitem \( \begin{pmatrix}
                  0  &3  \\
                  7 &0
                 \end{pmatrix} \)
      \partsitem \( \begin{pmatrix}  
                  0  &0  \\
                  0  &0
            \end{pmatrix}  \)
      \partsitem \( \begin{pmatrix}
                  1  &0  \\
                  0  &1
            \end{pmatrix}  \)
    \end{exparts*}
    \begin{answer}
      \begin{exparts}
         \partsitem This
           \begin{equation*}
             0=
             \begin{vmatrix}
               10-x  &-9  \\
               4     &-2-x
             \end{vmatrix}
             =(10-x)(-2-x)-(-36)
           \end{equation*}
           simplifies to the characteristic equation \( x^2-8x+16=0 \). 
           Because the equation factors into $(x-4)^2$ there is
           only one eigenvalue \( \lambda_1=4 \).
         \partsitem $0=(1-x)(3-x)-8=x^2-4x-5$; $\lambda_1=5$, $\lambda_2=-1$
         \partsitem \( x^2-21=0 \); 
           \( \lambda_1=\sqrt{21} \), $\lambda_2=-\sqrt{21}$
         \partsitem \( x^2=0 \); \( \lambda_1=0 \)
         \partsitem \( x^2-2x+1=0 \); \( \lambda_1=1 \)
       \end{exparts}  
     \end{answer}
  \recommended \item
    For each matrix, find the characteristic equation, and the
    eigenvalues and associated eigenvectors.
    \begin{exparts*}
      \partsitem \( \begin{pmatrix}
                  3  &0  \\
                  8  &-1
            \end{pmatrix}  \)
      \partsitem \( \begin{pmatrix}
                  3  &2  \\
                 -1  &0
            \end{pmatrix}  \)
    \end{exparts*}
    \begin{answer}
       \begin{exparts}
         \partsitem The characteristic equation is \( (3-x)(-1-x)=0 \).
           Its roots, the eigenvalues, are \( \lambda_1=3 \) and 
           \( \lambda_2=-1 \).
           For the eigenvectors we consider this equation.
           \begin{equation*}
             \begin{pmatrix}
               3-x  &0    \\
               8    &-1-x
             \end{pmatrix}
             \colvec{b_1  \\  b_2}
             =\colvec{0  \\  0}
           \end{equation*}
           For the eigenvector associated with $\lambda_1=3$,
           we consider the resulting linear system.
           \begin{equation*}
             \begin{linsys}{2}
               0\cdot b_1  &+  &0\cdot b_2  &=  &0  \\
               8\cdot b_1  &+  &-4\cdot b_2 &=  &0
             \end{linsys}
           \end{equation*}
           The eigenspace is the set of vectors whose second component is 
           twice the first component.
           \begin{equation*}
             \set{\colvec{b_2/2 \\ b_2}\suchthat b_2\in\C}
             \qquad
             \begin{pmatrix}
               3  &0  \\
               8  &-1
             \end{pmatrix}
             \colvec{b_2/2  \\ b_2}
             =3\cdot\colvec{b_2/2 \\ b_2}
           \end{equation*}
           (Here, the parameter is $b_2$ only because that is the variable that
           is free in the above system.)
           Hence, this is an eigenvector associated with the eigenvalue $3$.
           \begin{equation*}
             \colvec{1 \\ 2}
           \end{equation*}

           Finding an eigenvector associated with $\lambda_2=-1$ is similar.
           This system
           \begin{equation*}
             \begin{linsys}{2}
               4\cdot b_1  &+  &0\cdot b_2  &=  &0  \\
               8\cdot b_1  &+  &0\cdot b_2 &=  &0
             \end{linsys}
           \end{equation*}
           leads to the set of vectors whose first component is 
           zero.
           \begin{equation*}
             \set{\colvec{0 \\ b_2}\suchthat b_2\in\C}
             \qquad
             \begin{pmatrix}
               3  &0  \\
               8  &-1
             \end{pmatrix}
             \colvec{0  \\ b_2}
             =-1\cdot\colvec{0 \\ b_2}
           \end{equation*}
           And so this is an eigenvector associated with $\lambda_2$.
           \begin{equation*}
             \colvec{0 \\ 1}
           \end{equation*}
         \partsitem The characteristic equation is
           \begin{equation*}
             0=
             \begin{vmatrix}
               3-x  &2  \\
               -1   &-x               
             \end{vmatrix}
             =x^2-3x+2=(x-2)(x-1)
           \end{equation*}
           and so the eigenvalues are $\lambda_1=2$ and $\lambda_2=1$.
           To find eigenvectors, consider this system.
           \begin{equation*}
             \begin{linsys}{2}
               (3-x)\cdot b_1  &+  &2\cdot b_2  &=  &0  \\
               -1\cdot b_1     &-  &x\cdot b_2  &=  &0
             \end{linsys}
           \end{equation*}
           For $\lambda_1=2$ we get 
           \begin{equation*}
             \begin{linsys}{2}
                1\cdot b_1  &+  &2\cdot b_2  &=  &0  \\
               -1\cdot b_1  &-  &2\cdot b_2  &=  &0
             \end{linsys}
           \end{equation*}
           leading to this eigenspace and eigenvector.
           \begin{equation*}
             \set{\colvec{-2b_2 \\ b_2}
                   \suchthat b_2\in\C}
             \qquad
             \colvec{-2 \\ 1}
           \end{equation*}
           For $\lambda_2=1$ the system is 
           \begin{equation*}
             \begin{linsys}{2}
                2\cdot b_1  &+  &2\cdot b_2  &=  &0  \\
               -1\cdot b_1  &-  &1\cdot b_2  &=  &0
             \end{linsys}
           \end{equation*}
           leading to this.
           \begin{equation*}
             \set{\colvec{-b_2 \\ b_2}
                   \suchthat b_2\in\C}
             \qquad
             \colvec{-1 \\ 1}
           \end{equation*}
       \end{exparts}  
     \end{answer}
  \item
    Find the characteristic equation, and the
    eigenvalues and associated eigenvectors for this matrix.
    \textit{Hint.}
      The eigenvalues are complex.
    \begin{equation*}
      \begin{pmatrix}
         -2  &-1 \\
          5  &2
      \end{pmatrix}  
    \end{equation*}
    \begin{answer}
         The characteristic equation 
           \begin{equation*}
             0=
             \begin{vmatrix}
               -2-x  &-1  \\
               5     &2-x               
             \end{vmatrix}
             =x^2+1
           \end{equation*}
           has the complex roots $\lambda_1=i$ and $\lambda_2=-i$.
           This system 
           \begin{equation*}
             \begin{linsys}{2}
               (-2-x)\cdot b_1  &-  &1\cdot b_2      &=  &0  \\
               5\cdot b_1       &   &(2-x)\cdot b_2  &=  &0
             \end{linsys}
           \end{equation*}
           For $\lambda_1=i$ Gauss' method gives this reduction.
           \begin{equation*}
             \begin{linsys}{2}
                (-2-i)\cdot b_1  &-  &1\cdot b_2      &=  &0  \\
                5\cdot b_1       &-  &(2-i)\cdot b_2  &=  &0
             \end{linsys}
             \grstep{(-5/(-2-i))\rho_1+\rho_2}
             \begin{linsys}{2}
                (-2-i)\cdot b_1  &-  &1\cdot b_2      &=  &0  \\
                                 &   &0               &=  &0
             \end{linsys}
           \end{equation*}
           (For the calculation in the lower right get a common
           denominator
           \begin{equation*}
             \frac{5}{-2-i}-(2-i)
             =
             \frac{5}{-2-i}-\frac{-2-i}{-2-i}\cdot (2-i)
             =
             \frac{5-(-5)}{-2-i}
           \end{equation*}
           to see that it gives a $0=0$ equation.)
           These are the resulting eigenspace and  eigenvector.
           \begin{equation*}
             \set{\colvec{(1/(-2-i))b_2 \\ b_2}
                   \suchthat b_2\in\C}
             \qquad
             \colvec{1/(-2-i) \\ 1}
           \end{equation*}
           For $\lambda_2=-i$ the system 
           \begin{equation*}
             \begin{linsys}{2}
                (-2+i)\cdot b_1  &-  &1\cdot b_2      &=  &0  \\
                5\cdot b_1       &-  &(2+i)\cdot b_2  &=  &0
             \end{linsys}
             \grstep{(-5/(-2+i))\rho_1+\rho_2}
             \begin{linsys}{2}
                (-2+i)\cdot b_1  &-  &1\cdot b_2      &=  &0  \\
                                 &   &0               &=  &0
             \end{linsys}
           \end{equation*}
           leads to this.
           \begin{equation*}
             \set{\colvec{(1/(-2+i))b_2 \\ b_2}
                   \suchthat b_2\in\C}
             \qquad
             \colvec{1/(-2+i) \\ 1}
           \end{equation*}
    \end{answer}
  \item  
    Find the characteristic polynomial, the eigenvalues, and the associated
    eigenvectors of this matrix.
    \begin{equation*}
      \begin{pmatrix}
        1  &1  &1  \\
        0  &0  &1  \\
        0  &0  &1
      \end{pmatrix}
    \end{equation*}
    \begin{answer}
      The characteristic equation is
      \begin{equation*}
        0=
        \begin{vmatrix}
          1-x  &1   &1   \\
          0    &-x  &1   \\
          0    &0   &1-x
        \end{vmatrix}
        =(1-x)^2(-x)
      \end{equation*}
      and so the eigenvalues are $\lambda_1=1$ (this is a repeated root
      of the equation) and $\lambda_2=0$.
      For the rest, consider this system.
      \begin{equation*}
        \begin{linsys}{3}
          (1-x)\cdot b_1  &+  &b_2         &+  &b_3            &=  &0  \\
                          &   &-x\cdot b_2 &+  &b_3            &=  &0  \\
                          &   &            &   &(1-x)\cdot b_3 &= &0  
        \end{linsys}
      \end{equation*}
      When $x=\lambda_1=1$ then the solution set is this eigenspace.
      \begin{equation*}
        \set{\colvec{b_1 \\ 0 \\ 0}\suchthat b_1\in\C}
      \end{equation*}
      When $x=\lambda_2=0$ then the solution set is this eigenspace.
      \begin{equation*}
        \set{\colvec{-b_2 \\ b_2 \\ 0}\suchthat b_2\in\C}
      \end{equation*}
      So these are eigenvectors associated with $\lambda_1=1$ and 
      $\lambda_2=0$.
      \begin{equation*}
        \colvec{1 \\ 0 \\ 0}
        \qquad
        \colvec{-1 \\ 1 \\ 0}  
      \end{equation*}
    \end{answer}
  \recommended \item
    For each matrix, find the characteristic equation, and the
    eigenvalues and associated eigenvectors.
    \begin{exparts*}
      \partsitem \( \begin{pmatrix}
              3  &-2 &0  \\
             -2  &3  &0  \\
              0  &0  &5
            \end{pmatrix}  \)
      \partsitem \( \begin{pmatrix}
              0  &1   &0  \\
              0  &0   &1  \\
              4  &-17 &8
            \end{pmatrix}  \)
    \end{exparts*}
    \begin{answer}
      \begin{exparts}
         \partsitem The characteristic equation is
           \begin{equation*}
             0=
             \begin{vmatrix}
              3-x  &-2   &0  \\
             -2    &3-x  &0  \\
              0    &0    &5-x
             \end{vmatrix}
             =x^3-11x^2+35x-25=(x-1)(x-5)^2
           \end{equation*}
           and so the eigenvalues are $\lambda_1=1$ and also the
           repeated eigenvalue $\lambda_2=5$.
           To find eigenvectors, consider this system.
           \begin{equation*}
             \begin{linsys}{3}
               (3-x)\cdot b_1  &-  &2\cdot b_2      &   &   &=  &0  \\
               -2\cdot b_1     &+  &(3-x)\cdot b_2  &   &   &=  &0  \\
                               &   &                &   &(5-x)\cdot b_3 &= &0 
             \end{linsys}
           \end{equation*}
           For $\lambda_1=1$ we get 
           \begin{equation*}
             \begin{linsys}{3}
                2\cdot b_1     &-  &2\cdot b_2   &   &   &=  &0  \\
               -2\cdot b_1     &+  &2\cdot b_2   &   &   &=  &0  \\
                               &   &             &   &4\cdot b_3 &= &0 
             \end{linsys}
           \end{equation*}
           leading to this eigenspace and eigenvector.
           \begin{equation*}
             \set{\colvec{b_2 \\ b_2 \\ 0}
                   \suchthat b_2\in\C}
             \qquad
             \colvec{1 \\ 1 \\ 0}
           \end{equation*}
           For $\lambda_2=5$ the system is 
           \begin{equation*}
             \begin{linsys}{3}
                -2\cdot b_1  &-  &2\cdot b_2   &   &   &=  &0  \\
               -2\cdot b_1   &-  &2\cdot b_2   &   &   &=  &0  \\
                             &   &             &   &0\cdot b_3 &= &0 
             \end{linsys}
           \end{equation*}
           leading to this.
           \begin{equation*}
             \set{\colvec{-b_2 \\ b_2 \\ 0}+\colvec{0 \\ 0 \\ b_3}
                   \suchthat b_2,b_3\in\C}
             \qquad
             \colvec{-1 \\ 1 \\ 0},\,\colvec{0 \\ 0 \\ 1}
           \end{equation*}
         \partsitem The characteristic equation is
           \begin{equation*}
             0=
             \begin{vmatrix}
              -x   &1    &0  \\
              0    &-x   &1  \\
              4    &-17  &8-x
             \end{vmatrix}
             =-x^3+8x^2-17x+4=-1\cdot(x-4)(x^2-4x+1)
           \end{equation*}
           and the eigenvalues are $\lambda_1=4$ and (by using the
           quadratic equation) $\lambda_2=2+\sqrt{3}$ and 
           $\lambda_3=2-\sqrt{3}$.
           To find eigenvectors, consider this system.
           \begin{equation*}
             \begin{linsys}{3}
               -x\cdot b_1  &+  &b_2          &   &               &=  &0  \\
                            &   &-x\cdot b_2  &+  &b_3            &=  &0  \\
               4\cdot b_1   &-  &17\cdot b_2  &+  &(8-x)\cdot b_3 &= &0 
             \end{linsys}
           \end{equation*}
           Substituting $x=\lambda_1=4$ gives the system 
           \begin{equation*}
             \begin{linsys}{3}
               -4\cdot b_1  &+  &b_2          &   &           &= &0  \\
                            &   &-4\cdot b_2  &+  &b_3        &= &0  \\
               4\cdot b_1   &-  &17\cdot b_2  &+  &4\cdot b_3 &= &0 
             \end{linsys}                                              
             \grstep{\rho_1+\rho_3}
             \begin{linsys}{3}
               -4\cdot b_1  &+  &b_2          &   &           &= &0  \\
                            &   &-4\cdot b_2  &+  &b_3        &= &0  \\
                            &   &-16\cdot b_2  &+ &4\cdot b_3 &= &0 
             \end{linsys}                                              
             \grstep{-4\rho_2+\rho_3}
             \begin{linsys}{3}
               -4\cdot b_1  &+  &b_2          &   &           &= &0  \\
                            &   &-4\cdot b_2  &+  &b_3        &= &0  \\
                            &   &              &  &0          &= &0 
             \end{linsys}
           \end{equation*}
           leading to this eigenspace and eigenvector.
           \begin{equation*}
             V_4=\set{\colvec{(1/16)\cdot b_3 \\ (1/4)\cdot b_3 \\ b_3}
                   \suchthat b_2\in\C}
             \qquad
             \colvec{1 \\ 4 \\ 16}
           \end{equation*}

           Substituting $x=\lambda_2=2+\sqrt{3}$ gives the system 
           \begin{multline*}
             \begin{linsys}{3}
               (-2-\sqrt{3})\cdot b_1  
                      &+  &b_2          
                          &   &           &= &0  \\
                      &   &(-2-\sqrt{3})\cdot b_2  
                          &+  &b_3        &= &0  \\
               4\cdot b_1   
                      &-  &17\cdot b_2  
                          &+  &(6-\sqrt{3})\cdot b_3 &= &0 
             \end{linsys}                                   \\
             \grstep{(-4/(-2-\sqrt{3}))\rho_1+\rho_3}
             \begin{linsys}{3}
               (-2-\sqrt{3})\cdot b_1  
                      &+  &b_2          
                          &   &           &= &0  \\
                      &   &(-2-\sqrt{3})\cdot b_2  
                          &+  &b_3        &= &0  \\
                      &+  &(-9-4\sqrt{3})\cdot b_2  
                          &+  &(6-\sqrt{3})\cdot b_3 &= &0 
             \end{linsys}
           \end{multline*}
           (the middle coefficient in the third equation equals
           the number $(-4/(-2-\sqrt{3}))-17$; find a common denominator
           of $-2-\sqrt{3}$ and then rationalize the denominator by
           multiplying the top and bottom of the frsction by $-2+\sqrt{3}$)
           \begin{equation*}
             \grstep{((9+4\sqrt{3})/(-2-\sqrt{3}))\rho_2+\rho_3}
             \begin{linsys}{3}
               (-2-\sqrt{3})\cdot b_1  
                      &+  &b_2          
                          &   &           &= &0  \\
                      &   &(-2-\sqrt{3})\cdot b_2  
                          &+  &b_3        &= &0  \\
                      &   &                         
                          &   &0                     &= &0 
             \end{linsys}
           \end{equation*}
           which leads to this eigenspace and eigenvector.
           \begin{equation*}
             V_{2+\sqrt{3}}
             =\set{\colvec{(1/(2+\sqrt{3})^2)\cdot b_3  \\ 
                           (1/(2+\sqrt{3}))\cdot b_3    \\ 
                           b_3}
                    \suchthat b_3\in\C}
             \qquad
             \colvec{(1/(2+\sqrt{3})^2)  \\ 
                           (1/(2+\sqrt{3}))  \\ 
                           1}
           \end{equation*}

           Finally, substituting $x=\lambda_3=2-\sqrt{3}$ gives the system 
           \begin{multline*}
             \begin{linsys}{3}
               (-2+\sqrt{3})\cdot b_1  
                      &+  &b_2          
                          &   &           &= &0  \\
                      &   &(-2+\sqrt{3})\cdot b_2  
                          &+  &b_3        &= &0  \\
               4\cdot b_1   
                      &-  &17\cdot b_2  
                          &+  &(6+\sqrt{3})\cdot b_3 &= &0 
             \end{linsys}                                       \\
             \begin{aligned}
               &\grstep{(-4/(-2+\sqrt{3}))\rho_1+\rho_3}
               \begin{linsys}{3}
                 (-2+\sqrt{3})\cdot b_1  
                        &+  &b_2          
                            &   &           &= &0  \\
                        &   &(-2+\sqrt{3})\cdot b_2  
                            &+  &b_3        &= &0  \\
                        &   &(-9+4\sqrt{3})\cdot b_2  
                            &+  &(6+\sqrt{3})\cdot b_3 &= &0 
               \end{linsys}                                       \\
               &\grstep{((9-4\sqrt{3})/(-2+\sqrt{3}))\rho_2+\rho_3}
               \begin{linsys}{3}
                 (-2+\sqrt{3})\cdot b_1  
                        &+  &b_2          
                            &   &           &= &0  \\
                        &   &(-2+\sqrt{3})\cdot b_2  
                            &+  &b_3        &= &0  \\
                        &   &                         
                            &   &0                     &= &0 
               \end{linsys}
             \end{aligned}
           \end{multline*}
           which gives this eigenspace and eigenvector.
           \begin{equation*}
             V_{2-\sqrt{3}}
             =\set{\colvec{(1/(2+\sqrt{3})^2)\cdot b_3  \\ 
                           (1/(2-\sqrt{3}))\cdot b_3    \\ 
                           b_3}
                    \suchthat b_3\in\C}
             \qquad
             \colvec{(1/(-2+\sqrt{3})^2)  \\ 
                           (1/(-2+\sqrt{3}))  \\ 
                           1}
           \end{equation*}
      \end{exparts}
    \end{answer}
   \recommended \item
     Let \( \map{t}{\polyspace_2}{\polyspace_2} \) be
     \begin{equation*}
      a_0+a_1x+a_2x^2\mapsto
      (5a_0+6a_1+2a_2)-(a_1+8a_2)x+(a_0-2a_2)x^2.
     \end{equation*}
    Find its eigenvalues and the associated eigenvectors.
    \begin{answer}
      With respect to the natural basis $B=\sequence{1,x,x^2}$ 
      the matrix representation is this.
      \begin{equation*}
        \rep{t}{B,B}
        =
        \begin{pmatrix}
          5  &6  &2  \\
          0  &-1 &-8 \\
          1  &0  &-2 
        \end{pmatrix}
      \end{equation*}
      Thus the characteristic equation 
      \begin{equation*}
        0
        =
        \begin{pmatrix}
          5-x  &6    &2  \\
          0    &-1-x &-8 \\
          1    &0    &-2-x 
        \end{pmatrix}
        =(5-x)(-1-x)(-2-x)-48-2\cdot(-1-x)
      \end{equation*}
      is $0=-x^3+2x^2+15x-36=-1\cdot (x+4)(x-3)^2$.
      To find the associated eigenvectors, consider this system.
      \begin{equation*}
        \begin{linsys}{3}
          (5-x)\cdot b_1 &+ &6\cdot b_2      &+ &2\cdot b_3      &= &0 \\
                         &  &(-1-x)\cdot b_2 &- &8\cdot b_3      &= &0 \\
          b_1            &  &                &+ &(-2-x)\cdot b_3 &= &0 
        \end{linsys}
      \end{equation*}
      Plugging in $x=\lambda_1=-4$ gives
      \begin{equation*}
        \begin{linsys}{3}
                    9b_1 &+ &6\cdot b_2      &+ &2\cdot b_3      &= &0 \\
                         &  &3\cdot b_2      &- &8\cdot b_3      &= &0 \\
                     b_1 &  &                &+ & 2\cdot b_3     &= &0 
        \end{linsys}
        \grstep{-(1/9)\rho_1+\rho_3}\quad
        \grstep{(2/9)\rho_2+\rho_3}\quad
        \begin{linsys}{3}
                     9b_1 &+ &6\cdot b_2      &+ &2\cdot b_3      &= &0 \\
                         &  &3\cdot b_2       &- &8\cdot b_3      &= &0  
        \end{linsys}
      \end{equation*}
      The eigenspace and eigenvector are this.
           \begin{equation*}
             V_{-4}
             =\set{\colvec{(14/9)\cdot b_3  \\ 
                           (-8/3)\cdot b_3    \\ 
                           b_3}
                    \suchthat b_3\in\C}
             \qquad
             \colvec{14/9  \\ 
                     -8/3  \\ 
                       1}
           \end{equation*}

      Similarly, plugging in $x=\lambda_2=3$ gives
      \begin{equation*}
        \begin{linsys}{3}
                    2b_1 &+ &6\cdot b_2      &+ &2\cdot b_3      &= &0 \\
                         &  &-4\cdot b_2      &- &8\cdot b_3      &= &0 \\
                     b_1 &  &                &- & 5\cdot b_3     &= &0 
        \end{linsys}
        \grstep{-(1/2)\rho_1+\rho_3}\quad
        \grstep{-(3/4)\rho_2+\rho_3}\quad
        \begin{linsys}{3}
                     2b_1 &+ &6\cdot b_2      &+ &2\cdot b_3      &= &0 \\
                         &  &-4\cdot b_2       &- &8\cdot b_3      &= &0  
        \end{linsys}
      \end{equation*}
      with this eigenspace and eigenvector.
           \begin{equation*}
             V_{3}
             =\set{\colvec{5\cdot b_3  \\ 
                           -2\cdot b_3    \\ 
                           b_3}
                    \suchthat b_3\in\C}
             \qquad
             \colvec{5  \\ 
                     -2  \\ 
                       1}
           \end{equation*}
    \end{answer}
   \item 
     Find the eigenvalues and eigenvectors of this
     map \( \map{t}{\matspace_2}{\matspace_2} \).
     \begin{equation*}
         \begin{pmatrix}
            a  &b  \\
            c  &d
         \end{pmatrix}
       \mapsto
         \begin{pmatrix}
           2c    &a+c  \\
           b-2c  &d
         \end{pmatrix}
     \end{equation*}
     \begin{answer}
        $\lambda=1,
          \begin{pmatrix}
                   0  &0  \\
                   0  &1
          \end{pmatrix} \text{ and }
          \begin{pmatrix}
                   2  &3  \\
                   1  &0
          \end{pmatrix}$,           
          $\lambda=-2,
          \begin{pmatrix}
                  -1  &0  \\
                   1  &0
          \end{pmatrix}$,
          $\lambda=-1,
          \begin{pmatrix}
                  -2  &1  \\
                   1  &0
          \end{pmatrix}$  
       \end{answer}
   \recommended \item 
     Find the eigenvalues and associated eigenvectors of the
     differentiation operator
     \( \map{d/dx}{\polyspace_3}{\polyspace_3} \).
     \begin{answer}
       Fix the natural basis $B=\sequence{1,x,x^2,x^3}$. 
       The map's action is $1\mapsto 0$, $x\mapsto 1$, $x^2\mapsto 2x$,
       and $x^3\mapsto 3x^2$ and its representation is easy to compute.
       \begin{equation*}
         T=\rep{d/dx}{B,B}=
         \begin{pmatrix}
           0  &1  &0  &0  \\
           0  &0  &2  &0  \\
           0  &0  &0  &3  \\
           0  &0  &0  &0
         \end{pmatrix}_{B,B}
       \end{equation*}
       We find the eigenvalues with this computation.
       \begin{equation*}
         0=\deter{T-xI}=
         \begin{vmatrix}
           -x &1  &0  &0  \\
           0  &-x &2  &0  \\
           0  &0  &-x &3  \\
           0  &0  &0  &-x          
         \end{vmatrix}
         =x^4
       \end{equation*}
       Thus the map has the single eigenvalue $\lambda=0$.
       To find the associated eigenvectors, we solve
       \begin{equation*}
         \begin{pmatrix}
           0  &1  &0  &0  \\
           0  &0  &2  &0  \\
           0  &0  &0  &3  \\
           0  &0  &0  &0
         \end{pmatrix}_{B,B}
         \colvec{b_1 \\ b_2 \\ b_3 \\ b_4}_B
         =0\cdot\colvec{b_1 \\ b_2 \\ b_3 \\ b_4}_B
         \qquad\Longrightarrow\qquad
         \text{$b_2=0$, $b_3=0$, $b_4=0$}          
       \end{equation*}
       to get this eigenspace.
       \begin{equation*}
         \set{\colvec{b_1 \\ 0 \\ 0 \\ 0}_B
               \suchthat b_1\in\C}
         =\set{b_1+0\cdot x+0\cdot x^2+0\cdot x^3
               \suchthat b_1\in\C}
         =\set{b_1
               \suchthat b_1\in\C}
       \end{equation*}
      \end{answer}
   \item Prove that 
     the eigenvalues of a triangular matrix  (upper or lower triangular)
     are the entries on the diagonal.
     \begin{answer}
       The determinant of the triangular matrix $T-xI$ is the product 
       down the diagonal, and so it factors into the product of 
       the terms $t_{i,i}-x$.
     \end{answer}
  \recommended \item 
    Find the formula for the characteristic polynomial of a $\nbyn{2}$
    matrix.
    \begin{answer}
      Just expand the determinant of $T-xI$.
      \begin{equation*}
        \begin{vmatrix}
          a-x  &c  \\
          b    &d-x
        \end{vmatrix}
        =(a-x)(d-x)-bc
        =x^2+(-a-d)\cdot x +(ad-bc)
      \end{equation*}
    \end{answer}
  \item \label{exer:CharPolyTransWellDefed}
    Prove that 
    the characteristic polynomial of a transformation is well-defined.
    \begin{answer}
      Any two representations of that transformation are similar, and
      similar matrices have the same characteristic polynomial.  
    \end{answer}
  \item Prove or disprove: if all the eigenvalues of a matrix are $0$ 
    then it must be the zero matrix.
    \begin{answer}
      It is not true.
      All of the eigenvalues of this matrix are $0$.
      \begin{equation*}
        \begin{pmatrix}
          0  &1  \\
          0  &0
        \end{pmatrix}
      \end{equation*}
    \end{answer}
  \recommended \item 
    \begin{exparts}
      \partsitem Show that any non-\( \zero \) vector in any nontrivial 
        vector space can be a eigenvector.
        That is, given a \( \vec{v}\neq\zero \) from a nontrivial \( V \),
        show that there is a transformation \( \map{t}{V}{V} \) having a scalar
        eigenvalue \( \lambda\in\Re \) such that \( \vec{v}\in V_\lambda \).
      \partsitem What if we are given a  a scalar \( \lambda \)?
        Can any non-\( \zero \) member of any
        nontrivial vector space be an eigenvector associated with \( \lambda \)?
    \end{exparts}
    \begin{answer}
      \begin{exparts}
        \partsitem Use \( \lambda=1 \) and the identity map.
        \partsitem Yes, use the transformation that multiplies all 
          vectors by the scalar \( \lambda \).
      \end{exparts}  
     \end{answer}
  \recommended \item 
    Suppose that \( \map{t}{V}{V} \) and \( T=\rep{t}{B,B} \).
    Prove that the eigenvectors of \( T \) associated with \( \lambda \) are
    the non-\( \zero \) vectors in the kernel of the map represented
    (with respect to the same bases) by \( T-\lambda I \).
    \begin{answer}
      If $t(\vec{v})=\lambda\cdot\vec{v}$ then 
      $\vec{v}\mapsto\zero$ under the map $t-\lambda\cdot\identity$.
    \end{answer}
  \item 
    Prove that if $a,\ldots,\,d$ are all integers and \( a+b=c+d \) then
    \begin{equation*}
      \begin{pmatrix}
         a  &b  \\
         c  &d
      \end{pmatrix}
    \end{equation*}
    has integral eigenvalues, namely \( a+b \) and \( a-c \).
    \begin{answer}
      The characteristic equation 
      \begin{equation*}
        0=
        \begin{vmatrix}
          a-x  &b  \\
          c    &d-x 
        \end{vmatrix}
        =(a-x)(d-x)-bc
      \end{equation*}
      simplifies to $x^2+(-a-d)\cdot x + (ad-bc)$.
      Checking that the values $x=a+b$ and $x=a-c$ satisfy the equation 
      (under the $a+b=c+d$ condition) is routine. 
    \end{answer}
  \recommended \item
    Prove that if \( T \) is nonsingular and has eigenvalues
    \( \lambda_1,\dots,\lambda_n \) then \( T^{-1} \) has eigenvalues
    \( 1/\lambda_1,\dots,1/\lambda_n \).
    Is the converse true?
    \begin{answer}
      Consider an eigenspace $V_{\lambda}$.
      Any $\vec{w}\in V_{\lambda}$ is the image
      $\vec{w}=\lambda\cdot\vec{v}$ of some $\vec{v}\in V_{\lambda}$ 
      (namely, $\vec{v}=(1/\lambda)\cdot\vec{w}$).
      Thus, on $V_{\lambda}$ (which is a nontrivial subspace) 
      the action of $t^{-1}$ is 
      $t^{-1}(\vec{w})=\vec{v}=(1/\lambda)\cdot\vec{w}$,
      and so $1/\lambda$ is an eigenvalue of $t^{-1}$.
    \end{answer}
  \recommended \item
    Suppose that \( T \) is \( \nbyn{n} \) and \( c,d \) are scalars.
    \begin{exparts}
      \partsitem Prove that if \( T \) has the eigenvalue 
        \( \lambda \) with an associated
        eigenvector \( \vec{v} \) then \( \vec{v} \) is an eigenvector of
        \( cT+dI \) associated with eigenvalue \( c\lambda+d \).
      \partsitem Prove that if \( T \) is diagonalizable then so is
        \( cT+dI \).
    \end{exparts}
    \begin{answer}
      \begin{exparts}
        \partsitem We have 
          $(cT+dI)\vec{v}=cT\vec{v}+dI\vec{v}=c\lambda\vec{v}+d\vec{v}
               =(c\lambda+d)\cdot \vec{v}$.
        \partsitem Suppose that $S=PTP^{-1}$ is diagonal.
          Then $P(cT+dI)P^{-1}=P(cT)P^{-1}+P(dI)P^{-1}
                 =cPTP^{-1}+dI=cS+dI$ is also diagonal.
      \end{exparts}
    \end{answer}
  \recommended \item
    Show that \( \lambda \) is an eigenvalue of \( T \) if and only if the map
    represented by \( T-\lambda I \) is not an isomorphism.
    \begin{answer}
      The scalar $\lambda$ is an eigenvalue if and only if the transformation
      $t-\lambda \identity$ is singular.
      A transformation is singular if and only if it is not an isomorphism
      (that is, a transformation is an isomorphism if and only if it is
      nonsingular).
    \end{answer}
  \item \cite{Strang} 
    \begin{exparts}
      \partsitem Show that if \( \lambda \) is an eigenvalue of \( A \)
         then \( \lambda^k \) is an eigenvalue of \( A^k \).
      \partsitem What is wrong with this proof generalizing that?
         ``If \( \lambda \) is an eigenvalue of \( A \) and \( \mu \) is
         an eigenvalue for \( B \), then \( \lambda\mu \) is an eigenvalue
         for \( AB \), for, if \( A\vec{x}=\lambda\vec{x} \) and
         \( B\vec{x}=\mu\vec{x} \) then
         \( AB\vec{x}=A\mu\vec{x}=\mu A\vec{x}=\mu\lambda\vec{x} \)''?
    \end{exparts}
    \begin{answer}
      \begin{exparts}
        \partsitem Where the eigenvalue $\lambda$ is associated with the
          eigenvector $\vec{x}$ then
          $A^k\vec{x}=A\cdots A\vec{x}=A^{k-1}\lambda\vec{x}
            =\lambda A^{k-1}\vec{x}=\cdots=\lambda^k\vec{x}$.
          (The full details can be put in by doing induction on $k$.)
        \partsitem The eigenvector associated wih $\lambda$
          might not be an eigenvector associated with $\mu$.
      \end{exparts}
    \end{answer}
  \item 
    Do matrix-equivalent matrices have the same eigenvalues?
    \begin{answer}
      No.
      These are two same-sized, equal rank, matrices
      with different eigenvalues.
      \begin{equation*}
        \begin{pmatrix}
          1  &0  \\
          0  &1
        \end{pmatrix}
        \qquad
        \begin{pmatrix}
          1  &0  \\
          0  &2
        \end{pmatrix}
      \end{equation*}
    \end{answer}
  \item 
    Show that a square matrix with real entries and an odd number of rows
    has at least one real eigenvalue.
    \begin{answer}
      The characteristic polynomial has an odd power and so 
      has at least one real root.  
    \end{answer}
  \item 
    Diagonalize.
    \begin{equation*}
       \begin{pmatrix}
         -1  &2  &2  \\
          2  &2  &2  \\
         -3  &-6 &-6
       \end{pmatrix}
    \end{equation*}
    \begin{answer}
      The characteristic polynomial $x^3-5x^2+6x$ has distinct roots
      \( \lambda_1=0 \), \( \lambda_2=-2 \), and \( \lambda_3=-3 \).
      Thus the matrix can be diagonalized into this form.
      \begin{equation*}
         \begin{pmatrix}
            0  &0  &0  \\
            0  &-2 &0  \\
            0  &0  &-3
         \end{pmatrix}
      \end{equation*}    
    \end{answer}
  \item 
    Suppose that \( P \) is a nonsingular \( \nbyn{n} \) matrix.
    Show that 
    the \definend{similarity transformation}\index{similarity transformation}
    map \( \map{t_P}{\matspace_{\nbyn{n}}}{\matspace_{\nbyn{n}}} \)
    sending \( T\mapsto PTP^{-1} \)
    is an isomorphism.
    \begin{answer}
      We must show that it is one-to-one and onto, and that it respects the
      operations of matrix addition and scalar multiplication.

      To show that it is one-to-one, suppose that $t_P(T)=t_P(S)$,
      that is, suppose that $PTP^{-1}=PSP^{-1}$, and note that multiplying
      both sides on the left by $P^{-1}$ and on the right by $P$ gives that
      $T=S$.
      To show that it is onto, consider $S\in\matspace_{\nbyn{n}}$ and observe
      that $S=t_P(P^{-1}SP)$.

      The map $t_P$ preserves matrix addition since
      $t_P(T+S)=P(T+S)P^{-1}=(PT+PS)P^{-1}=PTP^{-1}+PSP^{-1}=t_P(T+S)$
      follows from properties of matrix multiplication and addition that 
      we have seen.
      Scalar multiplication is 
      similar:~$t_P(cT)=P(c\cdot T)P^{-1}=c\cdot (PTP^{-1})=c\cdot t_P(T)$.
    \end{answer}
  \puzzle \item 
    Show that if \( A \) is an \( n \) square matrix and each row (column)
    sums to \( c \) then \( c \) is a characteristic root of \( A \).
    \cite{MathMag67p232}
    \begin{answer}
      \answerasgiven %
      If the argument of the characteristic function of \( A \) is set equal
      to \( c \), adding the first \( (n-1) \) rows (columns) to the
      \( n \)th row (column) yields a determinant whose \( n \)th row
      (column) is zero.
      Thus \( c \) is a characteristic root of \( A \).  
    \end{answer}
\index{similarity|)}
\end{exercises}
