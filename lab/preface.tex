\chapter*{Preface}\pagestyle{preface}\thispagestyle{preface}
\setlength{\parskip}{.25ex}

\textit{WARNING! This is an incomplete draft,
no doubt riddled with errors.}
\smallskip

This collection supplements the text \nocite{Hefferon12}
\textit{Linear Algebra}\footnote{The text's home page 
\protect\url{http://joshua.smcvt.edu/linearalgebra} 
has the PDF, the ancillary materials, and the \protect\LaTeX{} source.}
with explorations to help students
solidify and extend their understanding of the subject, 
using the mathematical software \Sage{}.\footnote{See 
\url{http://www.sagemath.org} for the software and documentation.}

A major goal of any undergraduate Mathematics program is to move students 
toward a higher-level, more abstract, grasp of the subject.
For instance, Calculus classes work on elaborate computations
while later courses spend more effort on concepts and proofs, focussing
less on the details of calculations.  

The text \textit{Linear Algebra} fits into
this development process.
Its presentation works to bring students to a deeper understanding, 
but it does so by expecting
that for them at this point a good bit of calculation helps the process. 
Naturally it uses examples and practice problems
that are small-sized and have manageable numbers:~an 
assignment to by-hand multiply a pair of three by three matrices
of small integers will build intuition, whereas asking students to do that same 
question with twenty by twenty matrices
of ten decimal place numbers would be badgering. 
% (Even more worrying, having students  
% focus their intellectual energy on calculations instead of
% ideas and proofs misleads them as to
% what the subject is about.)

However, an instructor can be concerned that this misses a chance 
to make the point that Linear Algebra is widely applied,
or to develop students's understanding through explorations that are not 
hindered by the mechanics of paper and pencil. 
Mathematical software can mitigate these concerns by extending the reach of
what is reasonable 
to bigger systems, harder numbers, and longer computations.
% This manual extends students's ability to do problems in that way.
For instance, an advantage of learning how to handle larger jobs is that 
they are more like the ones that appear when students apply Linear 
Algebra to other areas.
Another advantage is that students see new ideas such as 
runtime growth measures.

Well then, why 
not teach straight from the computer?

Our goal is to 
develop a higher-level understanding of the material so we want to  
keep the focus on vector spaces and linear maps.
Our exposition takes computation to be
a tool to develop that understanding, not the main object.

Some instructors will find 
that their students are best served by
keeping a tight focus on the
core material, and leaving aside altogether 
the work in this manual. 
Other instructors
have students who will benefit from the increased reach that the software
provides.
This manual's existence, and status as a separate book, gives teachers the 
freedom to make the choice that suits their class.


\section{Why \Sage?}
In 
\textit{Open Source Mathematical Software\,}\citep{JoynerStein07}\footnote{See 
\protect\url{http://www.ams.org/notices/200710/tx071001279p.pdf} for the 
full text.}
the authors argue that for Mathematics the best way forward
is to use software that is Open Source.

\begin{quotation}\small
Suppose Jane is a well-known mathematician who announces
she has proved a theorem. We probably will believe
her, but she knows that she will be required to produce
a proof if requested. However, suppose now Jane says a
theorem is true based partly on the results of software. The
closest we can reasonably hope to get to a rigorous proof
(without new ideas) is the open inspection and ability to use
all the computer code on which the result depends. If the
program is proprietary, this is not possible. We have every
right to be distrustful, not only due to a vague distrust of
computers but because even the best programmers regularly
make mistakes.

If one reads the proof of Jane’s theorem in hopes of
extending her ideas or applying them in a new context, it
is limiting to not have access to the inner workings of the
software on which Jane’s result builds.
\end{quotation}  
Professionals choose their tools by balancing many factors but
this argument is persuasive.
We use \Sage{} because it is very capable 
so students can 
learn a great deal from it,
and because it is 
Free\footnote{The Free Software Foundation page 
\protect\url{http://www.gnu.org/philosophy/free-sw.html} 
gives background and a definition.} 
and Open Source.\footnote{See \protect\url{http://opensource.org/osd.html} 
for a definition.} 



\section{This manual}
This is Free.
Get the latest version from 
\url{http://joshua.smcvt.edu/linearalgebra}.
Also see that page for the license details and for 
the \LaTeX{} source.
I am glad to get feedback, especially from instructors
who have class-tested the material.
My contact information is on the same page. 

The computer output included here was generated automatically 
(I have automatically edited some lines).
This is my \Sage.
\begin{sageoutput}[d,0,1]
version()  
\end{sageoutput}



\section{Reading this manual}
Here I don't define all the terms or prove all the results.
So a person should read the material after covering the associated
chapter in the book, using that for reference.

The association between chapters here and chapters in the book is roughly:
\textit{Python and Sage} is an introduction that does not depend on the
book,
\textit{Gauss’s Method} is for Chapter One,
\textit{Vector Spaces} is for Chapter Two,
\textit{Matrices}, 
\textit{Maps}, and 
\textit{Singular Value Decomposition} go with Chapter Three,
\textit{Geometry of Linear Maps} goes best with Chapter Four,
and \textit{Eigenvalues} fits with the material from Chapter Five
(it mentions Jordan Form, but only relies on the material up to 
the traditional ending spot of Diagonalization.)






\section{Acknowledgements}
I am glad for this chance to thank the \Sage{} Development Team.
In particular,
without \citep{SageTeam12ref} this work would not have happened.
I am glad also for the chance to mention 
\citep{Beezer11} as an inspiration.
Finally, I am grateful to Saint Michael's College for the time to work on this.





\vspace{.5in}
\noindent\begin{tabular}[t]{@{}l}
\textit{We emphasize practice.}  \\
\hspace*{5em}--\cite{Suzuki}  \\[.5ex]
\textit{[A]n orderly presentation is not necessarily bad} \\ 
\quad\textit{but by itself may be insufficient.}  \\
\hspace*{5em}--\cite{Brandt}  
\end{tabular}

\vspace*{.25in}
\hbox{}
\hfill
\begin{tabular}[t]{l@{}}
Jim Hef{}feron \\
Mathematics, Saint Michael's College \\
Colchester, Vermont USA \\
2012-Sep-10
\end{tabular}
\vspace{1ex}
\endinput

TODO

1) Talk about how the chapter should be read after the associated book
chapter.