\chapter*{Preface}\pagestyle{preface}\thispagestyle{preface}


This is a collection of explorations that supplement the text
\textit{Linear Algebra}.\footnote{\protect\url{http://joshua.smcvt.edu/linearalgebra}}

The text develops the material using examples and practice problems
that are small-sized and have simple numbers.
This is natural: while a student must build intuition by
working through the examples and
associated exercises, large or more awkward
cases run the risk of more obscuring the issues than 
clarifying them.
That is, the labor involved in multiplying two twenty by twenty matrices
of ten decimal place numbers asks a great deal of the patience of the
reader, for only a tiny bit more payoff in intuition than the student would
get by 
multiplying two three by three matrices of small integers. 

But mathematical software 
extends the reach of students to work with more systems, or
with numbers that are more generic than the ones available to  
the text.
These are more like the systems and numbers that they will apply linear 
algebra to in other subjects.

In this manual students will examine the principles and operations 
from the text in the context
of the software \sage{}.\footnote{\url{www.sagemath.org}}


\section{Why \sage?}
\sage{} is a very powerful mathematical software systems but so are
many others.
This manual uses it because it is Free software.

In 
\textit{Open Source Mathematical Software}\footnote{Appeared as an opinion in the \protect\textit{Notices of the American Mathematical Society} \protect\url{http://www.ams.org/notices/200710/tx071001279p.pdf.}}
the authors argue that for Mathematics, using Open Source software
is the right way forward.

\begin{quotation}\small
Suppose Jane is a well-known mathematician who announces
she has proved a theorem. We probably will believe
her, but she knows that she will be required to produce
a proof if requested. However, suppose now Jane says a
theorem is true based partly on the results of software. The
closest we can reasonably hope to get to a rigorous proof
(without new ideas) is the open inspection and ability to use
all the computer code on which the result depends. If the
program is proprietary, this is not possible. We have every
right to be distrustful, not only due to a vague distrust of
computers but because even the best programmers regularly
make mistakes.

If one reads the proof of Jane’s theorem in hopes of
extending her ideas or applying them in a new context, it
is limiting to not have access to the inner workings of the
software on which Jane’s result builds.
\end{quotation}  
While professionals choose their tools by balancing many factors,
I find this argument compelling.
\sage{} is very capable, including at Linear Algebra, and students can 
learn a great deal from it.


\section{This book}
The latest version of this book is available from the home page of the project
\url{http://joshua.smcvt.edu/linearalgebra}.
Its license is Free; see that page for more detail.
You can also get the source there, if you are into \LaTeX.

Feedback has been very useful for me in the past so I am glad to hear 
any suggestions or corrections, especially from instructors.
My contact information is on the same page. 



\vspace{.5in}
\begin{flushright}
\begin{tabular}{l@{}}
Jim Hef{}feron \\
Mathematics, Saint Michael's College \\
2012-Sep-10
\end{tabular}  
\end{flushright}



\endinput