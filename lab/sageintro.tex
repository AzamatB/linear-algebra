\chapter{\python{} and \sage{}}

You need to learn enough \sage{} to cover the Linear Algebra
in this manual.
\sage{} uses the computer language \python{} so you need to start with 
some of that.




\section{\python}
\python{} is a popular computer language, often used for scripting,
that is appealing for its simple style and powerful libraries,
and because it is Free Software.
The significance of `scripting' is that \sage{} uses it in this way,
as a glue to bring together separate parts.

\python's home page is \href{http://www.python.org}{\url{www.python.org}}.
Download and installation instructions are there, as well as 
an excellent tutorial.
Here we will acquaint you with enough \python{} to get started, possibly
even if you have no programming experience whatsoever.

\smallskip
\textit{A note on \python{} 3.}
There is a new version of \python{}, called \python~3, with some differences.
% For instance, \lstinline[style=inline]!print! works a bit differently.
Here we stick to the older version 
because that is what \sage{} uses.


\subsection{Basics}
Start \python, for instance by running 
\lstinline[style=inline]!python!
from a command line.
You'll get a couple of lines of 
identifying information followed by three greater-than
characters.
\begin{lstlisting}[style=python]
>>>   
\end{lstlisting}
This is a prompt.
If you type \python{} code and \keyboardkey{Enter} then the system
will read your code, evaluate it, and print the result.
So the prompt lets you try things.
We will see below how to write and run entire \python{} programs
but for the moment we will stick to the experiment. 
(You can get out of the prompt with \textit{\keyboardkey{Ctrl}-D}.)

Try entering these expressions (double star is exponentiation).
\begin{lstlisting}[style=python]
>>> 2-(-1)
3
>>> 1+2*3
7
>>> 2**3
8  
\end{lstlisting}

Part of \python's appeal is that simple things tend to be simple to do.
Here is how you print something to the screen.
\begin{lstlisting}[style=python]
>>> print 1,"plus",2,"equals",3
1 plus 2 equals 3
\end{lstlisting}
You can often debug just by printing things to the screen at
various steps and having a straightforward print operator helps with that. 

Of course, variables give you a place to keep values.
\begin{lstlisting}[style=python]
>>> i=1
>>> i+1
2
\end{lstlisting}
In some programming languages you must declare the `type' of a variable
before you use it; for instance you would have to declare 
that $i$ is an integer before you could set $i=1$.
In contrast, \python{} deduces the type of a variable 
based on what you do to it\Dash above we assigned $1$ to $i$ 
so \python{} deduced that it must be an integer.
Further, we can change how we use the variable and \python{} will 
go along; here we change what is in $x$ from an integer to a string.
\begin{lstlisting}[style=python]
>>> x=1
>>> x
1
>>> x='a'
>>> x
'a'
\end{lstlisting}

\python{} complains by raising an \textit{error}.
Here we are trying to combine a string and an integer. 
\begin{lstlisting}[style=python]
>>> 'a'+1
Traceback (most recent call last):
  File "<stdin>", line 1, in <module>
TypeError: cannot concatenate 'str' and 'int' objects
\end{lstlisting}
The error message's bottom line is the useful one.

The hash mark \lstinline[style=inline]!#! makes the rest of a line a comment.
\begin{lstlisting}[style=python]
>>> t=2.2
>>> d=(0.5)*9.8*(t**2)  # d in meters
>>> d
23.716000000000005
\end{lstlisting}
(Comments are more useful in a program than at the prompt.)
Programmers often comment an entire line by starting 
that line with a hash. 

As in the listing above, we can can represent real 
numbers 
and even complex numbers.
\begin{lstlisting}[style=python]
>>> 5.774*3
17.322
>>> (3+2j)-(1-4j)
(2+6j)
\end{lstlisting}
Notice that \python{} uses `$j$' for the square
root of $-1$, not the `$i$' traditional in mathematics.

The examples above show addition, subtraction, multiplication, 
and exponentiation. 
Division is a bit awkward.
\python{} was originally designed to have the division bar
\lstinline[style=inline]!/! mean real number division 
when at least one of the numbers is real.
However between two integers the division bar was taken to mean 
a quotient, as in ``$2$ goes into $5$ with quotient~$2$ and remainder~$1$.''
\begin{lstlisting}[style=python]
>>> 5.2/2.0
2.6
>>> 5.2/2
2.6
>>> 5/2
2
\end{lstlisting}
This was a mistake and one of the changes in \python~3
is that the quotient operation will be \lstinline[style=inline]!//!
while the single-slash operator will be real division in all cases.
But for now, the simplest thing is to make sure that at least one
number in a division is real.
\begin{lstlisting}[style=python]
>>> x=5.2
>>> y=2
>>> (1.0*x)/y
2.6
\end{lstlisting}
Incidentally, do the remainder or modulus operation  
with a percent character:~\lstinline[style=inline]!5 % 2! returns 1.

Variables can also represent truth values; these are \textit{Booleans}.
\begin{lstlisting}[style=python]
>>> yankees_stink=True
>>> yankees_stink
True
\end{lstlisting}
You need the initial capital:
\lstinline[style=inline]!True!
or \lstinline[style=inline]!False!, not
\lstinline[style=inline]!true!
or \lstinline[style=inline]!false!.
 
Above we saw a string, consisting of text between single quotes.
You can use either single quotes or double quotes, as long as you use
the same at both ends of the string. 
Here, \lstinline[style=inline]!x! and \lstinline[style=inline]!y!
are double-quoted, which makes sense because they contain apostrophe's. 
\begin{lstlisting}[style=python]
>>> x="I'm Popeye the sailor man"
>>> y="I yam what I yam and that's all what I yam"
>>> x+', '+y
"I'm Popeye the sailor man, I yam what I yam and that's all what I yam"
\end{lstlisting}
The \lstinline[style=inline]!+! operation concatenates strings.
Inside a double-quoted string you can use \lstinline[style=inline]!\n! to
get a newline.

A string marked by three sets of quotes can contain line breaks.
\begin{lstlisting}[style=python]
>>> """THE ROAD TO WISDOM
... 
... The road to wisdom?
... -- Well, it's plain
... and simple to express:
... Err
... and err
... and err again
... but less
... and less
... and less. --Piet Hein"""
\end{lstlisting}
The triple dots at the start of each line is a prompt that \python's
read-eval-print loop gives when what you have typed is not complete.
One use for these strings is as documentation in a program.

A \python{} \textit{dictionary} is a finite function, that is, it is a finite
set of ordered pairs $\langle\text{key},\text{value}\rangle$ subject 
to the restriction that no key is repeated.
\begin{lstlisting}[style=python]
>>> english_words={'one':1, 'two':2, 'three':3}
>>> english_words['four']=4  
\end{lstlisting}
So dictionaries are a simple database.
But the fact that a dictionary is a set means that the elements come in 
no usable order.
\begin{lstlisting}[style=python]
>>> english_words
{'four': 4, 'three': 3, 'two': 2, 'one': 1}
\end{lstlisting}
(In particular, they do not just come in the reverse of the order
in which you entered them.)
And, an assignment to an existing key replaces the previous value. 
\begin{lstlisting}[style=python]
>>> english_words['one']=5
>>> english_words
{'four': 4, 'three': 3, 'two': 2, 'one': 5}
\end{lstlisting}
Dictionaries are central to \python, in part because looking up values 
in a dictionary is very fast.

While dictionaries are unordered, a \python{} \textit{list} is ordered.
\begin{lstlisting}[style=python]
>>> a=['alpha', 'beta', 'gamma']
>>> b=[]
>>> c=['delta']
>>> a
['alpha', 'beta', 'gamma']
>>> a+b+c
['alpha', 'beta', 'gamma', 'delta']
\end{lstlisting}
Get an element of the list by specifying its index, its place in the list,
inside square brackets.
Also, specifying two indices separated by a colon will get a \textit{slice} 
of the list. 
\begin{lstlisting}[style=python]
>>> a[1]
'beta'
>>> a[1:3]
['beta', 'gamma']
>>> a[1:2]
['beta']
\end{lstlisting}
Notice that lists are zero-offset indexed, that is, the initial element of the
list is numbered $0$.
You can count from the back by using negative indices.
\begin{lstlisting}[style=python]
>>> a[0]
'alpha'
>>> a[-1]
'gamma'
\end{lstlisting}
You can add to a list.
\begin{lstlisting}[style=python]
>>> c.append('epsilon')
>>> c
['delta', 'epsilon']  
\end{lstlisting}

A \textit{tuple} is like a list in that it is ordered.
\begin{lstlisting}[style=python]
>>> a=(0, 1, 2)
>>> a
(0, 1, 2)
>>> a[0]
0
\end{lstlisting}
Unlike a list, however, a tuple cannot change.
\begin{lstlisting}[style=python]
>>> a[0]=3
Traceback (most recent call last):
  File "<stdin>", line 1, in <module>
TypeError: 'tuple' object does not support item assignment
\end{lstlisting}
That is an advantage sometimes:
because tuples cannot change they can be dictionary keys while
list cannot be keys.
\begin{lstlisting}[style=python]
>>> a=['Jim', 2138]
>>> b=('Jim', 2138)
>>> d={a:1}
Traceback (most recent call last):
  File "<stdin>", line 1, in <module>
TypeError: unhashable type: 'list'
>>> d={b:1}
>>> d
{('Jim', 2138): 1}
\end{lstlisting}



\subsection{Flow of control}
\python{} supports the traditional ways of affecting the order of 
statement execution, with a twist.
\begin{lstlisting}[style=python]
>>> x=4
>>> if (x==0):
...     y=1
... else:
...     y=0
... 
>>> y
0
\end{lstlisting}
The twist is that while many languages use braces or some other syntax to
mark a block of code, \python{} uses indentation.
(We shall always indent with four spaces.)
Here, the single-line block \lstinline[style=inline]!y=1! is executed if $x$
equals $0$, otherwise $y$ is set to~$0$. 

Notice also that ``is equal to'' is represented by a double equals
\lstinline[style=inline]!==!. 
Single equals is the assignment
operation so that \lstinline[style=inline]!x=4! 
means ``$x$ is assigned the value~$4$.'' 

There are two variants on the above.
The first has only one branch and so forgoes the \lstinline[style=inline]!else!
\begin{lstlisting}[style=python]
>>> x=4
>>> y=0
>>> if (x==0):
...     y=1
... 
>>> y
0
\end{lstlisting}
while the second has more than two branches.
\begin{lstlisting}[style=python]
>>> x=2
>>> if (x==0):
...     y=1
... elif (x==1):
...     y=0
... else:
...     y=-1
... 
>>> y
-1
\end{lstlisting}

A specialty of computers is iteration, looping through the same steps.
\begin{lstlisting}[style=python]
>>> for i in range(5):
...     print i,"squared is",i**2
... 
0 squared is 0
1 squared is 1
2 squared is 4
3 squared is 9
4 squared is 16
\end{lstlisting}
The function \lstinline[style=inline]!range! returns a list of numbers.
\begin{lstlisting}[style=python]
 >>> range(10)
[0, 1, 2, 3, 4, 5, 6, 7, 8, 9]
>>> range(1,10)
[1, 2, 3, 4, 5, 6, 7, 8, 9] 
\end{lstlisting}
Observe that by default \lstinline[style=inline]!range! starts at $0$.
This makes sense because sequences are zero-indexed.
\begin{lstlisting}[style=python]
>>> x=[4,0,3,0]
>>> for i in range(len(x)):
...     if (x[i]==0):
...         print "item",i,"is zero"
...     else:
...         print "item",i,"is nonzero"
... 
item 0 is nonzero
item 1 is zero
item 2 is nonzero
item 3 is zero  
\end{lstlisting}
(We could instead have written \lstinline[style=inline]!for c in x:! since
the \lstinline[style=inline]!for! loop can iterate over any sequence, not just
over a sequence of integers.) 
Observe also that $9$ is the highest number in the sequence 
\lstinline[style=inline]!range(10)!.
This is defined so that \lstinline[style=inline]!range(10)+range(10,20)! is the
same as \lstinline[style=inline]!range(20)!.

A \lstinline[style=inline]!for! loop is designed to execute a certain
number of times.
The natural construct for a loop that will run an uncertain number of times
is this.
\begin{lstlisting}[style=python]
>>> n = 27
>>> i = 0
>>> while (n != 1):
...     if (n%2 == 0):
...         n = n/2
...     else:
...         n = 3*n+1
...     i = i+1
...     print "i=",i
... 
i= 1
i= 2
i= 3  
\end{lstlisting}
(The listing takes $111$ steps to finish; the \textit{Collatz conjecture} is 
that for any starting~$n$ this \lstinline[style=inline]!while! loop will 
terminate.)

The \lstinline[style=inline]!break! command will leave a loop.



\subsection{Functions}
A \textit{function} is a group of statements that executes when it is called,
and can return values to the caller.
Here is a naive version of the quadratic formula.
\begin{lstlisting}[style=python]
>>> def quad_formula(a,b,c):
...     discriminant=(b**2-4*a*c)**(0.5)
...     r1=(-1*b+discriminant)/(2.0*a)
...     r2=(-1*b-discriminant)/(2.0*a)
...     return (r1,r2)
... 
>>> quad_formula(1,0,-9)
(3.0, -3.0)
>>> quad_formula(1,2,1)
(-1.0, -1.0)
\end{lstlisting}
% One way it is naive is that it doesn't handle complex cases gracefully.
% \begin{lstlisting}[style=python]
% >>> quad_formula(1,1,1)
% Traceback (most recent call last):
%   File "<stdin>", line 1, in <module>
%   File "<stdin>", line 2, in quad_formula
% ValueError: negative number cannot be raised to a fractional power
% \end{lstlisting}
Functions organize code into blocks which may be 
run a number of different times or which may belong together conceptually. 
In a \python{} program the great majority of code is in functions. 

At the end of the \lstinline[style=inline]!def! line in parentheses are
the function's \textit{parameters}, which can take values 
passed in by the caller.
Functions can have \textit{optional parameters} that have a default value.
\begin{lstlisting}[style=python]
>>> def hello(name="Jim"):
...     print "Hello,",name
... 
>>> hello("Fred")
Hello, Fred
>>> hello()
Hello, Jim  
\end{lstlisting}
\sage{} uses this aspect of \python{} a great deal.


\subsection{Objects and modules}
In Mathematics, the real numbers is a set associated with some operations
such as addition and multiplication.
\python{} is \textit{object-oriented}, which means that we can similarly bundle
things together.
\begin{lstlisting}[style=python]
>>> class person(object):
...     def __init__(self,name,age):
...         self.name=name
...         self.age=age
...     def hello(self):
...         print "Hello",self.name
... 
>>> a=person("Jim",53)
>>> a.hello()
Hello Jim
>>> a.age
53  
\end{lstlisting}
So, \python{} has 
objects\Dash bundles of code and data.
You work with them by using the \textit{dot notation}, meaning that
to get the age data bundled with \lstinline[style=inline]!a!
you write \lstinline[style=inline]!a.age!, and to 
call the \lstinline[style=inline]!hello! function bundled
with \lstinline[style=inline]!a! you write
\lstinline[style=inline]!a.hello()! 
(a function bundled in this way is called a method).

You won't be writing your own classes here but you will be using ones from
the extensive libraries of code that others have written, including the
code for \sage. 
For instance, \python{} has a library, or \textit{module}, for math.
\begin{lstlisting}[style=python]
>>> import math
>>> math.pi
3.141592653589793
>>> math.factorial(4)
24
>>> math.cos(math.pi)
-1.0
\end{lstlisting}
The \lstinline[style=inline]!import! statement gets the module and makes
its contents available.

Another library is for random numbers.
\begin{lstlisting}[style=python]
>>> import random
>>> while (random.randint(1,10) != 1):
...     print "wrong"
... 
wrong
wrong
\end{lstlisting}



\subsection{Programs}
The read-eval-print loop is great for small-scale experiments but
for more than four or five lines you 
want to put your work in a separate file and run it as a standalone program.

To write the code, use a text editor; one example is 
\pgm{Emacs}\footnote{It may come with your operating system or see \protect\url{http://www.gnu.org/software/emacs}.}).
You should try to use an editor with support for \python{} such as 
syntax highlighting, where the editor colors text according to what part
it is of the language.

Here is one example.
Start your editor, open a file called \path{test.py}, and enter these lines.
Note the triple-quoted documentation string at the top of the file; 
good practice is to include this documentation in everything you write.
\begin{lstlisting}[style=python]
# test.py
"""test

A test program for Python. 
"""

import datetime
 
current = datetime.datetime.now()  # get a datetime object
print "the month number is", current.month
\end{lstlisting}
Run it under \python{} (for instance, from the command line
run \lstinline[style=inline]!python test.py!) and you should see
output like \lstinline[style=inline]!the month number is 9!.

Here is a small game (it has some \python{} constructs that
you haven't seen but that are simple to figure out).
\begin{lstlisting}[style=python]
# guessing_game.py
"""guessing_game

A toy game for demonstration.
"""
import random
CHOICE = random.randint(1,10)

def test_guess(guess):
    """Decide if the guess is correct and print a message.
    """
    if (guess < CHOICE):
        print "  Sorry, your guess is too low"
        return False
    elif (guess > CHOICE):
        print "  Sorry, your guess is too high"
        return False
    print "  You are right!"
    return True

flag = False
while (not flag):
    guess = int(raw_input("Guess an integer between 1 and 10: "))
    flag = test_guess(guess)
\end{lstlisting}
Here is the output.
\begin{lstlisting}
$ python guessing_game.py
Guess an integer between 1 and 10: 5
  Sorry, your guess is too low
Guess an integer between 1 and 10: 8
  Sorry, your guess is too high
Guess an integer between 1 and 10: 6
  Sorry, your guess is too low
Guess an integer between 1 and 10: 7
  You are right!
\end{lstlisting}  % $

As above, note the triple-quoted documentation strings both for the 
file as a whole and for the function.
Go to the directory containing \path{guessing_game.py} and start \python{}.
At the prompt type in \lstinline[style=inline]!import guessing_game!.
You will play through a round of the game (there is a way to avoid this
but it doesn't matter for this point) and then type
\lstinline[style=inline]!help("guessing_game")!.
\begin{lstlisting}
DESCRIPTION
    A toy game for demonstration.

FUNCTIONS
    test_guess(guess)
        Decide if the guess is correct and print a message. 
\end{lstlisting}
Obviously, \python{} got this help from your documentation strings.
In \python{}, and also in \sage, people try to always include documentation
that is accessible from the \lstinline[style=inline]!help! command.




%----------------------------------
\section{\sage}
Much of this book involves learning what \sage{} can do, so this is a very
brief walk-through of preliminaries.
First, install \sage{} (your system may supply it or you can
follow the directions at
\href{http://www.sagemath.org}{www.sagemath.org}).

\subsection{Command line}
We start with \sage's prompt, which is like \python's but adapted to 
mathematical work.
From a command line call \lstinline[style=inline]!sage!.
There is some initial text and then a prompt.
\begin{lstlisting}[style=python]
sage:  
\end{lstlisting}
(Leave the prompt by typing \lstinline[style=inline]!exit!
and \keyboardkey{Enter}.)

Experiment with some expressions.
\begin{lstlisting}[style=python]
sage: 2**3                                                                           
8
sage: 2^3
8
sage: 3*1+4*2
11
sage: 5 == 3+3
False
sage: sin(pi/3)
1/2*sqrt(3)
\end{lstlisting}
The second shows that \sage{} provides a convenient shortcut for exponentiation.
The fourth shows that \sage{} sometimes returns exact results, rather than an
approximation.
If you prefer the approximation then you can get that. 
\begin{lstlisting}[style=python]
sage: sin(pi/3).numerical_approx()
0.866025403784439
sage: sin(pi/3).n()
0.866025403784439  
\end{lstlisting}
The \lstinline[style=inline]!n()! is an abbreviation for the
\lstinline[style=inline]!numerical_approx()!.


\subsection{Script}
You can group \sage{} commands together in a file.
This way you can test the commands, 
and also reuse them without having to retype.

Create a file with the extension \path{sage} such as \path{sage_try.sage}.
\begin{lstlisting}[style=python]
def normal_curve(upper_limit):
    """Approximate area under the Normal curve from 0 to upper_limit.
    """
    stdev=1.0
    mu=0.0
    area=numerical_integral((1/sqrt(2*pi)*e^(-0.5*(x)^2)),
                            0,upper_limit)    
    print "area is",area[0]
\end{lstlisting}
Bring in the commands in this way.
\begin{lstlisting}[style=python]
sage: load "sage_try.sage"
sage: normal_curve(1.0)   
area is 0.341344746069  
\end{lstlisting}


\subsection{Notebook}
\sage{} also offers a browser-based interface, where you can set up
worksheets that can be run alone or with other people, where you can easily
view plots integrated with the text, and many other nice features.
We won't be using that interface, just in a desire to keep the attention
on the Linear Algebra, but it is a very nice environment. 
\endinput