\chapter{\python{} and \Sage{}}

To work through the Linear Algebra in this manual
you need to be acquainted with how to run \Sage. 
\Sage{} uses the computer language \python{} so we'll start with that.


\section{\python}
\python{} is a popular computer language, often used for scripting,
that is appealing for its simple style and powerful libraries.
The significance for us of `scripting' is that \Sage{} uses it in this way,
as a glue to bring together separate parts.

\python{} is Free.
If your operating system doesn't provide it then go to the home 
page \href{http://www.python.org}{\url{www.python.org}} and follow the
download and installation instructions.
Also at that site is \python's excellent tutorial.
That tutorial is thorough; 
here you will see only enough \python{} to get started.

\smallskip
\textit{Comment.}
There is a new version, \python~3, with some differences.
Here we stick to the older version 
because that is what \Sage{} uses.


\subsection{Basics}
Start \python, for instance by entering 
\inlinecode{python}
at a command line.
You'll get a couple of lines of 
information followed by three greater-than
characters.
\begin{lstlisting}[style=python]
>>>   
\end{lstlisting}
This is a prompt.
It lets you experiment:~if you type 
\python{} code and \keyboardkey{Enter} then the system
will read your code, evaluate it, and print the result.
We will see below how to write and run whole programs
but for now we will experiment.
You can always leave the prompt with \textit{\keyboardkey{Ctrl}-D}.

Try entering these expressions (double star is exponentiation).
\begin{lstlisting}[style=python]
>>> 2 - (-1)
3
>>> 1 + 2*3
7
>>> 2**3
8  
\end{lstlisting}

Part of \python's appeal is that simple things tend to be simple to do.
Here is how you print something to the screen.
\begin{lstlisting}[style=python]
>>> print 1, "plus", 2, "equals", 3
1 plus 2 equals 3
\end{lstlisting}
Often you can debug just by putting in commands to print things, 
and having a straightforward print operator helps with that. 

As in any other computer language, variables give you a place to keep values.
\begin{lstlisting}[style=python]
>>> i = 1
>>> i + 1
2
\end{lstlisting}
In some programming languages you must declare the `type' of a variable
before you use it; for instance you would have to declare 
that $i$ is an integer before you could set $i=1$.
In contrast, \python{} deduces the type of a variable 
based on what you do to it\Dash above we assigned $1$ to $i$ 
so \python{} figured that it must be an integer.
Further, we can change how we use the variable and \python{} will 
go along; here we change what is in $x$ from an integer to a string.
\begin{lstlisting}[style=python]
>>> x = 1
>>> x
1
>>> x = 'a'
>>> x
'a'
\end{lstlisting}

\python{} complains by raising an \textit{error}.
Here we are trying to combine a string and an integer. 
\begin{lstlisting}[style=python]
>>> 'a'+1
Traceback (most recent call last):
  File "<stdin>", line 1, in <module>
TypeError: cannot concatenate 'str' and 'int' objects
\end{lstlisting}
The error message's bottom line is the useful one.

The hash mark \inlinecode{\#} makes the rest of a line a comment.
\begin{lstlisting}[style=python]
>>> t = 2.2
>>> d = (0.5) * 9.8 * (t**2)  # d in meters
>>> d
23.716000000000005
\end{lstlisting}
(Comments are more useful in a program than at the prompt.)
Programmers often comment an entire line by starting 
that line with a hash. 

As in the listing above, we can represent real 
numbers 
and even complex numbers.
\begin{lstlisting}[style=python]
>>> 5.774 * 3
17.322
>>> (3+2j) - (1-4j)
(2+6j)
\end{lstlisting}
Notice that \python{} uses `$j$' for the square
root of $-1$, not the `$i$' traditional in mathematics.

The examples above show addition, subtraction, multiplication, 
and exponentiation. 
Division is a bit awkward.
\python{} was originally designed to have the division bar
\inlinecode{/} mean real number division 
when at least one of the numbers is real.
However between two integers the division bar was taken to mean 
a quotient, as in ``$2$ goes into $5$ with quotient~$2$ and remainder~$1$.''
\begin{lstlisting}[style=python]
>>> 5.2 / 2.0
2.6
>>> 5.2 / 2
2.6
>>> 5 / 2
2
\end{lstlisting}
This was a mistake and one of the changes in \python~3
is that the quotient operation will be \inlinecode{//}
while the single-slash operator will be real division in all cases.
In \python~2 the simplest thing is to make sure that at least one
number in a division is real.
\begin{lstlisting}[style=python]
>>> x = 5
>>> y = 2
>>> (1.0*x) / y
2.5
\end{lstlisting}
Incidentally, do the remainder operation (sometimes called `modulus')  
with a percent character:~\inlinecode{5 \% 2} returns 1.

Variables can also represent truth values; these are \textit{Booleans}.
\begin{lstlisting}[style=python]
>>> yankees_stink = True
>>> yankees_stink
True
\end{lstlisting}
You need the initial capital:
\inlinecode{True}
or \inlinecode{False}, not
\inlinecode{true}
or \inlinecode{false}.
 
Above we saw a string consisting of text between single quotes.
You can use either single quotes or double quotes, as long as you use
the same at both ends of the string. 
Here \inlinecode{x} and \inlinecode{y}
are double-quoted, which makes sense because they contain apostrophes. 
\begin{lstlisting}[style=python]
>>> x = "I'm Popeye the sailor man"
>>> y = "I yam what I yam and that's all what I yam"
>>> x + ', ' + y
"I'm Popeye the sailor man, I yam what I yam and that's all what I yam"
\end{lstlisting}
The \inlinecode{+} operation concatenates strings.
Inside a double-quoted string you can use slash-\textit{n} \inlinecode{\\n} to
get a newline.

A string marked by three sets of quotes can contain line breaks.
\begin{lstlisting}[style=python]
>>> """THE ROAD TO WISDOM
... 
... The road to wisdom?
... -- Well, it's plain
... and simple to express:
... Err
... and err
... and err again
... but less
... and less
... and less. --Piet Hein"""
\end{lstlisting}
The triple dots at the start of each line is a prompt that \python's
read-eval-print loop gives when what you have typed is not complete.
One use for triple-quoted strings is as documentation in a program;
we'll see that below.

A \python{} \textit{dictionary} is a finite function, that is, it is a finite
set of ordered pairs $\langle\text{key},\text{value}\rangle$ subject 
to the restriction that no key can be repeated.
\begin{lstlisting}[style=python]
>>> english_words = {'one': 1, 'two': 2, 'three': 3}
>>> english_words['one']
1
>>> english_words['four'] = 4  
\end{lstlisting}
Dictionaries are a simple database.
But the fact that a dictionary is a set means that the elements come in 
no apparently-sensible order.
\begin{lstlisting}[style=python]
>>> english_words
{'four': 4, 'three': 3, 'two': 2, 'one': 1}
\end{lstlisting}
(Don't be mislead by this example, 
they do not always just come in the reverse of the order
in which you entered them.)
If you assign to an existing key then that will replace the previous value. 
\begin{lstlisting}[style=python]
>>> english_words['one'] = 5
>>> english_words
{'four': 4, 'three': 3, 'two': 2, 'one': 5}
\end{lstlisting}
Dictionaries are central to \python, in part because looking up values 
in a dictionary is very fast.

While dictionaries are unordered, a \python{} \textit{list} is ordered.
\begin{lstlisting}[style=python]
>>> a=['alpha', 'beta', 'gamma']
>>> b=[]
>>> c=['delta']
>>> a
['alpha', 'beta', 'gamma']
>>> a+b+c
['alpha', 'beta', 'gamma', 'delta']
\end{lstlisting}
Get an element from a list by specifying its index, its place in the list,
inside square brackets.
Lists are zero-offset indexed, that is, the initial element of the
list is numbered $0$.
You can count from the back by using negative indices.
\begin{lstlisting}[style=python]
>>> a[0]
'alpha'
>>> a[1]
'beta'
>>> a[-1]
'gamma'
\end{lstlisting}
Also, specifying two indices separated by a colon will get a \textit{slice} 
of the list. 
\begin{lstlisting}[style=python]
>>> a[1:3]
['beta', 'gamma']
>>> a[1:2]
['beta']
\end{lstlisting}
You can add to a list.
\begin{lstlisting}[style=python]
>>> c.append('epsilon')
>>> c
['delta', 'epsilon']  
\end{lstlisting}
Lists can contain anything, including other lists.
\begin{lstlisting}
>>> x = 4
>>> a = ['alpha', ['beta', x]]
>>> a
['alpha', ['beta', 4]]  
\end{lstlisting}

The function \pythoncmd{range} returns a list of numbers.
\begin{lstlisting}[style=python]
 >>> range(10)
[0, 1, 2, 3, 4, 5, 6, 7, 8, 9]
>>> range(1,10)
[1, 2, 3, 4, 5, 6, 7, 8, 9] 
\end{lstlisting}
Observe that by default \pythoncmd{range} starts at $0$, which
is convenient because lists are zero-indexed.
Observe also that $9$ is the highest number in the sequence 
given by \inlinecode{range(10)}.
This makes \inlinecode{range(10)+range(10,20)} give 
the same list as \inlinecode{range(20)}.

One of the most common things you'll do with a list is to run through
it performing some action on each entry.
\python{} has a shortcut for this called \textit{list comprehension}.
\begin{lstlisting}
>>> a = [2**i for i in range(4)]
>>> a
[1, 2, 4, 8]
>>> [i-1 for i in a]
[0, 1, 3, 7]
\end{lstlisting}
This is a special form of a loop; we'll see more on loops below.

A \textit{tuple} is like a list in that it is ordered.
\begin{lstlisting}[style=python]
>>> a = (0, 1, 2)
>>> a
(0, 1, 2)
>>> a[0]
0
\end{lstlisting}
However, unlike a list a tuple cannot change.
\begin{lstlisting}[style=python]
>>> a[0] = 3
Traceback (most recent call last):
  File "<stdin>", line 1, in <module>
TypeError: 'tuple' object does not support item assignment
\end{lstlisting}
That is an advantage sometimes:
because tuples cannot change they can be keys in dictionaries, while
list cannot.
\begin{lstlisting}[style=python]
>>> a = ['Jim', 2138]
>>> b = ('Jim', 2138)
>>> d = {a: 1}
Traceback (most recent call last):
  File "<stdin>", line 1, in <module>
TypeError: unhashable type: 'list'
>>> d = {b: 1}
>>> d
{('Jim', 2138): 1}
\end{lstlisting}

\python{} has a special value \inlinecode{None} for you to use
where there is no sensible value.
For instance, if your program keeps track of a person's address and
includes a variable \pythoncmd{appt\_no} then \inlinecode{None} is
the right value for that variable when the person does not live in an
apartment.



\subsection{Flow of control}
\python{} supports the traditional ways of affecting the order of 
statement execution, with a twist.
\begin{lstlisting}[style=python]
>>> x = 4
>>> if (x == 0):
...     y = 1
... else:
...     y = 0
... 
>>> y
0
\end{lstlisting}
The twist is that while many languages use braces or some other syntax to
mark a block of code, \python{} uses indentation.
(We shall always indent with four spaces.)
Here, \python{} executes the single-line block \inlinecode{y = 1} if $x$
equals $0$, otherwise \python{} sets $y$ to~$0$. 

Notice also that double equals \inlinecode{==} means ``is equal to.'' 
We have already seen that single equals is the assignment
operation so that \inlinecode{x = 4} 
means ``$x$ is assigned the value~$4$.'' 

\python{} has two variants on the \pythoncmd{if} statement.
The first has only one branch
\begin{lstlisting}[style=python]
>>> x = 4
>>> y = 0
>>> if (x == 0):
...     y = 1
... 
>>> y
0
\end{lstlisting}
while the second has more than two branches.
\begin{lstlisting}[style=python]
>>> x = 2
>>> if (x == 0):
...     y = 1
... elif (x == 1):
...     y = 0
... else:
...     y = -1
... 
>>> y
-1
\end{lstlisting}

A specialty of computers is iteration, looping through the same steps.
\begin{lstlisting}[style=python]
>>> for i in range(5):
...     print i, "squared is", i**2
... 
0 squared is 0
1 squared is 1
2 squared is 4
3 squared is 9
4 squared is 16
\end{lstlisting}
A \pythoncmd{for} loop often involves a \pythoncmd{range}.
\begin{lstlisting}[style=python]
>>> x=[4,0,3,0]
>>> for i in range(len(x)):
...     if (x[i] == 0):
...         print "item",i,"is zero"
...     else:
...         print "item",i,"is nonzero"
... 
item 0 is nonzero
item 1 is zero
item 2 is nonzero
item 3 is zero  
\end{lstlisting}
We could instead have written \inlinecode{for c in x:} since
the \pythoncmd{for} loop can iterate over any sequence, not just
a sequence of integers.

A \pythoncmd{for} loop is designed to execute a certain
number of times.
The natural way to write a loop that will run an uncertain number of times
is \pythoncmd{while}.
\begin{lstlisting}[style=python]
>>> n = 27
>>> i = 0
>>> while (n != 1):
...     if (n%2 == 0):
...         n = n / 2
...     else:
...         n = 3*n + 1
...     i = i + 1
...     print "i=", i
... 
i= 1
i= 2
i= 3  
\end{lstlisting}
(This listing is incomplete; it takes $111$ steps to finish.)\footnote{The 
\protect\textit{Collatz conjecture} is 
that for any starting~$n$ this loop will 
terminate, but this is not known.}
Note that ``not equal'' is \lstinline[style=inline]@!=@. 

The \inlinecode{break} command gets you out of a loop right away.
\begin{lstlisting}
>>> for i in range(10):
...     if (i == 3):
...         break
...     print "i is", i
... 
i is 0
i is 1
i is 2  
\end{lstlisting}


\subsection{Functions}
A \textit{function} is a group of statements that executes when it is called,
and can return values to the caller.
Here is a naive version of the quadratic formula.
\begin{lstlisting}[style=python]
>>> def quad_formula(a, b, c):
...     discriminant = (b**2 - 4*a*c)**(0.5)
...     r1=(-1*b+discriminant) / (2.0*a)
...     r2=(-1*b-discriminant) / (2.0*a)
...     return (r1, r2)
... 
>>> quad_formula(1,0,-9)
(3.0, -3.0)
>>> quad_formula(1,2,1)
(-1.0, -1.0)
\end{lstlisting}
(One way that it is naive is that it doesn't handle complex roots gracefully.)
% \begin{lstlisting}[style=python]
% >>> quad_formula(1,1,1)
% Traceback (most recent call last):
%   File "<stdin>", line 1, in <module>
%   File "<stdin>", line 2, in quad_formula
% ValueError: negative number cannot be raised to a fractional power
% \end{lstlisting}
Functions organize code into blocks that may be 
run a number of different times or which may belong together conceptually. 
In a \python{} program the great majority of code is in functions. 

At the end of the \inlinecode{def} line, in parentheses, are
the function's \textit{parameters}. 
These can take values 
passed in by the caller.
Functions can have \textit{optional parameters} that have a default value.
\begin{lstlisting}[style=python]
>>> def hello(name="Jim"):
...     print "Hello,", name
... 
>>> hello("Fred")
Hello, Fred
>>> hello()
Hello, Jim  
\end{lstlisting}
\Sage{} uses this aspect of \python{} a great deal.

Functions can contain multiple \inlinecode{return} statements.
They always return something; 
if a function never executes a \inlinecode{return} then it will
return the value \inlinecode{None}.




\subsection{Objects and modules}
In Mathematics, the real numbers is a set associated with some operations
such as addition and multiplication.
\python{} is \textit{object-oriented}, which means that we can similarly bundle
together data and actions.
\begin{lstlisting}[style=python]
>>> class person(object):
...     def __init__(self, name, age):
...         self.name = name
...         self.age = age
...     def hello(self):
...         print "Hello", self.name
... 
>>> a=person("Jim", 53)
>>> a.hello()
Hello Jim
>>> a.age
53  
\end{lstlisting}
You work with objects by using the \textit{dot notation}:
to get the age data bundled with \inlinecode{a}
you write \inlinecode{a.age}, and to 
call the \inlinecode{hello} function bundled
with \inlinecode{a} you write
\inlinecode{a.hello()} 
(a function bundled in this way is called a \textit{method}).

You won't be writing your own classes in this lab manual 
but you will be using ones from
the extensive libraries of code that others have written, including the
code for \Sage. 
For instance, \python{} has a library, or \textit{module}, for math.
\begin{lstlisting}[style=python]
>>> import math
>>> math.pi
3.141592653589793
>>> math.factorial(4)
24
>>> math.cos(math.pi)
-1.0
\end{lstlisting}
The \inlinecode{import} statement gets the module and makes
its contents available.

Another library is for random numbers.
\begin{lstlisting}[style=python]
>>> import random
>>> while (random.randint(1,10) != 1):
...     print "wrong"
... 
wrong
wrong
\end{lstlisting}



\subsection{Programs}
The read-eval-print loop is great for small experiments but
for more than four or five lines you 
want to put your work in a separate file and run it as a standalone program.

To write the code, use a text editor; one example is 
\pgm{Emacs}\footnote{It may come with your operating system or 
see \protect\url{http://www.gnu.org/software/emacs}.}.
You should try to use an editor with support for \python{} such as
automatic indentation, and  
syntax highlighting, where the editor colors your code to make it easier to
read.

Here is one example.
Start your editor, open a file called \path{test.py}, and enter these lines.
Note the triple-quoted documentation string at the top of the file; 
good practice is to include this documentation in everything you write.
\begin{lstlisting}[style=python]
# test.py
"""test

A test program for Python. 
"""

import datetime
 
current = datetime.datetime.now()  # get a datetime object
print "the month number is", current.month
\end{lstlisting}
Run it under \python{} (for instance, from the command line
run \inlinecode{python test.py}) and you should see
output like \inlinecode{the month number is 9}.

Here is a small game (it has some \python{} constructs that
you haven't seen but that are straightforward).
\begin{lstlisting}[style=python]
# guessing_game.py
"""guessing_game

A toy game for demonstration.
"""
import random
CHOICE = random.randint(1,10)

def test_guess(guess):
    """Decide if the guess is correct and print a message.
    """
    if (guess < CHOICE):
        print "  Sorry, your guess is too low"
        return False
    elif (guess > CHOICE):
        print "  Sorry, your guess is too high"
        return False
    print "  You are right!"
    return True

flag = False
while (not flag):
    guess = int(raw_input("Guess an integer between 1 and 10: "))
    flag = test_guess(guess)
\end{lstlisting}
Here is the output.
\begin{lstlisting}
$ python guessing_game.py
Guess an integer between 1 and 10: 5
  Sorry, your guess is too low
Guess an integer between 1 and 10: 8
  Sorry, your guess is too high
Guess an integer between 1 and 10: 6
  Sorry, your guess is too low
Guess an integer between 1 and 10: 7
  You are right!
\end{lstlisting}  % $

As above, note the triple-quoted documentation strings both for the 
file as a whole and for the function.
Go to the directory containing \path{guessing_game.py} and start \python{}.
At the prompt type in \inlinecode{import guessing_game}.
You will play through a round of the game (there is a way to avoid this
but it doesn't matter here) and then type
\inlinecode{help("guessing_game")}.
You will see documentation that includes these lines.
\begin{lstlisting}
DESCRIPTION
    A toy game for demonstration.

FUNCTIONS
    test_guess(guess)
        Decide if the guess is correct and print a message. 
\end{lstlisting}
Obviously, \python{} got this from the file's documentation strings.
In \python{}, and also in \Sage, good practice is 
to always include documentation
that is accessible with the \inlinecode{help} command.




%----------------------------------
\section{\Sage}
Learning what \Sage{} can do is the object of much of this book 
so this is a very
brief walk-through of preliminaries.
First, if your system does not already supply it then install \Sage{} 
by following the directions at
\href{http://www.sagemath.org}{www.sagemath.org}.

\subsection{Command line}
\Sage's command line is like \python's but adapted to 
mathematical work.
First start \Sage,
for instance, enter \inlinecode{sage} into a command line window.
You get some initial text and then a prompt
(leave the prompt by typing \inlinecode{exit}
and \keyboardkey{Enter}.)
\begin{lstlisting}[style=python]
sage:  
\end{lstlisting}

Experiment with some expressions.
\begin{lstlisting}[style=python]
sage: 2**3                                                                           
8
sage: 2^3
8
sage: 3*1 + 4*2
11
sage: 5 == 3+3
False
sage: sin(pi/3)
1/2*sqrt(3)
\end{lstlisting}
The second expression 
shows that \Sage{} provides a convenient shortcut for exponentiation.
The fourth shows that \Sage{} sometimes returns exact results rather than an
approximation.
You can still get the approximation. 
\begin{lstlisting}[style=python]
sage: sin(pi/3).numerical_approx()
0.866025403784439
sage: sin(pi/3).n()
0.866025403784439  
\end{lstlisting}
The function \inlinecode{n()} is an abbreviation for 
\inlinecode{numerical_approx()}.


\subsection{Script}
You can group \Sage{} commands together in a file.
This way you can test the commands, 
and also reuse them without having to retype.

Create a file with the extension \path{.sage}, such as \path{sage_try.sage}.
Enter this function and save the file.
\begin{lstlisting}[style=python]
def normal_curve(upper_limit):
    """Approximate area under the Normal curve from 0 to upper_limit.
    """
    stdev = 1.0
    mu = 0.0
    area=numerical_integral((1/sqrt(2*pi) * e^(-0.5*(x)^2)),
                             0, upper_limit)    
    print "area is", area[0]
\end{lstlisting}
Bring in the commands in with a \Sagecmd{load} command.
\begin{lstlisting}[style=python]
sage: load "sage_try.sage"
sage: normal_curve(1.0)   
area is 0.341344746069  
\end{lstlisting}


\subsection{Notebook}
\Sage{} also offers a browser-based interface, where you can set up
worksheets to run alone or with other people, where you can easily
view plots integrated with the text, and many other nice features.

From the \Sage{} prompt run \inlinecode{notebook()} and
work through the tutorial.
\endinput


TODO:
  1) how to use notebook to do exercises?