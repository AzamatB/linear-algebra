\chapter{Vector Spaces}

\Sage{} can operate with vector spaces, for example by finding a basis for
a space.




%========================================
\section{Real $n$-spaces}

Start by creating a real vector space~$\Re^n$.
\begin{lstlisting}
sage: V = RR^3
sage: V
Vector space of dimension 3 over Real Field with 53 bits of precision  
\end{lstlisting}
You can test for membership.
\begin{lstlisting}
sage: v1 = vector(RR, [1, 2, 3])
sage: v1 in V
True
sage: v2 = vector(RR, [1, 2, 3, 4])
sage: v2 in V
False  
\end{lstlisting}

Consider the plane through the origin in $\Re^3$ described by the equation
$x-2y+2z=0$.
It is a subspace of $\Re^3$ and we can describe it as a span.
\begin{equation*}
  W=\set{\colvec{x \\ y \\ z}
    \suchthat x=2y-2z}
  =\set{\colvec{2 \\ 1 \\ 0}y+\colvec[r]{-2 \\ 0 \\ 1}z
        \suchthat y,z\in\Re}
\end{equation*}
You can create that subspace.
\begin{lstlisting}
sage: V = RR^3
sage: v1 = vector(RR, [2, 1, 0])
sage: v2 = vector(RR, [-2, 0, 1])
sage: W = V.span([v1, v2])
\end{lstlisting}
And you can check that it is the desired space by doing some membership
tests.
\begin{lstlisting}
sage: v3 = vector(RR, [0, 1, 1])
sage: v3 in W
True
sage: v4 = vector(RR, [1, 0, 0])
sage: v4 in W
False  
\end{lstlisting}

For a closer look at what's happening here, create a subspace of $\Re^4$.
(Some of the output in this chapter is edited to fit the page width.)
\begin{lstlisting}
sage: V = RR^4
sage: V
Vector space of dimension 4 over Real Field with 53 bits of precision
sage: v1 = vector(RR, [2, 0, -1, 0])
sage: W = V.span([v1])
sage: W
Vector space of degree 4 and dimension 1 over Real Field with 53 bits 
    of precision
Basis matrix:
[  1.00000000000000  0.000000000000000 -0.500000000000000  
    0.000000000000000]
\end{lstlisting}
\Sage{} has identified the dimension of the subspace and found
a basis containing one vector (it prefers the vector with a leading~$1$).
\Sage{} also mentions the ``Real Field'' because we could take scalars
from other number systems\Dash in the book's final chapter the scalars are
from the complex numbers\Dash but we will here stick with real numbers.
The ``53 bits of precision'' refers to the fact that \Sage{} is using
this computer's built-in model of real numbers.\footnote{This computer uses
IEEE~754 double-precision binary floating-point numbers;  
if you have programmed then you may know this number model as binary64.}  

As earlier, the membership set relation works here.
\begin{lstlisting}
sage: v2 = vector(RR, [2, 1, -1, 0])
sage: v2 in W
False
sage: v3 = vector(RR, [-4, 0, 2, 0])
sage: v3 in W
True  
\end{lstlisting}



\subsection{Basis}
\Sage{} will give you a basis for your space.
\begin{lstlisting}
sage: V=RR^2
sage: v=vector(RR, [1, -1])
sage: W = V.span([v])     
sage: W.basis()
[
(1.00000000000000, -1.00000000000000)
]  
\end{lstlisting}

Here is another example.
\begin{lstlisting}
sage: V=RR^3               
sage: v1 = vector(RR, [1, -1, 0]) 
sage: v2 = vector(RR, [1, 1, 0]) 
sage: W = V.span([v1, v2])       
sage: W.basis()
[
(1.00000000000000, 0.000000000000000, 0.000000000000000),
(0.000000000000000, 1.00000000000000, 0.000000000000000)
]
sage: W.basis()
[
(1.00000000000000, 0.000000000000000, 0.000000000000000),
(0.000000000000000, 1.00000000000000, 0.000000000000000)
]
\end{lstlisting}
Adding a linearly dependent vector $\vec{v}_3$ to the spanning
set doesn't change the space.
\begin{lstlisting}
sage: v3 = vector(RR, [2, 3, 0])
sage: W_prime = V.span([v1, v2, v3])
sage: W_prime.basis()
[
(1.00000000000000, 0.000000000000000, 0.000000000000000),
(0.000000000000000, 1.00000000000000, 0.000000000000000)
]  
\end{lstlisting}

In the prior example, notice that \Sage{} does not simply
give you back the linearly independent
vectors that you gave it.
Instead, \Sage{} takes the vectors from the spanning set 
as the rows of a matrix,
brings that matrix to reduced echelon form, and reports the nonzero 
rows (transposed to column vectors and so printed with parentheses) 
as the members of the basis.
Because each matrix has one and only one reduced echelon form, each 
vector subspace of real $n$-space has one and only one such
basis;
this is a \textit{canonical basis} for the space.

\Sage{} will show you that it keeps the canonical basis as defining the
space. 
\begin{lstlisting}
sage: W
Vector space of degree 3 and dimension 2 over Real Field with 53 bits 
    of precision
Basis matrix:
[ 1.00000000000000 0.000000000000000 0.000000000000000]
[0.000000000000000  1.00000000000000 0.000000000000000]
sage: W_prime
Vector space of degree 3 and dimension 2 over Real Field with 53 bits 
    of precision
Basis matrix:
[ 1.00000000000000 0.000000000000000 0.000000000000000]
[0.000000000000000  1.00000000000000 0.000000000000000]  
\end{lstlisting}

If you are quite keen on your own basis then \Sage{} will
accomodate you.
\begin{lstlisting}
sage: V = RR^3
sage: v1 = vector(RR, [1, 2, 3])
sage: v2 = vector(RR, [2, 1, 3])
sage: W = V.span_of_basis([v1, v2])
sage: W.basis()
[
(1.00000000000000, 2.00000000000000, 3.00000000000000),
(2.00000000000000, 1.00000000000000, 3.00000000000000)
]
sage: W
Vector space of degree 3 and dimension 2 over Real Field with 53 bits 
    of precision
User basis matrix:
[1.00000000000000 2.00000000000000 3.00000000000000]
[2.00000000000000 1.00000000000000 3.00000000000000]
\end{lstlisting}




\subsection{Space equality}

You can test spaces for equality.
\begin{lstlisting}
sage: V = RR^4
sage: v1 = vector(RR, [1, 0, 0, 0])
sage: v2 = vector(RR, [1, 1, 0, 0])
sage: W12 = V.span([v1, v2])
sage: v3 = vector(RR, [2, 1, 0, 0])
sage: W13 = V.span([v1, v3])  
\end{lstlisting}
Now use set membership to check that the two spaces $W_{12}$ and~$W_{13}$
are equal.
\begin{lstlisting}
sage: v3 in W12
True
sage: v2 in W13
True  
\end{lstlisting}
Then, since obviously $\vec{v}_1\in W_{12}$ and $\vec{v}_1\in W_{13}$, the two
spans are equal.

You can also just ask \Sage{}.
\begin{lstlisting}
sage: W12 == W13
True 
\end{lstlisting}

Your check that equality works would be half-hearted if you didn't have 
an unequal case.
\begin{lstlisting}
sage: v4 = vector(RR, [1, 1, 1, 1])
sage: W14 = V.span([v1, v4])
sage: v2 in W14
False
sage: v3 in W14                                 
False
sage: v4 in W12
False
sage: v4 in W13
False
sage: W12 == W14                                                              
False
sage: W13 == W14
False 
\end{lstlisting}

This illustrates a point about algorithms.
\Sage{} could check for equality of two spans 
by checking whether every member of the first spanning set is in the
second space and vice versa (since the two spanning sets are finite). 
But \Sage{} does something different.
For each space it maintains the canonical basis
and it checks for equality of the two spaces
just by checking that they have the same canonical bases.
These two algorithms 
have the same external behavior, in that both report correctly whether the
two spaces are equal, but the second is faster.
Finding the fastest way to do jobs is an important research area of computing.


\subsection{Space operations}
\Sage{} finds the intersection of two spaces.
Consider these members of $\Re^3$.
\begin{equation*}
  \vec{v}_1=\colvec{1 \\ 0 \\ 0}
  \quad \vec{v}_2=\colvec{0 \\ 1 \\ 0}
  \quad \vec{v}_3=\colvec{0 \\ 0 \\ 3}
\end{equation*}
We form two spans, the $xy$-plane $W_{12}=\spanof{\vec{v}_1, \vec{v}_2}$ 
and the $yz$-plane $W_{23}=\spanof{\vec{v}_2, \vec{v}_3}$.
The intersection of these two is the $y$-axis. 
\begin{lstlisting}
sage: V=RR^3
sage: v1 =  vector(RR, [1, 0, 0])
sage: v2 =  vector(RR, [0, 1, 0])
sage: W12 = V.span([v1, v2])
sage: W12.basis()
[
(1.00000000000000, 0.000000000000000, 0.000000000000000),
(0.000000000000000, 1.00000000000000, 0.000000000000000)
]
sage: v3 = vector(RR, [0, 0, 2])
sage: W23 = V.span([v2, v3])
sage: W23.basis()
[
(0.000000000000000, 1.00000000000000, 0.000000000000000),
(0.000000000000000, 0.000000000000000, 1.00000000000000)
]
sage: W = W12.intersection(W23)
sage: W.basis()
[
(0.000000000000000, 1.00000000000000, 0.000000000000000)
]
\end{lstlisting}

Remember that the span of the empty set is the trivial space.
\begin{lstlisting}
sage: v3 = vector(RR, [1, 1, 1])
sage: W3 = V.span([v3])
sage: W3.basis()
[
(1.00000000000000, 1.00000000000000, 1.00000000000000)
]
sage: W4 = W12.intersection(W3)
sage: W4.basis()
[

]
\end{lstlisting}

\Sage{} will also find the sum of spaces, the span of their union.
\begin{lstlisting}
sage: W5 = W12 + W3
sage: W5.basis()
[
(1.00000000000000, 0.000000000000000, 0.000000000000000),
(0.000000000000000, 1.00000000000000, 0.000000000000000),
(0.000000000000000, 0.000000000000000, 1.00000000000000)
]
sage: W5 == V
True
\end{lstlisting}






%========================================
\section{Other spaces}

You can extend these computations to
vector spaces that aren't a subspace of some $\Re^n$.
Just find a real space that matches the given one.

Consider this vector space set of quadratic polynomials
(under the usual operations of polynomial addition and scalar multiplication).
\begin{equation*}
  \set{ a_2x^2+a_1x+a_0\suchthat a_2-a_1=a_0}           
   =\set{ (a_1+a_0)x^2+a_1x+a_0 \suchthat a_1,a_0\in\Re}
\end{equation*}
It matches\footnote{The textbook's chapter on Maps Between Spaces makes 
``matches'' precise.}
this subspace of $\Re^3$.
\begin{equation*}
  \set{\colvec{a_1+a_0 \\ a_1 \\ a_0}\suchthat a_1,a_0\in\Re}
  =\set{\colvec{1 \\ 1 \\ 0}a_1+\colvec{1 \\ 0 \\ 1}a_0\suchthat a_1,a_0\in\Re}
\end{equation*}
\begin{lstlisting}
sage: V=RR^3
sage: v1 = vector(RR, [1, 1, 0])
sage: v2 = vector(RR, [1, 0, 1])
sage: W = V.span([v1, v2])
sage: W.basis()
[
(1.00000000000000, 0.000000000000000, 1.00000000000000),
(0.000000000000000, 1.00000000000000, -1.00000000000000)
]  
\end{lstlisting}

Similarly you can represent this space of $\nbyn{2}$ matrices
\begin{equation*}
  \set{\begin{mat}
         a  &b \\
         c  &d
       \end{mat}\suchthat \text{$a-b+c=0$ and $b+d=0$}}
\end{equation*}
by finding a real $n$-space just like it.
Rewrite the given space
\begin{equation*}
  \set{\begin{mat}
         a  &b \\
         c  &d
       \end{mat}\suchthat \text{$a=-c-d$ and $b=-d$}}
  =\set{\begin{mat}
         -1  &0 \\
          1  &0
       \end{mat}c
       +
       \begin{mat}
         -1  &-1 \\
          0  &1
       \end{mat}d
       \suchthat c,d\in\Re}
\end{equation*}
and then here is a natural matching real space.
\begin{lstlisting}
sage: V = RR^4
sage: v1 = vector(RR, [-1, 0, 1, 0])
sage: v2 = vector(RR, [-1, -1, 0, 1])
sage: W = V.span([v1, v2])
sage: W.basis()
[
(1.00000000000000, -0.000000000000000, -1.00000000000000, 
    -0.000000000000000),
(0.000000000000000, 1.00000000000000, 1.00000000000000, 
    -1.00000000000000)
]  
\end{lstlisting}
You could have gotten another matching space by going down the columns 
instead of across the rows, etc.  
But the important things about the space, such as its dimension, are 
unaffected by that choice.

\endinput


TODO:
