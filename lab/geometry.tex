\chapter{Geometry of Linear Maps}

\Sage{} can illustrate the geometric effect of linear maps.
Here we focus on transformations of the plane~$\Re^2$.  




%========================================
\section{Lines map to lines}
The pictures in this chapter are 
based on the observation that linear maps send lines to lines.

Consider a domain space~$\Re^d$ and codomain space~$\Re^c$, along with
the linear map~$h$.
We get a line in the domain by fixing a vector of slopes~$\vec{m}\in\Re^d$
and a vector of offsets from the origin~$\vec{b}$ and taking the set 
$\ell = \set{\vec{v}=\vec{m}\cdot s +\vec{b}\suchthat s\in\Re}$.
Then the image of this line~$h(\ell)$ is the set 
$\set{h(\vec{m}s+\vec{b})\suchthat s\in\Re}
=\set{h(\vec{m})s+h(\vec{b})\suchthat s\in\Re}$.
This is a line in the codomain~$\Re^c$ with the vector of 
slopes~$h(\vec{m})$ and the vector of offsets~$h(\vec{b})$. 

For example, consider the transformation $\map{t}{\Re^2}{\Re^2}$ 
that rotates vectors counterclockwise by $\pi/6$
\begin{equation*}
  \rep{t}{\stdbasis_2,\stdbasis_2}
  =
  \begin{mat}
    \cos(\pi/6)  &-\sin(\pi/6) \\
    \sin(\pi/6)  &\cos(\pi/6)
  \end{mat}
  = 
  \begin{mat}
    \sqrt{3}/2   &-1/2 \\
    1/2          &\sqrt{3}/2
  \end{mat}
\end{equation*}
(remember that this differs from the matrix given in the book because 
\Sage{} has vectors multiply from the left).
The plane line $y=3x+2$ is this set.
\begin{equation*}
  \ell=\set{\colvec{x \\ y}=\colvec{3 \\ 1}\cdot s+\colvec{2 \\ 0}\suchthat s\in\Re}
\end{equation*}
The rotated line is this set.
\begin{equation*}
  h(\ell)
  =
  \set{\colvec{x \\ y}=\frac{1}{2}\colvec{3\sqrt{3}+1 \\ -3+\sqrt{3}}\cdot s
                                  +\colvec{\sqrt{3} \\ 1}\suchthat s\in\Re}
\end{equation*}
\begin{sageoutput}
s = var('s')
ell = parametric_plot((3*s+2,1*s), (s, -10, 10))
ell.set_axes_range(-4, 4, -4, 4)
ell.save("sageoutput/plot_action0.png", figsize=2.5, fontsize=7)
\end{sageoutput}
\begin{center}
  \includegraphics{sageoutput/plot_action0.png}
\end{center}



So the routine finds the effect of the map 
\begin{equation*}
  \rowvec{x &y}
  \begin{mat}
    a &b  \\
    c &d
  \end{mat}
\end{equation*}
on the four corners of the 
square 
\begin{equation*}
  \colvec{0 \\ 0}\mapsunder{t}\colvec{0 \\ 0}\quad
  \colvec{1 \\ 0}\mapsunder{t}\colvec{a \\ b}\quad
  \colvec{1 \\ 1}\mapsunder{t}\colvec{a+c \\ b+d}\quad
  \colvec{0 \\ 1}\mapsunder{t}\colvec{c \\ d}
\end{equation*}
and plots four line segments.




%========================================
\section{The unit square}
Consider a linear transformation $\map{t}{\Re^2}{\Re^2}$ applied to
this unit square resting in the first quadrant.
\begin{center}
  \includegraphics{sageoutput/plot_action1.png}
\end{center}
This code generates that picture.
\begin{sageoutput}
load "plot_action.sage"
p = plot_square_action(1,0,0,1) 
p.set_axes_range(-0.5, 1.5, -0.5, 1.5) 
p.save("sageoutput/plot_action1.png")
\end{sageoutput}
\noindent The routine \inlinecode{plot_square_action(a, b, c, d)}
(whose code is at the end of this chapter)
shows the effect of the matrix
\begin{equation*}
  \rep{t}{\stdbasis_2,\stdbasis_2}=
  \begin{mat}
    a &b \\
    c &d
  \end{mat}
\end{equation*}
on a unit square.
The code above gives this routine 
the identity matrix, so it plots the square unchanged.

This routine is based on the observation that linear maps send lines to lines.
Consider the effect of the map $\map{h}{\Re^n}{\Re^m}$ 
on the line $\ell = \set{\vec{v}=\vec{m}\cdot s +\vec{b}\suchthat s\in\Re}$.
The image is the set 
$\set{h(\vec{m}s+\vec{b})\suchthat s\in\Re}
=\set{h(\vec{m})s+h(\vec{b})\suchthat s\in\Re}$, which is a line in $\Re^m$. 
So the routine finds the effect of the map 
\begin{equation*}
  \rowvec{x &y}
  \begin{mat}
    a &b  \\
    c &d
  \end{mat}
\end{equation*}
on the four corners of the 
square 
\begin{equation*}
  \colvec{0 \\ 0}\mapsunder{t}\colvec{0 \\ 0}\quad
  \colvec{1 \\ 0}\mapsunder{t}\colvec{a \\ b}\quad
  \colvec{1 \\ 1}\mapsunder{t}\colvec{a+c \\ b+d}\quad
  \colvec{0 \\ 1}\mapsunder{t}\colvec{c \\ d}
\end{equation*}
and plots four line segments.

For example, this code
\begin{sageoutput}
load "plot_action.sage"
q = plot_square_action(1,0,0,1) 
p = plot_square_action(1,2,3,4) 
q.set_axes_range(-6, 6, -1, 6) 
p.set_axes_range(-6, 6, -1, 6) 
q.save("sageoutput/plot_action2.png")
p.save("sageoutput/plot_action3.png")
\end{sageoutput}
\noindent generates these pictures showing the effect of the 
matrix.\footnote{The remaining examples in this chapter omit the 
fiddly lines that load, save, set the axis ranges, etc.}
\begin{equation*}
  \vcenteredhbox{\includegraphics{sageoutput/plot_action2.png}}
  \quad\mapsunder{\big (\begin{smallmatrix} 1 &2 \\ 3 &4 \end{smallmatrix}\big )}\quad
  \vcenteredhbox{\includegraphics{sageoutput/plot_action3.png}}
\end{equation*}
The colors are there to show that transformations can change
orientations.
Suppose that we take the natural order of colors to be red, orange, 
green, and then blue.
Then the domain square is a counterclockwise shape, while the transformed
square is clockwise.

Transformation acts uniformily?


Recall that any matrix $T$ factors as $H=PBQ$, 
where $P$ and $Q$ are nonsingular and $B$ is a partial-identity matrix.
Recall also that nonsingular matrices
factor into elementary matrices
$PBQ=T_nT_{n-1}\cdots T_jBT_{j-1}\cdots T_1$,
which are matrices that
come from the identity $I$ after one Gaussian step
\begin{equation*}
  I\grstep{k\rho_i}M_i(k) 
  \qquad 
  I\grstep{\rho_i\leftrightarrow\rho_j}P_{i,j}  
  \qquad
  I\grstep{k\rho_i+\rho_j}C_{i,j}(k) 
\end{equation*}
for $i\neq j$, $k\neq 0$.
So if we understand the effect of a linear map described
by a partial-identity matrix and the effect of the linear maps
described by the elementary matrices then we will in some sense
understand the effect of any linear map.
(To understand them we mean to give a description of their geometric effect;
the pictures below stick to transformations of $\Re^2$ for ease of drawing
but the principles extend for maps from any $\Re^n$ to any $\Re^m$.)







\section{Maps preserve lines through the origin}





\lstinputlisting{plot_action.sage}

\begin{center}
  \includegraphics{sageoutput/plot_action1.png}
\end{center}

\endinput


TODO:
