\chapter{Singular Value Decomposition}

Recall that a line through the origin in $\Re^n$ is 
$\set{r\cdot \vec{v}\suchthat r\in\Re}$. 
One of the defining properties of a linear map is that 
$h(r\cdot\vec{v})=r\cdot h(\vec{v})$.
So the action of $h$ on any 
line through the origin
is determined by the action
of $h$ on any nonzero vector in that line.

For instance consider the line~$y=2x$ in the plane.
\begin{equation*}
  \set{r\cdot\colvec{1 \\ 2}\suchthat r\in\Re}
\end{equation*}
If $\map{t}{\Re^2}{\Re^2}$ is represented by the matrix
\begin{equation*}
  \rep{t}{\stdbasis_2,\stdbasis_2}
  =
  \begin{mat}
    1 &2 \\
    3 &4
  \end{mat}
\end{equation*}
then here is the effect of $t$ on one vector in the line.
\begin{equation*}
  \colvec{1 \\ 2}\mapsunder{t}\colvec{7 \\ 10}
\end{equation*}
And here is the effect of $t$ on other members of the line.
\begin{equation*}
  \colvec{2 \\ 4}\mapsunder{t}\colvec{14 \\ 20}
  \qquad
  \colvec{-3 \\ -6}\mapsunder{t}\colvec{-21 \\ -30}
  \qquad
  \colvec{r \\ 2r}\mapsunder{t}\colvec{7r \\ 10r}
\end{equation*}
The map $t$'s action on the line is uniform.


\section{Unit circle}
The above means that one way to describe a transformation's action is to pick 
one nonzero element from each line through the origin and describe where
the transformation maps those elements.
An easy way to select one nonzero element from each line through the
origin is to take the upper half unit circle.
\begin{equation*}
  U=\set{\colvec{x \\ y}
         \suchthat 
         \text{$x=\cos(t)$, $y=\sin(t)$, $0\leq t<\pi$}}
  \qquad
  \vcenteredhbox{\includegraphics{sageoutput/plot_action10.png}}  
\end{equation*}
We produce the above 
graph by using a routine that draws the effect of an arbitrary 
matrix on the unit circle, and giving that routine the identity matrix. 
The colors will keep track the directionality in the set of
before and after pictures below.
\begin{sageoutput}
load "plot_action.sage"
p = plot_circle_action(1,0,0,1)  # identity matrix
p.set_axes_range(-1.5, 1.5, -0.5, 1.5) 
p.save("sageoutput/plot_action10.png")
\end{sageoutput}

Here 
is the transformation doubles the $x$~components of all vectors. 
\begin{sageoutput}[d,0,4;d,5,7]
load "plot_action.sage"
q = plot_circle_action(1,0,0,1) 
q.set_axes_range(-2, 2, -1, 2) 
q.save("sageoutput/plot_action11.png")
p = plot_circle_action(2,0,0,1) 
p.set_axes_range(-2, 2, -1, 2) 
p.save("sageoutput/plot_action11a.png")
\end{sageoutput}
\begin{equation*}
  \vcenteredhbox{\includegraphics{sageoutput/plot_action11.png}}
  \quad\mapsunder{\big (\begin{smallmatrix} 2 &0 \\ 0 &1 \end{smallmatrix}\big )}\quad
  \vcenteredhbox{\includegraphics{sageoutput/plot_action11a.png}}
\end{equation*}
And here is the
transformation that triples the $y$~components and multiplies 
$x$~components by $-1$. 
Note that it changes the orientation.
\begin{sageoutput}[d,0,4;d,5,7]
load "plot_action.sage"
q = plot_circle_action(1,0,0,1) 
q.set_axes_range(-2, 2, -4, 4) 
q.save("sageoutput/plot_action12.png")
p = plot_circle_action(-1,0,0,3) 
p.set_axes_range(-2, 2, -4, 4) 
p.save("sageoutput/plot_action12a.png")
\end{sageoutput}
\begin{equation*}
  \vcenteredhbox{\includegraphics{sageoutput/plot_action12.png}}
  \quad\mapsunder{\big (\begin{smallmatrix} -1 &0 \\ 0 &3 \end{smallmatrix}\big )}\quad
  \vcenteredhbox{\includegraphics{sageoutput/plot_action12a.png}}
\end{equation*}

Here is the first skew
transformation. 
\begin{sageoutput}[d,0,4;d,5,7]
load "plot_action.sage"
q = plot_circle_action(1,0,0,1) 
q.set_axes_range(-2, 2, -3, 3) 
q.save("sageoutput/plot_action13.png")
p = plot_circle_action(1,0,2,1) 
p.set_axes_range(-2, 2, -3, 3) 
p.save("sageoutput/plot_action13a.png")
\end{sageoutput}
\begin{equation*}
  \vcenteredhbox{\includegraphics{sageoutput/plot_action13.png}}
  \quad\mapsunder{\big (\begin{smallmatrix} 1 &0 \\ 2 &1 \end{smallmatrix}\big )}\quad
  \vcenteredhbox{\includegraphics{sageoutput/plot_action13a.png}}
\end{equation*}
\noindent Here is the second skew
transformation. 
\begin{sageoutput}[d,0,4;d,5,7]
load "plot_action.sage"
q = plot_circle_action(1,0,0,1) 
q.set_axes_range(-3, 3, -2, 2) 
q.save("sageoutput/plot_action14.png")
p = plot_circle_action(1,3,0,1) 
p.set_axes_range(-3, 3, -2, 2) 
p.save("sageoutput/plot_action14a.png")
\end{sageoutput}
\begin{equation*}
  \vcenteredhbox{\includegraphics{sageoutput/plot_action14.png}}
  \quad\mapsunder{\big (\begin{smallmatrix} 1 &3 \\ 0 &1 \end{smallmatrix}\big )}\quad
  \vcenteredhbox{\includegraphics{sageoutput/plot_action14a.png}}
\end{equation*}

And here is the generic map.
\begin{sageoutput}[d,0,4;d,5,7]
load "plot_action.sage"
q = plot_circle_action(1,0,0,1) 
q.set_axes_range(-2, 2, -4, 6) 
q.save("sageoutput/plot_action15.png")
p = plot_circle_action(1,2,3,4) 
p.set_axes_range(-2, 2, -4, 6) 
p.save("sageoutput/plot_action15a.png")
\end{sageoutput}
\begin{equation*}
  \vcenteredhbox{\includegraphics{sageoutput/plot_action15.png}}
  \quad\mapsunder{\big (\begin{smallmatrix} 1 &3 \\ 0 &1 \end{smallmatrix}\big )}\quad
  \vcenteredhbox{\includegraphics{sageoutput/plot_action15a.png}}
\end{equation*}



\section{Circles map to ellipses}
These pictures show the circles mapping to ellipses.
Recall that an ellipse in $\Re^2$ has a \textit{major axis} axis, 
the longer one, and a 
\textit{minor axis}.\footnote{Ellipses have a couple of special cases.
If the two axes have the same length then the ellipse is a circle.
If an axis has length zero then the ellipse is a line segment.}
We write $\sigma_1$ for half the length of the major axis, 
the distance from the center to the furthest-away point on the ellipse.
We write $\sigma_2$ for the length of the semi-minor axis.
These two axes are orthogonal.
\begin{sageoutput}
plot.options['axes_pad'] = 0.05
plot.options['fontsize'] = 7
plot.options['dpi'] = 500
plot.options['aspect_ratio'] = 1
sigma_1=3
sigma_2=1
E = ellipse((0,0), sigma_1, sigma_2)
E.save("sageoutput/svd01.png", figsize=3.5)
\end{sageoutput}
\begin{center}
  \includegraphics{sageoutput/svd01.png}
\end{center}

% transformationellipses map to ellipses and we have seen the special case
% of circles.)\footnote{The proof of this is harder than you may expect.
% Assuming $Ax^2+Bxy+Cy^2+Dx+Ey+F=0$  and
% substituting $a\hat{x}+c\hat{y}$ and~$b\hat{x}+d\hat{y}$ for $x$ and~$y$
% easily gives that the inverse of a conic is a conic.
% But assuming that the conic is an ellipse, so that the discriminant
% $B^2-4AC$ is less than~$0$,
% does not readily give that the discriminant of the inverse is less than~$0$.}

In general, under any linear map $\map{t}{\Re^n}{\Re^m}$ the 
unit sphere maps to a hyperellipse.
This is a version of the \textit{Singular Value Decomposition} of
matrices:
for any linear map $\map{t}{\Re^m}{\Re^n}$ there are bases
$B=\sequence{\vec{\beta}_1,\ldots,\vec{\beta}_m}$ for the domain and
$D=\sequence{\vec{\delta}_1,\ldots,\vec{\delta}_n}$ for the codomain
such that $t(\vec{\beta}_i)=\sigma_i\vec{\delta}_i$.

Here is the example of the generic $\nbyn{2}$~matrix.
\begin{sageoutput}
plot.options['axes_pad'] = 0.05
plot.options['fontsize'] = 7
plot.options['dpi'] = 500
plot.options['aspect_ratio'] = 1
M = matrix(RDF, [[1, 2], [3, 4]])
M.SVD()
beta_1 = vector(RR, [-0.404553584834, -0.914514295677])
beta_2 = vector(RR, [-0.914514295677, 0.404553584834])
delta_1 = vector(RR, [-0.576048436766, -0.81741556047])
delta_2 = vector(RR, [0.81741556047, -0.576048436766])
M*beta_1
M*beta_2
C = circle((0,0), 1)
P = C + plot(beta_1) + plot(beta_2)
P.save("sageoutput/svd02a.png", figsize=2)
Q = C + plot(M*beta_1,color='red') + plot(delta_1) 
Q = Q + plot(M*beta_2,color='red') + plot(delta_2)
Q.save("sageoutput/svd02b.png", figsize=4.5)
\end{sageoutput}
\begin{center}
  \vcenteredhbox{\includegraphics{sageoutput/svd02a.png}}
  \qquad
  \vcenteredhbox{\includegraphics{sageoutput/svd02b.png}}
\end{center}



\begin{center}
  \includegraphics{asy/ellipsoid.pdf}
\end{center}



\section{Source of plot\_action.sage}
\lstinputlisting{plot_action.sage}

\endinput


TODO:
