\chapter{Singular Value Decomposition}


Recall from the Geometry chapter that under a linear
map $\map{h}{\Re^d}{\Re^c}$ lines map to lines.
\begin{equation*}
\set{\vec{v}=\vec{m}\cdot s +\vec{b}\suchthat s\in\Re}
\quad\longrightarrow\quad
\set{h(\vec{m})\cdot s+h(\vec{b})\suchthat s\in\Re}  
\end{equation*}
A special case of that happens when $\vec{b}=\zero$.
Because a linear map sends the zero vector in the domain to the 
zero vector in the codomain, $h(\vec{b})=\zero$, so for any linear map
on real spaces,
a line throught the origin maps to a line through the origin.



\section{The upper half circle}
One of the defining properties of a linear map is that 
$h(r\cdot\vec{v})=r\cdot h(\vec{v})$.
This equation means that the action of $h$ on any line through the origin
$\ell=\set{r\cdot \vec{v}\suchthat r\in\Re}$ is determined by the action
of $h$ on any nonzero vector in that line.

For instance we can describe the line~$y=2x$ in this way.
\begin{equation*}
  \set{r\cdot\colvec{1 \\ 2}\suchthat r\in\Re}
\end{equation*}
If $\map{t}{\Re^2}{\Re^2}$ is represented by the matrix
\begin{equation*}
  \rep{t}{\stdbasis_2,stdbasis_2}
  =
  \begin{mat}
    1 &2 \\
    3 &4
  \end{mat}
\end{equation*}
then here is the effect of $t$ on one vector in the line.
\begin{equation*}
  \colvec{1 \\ 2}\mapsunder{t}\colvec{7 \\ 10}
\end{equation*}
And here is the effect of $t$ on other members of the line.
\begin{equation*}
  \colvec{2 \\ 4}\mapsunder{t}\colvec{14 \\ 20}
  \qquad
  \colvec{-3 \\ -6}\mapsunder{t}\colvec{-21 \\ -30}
  \qquad
  \colvec{r \\ 2r}\mapsunder{t}\colvec{7r \\ 10r}
\end{equation*}
So $t$'s action on the entire line is uniform.

This means that one way to describe a transformation's action is to pick 
one nonzero element from each line through the origin and describe where
the transformation maps those elements.
An easy way to select one nonzero element from each line through the
origin is to take the upper half circle.
\begin{equation*}
  U=\set{\colvec{x \\ y}
         \suchthat 
         \text{$x=\cos(t)$, $y=\sin(t)$, $0\leq t<\pi$}}
  \qquad
  \vcenteredhbox{\includegraphics{sageoutput/plot_action10.png}}  
\end{equation*}
As with the unit square that began this chapter we produce this 
graph by using a routine that draws the effect of an arbitrary 
matrix on the circle, and giving that routine the identity matrix. 
(As earlier, we will use the colors to track the directionality in a set of
before and after pictures.)
\begin{sageoutput}
load "plot_action.sage"
p = plot_circle_action(1,0,0,1)  # identity matrix
p.set_axes_range(-1.5, 1.5, -0.5, 1.5) 
p.save("sageoutput/plot_action10.png")
\end{sageoutput}

Using this visualization approach, here 
is the transformation doubles the $x$~components of all vectors. 
\begin{sageoutput}[d,0,4;d,5,7]
load "plot_action.sage"
q = plot_circle_action(1,0,0,1) 
q.set_axes_range(-2, 2, -1, 2) 
q.save("sageoutput/plot_action11.png")
p = plot_circle_action(2,0,0,1) 
p.set_axes_range(-2, 2, -1, 2) 
p.save("sageoutput/plot_action11a.png")
\end{sageoutput}
\begin{equation*}
  \vcenteredhbox{\includegraphics{sageoutput/plot_action11.png}}
  \quad\mapsunder{\big (\begin{smallmatrix} 2 &0 \\ 0 &1 \end{smallmatrix}\big )}\quad
  \vcenteredhbox{\includegraphics{sageoutput/plot_action11a.png}}
\end{equation*}
And here is the
transformation that triples the $y$~components and multiplies 
$x$~components by $-1$. 
Note that it changes the orientation.
\begin{sageoutput}[d,0,4;d,5,7]
load "plot_action.sage"
q = plot_circle_action(1,0,0,1) 
q.set_axes_range(-2, 2, -4, 4) 
q.save("sageoutput/plot_action12.png")
p = plot_circle_action(-1,0,0,3) 
p.set_axes_range(-2, 2, -4, 4) 
p.save("sageoutput/plot_action12a.png")
\end{sageoutput}
\begin{equation*}
  \vcenteredhbox{\includegraphics{sageoutput/plot_action12.png}}
  \quad\mapsunder{\big (\begin{smallmatrix} -1 &0 \\ 0 &3 \end{smallmatrix}\big )}\quad
  \vcenteredhbox{\includegraphics{sageoutput/plot_action12a.png}}
\end{equation*}

Here is the first skew
transformation. 
\begin{sageoutput}[d,0,4;d,5,7]
load "plot_action.sage"
q = plot_circle_action(1,0,0,1) 
q.set_axes_range(-2, 2, -3, 3) 
q.save("sageoutput/plot_action13.png")
p = plot_circle_action(1,0,2,1) 
p.set_axes_range(-2, 2, -3, 3) 
p.save("sageoutput/plot_action13a.png")
\end{sageoutput}
\begin{equation*}
  \vcenteredhbox{\includegraphics{sageoutput/plot_action13.png}}
  \quad\mapsunder{\big (\begin{smallmatrix} 1 &0 \\ 2 &1 \end{smallmatrix}\big )}\quad
  \vcenteredhbox{\includegraphics{sageoutput/plot_action13a.png}}
\end{equation*}
\noindent Here is the second skew
transformation. 
\begin{sageoutput}[d,0,4;d,5,7]
load "plot_action.sage"
q = plot_circle_action(1,0,0,1) 
q.set_axes_range(-3, 3, -2, 2) 
q.save("sageoutput/plot_action14.png")
p = plot_circle_action(1,3,0,1) 
p.set_axes_range(-3, 3, -2, 2) 
p.save("sageoutput/plot_action14a.png")
\end{sageoutput}
\begin{equation*}
  \vcenteredhbox{\includegraphics{sageoutput/plot_action14.png}}
  \quad\mapsunder{\big (\begin{smallmatrix} 1 &3 \\ 0 &1 \end{smallmatrix}\big )}\quad
  \vcenteredhbox{\includegraphics{sageoutput/plot_action14a.png}}
\end{equation*}

And here is the generic map.
\begin{sageoutput}[d,0,4;d,5,7]
load "plot_action.sage"
q = plot_circle_action(1,0,0,1) 
q.set_axes_range(-2, 2, -4, 6) 
q.save("sageoutput/plot_action15.png")
p = plot_circle_action(1,2,3,4) 
p.set_axes_range(-2, 2, -4, 6) 
p.save("sageoutput/plot_action15a.png")
\end{sageoutput}
\begin{equation*}
  \vcenteredhbox{\includegraphics{sageoutput/plot_action15.png}}
  \quad\mapsunder{\big (\begin{smallmatrix} 1 &3 \\ 0 &1 \end{smallmatrix}\big )}\quad
  \vcenteredhbox{\includegraphics{sageoutput/plot_action15a.png}}
\end{equation*}



\section{Circles map to ellipses}
These circles maps to ellipses.
In general, ellipses map to ellipses, so we have seen the special case
of circles.
The argument is here.
Consider a vector 
\begin{equation*}
  \vec{v}=\colvec{x \\ y}
\end{equation*}
that lies on an ellipse.
We can parametrize to get $x(t)=m\cos(t)$, $y(t)=n\sin(t)$ with 
$0\leq t< 2\pi$.





\section{Source of plot\_action.sage}
\lstinputlisting{plot_action.sage}

\endinput


TODO:
