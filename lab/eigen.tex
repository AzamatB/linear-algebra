\chapter{Eigenvalues}

Consider again the action of a skew map on the unit circle.
\begin{sageoutput}[d,0,4;d,5,7]
load "plot_action.sage"  
q = plot_circle_action(1,0,0,1) 
q.set_axes_range(-3, 3, -3, 3) 
q.save("sageoutput/plot_action100a.png")
p = plot_circle_action(2,2,0,2) 
p.set_axes_range(-3, 3, -3, 3) 
p.save("sageoutput/plot_action100b.png")
\end{sageoutput}
\begin{equation*}
  \vcenteredhbox{\includegraphics{sageoutput/plot_action100a.png}}
  \quad\mapsunder{\big (\begin{smallmatrix} 2 &2 \\ 0 &2 \end{smallmatrix}\big )}\quad
  \vcenteredhbox{\includegraphics{sageoutput/plot_action100b.png}}
  \tag{$*$}
\end{equation*}
There are twelve colors;
on the left they begin with red at $(x,y)=(1,0)$
while on the right they begin with red at $(1,2)$.
Here are the before and after vectors for that red point.
\begin{sageoutput}[d,0,1;d,2,4]
load "plot_action.sage"  
p = plot_before_after_action(2,2,0,2, [(1,0)], ['red']) 
p.set_axes_range(-0.5, 2, -0.5, 2) 
p.save("sageoutput/plot_action101.png", ticks=([1,2],[1,2]))
\end{sageoutput}
\begin{equation*}
  \vcenteredhbox{\includegraphics{sageoutput/plot_action101.png}}
\end{equation*}
The matrix's action on the red vector is to both resize and rotate.
\begin{sageoutput}
v = vector(RR, [1,0])
M = matrix(RR, [[2, 2], [0, 2]])
w = v*M
w.norm(), v.norm() 
angle = arccos(w*v/(w.norm()*v.norm())) 
angle 
\end{sageoutput}
\noindent The map resizes and rotates the vector at the other end of the 
half circle, at $(-1,0)$, by the
same amount as it does the vector ending at $(1,0)$. 
This is because a linear map has the same action on 
all vectors that lie on the same line through the origin.
% \begin{sageoutput}
% v = vector(RR, [-1,0])
% M = matrix(RR, [[2, 2], [0, 2]])
% w = v*M
% w.norm(), v.norm() 
% angle = arccos(w*v/(w.norm()*v.norm())) 
% angle 
% \end{sageoutput}
But intermediate vectors are not resized and rotated by the same
amount.
Here is the next of the twelve points on the half-circle.
\begin{sageoutput}
v = vector(RR, [cos(pi/12),sin(pi/12)])
M = matrix(RR, [[2, 2], [0, 2]])
w = v*M
w.norm(), v.norm() 
angle = arccos(w*v/(w.norm()*v.norm())) 
angle 
\end{sageoutput}

In Chapter~\ref{chap:SingularValueDecomposition} on 
Singular Value Decomposition we studied how
transformations resize vectors; the maximum resizing factors
are the singular values.
In this chapter we consider the turning.

\Sage{} will show us the rotation of each of the twelve points 
under the transformation.
At the end of this chapter is the source for a routine 
\inlinecode{find_angles}
that takes in the matrix entries and a list of input vectors, then
for each vector finds the angle between itself and its image and
returns a list of those angles.
Also there is the source of \inlinecode{circle_points} that takes in a
number and returns that many vectors evenly spaced around the half-circle. 
We can use those two to make a graph.
\begin{sageoutput}[d,0,1]
load "plot_action.sage"  
NUM_CIRCLE_PTS
angles = find_angles(2,2,0,2,NUM_CIRCLE_PTS)
angles[0:2]
ticks = ([0,pi/4,pi/2,3*pi/4,pi], [0,pi/2,pi])
p = scatter_plot(angles, marker='.', edgecolor='blue', ticks=ticks)
p.set_axes_range(0, pi, 0, pi) 
p.save("sageoutput/plot_action103.png", figsize=2.5, tick_formatter=pi)
\end{sageoutput}
\begin{center}
  \vcenteredhbox{\includegraphics{sageoutput/plot_action103.png}}
\end{center}
The most interesting point on this graph is where the angle is $0$\Dash
for this transformation there is a input vector that does not turn at all.
It is the vector halfway around the circle, with endpoint~$(0,1)$. 
We can see it on the graphic above labelled~($*$), where the green input
vector lies on the $y$-axis and so does its green image.
It is resized but it isn't rotated.

\subsection{Generic matrix}
We can do the same analysis for our usual generic $\nbyn{2}$~matrix.
Here is its action on the unit circle.
\begin{sageoutput}[d,0,4;d,5,7]
load "plot_action.sage"
q = plot_circle_action(1,0,0,1) 
q.set_axes_range(-3, 3, -5, 5) 
q.save("sageoutput/plot_action110a.png")
p = plot_circle_action(1,2,3,4) 
p.set_axes_range(-3, 3, -5, 5) 
p.save("sageoutput/plot_action110b.png")
\end{sageoutput}
\begin{equation*}
  \vcenteredhbox{\includegraphics{sageoutput/plot_action110a.png}}
  \quad\mapsunder{\big (\begin{smallmatrix} 1 &2 \\ 3 &4 \end{smallmatrix}\big )}\quad
  \vcenteredhbox{\includegraphics{sageoutput/plot_action110b.png}}
\end{equation*}
This graphs the angle between the each before arrow and its associated after
arrows.
\begin{sageoutput}[d,0,1]
load "plot_action.sage"  
# vecs = circle_points(1000)
angles = find_angles(1,2,3,4,1000)
ticks = ([0,pi/4,pi/2,3*pi/4,pi], [0,pi/2,pi])
p = scatter_plot(angles, marker='.', edgecolor='blue', ticks=ticks)
p.save("sageoutput/plot_action113.png", figsize=2.5, tick_formatter=pi)
\end{sageoutput}
\begin{center}
  \vcenteredhbox{\includegraphics{sageoutput/plot_action113.png}}
\end{center}
This graph has two interesting points, where $y=0$ and where 
$y=\pi$.
In the second case
the angle of $\pi$~radians means that the vector remains on the
same line through the origin but is multiplied by a negative scalar.
In both places the vector is not turned at all, only resized.

Vectors that are not turned by a transformation~$t$ but are only resized 
are \textit{eigenvectors} for~$t$, and the amount by which it is 
resized is the \textit{eigenvalue} associated with that vector.
For a fixed eigenvalue~$\lambda$, the set of vectors associated with
it is an \textit{eigenspace}.
\begin{sageoutput}[s,3,59,13]
M = matrix(RDF, [[1, 2], [3, 4]])
evs = M.eigenvectors_left()
evs
evs[0] 
evs[1]
\end{sageoutput}
Explain the \inlinecode{RDF} and the \inlinecode{left}.

The first information that \Sage{} gives, \inlinecode{evs[0]}, 
is that the set of vectors with a basis consisting of the single unit
vector with endpoint $(-0.909376709132, 0.415973557919)$ is an
eigenspace associated with the eigenvalue $-0.372281323269$.
\Sage{} also finds with \inlinecode{evs[1]} that a second eigenspace
has a basis consisting of the vector with endpoint
$(-0.565767464969, -0.824564840132)$, associated with the
eigenvalue~$5.37228132327$. 

We can draw those two as before and after pictures.
\begin{sageoutput}[d,0,3]
load "plot_action.sage"  
M = matrix(RDF, [[1, 2], [3, 4]])
evs = M.eigenvectors_left()  
p1 = plot_before_after_action(1,2,3,4, [evs[0][1][0]], ['red']) 
p2 = plot_before_after_action(1,2,3,4, [evs[1][1][0]], ['blue']) 
p = p1+p2
p.set_axes_range(-4, 1, -4, 1) 
p.save("sageoutput/plot_action114.png", figsize=3.5)
\end{sageoutput}
\noindent
Each before vector is a unit vector, so it is easy to pick it from among the
two.
The after vector is scaled from the before vector.
\begin{equation*}
  \vcenteredhbox{\includegraphics{sageoutput/plot_action114.png}}
\end{equation*}


\subsection{Characteristic polynomial}



\subsection{Attraction}
Another natural picture to draw is the action of linear maps on 
some points.

We start with the action of the linear transformation $\map{t}{\Re^2}{\Re^2}$
represented with respect to the standard bases by the generic matrix with
entries $1$, $2$, $3$, and $4$.
\begin{sageoutput}
load "plot_action.sage"
plot.options['figsize']=4.5
g = point_grid(3, 3)
p = plot_point_action(1, 2, 3, 4, g)
p.save("sageoutput/plot_action110.png")  
\end{sageoutput}
\begin{center}
  \includegraphics{"sageoutput/plot_action110.png"}
\end{center}



\section{Source of find\_angles}
This routine uses a formula for the angel between two vectors that 
always gives a positive value, that is, it is the unoriented angle.
That's fine for purpose here, which is to use the graph to 
roughly locate places where the action of the matrix does not turn the
vector. 
\lstinputlisting[firstline=200, lastline=221]{plot_action.sage}


\section{Source of plot\_before\_after\_action}
The only perhaps unexpected point in this routine, and the helper routine
before it, is that if the vector is not mapped very far then the helper
routine does not show an arrow but instead shows a circle.
\lstinputlisting[firstline=157, lastline=198]{plot_action.sage}


\endinput

TODO

1) Does python intro show 
   > x, y = 5, 7
construct?