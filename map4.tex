% Chapter 3, Section 4 _Linear Algebra_ Jim Hefferon
%  http://joshua.smcvt.edu/linalg.html
%  2001-Jun-12
\section{Matrix Operations}
The prior section shows how matrices represent linear maps.
When we see a new idea, a good strategy  
is to explore how it interacts with things
that we already understand.
In the first subsection below
we will see how the representation of a scalar product $r\cdot f$ relates 
to the representation of $f$, and
also how the representation of the 
sum of two maps $f+g$ relates to the representations of $f$ and $g$.
In the later subsections we will explore the representation of 
linear map composition and inverse.




\subsection{Sums and Scalar Products}

We start with an example showing the relationship between the representation
of a function and the representation of a scalar multiple of that function.

\begin{example}
Let \( \map{f}{V}{W} \) be a linear function 
represented with respect to some bases by this matrix.
\begin{equation*}
  \rep{f}{B,D}
  =
    \begin{mat}[r]
      1  &0  \\
      1  &1
    \end{mat}
\end{equation*}
Consider the scalar multiple map $\map{5f}{V}{W}$.
We want to see how to compute $\rep{5f}{B,D}$ from $\rep{f}{B,D}$.

The difference between the functions is that
if $f$ takes $\vec{v}\mapsto \vec{w}$ 
then $5f$ takes $\vec{v}\mapsto 5\vec{w}$.
So consider the representations of the domain and codomain vectors that are
associated by $f$.
\begin{equation*}
  \rep{\vec{v}}{B}
  =\colvec{v_1 \\ v_2}
  \qquad
  \rep{\vec{w}}{D}
  =\colvec{w_1 \\ w_2}
\end{equation*}
Where the basis $D$ is $\sequence{\vec{\delta}_1,\vec{\delta}_2}$
the representation above 
says that $\vec{w}=w_1\vec{\delta}_1+w_2\vec{\delta}_2$.
Since
$5\vec{w}=5\cdot(w_1\vec{\delta}_1+w_2\vec{\delta}_2)
=(5w_1)\vec{\delta}_1+(5w_2)\vec{\delta}_2$
we have that $5f$ associates $\vec{v}$ with the vector having this 
representation.
\begin{equation*}
  \rep{5\vec{w}}{D}
  =\colvec{5w_1 \\ 5w_2}
\end{equation*}
Changing the map from $f$ to $5f$ has the effect 
of changing the representation of the codomain vector by multiplying its
entries by $5$.

So \( \rep{5f}{B,D} \) is this matrix.
\begin{equation*}
  \rep{5f}{B,D}\cdot\colvec{v_1 \\ v_2}
  =
   \colvec{5v_1 \\ 5v_1+5v_2}
  \qquad  
  \rep{5f}{B,D}
  =
    \begin{mat}[r]
      5  &0  \\
      5  &5
    \end{mat}
\end{equation*}
Therefore going from the matrix representing $f$ to 
the matrix representing $5f$ just means multiplying all the entries 
by $5$.
\end{example}

Consider also how to compute the representation of the sum of two maps.

\begin{example}
Suppose that two linear maps with the same domain and codomain 
\( \map{f,g}{\Re^2}{\Re^2} \) are represented 
with respect to some bases $B$ and~$D$ by these matrices.
\begin{equation*}
  \rep{f}{B,D}=
    \begin{mat}
       1  &3  \\
       2  &0 
    \end{mat}
  \qquad
  \rep{g}{B,D}=
    \begin{mat}
      -2  &-1 \\
       2  &4
    \end{mat}
\end{equation*}
Recall the definition of the sum of two functions: if
$f$~does $\vec{v}\mapsto\vec{u}$ and 
$g$~does $\vec{v}\mapsto\vec{w}$ then
$f+g$ is the function $\vec{v}\mapsto\vec{u}+\vec{w}$.
If these are the representations of the vectors 
\begin{equation*}
  \rep{\vec{v}}{B}=\colvec{v_1 \\ v_2}
  \quad
  \rep{\vec{u}}{D}=\colvec{u_1 \\ u_2}
  \quad
  \rep{\vec{w}}{D}=\colvec{w_1 \\ w_2}
\end{equation*}
where $D=\sequence{\vec{\delta}_1,\vec{\delta}_2}$ then we have
$\vec{u}+\vec{w}=(u_1\vec{\delta}_1+u_2\vec{\delta}_2)
  +(w_1\vec{\delta}_1+w_2\vec{\delta}_2)
  =(u_1+w_1)\vec{\delta}_1+(u_2+w_2)\vec{\delta}_2$
and so this is the representation of the vector sum.
\begin{equation*}
  \rep{\vec{u}+\vec{w}}{D}=\colvec{u_1+w_1 \\ u_2+w_2}
\end{equation*}
% Thus the effect of adding the functions on the representations
% of the vectors is to add them entry-by-entry.

Since these represent the actions of $f$ and~$g$
\begin{equation*}
    \begin{mat}
       1  &3  \\
       2  &0 
    \end{mat}
    \colvec{v_1 \\ v_2}
    =\colvec{v_1+3v_2   \\ 2v_1}
  \qquad
    \begin{mat}
      -2  &-1 \\
       2  &4
    \end{mat}
    \colvec{v_1 \\ v_2}
    =\colvec{-2v_1-v_2   \\ 2v_1+4v_2}  
\end{equation*}
this represents the action of $f+g$.
\begin{equation*}
  \rep{f+g}{B,D}\cdot\colvec{v_1 \\ v_2}
  =
  \colvec{-v_1+2v_2 \\ 4v_1+4v_2}
\end{equation*}
Therefore, we 
compute the matrix representing the function sum by adding the entries of 
the two matrices representing the functions.
\begin{equation*} 
    \rep{f+g}{B,D}
    \begin{mat}
       -1  &2  \\
        4  &4 
    \end{mat}
\end{equation*}
\end{example}

\begin{definition} \label{def:SumScalarProdMats}
%<*df:SumScalarProdMats>
The \definend{scalar multiple}\index{scalar multiple!matrix}%
\index{matrix!scalar multiple} of a matrix is the
result of entry-by-entry scalar multiplication.
The \definend{sum\/}\index{sum!of matrices}\index{matrix!sum} of two
same-sized matrices is their entry-by-entry sum.
%</df:SumScalarProdMats>
\end{definition}

These operations extend the first chapter's addition and scalar
multiplication operations on vectors.

\begin{theorem}  \label{th:MatOpsRepMapOps}
%<*th:MatOpsRepMapOps>
Let \( \map{h,g}{V}{W} \) be linear maps represented with respect to
bases \( B,D \) by the matrices \( H \) and \( G \) and let $r$ be a scalar.
Then with respect to \( B,D \) the map 
\( \map{r\cdot h}{V}{W} \) is represented by
\( rH \)
and the map
\( \map{h+g}{V}{W} \) is represented
by \( H+G \).
%</th:MatOpsRepMapOps>
\end{theorem}

\begin{proof}
This is \nearbyexercise{exer:CorrspMapMatOps}.
Generalize the examples above.
\end{proof}

\begin{remark}
These two operations are simple.
However we did not define them this way
because they are simple.
We defined them
this way because they reflect the operations of function scalar multiplication
and function addition.
Their simplicity is a pleasant bonus.

In the next subsection we will define another operation, matrix multiplication. 
Based on the above two operations 
a person's first thought may be to take an entry-by-entry product of 
two matrices.
While in theory we could define whatever matrix operations 
we like, our program here is to instead be practical and
define the new operation so that it
combines the entries in the way that represents 
function composition.
That is, we are defining matrix operations by referencing function
operations.

We can express this point another way.
Recall Theorem~III.\ref{th:MatMultRepsFuncAppl},
which says that matrix-vector 
multiplication represents the application of a linear map.
Following it, Remark~III.\ref{rem:NotSurprising} notes that 
the theorem
justifies the definition of matrix-vector multiplication,
and so in some sense the theorem must hold.
If the definition was such that the theorem didn't hold
then we would adjust the definition until it did hold.
The above \nearbytheorem{th:MatOpsRepMapOps} 
is another example of such a result.
\end{remark}

A special case of scalar multiplication is multiplication by zero.
For any map $0\cdot h$ is the zero homomorphism and for any matrix
$0\cdot H$ is the matrix with all entries zero.

\begin{definition} \label{df:ZeroMatrix}
%<*df:ZeroMatrix>
A \definend{zero matrix}\index{zero matrix}\index{matrix!zero} % 
has all entries $0$.
We write $Z_{\nbym{n}{m}}$ or simply $Z$
(another common notation is $0_{\nbym{n}{m}}$ or just $0$).
%</df:ZeroMatrix>
\end{definition}

\begin{example}
The zero map from any three-dimensional space to any two-dimensional space
is represented by the \( \nbym{2}{3} \) zero matrix
\begin{equation*}
   Z=\begin{mat}[r]
     0  &0  &0  \\
     0  &0  &0
   \end{mat}
\end{equation*}
no matter what domain and codomain bases we use.
\end{example}



\begin{exercises}
  \recommended \item
    Perform the indicated operations, if defined.
    \begin{exparts}
      \partsitem \( \begin{mat}[r]
                 5  &-1  &2  \\
                 6  &1   &1
               \end{mat}
               +
               \begin{mat}[r]
                 2  &1   &4  \\
                 3  &0   &5
               \end{mat}    \)
      \partsitem \( 6\cdot\begin{mat}[r]
                 2  &-1  &-1 \\
                 1  &2   &3
               \end{mat}   \)
      \partsitem \( \begin{mat}[r]
                 2  &1  \\
                 0  &3
               \end{mat}
               +
               \begin{mat}[r]
                 2  &1  \\
                 0  &3
               \end{mat} \)
      \partsitem \( 4\begin{mat}[r]
                 1  &2  \\
                 3  &-1
               \end{mat}
               +
               5\begin{mat}[r]
                -1  &4  \\
                -2  &1
               \end{mat} \)
      \partsitem \( 3\begin{mat}[r]
                 2  &1  \\
                 3  &0
               \end{mat}
               +2
               \begin{mat}[r]
                 1  &1  &4 \\
                 3  &0  &5
               \end{mat} \)
    \end{exparts}
    \begin{answer} 
      \begin{exparts*}
        \partsitem \( \begin{mat}[r]
                   7  &0  &6  \\
                   9  &1  &6
                 \end{mat}  \)
        \partsitem \( \begin{mat}[r]
                  12  &-6 &-6 \\
                   6  &12 &18
                 \end{mat}  \)
        \partsitem \( \begin{mat}[r]
                   4  &2  \\
                   0  &6
                 \end{mat}  \)
        \partsitem \( \begin{mat}[r]
                  -1  &28 \\
                   2  &1
                 \end{mat}  \)
        \partsitem Not defined.
      \end{exparts*}  
   \end{answer}
  \item \label{exer:CorrspMapMatOps}
    Prove \nearbytheorem{th:MatOpsRepMapOps}.
    \begin{exparts}
      \partsitem Prove that matrix addition represents addition of linear maps.
      \partsitem Prove that matrix scalar multiplication represents scalar
         multiplication of linear maps.
    \end{exparts}
    \begin{answer}
      Represent the domain vector $\vec{v}\in V$ and the maps
      $\map{g,h}{V}{W}$ with respect to bases $B,D$ in the usual way.
      \begin{exparts}
      \partsitem The representation of
         $(g+h)\,(\vec{v})=g(\vec{v})+h(\vec{v})$
        \begin{multline*}
          \bigl( (g_{1,1}v_1+\dots+g_{1,n}v_n)\vec{\delta}_1  
                  +\cdots                     
                  +(g_{m,1}v_1+\dots+g_{m,n}v_n)\vec{\delta}_m \bigr) \\ 
           +\bigl( (h_{1,1}v_1+\cdots+h_{1,n}v_n)\vec{\delta}_1    
                   +\cdots                                  
                   +(h_{m,1}v_1+\dots+h_{m,n}v_n)\vec{\delta}_m \bigr) 
         \end{multline*}
         regroups
         \begin{equation*}
          =((g_{1,1}+h_{1,1})v_1+\dots+
                 (g_{1,1}+h_{1,n})v_n)\cdot\vec{\delta}_1       
            +\cdots                            
           +((g_{m,1}+h_{m,1})v_1+\dots+
                 (g_{m,n}+h_{m,n})v_n)\cdot\vec{\delta}_m
        \end{equation*} 
        to the entry-by-entry
        sum of the representation of \( g(\vec{v}) \) and the representation of
        \( h(\vec{v}) \).
      \partsitem The representation of
        $(r\cdot h)\,(\vec{v})=r\cdot \bigl(h(\vec{v})\bigr)$
        \begin{multline*}
          r\cdot \bigl(
               (h_{1,1}v_1+h_{1,2}v_2+\dots+h_{1,n}v_n)\vec{\delta}_1  
               +\dots      
               +(h_{m,1}v_1+h_{m,2}v_2+\dots+h_{m,n}v_n)\vec{\delta}_m\bigr) \\
          =(rh_{1,1}v_1+\dots+rh_{1,n}v_n)\cdot\vec{\delta}_1           
          +\dots                                    
          +(rh_{m,1}v_1+\dots+rh_{m,n}v_n)\cdot\vec{\delta}_m
        \end{multline*}
        is the  entry-by-entry
        multiple of \( r \) and the representation of \( h \).
     \end{exparts}  
   \end{answer}
  \recommended \item
    Prove each, assuming that the operations are defined, 
    where \( G \), \( H \), and 
    \( J \) are matrices, where
    \( Z \) is the zero matrix, and where \( r \) and \( s \) are scalars.
    \begin{exparts}
      \partsitem Matrix addition is commutative \( G+H=H+G \).
      \partsitem Matrix addition is associative \( G+(H+J)=(G+H)+J \).
      \partsitem The zero matrix is an additive identity \( G+Z=G \).
      \partsitem \( 0\cdot G=Z \)
      \partsitem \( (r+s)G=rG+sG \)
      \partsitem Matrices have an additive inverse \( G+(-1)\cdot G=Z \).
      \partsitem \( r(G+H)=rG+rH \)
      \partsitem \( (rs)G=r(sG) \)
    \end{exparts}
    \begin{answer}
      First, each of these properties
      is easy to check in an entry-by-entry way.
      For example, writing
      \begin{equation*}
        G=
        \begin{mat}
          g_{1,1}  &\ldots  &g_{1,n}  \\
          \vdots   &        &\vdots   \\
          g_{m,1}  &\ldots  &g_{m,n}       
        \end{mat}
        \qquad
        H=
        \begin{mat}
          h_{1,1}  &\ldots  &h_{1,n}  \\
          \vdots   &        &\vdots   \\
          h_{m,1}  &\ldots  &h_{m,n}       
        \end{mat}
      \end{equation*}
      then, by definition we have
      \begin{equation*}
        G+H=
        \begin{mat}
          g_{1,1}+h_{1,1}  &\ldots  &g_{1,n}+h_{1,n}  \\
          \vdots           &        &\vdots           \\
          g_{m,1}+h_{m,1}  &\ldots  &g_{m,n}+h_{m,n}       
        \end{mat}
        \qquad
        H+G=
        \begin{mat}
          h_{1,1}+g_{1,1}  &\ldots  &h_{1,n}+g_{1,n}  \\
          \vdots           &        &\vdots           \\
          h_{m,1}+g_{m,1}  &\ldots  &h_{m,n}+g_{m,n}       
        \end{mat}
      \end{equation*}
      and the two are equal since their entries are equal 
      $g_{i,j}+h_{i,j}=h_{i,j}+g_{i,j}$.
      That is, each of these is easy to check by using 
      \nearbydefinition{def:SumScalarProdMats} alone.

      However, each property
      is also easy to understand in terms of the represented
      maps, by applying \nearbytheorem{th:MatOpsRepMapOps} as well as
      the definition.
      \begin{exparts}
        \partsitem The two maps $g+h$ and $h+g$ are equal because
          $g(\vec{v})+h(\vec{v})=h(\vec{v})+g(\vec{v})$, as addition is 
          commutative in any vector space.
          Because the maps are the same, they must have the same
          representative. 
        \partsitem As with the prior answer, except that here we apply that
          vector space addition is associative.
        \partsitem As before, except that here we note that
          $g(\vec{v})+z(\vec{v})=g(\vec{v})+\zero=g(\vec{v})$.
        \partsitem Apply that
          $0\cdot g(\vec{v})=\zero=z(\vec{v})$.
        \partsitem Apply that
          $(r+s)\cdot g(\vec{v})=r\cdot g(\vec{v})+s\cdot g(\vec{v})$.
        \partsitem Apply the prior two items with $r=1$ and $s=-1$.
        \partsitem Apply that
          $r\cdot (g(\vec{v})+h(\vec{v}))=r\cdot g(\vec{v})+r\cdot h(\vec{v})$.
        \partsitem Apply that
          $(rs)\cdot g(\vec{v})=r\cdot(s\cdot g(\vec{v}))$.
      \end{exparts}
    \end{answer}
  \item 
    Fix domain and codomain spaces. 
    In general, one
    matrix can represent many different maps with respect to different bases.
    However, prove that a zero matrix represents only a zero map.
    Are there other such matrices?
    \begin{answer}
      For any \( V,W \) with bases \( B,D \), 
      the (appropriately-sized) zero matrix represents this map.
      \begin{equation*}
        \vec{\beta}_1\mapsto 0\cdot\vec{\delta}_1+\dots+0\cdot\vec{\delta}_m
        \quad\cdots\quad
        \vec{\beta}_n\mapsto 0\cdot\vec{\delta}_1+\dots+0\cdot\vec{\delta}_m
      \end{equation*}
      This is the zero map.

      There are no other matrices that represent only one map.
      For, suppose that \( H \) is 
      not the zero matrix.
      Then it has a nonzero entry; assume that \( h_{i,j}\neq 0 \).
      With respect to bases $B,D$, it represents \( \map{h_1}{V}{W} \) 
      sending
      \begin{equation*}
        \vec{\beta}_j\mapsto
        h_{1,j}\vec{\delta}_1+\dots+h_{i,j}\vec{\delta}_i
          +\dots+h_{m,j}\vec{\delta}_m
      \end{equation*}
      and with respect to $B,2\cdot D$ it also represents 
      \( \map{h_2}{V}{W} \) sending
      \begin{equation*}
        \vec{\beta}_j\mapsto
        h_{1,j}\cdot(2\vec{\delta}_1)+\dots+h_{i,j}\cdot(2\vec{\delta}_i)
          +\dots+h_{m,j}\cdot(2\vec{\delta}_m)
      \end{equation*}
      (the notation $2\cdot D$ means to double 
      all of the members of D). 
      These maps are easily seen to be unequal. 
     \end{answer}
  \recommended \item \label{exer:LinMapsIsoMatSp}
    Let \( V \) and \( W \) be vector spaces of dimensions \( n \) and \( m \).
    Show that the space \( \linmaps{V}{W} \) of linear maps from \( V \) to
    \( W \) is isomorphic to \( \matspace_{\nbym{m}{n}} \).
    \begin{answer}
      Fix bases \( B \) and \( D \) for \( V \) and \( W \), and consider  
      \( \map{\mbox{Rep}_{B,D}}{\linmaps{V}{W}}{\matspace_{\nbym{m}{n}}} \)
      associating each linear map with the matrix representing that map
      $h\mapsto \rep{h}{B,D}$.
      From the prior section we know that (under fixed bases) 
      the matrices correspond to linear maps, 
      so the representation map is one-to-one and onto.
      That it preserves linear operations is     
      \nearbytheorem{th:MatOpsRepMapOps}.
    \end{answer}
  \recommended \item 
    Show that it follows from the prior questions that
    for any six transformations
    \( \map{t_1,\dots,t_6}{\Re^2}{\Re^2} \)
    there are scalars \( c_1,\dots,c_6\in\Re \) such that
    \( c_1t_1+\dots+c_6t_6 \) is the zero map.
    (\textit{Hint:}~the six is slightly misleading.)
    \begin{answer}
      Fix bases and represent the transformations with
      \( \nbyn{2} \) matrices.
      The space of matrices \( \matspace_{\nbyn{2}} \) has dimension four,
      and hence the above six-element set is linearly dependent.
      By the prior exercise that extends to a dependence of maps.
      (The misleading part is only that there are six transformations, not
      five, so that we have more than we need to give the existence of the 
      dependence.) 
    \end{answer}
  \item 
    The \definend{trace}\index{trace}\index{matrix!trace} 
    of a square matrix is the sum of the entries on the main diagonal (the
    \( 1,1 \) entry
    plus the \( 2,2 \) entry, etc.;
    we will see the significance of the trace in Chapter Five).
    Show that \( \mbox{trace}(H+G)=\mbox{trace}(H)+\mbox{trace}(G)  \).
    Is there a similar result for scalar multiplication?
    \begin{answer}
      That the trace of a sum is the sum of the traces holds
      because both \( \text{trace}(H+G) \) and
      \( \text{trace}(H)+\text{trace}(G) \) are the sum of
      \( h_{1,1}+g_{1,1} \) with \( h_{2,2}+g_{2,2} \), etc.
      For scalar multiplication we have
      \( \mbox{trace}(r\cdot H)=r\cdot\mbox{trace}(H) \); the proof is easy.
      Thus the trace map is a homomorphism from $\matspace_{\nbyn{n}}$ to
      $\Re$.  
    \end{answer}
  \item 
    Recall that the \definend{transpose}\index{matrix!transpose}%
    \index{transpose!interaction with sum and scalar multiplication}
    of a matrix $M$ is another matrix, whose $i,j$ entry is the 
    $j,i$ entry of $M$.
    Verify these identities.
    \begin{exparts}
      \partsitem \( \trans{(G+H)}=\trans{G}+\trans{H} \)
      \partsitem \( \trans{(r\cdot H)}=r\cdot\trans{H} \)
    \end{exparts}
    \begin{answer} 
      \begin{exparts}
        \partsitem The \( i,j \) entry of \( \trans{(G+H)} \) is
          \( g_{j,i}+h_{j,i} \).
          That is also the \( i,j \) entry of \( \trans{G}+\trans{H} \).
        \partsitem The \( i,j \) entry of \( \trans{(r\cdot H)} \) is
          \( rh_{j,i} \),
          which is also the \( i,j \) entry of \( r\cdot\trans{H} \).
      \end{exparts}  
   \end{answer}
  \recommended \item 
    A square matrix is \definend{symmetric}\index{matrix!symmetric}%
    \index{symmetric matrix} if each \( i,j \) entry equals
    the \( j,i \) entry, that is, if the matrix equals its transpose.
    \begin{exparts}
      \partsitem Prove that for any square~$H$,
        the matrix \( H+\trans{H} \) is symmetric.
        Does every symmetric matrix have this form?
      \partsitem Prove that the set of \( \nbyn{n} \) symmetric matrices is
        a subspace of \( \matspace_{\nbyn{n}} \).
    \end{exparts}
    \begin{answer}  
      \begin{exparts}
        \partsitem For \( H+\trans{H} \), the \( i,j \) entry 
          is \( h_{i,j}+h_{j,i} \) and
          the \( j,i \) entry of  is \( h_{j,i}+h_{i,j} \).
          The two are equal and thus \( H+\trans{H} \) is symmetric.

          Every symmetric matrix does have that form, since we can write
          \( H=(1/2)\cdot(H+\trans{H}) \).
        \partsitem The set of symmetric matrices is nonempty as it
          contains the zero matrix.
          Clearly a scalar multiple of a symmetric matrix is symmetric.
          A sum \( H+G \) of two symmetric matrices is
          symmetric because \( h_{i,j}+g_{i,j}=h_{j,i}+g_{j,i} \) (since
          \( h_{i,j}=h_{j,i} \) and \( g_{i,j}=g_{j,i} \)).
          Thus the subset is nonempty and closed under the inherited
          operations, and so it is a subspace.
      \end{exparts} 
    \end{answer}
  \recommended \item 
    \begin{exparts}
      \partsitem How does matrix rank interact with 
        scalar multiplication\Dash can
        a scalar product of a rank~\( n \) matrix have rank less than \( n \)?
        Greater?
      \partsitem How does matrix rank interact with matrix 
        addition\Dash can a sum of
        rank~\( n \) matrices have rank less than \( n \)?
        Greater?
    \end{exparts}
    \begin{answer}  
      \begin{exparts}
        \partsitem Scalar multiplication leaves the rank of a matrix unchanged
          except that multiplication by zero leaves the matrix
          with rank zero.
          (This follows from the first theorem of the book, that multiplying a
          row by a nonzero 
          scalar doesn't change the solution set of the associated
          linear system.)
        \partsitem A sum of rank \( n \) matrices can have rank 
          less than \( n \).
          For instance,
          for any matrix \( H \), the sum \( H+(-1)\cdot H \) has rank zero.

          A sum of rank \( n \) matrices can have rank greater than \( n \).
          Here are rank~one matrices that sum to a rank~two matrix.
          \begin{equation*}
            \begin{mat}[r]
              1  &0  \\
              0  &0
            \end{mat}
            +\begin{mat}[r]
              0  &0  \\
              0  &1
            \end{mat}
            =\begin{mat}[r]
              1  &0  \\
              0  &1
            \end{mat}
          \end{equation*}
      \end{exparts}  
    \end{answer}
\end{exercises}











\subsection{Matrix Multiplication}
After representing addition and scalar multiplication of linear maps
in the prior subsection, the natural next map operation to consider 
is composition.

\begin{lemma} \label{lm:CompositionOfLinearMapsIsLinear}
%<*lm:CompositionOfLinearMapsIsLinear>
The composition of linear maps is linear.
%</lm:CompositionOfLinearMapsIsLinear>
\end{lemma}

\begin{proof}
\textit{(This argument has appeared earlier, 
as part of the proof of Theorem~I.\ref{th:IsoEquivRel}.)}  
%<*pf:CompositionOfLinearMapsIsLinear>
Let $\map{h}{V}{W}$ and $\map{g}{W}{U}$ be linear. 
The calculation
\begin{multline*}
  \composed{g}{h}\,\bigl(c_1\cdot\vec{v}_1+c_2\cdot\vec{v}_2\bigr)
     =g\bigl(\,h(c_1\cdot \vec{v}_1+c_2\cdot\vec{v}_2)\,\bigr)        
     =g\bigl(\,c_1\cdot h(\vec{v}_1)+c_2\cdot h(\vec{v}_2)\,\bigr) \\ 
     =c_1\cdot g\bigl(h(\vec{v}_1))+c_2\cdot g(h(\vec{v}_2)\bigr)  
     =c_1\cdot (\composed{g}{h})(\vec{v}_1)
                   +c_2\cdot (\composed{g}{h})(\vec{v}_2)
\end{multline*}
shows that $\map{\composed{g}{h}}{V}{U}$
preserves linear combinations.
%</pf:CompositionOfLinearMapsIsLinear>
\end{proof}

To see how the representation of the 
composite relates to the representations
of the compositors, consider an example.

\begin{example} \label{ex:RepCompLinMaps}
Let $\map{h}{\Re^4}{\Re^2}$ and $\map{g}{\Re^2}{\Re^3}$,
fix bases
\( B\subset\Re^4 \), \( C\subset\Re^2 \), \( D\subset\Re^3 \), 
and let these be the representations.
\begin{equation*}
  H=\rep{h}{B,C}
  =\begin{mat}[r]
       4  &6  &8  &2  \\
       5  &7  &9  &3
    \end{mat}_{B,C}
  \qquad
  G=\rep{g}{C,D}
  =\begin{mat}[r]
       1  &1  \\
       0  &1  \\
       1  &0
    \end{mat}_{C,D}
\end{equation*}
To 
represent the composition \( \map{\composed{g}{h}}{\Re^4}{\Re^3} \)
we start with a $\vec{v}$, represent $h$ of $\vec{v}$, 
and then represent $g$ of that.
The representation of $h(\vec{v})$ is the product of $h$'s matrix and 
$\vec{v}$'s vector.
\begin{equation*}
  \rep{\,h(\vec{v})\,}{C}
  =
  \begin{mat}[r]
    4  &6  &8  &2  \\
    5  &7  &9  &3
  \end{mat}_{B,C}
  \colvec{v_1 \\ v_2 \\ v_3 \\ v_4}_B  
  =
  \colvec{4v_1+6v_2+8v_3+2v_4 \\ 5v_1+7v_2+9v_3+3v_4}_C
\end{equation*}
The representation of $g(\,h(\vec{v})\,)$ is the product of $g$'s matrix
and $h(\vec{v})$'s vector.
\begin{multline*}
  \rep{\,g(h(\vec{v}))\,}{D}\!
  =\begin{mat}[r]
      1  &1  \\
      0  &1  \\
      1  &0
   \end{mat}_{C,D}
   \colvec{4v_1+6v_2+8v_3+2v_4 \\ 5v_1+7v_2+9v_3+3v_4}_C    \\
  =\colvec{1\cdot(4v_1+6v_2+8v_3+2v_4)+1\cdot(5v_1+7v_2+9v_3+3v_4) \\ 
               0\cdot(4v_1+6v_2+8v_3+2v_4)+1\cdot(5v_1+7v_2+9v_3+3v_4) \\
               1\cdot(4v_1+6v_2+8v_3+2v_4)+0\cdot(5v_1+7v_2+9v_3+3v_4)}_D
\end{multline*}
Distributing and regrouping on the $v$'s gives
\begin{equation*}
  =
  \colvec{(1\cdot4+1\cdot5)v_1+
                (1\cdot6+1\cdot7)v_2+
                (1\cdot8+1\cdot9)v_3+
                (1\cdot2+1\cdot3)v_4 \\
                (0\cdot4+1\cdot5)v_1+
                (0\cdot6+1\cdot7)v_2+
                (0\cdot8+1\cdot9)v_3+
                (0\cdot2+1\cdot3)v_4 \\
                (1\cdot4+0\cdot5)v_1+
                (1\cdot6+0\cdot7)v_2+
                (1\cdot8+0\cdot9)v_3+
                (1\cdot2+0\cdot3)v_4}_D
\end{equation*}
which we recognize as the result of this matrix-vector product.
\begin{equation*}
  =
  \begin{mat}
    1\cdot4+1\cdot5  &1\cdot6+1\cdot7  &1\cdot8+1\cdot9  &1\cdot2+1\cdot3 \\
    0\cdot4+1\cdot5  &0\cdot6+1\cdot7  &0\cdot8+1\cdot9  &0\cdot2+1\cdot3 \\
    1\cdot4+0\cdot5  &1\cdot6+0\cdot7  &1\cdot8+0\cdot9  &1\cdot2+0\cdot3
  \end{mat}_{B,D}
  \colvec{v_1 \\ v_2 \\ v_3 \\ v_4}_D
\end{equation*}
Thus
the matrix representing $\composed{g}{h}$ has the rows of $G$
combined with the columns of $H$.
\end{example}

\begin{definition}  \label{def:MatMult}
%<*df:MatMult>
The \definend{matrix-multiplicative product}%
\index{matrix!multiplication}\index{multiplication!matrix-matrix}
of the \( \nbym{m}{r} \)
matrix~\( G \) and the \( \nbym{r}{n} \) matrix~\( H \) is the
\( \nbym{m}{n} \) matrix~\( P \), where
\begin{equation*}
  p_{i,j}
  =
  g_{i,1}h_{1,j}+g_{i,2}h_{2,j}+\dots+g_{i,r}h_{r,j}
\end{equation*}
that is, the \( i,j \)-th entry of the product is the dot product of the
\( i \)-th row of the first matrix with the \( j \)-th column of the second.
\begin{equation*}
    GH=
    \begin{mat}
              &\vdots                     \\
      g_{i,1} &g_{i,2} &\ldots  &g_{i,r}  \\
              &\vdots
    \end{mat}
    \begin{mat}
              &h_{1,j}           \\
      \ldots  &h_{2,j} &\ldots   \\
              &\vdots            \\
              &h_{r,j}
    \end{mat}
  =
    \begin{mat}
              &\vdots            \\
      \ldots  &p_{i,j}  &\ldots  \\
              &\vdots
    \end{mat}
\end{equation*}
%</df:MatMult>
\end{definition}

\begin{example}
The matrices from \nearbyexample{ex:RepCompLinMaps} combine in this way.
\begin{equation*}
   \begin{mat}
    1\cdot4+1\cdot5  &1\cdot6+1\cdot7  &1\cdot8+1\cdot9  &1\cdot2+1\cdot3 \\
    0\cdot4+1\cdot5  &0\cdot6+1\cdot7  &0\cdot8+1\cdot9  &0\cdot2+1\cdot3 \\
    1\cdot4+0\cdot5  &1\cdot6+0\cdot7  &1\cdot8+0\cdot9  &1\cdot2+0\cdot3
    \end{mat}               
   =\begin{mat}[r]
      9  &13  &17  &5  \\
      5  &7   &9   &3  \\
      4  &6   &8   &2
    \end{mat}
\end{equation*}
\end{example}

\begin{example}
\begin{equation*}
  \begin{mat}[r]
     2  &0  \\
     4  &6  \\
     8  &2
  \end{mat}
  \begin{mat}[r]
    1  &3  \\
    5  &7
  \end{mat}
  =
  \begin{mat}
   2\cdot 1+0\cdot 5  &2\cdot 3+0\cdot 7  \\
   4\cdot 1+6\cdot 5  &4\cdot 3+6\cdot 7  \\
   8\cdot 1+2\cdot 5  &8\cdot 3+2\cdot 7
  \end{mat}
  =
  \begin{mat}[r]
    2  &6  \\
   34  &54 \\
   18  &38
  \end{mat}
\end{equation*}
\end{example}

We next check that our
definition of the matrix-matrix multiplication operation
does what we intend.

\begin{theorem}  \label{th:MatMultRepComp}
%<*th:MatMultRepComp>
\index{function!composition}\index{homomorphism!composition}
A composition of linear maps is represented by the matrix product
of the representatives.
%</th:MatMultRepComp>
\end{theorem}

\begin{proof}
This argument generalizes \nearbyexample{ex:RepCompLinMaps}.
%<*pf:MatMultRepComp0>
Let \( \map{h}{V}{W} \) and \( \map{g}{W}{X} \) be represented by
\( H \) and \( G \) with respect to bases
\( B\subset V \), \( C\subset W \), and \( D\subset X \), of sizes
\( n \), \( r \), and \( m \).
For any \( \vec{v}\in V \), the \( k \)-th component of
\( \rep{\,h(\vec{v})\,}{C} \) is
\begin{equation*}
  h_{k,1}v_1+\cdots+h_{k,n}v_n
\end{equation*}
and so the \( i \)-th component of
\( \rep{\,\composed{g}{h}\,(\vec{v})\,}{D} \) is this.
\begin{multline*}
  g_{i,1}\cdot(h_{1,1}v_1+\dots+h_{1,n}v_n)
  +g_{i,2}\cdot(h_{2,1}v_1+\dots+h_{2,n}v_n)  \\
  +\dots                                 
  +g_{i,r}\cdot(h_{r,1}v_1+\dots+h_{r,n}v_n)
\end{multline*}
Distribute and regroup on the \( v \)'s.
\begin{multline*}
  =(g_{i,1} h_{1,1}+g_{i,2} h_{2,1}+\dots+g_{i,r}h_{r,1})\cdot v_1    \\
   +\dots
   +(g_{i,1} h_{1,n}+g_{i,2} h_{2,n}
           +\dots+g_{i,r} h_{r,n})\cdot v_n
\end{multline*}
%</pf:MatMultRepComp0>
%<*pf:MatMultRepComp1>
Finish by recognizing that the coefficient of each \( v_j \)
\begin{equation*}
  g_{i,1}h_{1,j}+g_{i,2}h_{2,j}+\dots+g_{i,r}h_{r,j}
\end{equation*}
matches the definition of the \( i,j \) entry of the  product \( GH \).
%</pf:MatMultRepComp1>
\end{proof}

%<*MatMultArrowDiag0>
This
\definend{arrow diagram}\index{arrow diagram}
pictures the relationship between maps and matrices
(`wrt' abbreviates `with respect to').
%</MatMultArrowDiag0>
%\medskip
\begin{center}
  \includegraphics{ch3.20}
  %\quad
  %\includegraphics{ch3.19}  
\end{center}
%\medskip
%<*MatMultArrowDiag1>
Above the arrows, the maps show that the two ways of going from 
\( V \) to \( X \),
straight over via the composition or else in two steps by way of \( W \),
have the same effect 
\begin{equation*}
  \vec{v}\mapsunder{g\circ h}g(h(\vec{v}))
  \qquad
  \vec{v}\mapsunder{h}h(\vec{v})\mapsunder{g}g(h(\vec{v}))
\end{equation*}
(this is just the definition of composition).
Below the arrows, the matrices indicate that the product
does the same thing\Dash multiplying
$GH$ into the column vector $\rep{\vec{v}}{B}$ has the same
effect as  multiplying the column vector first by $H$ 
and then multiplying the result by $G$.
\begin{equation*}
   \rep{\composed{g}{h}}{B,D}
   =GH
   \qquad
   \rep{g}{C,D}\,\rep{h}{B,C}=GH
\end{equation*}
%</MatMultArrowDiag1>

\begin{example}
Because the number of columns on the left 
does not equal the number of rows on the right,
this product is not defined. 
\begin{equation*}
     \begin{mat}[r]
       -1  &2  &0  \\
        0  &10 &1.1
     \end{mat}
     \begin{mat}[r]
        0  &0  \\
        0  &2
     \end{mat}
\end{equation*}
\end{example}

One way to understand why the combination in the prior example is undefined
has to do with the underlying maps. 
We require that the sizes match 
because we want that the underlying
function composition is possible.
\begin{equation*}
  \text{dimension \( n \) space}
  \;\stackrel{h}{\longrightarrow}\;
  \text{dimension \( r \) space}
  \;\stackrel{g}{\longrightarrow}\;
  \text{dimension \( m \) space}
\end{equation*}
Thus, matrix product combines an $\nbym{m}{r}$ matrix~$G$
with an $\nbym{r}{n}$ matrix~$F$ to yield the
$\nbym{m}{n}$ result~$GF$.
Briefly,
`$\nbym{m}{r}\text{\ times\ }\nbym{r}{n}\text{\ equals\ }\nbym{m}{n}$'.

\begin{remark}
The order in which these things are written can be confusing.
In the prior
equation, the number written first $m$ is the dimension of
$g$'s codomain and is thus the number that appears last in the
map dimension description above.
The explanation is that while $h$ is done first
and then $g$, 
we write the composition as $\composed{g}{h}$,
from the notation `$g(h(\vec{v}))$'. 
(Some people try to lessen confusion by reading 
`$\composed{g}{h}$' aloud as ``$g$ following $h$.'')
That right to left order carries over to matrices:~$\composed{g}{h}$
is represented by $GH$.
\end{remark}

We can get insight into matrix-matrix product operation
by studying how the entries combine.
For instance, an alternative way to understand why we require above that
the sizes match
is that the row of the left-hand matrix must have the same number of
entries as the column of the right-hand matrix, or else some entry
will be left without a matching entry from the other matrix.

Another aspect of the combinatorics of
matrix multiplication is that in the definition of the $i,j$ entry
\begin{equation*} % \setlength{\fboxsep}{.15em}
  p_{i,j}
  =
  g_{i,\text{\highlight{$1 $}}}h_{\text{\highlight{$1 $}},j}
   +g_{i,\text{\highlight{$2 $}}}h_{\text{\highlight{$2 $}},j}
   +\dots+g_{i,\text{\highlight{$r $}}}h_{\text{\highlight{$r $}},j}
\end{equation*}
% \begin{equation*} \setlength{\fboxsep}{.15em}
%   p_{i,j}
%   =
%   g_{i,\text{\colorbox{lightgray}{$\scriptsize 1 $}}}h_{\text{\colorbox{lightgray}{$\scriptsize 1 $}},j}
%    +g_{i,\text{\colorbox{lightgray}{$\scriptsize 2 $}}}h_{\text{\colorbox{lightgray}{$\scriptsize 2 $}},j}
%    +\dots+g_{i,\text{\colorbox{lightgray}{$\scriptsize r $}}}h_{\text{\colorbox{lightgray}{$\scriptsize r $}},j}
% \end{equation*}
% \begin{equation*}
%   p_{i,j}
%   =
%   g_{i,\fbox{\scriptsize \( 1 \)}}h_{\fbox{\scriptsize \( 1 \)},j}
%    +g_{i,\fbox{\scriptsize \( 2 \)}}h_{\fbox{\scriptsize \( 2 \)},j}
%    +\dots+g_{i,\fbox{\scriptsize \( r \)}}h_{\fbox{\scriptsize \( r \)},j}
% \end{equation*}
the highlighted subscripts on the $g$'s are column indices while those on the
$h$'s indicate rows.
That is, the summation takes place over the columns of $G$
but over the rows of $H$\Dash
the definition treats left differently than right.
So we may reasonably suspect that \( GH \) 
can be unequal to \( HG \). 

\begin{example}   \label{ex:MatMultNoCommute}
Matrix multiplication is not commutative.
\begin{equation*}
    \begin{mat}[r]
      1  &2  \\
      3  &4
    \end{mat}
    \begin{mat}[r]
      5  &6  \\
      7  &8
    \end{mat}
  =
    \begin{mat}[r]
      19  &22  \\
      43  &50
    \end{mat}
  \qquad
    \begin{mat}[r]
      5  &6  \\
      7  &8
    \end{mat}
    \begin{mat}[r]
      1  &2  \\
      3  &4
    \end{mat}
  =
    \begin{mat}[r]
      23  &34  \\
      31  &46
    \end{mat}
\end{equation*}
% In fact, matrix multiplication hardly ever commutes, in that if we use
% multiply randomly chosen matrices both ways.
\end{example}

\begin{example}  \label{ex:MatMultNoCommuteII}
Commutativity can fail more dramatically:
\begin{equation*}
    \begin{mat}[r]
      5  &6  \\
      7  &8
    \end{mat}
    \begin{mat}[r]
      1  &2  &0 \\
      3  &4  &0
    \end{mat}
  =
    \begin{mat}[r]
      23  &34  &0  \\
      31  &46  &0
    \end{mat}
\end{equation*}
while
\begin{equation*}
    \begin{mat}[r]
      1  &2  &0 \\
      3  &4  &0
    \end{mat}
    \begin{mat}[r]
      5  &6  \\
      7  &8
    \end{mat}
\end{equation*}
isn't even defined.
\end{example}

\begin{remark}
The fact that
matrix multiplication is not commutative can be puzzling at first, 
perhaps because most
operations that people see in prior mathematics courses are commutative.
But matrix multiplication represents function composition, which
is not commutative: if
\( f(x)=2x \) and \( g(x)=x+1 \) then \( \composed{g}{f}(x)=2x+1 \)
while \( \composed{f}{g}(x)=2(x+1)=2x+2 \).
(True, this \( g \) is not linear and we might have hoped that linear functions
would commute but this shows that the failure of commutativity for
matrix multiplication fits into a larger context.)
\end{remark}

Except for the lack of commutativity,
matrix multiplication is algebraically well-behaved.
Below are some nice properties and more are in
\nearbyexercise{exer:NicePropsMatMult} and
\nearbyexercise{exer:MoreNicePropsMatMult}.

\begin{theorem}  \label{th:MatMultWellBehaved}
%<*th:MatMultWellBehaved>
If \( F \), \( G \), and \( H \) are matrices,
and the matrix products are defined,
then the product is associative
$(FG)H=F(GH)$
and distributes over matrix addition
$F(G+H)=FG+FH$ and $(G+H)F=GF+HF$.
%</th:MatMultWellBehaved>
\end{theorem}

\begin{proof}
%<*pf:MatMultWellBehaved0>
Associativity holds because matrix multiplication represents function
composition, which is associative:~the maps
\( \composed{(\composed{f}{g})}{h} \) and 
\( \composed{f}{(\composed{g}{h})} \) are equal 
as both send \( \vec{v} \) to \( f(g(h(\vec{v}))) \).
%</pf:MatMultWellBehaved0>

%<*pf:MatMultWellBehaved1>
Distributivity  is similar.
For instance, the first one goes
\( \composed{f}{(g+h)}\,(\vec{v})
    =f\bigl(\,(g+h)(\vec{v})\,\bigr)  
    =f\bigl(\,g(\vec{v})+h(\vec{v})\,\bigr)
    =f(g(\vec{v}))+f(h(\vec{v}))
    =\composed{f}{g}(\vec{v})+\composed{f}{h}(\vec{v}) \) 
(the third equality uses the linearity of $f$).
%</pf:MatMultWellBehaved1>
\end{proof}

\begin{remark} 
We could instead prove that result by slogging through the indices.
For example, for associativity
the \( i,j \)-th entry of \( (FG)H \) is
\begin{align*}
   \MoveEqLeft (f_{i,1}g_{1,1}+f_{i,2}g_{2,1}+\dots+f_{i,r}g_{r,1})h_{1,j}         \\
   &+(f_{i,1}g_{1,2}+f_{i,2}g_{2,2}+\dots+f_{i,r}g_{r,2})h_{2,j}   \\
   &\vdotswithin{+}                                       \\
   &+(f_{i,1}g_{1,s}+f_{i,2}g_{2,s}+\dots+f_{i,r}g_{r,s})h_{s,j}
\end{align*}
(where \( F \), \( G \), and \( H \) are \( \nbym{m}{r} \),
\( \nbym{r}{s} \), and \( \nbym{s}{n} \) matrices),
distribute
\begin{align*}
   \MoveEqLeft f_{i,1}g_{1,1}h_{1,j}+f_{i,2}g_{2,1}h_{1,j}+
            \dots+f_{i,r}g_{r,1}h_{1,j}                  \\ 
   &+f_{i,1}g_{1,2}h_{2,j}+f_{i,2}g_{2,2}h_{2,j}+
          \dots+f_{i,r}g_{r,2}h_{2,j}                      \\
   &\vdotswithin{+}                                           \\
   &+f_{i,1}g_{1,s}h_{s,j}+f_{i,2}g_{2,s}h_{s,j}+\dots
        +f_{i,r}g_{r,s}h_{s,j}
\end{align*}
and regroup around the \( f \)'s
\begin{align*}
   \MoveEqLeft f_{i,1}(g_{1,1}h_{1,j}+g_{1,2}h_{2,j}+\dots+g_{1,s}h_{s,j})     \\
   &+f_{i,2}(g_{2,1}h_{1,j}+g_{2,2}h_{2,j}+\dots+g_{2,s}h_{s,j})   \\
   &\vdotswithin{+}                                       \\
   &+f_{i,r}(g_{r,1}h_{1,j}+g_{r,2}h_{2,j}+\dots+g_{r,s}h_{s,j})
\end{align*}
to get the \( i,j \) entry of \( F(GH) \).

Contrast the two ways of verifying associativity.
The argument just above is hard to understand in that while
the calculations are easy to check, the arithmetic seems unconnected to any
idea.
The argument in the proof 
is shorter and says why this property ``really'' holds.
This illustrates the comments made at the start of
the chapter on vector spaces\Dash at least some 
of the time an argument from higher-level constructs is clearer.
\end{remark}

We have now seen how to construct 
the representation of the composition of two linear maps
from the representations of the two maps.
We have called the combination the product of the two matrices.
We
will explore this operation more in the next subsection.

\begin{exercises}
  \recommended \item 
    Compute, or state ``not defined''.
    \begin{exparts*}
      \partsitem
        $\begin{mat}[r]
          3  &1  \\
          -4 &2 
        \end{mat}
        \begin{mat}[r]
          0  &5  \\
          0  &0.5
        \end{mat}$
      \partsitem \(
        \begin{mat}[r]
          1  &1  &-1  \\
          4  &0  &3
        \end{mat}
        \begin{mat}[r]
          2  &-1 &-1  \\
          3  &1  &1   \\
          3  &1  &1
        \end{mat}   \)
      \partsitem \(
        \begin{mat}[r]
          2  &-7 \\
          7  &4
        \end{mat}
        \begin{mat}[r]
          1  &0  &5   \\
         -1  &1  &1   \\
          3  &8  &4
        \end{mat}   \)
      \partsitem \(
        \begin{mat}[r]
          5  &2  \\
          3  &1
        \end{mat}
        \begin{mat}[r]
          -1  &2   \\
           3  &-5
        \end{mat} \)
    \end{exparts*}
    \begin{answer} 
      \begin{exparts*}
        \partsitem
          $\begin{mat}[r]
            0  &15.5  \\
            0  &-19
          \end{mat}$
        \partsitem \(
          \begin{mat}[r]
            2  &-1  &-1  \\
           17  &-1  &-1
          \end{mat} \)
        \partsitem Not defined.
        \partsitem \( \begin{mat}[r]
                   1  &0  \\
                   0  &1
                 \end{mat}  \)
      \end{exparts*}  
    \end{answer}
  \recommended \item  
    Where
    \begin{equation*}
      A=
       \begin{mat}[r]
         1  &-1  \\
         2  &0   \\
       \end{mat}
      \quad
      B=
       \begin{mat}[r]
         5  &2   \\
         4  &4   \\
       \end{mat}
      \quad
      C=
       \begin{mat}[r]
        -2  &3   \\
        -4  &1   \\
       \end{mat}
    \end{equation*}
    compute or state `not defined'.
    \begin{exparts*}
      \partsitem \( AB \)
      \partsitem \( (AB)C \)
      \partsitem \( BC \)
      \partsitem \( A(BC) \)
    \end{exparts*}
    \begin{answer}  
      \begin{exparts*}
        \partsitem \(
          \begin{mat}[r]
            1  &-2   \\
            10 &4
          \end{mat}  \)
        \partsitem \( 
          \begin{mat}[r]
            1  &-2   \\
            10 &4
          \end{mat}
          \begin{mat}[r]
            -2 &3    \\
            -4 &1
          \end{mat}=
          \begin{mat}[r]
            6  &1    \\
           -36 &34
          \end{mat}  \)
        \partsitem \(
          \begin{mat}[r]
           -18 &17   \\
           -24 &16
          \end{mat}  \)
        \partsitem \(
          \begin{mat}[r]
            1  &-1  \\
            2  &0
          \end{mat}
          \begin{mat}[r]
           -18 &17   \\
           -24 &16
          \end{mat}=
          \begin{mat}[r]
            6  &1    \\
           -36 &34
          \end{mat}  \)
      \end{exparts*}  
    \end{answer}
  \item  
    Which products are defined?
    \begin{exparts*}
      \partsitem \( \nbym{3}{2} \)~times~\( \nbym{2}{3} \)
      \partsitem \( \nbym{2}{3} \)~times~\( \nbym{3}{2} \)
      \partsitem \( \nbym{2}{2} \)~times~\( \nbym{3}{3} \)
      \partsitem \( \nbym{3}{3} \)~times~\( \nbym{2}{2} \)
    \end{exparts*}
    \begin{answer}
      \begin{exparts*}
        \partsitem Yes.
        \partsitem Yes.
        \partsitem No.
        \partsitem No.
      \end{exparts*}
    \end{answer}
  \recommended \item 
    Give the size of the product or state ``not defined''.
    \begin{exparts}
      \partsitem a \( \nbym{2}{3} \) matrix times a \( \nbym{3}{1} \) matrix
      \partsitem a \( \nbym{1}{12} \) matrix times a \( \nbym{12}{1} \) matrix
      \partsitem a \( \nbym{2}{3} \) matrix times a \( \nbym{2}{1} \) matrix
      \partsitem a \( \nbym{2}{2} \) matrix times a \( \nbym{2}{2} \) matrix
    \end{exparts}
    \begin{answer}  
      \begin{exparts*}
        \partsitem \( \nbym{2}{1} \)
        \partsitem \( \nbym{1}{1} \)
        \partsitem Not defined.
        \partsitem \( \nbym{2}{2} \)
      \end{exparts*}  
     \end{answer}
  \recommended \item 
    Find the system of equations resulting from starting with
    \begin{equation*}
      \begin{linsys}{3}
        h_{1,1}x_1  &+  &h_{1,2}x_2  &+  &h_{1,3}x_3  &=  &d_1  \\
        h_{2,1}x_1  &+  &h_{2,2}x_2  &+  &h_{2,3}x_3  &=  &d_2  
       \end{linsys}
    \end{equation*}
    and making this change of variable (i.e., substitution).
    \begin{equation*}
      \begin{linsys}{2}
        x_1  &=  &g_{1,1}y_1  &+  &g_{1,2}y_2  \\
        x_2  &=  &g_{2,1}y_1  &+  &g_{2,2}y_2  \\
        x_3  &=  &g_{3,1}y_1  &+  &g_{3,2}y_2  
      \end{linsys}
    \end{equation*}
    \begin{answer}
      We have
      \begin{equation*}
        \begin{linsys}{3}
            h_{1,1}\cdot(g_{1,1}y_1+g_{1,2}y_2)  
               &+  &h_{1,2}\cdot (g_{2,1}y_1+g_{2,2}y_2)  
               &+  &h_{1,3}\cdot (g_{3,1}y_1+g_{3,2}y_2)  
               &=  &d_1  \\
            h_{2,1}\cdot(g_{1,1}y_1+g_{1,2}y_2)  
               &+  &h_{2,2}\cdot (g_{2,1}y_1+g_{2,2}y_2)  
               &+  &h_{2,3}\cdot (g_{3,1}y_1+g_{3,2}y_2)  
               &=  &d_2  
         \end{linsys}
      \end{equation*}
      which, after expanding and regrouping about the $y$'s yields this.
      \begin{equation*}
        \begin{linsys}{2}
            (h_{1,1}g_{1,1}+h_{1,2}g_{2,1}+h_{1,3}g_{3,1})y_1
              &+  &(h_{1,1}g_{1,2}+h_{1,2}g_{2,2}+h_{1,3}g_{3,2})y_2  
               &=  &d_1  \\
            (h_{2,1}g_{1,1}+h_{2,2}g_{2,1}+h_{2,3}g_{3,1})y_1
              &+  &(h_{2,1}g_{1,2}+h_{2,2}g_{2,2}+h_{2,3}g_{3,2})y_2  
               &=  &d_2
         \end{linsys}
      \end{equation*}
      We can express the starting system 
      and the system used for the substitutions
      in matrix language.
      \begin{equation*}
        \begin{mat}
          h_{1,1}  &h_{1,2}  &h_{1,3}  \\
          h_{2,1}  &h_{2,2}  &h_{2,3}
        \end{mat}
        \colvec{x_1 \\ x_2 \\ x_3}
        =H\colvec{x_1 \\ x_2 \\ x_3}=\colvec{d_1 \\ d_2}
        \qquad
        \begin{mat}
          g_{1,1}  &g_{1,2}  \\
          g_{2,1}  &g_{2,2}  \\
          g_{3,1}  &g_{3,2}
        \end{mat}
        \colvec{y_1 \\ y_2}
        =G\colvec{y_1 \\ y_2}=\colvec{x_1 \\ x_2 \\ x_3}
      \end{equation*}
      With this, the substitution is 
      $\vec{d}=H\vec{x}=H(G\vec{y})=(HG)\vec{y}$.
    \end{answer}
  \item 
    As \nearbydefinition{def:MatMult} points out, the matrix product
    operation generalizes the dot product.
    Is the dot product of a \( \nbym{1}{n} \) row vector
    and a \( \nbym{n}{1} \) column vector the same as their 
    matrix-multiplicative product?
    \begin{answer}
      Technically, no.
      The dot product operation yields a scalar while the matrix product
      yields a $\nbyn{1}$ matrix.
      However, we usually will ignore the distinction.
    \end{answer}
  \recommended \item 
    Represent the derivative map on \( \polyspace_n \)
    with respect to \( B,B \) where \( B \) is the natural basis
    \( \sequence{1,x,\ldots,x^n} \).
    Show that the product of this matrix with itself is defined;
    what map does it represent?
    \begin{answer}
      The action of $d/dx$ on $B$ is 
      $1\mapsto0$, $x\mapsto 1$, $x^2\mapsto 2x$, \ldots and so 
      this is its $\nbyn{(n+1)}$ matrix representation.
      \begin{equation*}
        \rep{\frac{d}{dx}}{B,B}=
        \begin{mat}
          0  &1 &0  &  &0  \\
          0  &0 &2  &  &0  \\
             &  &   &\ddots\\
          0  &0 &0  &  &n  \\
          0  &0 &0  &  &0
        \end{mat}
      \end{equation*}
      The product of this matrix with itself is defined because the matrix is
      square.
      \begin{equation*}
        \begin{mat}
          0  &1 &0  &  &0  \\
          0  &0 &2  &  &0  \\
             &  &   &\ddots\\
          0  &0 &0  &  &n  \\
          0  &0 &0  &  &0
        \end{mat}^2
        =
        \begin{mat}
          0  &0 &2  &0  &        &0       \\
          0  &0 &0  &6  &        &0       \\
             &  &   &   &\ddots  &        \\
          0  &0 &0  &  &         &n(n-1)  \\
          0  &0 &0  &  &         &0       \\
          0  &0 &0  &  &         &0
        \end{mat}
      \end{equation*}
      The map so represented is the composition
      \begin{equation*}
        p\;\mapsunder{\frac{d}{dx}}\;\frac{d\,p}{dx}
         \;\mapsunder{\frac{d}{dx}}\;\frac{d^2\,p}{dx^2}
      \end{equation*}
      which is the second derivative operation.  
     \end{answer}
  \item 
    \cite{Cleary}
    Match each type of matrix with all these descriptions that could fit:
    (i)~can be multiplied by its transpose to make a $\nbyn{1}$ matrix, 
    (ii)~is similar to the $\nbyn{3}$ matrix of all zeros,
    (iii)~can represent a linear map from $\Re^3$ to $\Re^2$ that is not onto,
    (iv)~can represent an isomorphism from $\Re^3$ to $\polyspace^2$.
    \begin{exparts}
      \item a $\nbym{2}{3}$ matrix whose rank is~$1$ 
      \item a $\nbyn{3}$ matrix that is nonsingular         
      \item a $\nbyn{2}$ matrix that is singular
      \item an $\nbym{n}{1}$ column vector 
    \end{exparts}
    \begin{answer}
      \begin{exparts}
        \item iii
        \item iv
        \item None
        \item None (or (i) if we allow multiplication from the left)
      \end{exparts}
    \end{answer}
  \item  
    Show that composition of linear transformations on \( \Re^1 \) is
    commutative.
    Is this true for any one-dimensional space?
    \begin{answer}
      It is true for all one-dimensional spaces.
      Let $f$ and $g$ be transformations of a one-dimensional space.
      We must show that 
      $\composed{g}{f}\,(\vec{v})=\composed{f}{g}\,(\vec{v})$
      for all vectors.
      Fix a basis $B$ for the space and then the transformations are 
      represented by $\nbyn{1}$ matrices.
      \begin{equation*}
        F=\rep{f}{B,B}=\begin{mat}
                        f_{1,1}
                     \end{mat}
        \qquad
        G=\rep{g}{B,B}=\begin{mat}
                        g_{1,1}
                     \end{mat}
      \end{equation*}
      Therefore, the compositions can be represented as $GF$ and $FG$.
      \begin{equation*}
        GF=\rep{\composed{g}{f}}{B,B}=\begin{mat}
                        g_{1,1}f_{1,1}
                     \end{mat}
        \qquad
        FG=\rep{\composed{f}{g}}{B,B}=\begin{mat}
                        f_{1,1}g_{1,1}
                     \end{mat}
      \end{equation*}
      These two matrices are equal and so the compositions have the same
      effect on each vector in the space.
     \end{answer}
  \item  
    Why is matrix multiplication not defined as entry-wise multiplication?
    That would be easier, and commutative too.
    \begin{answer}
      It would not represent linear map composition; 
      \nearbytheorem{th:MatMultRepComp} would fail.  
    \end{answer}
  \recommended \item \label{exer:NicePropsMatMult} 
    \begin{exparts}
      \partsitem Prove that $H^pH^q=H^{p+q}$ and $(H^p)^q=H^{pq}$ 
        for positive integers \( p,q \).
      \partsitem Prove that $(rH)^p=r^p\cdot H^p$ 
        for any positive integer \( p \) and scalar \( r\in\Re \).
    \end{exparts}
    \begin{answer}
      Each follows easily from the associated map fact.
      For instance, $p$ applications of the transformation $h$, following $q$
      applications, is simply $p+q$ applications.
    \end{answer}
  \recommended \item \label{exer:MoreNicePropsMatMult} 
    \begin{exparts}
      \partsitem How does matrix multiplication interact with 
        scalar multiplication:~is \( r(GH)=(rG)H \)?
        Is \( G(rH)=r(GH) \)?
      \partsitem  How does matrix multiplication interact with 
        linear combinations:~is \( F(rG+sH)=r(FG)+s(FH) \)?
        Is $(rF+sG)H=rFH+sGH$?
    \end{exparts}
    \begin{answer}  
      Although we can do these by going through the indices, they
      are best understood in terms of the represented maps.
      That is, fix spaces and bases so that the matrices 
      represent linear maps $f,g,h$.
       \begin{exparts}
        \partsitem Yes;  we have both 
          $r\cdot (\composed{g}{h})\,(\vec{v})
           =r\cdot g(\,h(\vec{v})\,)
           =\composed{(r\cdot g)}{h}\,(\vec{v})$
          and 
          $\composed{g}{(r\cdot h)}\,(\vec{v})=g(\,r\cdot h(\vec{v})\,)
            =r\cdot g(h(\vec{v}))
            =r\cdot (\composed{g}{h})\,(\vec{v})$
          (the second equality holds because of the linearity of $g$).
        \partsitem Both answers are yes.
          First, $\composed{f}{(rg+sh)}$ and
          $r\cdot(\composed{f}{g})+s\cdot(\composed{f}{h})$ both send 
          $\vec{v}$ to $r\cdot f(g(\vec{v}))+s\cdot f(h(\vec{v}))$; the
          calculation is as in the prior item
          (using the linearity of $f$ for the first one). 
          For the other, 
          $\composed{(rf+sg)}{h}$ and 
          $r\cdot(\composed{f}{h})+s\cdot(\composed{g}{h})$ both send
          $\vec{v}$ to $r\cdot f(h(\vec{v}))+s\cdot g(h(\vec{v}))$.
      \end{exparts}  
    \end{answer}
  \item \label{exer:TranspAndMult} 
    We can ask how the matrix product 
    operation interacts with the transpose operation. 
    \begin{exparts}  
      \partsitem Show that \( \trans{(GH)}=\trans{H}\trans{G} \).
      \partsitem A square matrix is 
        \definend{symmetric}\index{symmetric matrix}%
        \index{matrix!symmetric} if each 
        \( i,j \) entry equals the
        \( j,i \) entry, that is, if the matrix equals its own transpose.
        Show that
        the matrices \( H\trans{H} \) and \( \trans{H}H \) are symmetric.
      %\partsitem Is every symmetric matrix of that form?
    \end{exparts}
    \begin{answer}
      We have not seen a map interpretation of the transpose operation, so
      we will verify these by considering the entries.
      \begin{exparts}
        \partsitem  The \( i,j \) entry of \( \trans{GH} \) is the $j,i$ entry
          of $GH$, which is the dot product of the
          \( j \)-th row of \( G \) and the \( i \)-th column of \( H \).
          The \( i,j \) entry of \( \trans{H}\trans{G} \) is the dot product of
          the \( i \)-th row of \( \trans{H} \) and the \( j \)-th column of
          \( \trans{G} \), which is the
          dot product of the \( i \)-th column of \( H \) and the
          \( j \)-th row of \( G \).
          Dot product is commutative and so these two are equal.
        \partsitem By the prior item each equals its transpose, e.g.,
          $\trans{(H\trans{H})}=\trans{\trans{H}}\trans{H}=H\trans{H}$.
      \end{exparts}  
    \end{answer}
  \recommended \item
    Rotation of vectors in \( \Re^3 \) about an axis is a linear map.
    Show that linear maps do not commute by     
    showing geometrically that rotations do not commute.
    \begin{answer}
      Consider \( \map{r_x,r_y}{\Re^3}{\Re^3} \) rotating all vectors
      \( \pi/2 \)~radians
      counterclockwise about the \( x \) and \( y \)~axes 
      (counterclockwise in the sense that a person whose head is at
      \( \vec{e}_1 \) or \( \vec{e}_2 \) and whose feet are at the origin
      sees, when looking toward the origin, the rotation as
      counterclockwise).
      \begin{center}  \small
        \includegraphics{ch3.78}
        \qquad
        \includegraphics{ch3.79}
      \end{center}
      Rotating $r_x$ first and then $r_y$ is different than
      rotating $r_y$ first and then $r_x$.
      In particular, $r_x(\vec{e}_3)=-\vec{e}_2$
      so $\composed{r_y}{r_x}(\vec{e}_3)=-\vec{e}_2$,
      while $r_y(\vec{e}_3)=\vec{e}_1$ so
      $\composed{r_x}{r_y}(\vec{e}_3)=\vec{e}_1$,
      and hence the maps do not commute.
    \end{answer}
  \item \label{exer:MatProdPropsSpAndBas}
    In the proof of \nearbytheorem{th:MatMultWellBehaved} we used some maps.
    What are the domains and codomains?
    \begin{answer}
      It doesn't matter (as long as the spaces have the appropriate 
     dimensions).

      For associativity, 
      suppose that $F$ is $\nbym{m}{r}$, that $G$ is $\nbym{r}{n}$, and
      that $H$ is $\nbym{n}{k}$.
      We can take any $r$~dimensional space, 
      any $m$~dimensional space, any $n$~dimensional space, and any
      $k$~dimensional space\Dash for instance, 
      $\Re^r$, $\Re^m$, $\Re^n$, and $\Re^k$ will do.
      We can take any bases $A$, $B$, $C$, and $D$, for those spaces.
      Then, 
      with respect to $C,D$ the matrix $H$ represents a linear map $h$,
      with respect to $B,C$ the matrix $G$ represents a $g$,
      and with respect to $A,B$ the matrix $F$ represents an $f$.
      We can use those maps in the proof.
      
      The second half is similar, except that we add $G$ and $H$ 
      and so we must take them to represent maps with the same domain 
      and codomain.
    \end{answer}
  \item \label{exer:RankProdLeqRankFacts}
    How does matrix rank interact with matrix multiplication?
    \begin{exparts}
      \partsitem Can the product of rank \( n \) matrices have rank less 
        than \( n \)?
        Greater?
      \partsitem  Show that the rank of the product of two matrices is less 
        than or equal to the minimum of the rank of each factor.
    \end{exparts}
    \begin{answer}
     \begin{exparts}
      \partsitem The product of rank \( n \) matrices can have rank less 
        than or equal to \( n \) but not greater than \( n \).

        To see that the rank can fall,
        consider the maps \( \map{\pi_x,\pi_y}{\Re^2}{\Re^2} \) projecting onto
        the axes. 
        Each is rank one but their composition
        $\composed{\pi_x}{\pi_y}$, which is the zero map, is rank zero.
        That translates over to matrices representing those 
        maps in this way.
        \begin{equation*}
          \rep{\pi_x}{\stdbasis_2,\stdbasis_2}\cdot\rep{\pi_y}{\stdbasis_2,\stdbasis_2}=
          \begin{mat}[r]
            1  &0  \\
            0  &0
          \end{mat}
          \begin{mat}[r]
            0  &0  \\
            0  &1
          \end{mat}=
          \begin{mat}[r]
            0  &0  \\
            0  &0
          \end{mat}
        \end{equation*}

        To prove that the product of rank \( n \) matrices cannot have rank
        greater than \( n \), we can  apply the map result that the image of a
        linearly dependent set is linearly dependent.
        That is, if \( \map{h}{V}{W} \) and \( \map{g}{W}{X} \) both have rank
        \( n \) then a set in the range 
        \( \rangespace{\composed{g}{h}}  \) of size
        larger than \( n \) is the image under \( g \) of a set in \( W \) of
        size larger than \( n \) and so is linearly dependent
        (since the rank of \( h \) is \( n \)).
        Now, the image of a linearly dependent set is dependent, so any set of
        size larger than \( n \) in the range is dependent.
        (By the way, observe that the rank of \( g \) was not mentioned.
        See the next part.)  
       \partsitem Fix spaces and bases and consider the associated linear maps
          \( f \) and \( g \).
          Recall that the dimension of the image of a map (the map's rank) is
          less than or equal to the dimension of the domain, and consider
          the arrow diagram.
          \begin{equation*}
            \begin{matrix}
              V &\mapsunder{f} &\rangespace{f} &\mapsunder{g}
                &\rangespace{\composed{g}{f}}
            \end{matrix}
          \end{equation*}
          First, the image of \( \rangespace{f} \) must have dimension
          less than or equal to the dimension of \( \rangespace{f} \),
          by the prior sentence.
          On the other hand, \( \rangespace{f} \) is a subset of
          the domain of \( g \), and thus its image has dimension less than
          or equal the dimension of the domain of \( g \).
          Combining those two,
          the rank of a composition is less than or equal to the minimum
          of the two ranks.

          The matrix fact follows immediately.  
     \end{exparts}
    \end{answer}
  \item 
    Is `commutes with' an equivalence relation among $\nbyn{n}$
    matrices?
    \begin{answer}
      The `commutes with' relation is reflexive and symmetric.
      However, it is not transitive:~for instance, with
      \begin{equation*}
        G=\begin{mat}[r]
          1  &2  \\
          3  &4
        \end{mat}
        \quad
        H=\begin{mat}[r]
          1  &0  \\
          0  &1
        \end{mat}
        \quad
        J=\begin{mat}[r]
          5  &6  \\
          7  &8
        \end{mat}
      \end{equation*}
      $G$ commutes with $H$ and $H$ commutes with $J$, but $G$ does not
      commute with $J$.    
    \end{answer}
  \recommended \item \label{exer:ZeroDivisor}
    \textit{(We will use this exercise in the Matrix Inverses exercises.)}
    Here is another property of matrix multiplication that might be
    puzzling at first sight.
    \begin{exparts}
      \partsitem Prove that the composition of the projections
        \( \map{\pi_x,\pi_y}{\Re^3}{\Re^3} \)
        onto the \( x \) and \( y \)~axes
        is the zero map despite that neither one is itself the zero map.
      \partsitem Prove that the composition of the derivatives
        \( \map{d^2/dx^2,\,d^3/dx^3}{\polyspace_4}{\polyspace_4} \)
        is the zero map despite that neither is the zero map.
      \partsitem Give a matrix equation representing the first fact.
      \partsitem Give a matrix equation representing the second.
    \end{exparts}
    When two things multiply to give zero despite that neither is zero
    we say that each is
    a \definend{zero divisor}.\index{zero divisor}%
    \index{zero!divisor}
    \begin{answer}
      \begin{exparts}
        \partsitem Either of these.
          \begin{equation*}
            \colvec{x \\ y \\ z}
              \mapsunder{\pi_x}
            \colvec{x \\ 0 \\ 0}
              \mapsunder{\pi_y}
            \colvec[r]{0 \\ 0 \\ 0}
            \qquad
            \colvec{x \\ y \\ z}
              \mapsunder{\pi_y}
            \colvec{0 \\ y \\ 0}
              \mapsunder{\pi_x}
            \colvec[r]{0 \\ 0 \\ 0}
          \end{equation*}
        \partsitem The composition is the fifth derivative map
          $d^5/dx^5$ on the space of fourth-degree polynomials.
        \partsitem With respect to the natural bases,
          \begin{equation*}
            \rep{\pi_x}{\stdbasis_3,\stdbasis_3}=\begin{mat}[r]
              1  &0  &0  \\
              0  &0  &0  \\
              0  &0  &0
            \end{mat}
            \qquad
            \rep{\pi_y}{\stdbasis_3,\stdbasis_3}=\begin{mat}[r]
              0  &0  &0  \\
              0  &1  &0  \\
              0  &0  &0
            \end{mat}
          \end{equation*}
          and their product (in either order) is the zero matrix.
        \partsitem Where \( B=\sequence{1,x,x^2,x^3,x^4} \),
          \begin{equation*}
            \rep{\frac{d^2}{dx^2}}{B,B}=\begin{mat}[r]
              0  &0  &2  &0  &0  \\
              0  &0  &0  &6  &0  \\
              0  &0  &0  &0  &12 \\
              0  &0  &0  &0  &0  \\
              0  &0  &0  &0  &0
            \end{mat}
            \qquad
            \rep{\frac{d^3}{dx^3}}{B,B}=\begin{mat}[r]
              0  &0  &0  &6  &0  \\
              0  &0  &0  &0  &24 \\
              0  &0  &0  &0  &0  \\
              0  &0  &0  &0  &0  \\
              0  &0  &0  &0  &0
            \end{mat}
          \end{equation*}
          and their product (in either order) is the zero matrix.
      \end{exparts}  
    \end{answer}
  \item 
    Show that, for square matrices, \( (S+T)(S-T) \) need not equal
    \( S^2-T^2 \).
    \begin{answer}
      Note that \( (S+T)(S-T)=S^2-ST+TS-T^2 \), so a reasonable try is to look
      at matrices that do not commute so that $-ST$ and $TS$ don't 
      cancel:~with 
      \begin{equation*}
        S=\begin{mat}[r]
            1  &2  \\
            3  &4
          \end{mat}
        \quad
        T=\begin{mat}[r]
            5  &6  \\
            7  &8
          \end{mat}
      \end{equation*}
      we have the desired inequality.
      \begin{equation*}
        (S+T)(S-T)=\begin{mat}[r]
            -56  &-56  \\
            -88  &-88
          \end{mat}
        \qquad
        S^2-T^2=\begin{mat}[r]
            -60  &-68  \\
            -76  &-84
          \end{mat}
      \end{equation*}    
    \end{answer}
  \recommended \item \label{exer:IdMat} 
    Represent the identity transformation 
    $\map{\identity}{V}{V}$ with respect to $B,B$ for any
    basis $B$.
    This is the \definend{identity matrix}\index{matrix!identity} $I$.
    Show that this matrix plays the role in matrix multiplication that the
    number $1$ plays in real number multiplication:~$HI=IH=H$ (for all matrices
    $H$ for which the product is defined).
    \begin{answer}
      Because the identity map acts on the basis $B$ as
      $\vec{\beta}_1\mapsto\vec{\beta}_1$, \ldots,
      $\vec{\beta}_n\mapsto\vec{\beta}_n$,
      the representation is this.
      \begin{equation*}
        \begin{mat}
          1  &0  &0  &  &0 \\
          0  &1  &0  &  &0 \\
          0  &0  &1  &  &0 \\
             &   &   &\ddots  \\
          0  &0  &0  &  &1
        \end{mat}
      \end{equation*}
      The second part of the question is obvious from 
      \nearbytheorem{th:MatMultRepComp}.
    \end{answer}
  \item 
    In real number algebra, quadratic equations have at most two solutions.
    That is not so with matrix algebra.
    Show that the \( \nbyn{2} \) matrix equation \( T^2=I \)
    has more than two solutions,
    where $I$ is the identity matrix
    (this matrix has ones in its $1,1$ and $2,2$ entries and zeroes 
    elsewhere;~see \nearbyexercise{exer:IdMat}).
    \begin{answer}
      Here are four solutions.
      \begin{equation*}
        T=\begin{mat}[r]
          \pm 1  &0  \\
          0      &\pm 1
        \end{mat}
      \end{equation*}  
    \end{answer}
  \item 
    \begin{exparts}
      \partsitem Prove that for any \( \nbyn{2} \) matrix \( T \) 
        there are scalars
        \( c_0,\dots,c_4 \) that are not all $0$ such that the combination
        $c_4T^4+c_3T^3+c_2T^2+c_1T+c_0I$
        is the zero matrix
        (where $I$ is the $\nbyn{2}$ identity matrix, with $1$'s in its
        $1,1$ and $2,2$ entries and zeroes elsewhere;~see
        \nearbyexercise{exer:IdMat}).
      \partsitem Let \( p(x) \) be a polynomial 
        \( p(x)=c_nx^n+\dots+c_1x+c_0 \).
        If \( T \) is a square matrix we define \( p(T) \) to be the matrix
        \( c_nT^n+\dots+c_1T+I \)
        (where $I$ is the appropriately-sized identity matrix).
        Prove that for any square matrix there is a polynomial such that
        \( p(T) \) is the zero matrix.
      \partsitem The \definend{minimal polynomial}\index{minimal polynomial}%
        \index{matrix!minimal polynomial}
        $m(x)$ of a square matrix is the
        polynomial of least degree, and with leading coefficient \( 1 \),
        such that \( m(T) \) is the zero matrix.
        Find the minimal polynomial of this matrix.
        \begin{equation*}
          \begin{mat}[r]
            \sqrt{3}/2  &-1/2       \\
            1/2         &\sqrt{3}/2
          \end{mat}
        \end{equation*}
        (This is the representation with respect to $\stdbasis_2,\stdbasis_2$, 
        the standard basis, of a rotation through $\pi/6$~radians
        counterclockwise.)
    \end{exparts}
    \begin{answer} 
      \begin{exparts}
         \partsitem The vector space \( \matspace_{\nbyn{2}} \) has dimension
           four. 
           The set \( \set{T^4,\dots,T,I} \) has five
           elements and thus is linearly dependent.
         \partsitem Where \( T \) is \( \nbyn{n} \), 
           generalizing the argument from the
           prior item shows that there is such a polynomial of degree 
           \( n^2 \) or less,
           since \( \set{T^{n^2},\dots,T,I} \) is a \( n^2+1 \)-member
           subset of the $n^2$-dimensional space $\matspace_{\nbyn{n}}$.
         \partsitem First compute the powers
           \begin{equation*}
              T^2=
              \begin{mat}[r]
                 1/2         &-\sqrt{3}/2  \\
                 \sqrt{3}/2  &1/2
              \end{mat} 
              \qquad
              T^3=
              \begin{mat}[r]
                 0           &-1           \\
                 1           &0
              \end{mat}
             \qquad
              T^4=
              \begin{mat}[r]
                 -1/2        &-\sqrt{3}/2  \\
                 \sqrt{3}/2  &-1/2
              \end{mat}
           \end{equation*}
           (observe that rotating by $\pi/6$ three times results in a 
           rotation by $\pi/2$, which is indeed what $T^3$ represents).
           Then set \( c_4T^4+c_3T^3+c_2T^2+c_1T+c_0I \) 
           equal to the zero matrix 
           \begin{multline*}
             \begin{mat}[r]
                 -1/2        &-\sqrt{3}/2  \\
                 \sqrt{3}/2  &-1/2
              \end{mat}c_4
              +\begin{mat}[r]
                 0           &-1           \\
                 1           &0
              \end{mat}c_3
              +\begin{mat}[r]
                 1/2         &-\sqrt{3}/2  \\
                 \sqrt{3}/2  &1/2
              \end{mat}c_2
              +\begin{mat}[r]
                 \sqrt{3}/2  &-1/2       \\
                 1/2         &\sqrt{3}/2
               \end{mat}c_1
              +\begin{mat}[r]
                 1  &0  \\
                 0  &1 
               \end{mat}c_0                          \\
               =\begin{mat}[r]
                  0  &0  \\
                  0  &0 
               \end{mat}                     
           \end{multline*}
           to get this linear system.
           \begin{equation*}
             \begin{linsys}{5}
                -(1/2)c_4         &  &  &+ &(1/2)c_2
                    &+ &(\sqrt{3}/2)c_1  &+  &c_0  &=  &0      \\
                -(\sqrt{3}/2)c_4  &- &c_3 &- &(\sqrt{3}/2)c_2
                    &- &(1/2)c_1         &   &          &=  &0        \\
                 (\sqrt{3}/2)c_4  &+ &c_3 &+ &(\sqrt{3}/2)c_2
                    &+ &(1/2)c_1  &   &            &=  &0      \\
                -(1/2)c_4              &  &  &+ &(1/2)c_2
                    &+ &(\sqrt{3}/2)c_1  &+  &c_0  &=  &0
              \end{linsys} 
           \end{equation*}
           Apply Gaussian reduction.
           \begin{eqnarray*}
             &\grstep{-\rho_1+\rho_4}
             \;\grstep{\rho_2+\rho_3}
             &\begin{linsys}{5}
                -(1/2)c_4         &  &  &+ &(1/2)c_2
                    &+ &(\sqrt{3}/2)c_1  &+  &c_0  &=  &0      \\
                -(\sqrt{3}/2)c_4  &- &c_3 &- &(\sqrt{3}/2)c_2
                    &- &(1/2)c_1         &   &          &=  &0   \\
                                  &  &  &  &     
                    &  &                 &   &0    &=  &0      \\
                                  &  &  &  &     
                    &  &                 &   &0    &=  &0      
             \end{linsys}                                               \\
             &\;\grstep{-\sqrt{3}\rho_1+\rho_2}                   
             &\begin{linsys}{5}
                -(1/2)c_4         &  &  &+ &(1/2)c_2
                    &+ &(\sqrt{3}/2)c_1  &+  &c_0  &=  &0      \\
                                  &- &c_3 &- &\sqrt{3}c_2
                    &- &2c_1             &-  &\sqrt{3}c_0  &=  &0   \\
                                  &  &  &  &     
                    &  &                 &   &0    &=  &0      \\
                                  &  &  &  &     
                    &  &                 &   &0    &=  &0      
              \end{linsys} 
           \end{eqnarray*}
           Setting \( c_4 \), \( c_3 \), and \( c_2 \) to zero 
           makes \( c_1 \) and \( c_0 \) also come out to be zero so
           no degree one or degree zero polynomial will do.
           Setting \( c_4 \) and \( c_3 \) to zero (and $c_2$ to one)
           gives a linear system 
           \begin{equation*}
             \begin{linsys}{3}
               (1/2)     &+  &(\sqrt{3}/2)c_1  &+  &c_0          &=  &0 \\
               -\sqrt{3} &-  &2c_1             &-  &\sqrt{3}c_0  &=  &0
             \end{linsys}
           \end{equation*}
           with solution $c_1=-\sqrt{3}$ and $c_0=1$.
           Conclusion:~the polynomial 
           $m(x)=x^2-\sqrt{3}x+1$
           is minimal for the matrix $T$.
      \end{exparts}  
    \end{answer}
  \item 
    The infinite-dimensional space $\polyspace$ 
    of all finite-degree polynomials
    gives a memorable example of the non-commutativity of
    linear maps.
    Let \( \map{d/dx}{\polyspace}{\polyspace} \) be the usual derivative
    and let \( \map{s}{\polyspace}{\polyspace} \) be the \definend{shift} map.
    \begin{equation*}
      a_0+a_1x+\dots+a_nx^n
      \;\mapsunder{s}\;
      0+a_0x+a_1x^2+\dots+a_nx^{n+1}
    \end{equation*}
    Show that the two maps don't commute
    \( \composed{d/dx}{s}\neq \composed{s}{d/dx} \); in fact, 
    not only is \( (\composed{d/dx}{s})-(\composed{s}{d/dx}) \) not
    the zero map, it is the identity map.
    \begin{answer}
      The check is routine:
      \begin{equation*}
        a_0+a_1x+\dots+a_nx^n\mapsunder{s}a_0x+a_1x^2+\dots+a_nx^{n+1}
                  \mapsunder{d/dx} a_0+2a_1x+\dots+(n+1)a_nx^n
      \end{equation*}
      while
      \begin{equation*}
        a_0+a_1x+\dots+a_nx^n
                  \mapsunder{d/dx} a_1+\dots+na_nx^{n-1}
                  \mapsunder{s}a_1x+\dots+a_nx^n
      \end{equation*}
      so that under the map $(\composed{d/dx}{s})-(\composed{s}{d/dx})$ 
      we have 
        $a_0+a_1x+\dots+a_nx^n
                  \mapsto
                  a_0+a_1x+\dots+a_nx^n$.
    \end{answer}
  \item 
    Recall the notation for the sum of the sequence of numbers
    \( a_1, a_2, \dots, a_n \).
    \begin{equation*}
      \sum_{i=1}^{n}a_i=a_1+a_2+\dots+a_n
    \end{equation*}
    In this notation, the \( i,j \) entry of the product of \( G \) and
    \( H \) is this.
    \begin{equation*}
       p_{i,j}=\sum_{k=1}^{r} g_{i,k}h_{k,j}
    \end{equation*}
    Using this notation,
    \begin{exparts}
       \partsitem reprove that matrix multiplication is associative;
       \partsitem reprove \nearbytheorem{th:MatMultRepComp}.
    \end{exparts}
    \begin{answer}
       \begin{exparts}
         \partsitem Tracing through the remark at the end of the subsection
           gives that the \( i,j \) entry of \( (FG)H \) is this
           \begin{multline*}
             \sum_{t=1}^s\bigl(\sum_{k=1}^{r} f_{i,k}g_{k,t}\bigr)h_{t,j}
             =\sum_{t=1}^s\sum_{k=1}^{r} (f_{i,k}g_{k,t})h_{t,j} 
             =\sum_{t=1}^s\sum_{k=1}^{r} f_{i,k}(g_{k,t}h_{t,j})  \\ 
             =\sum_{k=1}^{r}\sum_{t=1}^s f_{i,k}(g_{k,t}h_{t,j}) 
             =\sum_{k=1}^{r}f_{i,k}\bigl(\sum_{t=1}^s g_{k,t}h_{t,j}\bigr)
           \end{multline*}
           (the first equality comes from using 
           the distributive law to multiply through
           the $h$'s, the second equality is the associative law for real
           numbers, the third is the commutative law for reals,
           and the fourth equality follows on using the distributive law to
           factor the $f$'s out),
           which is the \( i,j \) entry of \( F(GH) \).
         \partsitem The \( k \)-th component of \( h(\vec{v})\) is
           \begin{equation*}
             \sum_{j=1}^n h_{k,j}v_j
           \end{equation*}
           and so the \( i \)-th component of 
           \( \composed{g}{h}\,(\vec{v}) \) is this
           \begin{multline*}
              \sum_{k=1}^r g_{i,k}\bigl(\sum_{j=1}^n h_{k,j}v_j\bigr) 
              =\sum_{k=1}^r \sum_{j=1}^n g_{i,k}h_{k,j}v_j   
              =\sum_{k=1}^r \sum_{j=1}^n (g_{i,k}h_{k,j})v_j
                                                              \\ 
              =\sum_{j=1}^n \sum_{k=1}^r (g_{i,k}h_{k,j})v_j 
              =\sum_{j=1}^n (\sum_{k=1}^r g_{i,k}h_{k,j})\,v_j
           \end{multline*}
           (the first equality holds by using the distributive law to multiply
           the $g$'s through, the second equality represents the use of 
           associativity of reals, the third follows by commutativity of
           reals, and the fourth comes from using 
           the distributive law to factor the $v$'s out).
       \end{exparts} 
    \end{answer}
\end{exercises}





















\subsection{Mechanics of Matrix Multiplication}
In this subsection we consider matrix multiplication as a mechanical process, 
putting aside for the moment any implications about the underlying maps.

The striking thing about matrix multiplication is the
way rows and columns combine.
The \( i,j \) entry of the matrix product is the dot product of
row~$i$ of the left matrix with column~$j$
of the right one.
For instance,
here a second row and a third column combine to make a $2,3$~entry.
\begin{equation*}
\setlength{\fboxsep}{1.5pt}
    \begin{mat}
       \begin{array}{@{}cc@{}} 1  &1 \end{array}                         \\ 
       \text{\highlight{\( \begin{array}{@{}cc@{}}  0  &1  \end{array} \)}}   \\  
       \begin{array}{@{}cc@{}} 1  &0 \end{array}
    \end{mat}
    \begin{mat}
      \begin{array}{@{}c@{}}  4  \\  5  \end{array}
      &\begin{array}{@{}c@{}}  6  \\  7  \end{array}
      &\text{\highlight{ \(\begin{array}{@{}c@{}}  8  \\  9  \end{array}\) }}
      &\begin{array}{@{}c@{}}  2  \\  3  \end{array}
    \end{mat}
  =
    \begin{mat}[r]
      9  &13   &17                      &5  \\
      5  &7    &\text{\highlight{ \(9\) }}   &3  \\
      4  &6    &8                       &2
    \end{mat}
\end{equation*}
We can view this as the left matrix acting
by multiplying its rows, one at a time, into the columns of the right matrix.
Or, another perspective 
is that the right matrix uses its columns to
act on the left matrix's rows.
Below, we will examine actions from the left and from the right for some
simple matrices.

The action of a zero matrix is easy.

\begin{example}
Multiplying by an appropriately-sized zero matrix
from the left or from the right
results in a zero matrix.
\begin{equation*}
    \begin{mat}[r]
       0  &0  \\
       0  &0
    \end{mat}
    \begin{mat}[r]
       1  &3  &2   \\
      -1  &1  &-1
    \end{mat}
  =
    \begin{mat}[r]
       0  &0  &0   \\
       0  &0  &0
    \end{mat}
    \qquad
    \begin{mat}[r]
       2  &3  \\
       1  &4
    \end{mat}
    \begin{mat}[r]
       0  &0  \\
       0  &0
    \end{mat}
  =
    \begin{mat}[r]
       0  &0  \\
       0  &0
    \end{mat}
\end{equation*}
\end{example}

The next easiest to understand matrices, after the zero matrices,
are the ones with a single nonzero entry.

\begin{definition}  \label{df:UnitMatrix}
%<*df:UnitMatrix>
A matrix with all $0$'s except for a $1$ in the \( i,j \) entry
is an \( i,j \) \definend{unit}\index{matrix!unit}\index{unit matrix} 
matrix.
%</df:UnitMatrix>
\end{definition}

\begin{example}
This is the \( 1,2\, \) unit matrix with three rows and two columns,
multiplying from the left.
\begin{equation*}
    \begin{mat}[r]
       0  &1  \\
       0  &0  \\
       0  &0
    \end{mat}
    \begin{mat}[r]
       5  &6  \\
       7  &8
    \end{mat}
  =
    \begin{mat}[r]
       7  &8  \\
       0  &0  \\
       0  &0
    \end{mat}
\end{equation*}
Acting
from the left, an \( i,j \) unit matrix copies row~\( j \) of
the multiplicand into row~\( i \) of the result.
From the right an \( i,j \) unit matrix picks out column~\( i \) of
the multiplicand and copies it into column~\( j \) of the result.
\begin{equation*}
    \begin{mat}[r]
       1  &2  &3  \\
       4  &5  &6  \\
       7  &8  &9
    \end{mat}
    \begin{mat}[r]
       0  &1  \\
       0  &0  \\
       0  &0
    \end{mat}
  =
    \begin{mat}[r]
       0  &1  \\
       0  &4  \\
       0  &7
    \end{mat}
\end{equation*}
\end{example}

\begin{example}
Rescaling these matrices simply rescales the result. 
This is the action from the left of the matrix that is twice the one in the
prior example.
\begin{equation*}
    \begin{mat}[r]
       0  &2  \\
       0  &0  \\
       0  &0
    \end{mat}
    \begin{mat}[r]
       5  &6  \\
       7  &8
    \end{mat}
  =
    \begin{mat}[r]
      14  &16 \\
       0  &0  \\
       0  &0
    \end{mat}
\end{equation*}
And this is the action of the matrix that is $-3$~times the one 
from the prior example.
\begin{equation*}
    \begin{mat}[r]
       1  &2  &3  \\
       4  &5  &6  \\
       7  &8  &9
    \end{mat}
    \begin{mat}[r]
       0  &-3 \\
       0  &0  \\
       0  &0
    \end{mat}
  =
    \begin{mat}[r]
       0  &-3  \\
       0  &-12 \\
       0  &-21
    \end{mat}
\end{equation*}
\end{example}

Next in complication are matrices with two nonzero entries.
There are two cases.
If a left-multiplier has entries in different rows then their actions 
don't interact.

\begin{example}
\begin{align*}
    \begin{mat}[r]
       1  &0  &0  \\
       0  &0  &2  \\
       0  &0  &0
    \end{mat}
    \begin{mat}[r]
       1  &2  &3  \\
       4  &5  &6  \\
       7  &8  &9
    \end{mat}
  &=(  \begin{mat}[r]
          1  &0  &0  \\
          0  &0  &0  \\
          0  &0  &0
       \end{mat}
    +
       \begin{mat}[r]
          0  &0  &0  \\
          0  &0  &2  \\
          0  &0  &0
       \end{mat}     )
      \begin{mat}[r]
         1  &2  &3  \\
         4  &5  &6  \\
         7  &8  &9
      \end{mat}                      \\
  &=\begin{mat}[r]
       1  &2  &3  \\
       0  &0  &0  \\
       0  &0  &0
    \end{mat}
  +
    \begin{mat}[r]
       0  &0  &0  \\
      14  &16 &18 \\
       0  &0  &0
    \end{mat}                        \\
  &=\begin{mat}[r]
       1  &2  &3  \\
      14  &16 &18 \\
       0  &0  &0
    \end{mat}
\end{align*}
\end{example}

\par\noindent But if the left-multiplier's nonzero entries are in the same row
then that row of the result is a combination.
\begin{example}
\begin{align*}
    \begin{mat}[r]
       1  &0  &2  \\
       0  &0  &0  \\
       0  &0  &0
    \end{mat}
    \begin{mat}[r]
       1  &2  &3  \\
       4  &5  &6  \\
       7  &8  &9
    \end{mat}
  &=(  \begin{mat}[r]
          1  &0  &0  \\
          0  &0  &0  \\
          0  &0  &0
       \end{mat}
    +
       \begin{mat}[r]
          0  &0  &2  \\
          0  &0  &0  \\
          0  &0  &0
       \end{mat}     )
      \begin{mat}[r]
         1  &2  &3  \\
         4  &5  &6  \\
         7  &8  &9
      \end{mat}                     \\
  &=\begin{mat}[r]
       1  &2  &3  \\
       0  &0  &0  \\
       0  &0  &0
    \end{mat}
  +
    \begin{mat}[r]
      14  &16 &18 \\
       0  &0  &0  \\
       0  &0  &0
    \end{mat}                      \\
  &=\begin{mat}[r]
      15  &18 &21 \\
       0  &0  &0  \\
       0  &0  &0
    \end{mat}
\end{align*}
\end{example}

\par\noindent Right-multiplication acts in the same way, but with columns.

These observations about simple matrices
extend to arbitrary ones.

\begin{example}
Consider the columns of the product of two $\nbyn{2}$~matrices.
\begin{equation*}
  \begin{mat}
    g_{1,1}  &g_{1,2}    \\
    g_{2,1}  &g_{2,2}    
  \end{mat}
  \begin{mat}
    h_{1,1}  &h_{1,2}    \\
    h_{2,1}  &h_{2,2}    
  \end{mat}   
  =
  \begin{mat}
    g_{1,1}h_{1,1}+g_{1,2}h_{2,1}  &g_{1,1}h_{1,2}+g_{1,2}h_{2,2} \\
    g_{2,1}h_{1,1}+g_{2,2}h_{2,1}  &g_{2,1}h_{1,2}+g_{2,2}h_{2,2} 
  \end{mat}
\end{equation*}
Each column is the result of multiplying $G$ 
by the corresponding column of $H$.
\begin{equation*}
  G\colvec{h_{1,1} \\ h_{2,1}}  
  =
  \colvec{g_{1,1}h_{1,1}+g_{1,2}h_{2,1} \\ g_{2,1}h_{1,1}+g_{2,2}h_{2,1}}
  \qquad
  G\colvec{h_{1,2}  \\ h_{2,2}}    
  =\colvec{g_{1,1}h_{1,2}+g_{1,2}h_{2,2} \\ g_{2,1}h_{1,2}+g_{2,2}h_{2,2}} 
\end{equation*}
\end{example}


\begin{lemma} \label{lm:ColsAndRowsInMatrixMult}
%<*lm:ColsAndRowsInMatrixMult>
In a product of two matrices $G$ and $H$,
the columns of $GH$ are formed by taking $G$ times the columns of $H$
    \begin{equation*}
      G\cdot \begin{pmat}{c|@{\hspace{1em}}c@{\hspace{1em}}|c}
         % \MTFlushSpaceAbove
         % \vdotswithin{\vec{h}_1}       &        &\vdotswithin{\vec{h}_n}   
         % \MTFlushSpaceBelow
         \vdots       &        &\vdots   \\
         \vec{h}_1    &\cdots  &\vec{h}_n \\ 
         \vdots       &        &\vdots 
       \end{pmat}
      =\begin{pmat}{c|@{\hspace{1em}}c@{\hspace{1em}}|c}
         \vdots             &        &\vdots    \\
         G\cdot \vec{h}_1   &\cdots  &G\cdot\vec{h}_n \\ 
         \vdots             &        &\vdots 
       \end{pmat}
    \end{equation*}
and the rows of $GH$ are formed by taking the rows of $G$ times $H$
    \begin{equation*}
      \begin{pmat}{c}
        \cdots\; \vec{g}_1 \;\cdots\rule[-1.25ex]{0pt}{2ex}   \\ 
         \hline
         \vdots \\[.5ex]  
         \hline
        \rule{0pt}{2.5ex}\cdots\; \vec{g}_r \;\cdots 
       \end{pmat}\cdot H
      =\begin{pmat}{c}
         \cdots\; \vec{g}_1\cdot H \;\cdots\rule[-1.25ex]{0pt}{2ex} \\ 
        \hline
         \vdots\rule[-1ex]{0pt}{2ex} \\[.5ex]   
        \hline
         \rule{0pt}{2.5ex}\cdots\; \vec{g}_r\cdot H \;\cdots
       \end{pmat}
    \end{equation*}
(ignoring the extra parentheses).
%</lm:ColsAndRowsInMatrixMult>
\end{lemma}

\begin{proof}
%<*pf:ColsAndRowsInMatrixMult>
We will check that in a product of $\nbyn{2}$ matrices, 
the rows of the product equal the
product of the rows of~$G$ with the entire matrix~$H$.
\begin{align*}
  \begin{mat}
    g_{1,1}  &g_{1,2}    \\
    g_{2,1}  &g_{2,2}    
  \end{mat}
  \begin{mat}
    h_{1,1}  &h_{1,2}    \\
    h_{2,1}  &h_{2,2}    
  \end{mat}
  &=\begin{mat}
    \rowvec{g_{1,1}  &g_{1,2}}H  \\ 
    \rowvec{g_{2,1}  &g_{2,2}}H    
  \end{mat}          \\
  &=
  \begin{mat}
    \rowvec{g_{1,1}h_{1,1}+g_{1,2}h_{2,1}  &g_{1,1}h_{1,2}+g_{1,2}h_{2,2}}    \\ 
    \rowvec{g_{2,1}h_{1,1}+g_{2,2}h_{2,1}  &g_{2,1}h_{1,2}+g_{2,2}h_{2,2}} 
  \end{mat}
\end{align*}
We will leave the more general check as an exercise.
%</pf:ColsAndRowsInMatrixMult>
\end{proof}

An application of those observations is that there is a matrix
that just copies out the rows and columns.

\begin{definition}  \label{df:MainDiagonal}
%<*df:MainDiagonal>
The \definend{main diagonal}\index{matrix!main diagonal}
(or \definend{principle diagonal} or \definend{diagonal}) of a square matrix
goes from the upper left to the lower right.
%</df:MainDiagonal>
\end{definition}

\begin{definition} \label{df:IdentityMatrix}
%<*df:IdentityMatrix>
An \definend{identity matrix\/}\index{identity!matrix}\index{matrix!identity}
is square and every entry is $0$ except for $1$'s in the main diagonal.
\begin{equation*}
   I_{\nbyn{n}}=
      \begin{mat}
        1  &0  &\ldots  &0  \\
        0  &1  &\ldots  &0  \\
           &\vdots          \\
        0  &0  &\ldots  &1
      \end{mat}
\end{equation*}
%</df:IdentityMatrix>
\end{definition}

\begin{example}
Here is the \( \nbyn{2} \) identity matrix leaving its multiplicand unchanged
when it acts from the right.
\begin{equation*}
    \begin{mat}[r]
       1  &-2 \\
       0  &-2 \\
       1  &-1 \\
       4  &3
    \end{mat}
    \begin{mat}[r]
       1  &0  \\
       0  &1  \\
    \end{mat}
  =
    \begin{mat}[r]
       1  &-2 \\
       0  &-2 \\
       1  &-1 \\
       4  &3
    \end{mat}
\end{equation*}
\end{example}

\begin{example}
Here the \( \nbyn{3} \) identity
leaves its multiplicand unchanged both from the left
\begin{equation*}
    \begin{mat}[r]
       1  &0  &0  \\
       0  &1  &0  \\
       0  &0  &1
    \end{mat}
    \begin{mat}[r]
       2  &3  &6  \\
       1  &3  &8  \\
      -7  &1  &0
    \end{mat}
  =
    \begin{mat}[r]
       2  &3  &6  \\
       1  &3  &8  \\
      -7  &1  &0
    \end{mat}
\end{equation*}
and from the right.
\begin{equation*}
    \begin{mat}[r]
       2  &3  &6  \\
       1  &3  &8  \\
      -7  &1  &0
    \end{mat}
    \begin{mat}[r]
       1  &0  &0  \\
       0  &1  &0  \\
       0  &0  &1
    \end{mat}
  =
    \begin{mat}[r]
       2  &3  &6  \\
       1  &3  &8  \\
      -7  &1  &0
    \end{mat}
\end{equation*}
\end{example}

\noindent In short, an identity matrix is the identity element 
of the set of $\nbyn{n}$ matrices with
respect to the operation of matrix multiplication.

We next see two ways to generalize the identity matrix.
The first is that if we relax the ones to arbitrary reals then
the resulting matrix will rescale whole rows or columns.

\begin{definition}  \label{df:DiagonalMatrix}
%<*df:DiagonalMatrix>
A \definend{diagonal matrix\/}\index{matrix!diagonal}\index{diagonal matrix}
is square and has $0$'s off the main diagonal.
\begin{equation*}
    \begin{mat}
       a_{1,1}  &0        &\ldots     &0     \\
       0        &a_{2,2}  &\ldots     &0     \\
                &\vdots                      \\
       0        &0        &\ldots     &a_{n,n}
    \end{mat}
\end{equation*}
%</df:DiagonalMatrix>
\end{definition}

\begin{example}
From the left, 
the action of multiplication by a diagonal matrix is to
rescales the rows.
\begin{equation*}
    \begin{mat}[r]
       2  &0   \\
       0  &-1
    \end{mat}
    \begin{mat}[r]
       2  &1  &4  &-1  \\
      -1  &3  &4  &4
    \end{mat}
  =
    \begin{mat}[r]
       4  &2  &8  &-2  \\
       1  &-3 &-4 &-4
    \end{mat}
\end{equation*}
From the right such a matrix rescales the columns.
\begin{equation*}
     \begin{mat}[r]
        1  &2  &1  \\
        2  &2  &2
     \end{mat}
    \begin{mat}[r]
      3  &0  &0  \\
      0  &2  &0  \\
      0  &0  &-2
    \end{mat}
  =
     \begin{mat}[r]
        3  &4  &-2 \\
        6  &4  &-4
     \end{mat}
\end{equation*}
\end{example}

The second generalization of identity matrices is that 
we can put a single one in each row and column
in ways other than putting them down the diagonal.

\begin{definition}  \label{df:PermutationMatrix}
%<*df:PermutationMatrix>
A \definend{permutation matrix}%
\index{permutation!matrix}\index{matrix!permutation}
is square and is all $0$'s except for
a single~$1$ in each row and column.
%</df:PermutationMatrix>
\end{definition}

\begin{example}
From the left these matrices permute rows.
\begin{equation*}
     \begin{mat}[r]
        0  &0  &1  \\
        1  &0  &0  \\
        0  &1  &0
     \end{mat}
     \begin{mat}[r]
        1  &2  &3  \\
        4  &5  &6  \\
        7  &8  &9
     \end{mat}
  =
     \begin{mat}[r]
        7  &8  &9  \\
        1  &2  &3  \\
        4  &5  &6
     \end{mat}
\end{equation*}
From the right they permute columns.
\begin{equation*}
     \begin{mat}[r]
        1  &2  &3  \\
        4  &5  &6  \\
        7  &8  &9
     \end{mat}
     \begin{mat}[r]
        0  &0  &1  \\
        1  &0  &0  \\
        0  &1  &0
     \end{mat}
  =
     \begin{mat}[r]
        2  &3  &1  \\
        5  &6  &4  \\
        8  &9  &7
     \end{mat}
\end{equation*}
\end{example}

We finish this subsection by applying these observations to get matrices 
that perform Gauss's Method and Gauss-Jordan reduction.

\begin{example}
We have seen how to produce a matrix that will rescale rows.
Multiplying by this diagonal matrix rescales the second row of
the other matrix by a factor of three.
\begin{equation*}
    \begin{mat}[r]
      1  &0  &0  \\
      0  &3  &0  \\
      0  &0  &1
    \end{mat}
    \begin{mat}[r]
      0  &2    &1  &1  \\
      0  &1/3  &1  &-1 \\
      1  &0    &2  &0
    \end{mat}
  =
    \begin{mat}[r]
      0  &2    &1  &1  \\
      0  &1    &3  &-3 \\
      1  &0    &2  &0
    \end{mat}
\end{equation*}
We have seen how to produce a matrix that will swap rows.
Multiplying by this permutation matrix swaps the 
first and third rows.
\begin{equation*}
    \begin{mat}[r]
      0  &0  &1  \\
      0  &1  &0  \\
      1  &0  &0
    \end{mat}
    \begin{mat}[r]
      0  &2    &1  &1  \\
      0  &1    &3  &-3 \\
      1  &0    &2  &0
    \end{mat}
  =
    \begin{mat}[r]
      1  &0    &2  &0  \\
      0  &1    &3  &-3 \\
      0  &2    &1  &1
    \end{mat}
\end{equation*}
\end{example}

To see how to perform a row combination, 
we observe something about those two examples.
The matrix that rescales the second row by a factor of three arises in this 
way from the identity.
\begin{equation*}
    \begin{mat}[r]
      1  &0  &0  \\
      0  &1  &0  \\
      0  &0  &1
    \end{mat}
  \grstep{3\rho_2}
    \begin{mat}[r]
      1  &0  &0  \\
      0  &3  &0  \\
      0  &0  &1
    \end{mat}
\end{equation*}
Similarly, the matrix that swaps first and third rows arises in this way.
\begin{equation*}
    \begin{mat}[r]
      1  &0  &0  \\
      0  &1  &0  \\
      0  &0  &1
    \end{mat}
  \grstep{\rho_1\leftrightarrow\rho_3}
    \begin{mat}[r]
      0  &0  &1  \\
      0  &1  &0  \\
      1  &0  &0
    \end{mat}
\end{equation*}

\begin{example}
The \( \nbyn{3} \) matrix that arises as
\begin{equation*}
    \begin{mat}[r]
      1  &0  &0  \\
      0  &1  &0  \\
      0  &0  &1
    \end{mat}
  \grstep{-2\rho_2+\rho_3}
    \begin{mat}[r]
      1  &0  &0  \\
      0  &1  &0  \\
      0  &-2 &1
    \end{mat}
\end{equation*}
will, when it acts from the left,
perform the combination operation \( -2\rho_2+\rho_3 \).
\begin{equation*}
    \begin{mat}[r]
      1  &0  &0  \\
      0  &1  &0  \\
      0  &-2 &1
    \end{mat}
    \begin{mat}[r]
      1  &0    &2  &0  \\
      0  &1    &3  &-3 \\
      0  &2    &1  &1
    \end{mat}
  =
    \begin{mat}[r]
      1  &0    &2  &0  \\
      0  &1    &3  &-3 \\
      0  &0    &-5 &7
    \end{mat}
\end{equation*}
\end{example}

\begin{definition}   \label{df:ElementaryReductionMatrices}
%<*df:ElementaryReductionMatrices>
The \definend{elementary reduction matrices}%
\index{matrix!elementary reduction}\index{elementary!matrix}
result from applying a one Gaussian operation to an identity matrix.
\begin{enumerate}
  \item \( I\grstep{k\rho_i}M_i(k) \) for \( k\neq 0 \)
  \item \( I\grstep{\rho_i\leftrightarrow\rho_j}P_{i,j} \) for 
          \( i\neq j \)
  \item \( I\grstep{k\rho_i+\rho_j}C_{i,j}(k) \) for \( i\neq j \)
\end{enumerate}
%</df:ElementaryReductionMatrices>
\end{definition}

\begin{lemma}   \label{GrByMatMult}
%<*lm:GrByMatMult>
Matrix multiplication can do Gaussian reduction.
\begin{enumerate}
  \item If \( H\grstep{k\rho_i}G \) then \( M_i(k)H=G \).
  \item If \( H\grstep{\rho_i\leftrightarrow\rho_j}G \)
         then \( P_{i,j}H=G \).
  \item If \( H\grstep{k\rho_i+\rho_j}G \) then \( C_{i,j}(k)H=G \).
\end{enumerate}
%</lm:GrByMatMult>
\end{lemma}

\begin{proof}
Clear.
\end{proof}

\begin{example}
This is the first system, from the first chapter, on which we
performed Gauss's Method.
\begin{equation*}
  \begin{linsys}{3}
             &   &      &   &3x_3  &=  &9  \\
        x_1  &+  &5x_2  &-  &2x_3  &=  &2  \\
   (1/3)x_1  &+  &2x_2  &   &      &=  &3 
  \end{linsys}
\end{equation*}
We can reduce it with matrix multiplication.
Swap the first and third rows,
\begin{equation*}
    \begin{mat}[r]
       0  &0  &1  \\
       0  &1  &0  \\
       1  &0  &0
    \end{mat}
    \begin{amat}[r]{3}
       0    &0    &3   &9   \\
       1    &5    &-2  &2   \\
      1/3   &2    &0   &3
    \end{amat}
  =
    \begin{amat}[r]{3}
      1/3   &2    &0   &3   \\
       1    &5    &-2  &2   \\
       0    &0    &3   &9
    \end{amat}
\end{equation*}
triple the first row,
\begin{equation*}
    \begin{mat}[r]
       3  &0  &0  \\
       0  &1  &0  \\
       0  &0  &1
    \end{mat}
    \begin{amat}[r]{3}
      1/3   &2    &0   &3   \\
       1    &5    &-2  &2   \\
       0    &0    &3   &9
    \end{amat}
  =
    \begin{amat}[r]{3}
       1    &6    &0   &9   \\
       1    &5    &-2  &2   \\
       0    &0    &3   &9
    \end{amat}
\end{equation*}
and then add \( -1 \) times the first row to the second.
\begin{equation*}
    \begin{mat}[r]
       1  &0  &0  \\
      -1  &1  &0  \\
       0  &0  &1
    \end{mat}
    \begin{amat}[r]{3}
       1    &6    &0   &9   \\
       1    &5    &-2  &2   \\
       0    &0    &3   &9
    \end{amat}
  =
    \begin{amat}[r]{3}
       1    &6    &0   &9   \\
       0    &-1   &-2  &-7  \\
       0    &0    &3   &9
    \end{amat}
\end{equation*}
Now back substitution will give the solution.
\end{example}

\begin{example}
Gauss-Jordan reduction works the same way.
For the matrix ending the prior example, first adjust the leading entries
\begin{equation*}
    \begin{mat}[r]
       1  &0  &0  \\
       0  &-1 &0  \\
       0  &0  &1/3
    \end{mat}
    \begin{amat}[r]{3}
       1    &6    &0   &9   \\
       0    &-1   &-2  &-7  \\
       0    &0    &3   &9
    \end{amat}
  =
    \begin{amat}[r]{3}
       1    &6    &0   &9   \\
       0    &1    &2   &7   \\
       0    &0    &1   &3
    \end{amat}
\end{equation*}
and to finish, clear the third column and then the second column.
\begin{equation*}
    \begin{mat}[r]
       1  &-6 &0  \\
       0  &1  &0  \\
       0  &0  &1
    \end{mat}
    \begin{mat}[r]
       1  &0  &0  \\
       0  &1  &-2 \\
       0  &0  &1
    \end{mat}
    \begin{amat}[r]{3}
       1    &6    &0   &9   \\
       0    &1    &2   &7   \\
       0    &0    &1   &3
    \end{amat}
  =
    \begin{amat}[r]{3}
       1    &0    &0   &3   \\
       0    &1    &0   &1   \\
       0    &0    &1   &3
    \end{amat}
\end{equation*}
\end{example}

% We have observed the following result
% that we shall use in the next subsection.

\begin{corollary} \label{cor:ReducViaMatrices}
%<*co:ReducViaMatrices>
For any matrix \( H \) there are elementary reduction matrices
\( R_1 \), \ldots, \( R_r \) such that
\( R_r\cdot R_{r-1}\cdots R_1\cdot H \)
is in reduced echelon form.
%</co:ReducViaMatrices>
\end{corollary}

Until now we have taken the point of view that our primary objects of study
are vector spaces and the maps between them, and
have adopted matrices only for computational convenience.
This subsection show that this isn't the whole story.
Understanding matrices operations by how the entries combine can
be useful also. 
In the rest of this book we shall continue to focus on maps as the primary
objects but we will be pragmatic\Dash if the matrix point of view gives some
clearer idea then we will go with it.


\begin{exercises}
  \recommended \item
    Predict the result of each multiplication by an elementary
    reduction matrix, and then check by multiplying it out.
    \begin{exparts*}
      \partsitem \( \begin{mat}[r]
                 3  &0  \\
                 0  &1
               \end{mat}
               \begin{mat}[r]
                 1  &2  \\
                 3  &4
               \end{mat} \)
      \partsitem \( \begin{mat}[r]
                 1  &0  \\
                 0  &2
               \end{mat}
               \begin{mat}[r]
                 1  &2  \\
                 3  &4
               \end{mat} \)
      \partsitem \( \begin{mat}[r]
                 1  &0  \\
                -2  &1
               \end{mat}
               \begin{mat}[r]
                 1  &2  \\
                 3  &4
               \end{mat} \)
      \partsitem \( \begin{mat}[r]
                 1  &2  \\
                 3  &4
               \end{mat}
               \begin{mat}[r]
                 1  &-1 \\
                 0  &1
               \end{mat} \)
      \partsitem \( \begin{mat}[r]
                 1  &2  \\
                 3  &4
               \end{mat}
               \begin{mat}[r]
                 0  &1  \\
                 1  &0
               \end{mat} \)
    \end{exparts*}
    \begin{answer}
      \begin{exparts}
        \partsitem The second matrix has its first row multiplied 
          by \( 3 \).
          \begin{equation*}
            \begin{mat}[r]
              3  &6  \\
              3  &4
            \end{mat}
          \end{equation*}
        \partsitem The second matrix has its second row multiplied by \( 2 \).
          \begin{equation*}
            \begin{mat}[r]
              1  &2  \\
              6  &8
            \end{mat}
          \end{equation*}
        \partsitem The second matrix undergoes the combination operation
          of replacing the second row with \( -2 \) times the first row added
          to the second.
          \begin{equation*}
            \begin{mat}[r]
              1  &2  \\
              1  &0
            \end{mat}
          \end{equation*}
        \partsitem The first matrix undergoes the column operation of:~replace
          the
          second column by $-1$ times the first column plus the
          second.
          \begin{equation*}
            \begin{mat}[r]
              1  &1  \\
              3  &1
            \end{mat}
          \end{equation*}
        \partsitem The first matrix has its columns swapped.
          \begin{equation*}
            \begin{mat}[r]
              2  &1  \\
              4  &3
            \end{mat}
          \end{equation*}
      \end{exparts}   
    \end{answer}
  \item
    Predict the result of each multiplication by a
    diagonal matrix, and then check by multiplying it out.
    \begin{exparts*}
      \partsitem \( \begin{mat}[r]
                 3  &0  \\
                 0  &0
               \end{mat}
               \begin{mat}[r]
                 1  &2  \\
                 3  &4
               \end{mat} \)
      \partsitem \( \begin{mat}[r]
                 4  &0  \\
                 0  &2
               \end{mat}
               \begin{mat}[r]
                 1  &2  \\
                 3  &4
               \end{mat} \)
    \end{exparts*}
    \begin{answer}
      \begin{exparts}
        \partsitem The second matrix has its first row multiplied 
          by \( 3 \) and
          its second row multiplied by \( 0 \).
          \begin{equation*}
            \begin{mat}[r]
              3  &6  \\
              0  &0
            \end{mat}
          \end{equation*}
        \partsitem The second matrix has its first row multiplied 
          by \( 4 \) and
          its second row multiplied by \( 2 \).
          \begin{equation*}
            \begin{mat}[r]
              4  &8  \\
              6  &8
            \end{mat}
          \end{equation*}
      \end{exparts}   
    \end{answer}
  \recommended \item
    This table gives the number of hours of each 
    type done by each worker, and the associated pay rates.
    Use matrices to compute the wages due.
    \begin{center}
      \begin{tabular}[t]{l|cc}
        \multicolumn{1}{c}{\ }  
         &\multicolumn{1}{c}{\textit{regular}} 
         &\multicolumn{1}{c}{\textit{overtime}}  \\
        \cline{2-3}
        Alan      &40        &12        \\
        Betty     &35        &6         \\  
        Catherine &40        &18         \\  
        Donald    &28        &0         %\\  \cline{2-3}
      \end{tabular}
      \qquad
      \begin{tabular}[t]{l|c}
        \multicolumn{1}{c}{\ }   &\multicolumn{1}{c}{\textit{wage}}   \\
        \cline{2-2}
        regular   &$\$ 25.00$  \\
        overtime  &$\$ 45.00$  %\\  \cline{2-2}
      \end{tabular}
    \end{center}
    \textit{Remark.}
    This illustrates that in practice
    we often want to compute linear combinations of rows and
    columns in a context where we really aren't interested in any
    associated linear maps.
    \begin{answer}
      The pay due each person appears in the matrix product of the two
      arrays. 
    \end{answer}
  \recommended \item 
    The need to take linear combinations of rows and columns in tables of 
    numbers arises often in practice.
    For instance, this is a map of part of Vermont and New York.
    %\typeout{Ignore the overfull box here; it is bogus.}
    \begin{center}
      \parbox{2in}{In part because of Lake Champlain, there are
                      no roads directly connecting some pairs of towns.
                      For instance, there is no way to go from
                      Winooski to Grand Isle without going through
                      Colchester.
                      (To simplify the graph many other roads and towns
                      have been omitted.
                      From top to bottom of this map is
                      about forty miles.)}
      \quad
      \parbox{2in}{\includegraphics{ch3.99}}
    \end{center}
    \begin{exparts}
      \partsitem The \definend{adjacency matrix}\index{adjacency matrix}%
        \index{matrix!adjacency} 
        of a map is the square matrix
        whose \( i,j \) entry is the number of roads from city \( i \)
        to city \( j \).
        Produce the incidence matrix of this map (take the cities in
        alphabetical order).
      \partsitem A matrix is 
        \definend{symmetric}\index{matrix!symmetric} 
        if it equals its transpose.
        Show that an adjacency matrix is symmetric.
        (These are all two-way streets.
        Vermont doesn't have many one-way streets.)
      \partsitem What is the significance of the square of the 
        incidence matrix?
        The cube?
    \end{exparts}
    \begin{answer}
      \begin{exparts}
        \partsitem The adjacency matrix is this
          (e.g, the first row shows that there is only one connection
          including Burlington, the road to Winooski).
          \begin{equation*}
            \begin{mat}[r]
              0  &0  &0  &0  &1  \\
              0  &0  &1  &1  &1  \\
              0  &1  &0  &1  &0  \\
              0  &1  &1  &0  &0  \\
              1  &1  &0  &0  &0
            \end{mat}
          \end{equation*}
        \partsitem Because these are two-way roads, 
          any road connecting city~\( i \)
          to city~\( j \) gives a connection between city~\( j \) and 
          city~\( i \).
        \partsitem The square of the adjacency matrix tells how cities
          are connected
          by trips involving two roads.
      \end{exparts}  
   \end{answer}
  \item 
    Find the product of this matrix with its transpose.
    \begin{equation*}
      \begin{mat}
        \cos\theta  &-\sin\theta  \\
        \sin\theta  &\cos\theta
      \end{mat}
    \end{equation*}
    \begin{answer}
      The product is the identity matrix (recall that
      $\cos^2\theta+\sin^2\theta =1$).
      An explanation is that the given matrix represents, with respect to
      the standard bases, a rotation in \( \Re^2 \) of \( \theta \)
      radians while the transpose represents a rotation of \( -\theta \)
      radians.
      The two cancel.  
     \end{answer}
  \recommended \item 
    Prove that the diagonal matrices form a subspace of
    \( \matspace_{\nbyn{n}} \).
    What is its dimension?
    \begin{answer}
      The set of diagonal matrices is nonempty as the zero matrix is
      diagonal.
      Clearly it is closed under scalar multiples and sums.
      Therefore it is a subspace.
      The dimension is \( n \); here is a basis.
      \begin{equation*}
        \set{\begin{mat}
               1  &0  &\ldots    \\
               0  &0             \\
                  &   &\ddots    \\
               0  &0  &      &0
             \end{mat},\ldots,
             \begin{mat}
               0  &0  &\ldots    \\
               0  &0             \\
                  &   &\ddots    \\
               0  &0  &      &1
             \end{mat}  }
      \end{equation*} 
    \end{answer}
  \item 
    Does the identity matrix represent the identity map if the bases are
    unequal?
    \begin{answer}
      No.
      In \( \polyspace_1 \), with respect to the unequal bases
      \( B=\sequence{1,x} \) and \( D=\sequence{1+x,1-x} \),
      the identity transformation is represented by this matrix.
      \begin{equation*}
         \rep{\text{id}}{B,D}=
         \begin{mat}[r]
           1/2  &1/2  \\
           1/2  &-1/2
         \end{mat}_{B,D}
      \end{equation*}   
    \end{answer}
  \item 
    Show that every multiple of the identity commutes with every
    square matrix.
    Are there other matrices that commute with all square matrices?
    \begin{answer}
      For any scalar \( r \) and square matrix \( H \) we have
      \( (rI)H=r(IH)=rH=r(HI)=(Hr)I=H(rI) \).

      There are no other such matrices; here is an argument for $\nbyn{2}$ 
      matrices that is easily extended to $\nbyn{n}$.
      If a matrix commutes with all others then it commutes with this
      unit matrix.
      \begin{equation*}
        \begin{mat}
          0  &a  \\
          0  &c 
        \end{mat}
        =\begin{mat}
          a  &b  \\
          c  &d   
        \end{mat}
        \begin{mat}[r]
          0  &1  \\
          0  &0
        \end{mat}
        =\begin{mat}[r]
          0  &1  \\
          0  &0
        \end{mat}
        \begin{mat}
          a  &b  \\
          c  &d   
        \end{mat}
        =\begin{mat}
          c  &d  \\
          0  &0 
        \end{mat}
      \end{equation*}
      From this we first conclude that the upper left entry~$a$ 
      must equal its lower right entry~$d$.
      We also conclude that the lower left entry~$c$ is zero.
      The argument for the upper right entry~$b$ is similar.
    \end{answer}
  \item 
    Prove or disprove: nonsingular matrices commute.
    \begin{answer}
      It is false; these two don't commute.
      \begin{equation*}
         \begin{mat}[r]
           1  &2  \\
           3  &4
         \end{mat}
         \qquad
         \begin{mat}[r]
           5  &6  \\
           7  &8
         \end{mat}
      \end{equation*}  
    \end{answer}
  \recommended \item \label{exer:PermTimesTransEqId}
    Show that the product of a permutation matrix and its transpose
    is an identity matrix.
    \begin{answer}
      A permutation matrix has a single one in each row and column, and
      all its other entries are zeroes.
      Fix such a matrix.
      Suppose that the \( i \)-th row has its one in its \( j \)-th column.
      Then no other row has its one in the \( j \)-th column; every other
      row has a zero in the \( j \)-th column.
      Thus the dot product of the \( i \)-th row and any other row is zero.

      The \( i \)-th row of the product is made up of the dot products of the
      \( i \)-th row of the matrix and the columns of the transpose.
      By the last paragraph, all such dot products are zero except for the
      \( i \)-th one, which is one.  
    \end{answer}
  \item 
    Show that if the first and second rows of \( G \) are equal then so
    are the first and second rows of \( GH \).
    Generalize.
    \begin{answer}
      The generalization is to go from the first and second rows to the
      $i_1$-th and $i_2$-th rows.
      Row~$i$ of \( GH \) is made up of the dot products of
      row~$i$ of \( G \) and the columns of \( H \).
      Thus if rows \( i_1 \) and \( i_2 \) of \( G \) are equal then so are
      rows \( i_1 \) and \( i_2 \) of \( GH \).  
    \end{answer}
  \item 
    Describe the product of two diagonal matrices.
    \begin{answer}
      If the product of two diagonal matrices is defined\Dash if 
      both are $\nbyn{n}$\Dash then 
      the product of the diagonals is the diagonal
      of the products:~where \( G,H \) are equal-sized diagonal matrices,
      \( GH \) is all zeros except each that \( i,i \) entry is
      \( g_{i,i}h_{i,i} \).
    \end{answer}
  \item 
    Write
    \begin{equation*}
      \begin{mat}[r]
        1   &0  \\
        -3  &3
      \end{mat}
    \end{equation*}
    as the product of two elementary reduction matrices.
    \begin{answer}
      One way to produce this matrix from the identity is to use 
      the column operations
      of first multiplying the second column by three, and then adding the 
      negative of the resulting second column to the first.
      \begin{equation*}
        \begin{mat}[r]
          1  &0  \\
          0  &1
        \end{mat}
        \;\grstep{}\;
        \begin{mat}[r]
          1  &0  \\
          0  &3
        \end{mat}
        \;\grstep{}\;
        \begin{mat}[r]
          1  &0  \\
          -3  &3
        \end{mat}
      \end{equation*}
      In contrast with row operations, column operations are written from
      left to right, so this matrix product expresses
      doing the above two operations.
      \begin{equation*}
        \begin{mat}[r]
          1  &0  \\
          0  &3
        \end{mat}
        \begin{mat}[r]
          1  &0  \\
         -1  &1
        \end{mat}
      \end{equation*}
      \textit{Remark.}
      Alternatively, we could get the required matrix with row operations.
      Starting with the identity, first adding the negative of the first 
      row to the  second, and then multiplying the second row by three
      will work.
      Because we write successive row operations as matrix products from
      right to left, doing these two row operations is expressed with:~the 
      same matrix product.    
    \end{answer}
  \recommended \item 
    Show that if \( G \) has a row of zeros then \( GH \)
    (if defined) has a row of zeros.
    Does that work for columns?
    \begin{answer}
      The \( i \)-th row of \( GH \) is made up of the dot products of
      the \( i \)-th row of \( G \) with the columns of \( H \).
      The dot product of a zero row with a column is zero.

      It works for columns if stated correctly:~if \( H \) has a column of
      zeros then \( GH \) (if defined) has a column of zeros.
      The proof is easy.  
    \end{answer}
  \item 
    Show that the set of unit matrices forms a basis for 
    \( \matspace_{\nbym{n}{m}} \).
    \begin{answer}
      Perhaps the easiest way is to show that each \( \nbym{n}{m} \) matrix
      is a linear combination of unit matrices in one and only one way:
      \begin{equation*}
        c_1\begin{mat}
             1  &0  &\ldots  \\
             0  &0           \\
             \vdots
           \end{mat}
        +\dots+
        c_{n,m}\begin{mat}
             0  &0      &\ldots  \\
             \vdots              \\
             0  &\ldots &       &1
           \end{mat}
        =\begin{mat}
           a_{1,1} &a_{1,2} &\ldots          \\
           \vdots                            \\
           a_{n,1} &\ldots  &      &a_{n,m} 
         \end{mat}
      \end{equation*}
      has the unique solution \( c_1=a_{1,1} \), \( c_2=a_{1,2} \),
      etc.  
    \end{answer}
  \item  
    Find the formula for the $n$-th power of this matrix.
    \begin{equation*}
      \begin{mat}[r]
        1  &1  \\
        1  &0
      \end{mat}
    \end{equation*}
    \begin{answer}
      Call that matrix \( F \).
      We have
      \begin{equation*}
        F^2=\begin{mat}[r]
          2  &1  \\
          1  &1
        \end{mat}
        \quad
        F^3=\begin{mat}[r]
          3  &2  \\
          2  &1
        \end{mat}
        \quad
        F^4=\begin{mat}[r]
          5  &3  \\
          3  &2
        \end{mat}
      \end{equation*}
      In general, 
      \begin{equation*}
        F^n=\begin{mat}
          f_{n+1} &f_n  \\
          f_n     &f_{n-1}
        \end{mat}
      \end{equation*}
      where \( f_i \) is the \( i \)-th Fibonacci number
      \( f_i=f_{i-1}+f_{i-2} \) and \( f_0=0 \), \( f_1=1 \), 
      which we verify by induction, based on this equation.
      \begin{equation*}
        \begin{mat}
          f_{i-1}  &f_{i-2} \\
          f_{i-2}  &f_{i-3}
        \end{mat}
        \begin{mat}[r]
          1  &1  \\
          1  &0
        \end{mat}
        =\begin{mat}
           f_i     &f_{i-1}  \\
           f_{i-1} &f_{i-2}
        \end{mat}
      \end{equation*}
    \end{answer}
  \recommended \item
   The \definend{trace}\index{trace}\index{matrix!trace} 
   of a square matrix is the sum of the
   entries on its diagonal (its significance appears in
   Chapter Five).
   Show that \( \trace (GH)=\trace (HG)  \).
   \begin{answer}
     \textit{Chapter Five gives a less computational reason\Dash the
     trace of a matrix is the second coefficient in its characteristic
     polynomial\Dash but for now we can use indices.}
     We have
     \begin{align*}
       \trace (GH)
       &=(g_{1,1}h_{1,1}+g_{1,2}h_{2,1}+\dots+g_{1,n}h_{n,1})  \\
       &\text{}\quad+(g_{2,1}h_{1,2}+g_{2,2}h_{2,2}+\dots+g_{2,n}h_{n,2}) \\
       &\text{}\quad
        +\cdots+(g_{n,1}h_{1,n}+g_{n,2}h_{2,n}+\dots+g_{n,n}h_{n,n})
     \end{align*}
     while
     \begin{align*}
       \trace (HG)
       &=(h_{1,1}g_{1,1}+h_{1,2}g_{2,1}+\dots+h_{1,n}g_{n,1})  \\
       &\text{}\quad+(h_{2,1}g_{1,2}+h_{2,2}g_{2,2}+\dots+h_{2,n}g_{n,2}) \\
       &\text{}\quad
         +\cdots+(h_{n,1}g_{1,n}+h_{n,2}g_{2,n}+\dots+h_{n,n}g_{n,n})
     \end{align*}
     and the two are equal.  
    \end{answer}
  \recommended \item
    A square matrix is 
    \definend{upper triangular}\index{triangular matrix}%
    \index{matrix!triangular} 
    if its only nonzero
    entries lie above, or on, the diagonal.
    Show that the product of two upper triangular matrices is upper triangular.
    Does this hold for lower triangular also?
    \begin{answer}
      A matrix is upper triangular if and only if its $i,j$ entry is zero 
      whenever \( i>j \).
      Thus, if \( G,H \) are upper triangular then \( h_{i,j} \) and
      \( g_{i,j} \) are zero when \( i>j \).
      An entry in the product
      \( p_{i,j}=g_{i,1}h_{1,j}+\dots+g_{i,n}h_{n,j} \)
      is zero unless at least some of the terms are nonzero, that is, unless
      for at least some of the summands
      \( g_{i,r}h_{r,j} \) both \( i\leq r \) and \( r\leq j \).
      Of course, if \( i>j \) this cannot happen and so the product of two
      upper triangular matrices is upper triangular.
      (A similar argument works for lower triangular matrices.) 
   \end{answer}
 \item 
   A square matrix is a \definend{Markov matrix}\index{matrix!Markov} 
   if each entry is between zero
   and one and the sum along each row is one.
   Prove that a product of Markov matrices is Markov.
   \begin{answer}
     The sum along the \( i \)-th row of the product is this.
     \begin{align*}
       p_{i,1}+\cdots+p_{i,n}
       &=(h_{i,1}g_{1,1}+h_{i,2}g_{2,1}+\dots+h_{i,n}g_{n,1})  \\
       &\text{}\quad+(h_{i,1}g_{1,2}+h_{i,2}g_{2,2}+\dots+h_{i,n}g_{n,2}) \\
       &\text{}\quad
        +\dots+(h_{i,1}g_{1,n}+h_{i,2}g_{2,n}+\dots+h_{i,n}g_{n,n})  \\
       &=h_{i,1}(g_{1,1}+g_{1,2}+\dots+g_{1,n})  \\
       &\text{}\quad+h_{i,2}(g_{2,1}+g_{2,2}+\dots+g_{2,n}) \\
       &\text{}\quad
        +\dots+h_{i,n}(g_{n,1}+g_{n,2}+\dots+g_{n,n})  \\
       &=h_{i,1}\cdot 1+\dots+h_{i,n}\cdot 1  \\
       &=1
     \end{align*}  
    \end{answer}
  \recommended \item
    Give an example of two matrices of the same rank and size
    with squares of differing rank.
    \begin{answer}
      Matrices representing (say, with respect to
      \( \stdbasis_2,\stdbasis_2\subset\Re^2 \)) the maps that send
      \begin{equation*}
        \vec{\beta}_1 \mapsunder{h} \vec{\beta}_1
        \quad
        \vec{\beta}_2 \mapsunder{h} \zero
      \end{equation*}
      and
      \begin{equation*}
        \vec{\beta}_1 \mapsunder{g} \vec{\beta}_2
        \quad
        \vec{\beta}_2 \mapsunder{g} \zero
      \end{equation*}
      will do.  
    \end{answer}
  \item 
    Combine the two generalizations of the identity matrix,
    the one allowing entries to be other than ones, and the one allowing the
    single one in each row and column to be off the diagonal.
    What is the action of this type of matrix?
    \begin{answer}
      The combination is to have all entries of the matrix be zero
      except for one (possibly) nonzero entry in each row and column.
      We can write such a matrix as the product of a permutation matrix and
      a diagonal matrix, e.g.,
      \begin{equation*}
        \begin{mat}[r]
          0  &4  &0  \\
          2  &0  &0  \\
          0  &0  &-5
        \end{mat}
        =\begin{mat}[r]
          0  &1  &0  \\
          1  &0  &0  \\
          0  &0  &1
        \end{mat}
        \begin{mat}[r]
          4  &0  &0  \\
          0  &2  &0  \\
          0  &0  &-5
        \end{mat}
      \end{equation*}
      and its action is thus to rescale the rows and permute them.
    \end{answer}
  \item 
    On a computer multiplications have traditionally been 
    more costly than additions, so
    people have tried to in reduce the number of multiplications used to
    compute a matrix product.
    \begin{exparts}
      \partsitem How many real number multiplications do we need in the formula 
        we gave for the product of a
        \( \nbym{m}{r} \) matrix and a \( \nbym{r}{n} \) matrix?
      \partsitem 
        Matrix multiplication is associative, so all associations yield
        the same result.
        The cost in number of multiplications, however, varies.
        Find the association requiring the fewest real number multiplications
        to compute the matrix product of
        a \( \nbym{5}{10} \) matrix,
        a \( \nbym{10}{20} \) matrix,
        a \( \nbym{20}{5} \) matrix, and
        a \( \nbym{5}{1} \) matrix.
      \partsitem \textit{(Very hard.)}
        Find a way to multiply two \( \nbyn{2} \) matrices using only seven
        multiplications instead of the eight suggested by the naive approach.
    \end{exparts}
    \begin{answer}
      \begin{exparts}
        \partsitem Each entry \(p_{i,j}=g_{i,1}h_{1,j}+\dots+g_{1,r}h_{r,1}  \)
          takes \( r \) multiplications and there are \( m\cdot n \) entries.
          Thus there are \( m\cdot n\cdot r \) multiplications.
        \partsitem Let \( H_1 \) be \( \nbym{5}{10} \),
           let \( H_2 \) be \( \nbym{10}{20} \),
           let \( H_3 \) be \( \nbym{20}{5} \),
           let \( H_4 \) be \( \nbym{5}{1} \).
           Then, using the formula from the prior part, 
           \begin{center}
             \begin{tabular}{l|l}
               \multicolumn{1}{c}{\textit{this association}}  
                &\multicolumn{1}{c}{\textit{uses this many multiplications}}\\ 
               \hline
               \( ((H_1H_2)H_3)H_4 \)     &\( 1000+500+25=1525 \)  \\
               \( (H_1(H_2H_3))H_4 \)     &\( 1000+250+25=1275 \)  \\
               \( (H_1H_2)(H_3H_4) \)     &\( 1000+100+100=1200 \)  \\
               \( H_1(H_2(H_3H_4)) \)     &\( 100+200+50=350    \)  \\
               \( H_1((H_2H_3)H_4) \)     &\( 1000+50+50=1100    \)
             \end{tabular}
           \end{center}
           shows which is cheapest.
         \partsitem This is an improvement by S.~Winograd of a formula due to
           V.~Strassen:
           where \( w=aA-(a-c-d)(A-C+D) \),
           \begin{multline*}
               \begin{mat}
                 a  &b  \\
                 c  &d
               \end{mat}
               \begin{mat}
                 A  &B  \\
                 C  &D
                 \end{mat}    \\
               =
               \begin{mat}
                  aA+bB
                  &w+(c+d)(C-A)+(a+b-c-d)D \\
                  w+(a-c)(D-C)-d(A-B-C+D)
                  &w+(a-c)(D-C)+(c+d)(C-A)
               \end{mat}
           \end{multline*}
           takes seven multiplications and fifteen additions (save the
           intermediate results).
      \end{exparts}  
    \end{answer}
  \puzzle \item  
    \cite{Putnam90A5}
    If \( A \) and \( B \) are square matrices of the
    same size such that \( ABAB=0 \), does it follow that \( BABA=0 \)?
    \begin{answer}
      \answerasgiven 
      No, it does not.
      Let \( A \) and \( B \) represent, with respect to the standard bases,
      these transformations of \( \Re^3 \).
      \begin{equation*}
        \colvec{x \\ y \\ z}\mapsunder{a}\colvec{x \\ y \\ 0}
        \qquad
        \colvec{x \\ y \\ z}\mapsunder{a}\colvec{0 \\ x \\ y}
      \end{equation*}
      Observe that
      \begin{equation*}
        \colvec{x \\ y \\ z}\mapsunder{abab}\colvec[r]{0 \\ 0 \\ 0}
        \quad\text{but}\quad
        \colvec{x \\ y \\ z}\mapsunder{baba}\colvec{0 \\ 0 \\ x}.
      \end{equation*} 
    \end{answer}
  \item  
    \cite{Monthly66p1114}
    Demonstrate these four assertions to get an
    alternate proof that column rank equals row rank.
    \begin{exparts}
      \partsitem \( \vec{y}\cdot\vec{y}=\zero \) iff \( \vec{y}=\zero \).
      \partsitem \( A\vec{x}=\zero \) iff \( \trans{A}A\vec{x}=\zero \).
      \partsitem \( \dim(\rangespace{A})=\dim(\rangespace{\trans{A}A}) \).
      \partsitem \( \text{col rank}(A)=\text{col rank}(\trans{A})
        =\text{row rank}(A) \).
    \end{exparts}
    \begin{answer}
      \answerasgiven
      \begin{exparts}
        \partsitem Obvious.
        \partsitem If \( \trans{A}A\vec{x}=\zero \) then 
          \( \vec{y}\cdot\vec{y}=0 \)
          where \( \vec{y}=A\vec{x} \).
          Hence \( \vec{y}=\zero \) by (a).

          The converse is obvious.
        \partsitem By (b), \( A\vec{x}_1 \),\ldots,\( A\vec{x}_n \) 
          are linearly
          independent iff \( \trans{A}A\vec{x}_1 \),\ldots,
          \( \trans{A}A\vec{v}_n \) are linearly independent.
        \partsitem We have 
          \( \text{col rank}(A)=\text{col rank}(\trans{A}A)
          =\dim\set{\trans{A}(A\vec{x})\suchthat \text{all \(\vec{x}\)} }
          \leq \dim\set{\trans{A}\vec{y}\suchthat \text{all \(\vec{y}\)} }
          =\text{col rank}(\trans{A}) \).
          Thus also \( \text{col rank}(\trans{A})
                       \leq\text{col rank}(\trans{\trans{A}}) \)
          and so we have \( \text{col rank}(A)=\text{col rank}(\trans{A})
                    =\text{row rank}(A) \).
      \end{exparts} 
    \end{answer}
  \item  
    \cite{Monthly55p721}
    Prove
    (where \( A \) is an \( \nbyn{n} \) matrix and so defines
    a transformation of any \( n \)-dimensional space \( V \)
    with respect to \( B,B \) where \( B \) is a basis) that
    \( \dim(\rangespace{A}\intersection\nullspace{A})
           =\dim(\rangespace{A})-\dim(\rangespace{A^2}) \).
    Conclude
    \begin{exparts}
      \partsitem \( \nullspace{A}\subset\rangespace{A} \)
            iff
            \( \dim(\nullspace{A})=\dim(\rangespace{A})
                                   -\dim(\rangespace{A^2}) \);
      \partsitem \( \rangespace{A}\subseteq\nullspace{A} \)
            iff
            \( A^2=0 \);
      \partsitem \( \rangespace{A}=\nullspace{A} \)
            iff
            \( A^2=0 \)
            and \( \dim(\nullspace{A})=\dim(\rangespace{A}) \) ;
      \partsitem \( \dim(\rangespace{A}\intersection\nullspace{A})=0 \)
            iff
            \( \dim(\rangespace{A})=\dim(\rangespace{A^2}) \) ;
      \partsitem \textit{(Requires the Direct Sum subsection, 
            which is optional.)}
            \( V=\rangespace{A}\directsum\nullspace{A} \)
            iff
            \( \dim(\rangespace{A})=\dim(\rangespace{A^2}) \).
    \end{exparts}
    \begin{answer}
      \answerasgiven
      Let \( \sequence{\vec{z}_1,\dots,\vec{z}_k} \) be a basis for
      \( \rangespace{A}\intersection\nullspace{A} \)
      (\( k \) might be \( 0 \)).
      Let \( \vec{x}_1,\dots,\vec{x}_k\in V \) be such that
      \( A\vec{x}_i=\vec{z}_i \).
      Note \( \set{A\vec{x}_1,\dots,A\vec{x}_k} \) is linearly independent,
      and extend to a basis for \( \rangespace{A} \):
      \( A\vec{x}_1,\ldots,A\vec{x}_k,A\vec{x}_{k+1},\dots,A\vec{x}_{r_1} \)
      where \( r_1=\dim(\rangespace{A}) \).

      Now take \( \vec{x}\in V \).
      Write
      \begin{equation*}
        A\vec{x}=a_1(A\vec{x}_1)+\dots+a_{r_1}(A\vec{x}_{r_1})
      \end{equation*}
      and so
      \begin{equation*}
        A^2\vec{x}=a_1(A^2\vec{x}_1)+\dots+a_{r_1}(A^2\vec{x}_{r_1}).
      \end{equation*}
      But \( A\vec{x}_1,\dots,A\vec{x}_k\in\nullspace{A} \), so
      \( A^2\vec{x}_1=\zero,\dots,A^2\vec{x}_k=\zero \) and we now know
      \begin{equation*}
        A^2\vec{x}_{k+1},\dots,A^2\vec{x}_{r_1}
      \end{equation*}
      spans \( \rangespace{A^2} \).

      To see \( \set{A^2\vec{x}_{k+1},\dots,A^2\vec{x}_{r_1}} \)
      is linearly independent, write
      \begin{align*}
        b_{k+1}A^2\vec{x}_{k+1}+\dots+b_{r_1}A^2\vec{x}_{r_1}
        &=\zero                                                 \\
        A[b_{k+1}A\vec{x}_{k+1}+\dots+b_{r_1}A\vec{x}_{r_1}]
        &=\zero
      \end{align*}
      and, since
      \( b_{k+1}A\vec{x}_{k+1}+\dots+b_{r_1}A\vec{x}_{r_1}\in\nullspace{A} \)
      we get a contradiction unless it is \( \zero \) (clearly it is in
      \( \rangespace{A} \), but \( A\vec{x}_1,\ldots,A\vec{x}_k \) is a basis
      for \( \rangespace{A}\intersection\nullspace{A} \)).

      Hence \( \dim(\rangespace{A^2})=r_1-k=\dim(\rangespace{A})-
                \dim(\rangespace{A}\intersection\nullspace{A}) \).  
    \end{answer}
\end{exercises}























\subsection{Inverses}
\index{matrix!inverse|(}
%<*FunctionInverseReview0>
We finish this section by considering how to represent the 
inverse of a linear map.
%</FunctionInverseReview0>

%<*FunctionInverseReview1>
We first recall some things about
inverses.\appendrefs{function inverses}\spacefactor=1000\  
% Some functions have no inverse, or have an inverse on the left side 
%or right side only.
% 
% \begin{example}  \label{ex:ProjLeftInvOfEmbed}
Where 
\( \map{\pi}{\Re^3}{\Re^2} \) is the projection map 
and \( \map{\iota}{\Re^2}{\Re^3} \) is the embedding  
\begin{equation*}
  \colvec{x \\ y \\ z}
    \mapsunder{\pi}
  \colvec{x \\ y}
  \qquad
  \colvec{x \\ y}
    \mapsunder{\iota}
  \colvec{x \\ y \\ 0}
\end{equation*}
then the composition $\composed{\pi}{\iota}$ is the identity map
$\composed{\pi}{\iota}=\identity$ on $\Re^2$.
\begin{equation*}
  %\composed{\pi}{\eta}:\qquad
  \colvec{x \\ y}
    \mapsunder{\iota}
  \colvec{x \\ y \\ 0}
    \mapsunder{\pi}
  \colvec{x \\ y}
\end{equation*}
We say that
$\iota$ is a \definend{right inverse}\index{function!right inverse} 
of $\pi$
or, what is the same thing, 
that $\pi$ is a \definend{left inverse}\index{function!left inverse} 
of $\iota$.
%</FunctionInverseReview1>
%<*FunctionInverseReview2>
However, composition in the other order $\composed{\iota}{\pi}$ 
doesn't give the identity map\Dash here is a vector that is not 
sent to itself under $\composed{\iota}{\pi}$. 
\begin{equation*}
  %\composed{\eta}{\pi}:\qquad
  \colvec[r]{0 \\ 0 \\ 1}
    \mapsunder{\pi}
  \colvec[r]{0 \\ 0}
    \mapsunder{\iota}
  \colvec[r]{0 \\ 0 \\ 0}
\end{equation*}
%</FunctionInverseReview2>
%<*FunctionInverseReview3>
In fact,
$\pi$ has no left inverse at all.
For, if $f$ were to be a left inverse of $\pi$
then we would have
\begin{equation*}
  %\composed{f}{\pi}:\qquad
  \colvec{x \\ y \\ z}
    \mapsunder{\pi}
  \colvec{x \\ y}
    \mapsunder{f}
  \colvec{x \\ y \\ z}
\end{equation*}
for all of the infinitely many $z$'s.
But no function can send a single argument to more than one value.
% \end{example}
(An example of a function with no inverse on either side
is the zero transformation on $\Re^2$.)
%</FunctionInverseReview3>

%<*FunctionInverseReview4>
Some functions have a 
\definend{two-sided inverse},\index{function!two-sided inverse} 
another function
that is the inverse of the first both from the left and from the right.
For instance, the map given by 
$\vec{v}\mapsto 2\cdot \vec{v}$ has the two-sided inverse 
$\vec{v}\mapsto (1/2)\cdot\vec{v}$.  
% In this subsection we will focus on two-sided inverses.
% 
The appendix shows that a function
has a two-sided inverse if and only if it is both one-to-one and onto.
The appendix also shows that if a function $f$ has a two-sided inverse then 
it is unique, and so we call it 
`the'~inverse\index{inverse function}\index{function!inverse}
and denote it $f^{-1}$.
%</FunctionInverseReview4>

%<*FunctionInverseReview5>
In addition, recall that we have shown in Theorem~II.\ref{th:OOHomoEquivalence}
that if a linear map has a two-sided inverse
then that inverse is also linear. 
%</FunctionInverseReview5>

%<*FunctionInverseReview6>
Thus, our goal in this subsection is, where a linear $h$ has an inverse,
to find the relationship between $\rep{h}{B,D}$ and $\rep{h^{-1}}{D,B}$.
%</FunctionInverseReview6>

\begin{definition} \label{df:MatrixInverse}
%<*df:MatrixInverse>
A matrix \( G \) is a \definend{left inverse matrix}\index{inverse!left}
of the matrix \( H \) if \( GH \) is the identity matrix.
It is a \definend{right inverse matrix}\index{inverse!right}
if \( HG \) is the identity.
A matrix $H$ with a two-sided inverse is an \definend{invertible matrix}.
That two-sided inverse is
\definend{the inverse matrix}\index{inverse} and
is denoted \( H^{-1} \).\index{matrix!inverse, definition}
%</df:MatrixInverse>
\end{definition}

Because of the correspondence between linear maps and matrices,
statements about map inverses translate into statements about matrix inverses.

\begin{lemma}     \label{le:LeftAndRightInvEqual}
%<*lm:LeftAndRightInvEqual>
If a matrix has both a left inverse and a right inverse then the two are equal.
%</lm:LeftAndRightInvEqual>
\end{lemma}

\begin{theorem}   \label{th:MatrixInvertibleIffNonsingular}
\index{inverse!exists}\index{nonsingular}
%<*th:MatrixInvertibleIffNonsingular>
A matrix is invertible if and only if it is nonsingular.
%</th:MatrixInvertibleIffNonsingular>
\end{theorem}

\begin{proof}
%<*pf:MatrixInvertibleIffNonsingular>
\textit{(For both results.)}
Given a matrix $H$, fix spaces of appropriate dimension for the domain
and codomain.
Fix bases for these spaces.
With respect to these bases, $H$ represents a map $h$.
The statements are true about the map and therefore they are true about the
matrix.
%</pf:MatrixInvertibleIffNonsingular>
\end{proof}

\begin{lemma} \label{lem:ProdInvIsInv}
%<*lm:ProdInvIsInv>
A product of invertible matrices is invertible:~if
\( G \) and \( H \) are invertible and if \( GH \) is defined then
\( GH \) is invertible and \( (GH)^{-1}=H^{-1}G^{-1} \).
%</lm:ProdInvIsInv>
\end{lemma}

\begin{proof}
% \textit{(This is just like the prior proof except that it requires two maps.)}
%<*pf:ProdInvIsInv0>
Because the two matrices are invertible they are square.
Because their product is defined they must be square of the same dimension,
$\nbyn{n}$.
So by fixing a basis for $\Re^n$\Dash we can use the standard 
basis\Dash we get maps  
$\map{g,h}{\Re^n}{\Re^n}$ that are associated with the matrices,
$G=\rep{g}{\stdbasis_n,\stdbasis_n}$ and
$H=\rep{h}{\stdbasis_n,\stdbasis_n}$.
%</pf:ProdInvIsInv0>

%<*pf:ProdInvIsInv1>
Consider $h^{-1}g^{-1}$. 
By the prior paragraph this composition is defined.
This map is a two-sided inverse of \( gh \) since
\(
  (h^{-1}g^{-1})(gh)
  =
  h^{-1}(\identity)h
  =h^{-1}h
  =\identity
\)
and
\(
  (gh)(h^{-1}g^{-1})
  =
  g(\identity)g^{-1}
  =gg^{-1}
  =\identity
\).
the matrices representing the maps reflect this equality.
%</pf:ProdInvIsInv1>
\end{proof}

This is the arrow diagram\index{arrow diagram} giving the relationship
between map inverses and matrix inverses. 
It is a special case
of the diagram for function composition and matrix multiplication.
\smallskip
\begin{center}
    \includegraphics{ch3.21}
\end{center}

Beyond its place in our general program of 
seeing how to represent map operations, 
another reason for our interest in inverses comes from solving
linear systems.
A linear system is equivalent to a matrix equation, as here.
\begin{equation*}
  \begin{linsys}{2}
    x_1  &+  &x_2  &=  &3  \\
   2x_1  &-  &x_2  &=  &2  
  \end{linsys}
  \quad\Longleftrightarrow\quad
  \begin{mat}[r]
       1  &1  \\
       2  &-1
    \end{mat}
  \colvec{x_1 \\ x_2}
  =
  \colvec[r]{3 \\ 2}
\tag*{($*$)}\end{equation*}
By fixing spaces and bases 
(for instance, $\Re^2,\Re^2$ with the standard bases),
we take the matrix $H$ to represent a map $h$.
The matrix equation then becomes this linear map equation.
\begin{equation*}
  h(\vec{x})=\vec{d}
  \tag*{($**$)}
\end{equation*}
Asking for a solution to ($*$) is the same as  
asking in ($**$)
for the domain vector $\vec{x}$ that $h$ maps to the result 
$\vec{d}\,$.
If we had a left inverse map~$g$ then  
we could apply it to both sides
$\composed{g}{h}(\vec{x})=g(\vec{d})$,
which simplifies to
$\vec{x}=g(\vec{d})$.
In terms of the matrices,
we multiply $\rep{g}{C,B}\cdot\rep{\vec{d}}{C}$
to get $\rep{\vec{x}}{B}$.

\begin{example}  \label{ex:InverseByLinSys}
We can find a left inverse for the matrix just given
\begin{equation*}
    \begin{mat}
       m  &n  \\
       p  &q
    \end{mat}
    \begin{mat}[r]
       1  &1  \\
       2  &-1
    \end{mat}
  =
    \begin{mat}[r]
       1  &0  \\
       0  &1
    \end{mat}
\end{equation*}
by using Gauss's Method to solve the resulting linear system.
\begin{equation*}
  \begin{linsys}{4}
     m  &+  &2n  &    &   &   &    &=  &1     \\
     m  &-  &n   &    &   &   &    &=  &0     \\
        &   &    &    &p  &+  &2q  &=  &0     \\
        &   &    &    &p  &-  &q   &=  &1     
     \end{linsys}
\end{equation*}
Answer: \( m=1/3 \), \( n=1/3 \), \( p=2/3 \), and \( q=-1/3 \).
This matrix is actually the two-sided inverse of $H$;
the check is easy.
With it we can solve the system~($*$) above.
\begin{equation*}
  \colvec{x \\ y}
  =\begin{mat}[r]
       1/3  &1/3  \\
       2/3  &-1/3
    \end{mat}
  \colvec[r]{3 \\ 2}          
  =\colvec[r]{5/3 \\ 4/3}
\end{equation*}
\end{example}

\begin{remark}
Why do this when we have
Gauss's Method?
% takes less arithmetic
% (this assertion can be made precise by counting the 
% number of arithmetic operations,
% as computer algorithm designers do)? 
Beyond the conceptual appeal of representing the map inverse operation,
solving linear systems this way has 
at least two advantages.

First, once we have done the work of finding an inverse then 
solving a system with the
same coefficients but different constants is fast:~if
we change the entries on the right of the system~($*$) 
then we get a related problem
\begin{equation*}
    \begin{mat}[r]
       1  &1  \\
       2  &-1
    \end{mat}
  \colvec{x \\ y}
  =
  \colvec[r]{5 \\ 1}
\end{equation*}
that our inverse method solves quickly.
\begin{equation*}
  \colvec{x \\ y}
  =
    \begin{mat}[r]
       1/3  &1/3  \\
       2/3  &-1/3
    \end{mat}
  \colvec[r]{5 \\ 1}
  =
  \colvec[r]{2 \\ 3}
\end{equation*}
% In applications we often must solve many systems with the same matrix of
% coefficients.

Another advantage of inverses is that we can 
explore a system's sensitivity to changes in the constants.
For example, tweaking the $3$ on the right of the system~($*$) to
\begin{equation*}
    \begin{mat}[r]
       1  &1  \\
       2  &-1
    \end{mat}
  \colvec{x_1 \\ x_2}
  =
  \colvec[r]{3.01 \\ 2}
\end{equation*}
and solving with the inverse
\begin{equation*}
    \begin{mat}[r]
       1/3  &1/3  \\
       2/3  &-1/3
    \end{mat}
  \colvec[r]{3.01 \\ 2}
  =
  \colvec{(1/3)(3.01)+(1/3)(2) \\ (2/3)(3.01)-(1/3)(2)}
\end{equation*}
shows that the first component of the solution
changes by \( 1/3 \) of the tweak, while
the second component moves by \( 2/3 \) of the tweak.
This is \definend{sensitivity analysis}\index{sensitivity analysis}.
For instance, we could use it to decide how accurately
we must specify the data in a linear model to ensure that the solution has
a desired accuracy.
\end{remark}

We finish by describing the computational procedure
that we shall use to find the inverse matrix.

\begin{lemma} \label{lem:ComputeInvMat}
%<*lm:ComputeInvMat>
A matrix $H$ is invertible if and only if it can be written as the product of
elementary reduction matrices.
We can compute the inverse by applying to the 
identity matrix the same row steps, in the same order, as we use to
Gauss-Jordan reduce $H$. 
%</lm:ComputeInvMat>
\end{lemma}

\begin{proof}
%<*pf:ComputeInvMat0>
The matrix $H$ is invertible if and only if it is nonsingular and thus 
Gauss-Jordan reduces to the identity.
By \nearbycorollary{cor:ReducViaMatrices} we can do this reduction 
with elementary matrices.
\begin{equation*} 
   R_r\cdot R_{r-1}\dots R_1\cdot H=I 
   \tag{*}
\end{equation*}
%</pf:ComputeInvMat0>

%<*pf:ComputeInvMat1>
For the first sentence of the result, note that
elementary matrices are invertible (because elementary row operations
are reversible) and that their inverses are also
elementary.
Apply $R_r^{-1}$ from the left to both sides of ($*$). 
Then apply
$R_{r-1}^{-1}$, etc.
The result gives $H$ as the product of
elementary matrices $H=R_1^{-1}\cdots R_r^{-1}\cdot I$
(the $I$ here covers the trivial $r=0$ case).
%</pf:ComputeInvMat1>

%<*pf:ComputeInvMat2>
For the second sentence,
rewrite ($*$) as $(R_r\cdot R_{r-1}\dots R_1)\cdot H=I$
to recognize that 
$H^{-1}=R_r\cdot R_{r-1}\dots R_1\cdot I$.
Restated, applying $R_1$ to the identity, 
followed by $R_2$, etc., yields the inverse of $H$.
%</pf:ComputeInvMat2>
\end{proof}

\begin{example}
To find the inverse of
\begin{equation*}
    \begin{mat}[r]
       1  &1  \\
       2  &-1
    \end{mat}
\end{equation*}
we do Gauss-Jordan reduction, meanwhile performing the same operations on
the identity.
For clerical convenience we write the matrix and the identity side-by-side,
and do the reduction steps together.
\begin{eqnarray*}
    \begin{pmat}{rr|rr}
       1  &1   &1  &0  \\
       2  &-1  &0  &1
    \end{pmat}
  &\grstep{-2\rho_1+\rho_2}
  &\begin{pmat}{rr|rr}
        1  &1   &1  &0  \\
        0  &-3  &-2 &1
     \end{pmat}                 \\
  &\grstep{-1/3\rho_2}
  &\begin{pmat}{rr|rr}
        1  &1   &1   &0     \\
        0  &1   &2/3 &-1/3
     \end{pmat}                   \\
  &\grstep{-\rho_2+\rho_1}
  &\begin{pmat}{rr|rr}
        1  &0   &1/3 &1/3  \\
        0  &1   &2/3 &-1/3
     \end{pmat}
\end{eqnarray*}
This calculation has found the inverse.
\begin{equation*}
    \begin{mat}[r]
       1  &1  \\
       2  &-1
    \end{mat}^{-1}
  =
    \begin{mat}[r]
       1/3 &1/3  \\
       2/3 &-1/3
    \end{mat}
\end{equation*}
\end{example}

\begin{example} \label{exam:ThreeByThreeMatInv}
This one happens to start with a row swap.
\begin{eqnarray*}
     \begin{pmat}{rrr|rrr}
        0  &3  &-1  &1  &0  &0  \\
        1  &0  &1   &0  &1  &0  \\
        1  &-1 &0   &0  &0  &1
     \end{pmat}
  &\grstep{\rho_1\leftrightarrow\rho_2}
  &   \begin{pmat}{rrr|rrr}
         1  &0  &1   &0  &1  &0  \\
         0  &3  &-1  &1  &0  &0  \\
         1  &-1 &0   &0  &0  &1
      \end{pmat}                             \\
  &\grstep{-\rho_1+\rho_3}
  &   \begin{pmat}{rrr|rrr}
         1  &0  &1   &0  &1  &0  \\
         0  &3  &-1  &1  &0  &0  \\
         0  &-1 &-1  &0  &-1 &1
      \end{pmat}                             \\
  &\vdots                                  \\
  &\grstep{}
  &   \begin{pmat}{rrr|rrr}
         1  &0  &0   &1/4  &1/4  &3/4  \\
         0  &1  &0   &1/4  &1/4  &-1/4 \\
         0  &0  &1   &-1/4 &3/4  &-3/4
      \end{pmat}
\end{eqnarray*}
\end{example}

\begin{example}
We can detect
a non-invertible matrix when the left half won't
reduce to the identity.
\begin{equation*}
    \begin{pmat}{rr|rr}
       1  &1   &1  &0  \\
       2  &2   &0  &1
    \end{pmat}
  \grstep{-2\rho_1+\rho_2}
   \begin{pmat}{rr|rr}
       1  &1   &1  &0  \\
       0  &0   &-2 &1
    \end{pmat}
\end{equation*}
\end{example}

With this procedure we can give a formula for 
the inverse of a general $\nbyn{2}$ matrix, 
which is worth memorizing.
But larger matrices have more complex formulas so 
we will wait for more explanation in the next chapter.

\begin{corollary} \label{cor:TwoByTwoInv}
%<*co:TwoByTwoInv>
The inverse for a \( \nbyn{2} \) matrix exists and equals
\begin{equation*}
  \begin{mat}
    a  &b  \\
    c  &d
  \end{mat}^{-1}
  =
  \frac{1}{ad-bc}
  \begin{mat}
    d  &-b \\
   -c  &a
  \end{mat}
\end{equation*}
if and only if \( ad-bc\neq 0 \).
%</co:TwoByTwoInv>
\end{corollary}
\begin{proof}
%<*pf:TwoByTwoInv>
This computation is 
\nearbyexercise{exer:TwoByTwoInv}. 
%</pf:TwoByTwoInv>
\end{proof}

We have seen here, as in the Mechanics of Matrix Multiplication subsection,
that we can exploit the correspondence between
linear maps and matrices.
So we can fruitfully study both maps and matrices, translating back and forth
to whichever helps the most.

Over this whole 
section we have developed an algebra system for matrices.
We can compare it with the familiar algebra system for the real numbers.
Here we are working not with numbers but with matrices.
We have matrix addition and subtraction operations, 
and they work in much the same
way as the real number operations, except that they only combine same-sized
matrices.
We have scalar multiplication, which is in some ways another extension
of real number multiplication.
We also have a matrix multiplication operation 
and a multiplicative inverse.
These operations are somewhat like the familiar real number ones
(associativity, and distributivity over addition, for example), but
there are differences (failure of commutativity). 
This matrix system provides an example that algebra
systems other than the 
elementary real number system can be interesting and useful.


\begin{exercises}
  \item 
    Supply the intermediate steps in 
    \nearbyexample{exam:ThreeByThreeMatInv}.
    \begin{answer}
      Here is one way to proceed.
      \begin{multline*}
       \grstep{\rho_1\leftrightarrow\rho_2}\;
       \begin{pmat}{rrr|rrr}
              1  &0  &1   &0  &1  &0  \\
              0  &3  &-1  &1  &0  &0  \\
              1  &-1 &0   &0  &0  &1
           \end{pmat}                          
       \;\grstep{-\rho_1+\rho_3}\;
       \begin{pmat}{rrr|rrr}
              1  &0  &1   &0  &1  &0  \\
              0  &3  &-1  &1  &0  &0  \\
              0  &-1 &-1  &0  &-1 &1
           \end{pmat}                        \\                          
       \begin{aligned}
         &\grstep{(1/3)\rho_2+\rho_3}\;
         \begin{pmat}{rrr|rrr}
                1  &0  &1     &0   &1  &0  \\
                0  &3  &-1    &1   &0  &0  \\
                0  &0  &-4/3  &1/3 &-1 &1
             \end{pmat}                          
         \;\grstep[-(3/4)\rho_3]{(1/3)\rho_2}\;
         \begin{pmat}{rrr|rrr}
                1  &0  &1     &0    &1   &0    \\
                0  &1  &-1/3  &1/3  &0   &0    \\
                0  &0  &1     &-1/4 &3/4 &-3/4
             \end{pmat}                                  \\
         &\grstep[-\rho_3+\rho_1]{(1/3)\rho_3+\rho_2}\;
         \begin{pmat}{rrr|rrr}
                1  &0  &0     &1/4  &1/4 &3/4  \\
                0  &1  &0     &1/4  &1/4 &-1/4 \\
                0  &0  &1     &-1/4 &3/4 &-3/4
             \end{pmat}                         
       \end{aligned}
    \end{multline*} 
   \end{answer}
 \recommended \item 
    Use \nearbycorollary{cor:TwoByTwoInv} to decide if each matrix 
    has an inverse.
    \begin{exparts*}
      \partsitem
        $\begin{mat}[r]
           2  &1  \\
          -1  &1          
        \end{mat}$
      \partsitem
        $\begin{mat}[r]
          0  &4  \\
          1  &-3
        \end{mat}$
      \partsitem
        $\begin{mat}[r]
          2  &-3  \\
         -4  &6
        \end{mat}$
    \end{exparts*}
    \begin{answer}
      \begin{exparts*}
        \partsitem Yes, it has an inverse: $ad-bc=2\cdot 1-1\cdot(-1)\neq 0$.
        \partsitem Yes.
        \partsitem No.
      \end{exparts*}
    \end{answer}
  \recommended \item  
     For each invertible matrix in the prior problem, use
     \nearbycorollary{cor:TwoByTwoInv} to find its inverse.
     \begin{answer}
       \begin{exparts}
         \partsitem 
           $\displaystyle \frac{1}{2\cdot 1-1\cdot (-1)}
            \cdot\begin{mat}[r]
             1  &-1  \\
             1  &2
           \end{mat}
          =\frac{\displaystyle 1}{\displaystyle 3}\cdot\begin{mat}[r]
             1  &-1  \\
             1  &2
           \end{mat}
          =\begin{mat}[r]
             1/3  &-1/3  \\
             1/3  &2/3
           \end{mat}$
         \partsitem 
           $\displaystyle \frac{1}{0\cdot (-3)-4\cdot 1}
           \cdot\begin{mat}[r]
             -3  &-4  \\
             -1  &0
           \end{mat}
           =\begin{mat}[r]
             3/4  &1  \\
             1/4  &0
           \end{mat}$
         \partsitem The prior question shows that no inverse exists.
       \end{exparts}
     \end{answer}
  \recommended \item
    Find the inverse, if it exists, by using the Gauss-Jordan Method.
    Check the answers for the $\nbyn{2}$ matrices 
    with \nearbycorollary{cor:TwoByTwoInv}.
    \begin{exparts*}
      \partsitem $\begin{mat}[r]
                   3  &1  \\
                   0  &2
                 \end{mat}$
      \partsitem \( \begin{mat}[r]
                 2   &1/2  \\
                 3   &1
               \end{mat} \)
      \partsitem \( \begin{mat}[r]
                 2   &-4   \\
                -1   &2
               \end{mat} \)
      \partsitem \( \begin{mat}[r]
                 1   &1  &3  \\
                 0   &2  &4  \\
                 -1  &1  &0
               \end{mat} \)
      \partsitem \( \begin{mat}[r]
                 0   &1  &5  \\
                 0   &-2 &4  \\
                 2   &3  &-2
               \end{mat} \)
      \partsitem \( \begin{mat}[r]
                 2   &2  &3  \\
                 1   &-2 &-3 \\
                 4   &-2 &-3
               \end{mat} \)
    \end{exparts*}
    \begin{answer}
      \begin{exparts}
        \partsitem The reduction is routine. 
          \begin{equation*}
            \begin{pmat}{rr|rr}
              3  &1  &1  &0 \\
              0  &2  &0  &1
            \end{pmat}
            \;\grstep[(1/2)\rho_2]{(1/3)\rho_1}\;
            \begin{pmat}{rr|rr}
              1  &1/3  &1/3  &0   \\
              0  &1    &0    &1/2
            \end{pmat}
            \;\grstep{-(1/3)\rho_2+\rho_1}\;
            \begin{pmat}{rr|rr}
              1  &0    &1/3  &-1/6 \\
              0  &1    &0    &1/2
            \end{pmat}
          \end{equation*}
          This answer agrees with the answer from the check.
          \begin{equation*}
            \begin{mat}[r]
              3  &1  \\
              0  &2
            \end{mat}^{-1}
            =\frac{1}{3\cdot 2-0\cdot 1}\cdot
            \begin{mat}[r]
              2  &-1  \\
              0  &3
            \end{mat}
            =\frac{1}{6}\cdot
            \begin{mat}[r]
              2  &-1  \\
              0  &3
            \end{mat}
          \end{equation*}
        \partsitem This reduction is easy.
          \begin{multline*}
            \begin{pmat}{rr|rr}
              2  &1/2  &1  &0  \\
              3  &1    &0  &1
            \end{pmat}
            \;\grstep{-(3/2)\rho_1+\rho_2}\;
            \begin{pmat}{rr|rr}
              2  &1/2  &1     &0  \\
              0  &1/4  &-3/2  &1
            \end{pmat}                       \\
            \;\grstep[4\rho_2]{(1/2)\rho_1}\;
            \begin{pmat}{rr|rr}
              1  &1/4   &1/2   &0  \\
              0  &1     &-6    &4
            \end{pmat}                         
            \;\grstep{-(1/4)\rho_2+\rho_1}\;
            \begin{pmat}{rr|rr}
              1  &0     &2     &-1 \\
              0  &1     &-6    &4
            \end{pmat}
          \end{multline*}
          The check agrees.
          \begin{equation*}
            \frac{1}{2\cdot 1-3\cdot (1/2)}\cdot
            \begin{mat}[r]
              1  &-1/2  \\
              -3 &2
            \end{mat}
            =2\cdot
            \begin{mat}[r]
              1  &-1/2  \\
              -3 &2
            \end{mat}
          \end{equation*}
        \partsitem Trying the Gauss-Jordan reduction
          \begin{equation*}
            \begin{pmat}{rr|rr}
              2  &-4  &1  &0  \\
             -1  &2   &0  &1 
            \end{pmat}
            \;\grstep{(1/2)\rho_1+\rho_2}\;
            \begin{pmat}{rr|rr}
              2  &-4  &1   &0  \\
              0  &0   &1/2 &1 
            \end{pmat}
          \end{equation*}
          shows that the left side won't reduce to the identity, so no inverse
          exists.
          The check $ad-bc=2\cdot 2-(-4)\cdot (-1)=0$ agrees.
        \partsitem This produces an inverse.
          \begin{multline*}
            \begin{pmat}{rrr|rrr}
              1  &1  &3  &1  &0  &0  \\ 
              0  &2  &4  &0  &1  &0  \\
             -1  &1  &0  &0  &0  &1 
            \end{pmat}  
            \;\grstep{\rho_1+\rho_3}\;
            \begin{pmat}{rrr|rrr}
              1  &1  &3  &1  &0  &0  \\ 
              0  &2  &4  &0  &1  &0  \\
              0  &2  &3  &1  &0  &1 
            \end{pmat}
            \;\grstep{-\rho_2+\rho_3}\;
            \begin{pmat}{rrr|rrr}
              1  &1  &3  &1  &0  &0  \\ 
              0  &2  &4  &0  &1  &0  \\
              0  &0  &-1 &1  &-1 &1 
            \end{pmat}                                              \\
           \begin{aligned}
            &\grstep[-\rho_3]{(1/2)\rho_2}\;
            \begin{pmat}{rrr|rrr}
              1  &1  &3  &1  &0   &0  \\ 
              0  &1  &2  &0  &1/2 &0  \\
              0  &0  &1  &-1 &1   &-1
            \end{pmat}
            \;\grstep[-3\rho_3+\rho_1]{-2\rho_3+\rho_2}\;
            \begin{pmat}{rrr|rrr}
              1  &1  &0  &4  &-3   &3  \\ 
              0  &1  &0  &2  &-3/2 &2  \\
              0  &0  &1  &-1 &1    &-1
            \end{pmat}                                         \\
            &\grstep{-\rho_2+\rho_1}\;
            \begin{pmat}{rrr|rrr}
              1  &0  &0  &2  &-3/2 &1  \\ 
              0  &1  &0  &2  &-3/2 &2  \\
              0  &0  &1  &-1 &1    &-1
            \end{pmat}
           \end{aligned} 
          \end{multline*}
       \partsitem This is one way to do the reduction.
          \begin{multline*}
            \begin{pmat}{rrr|rrr}
              0  &1  &5  &1  &0  &0  \\
              0  &-2 &4  &0  &1  &0  \\ 
              2  &3  &-2 &0  &0  &1
            \end{pmat}
            \;\grstep{\rho_3\leftrightarrow\rho_1}\;
            \begin{pmat}{rrr|rrr}
              2  &3  &-2 &0  &0  &1  \\
              0  &-2 &4  &0  &1  &0  \\ 
              0  &1  &5  &1  &0  &0  
            \end{pmat}                                            \\
            \begin{aligned}
              &\;\grstep{(1/2)\rho_2+\rho_3}\;
              \begin{pmat}{rrr|rrr}
                2  &3  &-2 &0  &0   &1  \\
                0  &-2 &4  &0  &1   &0  \\ 
                0  &0  &7  &1  &1/2 &0  
              \end{pmat}                    
              \grstep[-(1/2)\rho_2 \\ (1/7)\rho_3]{(1/2)\rho_1}\;
              \begin{pmat}{rrr|rrr}
                1  &3/2  &-1 &0    &0     &1/2  \\
                0  &1    &-2 &0    &-1/2  &0    \\ 
                0  &0    &1  &1/7  &1/14  &0  
              \end{pmat}                                   \\
              &\;\grstep[\rho_3+\rho_1]{2\rho_3+\rho_2}\;
              \begin{pmat}{rrr|rrr}
                1  &3/2  &0  &1/7  &1/14  &1/2  \\
                0  &1    &0  &2/7  &-5/14 &0    \\ 
                0  &0    &1  &1/7  &1/14  &0  
              \end{pmat}                                   
              \grstep{-(3/2)\rho_2+\rho_1}\;
              \begin{pmat}{rrr|rrr}
                1  &0    &0  &-2/7 &17/28 &1/2  \\
                0  &1    &0  &2/7  &-5/14 &0    \\ 
                0  &0    &1  &1/7  &1/14  &0  
              \end{pmat}
            \end{aligned}
          \end{multline*}
        \partsitem There is no inverse.
          \begin{align*}
            \begin{pmat}{rrr|rrr}
              2  &2  &3  &1  &0  &0  \\
              1  &-2 &-3 &0  &1  &0  \\
              4  &-2 &-3 &0  &0  &1
            \end{pmat}
            &\;\grstep[-2\rho_1+\rho_3]{-(1/2)\rho_1+\rho_2}\;
            \begin{pmat}{rrr|rrr}
              2  &2  &3    &1     &0  &0  \\
              0  &-3 &-9/2 &-1/2  &1  &0  \\
              0  &-6 &-9   &-2    &0  &1
            \end{pmat}                                          \\
            &\grstep{-2\rho_2+\rho_3}\;
            \begin{pmat}{rrr|rrr}
              2  &2  &3    &1     &0   &0  \\
              0  &-3 &-9/2 &-1/2  &1   &0  \\
              0  &0  &0    &-1    &-2  &1
            \end{pmat}
          \end{align*}
          As a check, 
          note that the third column of the starting matrix is $3/2$ times 
          the second, and so it is indeed singular and therefore has no
          inverse.
      \end{exparts}
    \end{answer}
  \recommended \item 
    What matrix has this one for its inverse?
    \begin{equation*}
      \begin{mat}[r]
        1  &3  \\
        2  &5
       \end{mat}
     \end{equation*}
     \begin{answer}
       We can use \nearbycorollary{cor:TwoByTwoInv}. 
       \begin{equation*}
         \frac{1}{1\cdot 5-2\cdot 3}\cdot
         \begin{mat}[r]
           5  &-3  \\
          -2  &1
         \end{mat}
         =\begin{mat}[r]
           -5  &3  \\
            2  &-1
         \end{mat}
       \end{equation*}  
      \end{answer}
  \item 
   How does the inverse operation interact with scalar multiplication 
   and addition of matrices? 
   \begin{exparts}
      \partsitem What is the inverse of \( rH \)?
      \partsitem Is \( (H+G)^{-1}=H^{-1}+G^{-1} \)?
    \end{exparts}
    \begin{answer}
      \begin{exparts}
        \partsitem The proof that the inverse is 
          \( r^{-1}H^{-1}=(1/r)\cdot H^{-1} \) 
          (provided, of course, that the matrix is invertible) is easy.
        \partsitem No.
          For one thing, the fact that $H+G$ has an inverse doesn't imply that
          $H$ has an inverse or that $G$ has an inverse.
          Neither of these matrices is invertible but their sum is.
          \begin{equation*}
            \begin{mat}[r]
              1  &0  \\
              0  &0
            \end{mat}
            \qquad
            \begin{mat}[r]
              0  &0  \\
              0  &1
            \end{mat}
          \end{equation*}
          Another point is that just because $H$ and $G$ each has an inverse
          doesn't mean $H+G$ has an inverse; here is an example.
          \begin{equation*}
            \begin{mat}[r]
              1  &0  \\
              0  &1
            \end{mat}
            \qquad
            \begin{mat}[r]
              -1  &0  \\
               0  &-1
            \end{mat}
          \end{equation*}
          Still a third point is that, even if the two matrices have inverses,
          and the sum has an inverse, doesn't imply that the equation holds:
          \begin{equation*}
            \begin{mat}[r]
              2  &0  \\
              0  &2
            \end{mat}^{-1}
            =\begin{mat}[r]
              1/2  &0  \\
              0    &1/2
            \end{mat}^{-1}
            \qquad
            \begin{mat}[r]
              3  &0  \\
              0  &3
            \end{mat}^{-1}
            =\begin{mat}[r]
              1/3  &0  \\
              0    &1/3
            \end{mat}^{-1}
          \end{equation*}
          but
          \begin{equation*}
            \begin{mat}[r]
              5  &0  \\
              0  &5
            \end{mat}^{-1}
            =\begin{mat}[r]
              1/5  &0  \\
              0    &1/5
            \end{mat}^{-1}
          \end{equation*}
          and $(1/2)_+(1/3)$ does not equal $1/5$.
      \end{exparts}  
    \end{answer}
  \recommended \item 
    Is \( (T^k)^{-1}=(T^{-1})^k \)?
    \begin{answer}
      Yes:
      \( T^k(T^{-1})^k=(TT\cdots T)\cdot (T^{-1}T^{-1}\cdots T^{-1})
         =T^{k-1}(TT^{-1})(T^{-1})^{k-1}=\dots=I \). 
    \end{answer}
  \item 
    Is \( H^{-1} \) invertible?
    \begin{answer}
       Yes, the inverse of \( H^{-1} \) is \( H \).  
    \end{answer}
  \item 
    For each real number \( \theta \) let
    \( \map{t_\theta}{\Re^2}{\Re^2} \) be represented with respect to the
    standard bases by this matrix.
    \begin{equation*}
      \begin{mat}
         \cos\theta  &-\sin\theta  \\
         \sin\theta  &\cos\theta
      \end{mat}
    \end{equation*}
    Show that \( t_{\theta_1+\theta_2}=t_{\theta_1}\cdot t_{\theta_2} \).
    Show also that \( {t_{\theta}}^{-1}=t_{-\theta} \).
    \begin{answer}
      One way to check that the first is true is with
      the angle sum formulas from trigonometry.
      \begin{align*}
        \begin{mat}
          \cos(\theta_1+\theta_2) &-\sin(\theta_1+\theta_2)  \\
          \sin(\theta_1+\theta_2) &\cos(\theta_1+\theta_2)
        \end{mat}
        &=\begin{mat}
          \cos\theta_1\cos\theta_2-\sin\theta_1\sin\theta_2
            &-\sin\theta_1\cos\theta_2-\cos\theta_1\sin\theta_2  \\
          \sin\theta_1\cos\theta_2+\cos\theta_1\sin\theta_2
            &\cos\theta_1\cos\theta_2-\sin\theta_1\sin\theta_2
        \end{mat}                                                    \\
        &=\begin{mat}
          \cos\theta_1 &-\sin\theta_1  \\
          \sin\theta_1 &\cos\theta_1
        \end{mat}
        \begin{mat}
          \cos\theta_2 &-\sin\theta_2  \\
          \sin\theta_2 &\cos\theta_2
        \end{mat}
      \end{align*}
      Checking the second equation in this way is similar.

      Of course, the equations can be not just checked but also understood
      by recalling that $t_\theta$ is the map that rotates 
      vectors about the origin through an angle of \( \theta \)~radians.  
    \end{answer}
  \item \label{exer:TwoByTwoInv}
    Do the calculations for the proof of \nearbycorollary{cor:TwoByTwoInv}.
    \begin{answer}
      There are two cases.
      For the first case we assume that \( a \) is nonzero.
      Then
      \begin{equation*}
        \grstep{-(c/a)\rho_1+\rho_2}
        \begin{pmat}{cc|cc}
           a   &b           &1     &0   \\
           0   &-(bc/a)+d   &-c/a  &1
         \end{pmat}
        =\begin{pmat}{cc|cc}
           a   &b           &1     &0   \\
           0   &(ad-bc)/a   &-c/a  &1
         \end{pmat}
      \end{equation*}
      shows that the matrix is invertible (in this \( a\neq 0 \) case) 
      if and only if  \( ad-bc\neq 0 \).
      To find the inverse, we finish with the Jordan half of the reduction.
      \begin{equation*}
        \grstep[(a/ad-bc)\rho_2]{(1/a)\rho_1}
        \begin{pmat}{cc|cc}
           1   &b/a     &1/a         &0   \\
           0   &1       &-c/(ad-bc)  &a/(ad-bc)
         \end{pmat}                                
        \grstep{-(b/a)\rho_2+\rho_1}
        \begin{pmat}{cc|cc}
           1   &0       &d/(ad-bc)   &-b/(ad-bc)   \\
           0   &1       &-c/(ad-bc)  &a/(ad-bc)
         \end{pmat}                                
      \end{equation*}

      The other case is the \( a=0 \) case.
      We swap to get $c$ into the $1,1$ position.
      \begin{equation*}
        \grstep{\rho_1\leftrightarrow\rho_2}\;
        \begin{pmat}{cc|cc}
          c  &d  &0  &1  \\
          0  &b  &1  &0
        \end{pmat}
      \end{equation*}
      This matrix is nonsingular if and only if
      both \( b \) and \( c \) are nonzero
      (which, under the case assumption that \( a=0 \), 
      holds if and only if \( ad-bc\neq 0 \)).
      To find the inverse
      we do the Jordan half.
      \begin{equation*}
        \grstep[(1/b)\rho_2]{(1/c)\rho_1}\;
        \begin{pmat}{cc|cc}
          1  &d/c  &0       &1/c  \\
          0  &1    &1/b     &0
        \end{pmat}                        
        \;\grstep{-(d/c)\rho_2+\rho_1}\;
        \begin{pmat}{cc|cc}
          1  &0  &-d/bc  &1/c  \\
          0  &1  &1/b    &0
        \end{pmat}
      \end{equation*}
      (Note that this is what is required, since \( a=0 \) gives that
      \( ad-bc=-bc \)).
    \end{answer}
  \item 
    Show that this matrix 
    \begin{equation*}
      H=\begin{mat}[r]
          1  &0   &1  \\
          0  &1   &0
        \end{mat}
    \end{equation*}
    has infinitely many right inverses.
    Show also that it has no left inverse.
    \begin{answer}
      With $H$ a $\nbym{2}{3}$ matrix,
      in looking for a matrix $G$ such that the combination $HG$
      acts as the $\nbyn{2}$ identity we
      need $G$ to be $\nbym{3}{2}$. 
      Setting up the equation
      \begin{equation*}
          \begin{mat}[r]
             1  &0   &1  \\
             0  &1   &0
           \end{mat}
          \begin{mat}
             m  &n  \\
             p  &q  \\
             r  &s
          \end{mat}
        =
          \begin{mat}[r]
             1  &0  \\
             0  &1
          \end{mat}
      \end{equation*}
      and solving the resulting linear system
      \begin{equation*}
        \begin{linsys}{6}
           m  &   &   &   &+r &   &=    &1  \\
              &n  &   &   &   &+s &=    &0  \\
              &   &p  &   &   &   &=    &0  \\
              &   &   &q  &   &   &=    &1
         \end{linsys}
      \end{equation*}
      gives infinitely many solutions.
      \begin{equation*}
        \set{\colvec{m \\ n \\ p \\ q \\ r \\ s}
              =\colvec[r]{1 \\ 0 \\ 0 \\ 1 \\ 0 \\ 0}
              +r\cdot \colvec[r]{-1 \\ 0 \\ 0 \\ 0 \\ 1 \\ 0}
              +s\cdot \colvec[r]{0 \\ -1 \\ 0 \\ 0 \\ 0 \\ 1}
             \suchthat r,s\in\Re}
      \end{equation*}
      Thus $H$ has infinitely many right inverses.

      As for left inverses, the equation
      \begin{equation*}
         \begin{mat}
            a  &b  \\
            c  &d
         \end{mat}
         \begin{mat}[r]
             1  &0   &1  \\
             0  &1   &0
          \end{mat}
          =\begin{mat}[r]
             1  &0  &0  \\
             0  &1  &0  \\
             0  &0  &1
           \end{mat}
      \end{equation*}
      gives rise to a linear system with nine equations and four unknowns.
      \begin{equation*}
        \begin{linsys}{6}
          a & & & & & & & & & & &= &1 \\
            & &b& & & & & & & & &= &0 \\
          a & & & & & & & & & & &= &0 \\
            & & & &c& & & & & & &= &0 \\
            & & & & & &d& & & & &= &1 \\
            & & & &c& & & & & & &= &0 \\
            & & & & & & & &e& & &= &0 \\
            & & & & & & & & & &f&= &0 \\
            & & & & & & & &e& & &= &1 
        \end{linsys}
      \end{equation*}
      This system is inconsistent (the first equation conflicts
      with the third, as do the seventh and ninth) 
      and so there is no left inverse.
    \end{answer}
  \item 
    In the review of inverses example, % \nearbyexample{ex:ProjLeftInvOfEmbed},
    starting this subsection, how many left inverses has \( \iota \)?
    \begin{answer}
      With respect to the standard bases we have
      \begin{equation*}
        \rep{\iota}{\stdbasis_2,\stdbasis_3}
        =\begin{mat}[r]
          1  &0  \\
          0  &1  \\
          0  &0 
        \end{mat}
      \end{equation*}
      and setting up the equation to find the matrix inverse
      \begin{equation*}
        \begin{mat}
          a &b &c \\
          d &e &f
        \end{mat}
        \begin{mat}[r]
          1 &0  \\
          0 &1  \\
          0 &0
        \end{mat}
        =\begin{mat}[r]
          1 &0  \\
          0 &1 
        \end{mat}
        =\rep{\identity}{\stdbasis_2,\stdbasis_2}
      \end{equation*}
      gives rise to a linear system.
      \begin{equation*}
        \begin{linsys}{6}
          a & & & & & & & & & & &= &1 \\
            & &b& & & & & & & & &= &0 \\
            & & & & & &d& & & & &= &0 \\
            & & & & & & & &e& & &= &1 
        \end{linsys}
      \end{equation*}
      There are infinitely many solutions in $a,\ldots,f$ 
      to this system because two of these variables are entirely unrestricted
      \begin{equation*}
        \set{\colvec{a \\ b \\ c \\ d \\ e \\ f}
             =\colvec[r]{1 \\ 0 \\ 0 \\ 0 \\ 1 \\ 0}
              +c\cdot \colvec[r]{0 \\ 0 \\ 1 \\ 0 \\ 0 \\ 0}
              +f\cdot \colvec[r]{0 \\ 0 \\ 0 \\ 0 \\ 0 \\ 1}
             \suchthat c,f\in\Re}
      \end{equation*}
      and so there are infinitely many solutions to the matrix equation.
      \begin{equation*}
        \set{\begin{mat}
               1  &0  &c  \\
               0  &1  &f
             \end{mat}
             \suchthat c,f\in\Re}
      \end{equation*}
      With the bases still fixed at $\stdbasis_2,\stdbasis_2$, 
      for instance taking $c=2$ and 
      $f=3$ gives a matrix representing this map.
      \begin{equation*}
        \colvec{x \\ y \\ z}
        \;\mapsunder{f_{2,3}}\;
        \colvec{x+2z \\ y+3z}
      \end{equation*}
      The check that $\composed{f_{2,3}}{\iota}$ is the identity map on
      $\Re^2$ is easy.
    \end{answer}
  \item  
    If a matrix has infinitely many right-inverses, can it have infinitely
    many left-inverses?
    Must it have?
    \begin{answer}
      By \nearbylemma{le:LeftAndRightInvEqual} it cannot have infinitely many 
      left inverses, because
      a matrix with both left and right inverses has only one of each (and
      that one of each is one of both\Dash the left and right inverse matrices
      are equal).  
    \end{answer}
  \item Assume that $\map{g}{V}{W}$ is linear.
    One of these is true, the other is false.
    Which is which?
    \begin{exparts}
      \partsitem If $\map{f}{W}{V}$ is a left inverse of $g$ then $f$
        must be linear.
      \partsitem If $\map{f}{W}{V}$ is a right inverse of $g$ then $f$
        must be linear.
    \end{exparts}
    \begin{answer}
      \begin{exparts}
        \partsitem True,
          It must be linear, 
          as the proof from Theorem~II.\ref{th:OOHomoEquivalence} shows.
        \partsitem False.
          It may be linear, but it need not be.
          Consider the projection map $\map{\pi}{\Re^3}{\Re^2}$ described
          at the start of this subsection.
          Define $\map{\eta}{\Re^2}{\Re^3}$ in this way.
          \begin{equation*}
            \colvec{x  \\  y}\mapsto\colvec{x  \\ y \\ 1}
          \end{equation*}
          It is a right inverse of $\pi$ because $\composed{\pi}{\eta}$
          does this.  
          \begin{equation*}
            \colvec{x  \\  y}\mapsto\colvec{x  \\ y \\ 1}\mapsto\colvec{x \\ y}
          \end{equation*}
          It is not linear because it does not map the zero vector to the zero 
          vector.
      \end{exparts}
    \end{answer}
  \recommended \item
    Assume that \( H \) is invertible and that \( HG \) is the zero matrix.
    Show that \( G \) is a zero matrix.
    \begin{answer}
      The associativity of matrix multiplication gives
      on the one hand
      \( H^{-1}(HG)=H^{-1}Z=Z \), and on the other that
      \( H^{-1}(HG)=(H^{-1}H)G=IG=G \).
    \end{answer}
  \item 
    Prove that if \( H \) is invertible then
    the inverse commutes with a matrix \( GH^{-1}=H^{-1}G \) 
    if and only if $H$ itself commutes with that matrix \( GH=HG \).
    \begin{answer}
       Multiply both sides of the first equation by $H$.
    \end{answer}
  \recommended \item
    Show that if \( T \) is square and if \( T^4 \) is the zero matrix
    then \( (I-T)^{-1}=I+T+T^2+T^3 \).
    Generalize.
    \begin{answer}
      Checking that when $I-T$ is multiplied on both sides by that expression
      (assuming that $T^4$ is the zero matrix) then the result is the 
      identity matrix is easy.
      The obvious generalization is that if \( T^n \) is the zero matrix
      then \( (I-T)^{-1}=I+T+T^2+\cdots+T^{n-1} \); the check again is
      easy.  
    \end{answer}
  \recommended \item 
    Let \( D \) be diagonal.
    Describe \( D^2 \), \( D^3 \), \ldots\thinspace, etc.
    Describe \( D^{-1} \), \( D^{-2} \), \ldots\thinspace, etc.
    Define \( D^0 \) appropriately.
    \begin{answer}
      The powers of the matrix are formed by taking the powers of the
      diagonal entries.
      That is,  \( D^2 \) is all zeros except for diagonal entries of
      \( {d_{1,1}}^2 \), \( {d_{2,2}}^2 \), etc.
      This suggests defining \( D^0 \) to be the identity matrix.  
    \end{answer}
  \item
    Prove that any matrix row-equivalent to an invertible matrix is also
    invertible.
    \begin{answer}
      Assume that $B$ is row equivalent to $A$ and that $A$ is invertible.
      Because they are row-equivalent, there is a sequence of row steps 
      to reduce one to the other.
      We can do that reduction with matrices, for instance, $A$ can 
      change by row operations to $B$ as $B=R_n\cdots R_1A$.
      This equation gives $B$ as a product of invertible matrices and
      by \nearbylemma{lem:ProdInvIsInv} then, $B$ is also invertible.  
    \end{answer}
  \item 
    \textit{The first question below appeared as
    \nearbyexercise{exer:RankProdLeqRankFacts}.}
    \begin{exparts}
      \partsitem Show that the rank of the product of two matrices is less than
        or equal to the minimum of the rank of each.
      \partsitem Show that if \( T \)  and \( S \) are square then \( TS=I \)
         if and only if \( ST=I \).
    \end{exparts}
    \begin{answer}
      \begin{exparts}
        \partsitem See the answer to
          \nearbyexercise{exer:RankProdLeqRankFacts}.
        \partsitem We will show that both conditions are equivalent to
          the condition that the two matrices be nonsingular. 

          As \( T \) and \( S \) are square and their product is defined,
          they are equal-sized, say \( \nbyn{n} \).
          Consider the \( TS=I \) half.
          By the prior item the rank of \( I \) is less than or equal to
          the minimum of the rank of \( T \) and the rank of \( S \).
          But the rank of \( I \) is \( n \), so the rank of \( T \) and
          the rank of \( S \) must each be \( n \).
          Hence each is nonsingular.

          The same argument shows that \( ST=I \) implies that 
          each is nonsingular.
      \end{exparts}  
    \end{answer}
  \item 
    Show that the inverse of a permutation matrix is its transpose.
    \begin{answer}
      Inverses are unique, so we need only show that it works.
      The check appears above as
      \nearbyexercise{exer:PermTimesTransEqId}.  
    \end{answer}
  \item 
    \textit{The first two parts of this question appeared as
    \nearbyexercise{exer:TranspAndMult}.}
    \begin{exparts}
      \partsitem Show that \( \trans{(GH)}=\trans{H}\trans{G} \).
      \partsitem A square matrix is {\em symmetric\/} if each 
        \( i,j \) entry equals the
        \( j,i \) entry (that is, if the matrix equals its transpose).
        Show that 
        the matrices \( H\trans{H} \) and \( \trans{H}H \) are symmetric.
      \partsitem Show that the inverse of the transpose is the transpose 
        of the inverse.
      \partsitem Show that the inverse of a symmetric matrix is symmetric.
    \end{exparts}
    \begin{answer}
      \begin{exparts}
        \partsitem See the answer for \nearbyexercise{exer:TranspAndMult}.
        \partsitem See the answer for \nearbyexercise{exer:TranspAndMult}.
        \partsitem Apply the first part to  
          \( I=AA^{-1} \) to get
          $I=\trans{I}=\trans{(AA^{-1})}=\trans{(A^{-1})}\trans{A}$.
        \partsitem Apply the prior item with \( \trans{A}=A \),
          as \( A \) is symmetric.
      \end{exparts}  
    \end{answer}
  \recommended \item
    \textit{The items starting this question appeared as
    \nearbyexercise{exer:ZeroDivisor}.}
    \begin{exparts}
      \partsitem Prove that the composition of the projections
        \( \map{\pi_x,\pi_y}{\Re^3}{\Re^3} \) is the zero map despite that
        neither is the zero map.
      \partsitem Prove that the composition of the derivatives
        \( \map{d^2/dx^2,\,d^3/dx^3}{\polyspace_4}{\polyspace_4} \)
        is the zero map despite that neither map is the zero map.
      \partsitem Give matrix equations representing each of the prior two
        items.
    \end{exparts}
    When two things multiply to give zero despite that neither is zero, each is
    said to be a \definend{zero divisor}.\index{zero division}
    Prove that no zero divisor is invertible.
    \begin{answer}
      For the answer to the items making up the first half, see 
      \nearbyexercise{exer:ZeroDivisor}.
      For the proof in the second half, assume that $A$ is a zero divisor so
      there is a nonzero matrix $B$ with $AB=Z$ 
      (or else $BA=Z$; this case is similar), 
      If $A$ is invertible
      then $A^{-1}(AB)=(A^{-1}A)B=IB=B$ but also
      $A^{-1}(AB)=A^{-1}Z=Z$, contradicting that $B$ is nonzero.
    \end{answer}
  \item 
    In real number algebra, there are exactly two numbers, $1$ and $-1$, 
    that are their own multiplicative inverse.
    Does \( H^2=I \) have exactly two solutions for \( \nbyn{2} \)
    matrices?
    \begin{answer}
      Here are four solutions to \( H^2=I \).
      \begin{equation*}
        \begin{mat}[r]
          \pm 1  &0  \\
          0      &\pm 1
        \end{mat}
      \end{equation*}   
    \end{answer}
  \item 
   Is the relation `is a two-sided inverse of' transitive?
   Reflexive?
   Symmetric?
   \begin{answer}
     It is not reflexive since, for instance, 
     \begin{equation*}
       H=\begin{mat}[r]
         1  &0  \\
         0  &2
       \end{mat}
     \end{equation*}
     is not a two-sided inverse of itself.
     The same example shows that it is not transitive.
     That matrix has this two-sided inverse
     \begin{equation*}
       G=\begin{mat}[r]
         1  &0  \\
         0  &1/2
       \end{mat}
     \end{equation*}
     and while \( H \) is a two-sided inverse of \( G \) and \( G \)
     is a two-sided inverse of \( H \), we know that \( H \) is not a two-sided
     inverse of \( H \).
     However, the relation is symmetric:~if \( G \) is a two-sided inverse of 
     \( H \) then
     \( GH=I=HG \) and therefore \( H \) is also a two-sided
     inverse of \( G \).  
   \end{answer}
  \item  
    \cite{Monthly51p614}
    Prove: if the sum of the elements of a square
    matrix is \( k \), then the sum of the elements in each row of the
    inverse matrix is \( 1/k \).
    \begin{answer}
      \answerasgiven %
      Let \( A \) be \( \nbyn{m} \), non-singular, with the stated property.
      Let \( B \) be its inverse.
      Then for \( n\leq m \),
      \begin{equation*}
        1
        =\sum_{r=1}^{m}\delta_{nr}
        =\sum_{r=1}^{m}\sum_{s=1}^{m}b_{ns}a_{sr}
        =\sum_{s=1}^{m}\sum_{r=1}^{m}b_{ns}a_{sr}
        =k\sum_{s=1}^{m}b_{ns}
      \end{equation*}
      (\( A \) is singular if \( k=0 \)).  
   \end{answer}
\end{exercises}
\index{matrix!inverse|)}






