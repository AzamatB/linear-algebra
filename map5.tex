% Chapter 3, Section 5 _Linear Algebra_ Jim Hefferon
%  http://joshua.smcvt.edu/linalg.html
%  2001-Jun-12
\section{Change of Basis}
\index{change of basis|(}
Representations vary with the bases.
For instance, 
$\vec{e}_1\in\Re^2$ has two different representations
\begin{equation*}
  \rep{\vec{e}_1}{\stdbasis_2}=\colvec{1 \\ 0}
  \qquad
  \rep{\vec{e}_1}{B}=\colvec{1/2 \\ 1/2}
\end{equation*}
with respect to the standard basis and this one.
\begin{equation*}
  B=\sequence{\colvec{1 \\ 1},\colvec{1 \\ -1}}
\end{equation*}
The same is true for maps;
with respect to the basis pairs $\stdbasis_2,\stdbasis_2$ and $\stdbasis_2,B$, 
the identity map
has two different representations.
\begin{equation*}
  \rep{\text{id}}{\stdbasis_2,\stdbasis_2}=
   \begin{mat}[r]
     1  &0  \\
     0  &1
   \end{mat}
   \qquad
  \rep{\text{id}}{\stdbasis_2,B}=
   \begin{mat}[r]
     1/2  &1/2  \\
     1/2  &-1/2
   \end{mat}
\end{equation*}
With our point of view that the objects of our studies are vectors and
maps, by fixing bases 
we are adopting a scheme of tags or names for these objects that are
convenient for calculations.
We will now see how to translate among these names, so we will
see exactly how the representations vary as the bases vary.












\subsection{Changing Representations of Vectors}
In converting 
$\rep{\vec{v}}{B}$ to $\rep{\vec{v}}{D}$
the underlying vector $\vec{v}$ doesn't change.
Thus,  
this translation is accomplished by the identity map on the space,
described so that
the domain space vectors are represented with respect to $B$ and
the codomain space vectors are represented with respect to $D$. 
\index{arrow diagram}
%<*ChangeRepresentationOfVectorArrowDiagram>
\begin{equation*}
  \begin{CD}
    V_{\wrt{B}}                      \\
    @V\scriptsize\identity VV   \\
    V_{\wrt{D}}
  \end{CD}
\end{equation*}
(The diagram is
vertical to fit with the ones in the next subsection.)
%</ChangeRepresentationOfVectorArrowDiagram>

\begin{definition}  \label{df:ChangeOfBasisMatrix}
%<*df:ChangeOfBasisMatrix>
The \definend{change of basis matrix}\index{vector!representation of}%
\index{matrix!change of basis}\index{basis!change of}
for bases \( B,D\subset V \) is the representation of the identity
map  \( \map{\identity}{V}{V} \) with respect to those bases.
\begin{equation*}
  \rep{\identity}{B,D}=
  \begin{pmat}{c|@{\hspace{1em}}c@{\hspace{1em}}|c}
     \vdots                  &             &\vdots                  \\
     \rep{\vec{\beta}_1}{D}  &\;\cdots\;   &\rep{\vec{\beta}_n}{D}  \\
     \vdots                  &             &\vdots
  \end{pmat}
\end{equation*}
%\( B=\basis{\beta}{n} \).
%</df:ChangeOfBasisMatrix>
\end{definition}

\begin{remark}
Perhaps a better name would be `change of representation matrix' but this
one is standard.  
\end{remark}

\begin{lemma}  \label{le:ChBasisMatDoesChBases}
%<*lm:ChBasisMatDoesChBases>
Left-multiplication by the change of basis matrix for \( B,D \)
converts a representation with respect to \( B \) to one with respect to
\( D \).
Conversely, if left-multiplication by a matrix changes bases 
$M\cdot\rep{\vec{v}}{B}=\rep{\vec{v}}{D}$
then $M$ is a change of basis matrix.
%</lm:ChBasisMatDoesChBases>
\end{lemma}

\begin{proof}
%<*pf:ChBasisMatDoesChBases>
The first sentence holds 
because matrix-vector multiplication represents a map application
\( \rep{\identity}{B,D}\cdot\rep{\vec{v}}{B}=\rep{\,\identity(\vec{v})\,}{D}
  =\rep{\vec{v}}{D} \) for each \( \vec{v} \). 
For the second sentence,
with respect to $B,D$ the matrix $M$ represents a linear
map whose action is to map each vector to itself, and is therefore
the identity map.
%</pf:ChBasisMatDoesChBases>
\end{proof}

\begin{example}
With these bases for \( \Re^2 \),
\begin{equation*}
  B=
  \sequence{
            \colvec[r]{2 \\ 1},
            \colvec[r]{1 \\ 0} }
  \qquad
  D=
  \sequence{
            \colvec[r]{-1 \\ 1},
            \colvec[r]{1 \\ 1} }
\end{equation*}
because
\begin{equation*}
  \rep{\,\identity(\colvec[r]{2 \\ 1})}{D}
  =\colvec[r]{-1/2 \\ 3/2}_D
  \qquad
  \rep{\,\identity(\colvec[r]{1 \\ 0})}{D}
  =\colvec[r]{-1/2 \\ 1/2}_D
\end{equation*}
the change of basis matrix is this.
\begin{equation*}
  \rep{\rm id}{B,D}
  =
    \begin{mat}[r]
       -1/2  &-1/2  \\
        3/2  &1/2
    \end{mat}
\end{equation*}
For instance, if we 
finding the representations of
$\vec{e}_2$ 
\begin{equation*}
  \rep{\colvec[r]{0 \\ 1} }{B}
  =\colvec[r]{1 \\ -2}
  \qquad
  \rep{\colvec[r]{0 \\ 1} }{D}
  =\colvec[r]{1/2 \\ 1/2}
\end{equation*}
then the matrix will do the conversion. 
\begin{equation*}
    \begin{mat}[r]
       -1/2  &-1/2  \\
        3/2  &1/2
    \end{mat}
  \colvec[r]{1 \\ -2}
  =
  \colvec[r]{1/2 \\ 1/2}
\end{equation*}
\end{example}

We finish this subsection by recognizing that 
the change of basis matrices form a familiar set.

\begin{lemma}    \label{le:NonSingIsChBasis}
%<*lm:NonSingIsChBasis>
A matrix changes bases if and only if it is nonsingular.
%</lm:NonSingIsChBasis>
\end{lemma}

\begin{proof}
%<*pf:NonSingIsChBasis0>
For the `only if' direction, 
if left-multiplication by a matrix changes bases then
the matrix represents an invertible function,
simply because we can invert the function by changing the bases back.
Such a matrix is itself invertible, and so is nonsingular.
%</pf:NonSingIsChBasis0>

%<*pf:NonSingIsChBasis1>
To finish we will show that any nonsingular matrix $M$ 
performs a change of basis operation from any given starting basis $B$ 
to some ending basis.
Because the matrix is nonsingular it will Gauss-Jordan reduce to the
identity.
If the matrix is the identity~$I$ then the statement is obvious.
Otherwise there are elementary reduction matrices such that
$R_r\cdots R_1\cdot M=I$ with $r\geq 1$.
Elementary matrices are invertible and their inverses are also elementary
so multiplying both sides of that equation from the left 
by ${R_r}^{-1}$, then by ${R_{r-1}}^{-1}$, etc., gives 
$M$ as a product of elementary matrices
$M={R_1}^{-1}\cdots {R_r}^{-1}$.
(We've combined ${R_r}^{-1}I$ to make ${R_r}^{-1}$; 
because $r\geq 1$ we can always make the $I$ disappear in this way,
which we need to do because it isn't an elementary matrix.)
%</pf:NonSingIsChBasis1>

%<*pf:NonSingIsChBasis2>
Thus, we will be done if we show that elementary matrices 
change a given basis to another basis, for then 
${R_r}^{-1}$ changes $B$ to some other basis $B_r$, and
${R_{r-1}}^{-1}$ changes $B_r$ to some $B_{r-1}$, etc., and
the net effect is that $M$ changes $B$ to $B_1$.
We will prove this by covering the three types of elementary matrices
separately; here are the three cases.
\begin{equation*}
  M_{i}(k)
    \colvec{
       c_1     \\
       \vdots  \\
       c_i    \\
       \vdots  \\
       c_n  }
  =
    \colvec{
       c_1     \\
       \vdots  \\
       kc_i    \\
       \vdots  \\
       c_n  }
  \quad
  P_{i,j}
    \colvec{
       c_1     \\
       \vdots  \\
       c_i    \\
       \vdots  \\
       c_j    \\
       \vdots  \\
       c_n  }
  =
    \colvec{
       c_1     \\
       \vdots  \\
       c_j    \\
       \vdots  \\
       c_i    \\
       \vdots  \\
       c_n  }
  \quad
  C_{i,j}(k)
    \colvec{
       c_1     \\
       \vdots  \\
       c_i    \\
       \vdots  \\
       c_j    \\
       \vdots  \\
       c_n  }
  =
    \colvec{
       c_1     \\
       \vdots  \\
       c_i    \\
       \vdots  \\
       kc_i+c_j    \\
       \vdots  \\
       c_n  }
\end{equation*}
%</pf:NonSingIsChBasis2>
%<*pf:NonSingIsChBasis3>
Applying a row-multiplication matrix~$M_{i}(k)$
changes a representation with respect to
\( \sequence{\vec{\beta}_1,\dots,\vec{\beta}_i,\dots,\vec{\beta}_n } \)
to one with respect to
\( \sequence{\vec{\beta}_1,\dots,(1/k)\vec{\beta}_i,\dots,\vec{\beta}_n } \).
\begin{multline*}
   \vec{v}= c_1\cdot\vec{\beta}_1+\dots+c_i\cdot\vec{\beta}_i
                                   +\dots+c_n\cdot\vec{\beta}_n  
   \\  \mapsto\;                                                       
    c_1\cdot\vec{\beta}_1+\dots+kc_i\cdot(1/k)\vec{\beta}_i+\dots
                                  +c_n\cdot\vec{\beta}_n=\vec{v}     
\end{multline*}
We can easily see that the second one is a basis, given that the first is a 
basis and that $k\neq 0$ is a restriction in the definition of a 
row-multiplication matrix.
%</pf:NonSingIsChBasis3>
%<*pf:NonSingIsChBasis4>
Similarly, left-multiplication by a row-swap matrix $P_{i,j}$
changes a representation with respect to the basis
\( \sequence{\vec{\beta}_1,\dots,\vec{\beta}_i,\dots,\vec{\beta}_j,
  \dots,\vec{\beta}_n } \)
into one with respect to this basis
\( \sequence{\vec{\beta}_1,\dots,\vec{\beta}_j,\dots,\vec{\beta}_i,
  \dots,\vec{\beta}_n } \).
\begin{multline*}
   \vec{v}= c_1\cdot\vec{\beta}_1+\dots+c_i\cdot\vec{\beta}_i
                   +\dots+c_j\vec{\beta}_j+\dots+c_n\cdot\vec{\beta}_n  
   \\  \mapsto\;                                                       
    c_1\cdot\vec{\beta}_1+\dots+c_j\cdot\vec{\beta}_j+\dots
               +c_i\cdot\vec{\beta}_i+\dots+c_n\cdot\vec{\beta}_n=\vec{v}     
\end{multline*}
%</pf:NonSingIsChBasis4>
%<*pf:NonSingIsChBasis5>
And, a representation with respect to
\( \sequence{\vec{\beta}_1,\dots,\vec{\beta}_i,\dots,\vec{\beta}_j,
  \dots,\vec{\beta}_n } \)
changes via left-multiplication by a row-combination matrix
$C_{i,j}(k)$ into a representation with respect to 
\( \sequence{\vec{\beta}_1,\dots,\vec{\beta}_i-k\vec{\beta}_j,
  \dots,\vec{\beta}_j,\dots,\vec{\beta}_n } \)
\begin{multline*}
   \vec{v}= c_1\cdot\vec{\beta}_1+\dots+c_i\cdot\vec{\beta}_i
                       +c_j\vec{\beta}_j+\dots+c_n\cdot\vec{\beta}_n  
   \\  \mapsto\;                                                       
    c_1\cdot\vec{\beta}_1+\dots+c_i\cdot(\vec{\beta}_i-k\vec{\beta}_j)+\dots
          +(kc_i+c_j)\cdot\vec{\beta}_j+\dots+c_n\cdot\vec{\beta}_n=\vec{v} 
\end{multline*}
(the definition of reduction matrices specifies that \( i\neq j \) and
\( k\neq 0 \)).
%</pf:NonSingIsChBasis5>
\end{proof}

\begin{corollary}  \label{co:MatrixNonsingularIffChangesBasis}
%<*co:MatrixNonsingularIffChangesBasis>
A matrix is nonsingular if and only if it represents the identity map
with respect to some pair of bases.
%</co:MatrixNonsingularIffChangesBasis>
\end{corollary}

%\begin{proof}
%One direction of the double implication is easy:~if a matrix represents the
%identity map then it must be square and, 
%because the identity map has a rank equal to the dimension
%of its domain, it must have a rank equal to the number of its
%columns, and so must be nonsingular.

%For the other implication we must show that if a vector space 
%$V$ is $n$-dimensional and $H$ is a
%nonsingular $\nbyn{n}$ matrix, then there are bases $B$ and $D$ such
%that $H=\rep{\text{id}}{B,D}$.

%Fix any starting basis $B$.
%Write $H$ as a product of elementary reduction matrices
%$H=R_1\cdots R_r$.
%By the previous result, we have that $R_r$ is the change of basis matrix
%from the basis $B$ to some other basis $B_r$.
%Similarly, the previous result gives that $R_{r-1}$ is the change of basis
%matrix from $B_r$ to some $B_{r-1}$, etc.
%Thus
%\begin{equation*}
%  H=R_1\cdots R_r=\rep{\text{id}}{B_2,B_1}\cdots\rep{\text{id}}{B,B_r}
%     =\rep{\text{id}}{B,B_1}
%\end{equation*}
%and we are finished on recognizing that $B_1$ is the required basis $D$.
%\end{proof}

In the next subsection we will
see how to translate among representations of 
maps, that is, how to change
$\rep{h}{B,D}$ to $\rep{h}{\hat{B},\hat{D}}$.
The above corollary is a special case of this, where the domain and range are
the same space, and where the map is the identity map.


\begin{exercises}
  \recommended \item 
    In \( \Re^2 \), where
    \begin{equation*}
      D=\sequence{\colvec[r]{2 \\ 1},\colvec[r]{-2 \\ 4}}
    \end{equation*}
    find the change of basis matrices from \( D \) to \( \stdbasis_2 \) and 
    from \( \stdbasis_2 \) to \( D \).
    Multiply the two.
    \begin{answer}
      For the matrix to change bases from $D$ to $\stdbasis_2$ we need that
      $\rep{\identity(\vec{\delta}_1)}{\stdbasis_2}
        =\rep{\vec{\delta}_1}{\stdbasis_2}$ 
      and that 
      $\rep{\identity(\vec{\delta}_2)}{\stdbasis_2}
        =\rep{\vec{\delta}_2}{\stdbasis_2}$.
      Of course, the representation of a vector in $\Re^2$ with respect to 
      the standard basis is easy.
      \begin{equation*}
        \rep{\vec{\delta}_1}{\stdbasis_2}=\colvec[r]{2 \\ 1}
        \qquad
        \rep{\vec{\delta}_2}{\stdbasis_2}=\colvec[r]{-2 \\ 4}
      \end{equation*}
      Concatenating those two together to make the columns of the change of
      basis matrix gives this. 
      \begin{equation*}
        \rep{\identity}{D,\stdbasis_2}
        =\begin{mat}[r]
          2     &-2    \\
          1     &4
        \end{mat}
      \end{equation*}
      For the change of basis matrix in the other direction we can
      calculate $\rep{\identity(\vec{e}_1)}{D}=\rep{\vec{e}_1}{D}$ and 
      $\rep{\identity(\vec{e}_2)}{D}=\rep{\vec{e}_2}{D}$ (this job is routine)
      or we can take the inverse of the above matrix.
      Because of the formula for the inverse of a $\nbyn{2}$ matrix, 
      this is easy. 
      \begin{equation*}
        \rep{\identity}{\stdbasis_2,D}=
        \frac{1}{10}\cdot\begin{mat}[r]
          4  &2  \\
         -1  &2
        \end{mat}
        =\begin{mat}[r]
          4/10  &2/10  \\
         -1/10  &2/10
        \end{mat}
      \end{equation*}
    \end{answer}
  \recommended \item \label{exer:ChngeBasMatsRTwo}
    Find the change of basis matrix for \( B,D\subseteq\Re^2 \).
    \begin{exparts*}
      \partsitem \( B=\stdbasis_2 \),
        \( D=\sequence{\vec{e}_2,\vec{e}_1} \)
      \partsitem \( B=\stdbasis_2 \),
        \( D=\sequence{\colvec[r]{1 \\ 2},\colvec[r]{1 \\ 4}} \)
      \partsitem  \( B=\sequence{\colvec[r]{1 \\ 2},\colvec[r]{1 \\ 4}} \),
        \( D=\stdbasis_2 \)
      \partsitem  \( B=\sequence{\colvec[r]{-1 \\ 1},\colvec[r]{2 \\ 2}} \),
        \( D=\sequence{\colvec[r]{0 \\ 4},\colvec[r]{1 \\ 3}} \)
    \end{exparts*}
    \begin{answer}
      In each case, concatenate the columns 
      $\rep{\identity(\vec{\beta}_1)}{D}=\rep{\vec{\beta}_1}{D}$
      and $\rep{\identity(\vec{\beta}_2)}{D}=\rep{\vec{\beta}_2}{D}$
      to make the change of basis matrix
      $\rep{\identity}{B,D}$.
      \begin{exparts*}
        \partsitem \( \begin{mat}[r]
                   0  &1  \\
                   1  &0
                 \end{mat} \)
        \partsitem \( \begin{mat}[r]
                   2  &-1/2  \\
                   -1 &1/2
                 \end{mat} \)
        \partsitem \( \begin{mat}[r]
                   1  &1     \\
                   2  &4
                 \end{mat} \)
        \partsitem \( \begin{mat}[r]
                   1  &-1    \\
                  -1  &2
                 \end{mat} \)
      \end{exparts*}  
    \end{answer}
  \item 
    For the bases in \nearbyexercise{exer:ChngeBasMatsRTwo}, 
    find the change of basis matrix in the other direction, from $D$ to $B$.
    \begin{answer}
       One way to go is to find 
       $\rep{\vec{\delta}_1}{B}$ and $\rep{\vec{\delta}_2}{B}$,
       and then concatenate them into the columns of the desired 
       change of basis matrix.
       Another way is to find the inverse of the matrices that answer
       \nearbyexercise{exer:ChngeBasMatsRTwo}. 
       \begin{exparts*}
        \partsitem
          $\begin{mat}[r]
            0  &1  \\
            1  &0
          \end{mat}$
        \partsitem \( \begin{mat}[r]
            1  &1  \\
            2  &4
          \end{mat} \)
        \partsitem \( \begin{mat}[r]
            2  &-1/2  \\
            -1 &1/2
          \end{mat} \)
        \partsitem \( \begin{mat}[r]
            2  &1  \\
            1  &1
          \end{mat} \)
      \end{exparts*} 
    \end{answer}
  \recommended \item
    Find the change of basis matrix for each \( B,D\subseteq\polyspace_2 \).
    \begin{exparts*}
      \partsitem \( B=\sequence{1,x,x^2},
               D=\sequence{x^2,1,x} \)
      \partsitem \( B=\sequence{1,x,x^2},
               D=\sequence{1,1+x,1+x+x^2} \)
      \partsitem \( B=\sequence{2,2x,x^2},
               D=\sequence{1+x^2,1-x^2,x+x^2} \)
    \end{exparts*}
    \begin{answer}
      The columns vector representations
      $\rep{\identity(\vec{\beta}_1)}{D}=\rep{\vec{\beta}_1}{D}$,
      and $\rep{\identity(\vec{\beta}_2)}{D}=\rep{\vec{\beta}_2}{D}$,
      and $\rep{\identity(\vec{\beta}_3)}{D}=\rep{\vec{\beta}_3}{D}$
      make the change of basis matrix
      $\rep{\identity}{B,D}$.
      \begin{exparts*}
        \partsitem \( \begin{mat}[r]
                   0  &0  &1  \\
                   1  &0  &0  \\
                   0  &1  &0
                 \end{mat} \)
        \partsitem \( \begin{mat}[r]
                   1  &-1 &0  \\
                   0  &1  &-1 \\
                   0  &0  &1
                 \end{mat} \)
        \partsitem \( \begin{mat}[r]
                   1  &-1 &1/2  \\
                   1  &1  &-1/2 \\
                   0  &2  &0
                 \end{mat} \)
      \end{exparts*}  
      E.g., for the first column of the first matrix,
      $1=0\cdot x^2+1\cdot 1+0\cdot x$.
     \end{answer}
  \recommended \item 
    Decide if each changes bases on \( \Re^2 \).
    To what basis is \( \stdbasis_2 \) changed?
    \begin{exparts*}
      \partsitem \( \begin{mat}[r]
                 5  &0  \\
                 0  &4
               \end{mat}  \)
      \partsitem \( \begin{mat}[r]
                 2  &1  \\
                 3  &1
               \end{mat}  \)
      \partsitem \( \begin{mat}[r]
                -1  &4  \\
                 2  &-8
               \end{mat}  \)
      \partsitem \( \begin{mat}[r]
                 1  &-1 \\
                 1  &1
               \end{mat}  \)
    \end{exparts*}
    \begin{answer}
       A matrix changes bases if and only if it is nonsingular.
       \begin{exparts}
        \partsitem This matrix is nonsingular and so changes bases.
          Finding to what basis \( \stdbasis_2 \) is changed 
          means finding $D$ such that 
          \begin{equation*}
            \rep{\identity}{\stdbasis_2,D}=
            \begin{mat}[r]
                 5  &0  \\
                 0  &4
            \end{mat}
          \end{equation*}
          and by the definition of how a matrix represents a linear map,
          we have this.
          \begin{equation*}
            \rep{\identity(\vec{e}_1)}{D}=\rep{\vec{e}_1}{D}=\colvec[r]{5 \\ 0}
            \qquad
            \rep{\identity(\vec{e}_2)}{D}=\rep{\vec{e}_2}{D}=\colvec[r]{0 \\ 4}
          \end{equation*}
          Where
          \begin{equation*}
            D=\sequence{\colvec{x_1 \\ y_1},\colvec{x_2 \\ y_2}}
          \end{equation*}
          we can either solve the system
          \begin{equation*}
            \colvec[r]{1 \\ 0}
            =5\colvec{x_1 \\ y_1}+0\colvec{x_2 \\ y_1}
            \qquad
            \colvec[r]{0 \\ 1}
            =0\colvec{x_1 \\ y_1}+4\colvec{x_2 \\ y_1}
          \end{equation*}
          or else just spot the answer
          (thinking of the proof of \nearbylemma{le:NonSingIsChBasis}).
          \begin{equation*}
            D=\sequence{\colvec[r]{1/5 \\ 0},
                        \colvec[r]{0 \\ 1/4}}
          \end{equation*}
        \partsitem Yes, this matrix is nonsingular and so changes bases.
          To calculate $D$, we proceed as above with 
          \begin{equation*}
            D=\sequence{\colvec{x_1 \\ y_1},
                        \colvec{x_2 \\ y_2}}
          \end{equation*}
          to solve 
          \begin{equation*}
            \colvec[r]{1 \\ 0}
            =2\colvec{x_1 \\ y_1}+3\colvec{x_2 \\ y_1}
            \quad\text{and}\quad
            \colvec[r]{0 \\ 1}
            =1\colvec{x_1 \\ y_1}+1\colvec{x_2 \\ y_1}
          \end{equation*}
          and get this.
          \begin{equation*}
            D=\sequence{\colvec[r]{-1 \\ 3},
                        \colvec[r]{1 \\ -2}}
          \end{equation*}
        \partsitem No, this matrix does not change bases because it
           is singular.
        \partsitem Yes, this matrix changes bases because it is nonsingular.
          The calculation of the changed-to basis is as above. 
          \begin{equation*}
            D=\sequence{\colvec[r]{1/2 \\ -1/2},
                        \colvec[r]{1/2 \\ 1/2}}
          \end{equation*}
      \end{exparts}  
    \end{answer}
  \item 
    Find bases such that this matrix represents the identity map
    with respect to those bases.
    \begin{equation*}
      \begin{mat}[r]
        3  &1  &4  \\
        2  &-1 &1  \\
        0  &0  &4
      \end{mat}
    \end{equation*}
    \begin{answer}
      This question has many different solutions.
      One way to proceed is to make up any basis $B$ for any space,
      and then compute the appropriate $D$ (necessarily for the same space,
      of course).
      Another, easier, way to proceed is to fix the codomain as $\Re^3$ and
      the codomain basis as $\stdbasis_3$.
      This way (recall that the representation of any vector with respect
      to the standard basis is just the vector itself),
      we have this. 
      \begin{equation*}
        B=\sequence{\colvec[r]{3 \\ 2 \\ 0},
                    \colvec[r]{1 \\ -1 \\ 0},
                    \colvec[r]{4 \\ 1 \\ 4}  }
        \qquad
        D=\stdbasis_3
      \end{equation*}  
    \end{answer}
  \item 
    Consider the vector space of real-valued functions with basis
    \( \sequence{\sin(x),\cos(x)} \).
    Show that \( \sequence{2\sin(x)+\cos(x),3\cos(x)} \)
    is also a basis for this space.
    Find the change of basis matrix in each direction.
    \begin{answer}
      Checking that \( B=\sequence{2\sin(x)+\cos(x),3\cos(x)} \) is a basis
      is routine.
      Call the natural basis $D$.
      To compute the change of basis matrix $\rep{\identity}{B,D}$ we must
      find $\rep{2\sin(x)+\cos(x)}{D}$ and $\rep{3\cos(x)}{D}$, that is,
      we need $x_1,y_1, x_2,y_2$ such that these equations hold.
      \begin{align*}
        x_1\cdot \sin(x)+y_1\cdot\cos(x) &= 2\sin(x)+\cos(x) \\
        x_2\cdot \sin(x)+y_2\cdot\cos(x) &= 3\cos(x) 
      \end{align*}
      Obviously this is the answer.
      \begin{equation*}
        \rep{\identity}{B,D}
        =\begin{mat}[r]
          2  &0  \\
          1  &3
        \end{mat}
      \end{equation*}
      For the change of basis matrix in the other direction we could 
      look for $\rep{\sin(x)}{B}$ and $\rep{\cos(x)}{B}$ by solving these. 
      \begin{align*}
        w_1\cdot (2\sin(x)+\cos(x))+z_1\cdot(3\cos(x)) &= \sin(x) \\
        w_2\cdot (2\sin(x)+\cos(x))+z_2\cdot(3\cos(x)) &= \cos(x) 
      \end{align*}
      An easier method is to find the inverse of the matrix found above.
      \begin{equation*}
        \rep{\identity}{D,B}
        =\begin{mat}[r]
          2  &0  \\
          1  &3  
        \end{mat}^{-1}
        =\frac{1}{6}\cdot\begin{mat}[r]
          3  &0  \\
          -1  &2  
        \end{mat}
        =\begin{mat}[r]
           1/2  &0  \\
          -1/6  &1/3
         \end{mat}
      \end{equation*}
    \end{answer}
  \item 
    Where does this matrix
    \begin{equation*}
        \begin{mat}
           \cos(2\theta)   &\sin(2\theta)   \\
           \sin(2\theta)   &-\cos(2\theta)
        \end{mat}
    \end{equation*}
    send the standard basis for \( \Re^2 \)?
    Any other bases?
    \textit{Hint.}
    Consider the inverse.
    \begin{answer} 
      We start by taking
      the inverse of the matrix, that is, by deciding what is the inverse to
      the map of interest.
      \begin{equation*}
        \rep{\identity}{D,\stdbasis_2}
        \rep{\identity}{D,\stdbasis_2}^{-1}
        =\frac{1}{-\cos^2(2\theta)-\sin^2(2\theta)}\cdot\begin{mat}
           -\cos(2\theta)   &-\sin(2\theta)  \\
           -\sin(2\theta)   &\cos(2\theta)
        \end{mat}
        =\begin{mat}
           \cos(2\theta)   &\sin(2\theta)  \\
           \sin(2\theta)   &-\cos(2\theta)
        \end{mat}
      \end{equation*}
      This is more tractable than the representation the other way
      because this matrix is the concatenation of these two column vectors
      \begin{equation*}
        \rep{\vec{\delta}_1}{\stdbasis_2}
           =\colvec{\cos(2\theta) \\ \sin(2\theta)}
        \qquad 
        \rep{\vec{\delta}_2}{\stdbasis_2}
           =\colvec{\sin(2\theta) \\ -\cos(2\theta)}
      \end{equation*}
      and representations with respect to $\stdbasis_2$ are transparent.
      \begin{equation*}
        \vec{\delta}_1=\colvec{\cos(2\theta) \\ \sin(2\theta)}
        \qquad 
        \vec{\delta}_2=\colvec{\sin(2\theta) \\ -\cos(2\theta)}
      \end{equation*}
      This pictures the action of the map that transforms $D$ to $\stdbasis_2$
      (it is, again, the inverse of the map that is the answer to
      this question).
      The line lies at an angle $\theta$ to the $x$~axis.
      \begin{center}  \small
        \includegraphics{ch3.81}
      \end{center}
      This map reflects vectors over that line.
      Since reflections are self-inverse, the answer to the question is:~the
      original map reflects about the line
      through the origin with angle of elevation $\theta$. 
      (Of course, it does this to any basis.)
    \end{answer}
  \recommended \item
    What is the change of basis matrix with respect to \( B,B \)?
    \begin{answer}
       The appropriately-sized identity matrix.  
     \end{answer}
  \item 
    Prove that a matrix changes bases if and only if it is invertible.
    \begin{answer}
       Each is true if and only if the matrix is nonsingular.
    \end{answer}
  \item 
    Finish the proof of \nearbylemma{le:NonSingIsChBasis}.
    \begin{answer}
      What remains is to show that 
      left multiplication by a reduction matrix represents a
      change from another basis to \( B=\basis{\beta}{n} \).

      Application of a row-multiplication matrix \( M_i(k) \) translates a
      representation with respect to the basis
      \( \sequence{\vec{\beta}_1,\dots,k\vec{\beta}_i,\dots,\vec{\beta}_n} \)
      to one with respect to \( B \), as here.
      \begin{equation*}
         \vec{v}=c_1\cdot\vec{\beta}_1+\dots+c_i\cdot(k\vec{\beta}_i)
                   +\dots+c_n\cdot\vec{\beta}_n  
         \;\mapsto\;                                                       
         c_1\cdot\vec{\beta}_1+\dots
             +(kc_i)\cdot\vec{\beta}_i+\dots+c_n\cdot\vec{\beta}_n=\vec{v}
      \end{equation*}
      Applying a row-swap matrix \( P_{i,j} \) translates a representation
      with respect to
      \( \sequence{\vec{\beta}_1,\dots,\vec{\beta}_j,\dots,
        \vec{\beta}_i,\dots,\vec{\beta}_n} \)
      to one with respect to
            \( \sequence{\vec{\beta}_1,\dots,\vec{\beta}_i,\dots,
        \vec{\beta}_j,\dots,\vec{\beta}_n} \).
      Finally, applying a row-combination matrix \( C_{i,j}(k) \) changes a
      representation with respect to
      \( \sequence{\vec{\beta}_1,\dots,\vec{\beta}_i+k\vec{\beta}_j,\dots,
        \vec{\beta}_j,\dots,\vec{\beta}_n} \)
      to one with respect to \( B \).  
      \begin{multline*}
         \vec{v}= c_1\cdot\vec{\beta}_1+\dots
                     +c_i\cdot(\vec{\beta}_i+k\vec{\beta}_j)
                       +\dots+c_j\vec{\beta}_j+\dots+c_n\cdot\vec{\beta}_n  
         \\  \mapsto\;                                                       
         c_1\cdot\vec{\beta}_1+\dots+c_i\cdot\vec{\beta}_i
               +\dots+(kc_i+c_j)\cdot\vec{\beta}_j
               +\dots+c_n\cdot\vec{\beta}_n=\vec{v}     
      \end{multline*}
      (As in the part of the proof in the body of this subsection, 
      the various conditions on the
      row operations, e.g., that the scalar $k$ is nonzero, assure that these
      are all bases.)
    \end{answer}
  \recommended \item 
    Let \( H \) be a \( \nbyn{n} \) nonsingular matrix.
    What basis of \( \Re^n \) does \( H \) change to the standard basis?
    \begin{answer}
       Taking $H$ as a change of basis matrix 
       $H=\rep{\identity}{B,\stdbasis_n}$, 
       its columns are
       \begin{equation*}
         \colvec{h_{1,i} \\ \vdots \\ h_{n,i}}
         =\rep{\identity(\vec{\beta}_i)}{\stdbasis_n}
         =\rep{\vec{\beta}_i}{\stdbasis_n}
       \end{equation*}
       and, because representations with respect to the standard basis
       are transparent, we have this.
       \begin{equation*}
         \colvec{h_{1,i} \\ \vdots \\ h_{n,i}}
         =\vec{\beta}_i
       \end{equation*}
       That is, the basis is the one composed of the columns of \( H \).  
    \end{answer}
  \recommended \item \label{exer:AnyNonZeroRepChgTOAnyOther}
    \begin{exparts}
      \partsitem  In \( \polyspace_3 \) with basis
        \( B=\sequence{1+x,1-x,x^2+x^3,x^2-x^3} \) we have this
        representation.
        \begin{equation*}
          \rep{1-x+3x^2-x^3}{B}=
            \colvec[r]{0 \\ 1 \\ 1 \\ 2}_B
        \end{equation*}
        Find a basis \( D \) 
        giving this different representation for the same
        polynomial. 
        \begin{equation*}
          \rep{1-x+3x^2-x^3}{D}=
            \colvec[r]{1 \\ 0 \\ 2 \\ 0}_D
        \end{equation*}
      \partsitem State and prove that we can change any nonzero vector
        representation to any other.
    \end{exparts}
    \noindent\textit{Hint.}
    The proof of \nearbylemma{le:NonSingIsChBasis}
    is constructive\Dash it not only says the bases change, it shows
    how they change.
    \begin{answer}
      \begin{exparts}
        \partsitem We can change the starting vector representation
          to the ending one through a sequence of row operations.
          The proof tells us what how the bases change. 
          We start by swapping the first and second rows
          of the representation with respect to $B$ to get a representation
          with respect to a new basis $B_1$.
          \begin{equation*}
            \rep{1-x+3x^2-x^3}{B_1}=
              \colvec[r]{1 \\ 0 \\ 1 \\ 2}_{B_1}
            \qquad
            B_1=\sequence{1-x,1+x,x^2+x^3,x^2-x^3}
          \end{equation*}
          We next add \( -2 \) times the third row of the vector
          representation to the fourth row.
          \begin{equation*}
            \rep{1-x+3x^2-x^3}{B_3}=
              \colvec[r]{1 \\ 0 \\ 1 \\ 0}_{B_2}
            \qquad
            B_2=\sequence{1-x,1+x,3x^2-x^3,x^2-x^3}
          \end{equation*}
          (The third element of \( B_2 \) is the third element of 
          \( B_1 \) minus \( -2 \) times the fourth element of $B_1$.)
          Now we can finish by doubling the third row.
          \begin{equation*}
            \rep{1-x+3x^2-x^3}{D}=
              \colvec[r]{1 \\ 0 \\ 2 \\ 0}_{D}
            \qquad
            D=\sequence{1-x,1+x,(3x^2-x^3)/2,x^2-x^3}
          \end{equation*}
        \partsitem 
          Here are three different approaches to stating such a result.
          The first is the assertion:~where $V$ is a vector space with
          basis $B$ and $\vec{v}\in V$ is nonzero, for any nonzero column
          vector $\vec{z}$ 
          (whose number of components equals the dimension of $V$) 
          there is a change of basis matrix $M$ such that
          $M\cdot \rep{\vec{v}}{B}=\vec{z}$. 
          The second possible statement:~for any ($n$-dimensional)
          vector space $V$ and any nonzero
          vector \( \vec{v}\in V \), where \( \vec{z}_1, \vec{z}_2\in\Re^n \)
          are nonzero, there are bases \( B, D\subset V \) such that
          \( \rep{\vec{v}}{B}=\vec{z}_1 \) and 
          \( \rep{\vec{v}}{D}=\vec{z}_2 \).
          The third is:~for any nonzero $\vec{v}$ member of 
          any vector space (of dimension~$n$) and any nonzero column vector
          (with $n$ components) there is a basis such that $\vec{v}$ is 
          represented with respect to that basis by that column vector.

          The first and second statements follow easily from the third.
          The first follows because the third statement gives a basis $D$
          such that $\rep{\vec{v}}{D}=\vec{z}$ and then 
          $\rep{\identity}{B,D}$ is the desired~$M$.
          The second follows from the third because it is just a
          doubled application of it.

          A way to prove the third is as in the answer to the first part
          of this question.
          Here is a sketch.
          Represent $\vec{v}$ with respect to any basis $B$ with a column
          vector $\vec{z}_1$.
          This column vector must have a nonzero component because $\vec{v}$
          is a nonzero vector.
          Use that component in a sequence of row operations to convert 
          $\vec{z}_1$ to $\vec{z}$.
          (We could fill out this sketch as an induction 
          argument on the dimension of $V$.)  
       \end{exparts}  
     \end{answer}
  \item 
    Let \( V,W \) be vector spaces, and let \( B,\hat{B} \) be bases for
    \( V \) and \( D,\hat{D} \) be bases for \( W \).
    Where \( \map{h}{V}{W} \) is linear, find a formula relating
    \( \rep{h}{B,D} \) to \( \rep{h}{\hat{B},\hat{D}} \).
    \begin{answer}
      This is the topic of the next subsection.
    \end{answer}
  \recommended \item
    Show that the columns of an \( \nbyn{n} \) change of basis matrix
    form a basis for \( \Re^n \).
    Do all bases appear in that way:~can 
    the vectors from any $\Re^n$ basis make the columns of a change of 
    basis matrix?
    \begin{answer}
      A change of basis matrix is nonsingular and thus
      has rank equal to the number of its columns.
      Therefore its set of columns is a linearly independent subset of size 
      $n$ in $\Re^n$ and it is thus a basis.
      The answer to the second half is also `yes'; all implications in the 
      prior sentence reverse
      (that is, all of the `if \ldots then~\ldots' parts of the prior sentence
      convert to `if and only if' parts).
    \end{answer}
  \recommended \item 
    Find a matrix having this effect.
    \begin{equation*}
      \colvec[r]{1 \\ 3}
      \;\mapsto\;
      \colvec[r]{4 \\ -1}
    \end{equation*}
    That is, find a $M$ that left-multiplies the 
    starting vector to yield the ending vector.
    Is there a matrix having these two effects?
    \begin{exparts*}
      \partsitem
        $
          \colvec[r]{1 \\ 3}\mapsto\colvec[r]{1 \\ 1}
          \quad
          \colvec[r]{2 \\ -1}\mapsto\colvec[r]{-1 \\ -1}
        $
      \partsitem $
          \colvec[r]{1 \\ 3}\mapsto\colvec[r]{1 \\ 1}
          \quad
          \colvec[r]{2 \\ 6}\mapsto\colvec[r]{-1 \\ -1}
        $
   \end{exparts*}
   Give a necessary and sufficient condition for there to be a
   matrix such that
   $\vec{v}_1\mapsto\vec{w}_1$ and $\vec{v}_2\mapsto\vec{w}_2$.
    \begin{answer}
      In response to the first half of the question, 
      there are infinitely many such matrices. 
      One of them  represents with respect to
      \( \stdbasis_2 \) the transformation of \( \Re^2 \) with this action.
      \begin{equation*}
        \colvec[r]{1 \\ 0}\mapsto\colvec[r]{4 \\ 0}
        \qquad
        \colvec[r]{0 \\ 1}\mapsto\colvec[r]{0 \\ -1/3}
      \end{equation*}  
     The problem of specifying two distinct input/output pairs is a bit 
     trickier.
     The fact that matrices have a linear action precludes some possibilities.
     \begin{exparts}
       \partsitem Yes, there is such a matrix.
         These conditions
         \begin{equation*}
           \begin{mat}
             a  &b  \\
             c  &d
           \end{mat}
           \colvec[r]{1 \\ 3}
           =
           \colvec[r]{1 \\ 1}
           \qquad
           \begin{mat}
             a  &b  \\
             c  &d
           \end{mat}
           \colvec[r]{2 \\ -1}
           =
           \colvec[r]{-1 \\ -1}
         \end{equation*}
         can be solved
         \begin{equation*}
           \begin{linsys}{4}
             a  &+  &3b  &   &   &   &   &=  &1  \\
                &   &    &   &c  &+  &3d &=  &1  \\
            2a  &-  &b   &   &   &   &   &=  &-1 \\
                &   &    &   &2c &-  &d  &=  &-1 
           \end{linsys}
         \end{equation*}
         to give this matrix.
         \begin{equation*}
           \begin{mat}[r]
             -2/7  &3/7 \\
             -2/7  &3/7 
           \end{mat}
         \end{equation*}
       \partsitem No, because
         \begin{equation*}
           2\cdot\colvec[r]{1 \\ 3}=\colvec[r]{2 \\ 6}
           \quad\text{but}\quad
           2\cdot\colvec[r]{1 \\ 1}\neq\colvec[r]{-1 \\ -1}
         \end{equation*}
         no linear action can produce this effect.
       \partsitem A sufficient condition is that 
         \( \set{\vec{v}_1,\vec{v}_2} \) be linearly independent, but
         that's not a necessary condition.
         A necessary and sufficient condition is that any linear dependences
         among the starting vectors appear also among the ending vectors.
         That is,
         \begin{equation*}
           c_1\vec{v}_1+c_2\vec{v}_2=\zero
           \quad\text{implies}\quad
           c_1\vec{w}_1+c_2\vec{w}_2=\zero.
         \end{equation*}
         The proof of this condition is routine.
     \end{exparts} 
   \end{answer}
\end{exercises}

















\subsection{Changing Map Representations}
\index{matrix equivalence|(}
The first subsection shows how to convert the representation
of a vector with respect to one basis to the representation of that
same vector with respect to another basis.
Here we will see how to
convert the representation of a map  with
respect to one pair of bases  to the representation of that
map with respect to a different pair, how to change
$\rep{h}{B,D}$ to $\rep{h}{\hat{B},\hat{D}}$.

That is, we want the relationship between the matrices in this 
arrow diagram.\index{arrow diagram}
%<*ChangeRepresentationOfMapArrowDiagram>
\begin{equation*}
  \begin{CD}
    V_{\wrt{B}}                   @>h>H>        W_{\wrt{D}}       \\
    @V{\text{\scriptsize$\identity$}} VV                @V{\text{\scriptsize$\identity$}} VV \\
    V_{\wrt{\hat{B}}}             @>h>\hat{H}>  W_{\wrt{\hat{D}}}
  \end{CD}
\end{equation*}
To move from the lower-left of this diagram 
to the lower-right we can either go straight over, or
else up to $V_B$ then over to $W_D$ and then down.
So 
we can calculate $\hat{H}=\rep{h}{\hat{B},\hat{D}}$ 
either by simply using $\hat{B}$ and $\hat{D}$,
or else by first changing bases with $\rep{\identity}{\hat{B},B}$ 
then multiplying by \( H=\rep{h}{B,D} \)
and then changing bases with $\rep{\identity}{D,\hat{D}}$.

This equation summarizes.
\begin{equation*}
   \hat{H}=
   \rep{\identity}{D,\hat{D}}\cdot H\cdot \rep{\identity}{\hat{B},B}
\tag*{\text{($*$})}\end{equation*}
%</ChangeRepresentationOfMapArrowDiagram>
(To compare this equation with the sentence before it, remember to read the
equation from right to left because we read
function composition from right to left and matrix multiplication 
represents composition.)

\begin{example}
The matrix
\begin{equation*}
  T=\begin{mat}[r]
     \cos(\pi/6)  &-\sin(\pi/6)  \\
     \sin(\pi/6)  &\cos(\pi/6)
  \end{mat}
  =
  \begin{mat}[r]
     \sqrt{3}/2  &-1/2  \\
     1/2         &\sqrt{3}/2
  \end{mat}
\end{equation*}
represents, with respect to \( \stdbasis_2,\stdbasis_2 \),
the transformation \( \map{t}{\Re^2}{\Re^2} \) that rotates vectors
\( \pi/6 \) radians counterclockwise.
\begin{center}
  \includegraphics{ch3.22}
\end{center}
We can translate that to a representation 
with respect to
\begin{equation*}
  \hat{B}=\sequence{
              \colvec[r]{1 \\ 1}
              \colvec[r]{0 \\ 2} }
  \qquad
  \hat{D}=\sequence{
              \colvec[r]{-1 \\ 0}
              \colvec[r]{2 \\ 3} }
\end{equation*}
by using the arrow diagram and formula~($*$) above.
\begin{equation*}
  \begin{CD}
    \Re^2_{\wrt{\stdbasis_2}}              @>t>T>        \Re^2_{\wrt{\stdbasis_2}}     \\
    @V\text{\scriptsize$\identity$} VV                @V\text{\scriptsize$\identity$} VV \\
    \Re^2_{\wrt{\hat{B}}}          @>t>\hat{T}>  \Re^2_{\wrt{\hat{D}}}
  \end{CD}
  \qquad
   \hat{T}=
   \rep{\identity}{\stdbasis_2,\hat{D}}\cdot T\cdot \rep{\identity}{\hat{B},\stdbasis_2}
\end{equation*}
Note that $\rep{\identity}{\stdbasis_2,\hat{D}}$ is
the matrix inverse of $\rep{\identity}{\hat{D},\stdbasis_2}$.
\begin{align*}
   \rep{t}{\hat{B},\hat{D}}
  &=\begin{mat}[r]
     -1     &2   \\
     0      &3
  \end{mat}^{-1}
  \begin{mat}[r]
     \sqrt{3}/2  &-1/2  \\
     1/2         &\sqrt{3}/2
  \end{mat}
  \begin{mat}[r]
     1      &0   \\
     1      &2
  \end{mat}                              \\                            
  &=\begin{mat}[r]
     (5-\sqrt{3})/6   &(3+2\sqrt{3})/3 \\
     (1+\sqrt{3})/6  &\sqrt{3}/3
  \end{mat}
\end{align*}
Although the new matrix is messier,
the map that it represents is the same. 
For instance, to replicate the effect of $t$ in the picture, 
start with $\hat{B}$,
\begin{equation*}
  \rep{\colvec[r]{1 \\ 3}}{\hat{B}}=\colvec[r]{1 \\ 1}_{\hat{B}}
\end{equation*}
apply $\hat{T}$,
\begin{equation*}
  \begin{mat}[r]
     (5-\sqrt{3})/6   &(3+2\sqrt{3})/3 \\
     (1+\sqrt{3})/6  &\sqrt{3}/3
  \end{mat}_{\hat{B},\hat{D}}
  \colvec[r]{1 \\ 1}_{\hat{B}}
  =
  \colvec[r]{(11+3\sqrt{3})/6 \\ (1+3\sqrt{3})/6}_{\hat{D}}
\end{equation*}
and check it against $\hat{D}$
\begin{equation*}
  \frac{11+3\sqrt{3}}{6}\cdot\colvec[r]{-1 \\ 0}
  +\frac{1+3\sqrt{3}}{6}\cdot\colvec[r]{2 \\ 3}
  =\colvec[r]{(-3+\sqrt{3})/2 \\ (1+3\sqrt{3})/2}
\end{equation*}
and it gives the same outcome as above.
\end{example}

\begin{example} \label{ex:DiagizedMat}
We may make the matrix simpler by changing bases.
On \( \Re^3 \) the map
\begin{equation*}
  \colvec{x \\ y \\ z}\mapsunder{t}\colvec{y+z \\ x+z \\ x+y}
\end{equation*}
is represented with respect to the standard basis in this way.
\begin{equation*}
  \rep{t}{\stdbasis_3,\stdbasis_3}=
  \begin{mat}[r]
    0  &1  &1  \\
    1  &0  &1  \\
    1  &1  &0
  \end{mat}
\end{equation*}
Represented with respect to
\begin{equation*}
  B=\sequence{\colvec[r]{1 \\ -1 \\ 0},
                             \colvec[r]{1 \\ 1 \\ -2},
                             \colvec[r]{1 \\ 1 \\ 1}}
\end{equation*} 
gives a matrix 
that is diagonal.
\begin{equation*} 
  \rep{t}{B,B}=
  \begin{mat}[r]
   -1  &0  &0  \\
    0  &-1 &0  \\
    0  &0  &2
  \end{mat}
\end{equation*}
\end{example}

Naturally we usually prefer basis changes that make the
representation easier to understand.
We say that a map or matrix
has been \definend{diagonalized}\index{matrix!diagonalized}
when its representation is diagonal with respect to $B,B$, that is,
with respect to equal starting
and ending bases. 
In Chapter Five we shall see which maps and matrices are diagonalizable.
In the rest of this subsection we consider the easier case 
where representations are with respect to $B,D$, which are  
possibly different starting and ending bases.
Recall that the prior subsection 
shows that a matrix changes bases if and only if it is nonsingular.
That gives us another version of the above  arrow diagram
and equation~($*$) from the start of this subsection.

\begin{definition}  \label{def:MatEquiv}
%<*df:MatEquiv>
Same-sized matrices \( H \) and \( \hat{H} \) are
\definend{matrix equivalent\/}\index{matrix equivalence!definition}%
\index{matrix!equivalent}\index{equivalence relation!matrix equivalence}
if there are nonsingular matrices \( P \) and \( Q \) such that
$\hat{H}=PHQ$.
%</df:MatEquiv>
\end{definition}

\begin{corollary} \label{le:MatEqIsSameMap}
%<*co:MatEqIsSameMap>
Matrix equivalent matrices represent the same map, with respect to appropriate
pairs of bases.
%</co:MatEqIsSameMap>
\end{corollary}

\nearbyexercise{exer:MatEqIsEqRel} checks that
matrix equivalence is an equivalence relation.
Thus it  partitions\index{partition!matrix equivalence classes} 
the set of matrices into matrix equivalence classes.
\begin{center}
  \raisebox{.4in}{\begin{tabular}{l}
                    All matrices:
                  \end{tabular}}
  \includegraphics{ch3.23}
  \raisebox{.4in}{\begin{tabular}{l}
                     $H$ matrix equivalent \\ to $\hat{H}$
                  \end{tabular}}
\end{center}
We can get some insight into the classes by comparing matrix equivalence
with row equivalence
(remember that matrices are row equivalent when they can be reduced to each
other by row operations).
In $\hat{H}=PHQ$, the matrices $P$ and $Q$ are nonsingular and 
thus we can write each as a product of elementary reduction matrices
(\nearbylemma{lem:ComputeInvMat}).
Left-multiplication by the reduction matrices making up $P$
has the effect of performing row operations.
Right-multiplication by the reduction matrices making up $Q$
performs column operations.
Therefore, matrix equivalence is a generalization of row equivalence\Dash two
matrices are row equivalent if one can be converted to the other by
a sequence of row reduction steps, while
two matrices are matrix equivalent if one can be converted to the other by a 
sequence of row reduction steps followed by a sequence of column reduction
steps. 

Thus, if matrices are row equivalent then they are also
matrix equivalent (since we can take $Q$ to be the identity matrix and so 
perform no column operations).
The converse, however, does not hold:
two matrices can be matrix equivalent but not row equivalent.

\begin{example}
These two
\begin{equation*} 
  \begin{mat}[r]
    1  &0  \\
    0  &0
  \end{mat}
  \qquad
  \begin{mat}[r]
    1  &1  \\
    0  &0
  \end{mat}
\end{equation*}
are matrix equivalent because the second reduces to the first by
the column operation of taking $-1$ times the first column and adding
to the second.
They are not row equivalent because they have different reduced echelon
forms (in fact, both are already in reduced form).
\end{example}

We will close this section by finding 
a set of representatives
for the matrix equivalence classes.\appendrefs{class representatives}%
\index{representative!of matrix equivalence classes}

\begin{theorem}  \label{th:CanonFormForMatEquiv}
\index{matrix equivalence!canonical form}
\index{canonical form!for matrix equivalence}
%<*th:CanonFormForMatEquiv>
Any \( \nbym{m}{n} \) matrix of rank \( k \) is matrix equivalent to
the \( \nbym{m}{n} \) matrix that is all zeros except that
the first \( k \) diagonal entries are ones.
\begin{equation*}
    \begin{mat}
      1  &0      &\ldots &0  &0  &\ldots  &0  \\
      0  &1      &\ldots &0  &0  &\ldots  &0  \\
         &\vdots                              \\
      0  &0      &\ldots &1  &0  &\ldots  &0  \\
      0  &0      &\ldots &0  &0  &\ldots  &0  \\
         &\vdots                              \\
      0  &0      &\ldots &0  &0  &\ldots  &0
    \end{mat}
\end{equation*}
%</th:CanonFormForMatEquiv>
\end{theorem}

%<*BlockPartialIdentityForm>
\noindent This is a 
\definend{block partial-identity}\index{matrix!block} 
form.
\begin{equation*}
    \begin{pmat}{c|c}
      I  &Z  \\  \hline
      Z  &Z
    \end{pmat}
\end{equation*}
%</BlockPartialIdentityForm>

\begin{proof}
%<*pf:CanonFormForMatEquiv>
As discussed above, Gauss-Jordan reduce the given matrix
and combine all the reduction matrices used there to make \( P \).
Then use the leading entries to do column reduction and
finish by swapping columns to put the leading ones on the diagonal.
Combine the reduction matrices used for those column operations into
\( Q \). 
%</pf:CanonFormForMatEquiv>
\end{proof}

\begin{example}
We illustrate the proof by finding the $P$ and $Q$ for this matrix. 
\begin{equation*}
    \begin{mat}[r]
       1  &2  &1  &-1  \\
       0  &0  &1  &-1  \\
       2  &4  &2  &-2
    \end{mat}
\end{equation*}
First Gauss-Jordan row-reduce.
\begin{equation*}
    \begin{mat}[r]
       1  &-1 &0    \\
       0  &1  &0    \\
       0  &0  &1
    \end{mat}
    \begin{mat}[r]
       1  &0  &0    \\
       0  &1  &0    \\
       -2 &0  &1
    \end{mat}
    \begin{mat}[r]
       1  &2  &1  &-1  \\
       0  &0  &1  &-1  \\
       2  &4  &2  &-2
    \end{mat}
  =
    \begin{mat}[r]
       1  &2  &0  &0   \\
       0  &0  &1  &-1  \\
       0  &0  &0  &0
    \end{mat}
\end{equation*}
Then column-reduce, which involves right-multiplication.
\begin{equation*}
    \begin{mat}[r]
       1  &2  &0  &0   \\
       0  &0  &1  &-1  \\
       0  &0  &0  &0
    \end{mat}
    \begin{mat}[r]
       1  &-2 &0  &0   \\
       0  &1  &0  &0   \\
       0  &0  &1  &0   \\
       0  &0  &0  &1
    \end{mat}
    \begin{mat}[r]
       1  &0  &0  &0   \\
       0  &1  &0  &0   \\
       0  &0  &1  &1   \\
       0  &0  &0  &1
    \end{mat}
  =
    \begin{mat}[r]
       1  &0  &0  &0   \\
       0  &0  &1  &0  \\
       0  &0  &0  &0
    \end{mat}
\end{equation*}
Finish by swapping columns.
\begin{equation*}
    \begin{mat}[r]
       1  &0  &0  &0   \\
       0  &0  &1  &0  \\
       0  &0  &0  &0
    \end{mat}
    \begin{mat}[r]
       1  &0  &0  &0   \\
       0  &0  &1  &0   \\
       0  &1  &0  &0   \\
       0  &0  &0  &1
    \end{mat}
  =
    \begin{mat}[r]
       1  &0  &0  &0   \\
       0  &1  &0  &0  \\
       0  &0  &0  &0
    \end{mat}
\end{equation*}
Finally, combine the left-multipliers together as $P$ and the
right-multipliers together as $Q$ to get the $PHQ$ equation.
\begin{equation*}
    \begin{mat}[r]
       1  &-1 &0    \\
       0  &1  &0    \\
       -2 &0  &1
    \end{mat}
    \begin{mat}[r]
       1  &2  &1  &-1  \\
       0  &0  &1  &-1  \\
       2  &4  &2  &-2
    \end{mat}
    \begin{mat}[r]
       1  &0  &-2  &0   \\
       0  &0  &1  &0   \\
       0  &1  &0  &1   \\
       0  &0  &0  &1
    \end{mat}
    =
    \begin{mat}[r]
       1  &0  &0  &0   \\
       0  &1  &0  &0  \\
       0  &0  &0  &0
    \end{mat}
\end{equation*}
\end{example}

\begin{corollary}  \label{co:MatrixEquivalentIffSameRank}
%<*co:MatrixEquivalentIffSameRank>
Two same-sized matrices are matrix equivalent if and only if they
have the same rank.
%</co:MatrixEquivalentIffSameRank>
\end{corollary}

That is, the matrix equivalence classes are
characterized\index{characterizes} by rank.

\begin{proof}
%<*pf:MatrixEquivalentIffSameRank>
Two same-sized matrices with the same rank
are equivalent to the same block partial-identity matrix.
%</pf:MatrixEquivalentIffSameRank>
\end{proof}

\begin{example}
%Now that we know that the block partial-identity matrices form
%canonical representatives of the matrix-equivalence classes, we can
%see what the classes look like and how many classes there are.
The $\nbyn{2}$ matrices have
only three possible ranks:~zero, one, or~two.
Thus there are three matrix-equivalence classes.
\index{partition!matrix equivalence classes}
\begin{center} % make an exercise comparing these classes with the ones for row-equivalence?
  \raisebox{.4in}{\begin{tabular}{l}
                    All $\nbyn{2}$ matrices:
                  \end{tabular}}
  \includegraphics{ch3.24}
  \raisebox{.4in}{\begin{tabular}{l}
                    Three equivalence \\ classes
                  \end{tabular}}
\end{center}
Each class consists of all of the $\nbyn{2}$ matrices with the same rank. 
There is only one rank~zero matrix so that class has only
one member. 
The other two classes have infinitely many members.
\end{example}

In this subsection
we have seen how to change the representation of a map with
respect to a first pair of bases to one with respect to a second pair.
That led to a definition describing when matrices are equivalent in
this way.
Finally we noted that,
with the proper choice of (possibly different) starting and ending bases, any
map can be represented in block partial-identity form.

One of the nice things about this representation is that,
in some sense, we can completely understand the map when we express it
in this way:
if the bases are
\( B=\sequence{\vec{\beta}_1,\dots,\vec{\beta}_n} \)
and
\( D=\sequence{\vec{\delta}_1,\dots,\vec{\delta}_m} \)
then the map sends
\begin{equation*}
  c_1\vec{\beta}_1+\dots+c_k\vec{\beta}_k+c_{k+1}\vec{\beta}_{k+1}+\dots
     +c_n\vec{\beta}_n
  \;\longmapsto\;
  c_1\vec{\delta}_1+\dots+c_k\vec{\delta}_k+\zero+\dots+\zero
\end{equation*}
where \( k \) is the map's rank.
Thus, we can understand any linear map as a kind of projection.
\begin{equation*}
  \colvec{c_1 \\ \vdots \\ c_k \\ c_{k+1} \\ \vdots \\ c_n}_B
  \;\mapsto\;
  \colvec{c_1 \\ \vdots \\ c_k \\ 0 \\ \vdots \\ 0}_D
\end{equation*}
Of course, ``understanding'' a map expressed in this way 
requires that we understand the relationship between \( B \) and \( D \).
Nonetheless,
this is a good classification of linear maps.

\begin{exercises}
  \recommended \item 
    Decide if these matrices are matrix equivalent.
    \begin{exparts}
      \partsitem \( \begin{mat}[r]
                 1  &3  &0  \\
                 2  &3  &0
               \end{mat} \),
            \( \begin{mat}[r]
                 2  &2  &1  \\
                 0  &5  &-1
               \end{mat} \)
      \partsitem \( \begin{mat}[r]
                 0  &3  \\
                 1  &1
               \end{mat} \),
            \( \begin{mat}[r]
                 4  &0  \\
                 0  &5
               \end{mat} \)
      \partsitem \( \begin{mat}[r]
                 1  &3  \\
                 2  &6
               \end{mat} \),
            \( \begin{mat}[r]
                 1  &3  \\
                 2  &-6
               \end{mat} \)
    \end{exparts}
    \begin{answer}
      \begin{exparts}
        \partsitem Yes, each has rank two.
        \partsitem Yes, they have the same rank.
        \partsitem No, they have different ranks.
      \end{exparts}  
    \end{answer}
  \recommended \item
    Find the canonical representative of the matrix-equivalence class of
    each matrix.
    \begin{exparts*}
      \partsitem \( \begin{mat}[r]
                 2  &1  &0  \\
                 4  &2  &0
               \end{mat} \)
      \partsitem \( \begin{mat}[r]
                 0  &1  &0  &2  \\
                 1  &1  &0  &4  \\
                 3  &3  &3  &-1
               \end{mat} \)
    \end{exparts*}
    \begin{answer}
      We need only decide what the rank of each is.
      \begin{exparts*}
        \partsitem \( \begin{mat}[r]
                   1  &0  &0  \\
                   0  &0  &0
                 \end{mat} \)
        \partsitem \( \begin{mat}[r]
                   1  &0  &0  &0  \\
                   0  &1  &0  &0  \\
                   0  &0  &1  &0
                 \end{mat} \)
      \end{exparts*}  
    \end{answer}
  \item 
    Suppose that, with respect to
    \begin{equation*}
      B=\stdbasis_2
      \qquad
      D=\sequence{\colvec[r]{1 \\ 1},\colvec[r]{1 \\ -1}}
    \end{equation*}
    the transformation \( \map{t}{\Re^2}{\Re^2} \) is represented by
    this matrix.
    \begin{equation*}
      \begin{mat}[r]
        1  &2  \\
        3  &4
      \end{mat}
    \end{equation*}
    Use change of basis matrices to represent \( t  \) with respect
    to each pair.
    \begin{exparts}
      \partsitem \( \hat{B}=\sequence{\colvec[r]{0 \\ 1},\colvec[r]{1 \\ 1}} \),
        \( \hat{D}=\sequence{\colvec[r]{-1 \\ 0},\colvec[r]{2 \\ 1}} \)
      \partsitem \( \hat{B}=\sequence{\colvec[r]{1 \\ 2},\colvec[r]{1 \\ 0}} \),
        \( \hat{D}=\sequence{\colvec[r]{1 \\ 2},\colvec[r]{2 \\ 1}} \)
    \end{exparts}
    \begin{answer}
      Recall the diagram
      and the formula.
      \begin{equation*}
        \begin{CD}
          \Re^2_{\wrt{B}}                   @>t>T>        \Re^2_{\wrt{D}}       \\
          @V\scriptsize\identity VV             @V\scriptsize\identity VV \\
          \Re^2_{\wrt{\hat{B}}}             @>t>\hat{T}>  \Re^2_{\wrt{\hat{D}}}
        \end{CD}
        \qquad \hat{T}=
         \rep{\identity}{D,\hat{D}}\cdot T\cdot \rep{\identity}{\hat{B},B}
      \end{equation*}
      \begin{exparts}
        \partsitem These two 
          \begin{equation*}
            \colvec[r]{1 \\ 1}=1\cdot\colvec[r]{-1 \\ 0}
                            +1\cdot\colvec[r]{2 \\ 1}
            \qquad
            \colvec[r]{1 \\ -1}=(-3)\cdot\colvec[r]{-1 \\ 0}
                            +(-1)\cdot\colvec[r]{2 \\ 1}
          \end{equation*}
          show that
          \begin{equation*}
            \rep{\identity}{D,\hat{D}}=\begin{mat}[r]
              1  &-3  \\
              1  &-1
            \end{mat}
          \end{equation*}
          and similarly these two
          \begin{equation*}
            \colvec[r]{0 \\ 1}=0\cdot\colvec[r]{1 \\ 0}
                            +1\cdot\colvec[r]{0 \\ 1}
            \qquad
            \colvec[r]{1 \\  1}=1\cdot\colvec[r]{1 \\ 0}
                            +1\cdot\colvec[r]{0 \\ 1}
          \end{equation*}
          give the other nonsingular matrix.
          \begin{equation*}
            \rep{\identity}{\hat{B},B}=\begin{mat}[r]
              0  &1  \\
              1  &1
            \end{mat}
          \end{equation*}
          Then the answer is this.
          \begin{equation*}
            \hat{T}=
            \begin{mat}[r]
              1  &-3  \\
              1  &-1
            \end{mat}
            \begin{mat}[r]
              1  &2  \\
              3  &4
            \end{mat}
            \begin{mat}[r]
              0  &1  \\
              1  &1
            \end{mat}
            =\begin{mat}[r]
              -10  &-18  \\
              -2   &-4 
            \end{mat}
          \end{equation*}
          Although not strictly necessary, a check is reassuring.
          Arbitrarily fixing 
          \begin{equation*}
            \vec{v}=\colvec[r]{3 \\ 2}
          \end{equation*}
          we have that
          \begin{equation*}
            \rep{\vec{v}}{B}=\colvec[r]{3 \\ 2}_B
            \qquad
            \begin{mat}[r]
              1  &2  \\
              3  &4
            \end{mat}_{B,D}
            \colvec[r]{3 \\ 2}_B
            =\colvec[r]{7 \\ 17}_D
          \end{equation*}
          and so $t(\vec{v})$ is this.
          \begin{equation*}
            7\cdot\colvec[r]{1 \\ 1}+17\cdot\colvec[r]{1 \\ -1}=\colvec[r]{24 \\ -10}
          \end{equation*}
          Doing the calculation with respect to $\hat{B},\hat{D}$ starts with
          \begin{equation*}
            \rep{\vec{v}}{\hat{B}}=\colvec[r]{-1 \\ 3}_{\hat{B}}
            \qquad
            \begin{mat}[r]
              -10  &-18  \\
               -2  &-4
            \end{mat}_{\hat{B},\hat{D}}
            \colvec[r]{-1 \\ 3}_{\hat{B}}
            =\colvec[r]{-44 \\ -10}_{\hat{D}}
          \end{equation*}
          and then checks that this is the same result.
          \begin{equation*}
            -44\cdot\colvec[r]{-1 \\ 0}-10\cdot\colvec[r]{2 \\ 1}=\colvec[r]{24 \\ -10}
          \end{equation*}
    \partsitem These two
      \begin{equation*}
        \colvec[r]{1 \\ 1}=\frac{1}{3}\cdot\colvec[r]{1 \\ 2}
                  +\frac{1}{3}\cdot\colvec[r]{2 \\ 1}
        \qquad
        \colvec[r]{1 \\ -1}=-1\cdot\colvec[r]{1 \\ 2}
                  +1\cdot\colvec[r]{2 \\ 1}
      \end{equation*}
      show that
      \begin{equation*}
        \rep{\identity}{D,\hat{D}}=\begin{mat}[r]
          1/3  &-1  \\
          1/3  &1
        \end{mat}
      \end{equation*}
      and these two
      \begin{equation*}
        \colvec[r]{1 \\ 2}=1\cdot\colvec[r]{1 \\ 0}
                  +2\cdot\colvec[r]{0 \\ 1}
        \qquad
        \colvec[r]{1 \\ 0}=-1\cdot\colvec[r]{1 \\ 0}
                  +0\cdot\colvec[r]{0 \\ 1}
      \end{equation*}
      show this.
      \begin{equation*}
        \rep{\identity}{\hat{B},B}=\begin{mat}[r]
          1  &1  \\
          2  &0
        \end{mat}
      \end{equation*}
      With those, the conversion goes in this way.
      \begin{equation*}
        \hat{T}=\begin{mat}[r]
          1/3  &-1  \\
          1/3  &1
        \end{mat}
        \begin{mat}[r]
          1  &2  \\
          3  &4
        \end{mat}
        \begin{mat}[r]
          1  &1  \\
          2  &0
        \end{mat}
        =\begin{mat}[r]
          -28/3  &-8/3  \\
          38/3   &10/3
        \end{mat}
      \end{equation*}
      As in the prior item, a check provides some confidence that we did
      this calculation without mistakes.
      We can for instance, fix the vector
      \begin{equation*}
        \vec{v}=\colvec[r]{-1 \\ 2}
      \end{equation*}
      (this is arbitrary, taken from thin air).
      Now we have
      \begin{equation*}
        \rep{\vec{v}}{B}=\colvec[r]{-1 \\ 2}
        \qquad
        \begin{mat}[r]
          1  &2  \\
          3  &4
        \end{mat}_{B,D}
        \colvec[r]{-1  \\ 2}_{B}
        =\colvec[r]{3  \\ 5}_D
      \end{equation*}
      and so $t(\vec{v})$ is this vector.
      \begin{equation*}
        3\cdot\colvec[r]{1 \\ 1}+5\cdot\colvec[r]{1 \\ -1}=\colvec[r]{8 \\ -2}
      \end{equation*}
      With respect to $\hat{B},\hat{D}$ we first calculate
      \begin{equation*}
        \rep{\vec{v}}{\hat{B}}=\colvec[r]{1 \\ -2}
        \qquad
        \begin{mat}[r]
          -28/3  &-8/3  \\
          38/3   &10/3
        \end{mat}_{\hat{B},\hat{D}}
        \colvec[r]{1 \\ -2}_{\hat{B}}
        =\colvec[r]{-4 \\ 6}_{\hat{D}}
      \end{equation*}
      and, sure enough, that is the same result for $t(\vec{v})$.
      \begin{equation*}
        -4\cdot\colvec[r]{1 \\ 2}+6\cdot\colvec[r]{2 \\ 1}=\colvec[r]{8 \\ -2}
      \end{equation*}
      \end{exparts} 
    \end{answer}
  \recommended \item 
    What sizes are \( P \) and \( Q \) in the equation $\hat{H}=PHQ$?
    \begin{answer}
      Where \( H \) and \( \hat{H} \) are \( \nbym{m}{n} \), the
      matrix \( P \) is \( \nbyn{m} \) while \( Q \) is \( \nbyn{n} \).  
    \end{answer}
  \recommended \item
    Use \nearbytheorem{th:CanonFormForMatEquiv} to show that a square matrix
    is nonsingular if and only if it is equivalent to an identity matrix.
    \begin{answer}
        Any \( \nbyn{n} \) matrix is nonsingular if and only if it has 
        rank \( n \), that is, by  \nearbytheorem{th:CanonFormForMatEquiv},
        if and only if it is matrix equivalent to 
        the $\nbyn{n}$ matrix whose diagonal is all ones.  
    \end{answer}
  \recommended \item
    Show that, where \( A \) is a nonsingular square matrix, if
    \( P \) and \( Q \) are nonsingular square matrices such that \( PAQ=I \)
    then \( QP=A^{-1} \).
    \begin{answer}
      If \( PAQ=I \) then \( QPAQ=Q \), so \( QPA=I \), and so
      \( QP=A^{-1} \).  
    \end{answer}
  \recommended \item
    Why does \nearbytheorem{th:CanonFormForMatEquiv} not show that every matrix
    is diagonalizable (see \nearbyexample{ex:DiagizedMat})?
    \begin{answer}
      By the definition following  \nearbyexample{ex:DiagizedMat}, a matrix
      $M$ is diagonalizable if it represents $M=\rep{t}{B,D}$
      a transformation with the property that there is some basis
      $\hat{B}$ such that $\rep{t}{\hat{B},\hat{B}}$ is a diagonal 
      matrix\Dash the starting and ending bases must be equal. 
      But \nearbytheorem{th:CanonFormForMatEquiv}  says only that there are 
      $\hat{B}$ and $\hat{D}$ such that we can 
      change to a representation $\rep{t}{\hat{B},\hat{D}}$ and get a diagonal
      matrix.
      We have no reason to suspect that we could pick the two
      $\hat{B}$ and $\hat{D}$ so that they are equal.
    \end{answer}
  \item 
    Must matrix equivalent matrices have matrix equivalent transposes?
    \begin{answer}
      Yes.
      Row rank equals column rank, so the rank of the transpose equals
      the rank of the matrix.
      Same-sized matrices with equal ranks are matrix equivalent.  
    \end{answer}
  \item 
    What happens in \nearbytheorem{th:CanonFormForMatEquiv} if \( k=0 \)?
    \begin{answer}
      Only a zero matrix has rank zero.  
    \end{answer}
  \recommended \item \label{exer:MatEqIsEqRel}
    Show that matrix-equivalence is an equivalence relation.
    \begin{answer}
      For reflexivity, to show that any matrix is matrix equivalent to
      itself, 
      take \( P \) and \( Q \) to be identity matrices.
      For symmetry, if \( H_1=PH_2Q \) then
      \( H_2=P^{-1}H_1Q^{-1} \) (inverses exist because $P$ and $Q$ are
      nonsingular).
      Finally, for transitivity, assume that \( H_1=P_2H_2Q_2 \) and 
      that \( H_2=P_3H_3Q_3 \).
      Then substitution gives 
      \( H_1=P_2(P_3H_3Q_3)Q_2=(P_2P_3)H_3(Q_3Q_2) \).
      A product of nonsingular matrices is nonsingular (we've shown that
      the product of invertible matrices is invertible;~in fact, we've shown
      how to calculate the inverse) and so \( H_1 \) is therefore
      matrix equivalent to \( H_3 \).  
     \end{answer}
  \recommended \item 
    Show that a zero matrix is alone in its matrix equivalence
    class.
    Are there other matrices like that?
    \begin{answer}
      By \nearbytheorem{th:CanonFormForMatEquiv}, a zero matrix is alone 
      in its class because it is the only \( \nbym{m}{n} \) of rank zero.
      No other matrix is alone in its class; 
      any nonzero scalar product of a matrix
      has the same rank as that matrix.  
    \end{answer}
  \item 
    What are the matrix equivalence classes of matrices of
    transformations on \( \Re^1 \)?
    \( \Re^3 \)?
    \begin{answer}
      There are two matrix-equivalence classes of \( \nbyn{1} \)
      matrices\Dash those of rank zero and those of rank one.
      The  \( \nbyn{3} \) matrices fall into four matrix equivalence
      classes.
    \end{answer}
  \item 
    How many matrix equivalence classes are there?
    \begin{answer}
      For \( \nbym{m}{n} \) matrices there are classes for each possible
      rank: where \( k \)  is the minimum of \( m \) and \( n \) there are
      classes for the matrices of rank \( 0 \), \( 1 \), \ldots, \( k \).
      That's \( k+1 \) classes.  
      (Of course, totaling over all sizes of matrices we get infinitely
      many classes.)
    \end{answer}
  \item 
    Are matrix equivalence classes closed under scalar
    multiplication?
    Addition?
    \begin{answer}
      They are closed under nonzero scalar multiplication, since
      a nonzero scalar multiple of a matrix has the same rank as does the
      matrix.
      They are not closed under addition,
      for instance, \( H+(-H) \) has rank zero.  
    \end{answer}
  \item 
    Let \( \map{t}{\Re^n}{\Re^n} \) represented by
    \( T \) with respect to \( \stdbasis_n,\stdbasis_n \).
    \begin{exparts}
      \partsitem Find $\rep{t}{B,B}$ in this specific case.
        \begin{equation*}
          T=\begin{mat}[r]
            1  &1  \\
            3  &-1
          \end{mat}
          \qquad
          B=\sequence{\colvec[r]{1  \\ 2},
                      \colvec[r]{-1 \\ -1}}
        \end{equation*}
      \partsitem Describe $\rep{t}{B,B}$ in the general case where
        \( B=\basis{\beta}{n} \).
    \end{exparts}
    \begin{answer}
      \begin{exparts}
        \partsitem We have
          \begin{equation*}
            \rep{\identity}{B,\stdbasis_2}=
            \begin{mat}[r]
              1  &-1  \\
              2  &-1
            \end{mat}
            \qquad
            \rep{\identity}{\stdbasis_2,B}=\rep{\identity}{B,\stdbasis_2}^{-1}=
            \begin{mat}[r]
              1  &-1  \\
              2  &-1
            \end{mat}^{-1}
            =\begin{mat}[r]
              -1  &1  \\
              -2  &1 
            \end{mat}
          \end{equation*}
          and thus the answer is this.
          \begin{equation*}
            \rep{t}{B,B}
            =
            \begin{mat}[r]
              1  &-1  \\
              2  &-1
            \end{mat}
            \begin{mat}[r]
              1  &1  \\
              3  &-1   
            \end{mat}
            \begin{mat}[r]
              -1  &1  \\
              -2  &1 
            \end{mat}
            =\begin{mat}[r]
              -2  &0   \\
              -5  &2 
            \end{mat}
          \end{equation*}
          As a quick check, we can take a vector at random
          \begin{equation*}
            \vec{v}=\colvec[r]{4  \\  5}
          \end{equation*}
          giving
          \begin{equation*}
            \rep{\vec{v}}{\stdbasis_2}=\colvec[r]{4 \\ 5}
            \qquad
            \begin{mat}[r]
              1  &1  \\
              3  &-1   
            \end{mat}
            \colvec[r]{4 \\ 5}
            =\colvec[r]{9 \\ 7}=t(\vec{v})
          \end{equation*}
          while the calculation with respect to $B,B$ 
          \begin{equation*}
            \rep{\vec{v}}{B}=\colvec[r]{1 \\ -3}
            \qquad
            \begin{mat}[r]
              -2  &0   \\
              -5  &2 
            \end{mat}_{B,B}
            \colvec[r]{1 \\ -3}_B
            =\colvec[r]{-2 \\ -11}_B
          \end{equation*}
          yields the same result.
          \begin{equation*}
            -2\cdot\colvec[r]{1 \\ 2}-11\cdot\colvec[r]{-1 \\ -1}
               =\colvec[r]{9 \\ 7}
          \end{equation*}
       \partsitem We have
          \begin{equation*}
            \begin{CD}
              \Re^2_{\wrt{\stdbasis_2}}         @>t>T>     \Re^2_{\wrt{\stdbasis_2}}    \\
              @V\scriptsize\identity VV         @V\scriptsize\identity VV \\
              \Re^2_{\wrt{B}}        @>t>\hat{T}>  \Re^2_{\wrt{B}}
            \end{CD}
        \qquad 
            \rep{t}{B,B}=\rep{\identity}{\stdbasis_2,B}
                          \cdot T
                          \cdot \rep{\identity}{B,\stdbasis_2}
          \end{equation*}
          and, as in the first item of this question
          \begin{equation*}
            \rep{\identity}{B,\stdbasis_2}
            =\begin{pmat}{c|c|c}
              \vec{\beta}_1 &\;\cdots\; &\vec{\beta}_n
            \end{pmat}
            \qquad
            \rep{\identity}{\stdbasis_2,B}=\rep{\identity}{B,\stdbasis_2}^{-1}
          \end{equation*}
          so, writing $Q$ for the matrix whose columns are the basis vectors,
          we have that $\rep{t}{B,B}=Q^{-1}TQ$.
      \end{exparts}
    \end{answer}
  \item 
    \begin{exparts}
       \partsitem Let \( V \) have bases \( B_1 \) and \( B_2 \) and 
         suppose that \( W \) has the basis \( D \).
         Where \( \map{h}{V}{W} \), find the formula that computes
         \( \rep{h}{B_2,D} \) from \( \rep{h}{B_1,D} \).
       \partsitem Repeat the prior question with one basis 
         for \( V \) and two bases for \( W \).
     \end{exparts}
     \begin{answer}
       \begin{exparts}
         \partsitem The adapted form of the arrow diagram is this.
           \begin{equation*}
             \begin{CD}
               V_{\wrt{B_1}}                   @>h>H>        W_{\wrt{D}}       \\
               @V\scriptsize\identity VQV      @V\scriptsize\identity VPV \\
               V_{\wrt{B_2}}             @>h>\hat{H}>  W_{\wrt{D}}
             \end{CD}
           \end{equation*}
           Since there is no need to change bases in 
           \( W \) (or we can
           say that the change of basis matrix $P$ is the identity), we have
           \( \rep{h}{B_2,D}=\rep{h}{B_1,D}\cdot Q \) where
           \( Q=\rep{\identity}{B_2,B_1} \).
         \partsitem Here, this is the arrow diagram. 
           \begin{equation*}
             \begin{CD}
               V_{\wrt{B}}                   @>h>H>   W_{\wrt{D_1}}       \\
               @V\scriptsize\identity VQV      @V\scriptsize\identity VPV \\
               V_{\wrt{B}}             @>h>\hat{H}>  W_{\wrt{D_2}}
             \end{CD}
           \end{equation*}
           We have that \( \rep{h}{B,D_2}=P\cdot \rep{h}{B,D_1} \) where
           \( P=\rep{\identity}{D_1,D_2} \).
       \end{exparts}  
      \end{answer}
  \item 
    \begin{exparts}
      \partsitem If two matrices are matrix-equivalent and invertible,
        must their
        inverses be matrix-equivalent?
      \partsitem If two matrices have matrix-equivalent inverses, must the two
        be matrix-equivalent?
      \partsitem If two matrices are square and matrix-equivalent, must their
        squares be matrix-equivalent?
      \partsitem If two matrices are square and have matrix-equivalent squares,
        must they be matrix-equivalent?
    \end{exparts}
    \begin{answer}
      \begin{exparts}
        \partsitem Here is the arrow diagram, and a version of that diagram
          for inverse functions.
          \begin{equation*}
           \begin{CD}
             V_{\wrt{B}}                   @>h>H>      W_{\wrt{D}}       \\
             @V\scriptsize\identity VQV       @V\scriptsize\identity VPV \\
             V_{\wrt{\hat{B}}}             @>h>\hat{H}> W_{\wrt{\hat{D}}}
            \end{CD}
            \hspace*{4em}\qquad
           \begin{CD}
             V_{\wrt{B}}              @<h^{-1}<H^{-1}<    W_{\wrt{D}}       \\
             @V\scriptsize\identity VQV       @V\scriptsize\identity VPV \\
             V_{\wrt{\hat{B}}}        @<h^{-1}<\hat{H}^{-1}< W_{\wrt{\hat{D}}}
            \end{CD}
           \end{equation*}
           Yes, the inverses of the matrices represent the 
           inverses of the maps.
           That is, we can move from the lower right to the lower left by
           moving up, then left, then down.
           In other words, where \( \hat{H}=PHQ \) (and  \( P,Q \) invertible)
           and \( H,\hat{H} \) are invertible then
           \( \hat{H}^{-1}=Q^{-1}H^{-1}P^{-1} \).
        \partsitem Yes; this is the prior part repeated in different terms.
        \partsitem No, we need another assumption:~if \( H \) represents 
          \( h \) with respect to the same starting as ending bases \( B,B \), 
          for some \( B \) then \( H^2 \) represents
          \( \composed{h}{h} \).
          As a specific example, 
          these two matrices are both rank one and so they are
          matrix equivalent
          \begin{equation*}
             \begin{mat}[r]
               1  &0  \\
               0  &0
             \end{mat}
             \qquad
             \begin{mat}[r]
               0  &0  \\
               1  &0
             \end{mat}
          \end{equation*}
          but the squares are not matrix equivalent\Dash the square of the 
          first has rank one while the square of the second has rank zero.
        \partsitem No.
          These two are not matrix equivalent but have matrix equivalent
          squares.
          \begin{equation*}
             \begin{mat}[r]
               0  &0  \\
               0  &0
             \end{mat}
             \qquad
             \begin{mat}[r]
               0  &0  \\
               1  &0
             \end{mat} 
          \end{equation*}
      \end{exparts}  
    \end{answer}
  \recommended \item
    Square matrices are \definend{similar} if they represent the same
    transformation, but each with respect to the same ending as starting
    basis.
    That is, \( \rep{t}{B_1,B_1} \) is similar to \( \rep{t}{B_2,B_2} \).
    \begin{exparts}
      \partsitem Give a  definition of matrix similarity like that of
        \nearbydefinition{def:MatEquiv}.
      \partsitem Prove that similar matrices are matrix equivalent.
      \partsitem Show that similarity is an equivalence relation.
      \partsitem Show that if \( T \) is similar to \( \hat{T} \) then
        \( T^2 \) is similar to \( \hat{T}^2 \), the cubes are similar, etc.
        \textit{Contrast with the prior exercise.}
      \partsitem Prove that there are matrix equivalent matrices 
        that are not similar.
    \end{exparts}
    \begin{answer}
      \begin{exparts}
        \partsitem The 
          arrow diagram suggests the definition.
          \begin{equation*}
            \begin{CD}
              V_{\wrt{B_1}}                   @>t>T>        V_{\wrt{B_1}}       \\
              @V\scriptsize\identity VV        @V\scriptsize\identity VV \\
              V_{\wrt{B_2}}             @>t>\hat{T}>  V_{\wrt{B_2}}
            \end{CD}
          \end{equation*}
          Call matrices \( T, \hat{T} \) \definend{similar} if there
          is a nonsingular matrix \( P \) such that
          \( \hat{T}=P^{-1}TP \).
        \item Take \( P^{-1} \) to be \( P \) and take \( P \) to be \( Q \).
        \item \textit{This is as in \nearbyexercise{exer:MatEqIsEqRel}.}
          Reflexivity is obvious: \( T=I^{-1}TI \).
          Symmetry is also easy: \( \hat{T}=P^{-1}TP \) implies that 
          \( T=P\hat{T}P^{-1} \) (multiply the first equation from the right
          by $P^{-1}$ and from the left by $P$).
          For transitivity, assume that \( T_1={P_2}^{-1}T_2P_2 \) and that 
          \( T_2={P_3}^{-1}T_3P_3 \).
          Then \( T_1={P_2}^{-1}({P_3}^{-1}T_3P_3)P_2
                     =({P_2}^{-1}{P_3}^{-1})T_3(P_3P_2) \) and we are finished
          on noting that \( P_3P_2 \) is an invertible matrix with inverse
          \( {P_2}^{-1}{P_3}^{-1} \).
        \partsitem Assume  that \( \hat{T}=P^{-1}TP \).
          For the squares:
          \( \hat{T}^2=(P^{-1}TP)(P^{-1}TP)
                      =P^{-1}T(PP^{-1})TP=P^{-1}T^2P \).
          Higher powers follow by induction.
        \partsitem These two are matrix equivalent but their squares are not
          matrix equivalent.
          \begin{equation*}
             \begin{mat}[r]
               1  &0  \\
               0  &0
             \end{mat}
             \qquad
             \begin{mat}[r]
               0  &0  \\
               1  &0
             \end{mat}
          \end{equation*}
          By the prior item, matrix similarity and matrix equivalence are thus
          different.
      \end{exparts}  
   \end{answer}
\index{matrix equivalence|)}
\index{change of basis|)}
\end{exercises}
