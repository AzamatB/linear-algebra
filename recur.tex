% Chapter 4, Topic _Linear Algebra_ Jim Hefferon
%  http://joshua.smcvt.edu/linearalgebra
%  2001-Jun-12
\topic{Linear Recurrences}
\index{linear recurrence|(}
\index{recurrence relation|(}

In 1202 Leonardo of Pisa, known as Fibonacci, posed this problem.
\begin{quotation}
  \noindent A certain man put a pair of rabbits in a place
  surrounded on all sides by a wall.
  How many pairs of rabbits can be produced from that pair in a year if it
  is supposed that every month each pair begets a new pair which from the
  second month on becomes productive?
\end{quotation}
This moves past an elementary exponential growth model for
populations
to include that newborns are not fertile for some period, here a month.
However, it retains other simplifying assumptions such as  
that there is an age after which the rabbits are infertile.

To get next month's total number of pairs 
we add the number of pairs alive going into next month to
the number of pairs that will be newly born next month.
The latter equals the number of pairs that will be 
productive going into next month, which is the number that next month will 
have been alive for at least two months.
\begin{equation*}
  F(n)=F(n-1)+F(n-2)  \qquad \text{where $F(0)=0$, $F(1)=1$}
  \tag{$*$}
\end{equation*}
On the left is a
\definend{recurrence relation}.\index{recurrence}
It gets that name 
because $F$ recurs in its own defining equation.
On the right are the initial conditions.
From~($*$) we can compute $F(2)$, $F(3)$, etc., 
to work up to the answer for Fibonacci's question.
\begin{center}
  \begin{tabular}{r|ccccccccccccc}
    \textit{month}~$n$
     &0  &1  &2  &3  &4  &5  &6  &7  &8  &9  &10  &11  &12  \\ \hline
    \textit{pairs}~$F(n)$
     &0  &1  &1  &2  &3  &5  &8  &13  &21  &34  &55  &89  &144   
  \end{tabular}
\end{center}
We will use linear algebra to get a formula that
calculates $F(n)$ without having to first calculate the intermediate 
values $F(2)$, $F(3)$, etc.

We start by giving ($*$) a matrix formulation.
\begin{equation*}
  \colvec{F(n) \\ F(n-1)}
  =
  \begin{mat}[r]
    1  &1   \\
    1  &0
  \end{mat}
  \colvec{F(n-1) \\ F(n-2)}
  \qquad\text{where $\colvec{F(1) \\ F(0)}=\colvec{1 \\  0}$}
\end{equation*}  
Write $T$ for the matrix and 
$\vec{v}_{n}$ for the vector with components $F(n)$ and $F(n-1)$ so that
$\vec{v}_{n}=T^{n-1}\vec{v}_1$ for $n\geq 1$.
If we diagonalize $T$ then we 
have a fast way to compute its powers:~where $T=PDP^{-1}$ then 
$T^n=PD^nP^{-1}$ and the $n$-th power of the diagonal matrix $D$ is the 
diagonal matrix whose entries are the $n$-th powers of 
the entries of $D$.

The characteristic equation of $T$ is $\lambda^2-\lambda-1=0$.
The quadratic formula gives its roots as $(1+\sqrt{5})/2$ and 
$(1-\sqrt{5})/2$. 
(These are sometimes called ``golden ratios;''
see \cite{Falbo}.)
Diagonalizing gives this.
\begin{equation*}
  \begin{mat}[r]
    1  &1  \\
    1  &0
  \end{mat}
  =\begin{mat}
     \frac{1+\sqrt{5}}{2}  &\frac{1-\sqrt{5}}{2} \\
     % \sfrac{1+\sqrt{5}}{2}  &\sfrac{1-\sqrt{5}}{2} \\
     1                     &1
   \end{mat}
   \begin{mat}
     \frac{1+\sqrt{5}}{2}  &0   \\
     0                     &\frac{1-\sqrt{5}}{2}
   \end{mat}
   \begin{mat}
     \frac{1}{\sqrt{5}}     &-(\frac{1-\sqrt{5}}{2\sqrt{5}})  \\
     \frac{-1}{\sqrt{5}}    &\frac{1+\sqrt{5}}{2\sqrt{5}}       
   \end{mat}
\end{equation*} 
Introducing the vectors and taking the $n$-th power, we have
\begin{multline*}
  \colvec{F(n) \\ F(n-1)}
  =\begin{mat}[r]
    1  &1  \\
    1  &0
  \end{mat}^{n-1}
  \colvec{f(1) \\ f(0)}                                          \\
  =\begin{mat}
     \frac{1+\sqrt{5}}{2}  &\frac{1-\sqrt{5}}{2} \\
     1                     &1
   \end{mat}
   \begin{mat}
     \left(\frac{1+\sqrt{5}}{2}\right)^{n-1}  &0   \\
     0                                  &\left(\frac{1-\sqrt{5}}{2}\right)^{n-1}
   \end{mat}
   \begin{mat}
     \frac{1}{\sqrt{5}}     &-(\frac{1-\sqrt{5}}{2\sqrt{5}})  \\
     \frac{-1}{\sqrt{5}}    &\frac{1+\sqrt{5}}{2\sqrt{5}}       
   \end{mat}
  \colvec[r]{1 \\ 0}                                             
\end{multline*}
The calculation is ugly but not hard.
\begin{align*}
  \colvec{F(n) \\ F(n-1)}
  &=\begin{mat}
     \frac{1+\sqrt{5}}{2}  &\frac{1-\sqrt{5}}{2} \\
     1                     &1
   \end{mat}
   \begin{mat}
     \left(\frac{1+\sqrt{5}}{2}\right)^{n-1}  &0   \\
     0                                  &\left(\frac{1-\sqrt{5}}{2}\right)^{n-1}
   \end{mat}
  \colvec{\frac{1}{\sqrt{5}} \\ -\frac{1}{\sqrt{5}}}       \\ 
  &=\frac{1}{\sqrt{5}}
    \begin{mat}
     \frac{1+\sqrt{5}}{2}  &\frac{1-\sqrt{5}}{2} \\
     1                     &1
   \end{mat}
  \colvec{\left(\frac{1+\sqrt{5}}{2}\right)^{n-1} \\ 
          -\left(\frac{1-\sqrt{5}}{2}\right)^{n-1}}    \\    
  &=\frac{1}{\sqrt{5}}
  \colvec{\left(\frac{1+\sqrt{5}}{2}\right)^{n}  
          -\left(\frac{1-\sqrt{5}}{2}\right)^{n}  \\
          \left(\frac{1+\sqrt{5}}{2}\right)^{n-1}  
          -\left(\frac{1-\sqrt{5}}{2}\right)^{n-1}}   
\end{align*}
We want the first component.
\begin{equation*}
  F(n)=\frac{1}{\sqrt{5}}\left[\left(\frac{1+\sqrt{5}}{2}\right)^{n}
                               -\left(\frac{1-\sqrt{5}}{2}\right)^{n}\right]
\end{equation*}
This formula gives the value of any member of the sequence
without having to first find the intermediate values.  

Because $(1-\sqrt{5})/2\approx -0.618$
has absolute value less than one, its powers go to zero and so
the $F(n)$ formula is dominated by its first term. 
Although we have extended the elementary model of 
population growth by adding a delay period 
before the onset of fertility, we nonetheless 
still get a function that is asymptotically exponential.

In general, a 
\definend{homogeneous linear recurrence relation of order $k$}\index{recurrence}\index{linear recurrence!definition} 
has this form.
\begin{equation*}
  f(n)=a_{n-1}f(n-1)+a_{n-2}f(n-2)+\dots+a_{n-k}f(n-k)
\end{equation*}
This recurrence relation is homogeneous
because it has no constant term, i.e, we can rewrite it as
$0=-f(n)+a_{n-1}f(n-1)+a_{n-2}f(n-2)+\dots+a_{n-k}f(n-k)$.
It is of 
order~$k$ because it uses $k$-many prior terms to calculate $f(n)$.
The relation, cimbined with 
initial conditions\index{recurrence!initial conditions} 
giving values for 
$f(0)$, \ldots, $f(k-1)$,
completely determines a sequence, 
simply because 
we can compute $f(n)$ by first computing $f(k)$, $f(k+1)$, etc.
As with the Fibonacci case we will find a formula that solves the
recurrence, that directly gives $f(n)$  

Let $V$ be the set of functions with domain
$\N=\set{0,1,2,\ldots}$ and codomain $\Re$.  
(Where convenient we sometimes use the domain $\Z^+=\set{1,2,\ldots}$.)
This is a vector space under the usual meaning for addition and
scalar multiplication, that
$f+g$ is the map $x\mapsto f(x)+g(x)$ and 
$cf$ is the map $x\mapsto c\cdot f(x)$.

If we put aside any initial conditions and look only at the recurrence, 
then there may be many functions satisfying the relation.
For example, the Fibonacci recurrence that each value beyond the initial ones
is the sum of the prior two is satisfied by the 
function $L$ whose first few values 
are $L(0)=2$, $L(1)=1$, $L(2)=3$, $L(3)=4$, and
$L(4)=7$.

Fix a homogeneous linear recurrence relation of order~$k$ and
consider the subset $S$ of functions satisfying the relation (without
initial conditions).
This $S$ is a subspace of $V$.
It is nonempty because the zero function is a solution, by homogeneity.
It is closed under addition because if $f_1$ and $f_2$ are solutions then
this holds.
\begin{multline*}
  -(f_1+f_2)(n)+a_{n-1}(f_1+f_2)(n-1)+\dots+a_{n-k}(f_1+f_2)(n-k) \\  
  \begin{aligned}
    &=(-f_1(n)+\dots+a_{n-k}f_1(n-k))          \\
    &\qquad\hbox{}+(-f_2(n)+\dots+a_{n-k}f_2(n-k))     \\
    &=0+0=0
  \end{aligned}
\end{multline*}
It is also closed under scalar multiplication.
\begin{multline*}
  -(rf_1)(n)+a_{n-1}(rf_1)(n-1)+\dots+a_{n-k}(rf_1)(n-k) \\  
  \begin{aligned}
    &=r\cdot (-f_1(n)+\dots+a_{n-k}f_1(n-k))   \\
    &=r\cdot 0                                    \\
    &=0
  \end{aligned}
\end{multline*}
We can find the dimension of $S$.
Where $k$ is the order of the recurrence, 
consider this map from the set of functions~$S$ to the set of 
$k$-tall vectors. 
\begin{equation*}
  f 
  \mapsto 
  \colvec{f(0) \\ f(1) \\ \vdots \\ f(k-1)}
\end{equation*}
\nearbyexercise{exer:SeqToRnLinMap} shows that this is linear.
Any solution of the recurrence is uniquely determined by the $k$-many
initial conditions so this map is one-to-one and onto.
Thus it is an isomorphism, and $S$ has dimension $k$. 

So we can describe the
set of solutions of our linear homogeneous recurrence relation of order~$k$
by finding a basis consisting of
$k$-many linearly independent functions. 
To produce those we give 
the recurrence a matrix formulation.
\begin{equation*}
  \colvec{f(n) \\ f(n-1) \\ \vdots  \\ f(n-k+1)}
  =
  \begin{mat}
    a_{n-1}  &a_{n-2}  &a_{n-3}  &\ldots  &a_{n-k+1} &a_{n-k}  \\
    1    &0        &0        &\ldots  &0         &0        \\
    0    &1        &0                                      \\
    0    &0        &1                                      \\
    \vdots &\vdots &         &\ddots  &           &\vdots  \\
    0    &0        &0        &\ldots   &1          &0
  \end{mat}
  \colvec{f(n-1) \\ f(n-2) \\ \vdots  \\ f(n-k)}
\end{equation*}
Call the matrix~$A$.
We want its characteristic function,
the determinant of $A-\lambda I$.
The pattern in the $\nbyn{2}$ case 
\begin{equation*}
  \begin{mat}
    a_{n-1}-\lambda  &a_{n-2} \\
    1               &-\lambda
  \end{mat}
  =\lambda^2-a_{n-1}\lambda-a_{n-2}
\end{equation*}
and the $\nbyn{3}$ case
\begin{equation*}
  \begin{mat}
    a_{n-1}-\lambda  &a_{n-2}   &a_{n-3}  \\
    1               &-\lambda  &0        \\
    0               &1         &-\lambda
  \end{mat}
  =-\lambda^3+a_{n-1}\lambda^2+a_{n-2}\lambda+a_{n-3}
\end{equation*}
leads us to expect, 
and \nearbyexercise{exer:CharEqnIsDeter} verifies, that
this is the characteristic equation.
\begin{align*}
  0 &=
  \begin{vmat}
    a_{n-1}-\lambda &a_{n-2}  &a_{n-3}  &\ldots  &a_{n-k+1} &a_{n-k}  \\
    1              &-\lambda &0        &\ldots  &0         &0        \\
    0              &1        &-\lambda                                      \\
    0              &0        &1                                      \\
    \vdots         &\vdots &         &\ddots   &           &\vdots  \\
    0             &0        &0        &\ldots   &1          &-\lambda
  \end{vmat}                                                           \\
  &=\pm(-\lambda^k+a_{n-1}\lambda^{k-1}+a_{n-2}\lambda^{k-2}
       +\dots+a_{n-k+1}\lambda+a_{n-k})
\end{align*}
The $\pm$ is not relevant to find the roots
so we drop it.
We say that the polynomial 
$-\lambda^k+a_{n-1}\lambda^{k-1}+a_{n-2}\lambda^{k-2}
       +\dots+a_{n-k+1}\lambda+a_{n-k}$
is associated with the recurrence 
relation.\index{recurrence!associated polynomial}\index{polynomial!associated with recurrence}

If the characteristic equation has no repeated roots 
then the matrix is diagonalizable
and we can, in theory, get a formula for $f(n)$, as in the Fibonacci case.
But because we know that the subspace of solutions has dimension~$k$ 
we do not need to do the diagonalization calculation, provided that 
we can exhibit $k$ different 
linearly independent functions satisfying the relation.

Where $r_1$, $r_2$, \ldots, $r_k$ are the distinct roots,
consider the functions of powers of those roots, $f_{r_1}(n)=r_1^n$ through
$f_{r_k}(n)=r_k^n$.
\nearbyexercise{exer:SoltnsLinRecur} shows that 
each is a solution of the recurrence and that they form a 
linearly independent set.
% So given the homogeneous linear recurrence
% $f(n+1)=a_nf(n)+\dots+a_{n-k}f(n-k)$
% (that is, $0=-f(n+1)+a_nf(n)+\dots+a_{n-k}f(n-k)$)
% we consider the associated equation
% $0=-\lambda^k+a_n\lambda^{k-1}+\dots+a_{n-k+1}\lambda+a_{n-k}$.
So, if the roots of 
the associated polynomial are distinct, 
any solution of the relation has the form 
$f(n)=c_1r_1^n+c_2r_2^n+\dots+c_kr_k^n$ for some scalars $c_1, \dots, c_n$.
(The case of repeated roots is similar but we won't
cover it here; see any text on Discrete Mathematics.)

Now we bring in the initial conditions. 
Use them to solve for $c_1$, \ldots, $c_n$.
For instance, the polynomial associated with the Fibonacci relation is
$-\lambda^2+\lambda+1$, whose roots are $r_1=(1+\sqrt{5})/2$
and $r_2=(1-\sqrt{5})/2$
and so any solution of the Fibonacci recurrence 
has the form $f(n)=c_1((1+\sqrt{5})/2)^n+c_2((1-\sqrt{5})/2)^n$.
Use the Fibonacci initial conditions for $n=0$ and $n=1$ 
\begin{equation*}
  \begin{linsys}{2}
    c_1               &+  &c_2               &=  &0  \\
    (1+\sqrt{5}/2)c_1 &+  &(1-\sqrt{5}/2)c_2 &=  &1
  \end{linsys}
\end{equation*}
and solve to get $c_1=1/\sqrt{5}$ and $c_2=-1/\sqrt{5}$, as we found above.

We close by considering the nonhomogeneous case,
where the relation has the form
$f(n+1)=a_nf(n)+a_{n-1}f(n-1)+\dots+a_{n-k}f(n-k)+b$ for some nonzero $b$.
We only need a small adjustment
to make the transition from the homogeneous case.

This classic example illustrates:
in 1883, Edouard Lucas posed the
Tower of Hanoi\index{Tower of Hanoi} problem.
\begin{quotation}
  \noindent
  In the great temple at Benares, beneath the dome which marks the center 
  of the world, rests a brass plate in which are fixed three diamond needles,
  each a cubit high and as thick as the body of a bee.
  On one of these needles, at the creation,
  God placed sixty four disks of pure gold, the largest disk resting on 
  the brass plate, and the others getting smaller and smaller up to the
  top one.
  This is the Tower of Brahma.
  Day and night unceasingly the priests transfer the disks from one diamond
  needle to another according to the fixed and immutable laws of Bram-ah,
  which require that the priest on duty must not move more than one disk at a 
  time and that he must place this disk on a needle so that there is no
  smaller disk below it.
  When the sixty-four disks shall have been thus transferred from the needle
  on which at the creation God placed them to one of the other needles,
  tower, temple, and Brahmins alike will crumble into dusk, and with
  a thunderclap the world will vanish.
  (Translation of \cite{DeParville} from \cite{Ball}.)
\end{quotation}
We put aside the question of why the priests don't sit down for a while
and have the world last a little longer, and instead ask
how many disk moves it will take.
Before tackling the sixty~four disk problem 
we will consider the 
problem for three disks.

To begin, all three disks are on the same needle.
\begin{center}
  \includegraphics{ch5.6}
\end{center}
After the three moves of taking the small disk to the far needle, 
the mid-sized disk to the
middle needle, and then the small disk to the middle needle,
we have this. 
\begin{center}
  \includegraphics{ch5.7}
\end{center}
Now we can  move the big disk to the far needle.
Then to finish we repeat the three-move process on the two smaller 
disks, this time so that they end up on the third needle, 
on top of the big disk.

That sequence of moves is the best that we can do.
To move the bottom disk
at a minimum we must first move the smaller disks to the middle needle, 
then move the big one,
and then move all the smaller ones from the middle needle 
to the ending needle. 
Since this minimum suffices, we get this recurrence.
\begin{equation*}
  T(n)=T(n-1)+1+T(n-1)=2T(n-1)+1 \qquad \text{where $T(1)=1$}
\end{equation*} 
Here are the first few values of $T$.
\begin{center}
  \begin{tabular}{r|cccccccccc}
    \textit{disks}~$n$             
       &1  &2  &3  &4  &5  &6     &7    &8    &9   &10  \\
    \hline
    \textit{moves}~$T(n)$ 
       &1  &3  &7  &15  &31  &63  &127  &255  &511 &1023 
  \end{tabular}
\end{center}
Of course, these numbers are one less than a power of two.
To derive this 
write the original relation as $-1=-T(n)+2T(n-1)$.
Consider $0=-T(n)+2T(n-1)$, a linear homogeneous recurrence of order~$1$. 
Its associated polynomial is $-\lambda+2$, 
with the single root $r_1=2$.
Thus
functions satisfying the homogeneous relation take the form $c_12^n$.

That's the homogeneous solution.
Now we need a particular solution. 
Because the nonhomogeneous relation $-1=-T(n)+2T(n-1)$ is so simple, 
we can by eye spot a particular solution $T(n)=-1$.
Any solution of the recurrence 
$T(n)=2T(n-1)+1$ (without initial conditions)
is the sum of the homogeneous solution and the
particular solution: $c_12^n-1$.
Now the initial condition $T(1)=1$ gives that $c_1=1$ and we've gotten
the formula that generates the table:~the $n$-disk Tower of Hanoi problem 
requires $T(n)=2^n-1$ moves.

Finding a particular solution in more complicated cases is, 
perhaps not surprisingly,
more complicated.
A delightful and rewarding, but challenging, source 
is \cite{GrahamKnuthPatashnik}.
For more on the Tower of Hanoi see \cite{Ball}, \cite{Gardner57}, 
and \cite{Hofstadter}.
Some computer code follows the 
exercises.

\begin{exercises}
  \item 
   How many months until the number of Fibonacci rabbit pairs
   passes a thousand?
   Ten thousand?
   A million?
   \begin{answer}
     We use the formula
     \begin{equation*}
       F(n)=\frac{1}{\sqrt{5}}\left[\left(\frac{1+\sqrt{5}}{2}\right)^{n}
                               -\left(\frac{1-\sqrt{5}}{2}\right)^{n}\right]
     \end{equation*}
     As observed earlier, $(1+\sqrt{5})/2$ is larger than one 
     while $(1+\sqrt{5})/2$ has absolute value less than one.
\begin{lstlisting}
sage: phi = (1+5^(0.5))/2
sage: psi = (1-5^(0.5))/2
sage: phi
1.61803398874989
sage: psi
-0.618033988749895       
\end{lstlisting}
     So the value of the expression is dominated by the first term.
     Solving $1000=(1/\sqrt{5})\cdot ((1+\sqrt{5})/2)^n$ gives
     this.
\begin{lstlisting}
sage: a = ln(1000*5^(0.5))/ln(phi)
sage: a
16.0271918385296
sage: psi^(17)
-0.000280033582072583       
\end{lstlisting}
   So by the seventeenth power, the second term does not contribute 
   enough to change the roundoff.
   For the ten thousand and million calculations the situation is even
   more extreme.
\begin{lstlisting}
sage: b = ln(10000*5^(0.5))/ln(phi)
sage: b
20.8121638053112
sage: c = ln(1000000*5^(0.5))/ln(phi)
sage: c
30.3821077388746    
\end{lstlisting}
   The answers in these cases are $21$ and $31$.
   \end{answer}
  \item 
   Solve each homogeneous linear recurrence relations.
   \begin{exparts}
     \partsitem $f(n)=5f(n-1)-6f(n-2)$   
     \partsitem $f(n)=4f(n-2)$   
     \partsitem $f(n)=5f(n-1)-2f(n-2)-8f(n-3)$   
   \end{exparts}
   \begin{answer}
    \begin{exparts}
      \partsitem 
        We express the relation in matrix form.
        \begin{equation*}
          \colvec{f(n) \\ f(n-1)}
          =
          \begin{mat}[r]
            5  &-6  \\
            1  &0
          \end{mat}
          \colvec{f(n-1) \\ f(n-2)}
        \end{equation*}
        The characteristic equation of the matrix
        \begin{equation*}
          \begin{vmat}
            5-\lambda &-6       \\
            1         &-\lambda
          \end{vmat}
          =\lambda^2-5\lambda+6 
        \end{equation*}
        has roots of $2$ and $3$.
        Any function of the form
        $f(n)=c_12^n+c_23^n$
        satisfies the recurrence.
      \partsitem 
        The matrix expression of the relation is 
        \begin{equation*}
          \colvec{f(n) \\ f(n-1)}
          =
          \begin{mat}[r]
            0  &4  \\
            1  &0  
          \end{mat}
          \colvec{f(n-1) \\ f(n-2)}
        \end{equation*}
        and the characteristic equation
        \begin{equation*}
          \begin{vmat}
            \lambda^2-2       
          \end{vmat}
          =(\lambda-2)(\lambda+2)
        \end{equation*}
        has the two roots $2$ and $-2$.
        Any function of the form
        $f(n)=c_12^n+c_2(-2)^n$
        satisfies this recurrence.
      \partsitem 
        In matrix form the relation
        \begin{equation*}
          \colvec{f(n) \\ f(n-1) \\ f(n-2)}
          =
          \begin{mat}[r]
            5  &-2  &-8  \\
            1  &0  &0  \\
            0  &1  &0
          \end{mat}
          \colvec{f(n-1) \\ f(n-2) \\ f(n-3)}
        \end{equation*}
        has a characteristic equation with roots $-1$, $2$, and $4$.
        Any combination of the form
        $c_1(-1)^n+c_22^n+c_34^n$ solves the recurrence.
    \end{exparts} 
   \end{answer}
  \item \label{exer:SolvePartRecurSoltn} 
    Give a formula for the relations of the prior exercise, with
    these initial conditions.
    \begin{exparts}
      \partsitem $f(0)=1$, $f(1)=1$
      \partsitem $f(0)=0$, $f(1)=1$
      \partsitem $f(0)=1$, $f(1)=1$, $f(2)=3$.      
    \end{exparts}
  \item \label{exer:SeqToRnLinMap}
    Check that the isomorphism given between $S$ and $\Re^k$ is a linear map. 
    We argue above that this map is one-to-one.
    What is its inverse?
  \item \label{exer:CharEqnIsDeter}
    Show that the characteristic equation of the matrix is as stated, that is,
    is the polynomial associated with the relation.
    (\textit{Hint:} expanding down the final column and using induction 
    will work.)  
  \item \label{exer:SoltnsLinRecur}
    Given a homogeneous linear recurrence relation
    $f(n)=a_nf(n-1)+\dots+a_{n-k}f(n-k)$, let $r_1$, \ldots, $r_k$ be the
    roots of the associated polynomial.
    \begin{exparts}
      \partsitem Prove that each function 
         $f_{r_i}(n)=r_k^n$
         satisfies the recurrence (without initial conditions).
      \partsitem Prove that no $r_i$ is $0$.
      \partsitem Prove that the set 
        $\set{f_{r_1},\dots,f_{r_k}}$
        is linearly independent.
    \end{exparts}
  \item 
    (This refers to the value $T(64)=18,446,744,073,709,551,615$
    given in the computer code below.)
    Transferring one disk per second, how many years would it take
    the priests at the Tower of Hanoi to finish the job?
\end{exercises}

\announcecomputercode
This code generates the first few values of a function 
defined by a recurrence and initial conditions.
It is in the Scheme dialect of LISP,
specifically, \cite{ChickenScheme}.

After loading an extension that keeps the computer from switching to
floating point numbers when the integers get large,
the Tower of Hanoi function is straightforward.
\begin{lstlisting}
(require-extension numbers)

(define (tower-of-hanoi-moves n) 
    (if (= n 1)
       1
       (+ (* (tower-of-hanoi-moves (- n 1)) 
              2) 
           1) ) )

; Two helper funcitons
(define (first-few-outputs proc n)
    (first-few-outputs-aux proc n '()) )

(define (first-few-outputs-aux proc n lst)
    (if (< n 1)
    lst 
    (first-few-outputs-aux proc (- n 1) (cons (proc n) lst)) ) )
\end{lstlisting}
\noindent (For readers unused to recursive code:~to compute $T(64)$, 
the computer wants to compute $2*T(63)-1$, which requires
computing $T(63)$.
The computer puts the `times $2$' and the `plus $1$' aside for a moment.
It computes  $T(63)$ by using this same piece of code (that's 
what `recursive' means), and to do that it wants to compute $2*T(62)-1$.
This keeps up until, after $63$ steps, the computer
tries to compute $T(1)$.
It then returns $T(1)=1$, 
which allows the computation of $T(2)$ to proceed,
etc., until the original computation of $T(64)$ finishes.)

The helper functions give a table of the first few 
values.
% (Some language notes:~\texttt{'()} is the empty list
% and \texttt{cons} pushes something onto the start of a 
% list.
% Note that, in the last line, the function \texttt{proc}
% is called on argument \texttt{n}.)
Here is the session at the prompt.
\begin{lstlisting}
#;1> (load "hanoi.scm")
; loading hanoi.scm ...
; loading /var/lib//chicken/6/numbers.import.so ...
; loading /var/lib//chicken/6/chicken.import.so ...
; loading /var/lib//chicken/6/foreign.import.so ...
; loading /var/lib//chicken/6/numbers.so ...
#;2> (tower-of-hanoi-moves 64)
18446744073709551615
#;3> (first-few-outputs tower-of-hanoi-moves 64)
(1 3 7 15 31 63 127 255 511 1023 2047 4095 8191 16383 32767 65535 131071 262143 524287 1048575 
2097151 4194303 8388607 16777215 33554431 67108863 134217727 268435455 536870911 1073741823 
2147483647 4294967295 8589934591 17179869183 34359738367 68719476735 137438953471 274877906943 
549755813887 1099511627775 2199023255551 4398046511103 8796093022207 17592186044415 
35184372088831 70368744177663 140737488355327 281474976710655 562949953421311 1125899906842623 
2251799813685247 4503599627370495 9007199254740991 18014398509481983 36028797018963967 
72057594037927935 144115188075855871 288230376151711743 576460752303423487 1152921504606846975 
2305843009213693951 4611686018427387903 9223372036854775807 18446744073709551615)
\end{lstlisting}
\noindent This is a list of $T(1)$ through $T(64)$ (the session was 
edited to put in line breaks for readability).
\index{recurrence relation|)}
\index{linear recurrence|)}
\endinput

