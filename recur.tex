% Chapter 4, Topic _Linear Algebra_ Jim Hefferon
%  http://joshua.smcvt.edu/linalg.html
%  2001-Jun-12
\topic{Linear Recurrences}
\index{linear recurrences|(}

In 1202 Leonardo of Pisa, known as Fibonacci, posed this problem.
\begin{quotation}
  \noindent A certain man put a pair of rabbits in a place 
  surrounded on all sides by a wall.
  How many pairs of rabbits can be produced from that pair in a year if it
  is supposed that every month each pair begets a new pair which from the 
  second month on becomes productive? 
\end{quotation}
This moves past an elementary exponential growth model for
populations
to include the fact that there is an initial period where 
newborns are not fertile.
However, it retains other simplifying assumptions such as  
that there is no gestation period and no mortality.

To get next month's total number of pairs 
we add the number of pairs alive this month to
the number of pairs that will be newly born next month.
The latter number is the number of pairs of parents that will be 
productive next month, which is the number that next month will 
have been alive for at least two months,
and that is the number that were alive last month.
\begin{equation*}
  f(n+1)=f(n)+f(n-1)  \qquad \text{where $f(0)=0$, $f(1)=1$}
\end{equation*}
We call this a 
\definend{recurrence relation}\index{recurrence}
because $f$ recurs in its own defining equation.
With it we can answer Fibonacci's twelve-month question.
\begin{center}
  \begin{tabular}{r|ccccccccccccc}
    \textit{month}
     &0  &1  &2  &3  &4  &5  &6  &7  &8  &9  &10  &11  &12  \\ \hline
    \textit{pairs}
     &0  &1  &1  &2  &3  &5  &8  &13  &21  &34  &55  &89  &144   
  \end{tabular}
\end{center}
The sequence of numbers defined by the above equation is the 
\definend{Fibonacci sequence}\index{Fibonacci sequence}.
We will give a formula to
calculate $f(n+1)$ without having to first calculate $f(n)$, $f(n-1)$, etc.

We can give the recurrence a matrix formulation.
\begin{equation*}
  \begin{mat}[r]
    1  &1   \\
    1  &0
  \end{mat}
  \colvec{f(n) \\ f(n-1)}
  =
  \colvec{f(n+1) \\ f(n)}
  \qquad\text{where $\colvec{f(1) \\ f(0)}=\colvec{1 \\  0}$}
\end{equation*}  
Writing $T$ for the matrix and 
$\vec{v}_{n}$ for the vector with components $f(n+1)$ and $f(n)$,
we have that $\vec{v}_n=T^n\vec{v}_0$.
The advantage of this formulation comes from diagonalizing $T$ 
because then we 
have a fast way to compute its powers:~if $T=PDP^{-1}$ then 
$T^n=PD^nP^{-1}$ and the $n$-th power of the diagonal matrix $D$ is the 
diagonal matrix whose entries are the $n$-th powers of 
the entries of $D$.

The characteristic equation of $T$ is $\lambda^2-\lambda-1$.
The quadratic formula gives its roots as $(1+\sqrt{5})/2$ and 
$(1-\sqrt{5})/2$. 
(\textit{Remark:}
these are sometimes called ``golden ratios;''
see \cite{Falbo}.)
Diagonalizing gives this.
\begin{equation*}
  \begin{mat}[r]
    1  &1  \\
    1  &0
  \end{mat}
  =\begin{mat}
     \frac{1+\sqrt{5}}{2}  &\frac{1-\sqrt{5}}{2} \\
     1                     &1
   \end{mat}
   \begin{mat}
     \frac{1+\sqrt{5}}{2}  &0   \\
     0                     &\frac{1-\sqrt{5}}{2}
   \end{mat}
   \begin{mat}
     \frac{1}{\sqrt{5}}     &-(\frac{1-\sqrt{5}}{2\sqrt{5}})  \\
     \frac{-1}{\sqrt{5}}    &\frac{1+\sqrt{5}}{2\sqrt{5}}       
   \end{mat}
\end{equation*} 
Introducing the vectors and taking the $n$-th power, we have
\begin{multline*}
  \colvec{f(n+1) \\ f(n)}
  =\begin{mat}[r]
    1  &1  \\
    1  &0
  \end{mat}^n
  \colvec{f(1) \\ f(0)}                                          \\
  =\begin{mat}
     \frac{1+\sqrt{5}}{2}  &\frac{1-\sqrt{5}}{2} \\
     1                     &1
   \end{mat}
   \begin{mat}
     \left(\frac{1+\sqrt{5}}{2}\right)^n  &0   \\
     0                                    &\left(\frac{1-\sqrt{5}}{2}\right)^n
   \end{mat}
   \begin{mat}
     \frac{1}{\sqrt{5}}     &-(\frac{1-\sqrt{5}}{2\sqrt{5}})  \\
     \frac{-1}{\sqrt{5}}    &\frac{1+\sqrt{5}}{2\sqrt{5}}       
   \end{mat}
  \colvec[r]{1 \\ 0}                                             
\end{multline*}
The calculation is ugly but not hard.
\begin{align*}
  \colvec{f(n+1) \\ f(n)}
  &=\begin{mat}
     \frac{1+\sqrt{5}}{2}  &\frac{1-\sqrt{5}}{2} \\
     1                     &1
   \end{mat}
   \begin{mat}
     \left(\frac{1+\sqrt{5}}{2}\right)^n  &0   \\
     0                                    &\left(\frac{1-\sqrt{5}}{2}\right)^n
   \end{mat}
  \colvec{\frac{1}{\sqrt{5}} \\ -\frac{1}{\sqrt{5}}}       \\                     &=\begin{mat}
     \frac{1+\sqrt{5}}{2}  &\frac{1-\sqrt{5}}{2} \\
     1                     &1
   \end{mat}
  \colvec{\frac{1}{\sqrt{5}}\left(\frac{1+\sqrt{5}}{2}\right)^n \\ 
          -\frac{1}{\sqrt{5}}\left(\frac{1-\sqrt{5}}{2}\right)^n}         \\    
  &=
  \colvec{\frac{1}{\sqrt{5}}\left(\frac{1+\sqrt{5}}{2}\right)^{n+1}  
          -\frac{1}{\sqrt{5}}\left(\frac{1-\sqrt{5}}{2}\right)^{n+1}  \\
          \frac{1}{\sqrt{5}}\left(\frac{1+\sqrt{5}}{2}\right)^n  
          -\frac{1}{\sqrt{5}}\left(\frac{1-\sqrt{5}}{2}\right)^n}   
\end{align*}
We want the second component of that equation.
\begin{equation*}
  f(n)=\frac{1}{\sqrt{5}}\left[\left(\frac{1+\sqrt{5}}{2}\right)^n
                               -\left(\frac{1-\sqrt{5}}{2}\right)^n\right]
\end{equation*}
This formula finds the value of any member of the sequence
without having to first find the intermediate values.  
Notice that $(1-\sqrt{5})/2\approx −0.618$
has absolute value less than one and so its powers go to zero.
Thus the formula giving $f(n)$ is dominated by its first term. 

Although we have extended the elementary model of 
population growth by adding a delay period 
before the onset of fertility, we nonetheless 
still get a function that is asymptotically exponential.

In general, a 
\definend{linear recurrence relation}\index{recurrence}\index{linear recurrence} 
(or \definend{difference equation}\index{difference equation})
has this form.
\begin{equation*}
  f(n+1)=a_nf(n)+a_{n-1}f(n-1)+\dots+a_{n-k}f(n-k)
\end{equation*}
This recurrence relation is 
\definend{homogeneous}\index{recurrence!homogeneous}\index{difference equation!homogeneous} 
because there is no constant term, i.e, we can rewrite it into
the form $0=-f(n+1)+a_nf(n)+a_{n-1}f(n-1)+\dots+a_{n-k}f(n-k)$.
We say that this relation is of 
\definend{order}\index{order!of a recurrence}~$k$.
The relation along with the 
\definend{initial conditions}\index{recurrence!initial conditions} 
$f(0)$, \ldots, $f(k)$
completely determines a sequence.
For instance, 
the Fibonacci relation is of order two and,  
along with the two initial conditions $f(0)=0$ and $f(1)=1$, it
determines the Fibonacci sequence simply because 
we can compute any $f(n)$ by first computing $f(2)$, $f(3)$, etc.
We shall see how to use linear algebra to solve a linear
recurrence relation\Dash to 
find a formula that computes the $n$-th member of the sequence
without having to first compute the values of the prior members.  

Let $V$ be the set of functions with domain
$\N=\set{0,1,2,\ldots}$. 
(We shall use the codomain $\Re$ but 
we could use others, such as $\C$.
Below we sometimes have domain $\set{1,2,\ldots}$
but it is not an important difference.)
This is a vector space with the usual meaning for addition and
scalar multiplication of functions, that
the action of $f+g$ is $x\mapsto f(x)+g(x)$ and the action
of $cf$ is $x\mapsto c\cdot f(x)$.

If we omit the initial conditions then there may be many functions
satisfying a recurrence.
For example, the function $g$ whose first few values 
are $g(0)=1$, $g(1)=2$, $g(2)=3$, $g(3)=4$, and
$g(4)=7$
solves the Fibonacci relation without the Fibonacci initial conditions.

Fix a relation and
consider the subset $S$ of functions satisfying the relation without
initial conditions.
We claim that it is a subspace of $V$.
It is nonempty because the zero function is a solution.
It is closed under addition since if $f_1$ and $f_2$ are solutions, then
this holds.
\begin{multline*}
  a_{n+1}(f_1+f_2)(n+1)+\dots+a_{n-k}(f_1+f_2)(n-k) \\  
  \begin{aligned}
    &=(a_{n+1}f_1(n+1)+\dots+a_{n-k}f_1(n-k))          \\
    &\qquad\hbox{}+(a_{n+1}f_2(n+1)+\dots+a_{n-k}f_2(n-k))     \\
    &=0
  \end{aligned}
\end{multline*}
It is also closed under scalar multiplication.
\begin{multline*}
  a_{n+1}(rf_1)(n+1)+\dots+a_{n-k}(rf_1)(n-k) \\  
  \begin{aligned}
    &=r(a_{n+1}f_1(n+1)+\dots+a_{n-k}f_1(n-k))   \\
    &=r\cdot 0                                    \\
    &=0
  \end{aligned}
\end{multline*}
We can find the dimension of $S$.
Consider this map from the set of functions $S$ to the set of 
$k$-tall vectors. 
\begin{equation*}
  f 
  \mapsto 
  \colvec{f(0) \\ f(1) \\ \vdots \\ f(k)}
\end{equation*}
\nearbyexercise{exer:SeqToRnLinMap} shows that this map is linear.
Because
any solution of the recurrence is uniquely determined by the $k$
initial conditions, this map is one-to-one and onto.
Thus it is an isomorphism and thus $S$ has dimension $k$,
the order of the recurrence. 

So we can describe the
set of solutions of our linear homogeneous recurrence relation of degree~$k$
(again, without any initial conditions)
by taking linear combinations of a set having 
only $k$-many linearly independent functions. 

To produce those equations we give 
the recurrence $f(n+1)=a_nf(n)+\dots+a_{n-k}f(n-k)$
a matrix formulation.
\begin{equation*}
  \begin{mat}
    a_n  &a_{n-1}  &a_{n-2}  &\ldots  &a_{n-k+1} &a_{n-k}  \\
    1    &0        &0        &\ldots  &0         &0        \\
    0    &1        &0                                      \\
    0    &0        &1                                      \\
    \vdots &\vdots &         &\ddots  &           &\vdots  \\
    0    &0        &0        &\ldots   &1          &0
  \end{mat}
  \colvec{f(n) \\ f(n-1) \\ \vdots  \\ f(n-k)}
  =
  \colvec{f(n+1) \\ f(n) \\ \vdots  \\ f(n-k+1)}
\end{equation*}
We want the characteristic function of the matrix,
the determinant of $A-\lambda I$ where the above matrix is $A$.
The pattern in the $\nbyn{2}$ case 
\begin{equation*}
  \begin{mat}
    a_n-\lambda  &a_{n-1} \\
    1            &-\lambda
  \end{mat}
  =\lambda^2-a_n\lambda-a_{n-1}
\end{equation*}
and the $\nbyn{3}$ case
\begin{equation*}
  \begin{mat}
    a_n-\lambda  &a_{n-1}   &a_{n-2}  \\
    1            &-\lambda  &0        \\
    0            &1         &-\lambda
  \end{mat}
  =-\lambda^3+a_n\lambda^2+a_{n-1}\lambda+a_{n-2}
\end{equation*}
leads us to expect 
(and \nearbyexercise{exer:CharEqnIsDeter} verifies) that
this is the characteristic equation.
\begin{multline*}
  \begin{vmat}
    a_n-\lambda &a_{n-1}  &a_{n-2}  &\ldots  &a_{n-k+1} &a_{n-k}  \\
    1    &-\lambda &0        &\ldots  &0         &0        \\
    0    &1        &-\lambda                                      \\
    0    &0        &1                                      \\
    \vdots &\vdots &         &\ddots   &           &\vdots  \\
    0    &0        &0        &\ldots   &1          &-\lambda
  \end{vmat}                                                          \\
  =\pm(-\lambda^k+a_n\lambda^{k-1}+a_{n-1}\lambda^{k-2}
       +\dots+a_{n-k+1}\lambda+a_{n-k})
\end{multline*}
The $\pm$ is irrelevant to find the roots
so we will drop it.
We say that the polynomial is `associated' with the recurrence 
relation.\index{recurrence!associated polynomial}\index{polynomial!associated with recurrence}

If  
$-\lambda^k+a_n\lambda^{k-1}+a_{n-1}\lambda^{k-2}
   +\dots+a_{n-k+1}\lambda+a_{n-k}$
has no repeated roots 
then the matrix is diagonalizable
and we can, in theory, get a formula for $f(n)$ as in the Fibonacci case.
But because we know that the subspace of solutions has dimension $k$, 
we do not need to do the diagonalization calculation provided that 
we can exhibit $k$ linearly independent functions satisfying the relation.

Where $r_1$, $r_2$, \ldots, $r_k$ are the distinct roots,
consider the functions $f_{r_1}(n)=r_1^n$ through
$f_{r_k}(n)=r_k^n$ of powers of those roots.
\nearbyexercise{exer:SoltnsLinRecur} shows that 
each is a solution of the recurrence and that they form a 
linearly independent set.
% So given the homogeneous linear recurrence
% $f(n+1)=a_nf(n)+\dots+a_{n-k}f(n-k)$
% (that is, $0=-f(n+1)+a_nf(n)+\dots+a_{n-k}f(n-k)$)
% we consider the associated equation
% $0=-\lambda^k+a_n\lambda^{k-1}+\dots+a_{n-k+1}\lambda+a_{n-k}$.
So if the roots $r_1$, \ldots, $r_k$ of the associated polynomial 
are distinct then
any solution of the relation has the form 
$f(n)=c_1r_1^n+c_2r_2^n+\dots+c_kr_k^n$ for $c_1, \dots, c_n\in\Re$.
(The case of repeated roots is similar but we won't
cover it here; see any text on Discrete Mathematics.)

Now we bring in the initial conditions. 
Use them to solve for $c_1$, \ldots, $c_n$.
For instance, the polynomial associated with the Fibonacci relation is
$-\lambda^2+\lambda+1$, whose roots are $(1\pm\sqrt{5})/2$
and so any solution of the Fibonacci equation 
has the form $f(n)=c_1((1+\sqrt{5})/2)^n+c_2((1-\sqrt{5})/2)^n$.
Including the initial conditions for the cases $n=0$ and $n=1$ gives
\begin{equation*}
  \begin{linsys}{2}
    c_1               &+  &c_2               &=  &0  \\
    (1+\sqrt{5}/2)c_1 &+  &(1-\sqrt{5}/2)c_2 &=  &1
  \end{linsys}
\end{equation*}
which yields $c_1=1/\sqrt{5}$ and $c_2=-1/\sqrt{5}$, as we found above.

We close by considering the nonhomogeneous case,
where the relation has the form
$f(n+1)=a_nf(n)+a_{n-1}f(n-1)+\dots+a_{n-k}f(n-k)+b$ for some nonzero $b$.
We only need a small adjustment
to make the transition from the homogeneous case.
This classic example illustrates.

In 1883, Edouard Lucas posed the following problem,
today called the \definend{Tower of Hanoi}\index{Tower of Hanoi}.
\begin{quotation}
  \noindent
  In the great temple at Benares, beneath the dome which marks the center 
  of the world, rests a brass plate in which are fixed three diamond needles,
  each a cubit high and as thick as the body of a bee.
  On one of these needles, at the creation,
  God placed sixty four disks of pure gold, the largest disk resting on 
  the brass plate, and the others getting smaller and smaller up to the
  top one.
  This is the Tower of Brahma.
  Day and night unceasingly the priests transfer the disks from one diamond
  needle to another according to the fixed and immutable laws of Bram-ah,
  which require that the priest on duty must not move more than one disk at a 
  time and that he must place this disk on a needle so that there is no
  smaller disk below it.
  When the sixty-four disks shall have been thus transferred from the needle
  on which at the creation God placed them to one of the other needles,
  tower, temple, and Brahmins alike will crumble into dusk, and with
  a thunderclap the world will vanish.
  (Translation of \cite{DeParville} from \cite{Ball}.)
\end{quotation}
How many disk moves will it take?
Instead of tackling the sixty~four disk problem right away, 
we will consider the 
problem for smaller numbers of disks, starting with three.

To begin, all three disks are on the same needle.
\begin{center}
  \includegraphics{ch5.6}
\end{center}
After moving the small disk to the far needle, the mid-sized disk to the
middle needle, and then moving the small disk to the middle needle
we have this. 
\begin{center}
  \includegraphics{ch5.7}
\end{center}
Now we can  move the big disk over.
Then to finish we repeat the process of moving the two smaller 
disks, this time so that they end up on the third needle, 
on top of the big disk.

So to move the bottom disk
at a minimum we must first move the smaller disks to the middle needle, 
then move the big one,
and then move all the smaller ones from the middle needle 
to the ending needle. 
Since this minimum suffices, we get this recurrence
for the number of moves.
\begin{equation*}
  T(n+1)=T(n)+1+T(n)=2T(n)+1 \qquad \text{where $T(1)=1$}
\end{equation*} 
We can easily compute the first few values of $T$.
\begin{center}
  \begin{tabular}{r|cccccccccc}
    $n$    &1  &2  &3  &4  &5  &6     &7    &8    &9   &10  \\
    \hline
    $T(n)$ &1  &3  &7  &15  &31  &63  &127  &255  &511 &1023 
  \end{tabular}
\end{center}
Of course, those numbers are one less than a power of two.
To derive this equation instead of just guessing at it, 
we write the original relation as $-1=-T(n+1)+2T(n)$, consider 
the homogeneous relation $0=-T(n)+2T(n-1)$, 
get its associated polynomial $-\lambda+2$, 
which obviously has the single root $r_1=2$, and conclude that
functions satisfying the homogeneous relation take the form $T(n)=c_12^n$.

That's the homogeneous solution.
Now we need a particular solution. 
Because the nonhomogeneous relation $-1=-T(n+1)+2T(n)$ is so simple, 
in a few minutes (or by remembering the table) 
we can spot a particular solution, $T(n)=-1$.
So we have that (without yet considering the initial condition) any solution
of $T(n+1)=2T(n)+1$ is the sum of the homogeneous solution and this
particular solution: $T(n)=c_12^n-1$.
The initial condition $T(1)=1$ now gives that $c_1=1$, and we've gotten
the formula that generates the table:~the $n$-disk Tower of Hanoi problem 
requires a minimum of $2^n-1$ moves.

Finding a particular solution in more complicated cases is, unsurprisingly,
more complicated.
A delightful and rewarding, but challenging, source on recurrence relations
is \cite{GrahamKnuthPatashnik}.
For more on the Tower of Hanoi, \cite{Ball} or \cite{Gardner57}
are good starting points. 
So is \cite{Hofstadter}.
Some computer code for trying some recurrence relations follows the 
exercises.

\begin{exercises}
  \item 
   Solve each homogeneous linear recurrence relations.
   \begin{exparts}
     \partsitem $f(n+1)=5f(n)-6f(n-1)$   
     \partsitem $f(n+1)=4f(n-1)$   
     \partsitem $f(n+1)=5f(n)-2f(n-1)-8f(n-2)$   
   \end{exparts}
   \begin{answer}
    \begin{exparts}
      \partsitem 
        We express the relation in matrix form.
        \begin{equation*}
          \begin{mat}[r]
            5  &-6  \\
            1  &0
          \end{mat}
          \colvec{f(n) \\ f(n-1)}
          =
          \colvec{f(n+1) \\ f(n)}
        \end{equation*}
        The characteristic equation of the matrix
        \begin{equation*}
          \begin{vmat}
            5-\lambda &-6       \\
            1         &-\lambda
          \end{vmat}
          =\lambda^2-5\lambda+6 
        \end{equation*}
        has roots of $2$ and $3$.
        Any function of the form
        $f(n)=c_12^n+c_23^n$
        satisfies the recurrence.
      \partsitem 
        The matrix expression of the relation is 
        \begin{equation*}
          \begin{mat}[r]
            0  &4  \\
            1  &0  
          \end{mat}
          \colvec{f(n) \\ f(n-1)}
          =
          \colvec{f(n+1) \\ f(n)}
        \end{equation*}
        and the characteristic equation
        \begin{equation*}
          \begin{vmat}
            \lambda^2-2       
          \end{vmat}
          =(\lambda-2)(\lambda+2)
        \end{equation*}
        has the two roots $2$ and $-2$.
        Any function of the form
        $f(n)=c_12^n+c_2(-2)^n$
        satisfies this recurrence.
      \partsitem 
        In matrix form the relation
        \begin{equation*}
          \begin{mat}[r]
            5  &-2  &-8  \\
            1  &0  &0  \\
            0  &1  &0
          \end{mat}
          \colvec{f(n) \\ f(n-1) \\ f(n-2)}
          =
          \colvec{f(n+1) \\ f(n) \\ f(n-1)}
        \end{equation*}
        has a characteristic equation with roots $-1$, $2$, and $4$.
        Any combination of the form
        $c_1(-1)^n+c_22^n+c_34^n$ solves the recurrence.
    \end{exparts} 
   \end{answer}
  \item \label{exer:SolvePartRecurSoltn} 
    Give a formula for the relations of the prior exercise, with
    these initial conditions.
    \begin{exparts}
      \partsitem $f(0)=1$, $f(1)=1$
      \partsitem $f(0)=0$, $f(1)=1$
      \partsitem $f(0)=1$, $f(1)=1$, $f(2)=3$.      
    \end{exparts}
  \item \label{exer:SeqToRnLinMap}
    Check that the isomorphism given between $S$ and $\Re^k$ is a linear map. 
    We argue above that this map is one-to-one.
    What is its inverse?
  \item \label{exer:CharEqnIsDeter}
    Show that the characteristic equation of the matrix is as stated, that is,
    is the polynomial associated with the relation.
    (Hint: expanding down the final column, and using induction will work.)  
  \item \label{exer:SoltnsLinRecur}
    Given a homogeneous linear recurrence relation
    $f(n+1)=a_nf(n)+\dots+a_{n-k}f(n-k)$, let $r_1$, \ldots, $r_k$ be the
    roots of the associated polynomial.
    \begin{exparts}
      \partsitem Prove that each function 
         $f_{r_i}(n)=r_k^n$
         satisfies the recurrence (without initial conditions).
      \partsitem Prove that no $r_i$ is $0$.
      \partsitem Prove that the set 
        $\set{f_{r_1},\dots,f_{r_k}}$
        is linearly independent.
    \end{exparts}
  \item 
    (This refers to the value $T(64)=18,446,744,073,709,551,615$
    given in the computer code below.)
    Transferring one disk per second, how many years would it take
    the priests at the Tower of Hanoi to finish the job?
\end{exercises}

\announcecomputercode
This code allows the generation of the first few values of a function 
defined by a recurrence and initial conditions.
It is in the Scheme dialect of LISP
(specifically, it shows A.~Jaffer's free scheme interpreter SCM 
although any Scheme implementation should work).

First,
the Tower of Hanoi code is a straightforward implementation of the recurrence.
\begin{lstlisting}
(define (tower-of-hanoi-moves n) 
    (if (= n 1)
       1
       (+ (* (tower-of-hanoi-moves (- n 1)) 
              2) 
           1) )  )
\end{lstlisting}
\noindent (Note for readers unused to recursive code:~to compute $T(64)$, 
the computer wants to compute $2*T(63)-1$, which requires
computing $T(63)$.
The computer puts the `times $2$' and the `plus $1$' aside for a moment
to do that.
It computes  $T(63)$ by using this same piece of code (that's 
what `recursive' means), and to do that it wants to compute $2*T(62)-1$.
This keeps up (the next step is to try to do $T(62)$ while it holds the other 
arithmetic in waiting), until after $63$ steps the computer
tries to compute $T(1)$.
It then returns $T(1)=1$, 
which allows the computation of $T(2)$ can proceed,
etc., up until the original computation of $T(64)$ finishes.)

The next routine calculates a table of the first few 
values.
(Some language notes:~\texttt{'()} is the empty list, that is, 
the empty sequence, and \texttt{cons} pushes something onto the start of a 
list.
Note that, in the last line, the procedure \texttt{proc}
is called on argument \texttt{n}.)
\begin{lstlisting}
(define (first-few-outputs proc n)
    (first-few-outputs-helper proc n '()) )
;
(define (first-few-outputs-aux proc n lst)
    (if (< n 1)
    lst 
    (first-few-outputs-aux proc (- n 1) (cons (proc n) lst)) ) )
\end{lstlisting}
\noindent The session at the SCM prompt went like this.
\begin{lstlisting}
>(first-few-outputs tower-of-hanoi-moves 64)
(1 3 7 15 31 63 127 255 511 1023 2047 4095 8191 16383 32767 
65535 131071 262143 524287 1048575 2097151 4194303 8388607 
16777215 33554431 67108863 134217727 268435455 536870911 
1073741823 2147483647 4294967295 8589934591 17179869183 
34359738367 68719476735 137438953471 274877906943 549755813887 
1099511627775 2199023255551 4398046511103 8796093022207 
17592186044415 35184372088831 70368744177663 140737488355327 
281474976710655 562949953421311 1125899906842623 
2251799813685247 4503599627370495 9007199254740991 
18014398509481983 36028797018963967 72057594037927935 
144115188075855871 288230376151711743 576460752303423487 
1152921504606846975 2305843009213693951 4611686018427387903 
9223372036854775807 18446744073709551615)
\end{lstlisting}
\noindent This is a list of $T(1)$ through $T(64)$.
% (The $120$~mSec came on a 50~mHz '486 running in an XTerm of XWindow
% under Linux.
% The session was edited to put line breaks between numbers.)
\index{linear recurrences|)}
\endinput

