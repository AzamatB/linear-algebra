% Chapter 2, Section 2 _Linear Algebra_ Jim Hefferon
%  http://joshua.smcvt.edu/linalg.html
%  2001-Jun-10
\section{Linear Independence}
The prior section shows that we can understand a vector space
as a span, 
as an unrestricted linear combination of some of its elements.
For example, the space of linear polynomials $\set{a+bx\suchthat a,b\in\Re}$ 
is spanned by the set $\set{1,x}$.
The prior section also showed that a space can have many sets that span it.
The space of linear polynomials is also spanned by 
$\set{1,2x}$ and $\set{1,x,2x}$.

At the end of that section we described some spanning sets as `minimal'
but we never precisely defined that word.
We could mean one of two things.
We could mean that a spanning set is minimal if it 
contains the smallest number of members of any set with the same span.
With this meaning $\set{1,x,2x}$ is not minimal because it has 
three members while we've given spanning sets that have two.
Or we could mean that a spanning set is minimal when it has no elements 
that we can remove without changing the span.
Under this meaning $\set{1,x,2x}$ is not minimal because 
removing the \( 2x \) to get \( \set{1,x} \) leaves the
span unchanged.

The first sense of minimality appears to be a global requirement, 
in that to check if a spanning set is minimal 
we seemingly must look at all the spanning sets of 
a subspace and find one with the least number of elements. 
The second sense of minimality is local since
we need to look only at the set and consider the
span with and without various elements.
For instance, using the second sense
we could compare the span of $\set{1,x,2x}$ 
with the span of $\set{1,x}$ and
note that the $2x$ is a ``repeat'' in that its removal doesn't shrink the span.

In this section we will use the second sense of `minimal spanning set'
because of this technical convenience.
However, the most important result of this book is that the two senses 
coincide. 
We will prove that in the next section.









\subsection{Definition and Examples}
% We first characterize when a vector can be
% removed from a set without changing the span of that set.

% For that, note that if a vector $\vec{v}$ is not a member of a set $S$ 
% then the union $S\union\set{\vec{v}}$ 
% and the set $S$ differ only in that
% the former contains $\vec{v}$.

\begin{lemma}  \label{le:VecInSpanIffSpanUnchByAddVec}
If \( S \) is a subset of a vector space $V$  and 
$\vec{v}\in V$ then
$\spanof{S}=\spanof{S\union\set{\vec{v}}}$
if and only if
$\vec{v}\in\spanof{S}$.
\end{lemma}

\begin{proof}
The left to right implication is easy.
If $\spanof{S}=\spanof{S\union\set{\vec{v}}}$
then, since \( \vec{v}\in\spanof{S\union\set{\vec{v}}} \),
the equality of the two sets gives that \( \vec{v}\in\spanof{S} \).

For the right to left implication assume that \( \vec{v}\in \spanof{S} \) to
show that \( \spanof{S}=\spanof{S\union\set{\vec{v}}} \) by 
mutual set inclusion.
The inclusion \( \spanof{S}\subseteq\spanof{S\union\set{\vec{v}}} \) is
obvious.
For \( \spanof{S}\supseteq\spanof{S\union\set{\vec{v}}} \),
write an element of \( \spanof{S\union\set{\vec{v}}} \) as
\( d_0\vec{v}+d_1\vec{s}_1+\dots+d_m\vec{s}_m \)
and substitute \( \vec{v} \)'s expansion
as a linear combination of members of the same set
\( d_0(c_0\vec{t}_0+\dots+c_k\vec{t}_k)+d_1\vec{s}_1+\dots+d_m\vec{s}_m \).
This is a linear combination of linear combinations and so 
distributing \( d_0 \) results in
a linear combination of vectors from \( S \).
Hence each member of $\spanof{S\union\set{\vec{v}}}$ is also 
a member of $\spanof{S}$.
\end{proof}

\begin{example}
In \( \Re^3 \), where
\begin{equation*}
  \vec{v}_1=\colvec{1 \\ 0 \\ 0}\quad
  \vec{v}_2=\colvec{0 \\ 1 \\ 0}\quad
  \vec{v}_3=\colvec{2 \\ 1 \\ 0}
\end{equation*}
the spans \( \spanof{\set{\vec{v}_1,\vec{v}_2}} \) and
\( \spanof{\set{\vec{v}_1,\vec{v}_2,\vec{v}_3}} \) are equal since
\( \vec{v}_3 \) is in the span \( \spanof{\set{\vec{v}_1,\vec{v}_2}} \).
\end{example}

The lemma says that if we have a spanning set 
then we can remove a $\vec{v}$ to get a new set $S$ 
with the same span if and only if
$\vec{v}$ is a linear combination of vectors from $S$.
Thus, under the second sense described in this section's opening, 
a spanning set is minimal
if and only if it contains
no vectors that are linear combinations of the others in that set.

\begin{definition}
\label{def:LinInd}
A subset of a vector space is
\definend{linearly independent}\index{linearly independent}%
\index{sets!dependent, independent}
if none of its elements is a linear combination of the others.
Otherwise it is \definend{linearly dependent}\index{linearly dependent}.
\end{definition}

Here is an important observation:
although this way of writing one vector as a combination of the others
\begin{equation*}
   \vec{s}_0=\lincombo{c}{\vec{s}}
\end{equation*}
visually sets \( \vec{s}_0 \) off from the other vectors, algebraically
there is nothing special about it in that equation.
For any \( \vec{s}_i \) with a coefficient $c_i$ that is non-$0$
we can rewrite the relationship to set off \( \vec{s}_i \).
\begin{equation*}
   \vec{s}_i=(1/c_i)\vec{s}_0+\dots
              +(-c_{i-1}/c_i)\vec{s}_{i-1}+(-c_{i+1}/c_i)\vec{s}_{i+1}
              +\dots+(-c_n/c_i)\vec{s}_n
\end{equation*}
When we don't want to single out any vector by writing it alone on
one side of the equation we will instead say that
\( \vec{s}_0,\vec{s}_1,\dots,\vec{s}_n \) are in a
\definend{linear relationship}\index{linear relationship}%
\index{relationship!linear} 
and write the relationship
with all of the vectors on the same side.
The next result rephrases the linear independence definition in this style.
It gives what is usually the easiest way to compute whether
a finite set is dependent or independent.

\begin{lemma}   \label{le:LDIffANonTrivLinRel}
A subset \( S \) of a vector space is linearly independent if and only if 
among distinct \( \vec{s}_1,\dots,\vec{s}_n\in S \) 
the only linear relationship 
\begin{equation*}
   c_1\vec{s}_1+\dots+c_n\vec{s}_n=\zero
   \qquad c_1,\dots,c_n\in\Re
\end{equation*}
is the trivial one \( c_1=0,\dots,\,c_n=0 \).
\end{lemma}

\begin{proof}
% This is a direct consequence of the observation above.
If the set \( S \) is linearly independent then no vector
$\vec{s}_i$ 
is a linear combination of other vectors from $S$
so there is no linear relationship where some of the $\vec{s}\,$'s have 
nonzero coefficients.

If \( S \) is not linearly independent then some \( \vec{s}_i \) is a linear
combination 
$\vec{s}_i=c_1\vec{s}_1+\dots+c_{i-1}\vec{s}_{i-1}
    +c_{i+1}\vec{s}_{i+1}+\dots+c_n\vec{s}_n$
of other vectors from \( S \). 
Subtracting $\vec{s}_i$ from both sides
gives a relationship
involving a nonzero coefficient,
the \( -1 \) in front of \( \vec{s}_i \).
\end{proof}

\begin{example}  \label{ex:StaticsLIAndLD}
In the vector space of two-wide row vectors, the two-element set
\( \set{ \rowvec{40 &15},\rowvec{-50 &25}} \) is linearly independent.
To check this, set
\begin{equation*}
  c_1\cdot\rowvec{40 &15}+c_2\cdot\rowvec{-50 &25}=\rowvec{0 &0}
\end{equation*}
and solving the resulting system
\begin{equation*}
  \begin{linsys}{2}
    40c_1  &-  &50c_2  &=  &0  \\
    15c_1  &+  &25c_2  &=  &0  
   \end{linsys}
  \;\grstep{-(15/40)\rho_1+\rho_2}\;
  \begin{linsys}{2}
     40c_1  &- &50c_2       &=  &0  \\
            &  &(175/4)c_2  &=  &0  
   \end{linsys}
\end{equation*}
shows that both \( c_1 \) and \( c_2 \) are zero. 
So the only linear relationship between the two given row vectors
is the trivial relationship.

In the same vector space,
\( \set{ \rowvec{40 &15},\rowvec{20 &7.5}} \) is linearly dependent since
we can satisfy
\begin{equation*}
  c_1\rowvec{40 &15}+c_2\cdot\rowvec{20 &7.5}=\rowvec{0 &0}
\end{equation*}
with \( c_1=1 \) and \( c_2=-2 \).
\end{example}

\begin{remark}  \label{rem:StaticsLIAndLD}
Recall the Statics example that began this book.
We first set the unknown-mass objects at \( 40 \)~cm and \( 15 \)~cm and
got a balance,
and then we set the objects at \( -50 \)~cm and \( 25 \)~cm and got a balance.
With those two pieces of information we could compute values of the unknown
masses.
Had we instead first set the unknown-mass objects at \( 40 \)~cm and 
\( 15 \)~cm, and then at
\( 20 \)~cm and \( 7.5 \)~cm, we would not have been able to compute the values
of the unknown masses.
Intuitively, 
the problem is that the \( \rowvec{20 &7.5} \) information is a ``repeat'' 
of the
$\rowvec{40 &15}$ information\Dash that is, $\rowvec{20 &7.5}$ is in the 
span of the set $\set{\rowvec{40 &15}}$\Dash and so we would be trying to 
solve a two-unknowns problem with what is essentially one piece of information.
\end{remark}

\begin{example}
The set \( \set{1+x,1-x} \) is linearly independent in \( \polyspace_2 \), the
space of quadratic polynomials with real coefficients, because
\begin{equation*}
   0+0x+0x^2
   =
   c_1(1+x)+c_2(1-x)
   =
   (c_1+c_2)+(c_1-c_2)x+0x^2
\end{equation*}
gives 
\begin{eqnarray*}
  \begin{linsys}{2}
    c_1  &+  &c_2  &=  &0  \\
    c_1  &-  &c_2  &=  &0  
   \end{linsys}
  &\grstep{-\rho_1+\rho_2}
  &\begin{linsys}{2}
     c_1  &+  &c_2  &=  &0  \\
          &   &2c_2 &=  &0
  \end{linsys}
\end{eqnarray*}
since polynomials are equal only if their coefficients are equal.
Thus, the only linear relationship between these two members of
$\polyspace_2$ is the trivial one.
\end{example}

\begin{example}
The rows of this matrix
\begin{equation*}
  A=
  \begin{mat}[r]
    2  &3  &1  &0  \\
    0  &-1 &0  &-2 \\
    0  &0  &0  &1
  \end{mat}
\end{equation*}
form a linearly independent set.
This is easy to check but we know already that it is true:
at the end of the first chapter, in Lemma~One.III.\ref{le:EchFormNoLinCombo},
we showed that the rows of any echelon form matrix are linearly independent.
\end{example}

\begin{example}
In \( \Re^3 \), where
\begin{equation*}
   \vec{v}_1=\colvec{3 \\ 4 \\ 5}
   \quad
   \vec{v}_2=\colvec{2 \\ 9 \\ 2}
   \quad
   \vec{v}_3=\colvec{4 \\ 18 \\ 4}
\end{equation*}
the set \( S=\set{\vec{v}_1,\vec{v}_2,\vec{v}_3} \)
is linearly dependent because this is a relationship
\begin{equation*}
  0\cdot\vec{v}_1
  +2\cdot\vec{v}_2
  -1\cdot\vec{v}_3
  =\zero
\end{equation*}
where not all of the scalars are zero (the fact that some 
of the scalars are zero doesn't matter).
\end{example}

\begin{remark}  \label{rem:WhyLIIsNonTrivLinRel}
That example illustrates why, 
although \nearbydefinition{def:LinInd} is a clearer
statement of what independence is,
\nearbylemma{le:LDIffANonTrivLinRel} is more useful for
computations.
Working straight from the definition, someone trying to compute whether $S$
is linearly independent would start by setting
\( \vec{v}_1=c_2\vec{v}_2+c_3\vec{v}_3 \)
and concluding that there are no such $c_2$ and $c_3$.
But knowing that the first vector is not
dependent on the other two is not enough.
This person would have to go on to try
\( \vec{v}_2=c_1\vec{v}_1+c_3\vec{v}_3 \) 
to find the dependence $c_1=0$, \( c_3=1/2 \).
\nearbylemma{le:LDIffANonTrivLinRel} 
gets the same conclusion with only one computation.
\end{remark}

\begin{example} \label{ex:EmSetLI}
The empty subset\index{sets!empty} of a vector space is linearly independent.
There is no nontrivial linear relationship among its members as it has
no members.
\end{example}

\begin{example} \label{ex:SetWithZeroVecLD}
In any vector space, any subset containing the zero vector is linearly 
dependent.
For example, in the space $\polyspace_2$ of quadratic polynomials, 
consider the subset $\set{1+x,x+x^2,0}$.

One way to see that this subset is linearly 
dependent is to use \nearbylemma{le:LDIffANonTrivLinRel}:~we have  
$0\cdot\vec{v}_1+0\cdot\vec{v}_2+1\cdot\zero=\zero$, and this is a nontrivial
relationship as not all of the coefficients are zero.
Another way to see that this subset is
linearly dependent is to go straight to \nearbydefinition{def:LinInd}:~we
can express the third member of the subset
as a linear combination of the 
first two, namely, $c_1\vec{v}_1+c_2\vec{v}_2=\zero$ is 
satisfied by taking $c_1=0$ and $c_2=0$
(in contrast to the lemma, the definition allows all of the coefficients to 
be zero).

There is subtler way to see that this subset is dependent.
The zero vector is equal to the trivial sum, 
the sum of the empty set.
So a set containing the zero vector has an element that
is a combination of a set of other vectors from the 
set, specifically,
the zero vector is a combination of the empty set.
\end{example}

The above examples, especially \nearbyexample{ex:StaticsLIAndLD},
underline the discussion that begins this section.
The next result 
says that given a finite set,
we can produce a linearly independent subset
by discarding what \nearbyremark{rem:StaticsLIAndLD}
calls ``repeats.''


\begin{theorem}
\label{th:AlwaysAnLDSubset}
In a vector space,
any finite subset has a linearly independent subset with the same span.
\end{theorem}

\begin{proof}
If the set  \( S=\set{ \vec{s}_1,\dots,\vec{s}_n} \) is linearly independent
then $S$ itself satisfies the statement, so 
assume that it is linearly dependent.

By the definition of dependence,
there is a vector \( \vec{s}_i \) that is a linear combination of
the others.
Call that vector \( \vec{v}_1 \).
Discard it:
define the set \( S_1=S-\set{\vec{v}_1} \).
By \nearbylemma{le:VecInSpanIffSpanUnchByAddVec}, the span does not
shrink \( \spanof{S_1}=\spanof{S} \).

Now, if \( S_1 \) is linearly independent then we are finished.
Otherwise iterate the prior paragraph: 
take a vector $\vec{v}_2$ 
that is a linear combination of
other members of $S_1$ and discard it
to derive \( S_2=S_1-\set{\vec{v}_2} \)
such that \( \spanof{S_2}=\spanof{S_1} \).
Repeat this until a linearly independent set $S_j$ appears;
one must appear eventually because \( S \) is finite
and the empty set is linearly independent.
(Formally, this argument uses
induction on $n$, the number of elements in the starting set.
\nearbyexercise{exer:FillIndDetProofSetHasLISub} asks for the details.)
\end{proof}

\begin{example}  \label{ex:ShrinkSetSameSpan}
This set spans \( \Re^3 \) (the check is routine).
\begin{equation*}
  S=\set{\colvec{1 \\ 0 \\ 0},
     \colvec{0 \\ 2 \\ 0},
     \colvec{1 \\ 2 \\ 0},
     \colvec{0 \\ -1 \\ 1},
     \colvec{3 \\ 3 \\ 0}  }
\end{equation*}
Looking for a linear relationship
\begin{equation*}
  c_1\colvec{1 \\ 0 \\ 0}
  +c_2\colvec{0 \\ 2 \\ 0}
  +c_3\colvec{1 \\ 2 \\ 0}
  +c_4\colvec{0 \\ -1 \\ 1}
  +c_5\colvec{3 \\ 3 \\ 0}
  =\colvec{0 \\ 0 \\ 0}\qquad\tag{$*$}
\end{equation*}
gives a system 
\begin{equation*}
  \begin{linsys}{5}
     c_1  &   &      &+  &c_3   &+  &    &+  &3c_5 &= &0  \\
          &   &2c_2  &+  &2c_3  &-  &c_4 &+  &3c_5 &= &0  \\
          &   &      &   &     &   &c_4  &   &     &= &0  
\end{linsys}
\end{equation*}
with leading variables $c_1$, $c_2$, and $c_4$ and
free variables $c_3$ and $c_5$.
Parametrizing the solution set
\begin{equation*}
  \set{\colvec{c_1 \\ c_2 \\ c_3 \\ c_4 \\ c_5}=
     c_3\colvec{-1 \\ -1 \\ 1 \\ 0 \\ 0}
     +c_5\colvec{-3 \\ -3/2 \\ 0 \\ 0 \\ 1}
     \suchthat c_3,c_5\in\Re }
\end{equation*}
shows that there are nontrivial solutions to equation~($*$).
For instance, setting \( c_3=0 \) and \( c_5=1 \)
gives that $c_1=-3$, $c_2=-3/2$, and $c_4=0$, and so the fifth
vector is a linear combination of the first two.

Because of that,
\nearbylemma{le:VecInSpanIffSpanUnchByAddVec} says that 
discarding the fifth vector to get
\begin{equation*}
  S_1=\set{\colvec{1 \\ 0 \\ 0},
     \colvec{0 \\ 2 \\ 0},
     \colvec{1 \\ 2 \\ 0},
     \colvec{0 \\ -1 \\ 1}  }
\end{equation*}
leaves the span unchanged $\spanof{S_1}=\spanof{S}$.

Similarly, setting \( c_3=1 \) and \( c_5=0 \)
gives a linear dependence in equation~($*$) above.
Since $c_5=0$ this is a relationship among the first four vectors,
the members of $S_1$.
Thus we can discard $c_3$'s vector from $S_1$ to get
\begin{equation*}
  S_2=\set{\colvec{1 \\ 0 \\ 0},
     \colvec{0 \\ 2 \\ 0},
     \colvec{0 \\ -1 \\ 1}  }
\end{equation*}
with the same span as $S_1$, and therefore the same span as $S$, 
but with one difference.
We can easily check that $S_2$ is linearly independent  
and so discarding any of its elements will shrink the span.
\end{example}

That example makes clear the general method: 
given a finite set of vectors, we first
write the system to find a linear dependence.
Then discarding any vectors 
associated with the free variables of that system will leave the span
unchanged. 

\nearbytheorem{th:AlwaysAnLDSubset} describes producing a linearly 
independent set by shrinking, that is, by taking subsets. 
We finish this subsection by considering
how linear independence and dependence, which are properties of sets, interact
with the subset relation between sets.

\begin{lemma}  \label{le:SubsetPreserveLI}
Any subset of a linearly independent set is also linearly independent.
Any superset of a linearly dependent set is also linearly dependent.
\end{lemma}

\begin{proof}
This is clear.
\end{proof}

Restated, independence is preserved by subset 
and dependence is preserved by superset.

Those are two of the four possible cases.
The third case, whether linear dependence is preserved by the subset operation,
is covered by \nearbyexample{ex:ShrinkSetSameSpan}, which gives
a linearly dependent set $S$ with a subset $S_1$ 
that is linearly dependent and another subset $S_2$
that is linearly independent.

That leaves one case, whether linear independence is preserved by superset.
The next example shows what can happen.

\begin{example} \label{ex:LinindSetsAndSuper}
In each of these three paragraphs
the subset $S$ is linearly independent.

For the set
\begin{equation*}
  S
   =\set{\colvec{1 \\ 0 \\ 0}}
\end{equation*}
the span \( \spanof{S} \) is the \( x \)-axis.
Here are two supersets of $S$, one linearly dependent and the other linearly
independent.
\begin{center}
     dependent:
     \( \set{
         \colvec{1 \\ 0 \\ 0},
         \colvec{-3 \\ 0 \\ 0}} \)      
     \qquad
     independent:
     \( \set{
         \colvec{1 \\ 0 \\ 0},
         \colvec{0 \\ 1 \\ 0}} \)      
\end{center}
Checking the dependence or independence of these sets is easy.

For
\begin{equation*}
  S
   =\set{\colvec{1 \\ 0 \\ 0},
          \colvec{0 \\ 1 \\ 0}
                 }
\end{equation*}
the span \( \spanof{S} \) is the \( xy \)-plane.
These are two supersets.
\begin{center}
     dependent:
     \( \set{
         \colvec{1 \\ 0 \\ 0},
         \colvec{0 \\ 1 \\ 0},
         \colvec{3 \\ -2 \\ 0} } \)       
     \qquad
     independent:
     \( \set{
         \colvec{1 \\ 0 \\ 0},
         \colvec{0 \\ 1 \\ 0},
         \colvec{0 \\ 0 \\ 1} } \)    
\end{center}

If
\begin{equation*}
   S =\set{\colvec{1 \\ 0 \\ 0},
          \colvec{0 \\ 1 \\ 0},
          \colvec{0 \\ 0 \\ 1}        } 
\end{equation*}
then \( \spanof{S}=\Re^3 \).
A linearly dependent superset is 
\begin{center}
     dependent:
     \( \set{
         \colvec{1 \\ 0 \\ 0},
         \colvec{0 \\ 1 \\ 0},
         \colvec{0 \\ 0 \\ 1},
         \colvec{2 \\ -1 \\ 3} } \)     
\end{center}
but there are no linearly independent supersets of $S$.
The reason is that for any vector that we would add to make a
superset, the linear dependence equation
\begin{equation*}
  \colvec{x \\ y \\ z}
  =c_1\colvec{1 \\ 0 \\ 0}
   +c_2\colvec{0 \\ 1 \\ 0}
   +c_3\colvec{0 \\ 0 \\ 1}
\end{equation*}
has a solution $c_1=x$, $c_2=y$, and $c_3=z$.
\end{example}

So, in general, a linearly independent set 
may have a superset that is dependent.
And, in general, a  
linearly independent set may  have a superset that is independent.
We can characterize when the superset is one and when it is the other.

\begin{lemma} \label{le:SUnionXiLIIffXiNotInSpan}
If \( S \) is a linearly independent subset of a vector space \( V \) then
for any \( \vec{v}\in V \) with \( \vec{v}\not\in S \), the set
$S\union\set{\vec{v}}$ is linearly dependent
if and only if
$\vec{v}\in\spanof{S}$.
\end{lemma}

\begin{proof}
The right-to-left implication is clear: if \( \vec{v}\in\spanof{S} \) then
\( \vec{v}=\lincombo{c}{\vec{s}} \) where each \( \vec{s}_i\in S \) and
\( c_i\in\Re \), and so \( \zero=(-1)\vec{v}+\lincombo{c}{\vec{s}} \)
is a nontrivial linear relationship among elements of
\( S\union\set{\vec{v}} \).

The other implication requires the assumption that \( S \) is linearly
independent.
Suppose that
\( S\union\set{\vec{v}} \) is linearly dependent.
Then there is a
nontrivial \( c_0\vec{v}+\lincombo{c}{\vec{s}}=\zero \),
and independence of $S$ then implies that \( c_0\neq 0 \) 
or else it would be a nontrivial relationship among members of \( S \).
Rewrite this equation as
\( \vec{v}=-(c_1/c_0)\vec{s}_1-\dots-(c_n/c_0)\vec{s}_n \) to get
that \( \vec{v}\in\spanof{S} \).
\end{proof}

\noindent
(Compare this result with \nearbylemma{le:VecInSpanIffSpanUnchByAddVec}.
Both say, roughly, that $\vec{v}$ is a ``repeat'' if it is in the
span of $S$.
However, note the additional hypothesis here of linear independence.)

\begin{corollary}
\label{cor:LDMeansLC}
A subset \( S=\set{\vec{s}_1,\dots,\vec{s}_n} \) of a vector space
is linearly dependent if and only if some \( \vec{s_i} \)
is a linear combination of the vectors 
\( \vec{s}_1 \), \ldots, \( \vec{s}_{i-1} \)
listed before it.
\end{corollary}

\begin{proof}
Consider \( S_0=\set{} \), \( S_1=\set{\vec{s_1}} \),
\( S_2=\set{\vec{s}_1,\vec{s}_2 } \), etc.
Some index \( i\geq 1 \) is the first one with
\( S_{i-1}\union\set{\vec{s}_i } \)
linearly dependent, and there \( \vec{s}_i\in\spanof{ S_{i-1} } \).
\end{proof}

We can restate \nearbylemma{le:SUnionXiLIIffXiNotInSpan}
in terms of independence instead of dependence:
if \( S \) is linearly independent and \( \vec{v}\not\in S \) then
the set \( S\union\set{\vec{v}} \)
is also linearly independent if and only if \( \vec{v}\not\in\spanof{S}. \)
Applying \nearbylemma{le:VecInSpanIffSpanUnchByAddVec},
we conclude that
if \( S \) is linearly independent and \( \vec{v}\not\in S \)
then \( S\union\set{\vec{v}} \) is also
linearly independent if and only if
\( \spanof{S\union\set{\vec{v}}}\neq\spanof{S} \).
Briefly,
when passing from $S$ to a superset $S_1$,
to preserve linear independence we must expand the span
$\spanof{S_1}\supset\spanof{S}$.

\nearbyexample{ex:LinindSetsAndSuper} shows that some 
linearly independent sets are maximal\Dash have as many elements as 
possible\Dash in
that they have no supersets that are linearly independent.
By the prior paragraph, 
a linearly independent sets is maximal if and only if it
spans the 
entire space, because then no vector exists that is not already in the span.

This table summarizes the interaction between the properties of independence
and dependence and the relations of subset and superset.
\medskip
\begin{center} \renewcommand{\arraystretch}{1.1}
  \begin{tabular}[b]{r|cc} 
                        \multicolumn{1}{c}{}
                        &\multicolumn{1}{c}{\( S_1\subset S \)}
                        &\multicolumn{1}{c}{\( S_1\supset S \)}      \\
     \cline{2-3}\rule{0em}{12pt} % make box a bit taller
          \textit{$S$ independent} 
              &\( S_1 \) must be independent   &\( S_1 \) may be either\\
     % \cline{2-3}\noalign{\smallskip}
          \textit{$S$ dependent} 
              &\( S_1 \) may be either &\( S_1 \) must be dependent    \\
     % \cline{2-3}
   \end{tabular}
\end{center}
\medskip
In developing this table we've uncovered an
intimate relationship between linear independence and span.
Complementing the fact that
a spanning set is minimal if and only if it is linearly independent,
a linearly independent set is maximal if and only if it spans the space.

In summary,
we have introduced the definition of linear independence to 
formalize the idea of the minimality of a spanning set.
We have developed some properties of this idea.
The most important is \nearbylemma{le:SUnionXiLIIffXiNotInSpan}, which 
tells us that a linearly independent set is maximal when it spans the space.


\begin{exercises}
  \recommended \item
    Decide whether each subset of \( \Re^3 \) is linearly dependent or
    linearly independent. 
    \begin{exparts}
      \partsitem \( \set{\colvec{1 \\ -3 \\ 5},
                    \colvec{2 \\ 2 \\ 4},
                    \colvec{4 \\ -4 \\ 14} }  \)
      \partsitem \( \set{\colvec{1 \\ 7 \\ 7},
                    \colvec{2 \\ 7 \\ 7},
                    \colvec{3 \\ 7 \\ 7} }  \)
      \partsitem \( \set{\colvec{0 \\ 0 \\ -1},
                    \colvec{1 \\ 0 \\ 4} }  \)
      \partsitem \( \set{\colvec{9 \\ 9 \\ 0},
                    \colvec{2 \\ 0 \\ 1},
                    \colvec{3 \\ 5 \\ -4},
                    \colvec{12 \\ 12 \\ -1} }  \)
    \end{exparts}
    \begin{answer}
      For each of these, when the subset is independent you must prove it, and
      when the subset is dependent you must give an example of a dependence.
      \begin{exparts}
        \partsitem It is dependent.
          Considering
          \begin{equation*}
             c_1\colvec{1 \\ -3 \\ 5}
             +c_2\colvec{2 \\ 2 \\ 4}
             +c_3\colvec{4 \\ -4 \\ 14}
             =\colvec{0 \\ 0 \\ 0}
          \end{equation*}
          gives this linear system.
          \begin{equation*}
            \begin{linsys}{3}
              c_1  &+  &2c_2  &+  &4c_3  &=  &0  \\
              -3c_1&+  &2c_2  &-  &4c_3  &=  &0  \\
              5c_1 &+  &4c_2  &+  &14c_3 &=  &0  
            \end{linsys}
          \end{equation*}
          Gauss' method 
          \begin{equation*}
            \begin{amat}[r]{3}
              1  &2  &4  &0  \\
              -3 &2  &-4 &0  \\
              5  &4  &14 &0
            \end{amat}
            \grstep[-5\rho_1+\rho_3]{3\rho_1+\rho_2}
            \;\grstep{(3/4)\rho_2+\rho_3}            
            \begin{amat}[r]{3}
              1  &2  &4  &0  \\
              0  &8  &8  &0  \\
              0  &0  &0  &0
            \end{amat}
          \end{equation*}
          yields a free variable, so there are infinitely many solutions.
          For an example of a particular dependence we can set $c_3$ to be,
          say, $1$.  Then we get
          \( c_2=-1 \) and \( c_1=-2 \).
        \partsitem It is dependent.
          The linear system that arises here
          \begin{equation*}
            \begin{amat}[r]{3}
              1  &2  &3  &0  \\
              7  &7  &7  &0  \\
              7  &7  &7  &0
            \end{amat}
            \;\grstep[-7\rho_1+\rho_3]{-7\rho_1+\rho_2}
            \;\grstep{-\rho_2+\rho_3}\;
            \begin{amat}[r]{3}
              1  &2  &3   &0  \\
              0  &-7 &-14 &0  \\
              0  &0  &0   &0
            \end{amat}
          \end{equation*}
          has infinitely many solutions.
          We can get a particular solution by taking $c_3$ to be, say,
          $1$, and back-substituting to get the resulting $c_2$ and $c_1$.
        \partsitem It is linearly independent.
          The system
          \begin{equation*}
            \begin{amat}[r]{2}
              0  &1  &0  \\
              0  &0  &0  \\
              -1 &4  &0
            \end{amat}
            \;\grstep{\rho_1\leftrightarrow\rho_2}
            \;\grstep{\rho_3\leftrightarrow\rho_1}\;
            \begin{amat}{2}
              -1 &4  &0  \\
              0  &1  &0  \\
              0  &0  &0  
            \end{amat}
          \end{equation*}
          has only the solution $c_1=0$ and $c_2=0$.
          (We could also have gotten the answer by inspection\Dash the second
          vector is obviously not a multiple of the first, and vice versa.)
        \partsitem It is linearly dependent.
          The linear system
          \begin{equation*}
            \begin{amat}[r]{4}
              9  &2  &3  &12  &0  \\
              9  &0  &5  &12  &0  \\
              0  &1  &-4 &-1  &0
            \end{amat}
          \end{equation*}
          has more unknowns than equations, and so Gauss' method
          must end with at least one variable free (there can't be a 
          contradictory equation because the system is homogeneous, and so
          has at least the solution of all zeroes).
          To exhibit a combination, we can do the reduction 
          \begin{equation*}
            \grstep{-\rho_1+\rho_2}
            \;\grstep{(1/2)\rho_2+\rho_3}\;
            \begin{amat}[r]{4}
              9  &2  &3  &12  &0  \\
              0  &-2 &2  &0   &0  \\
              0  &0  &-3 &-1  &0
            \end{amat}
          \end{equation*}
          and take, say,  $c_4=1$.
          Then we have that $c_3=-1/3$, $c_2=-1/3$, and $c_1=-31/27$.
      \end{exparts}  
    \end{answer}
  \recommended \item 
    Which of these subsets of \( \polyspace_3 \) are
    linearly dependent and which are independent?
    \begin{exparts}
      \partsitem \( \set{3-x+9x^2,5-6x+3x^2,1+1x-5x^2} \)
      \partsitem \( \set{-x^2,1+4x^2} \)
      \partsitem \( \set{2+x+7x^2,3-x+2x^2,4-3x^2} \)
      \partsitem \( \set{8+3x+3x^2,x+2x^2,2+2x+2x^2,8-2x+5x^2} \)
    \end{exparts}
    \begin{answer}
      In the cases of independence, you must prove that it is independent.
      Otherwise, you must exhibit a dependence.
      (Here we give a specific dependence but others are possible.)
      \begin{exparts}
        \partsitem This set is independent.
          Setting up the relation
          \( c_1(3-x+9x^2)+c_2(5-6x+3x^2)+c_3(1+1x-5x^2)=0+0x+0x^2 \)
          gives a linear system 
          \begin{equation*}
            \begin{amat}[r]{3}
              3  &5  &1  &0  \\
              -1 &-6 &1  &0  \\
              9  &3  &-5 &0  
            \end{amat}
            \;\grstep[-3\rho_1+\rho_3]{(1/3)\rho_1+\rho_2}
            \;\grstep{3\rho_2}
            \;\grstep{-(12/13)\rho_2+\rho_3}\;
            \begin{amat}[r]{3}
              3  &5   &1        &0  \\
              0  &-13 &4        &0  \\
              0  &0   &-128/13  &0  
            \end{amat}
          \end{equation*}
          with only one solution: \( c_1=0 \), \( c_2=0 \), and \( c_3=0 \).
        \partsitem This set is independent.
           We can see this by inspection, straight from the definition
           of linear independence.
           Obviously neither is a multiple of the other.
        \partsitem This set is linearly independent.
           The linear system reduces in this way
           \begin{equation*}
             \begin{amat}[r]{3}
               2  &3  &4  &0  \\
               1  &-1 &0  &0  \\
               7  &2  &-3 &0  
             \end{amat}
             \;\grstep[-(7/2)\rho_1+\rho_3]{-(1/2)\rho_1+\rho_2}
             \;\grstep{-(17/5)\rho_2+\rho_3}\;
             \begin{amat}[r]{3}
               2  &3    &4      &0  \\
               0  &-5/2 &-2     &0  \\
               0  &0    &-51/5  &0  
             \end{amat}
           \end{equation*}
           to show that there is only the solution $c_1=0$, 
           $c_2=0$, and $c_3=0$.
        \partsitem This set is linearly dependent.
           The linear system
           \begin{equation*}
             \begin{amat}[r]{4}
               8  &0  &2  &8  &0  \\ 
               3  &1  &2  &-2 &0  \\
               3  &2  &2  &5  &0
             \end{amat}
           \end{equation*}
           must, after reduction, end with at least one variable free
           (there are more variables than equations, and there is no
           possibility of a contradictory equation because the system is
           homogeneous).
           We can take the free variables as parameters to describe the
           solution set.
           We can then set the parameter to a nonzero value to get a
           nontrivial linear relation. 
      \end{exparts}  
     \end{answer}
  \recommended \item
    Prove that each set \( \set{f,g} \) is linearly independent in the
    vector space of all functions from \( \Re^+ \) to \( \Re \).
    \begin{exparts}
      \partsitem \( f(x)=x \) and \( g(x)=1/x \)
      \partsitem \( f(x)=\cos(x) \) and \( g(x)=\sin(x) \)
      \partsitem \( f(x)=e^x \) and \( g(x)=\ln(x) \)
    \end{exparts}
    \begin{answer}
      Let $Z$ be the zero function $Z(x)=0$, which is the additive identity in
      the vector space under discussion.
      \begin{exparts}
        \partsitem This set is linearly independent.  
          Consider \( c_1\cdot f(x)+c_2\cdot g(x)=Z(x) \).
          Plugging in \( x=1 \) and \( x=2 \) gives a linear system 
          \begin{equation*}
            \begin{linsys}{2}
              c_1\cdot 1  &+  &c_2\cdot 1     &=  &0  \\
              c_1\cdot 2  &+  &c_2\cdot (1/2) &=  &0
            \end{linsys}
          \end{equation*}
          with the unique solution \( c_1=0 \), \( c_2=0 \).
        \partsitem This set is linearly independent.  
          Consider \( c_1\cdot f(x)+c_2\cdot g(x)=Z(x) \) and 
          plug in \( x=0 \) and \( x=\pi/2 \) to get 
          \begin{equation*}
            \begin{linsys}{2}
              c_1\cdot 1  &+  &c_2\cdot 0     &=  &0  \\
              c_1\cdot 0  &+  &c_2\cdot 1     &=  &0
            \end{linsys}
          \end{equation*}
          which obviously gives that \( c_1=0 \), \( c_2=0 \).
        \partsitem This set is also linearly independent.  
          Considering \( c_1\cdot f(x)+c_2\cdot g(x)=Z(x) \) and 
          plugging in \( x=1 \) and \( x=e \) 
          \begin{equation*}
            \begin{linsys}{2}
              c_1\cdot e    &+  &c_2\cdot 0     &=  &0  \\
              c_1\cdot e^e  &+  &c_2\cdot 1     &=  &0
            \end{linsys}
          \end{equation*}
          gives that \( c_1=0 \) and \( c_2=0 \).
      \end{exparts}  
     \end{answer}
  \recommended \item 
    Which of these subsets of the space of real-valued functions
    of one real variable is linearly dependent and which is linearly
    independent?
    (Note that we have abbreviated some constant functions;~e.g., 
     in the first item, 
     the `$2$' stands for the constant function $f(x)=2$.)
    \begin{exparts*}
      \partsitem \( \set{2,4\sin^2(x),\cos^2(x)} \)
      \partsitem \( \set{1,\sin(x),\sin(2x)} \)
      \partsitem \( \set{x,\cos(x)} \)
      \partsitem \( \set{(1+x)^2,x^2+2x,3} \)
      \partsitem \( \set{\cos(2x),\sin^2(x),\cos^2(x)} \)
      \partsitem \( \set{0,x,x^2} \)
    \end{exparts*}
    \begin{answer}
      In each case, if the set is independent then you must prove that, 
      and if it is
      dependent then you must exhibit a dependence.
      \begin{exparts}
        \partsitem This set is dependent.
          The familiar relation $\sin^2(x)+\cos^2(x)=1$ shows that
          $2=c_1\cdot(4\sin^2(x))+c_2\cdot(\cos^2(x))$ is satisfied by
          $c_1=1/2$ and $c_2=2$.
        \partsitem This set is independent.
          Consider the relationship
          $c_1\cdot 1+c_2\cdot\sin(x)+c_3\cdot\sin(2x)=0$
          (that `$0$' is the zero function).
          Taking $x=0$, $x=\pi/2$ and $x=\pi/4$ gives this system.
          \begin{equation*}
            \begin{linsys}{3}
               c_1  &   &                &   &      &=  &0  \\
               c_1  &+  &c_2             &   &      &=  &0  \\
               c_1  &+  &(\sqrt{2}/2)c_2 &+  &c_3   &=  &0  
            \end{linsys}
          \end{equation*}
          whose only solution is 
          $c_1=0$, $c_2=0$, and $c_3=0$. 
        \partsitem By inspection, this set is independent.
          Any dependence $\cos(x)=c\cdot x$ is not possible since the cosine
          function is not a multiple of the identity function
          (we are applying \nearbycorollary{cor:LDMeansLC}).
        \partsitem By inspection, we spot that there is a dependence.
          Because $(1+x)^2=x^2+2x+1$, we get that
          $c_1\cdot(1+x)^2+c_2\cdot(x^2+2x)=3$ is satisfied by 
          $c_1=3$ and $c_2=-3$.
        \partsitem This set is dependent.
          The easiest way to see that is to recall the trigonometric
          relationship $\cos^2(x)-\sin^2(x)=\cos(2x)$.
          (\textit{Remark.} 
          A person who doesn't recall this, and tries some $x$'s,
          simply never gets a system leading to a unique solution, and
          never gets to conclude that the set is independent.
          Of course, this person might wonder if they simply never tried the
          right set of $x$'s, but a few tries will lead most people to 
          look instead for a dependence.)
        \partsitem This set is dependent, because it contains the 
          zero object in the vector space, the zero polynomial.
      \end{exparts}  
     \end{answer}
  \item 
    Does the equation \( \sin^2(x)/\cos^2(x)=\tan^2(x) \) show that
    this set of functions
    \( \set{\sin^2(x),\cos^2(x),\tan^2(x)} \) is a linearly dependent
    subset of the set of all real-valued functions with domain
    the interval \( (-\pi/2..\pi/2) \) of real numbers between 
    \( -\pi/2 \) and \( \pi/2) \)?
    \begin{answer}
      No, that equation is not a linear relationship.
      In fact this set is independent, as the system arising from taking
      \( x \) to be \( 0 \), \( \pi/6 \) and \( \pi/4 \) shows.  
    \end{answer}
  \item  
    Why does \nearbylemma{le:LDIffANonTrivLinRel} say ``distinct''?
    \begin{answer}
      To emphasize that the equation 
      \( 1\cdot\vec{s}+(-1)\cdot\vec{s}=\zero \)
      does not make the set dependent.
    \end{answer}
  \recommended \item
    Show that the nonzero rows of an echelon form matrix form a linearly
    independent set.
    \begin{answer}
      We have already showed this: the Linear Combination
      Lemma and its corollary state that in an echelon form matrix, 
      no nonzero row is a linear combination of the others.  
    \end{answer}
  \recommended \item
     \begin{exparts}
       \partsitem Show that if the set \( \set{\vec{u},\vec{v},\vec{w}} \)
          is linearly independent set then so is the set
          \( \set{\vec{u},\vec{u}+\vec{v},\vec{u}+\vec{v}+\vec{w}} \).
       \partsitem  What is the relationship between the linear independence
         or dependence of the set \( \set{\vec{u},\vec{v},\vec{w}} \) and the
         independence or dependence of
         \( \set{\vec{u}-\vec{v},\vec{v}-\vec{w},\vec{w}-\vec{u}} \)?
     \end{exparts}
     \begin{answer}
       \begin{exparts}
         \partsitem Assume that the set 
           \( \set{\vec{u},\vec{v},\vec{w}} \) is linearly 
           independent, so that any relationship
           $d_0\vec{u}+d_1\vec{v}+d_2\vec{w}=\zero$ leads to the conclusion 
           that $d_0=0$, $d_1=0$, and $d_2=0$.

           Consider the relationship
           \( c_1(\vec{u})+c_2(\vec{u}+\vec{v})+c_3(\vec{u}+\vec{v}+\vec{w})
           =\zero \).
           Rewrite it to get
           \( (c_1+c_2+c_3)\vec{u}+(c_2+c_3)\vec{v}+(c_3)\vec{w}=\zero \).
           Taking $d_0$ to be $c_1+c_2+c_3$, taking $d_1$ to be $c_2+c_3$, 
           and taking $d_2$ to be $c_3$ we have this system.
           \begin{equation*}
             \begin{linsys}{3}
               c_1  &+  &c_2  &+  &c_3  &=  &0  \\
                    &   &c_2  &+  &c_3  &=  &0  \\
                    &   &     &   &c_3  &=  &0
             \end{linsys}
           \end{equation*}
           Conclusion:~the $c$'s are all zero, and so the set is linearly
           independent.
        \partsitem The second set is dependent
           \begin{equation*}
             1\cdot(\vec{u}-\vec{v})
             +1\cdot(\vec{v}-\vec{w})
             +1\cdot(\vec{w}-\vec{u})
             =\zero
           \end{equation*}
           whether or not the first set is independent.
      \end{exparts}
    \end{answer}
  \item 
    \nearbyexample{ex:EmSetLI} shows that the empty set is 
    linearly independent.   
    \begin{exparts}
      \partsitem When is a one-element set linearly independent?
      \partsitem How about a set with two elements?
    \end{exparts}
    \begin{answer}
      \begin{exparts}
        \partsitem A singleton set $\set{\vec{v}}$ is linearly independent 
          if and only if $\vec{v}\neq\zero$.
          For the `if' direction, with $\vec{v}\neq\zero$, 
          we can apply \nearbylemma{le:LDIffANonTrivLinRel} by considering 
          the relationship
          \( c\cdot\vec{v}=\zero \) and noting that the only solution
          is the trivial one:~$c=0$.
          For the `only~if' direction, just recall that 
          \nearbyexample{ex:SetWithZeroVecLD}
          shows that $\set{\zero}$ is linearly dependent, and so if the set
          $\set{\vec{v}}$ is linearly independent then $\vec{v}\neq\zero$. 

          (\textit{Remark.} 
          Another answer is to say that this is the special case of
          \nearbylemma{le:SUnionXiLIIffXiNotInSpan} where \( S=\emptyset \).)
        \partsitem A set with two elements is linearly independent 
          if and only if neither member is a  multiple of the other 
          (note that if one is the zero vector then it is a multiple of the
          other).
          This is an equivalent statement:~a set is linearly dependent if and
          only if one element is a multiple of the other.

          The proof is easy.
          A set $\set{\vec{v}_1,\vec{v}_2}$ is linearly dependent if and only
          if there is a relationship $c_1\vec{v}_1+c_2\vec{v}_2=\zero$ 
          with either $c_1\neq 0$ or $c_2\neq 0$ (or both).
          That holds if and only if $\vec{v}_1=(-c_2/c_1)\vec{v}_2$
          or $\vec{v}_2=(-c_1/c_2)\vec{v}_1$ (or both).
       \end{exparts}   
     \end{answer}
  \item  
    In any vector space \( V \), the empty set is linearly independent.
    What about all of \( V \)?
    \begin{answer}
      This set is linearly dependent set because it contains the zero vector.  
    \end{answer}
  \item 
    Show that if \( \set{\vec{x},\vec{y},\vec{z}} \) is linearly
    independent then so are all of its proper 
    subsets:~\( \set{\vec{x},\vec{y}} \),
    \( \set{\vec{x},\vec{z}} \), \( \set{\vec{y},\vec{z}} \),
    \( \set{\vec{x}} \),\( \set{\vec{y}} \), \( \set{\vec{z}} \),
    and \( \set{} \).
    Is that `only if' also?
    \begin{answer}
      \nearbylemma{le:SubsetPreserveLI} gives the `if' half.
      The converse (the `only if' statement) does not hold. 
      An example is to consider the vector space \( \Re^2 \) and
      these vectors.
      \begin{equation*}
         \vec{x}=\colvec{1 \\ 0},\quad
         \vec{y}=\colvec{0 \\ 1},\quad
         \vec{z}=\colvec{1 \\ 1}
      \end{equation*} 
    \end{answer}
  \item 
    \begin{exparts}
      \partsitem Show that this 
        \begin{equation*}
          S=\set{\colvec{1 \\ 1 \\ 0},\colvec{-1 \\ 2 \\ 0}}
        \end{equation*}
        is a linearly independent subset of \( \Re^3 \).
      \partsitem Show that
        \begin{equation*}
          \colvec{3 \\ 2 \\ 0}
        \end{equation*}
        is in the span of $S$ by finding \( c_1 \) and \( c_2 \) 
        giving a linear relationship.
        \begin{equation*}
          c_1\colvec{1 \\ 1 \\ 0}
          +c_2\colvec{-1 \\ 2 \\ 0}
          =\colvec{3 \\ 2 \\ 0}
        \end{equation*}
        Show that the pair \( c_1,c_2 \) is unique.
      \partsitem Assume that \( S \) is a subset of a vector space and that
        \( \vec{v} \) is in \( \spanof{S} \), so that \( \vec{v} \) is a
        linear combination of vectors from \( S \).
        Prove that if \( S \) is linearly independent then a linear combination
        of vectors from \( S \) adding to \( \vec{v} \)
        is unique (that is, unique up to reordering
        and adding or taking away terms of the form \( 0\cdot\vec{s} \)).
        Thus \( S \) as a spanning set is minimal in this strong sense:
        each vector in \( \spanof{S} \) is ``hit'' a minimum number of
        times\Dash only once.
      \partsitem
        Prove that it can happen when \( S \) is not linearly 
        independent that distinct linear combinations sum to the same vector.
    \end{exparts}
    \begin{answer}
      \begin{exparts}
        \partsitem The linear system arising from
          \begin{equation*}
            c_1\colvec{1 \\ 1 \\ 0}
            +c_2\colvec{-1 \\ 2 \\ 0}
            =\colvec{0 \\ 0 \\ 0}
          \end{equation*}
          has the unique solution \( c_1=0 \) and \( c_2=0 \).
        \partsitem The linear system arising from
          \begin{equation*}
            c_1\colvec{1 \\ 1 \\ 0}
            +c_2\colvec{-1 \\ 2 \\ 0}
            =\colvec{3 \\ 2 \\ 0}
          \end{equation*}
          has the unique solution \( c_1=8/3 \) and \( c_2=-1/3 \).
        \partsitem Suppose that \( S \) is linearly independent.
          Suppose that we have both $\vec{v}=c_1\vec{s}_1+\dots+c_n\vec{s}_n$
          and $\vec{v}=d_1\vec{t}_1+\dots+d_m\vec{t}_m$
          (where the vectors are members of $S$).
          Now, 
          \begin{equation*}
            c_1\vec{s}_1+\dots+c_n\vec{s}_n
            =\vec{v}
            =d_1\vec{t}_1+\dots+d_m\vec{t}_m
          \end{equation*}
          can be rewritten in this way.
          \begin{equation*}
            c_1\vec{s}_1+\dots+c_n\vec{s}_n
            -d_1\vec{t}_1-\dots-d_m\vec{t}_m
            =\zero
          \end{equation*}
          Possibly some of the $\vec{s}\,$'s equal some of the $\vec{t}\,$'s;
          we can combine the associated coefficients 
          (i.e., if $\vec{s}_i=\vec{t}_j$ then
          $\cdots+c_i\vec{s}_i+\dots-d_j\vec{t}_j-\cdots$ can be rewritten
          as $\cdots+(c_i-d_j)\vec{s}_i+\cdots$).
          That equation is a linear relationship among  
          distinct (after the combining is done) members of the set $S$.
          We've assumed that $S$ is linearly independent, so all of the 
          coefficients are zero.
          If $i$ is such that $\vec{s}_i$ does not equal any $\vec{t}_j$
          then $c_i$ is zero.
          If $j$ is such that $\vec{t}_j$ does not equal any $\vec{s}_i$
          then $d_j$ is zero.
          In the final case, we have that $c_i-d_j=0$ and so $c_i=d_j$.  

          Therefore, the original two sums are the same, except perhaps for
          some $0\cdot\vec{s}_i$ or $0\cdot\vec{t}_j$ terms that we can
          neglect.
        \partsitem
          This set is not linearly independent:
          \begin{equation*}
            S=\set{\colvec{1 \\ 0},\colvec{2 \\ 0}}\subset\Re^2
          \end{equation*}
          and these two linear combinations give the same result
          \begin{equation*}
            \colvec{0 \\ 0}=2\cdot\colvec{1 \\ 0}-1\cdot\colvec{2 \\ 0}
                            =4\cdot\colvec{1 \\ 0}-2\cdot\colvec{2 \\ 0}
          \end{equation*}
          Thus, a linearly dependent set might have indistinct sums.

          In fact, this stronger statement holds:~if a set is linearly 
          dependent then it must have the property that there are two 
          distinct linear combinations that sum to the same vector.
          Briefly, where \( c_1\vec{s}_1+\dots+c_n\vec{s}_n=\zero \) then
          multiplying both sides of the relationship by two gives another 
          relationship.
          If the first relationship is nontrivial then the second is also.
      \end{exparts}  
    \end{answer}
  \item  \label{exer:PolyZeroFcnOnlyIfZeroPol}
     Prove that a polynomial gives rise to the zero function if and only if
     it is the zero polynomial.
     (\textit{Comment.}
     This question is not a Linear Algebra matter, but we often use the result.
     A polynomial gives rise to a function in the natural 
     way:~$x\mapsto c_nx^n+\dots+c_1x+c_0$.)
     \begin{answer}
       In this `if and only if' statement, the `if' half is clear\Dash if 
       the polynomial is the zero polynomial then the function that arises 
       from the action of the polynomial must be the zero 
       function $x\mapsto 0$. 
       For `only if' we write $p(x)=c_nx^n+\dots+c_0$. 
       Plugging in zero $p(0)=0$ gives that $c_0=0$.
       Taking the derivative and plugging in zero $p^\prime(0)=0$ gives 
       that $c_1=0$.
       Similarly we get that each $c_i$ is zero, and $p$ is the zero 
       polynomial.
     \end{answer}
  \item
    Return to Section 1.2 and redefine point, line, plane,
    and other linear surfaces to avoid degenerate cases.
    \begin{answer}
      The work in this section suggests that we should define 
      an \( n \)-dimensional
      non-degenerate linear surface as the span of a linearly
      independent set of \( n \) vectors.  
    \end{answer}
  \item 
    \begin{exparts}  
      \partsitem Show that any set of four vectors in \( \Re^2 \) is 
         linearly dependent.
      \partsitem Is this true for any set of five?
         Any set of three?
      \partsitem What is the most number of elements that a 
         linearly independent subset of $\Re^2$ can have?
    \end{exparts}
    \begin{answer}
      \begin{exparts}
        \partsitem For any $a_{1,1}$, \ldots, $a_{2,4}$,
          \begin{equation*}
            c_1\colvec{a_{1,1} \\ a_{2,1}}
            +c_2\colvec{a_{1,2} \\ a_{2,2}}
            +c_3\colvec{a_{1,3} \\ a_{2,3}}
            +c_4\colvec{a_{1,4} \\ a_{2,4}}
            =\colvec{0 \\ 0}
          \end{equation*}
          yields a linear system
          \begin{equation*}
             \begin{linsys}{4}
              a_{1,1}c_1 &+ &a_{1,2}c_2 &+ &a_{1,3}c_3 &+ &a_{1,4}c_4 &= &0  \\
              a_{2,1}c_1 &+ &a_{2,2}c_2 &+ &a_{2,3}c_3 &+ &a_{2,4}c_4 &= &0  
             \end{linsys}
          \end{equation*}
        that has infinitely many solutions (Gauss' method leaves at least
        two variables free).
        Hence there are nontrivial linear relationships among the given
        members of $\Re^2$.
      \partsitem Any set five vectors is a superset of a set of four vectors,
        and so is linearly dependent.

        With three vectors from $\Re^2$, the argument from the prior item 
        still applies, with the slight change that Gauss' method now only 
        leaves at least one variable free (but that still gives infinitely many
        solutions).
      \partsitem The prior item shows that no three-element subset of $\Re^2$
        is independent.
        We know that there are two-element subsets of $\Re^2$ that are 
        independent\Dash one is
        \begin{equation*}
          \set{\colvec{1  \\ 0},\colvec{0  \\ 1}}
        \end{equation*} 
        and so the answer is two.
     \end{exparts}  
    \end{answer}
  \recommended \item
    Is there a set of four vectors in \( \Re^3 \), any three of which
    form a linearly independent set?
    \begin{answer}
      Yes; here is one.
      \begin{equation*}
         \set{\colvec{1 \\ 0 \\ 0},
              \colvec{0 \\ 1 \\ 0},
              \colvec{0 \\ 0 \\ 1},
              \colvec{1 \\ 1 \\ 1} }
      \end{equation*}  
    \end{answer}
  \item  
    Must every linearly dependent set have a subset that is dependent and
    a subset that is independent?
    \begin{answer}
      Yes.
      The two improper subsets, the entire set and the empty subset, serve as
      examples.  
    \end{answer}
  \item  
    In \( \Re^4 \), what is the biggest linearly independent
    set you can find?
    The smallest?
    The biggest linearly dependent set?
    The smallest?
    (`Biggest' and `smallest' mean that there are no supersets or subsets 
    with the same property.)
    \begin{answer} 
      In \( \Re^4 \) the biggest linearly independent set has
      four vectors.
      There are many examples of such sets, this is one.
      \begin{equation*}
        \set{\colvec{1 \\ 0 \\ 0 \\ 0},
             \colvec{0 \\ 1 \\ 0 \\ 0},
             \colvec{0 \\ 0 \\ 1 \\ 0},
             \colvec{0 \\ 0 \\ 0 \\ 1}  }
      \end{equation*}
      To see that no set with five or more vectors can be independent, set up
      \begin{equation*}
             c_1\colvec{a_{1,1} \\ a_{2,1} \\ a_{3,1} \\ a_{4,1}}
            +c_2\colvec{a_{1,2} \\ a_{2,2} \\ a_{3,2} \\ a_{4,2}}
            +c_3\colvec{a_{1,3} \\ a_{2,3} \\ a_{3,3} \\ a_{4,3}}
            +c_4\colvec{a_{1,4} \\ a_{2,4} \\ a_{3,4} \\ a_{4,4}}
            +c_5\colvec{a_{1,5} \\ a_{2,5} \\ a_{3,5} \\ a_{4,5}}
             =\colvec{0 \\ 0 \\ 0 \\ 0}
      \end{equation*}
      and note that the resulting linear system 
      \begin{equation*}
        \begin{linsys}{5}
          a_{1,1}c_1 &+ &a_{1,2}c_2 &+ &a_{1,3}c_3 
              &+ &a_{1,4}c_4 &+ &a_{1,5}c_5 &=  &0   \\
          a_{2,1}c_1 &+ &a_{2,2}c_2 &+ &a_{2,3}c_3 
              &+ &a_{2,4}c_4 &+ &a_{2,5}c_5 &=  &0   \\
          a_{3,1}c_1 &+ &a_{3,2}c_2 &+ &a_{3,3}c_3 
              &+ &a_{3,4}c_4 &+ &a_{3,5}c_5 &=  &0   \\
          a_{4,1}c_1 &+ &a_{4,2}c_2 &+ &a_{4,3}c_3 
              &+ &a_{4,4}c_4 &+ &a_{4,5}c_5 &=  &0     
        \end{linsys}
      \end{equation*}
      has four equations and five unknowns, 
      so Gauss' method must end with at least one \( c \) variable free,
      so there are infinitely many solutions,
      and so the above linear relationship among the four-tall vectors has
      more solutions than just the trivial solution.

      The smallest linearly independent set is the empty set.

      The biggest linearly dependent set is \( \Re^4 \).
      The smallest is \( \set{\zero} \).  
    \end{answer}
  \recommended \item 
    Linear independence and linear dependence are properties of sets.
    We can thus naturally ask how those properties act with respect to
    the familiar elementary set relations and operations.  
    In this body of this subsection we have covered the subset and superset
    relations.
    We can also consider the operations of intersection, complementation, 
    and union.
    \begin{exparts}
       \partsitem How does linear independence relate to intersection:~can
         an intersection of linearly independent sets be independent?
         Must it be?
       \partsitem How does linear independence relate to complementation?
       \partsitem Show that the union of two linearly independent sets
          can be linearly independent.
       \partsitem Show that
          the union of two linearly independent sets need not be
          linearly independent.
    \end{exparts}
    \begin{answer}
      \begin{exparts}
        \partsitem The intersection of two linearly independent sets
          $S\intersection T$ must be linearly
          independent as it is a subset of the linearly independent set $S$
          (as well as the linearly independent set $T$ also, of course).
        \partsitem The complement of a linearly independent set is linearly
          dependent as it contains the zero vector.
        \partsitem A simple example in \( \Re^2 \) is these two sets.
          \begin{equation*}
            S=\set{\colvec{1 \\ 0}}
            \qquad
            T=\set{\colvec{0 \\ 1}}          
          \end{equation*}
          A somewhat subtler example, again in \( \Re^2 \), is these two.
          \begin{equation*}
            S=\set{\colvec{1 \\ 0}}
            \qquad
            T=\set{\colvec{1 \\ 0}, \colvec{0 \\ 1}}
          \end{equation*}
        \partsitem We must produce an example.
          One, in \( \Re^2 \), is
          \begin{equation*}
            S=\set{\colvec{1 \\ 0}}
            \qquad
            T=\set{\colvec{2 \\ 0}}
          \end{equation*}
          since the linear dependence of \( S_1\union S_2 \) is easy to see.
      \end{exparts}
    \end{answer}
  \item \textit{Continued from prior exercise.} 
    What is the interaction between the property of linear independence
    and the operation of union?
    \begin{exparts}
       \partsitem We might conjecture that the union $S\union T$ of 
          linearly independent sets 
          is linearly independent if and only if their spans have a trivial 
          intersection $\spanof{S}\intersection \spanof{T}=\set{\zero}$.
          What is wrong with this argument for the `if' direction
          of that conjecture? 
          ``If the union $S\union T$ is linearly independent then
          the only solution to          
          $c_1\vec{s}_1+\cdots+c_n\vec{s}_n
            +d_1\vec{t}_1+\cdots+d_m\vec{t}_m
            =\zero$
          is the trivial one $c_1=0$, \ldots, $d_m=0$. 
          So any member of the intersection of the spans
          must be the zero vector because in 
          $c_1\vec{s}_1+\cdots+c_n\vec{s}_n
            =d_1\vec{t}_1+\cdots+d_m\vec{t}_m$
          each scalar is zero.''
       \partsitem Give an example showing that the conjecture is false.
       \partsitem Find linearly independent sets \( S \) and \( T \)
          so that the union of \( S-(S\cap T)\) and \( T-(S\cap T) \)
          is linearly independent, but the union \( S\cup T \) is
          not linearly independent.
       \partsitem Characterize when the union of two linearly independent sets
          is linearly independent, in terms of the intersection of spans.
    \end{exparts}
    \begin{answer}
      \begin{exparts}
        \partsitem \nearbylemma{le:LDIffANonTrivLinRel} 
          requires that the vectors 
          $\vec{s}_1,\ldots,\vec{s}_n,\vec{t}_1,\ldots,\vec{t}_m$
          be distinct. 
          But we could have that the union $S\cup T$ is linearly 
          independent with some $\vec{s}_i$ equal to some $\vec{t}_j$.
        \partsitem One example in \( \Re^2 \) is these two.
          \begin{equation*}
            S=\set{\colvec{1 \\ 0}}
            \qquad
            T=\set{\colvec{1 \\ 0}, \colvec{0 \\ 1}}
          \end{equation*}
        \partsitem 
          An example from \( \Re^2 \) is these sets.
          \begin{equation*}
            S=\set{\colvec{1 \\ 0}, \colvec{0 \\ 1}}
            \qquad
            T=\set{\colvec{1 \\ 0}, \colvec{1 \\ 1}}
          \end{equation*}
        \partsitem The union of two linearly independent sets $S\union T$
          is linearly independent if and only if their spans of \( S \) 
          and \( T-(S\cap T)\) have a trivial intersection 
          $\spanof{S}\intersection \spanof{T-(S\cap T)}=\set{\zero}$.
          To prove that, assume that \( S \) and \( T \) are linearly 
          independent subsets of some vector space.

          For the `only if' direction, assume that the intersection of
          the spans is trivial 
          \( \spanof{S}\intersection \spanof{T-(S\cap T)}=\set{\zero} \).
          Consider the set $S\union (T-(S\cap T))=S\union T$ and
          consider the linear relationship 
          $c_1\vec{s}_1+\dots+c_n\vec{s}_n
            +d_1\vec{t}_1+\dots+d_m\vec{t}_m=\zero$.
          Subtracting gives
          $c_1\vec{s}_1+\dots+c_n\vec{s}_n=
            -d_1\vec{t}_1-\dots-d_m\vec{t}_m$.
          The left side of that equation sums to a vector in $\spanof{S}$, and
          the right side is a vector in $\spanof{T-(S\cap T)}$.
          Therefore, since the intersection of the spans is trivial, both
          sides equal the zero vector.
          Because $S$ is linearly independent, all of the $c$'s are zero.
          Because $T$ is linearly independent so also is  
          $T-(S\cap T)$ linearly independent, 
          and therefore all of the $d$'s are zero.
          Thus, the original linear relationship among members of 
          $S\union T$ only holds if all of the coefficients are zero.
          Hence, $S\union T$ is linearly independent.

          For the `if' half we can make the same argument in reverse.
          Suppose that the union $S\union T$ is linearly independent.
          Consider a linear relationship among members of 
          $S$ and $T-(S\cap T)$.        
          $c_1\vec{s}_1+\cdots+c_n\vec{s}_n
            +d_1\vec{t}_1+\cdots+d_m\vec{t}_m
            =\zero$
          Note that no $\vec{s}_i$ is equal to a $\vec{t}_j$
          so that is a combination of distinct vectors, as required by 
          \nearbylemma{le:LDIffANonTrivLinRel}.
          So the only solution 
          is the trivial one $c_1=0$, \ldots, $d_m=0$. 
          Since any vector $\vec{v}$ 
          in the intersection of the spans
          $\spanof{S}\intersection \spanof{T-(S\cap T)}$ 
          we can write 
          $\vec{v}=c_1\vec{s}_1+\cdots+c_n\vec{s}_n
            =-d_1\vec{t}_1-\cdots-d_m\vec{t}_m$,
          and it must be the zero vector because each scalar is zero.
      \end{exparts}  
    \end{answer}
  \recommended \item \label{exer:FillIndDetProofSetHasLISub}
    For \nearbytheorem{th:AlwaysAnLDSubset},
    \begin{exparts}
       \partsitem fill in the induction for the proof;
       \partsitem give an alternate proof that starts with the empty
         set and builds
         a sequence of linearly independent subsets of the given finite set
         until one appears with the same span as the given set.
    \end{exparts}
    \begin{answer}
      \begin{exparts}
         \partsitem We do induction on the number of vectors in the finite set
           \( S \).

           The base case is that $S$ has no elements.
           In this case $S$ is linearly independent and there is nothing to 
           check\Dash a subset of $S$ that has the same span as $S$ is $S$
           itself.

           For the inductive step assume that the theorem is true for all 
           sets of size $n=0$, $n=1$, \ldots, $n=k$ 
           in order to prove that it holds when \( S \) has $n=k+1$ elements.
           If the $k+1$-element set \( S=\set{\vec{s}_0,\dots,\vec{s}_{k}} \) 
           is linearly independent then the theorem is trivial,
           so assume that it is dependent.
           By \nearbycorollary{cor:LDMeansLC} there is an \( \vec{s}_i \)
           that is a linear combination of other vectors in \( S \).
           Define \( S_1=S-\set{\vec{s}_i} \) and note that 
           \( S_1 \) has the same span as \( S \) by
           \nearbylemma{le:VecInSpanIffSpanUnchByAddVec}.
           The set \( S_1 \) has \( k \) elements and 
           so the inductive hypothesis
           applies to give that it has a linearly independent subset with the 
           same span.
           That subset of \( S_1 \) is the desired subset of \( S \).
         \partsitem Here is a sketch of the argument.
           We have left out the induction argument details.

           If the finite set \( S \) is empty then there is nothing to prove.
           If \( S=\set{\zero} \) then the empty subset will do.

           Otherwise, take some nonzero vector \( \vec{s}_1\in S \)
           and define \( S_1=\set{\vec{s}_1} \).
           If \( \spanof{S_1}=\spanof{S} \) then
           we are finished with this proof by noting that \( S_1 \) is linearly
           independent.

           If not, then there is a nonzero
           vector \( \vec{s}_2\in S-\spanof{S_1} \)
           (if every \( \vec{s}\in S \) is in \( \spanof{S_1} \) then
           \( \spanof{S_1}=\spanof{S} \)).
           Define \( S_2=S_1\union\set{\vec{s}_2} \).
           If \( \spanof{S_2}=\spanof{S} \) then 
           we are finished 
           by using \nearbytheorem{cor:LDMeansLC}
           to show that \( S_2 \) is linearly independent.

           Repeat the last paragraph until a set with a big enough
           span appears.
           That must eventually happen
           because \( S \) is finite, and
           \( \spanof{S} \) will be reached at worst when we have
           used every vector from
           \( S \).
      \end{exparts}  
    \end{answer}
  \item 
     With a little calculation we can get formulas to determine whether or
     not a set of vectors is linearly independent. 
     \begin{exparts}
       \partsitem Show that this subset of \( \Re^2 \)
         \begin{equation*}
           \set{\colvec{a \\ c},\colvec{b \\ d}}
         \end{equation*}
         is linearly independent if and only if \( ad-bc\neq 0 \).
       \partsitem Show that this subset of \( \Re^3 \)
         \begin{equation*}
           \set{\colvec{a \\ d \\ g},
                \colvec{b \\ e \\ h},
                \colvec{c \\ f \\ i}  }
         \end{equation*}
         is linearly independent iff
         \( aei+bfg+cdh-hfa-idb-gec \neq 0 \).
       \partsitem When is this subset of \( \Re^3 \)
         \begin{equation*}
           \set{\colvec{a \\ d \\ g},
                \colvec{b \\ e \\ h} }
         \end{equation*}
         linearly independent?
       \partsitem This is an opinion question:~for
         a set of four vectors from \( \Re^4 \),
         must there be a formula involving the sixteen entries 
         that determines independence of the set?
         (You needn't produce such a formula, just decide if one exists.)
    \end{exparts}
    \begin{answer}
      \begin{exparts}
        \partsitem 
          Assuming first that \( a\neq 0 \),
          \begin{equation*}
            x\colvec{a \\ c}
            +y\colvec{b \\ d}
            =\colvec{0 \\ 0}
          \end{equation*}
          gives
          \begin{equation*}
            \begin{linsys}{2}
               ax  &+  &by &=  &0  \\
               cx  &+  &dy &=  &0  
            \end{linsys}
            \;\grstep{-(c/a)\rho_1+\rho_2}\;
            \begin{linsys}{2}
               ax  &+  &by           &=  &0  \\
                   &   &(-(c/a)b+d)y &=  &0  
             \end{linsys}
          \end{equation*}
          which has a solution if and only if
          \( 0\neq-(c/a)b+d=(-cb+ad)/d \)
          (we've assumed in this case that \( a\neq 0 \), and so 
          back substitution yields a unique solution).

          The \( a=0 \) case is also not hard\Dash break it into the 
          \( c\neq 0 \) and \( c=0 \) subcases and 
          note that in these cases \( ad-bc=0\cdot d-bc \).

          \textit{Comment.}
          An earlier exercise showed that a two-vector set is linearly
          dependent if and only if either vector is a scalar multiple of the
          other.
          We could also use that to make the calculation.
        \partsitem The equation
          \begin{equation*}
            c_1\colvec{a \\ d \\ g}
            +c_2\colvec{b \\ e \\ h}
            +c_3\colvec{c \\ f \\ i}
            =\colvec{0 \\ 0 \\ 0}
          \end{equation*}
         gives rise to a homogeneous linear system.
         We proceed by writing it in matrix form and applying Gauss' method.

         We first reduce the matrix to upper-triangular.
         Assume that \( a\neq 0 \).
         \begin{eqnarray*}
           \grstep{(1/a)\rho_1}
           \begin{amat}{3}
              1   &b/a   &c/a  &0 \\
              d   &e     &f    &0 \\
              g   &h     &i    &0
            \end{amat}                                            
           &\grstep[-g\rho_1+\rho_3]{-d\rho_1+\rho_2}
           &\begin{amat}{3}
              1   &b/a           &c/a        &0   \\
              0   &(ae-bd)/a     &(af-cd)/a  &0   \\
              0   &(ah-bg)/a     &(ai-cg)/a  &0
            \end{amat}                                            \\
           &\grstep{(a/(ae-bd))\rho_2}
           &\begin{amat}{3}
              1   &b/a           &c/a             &0  \\
              0   &1             &(af-cd)/(ae-bd) &0  \\
              0   &(ah-bg)/a     &(ai-cg)/a       &0
            \end{amat}
         \end{eqnarray*}
         (where we've assumed for the moment that \( ae-bd\neq 0 \) in order
         to do the row reduction step).
         Then, under the assumptions, we get this.
         \begin{eqnarray*}
           &\grstep{((ah-bg)/a)\rho_2+\rho_3}
           &\begin{amat}{3}
              1   &\frac{b}{a}   &\frac{c}{a}                           &0 \\
              0   &1             &\frac{af-cd}{ae-bd}                   &0 \\
              0   &0             &\frac{aei+bgf+cdh-hfa-idb-gec}{ae-bd} &0
            \end{amat}
         \end{eqnarray*}
         shows that the original system is nonsingular
         if and only if the \( 3,3 \) entry is nonzero.
         This fraction is defined because of the \( ae-bd\neq 0 \) assumption,
         and it will equal zero if and only if its numerator equals zero. 

         We next worry about the assumptions.
         First, if \( a\neq 0 \) but \( ae-bd=0 \) then we swap
         \begin{eqnarray*}
           \begin{amat}{3}
              1   &b/a           &c/a        &0   \\
              0   &0             &(af-cd)/a  &0   \\
              0   &(ah-bg)/a     &(ai-cg)/a  &0
            \end{amat}                               
           &\grstep{\rho_2\leftrightarrow\rho_3}
           &\begin{amat}{3}
              1   &b/a           &c/a        &0   \\
              0   &(ah-bg)/a     &(ai-cg)/a  &0   \\
              0   &0             &(af-cd)/a  &0
            \end{amat}
         \end{eqnarray*}
         and conclude that the system is nonsingular if and only if either
         \( ah-bg=0 \) or \( af-cd=0 \).
         That's the same as asking that their product be zero:
         \begin{align*}
            ahaf-ahcd-bgaf+bgcd
            &=0                   \\
            ahaf-ahcd-bgaf+aegc
            &=0                   \\
            a(haf-hcd-bgf+egc)
            &=0
         \end{align*}
         (in going from the first line to the second we've applied the
         case assumption that $ae-bd=0$ by substituting $ae$ for $bd$).
         Since we are assuming that \( a\neq 0 \), 
         we have that \( haf-hcd-bgf+egc=0 \).
         With $ae-bd=0$ we can rewrite this to fit the form we need:~in
         this \( a\neq 0 \) and \( ae-bd=0 \) case, the given system
         is nonsingular when
         \( haf-hcd-bgf+egc-i(ae-bd)=0 \), as required.

         The remaining cases have the same character.
         Do the \( a=0 \) but \( d\neq 0 \) case and the \( a=0 \) and
         \( d=0 \) but \( g\neq 0 \) case by first swapping rows and
         then going on as above.
         The \( a=0 \), \( d=0 \), and \( g=0 \) case is easy\Dash a set with a
         zero vector is linearly dependent, and the formula comes out
         to equal zero.
       \partsitem It is linearly dependent if and only if either vector is a
         multiple of the other.
         That is, it is not independent iff
         \begin{equation*}
           \colvec{a \\ d \\ g}=r\cdot\colvec{b \\ e \\ h}
           \quad\text{or}\quad
           \colvec{b \\ e \\ h}=s\cdot\colvec{a \\ d \\ g}
         \end{equation*}
         (or both) for some scalars $r$ and $s$.
         Eliminating $r$ and $s$ in order to restate this condition only in
         terms of the given letters $a$, $b$, $d$, $e$, $g$, $h$, we have that 
         it is not independent\Dash it is dependent\Dash iff
         \( ae-bd=ah-gb=dh-ge \).
       \partsitem Dependence or independence is a function of the
         indices, so there
         is indeed a formula (although at first glance a person might think
         the formula involves cases: ``if the first component of the first
         vector is zero then \ldots'', this guess turns out not to be 
         correct).
      \end{exparts}  
    \end{answer}
  \recommended \item  
    \begin{exparts}
      \partsitem Prove that a set of two perpendicular 
        nonzero vectors from
        \( \Re^n \) is linearly independent when \( n>1 \).
      \partsitem What if \( n=1 \)?
        \( n=0 \)?
      \partsitem Generalize to more than two vectors.
    \end{exparts}
    \begin{answer}
      Recall that two vectors from \( \Re^n \) are perpendicular if and
      only if their dot product is zero.
      \begin{exparts}
         \partsitem Assume that \( \vec{v} \) and \( \vec{w} \) are
           perpendicular nonzero vectors in $\Re^n$, with \( n>1 \).
           With the linear relationship \( c\vec{v}+d\vec{w}=\zero \), 
           apply \( \vec{v} \) to both
           sides to conclude that \( c\cdot\norm{\vec{v}}^2+d\cdot 0=0 \).
           Because \( \vec{v}\neq\zero \) we have that \( c=0 \).
           A similar application of \( \vec{w} \) shows that \( d=0 \).
         \partsitem Two vectors in \( \Re^1 \) are perpendicular if and only if
           at least one of them is zero.

           We define \( \Re^0 \) to be a trivial space, and so both $\vec{v}$
           and $\vec{w}$ are the zero vector.
         \partsitem The right generalization is to look at a set
           \( \set{\vec{v}_1,\dots,\vec{v}_n}\subseteq\Re^k \) of vectors
           that are \definend{mutually orthogonal} 
           (also called \definend{pairwise perpendicular}):~if
           \( i\neq j \) then \( \vec{v}_i \) is perpendicular to
           \( \vec{v}_j \).
           Mimicking the proof of the first item above shows that such a set of
           nonzero vectors is linearly independent.
      \end{exparts}  
    \end{answer}
  \item 
    Consider the set of functions from the open interval $(-1..1)$ 
    to $\Re$.
    \begin{exparts}
      \partsitem Show that 
         this set is a vector space under the usual operations.
      \partsitem Recall the formula for the sum of an infinite geometric 
         series: 
         \( 1+x+x^2+\cdots=1/(1-x) \) for all \( x\in(-1..1) \).
         Why does this not express a dependence inside of
         the set $\set{g(x)=1/(1-x),f_0(x)=1,f_1(x)=x,f_2(x)=x^2,\ldots}$
         (in the vector space that we are considering)?
         (\textit{Hint.}
         Review the definition of linear combination.)
      \partsitem Show that the set in the prior item is linearly independent.
    \end{exparts}
    This shows that some vector spaces exist with linearly independent subsets
    that are infinite.
    \begin{answer}
      \begin{exparts}
        \partsitem This check is routine.
        \partsitem The summation is infinite (has infinitely many summands).
          The definition of linear combination involves only finite sums.
        \partsitem No nontrivial finite sum of members of 
           \( \set{g,f_0,f_1,\ldots} \) adds to the zero object:~assume that
           \begin{equation*}
              c_0\cdot (1/(1-x))+c_1\cdot 1+\dots+c_n\cdot x^n=0
           \end{equation*}
           (any finite sum uses a highest power, here \( n \)).
           Multiply both sides by \( 1-x \) to conclude that each coefficient 
           is zero, because a polynomial describes the zero function only when 
           it is the zero polynomial.
      \end{exparts}
     \end{answer}
  \item 
    Show that, where \( S \) is a subspace of \( V \), if a subset $T$ of
    \( S \) is linearly independent in \( S \) then $T$ is also linearly
    independent in \( V \).
    Is that `only if'?
    \begin{answer}
      It is both `if' and `only if'.

      Let \( T \) be a subset of the subspace \( S \) of the vector space
      \( V \).
      The assertion that any linear relationship 
      $c_1\vec{t}_1+\dots+c_n\vec{t}_n=\zero$ among members of \( T \)
      must be the trivial relationship $c_1=0$, \ldots, $c_n=0$
      is a statement that 
      holds in \( S \) if and only if it holds in \( V \),
      because the subspace \( S \) inherits its addition and 
      scalar multiplication operations from \( V \).  
    \end{answer}
\end{exercises}
