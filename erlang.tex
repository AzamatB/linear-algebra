% Chapter 3, Topic _Linear Algebra_ Jim Hefferon
%  http://joshua.smcvt.edu/linalg.html
%  2001-Jun-12
\topic{Orthonormal Matrices}
\index{orthonormal matrix|(}
\index{matrix!orthonormal|(}
In \emph{The Elements},\index{Euclid} Euclid considers two figures to be
the same if they have the same size and shape.
That is, while the triangles below are not equal because they are not the same 
set of points,
they are \definend{congruent}\index{congruent figures}\Dash essentially 
indistinguishable
for Euclid's purposes\Dash because we can imagine 
picking the plane up,
sliding it over and rotating it a bit,
although not warping or stretching it,    
and then putting it back down, to superimpose the first figure on
the second.
(Euclid never explicitly states this principle 
but he uses it often \cite{Casey}.)
\begin{center}
  \includegraphics{ch3.61}
\end{center}
In modern terminology, ``picking the plane up~\ldots'' 
is considering a
map from the plane to itself. 
Euclid considers only  
transformations of the plane
that may slide or turn the plane but not bend or stretch it.
Accordingly, we define a map $\map{f}{\Re^2}{\Re^2}$ to be 
\definend{distance-preserving}\index{distance-preserving}%
\index{map!distance-preserving} 
or a \definend{rigid motion}\index{rigid motion} or an
\definend{isometry},\index{isometry}
if for all points $P_1,P_2\in\Re^2$,
the distance from $f(P_1)$ to $f(P_2)$ equals the distance from
$P_1$ to $P_2$. 
We also define a plane \definend{figure}\index{plane figure} 
to be a set of points in the plane
and we say that two figures are 
\definend{congruent}\index{congruent plane figures}%
\index{plane figure!congruence} if there is a 
distance-preserving map from the plane to itself that carries one figure 
onto the other.

Many statements from Euclidean geometry 
follow easily from these definitions.
Some are: (i)~collinearity is invariant under any distance-preserving map
(that is, if $P_1$, $P_2$, and $P_3$ are collinear then so are
$f(P_1)$, $f(P_2)$, and $f(P_3)$),
(ii)~betweeness is invariant under any distance-preserving map
(if $P_2$ is between $P_1$ and $P_3$ then so is
$f(P_2)$ between $f(P_1)$ and $f(P_3)$),
(iii)~the property of being a triangle is invariant under 
any distance-preserving map 
(if a figure is a triangle then the image of that figure is also a triangle),
(iv)~and the property of being a circle is invariant under any
distance-preserving map.
In 1872, F.~Klein\index{Klein, F.} suggested that we can define 
Euclidean geometry as the study of properties that
are invariant under these maps.
(This forms part of Klein's Erlanger\index{Erlanger Program} Program, 
which proposes the organizing principle that we can describe each kind of 
geometry\Dash Euclidean, projective, etc.\Dash 
as the study of the properties that are
invariant under some group of transformations.
The word `group' here means more than just `collection', 
but that lies outside of our scope.)

We can use linear algebra to characterize the distance-preserving maps
of the plane.

We must first observe that
there are distance-preserving transformations of the plane that are not linear.
The obvious example is this \emph{translation}.\index{translation}
\begin{equation*}
  \colvec{x \\ y}
  \quad\mapsto\quad
  \colvec{x \\ y}+\colvec[r]{1 \\ 0}=\colvec{x+1 \\ y}
\end{equation*}
However, 
this example turns out to be the only example, in the 
sense that if $f$ is distance-preserving and sends $\zero$ to $\vec{v}_0$ 
then the map $\vec{v}\mapsto f(\vec{v})-\vec{v}_0$ is linear.
That will follow immediately from this statement:~a map $t$ that is 
distance-preserving and sends $\zero$ to itself is linear.
To prove this equivalent statement, let
\begin{equation*}
  t(\vec{e}_1)=\colvec{a \\ b}
  \qquad
  t(\vec{e}_2)=\colvec{c \\ d}
\end{equation*}
for some $a,b,c,d\in\Re$. 
Then to show that $t$ is linear we can show that 
it can be represented by a matrix, that is, that $t$ acts in this way
for all $x,y\in\Re$.
\begin{equation*}
  \vec{v}=\colvec{x  \\  y}
  \mapsunder{t}
  \colvec{ax+cy  \\  bx+dy}
\tag*{\text{($*$)}}\end{equation*}
Recall that if we fix three non-collinear points 
then we can determine any point 
by giving its distance from those three.
So we can determine any point $\vec{v}$ in the domain by its distance from
$\zero$, $\vec{e}_1$, and $\vec{e}_2$. 
Similarly, we can determine any point $t(\vec{v})$ 
in the codomain by its distance from
the three fixed points $t(\zero)$, $t(\vec{e}_1)$, and $t(\vec{e}_2)$
(these three are not collinear because, as mentioned above, 
collinearity is invariant and
$\zero$, $\vec{e}_1$, and $\vec{e}_2$ are not collinear).
In fact, because $t$ is distance-preserving, we can say more:~for the 
point $\vec{v}$ in the plane that is determined by being 
the distance $d_0$ from $\zero$, 
the distance $d_1$ from $\vec{e}_1$, and the distance $d_2$ from $\vec{e}_2$,  
its image $t(\vec{v})$ must be the unique point in the codomain 
that is determined by being $d_0$ from $t(\zero)$, 
$d_1$ from $t(\vec{e}_1)$, 
and $d_2$ from $t(\vec{e}_2)$.
Because of the uniqueness, 
checking that the action in ($*$) works in the 
$d_0$, $d_1$, and $d_2$ cases
\begin{equation*}
  \dist(\colvec{x \\ y},\zero) 
  =\dist(t(\colvec{x \\ y}),t(\zero))
  =\dist(\colvec{ax+cy \\ bx+dy},\zero)
\end{equation*}
(we assumed that $t$ maps $\zero$ to itself)
\begin{equation*}
  \dist(\colvec{x \\ y},\vec{e}_1)
  =\dist(t(\colvec{x \\ y}),t(\vec{e}_1))
  =\dist(\colvec{ax+cy \\ bx+dy},\colvec{a \\ b})  
\end{equation*}
and
\begin{equation*}
  \dist(\colvec{x \\ y},\vec{e}_2)
  =\dist(t(\colvec{x \\ y}),t(\vec{e}_2))
  =\dist(\colvec{ax+cy \\ bx+dy},\colvec{c \\ d})
\end{equation*}
suffices to show that ($*$) describes $t$.
Those checks are routine.

Thus we can write any distance-preserving $\map{f}{\Re^2}{\Re^2}$ as
$f(\vec{v})=t(\vec{v})+\vec{v}_0$ for some constant vector $\vec{v}_0$
and linear map $t$ that is distance-preserving.
So what is left in order to understand distance-preserving maps is to
understand distance-preserving linear maps.

Not every linear map is distance-preserving. 
For example
$\vec{v}\mapsto 2\vec{v}$ does not preserve distances.

But there is a neat characterization:~a linear transformation $t$ of the
plane is distance-preserving if and only if both 
$\norm{t(\vec{e}_1)}=\norm{t(\vec{e}_2)}=1$, 
and $t(\vec{e}_1)$ is orthogonal to $t(\vec{e}_2)$.
The `only if' half of that statement is easy\Dash because $t$ is 
distance-preserving it must preserve the lengths of vectors 
and because $t$ is distance-preserving the Pythagorean theorem shows
that it must preserve orthogonality.
To show the `if' half we can check that the map preserves lengths
of vectors because then for all
$\vec{p}$ and $\vec{q}$ the distance between the two is preserved
$\norm{t(\vec{p}-\vec{q}\,)}=\norm{t(\vec{p})-t(\vec{q}\,)}
=\norm{\vec{p}-\vec{q}\,}$.
For that check let
\begin{equation*}
  \vec{v}=\colvec{x \\ y} 
  \quad 
  t(\vec{e}_1)=\colvec{a \\ b}
  \quad
  t(\vec{e}_2)=\colvec{c \\ d}
\end{equation*}
and with the `if' assumptions that
$a^2+b^2=c^2+d^2=1$ and $ac+bd=0$ we have this.
\begin{align*}
  \norm{t(\vec{v}\,)}^2
  &= (ax+cy)^2+(bx+dy)^2  \\
  &= a^2x^2+2acxy+c^2y^2+b^2x^2+2bdxy+d^2y^2 \\
  &= x^2(a^2+b^2)+y^2(c^2+d^2)+2xy(ac+bd)  \\
  &= x^2 + y^2 \\
  &= \norm{\vec{v}\,}^2
\end{align*}

One thing that is neat
about this characterization is that we can easily recognize
matrices that represent such a map with respect to the standard bases:
the columns
are of length one and are mutually orthogonal.
This is an 
\definend{orthonormal matrix}\index{orthonormal matrix}%
\index{matrix!orthonormal}
or
\definend{orthogonal matrix}\index{orthogonal matrix}\index{matrix!orthogonal}
(people often use the second term to mean not just that the columns are
orthogonal but also that they have length one). 

We can use this to
understand the geometric actions of
distance-preserving maps.
Because $\norm{t(\vec{v}\,)}=\norm{\vec{v}\,}$, the map~$t$ 
sends any $\vec{v}$ somewhere on the circle about the origin that has
radius equal to the length of $\vec{v}$.
In particular,
$\vec{e}_1$ and $\vec{e}_2$ map to the unit circle.
What's more, %because of the orthogonality restriction,
once we fix the unit vector $\vec{e}_1$ as mapped
to the vector with components $a$ and $b$ then there are only two places  
where $\vec{e}_2$ can go if its image is to be perpendicular 
to the first vector's image:~it can map either to one
where $\vec{e}_2$ maintains its position a quarter circle clockwise
from $\vec{e}_1$
\begin{center}
  \begin{tabular}{@{}c@{}}\includegraphics{ch3.62}\end{tabular}
  \qquad
  $\rep{t}{\stdbasis_2,\stdbasis_2}=
  \begin{mat}
    a  &-b  \\
    b  &a
  \end{mat}$ 
\end{center}
or to one where it goes a quarter circle counterclockwise.
\begin{center}
  \begin{tabular}{@{}c@{}}\includegraphics{ch3.63}\end{tabular}
  \qquad
  $\rep{t}{\stdbasis_2,\stdbasis_2}=
  \begin{mat}
    a  &b  \\
    b  &-a
  \end{mat}$
\end{center}

We can geometrically describe these two cases.
Let $\theta$ be the counterclockwise
angle between the $x$-axis and the image of $\vec{e}_1$.
The first matrix above represents, with respect to the standard bases,
a \definend{rotation}\index{rotation}\index{linear map!rotation} 
of the plane by $\theta$ radians.
\begin{center}
  \begin{tabular}{@{}c@{}}\includegraphics{ch3.62}\end{tabular}
  \qquad
  $\colvec{x \\ y}
  \mapsunder{t}
  \colvec{x\cos\theta-y\sin\theta  \\ x\sin\theta+y\cos\theta}$ 
\end{center}
The second matrix above represents 
a \definend{reflection}\index{reflection}\index{linear map!reflection} 
of the plane through the line 
bisecting the angle between $\vec{e}_1$ and $t(\vec{e}_1)$.
\begin{center}
  \begin{tabular}{@{}c@{}}\includegraphics{ch3.64}\end{tabular}
  \qquad
  $\colvec{x \\ y}\mapsunder{t}
  \colvec{x\cos\theta+y\sin\theta  \\  x\sin\theta-y\cos\theta}$ 
\end{center}
(This picture shows $\vec{e}_1$ reflected up into the first 
quadrant and $\vec{e}_2$ reflected down into the fourth quadrant.)

Note: in the domain  
the angle between $\vec{e}_1$ and $\vec{e}_2$ runs counterclockwise,
and in the first map above the angle from 
$t(\vec{e}_1)$ to $t(\vec{e}_2)$ is also counterclockwise,
so it preserves the orientation of the angle.
But the second map reverses the orientation.
A distance-preserving map is 
\definend{direct}\index{direct map}\index{orientation preserving map} if it 
preserves orientations
and \definend{opposite}\index{opposite map}\index{orientation reversing map} 
if it reverses orientation.

So, we have characterized the Euclidean study of congruence.
It considers, for plane figures, the properties that are invariant 
under combinations of (i)~a rotation followed by a translation,
or (ii)~a reflection followed by a translation
(a reflection followed by a non-trivial 
translation is a \definend{glide reflection}\index{reflection!glide}).

Another idea, besides congruence of figures, encountered in 
elementary geometry is
that figures are \definend{similar}\index{similar triangles}\index{triangles!similar} 
if they are congruent after a change of scale. 
These two triangles are similar since the second is 
the same shape as the first, but $3/2$-ths the size.
\begin{center}
  \includegraphics{ch3.65}
\end{center}
From the above work we have that figures are similar if there
is an orthonormal matrix $T$ such that the points $\vec{q}$ 
on one figure are the images of 
the points $\vec{p}$ on the other figure by 
$\vec{q}=(kT)\vec{v}+\vec{p}_0$
for some nonzero real number $k$ and constant vector $\vec{p}_0$.

Although these ideas are from Euclid, mathematics is timeless and
they are still in use today. 
One application of the maps studied above is in computer graphics.
We can, for example, animate this top view of a cube 
by putting together film frames of it rotating; that's a rigid motion. 
\begin{center}
  \begin{tabular}{ccc}
    \includegraphics{ch3.66}
    &\includegraphics{ch3.67}
    &\includegraphics{ch3.68}                       \\
    Frame 1      &Frame 2       &Frame 3       
  \end{tabular}
\end{center}
We could also make the cube appear to be moving away from us by producing film
frames of it shrinking, which gives us figures that are similar.
\begin{center}
  \begin{tabular}{ccc}
    \includegraphics{ch3.69}
    &\includegraphics{ch3.70}
    &\includegraphics{ch3.71}                       \\
    Frame 1:        &Frame 2:       &Frame 3:        
  \end{tabular}
\end{center}
Computer graphics incorporates techniques
from linear algebra in many other ways (see \nearbyexercise{exer:HomoCrds}).

% So the analysis above of distance-preserving maps
% is useful as well as interesting.
A beautiful book that explores some of this area is \cite{Weyl}.
More on groups, of transformations and otherwise, is in any book
on Modern Algebra, for instance \cite{BirkhoffMaclane}.
More on Klein and the Erlanger Program is in \cite{Yaglom}.
 

\begin{exercises}
  \item 
    Decide if each of these is an orthonormal matrix.
    \begin{exparts}
      \partsitem $\begin{mat}[r]
                    1/\sqrt{2} &-1/\sqrt{2}  \\
                   -1/\sqrt{2} &-1/\sqrt{2}
                  \end{mat}$
      \partsitem $\begin{mat}[r]
                    1/\sqrt{3} &-1/\sqrt{3}  \\
                   -1/\sqrt{3} &-1/\sqrt{3}
                  \end{mat}$
      \partsitem $\begin{mat}[r]
                    1/\sqrt{3} &-\sqrt{2}/\sqrt{3}  \\
                   -\sqrt{2}/\sqrt{3} &-1/\sqrt{3}
                  \end{mat}$
    \end{exparts}
    \begin{answer}
      \begin{exparts}
        \partsitem Yes.
        \partsitem No, the columns do not have length one.
        \partsitem Yes.
      \end{exparts}
    \end{answer}
  \item 
    Write down the formula for each of these distance-preserving maps.
    \begin{exparts}  
      \partsitem the map that rotates $\pi/6$ radians, and then
        translates by $\vec{e}_2$
      \partsitem the map that reflects about the line $y=2x$
      \partsitem the map that reflects about $y=-2x$ and translates over $1$
        and up $1$
    \end{exparts}
    \begin{answer}
      Some of these are nonlinear, 
      because they involve a nontrivial translation.
      \begin{exparts}
        \partsitem 
          $\colvec{x \\ y}
           \mapsto
           \begin{mat}
             x\cdot\cos(\pi/6)-y\cdot\sin(\pi/6) \\
             x\cdot\sin(\pi/6)+y\cdot\cos(\pi/6) 
           \end{mat}
           +\colvec[r]{0 \\ 1}
           =\begin{mat}
             x\cdot(\sqrt{3}/2)-y\cdot(1/2)+0 \\
             x\cdot(1/2)+y\cdot\cos(\sqrt{3}/2)+1 
           \end{mat}$
        \partsitem The line $y=2x$ makes an angle of $\arctan(2/1)$
          with the $x$-axis.
          Thus $\sin\theta=2/\sqrt{5}$ and $\cos\theta=1/\sqrt{5}$.
          \begin{equation*}
          \colvec{x \\ y}
           \mapsto
           \begin{mat}
             x\cdot(1/\sqrt{5})-y\cdot(2/\sqrt{5}) \\
             x\cdot(2/\sqrt{5})+y\cdot(1/\sqrt{5}) 
           \end{mat}
           \end{equation*}
        \partsitem           
           $\colvec{x \\ y}
           \mapsto
           \begin{mat}
             x\cdot(1/\sqrt{5})-y\cdot(-2/\sqrt{5}) \\
             x\cdot(-2/\sqrt{5})+y\cdot(1/\sqrt{5}) 
           \end{mat}
           +\colvec[r]{1 \\ 1}
           =\begin{mat}
             x/\sqrt{5}+2y/\sqrt{5}+1 \\
             -2x/\sqrt{5}+y/\sqrt{5}+1 
           \end{mat}$
      \end{exparts}
    \end{answer}
  \item \label{exer:IsometryFacts}
     \begin{exparts}
       \partsitem The proof that a map that is distance-preserving and 
         sends the zero vector to itself incidentally shows that
         such a map is one-to-one and onto
         (the point in the domain determined by $d_0$, $d_1$, and $d_2$
         corresponds to the point in the codomain determined by those
         three).
         Therefore any distance-preserving map has an inverse.
         Show that the inverse is also distance-preserving.
       \partsitem Prove that congruence is an equivalence relation 
         between plane figures.
     \end{exparts}
     \begin{answer}
       \begin{exparts}
         \partsitem Let $f$ be distance-preserving and consider $f^{-1}$.
           Any two points in the codomain can be written as $f(P_1)$ and
           $f(P_2)$.
           Because $f$ is distance-preserving, the distance from $f(P_1)$
           to $f(P_2)$ equals the distance from $P_1$ to $P_2$.
           But this is exactly what is required for $f^{-1}$ to be
           distance-preserving.
         \partsitem Any plane figure $F$ is congruent to itself via the
           identity map $\map{\identity}{\Re^2}{\Re^2}$, which is obviously
           distance-preserving.
           If $F_1$ is congruent to $F_2$ (via some $f$) then 
           $F_2$ is congruent to $F_1$ via $f^{-1}$, which is 
           distance-preserving by the prior item.
           Finally, if $F_1$ is congruent to $F_2$ (via some $f$) and
           $F_2$ is congruent to $F_3$ (via some $g$) then $F_1$ is
           congruent to $F_3$ via $\composed{g}{f}$, which is easily checked
           to be distance-preserving.
       \end{exparts}
     \end{answer}
  \item \label{exer:HomoCrds}
    In practice the matrix for the distance-preserving linear transformation
    and the translation are often combined into one.
    Check that these two computations yield the same
    first two components.
        \begin{equation*}
           \begin{mat}
             a  &c  \\
             b  &d
           \end{mat}
           \colvec{x  \\ y}
           +\colvec{e \\ f}
           \qquad
           \begin{mat}
             a  &c  &e \\
             b  &d  &f \\
             0  &0  &1
           \end{mat}
           \colvec{x  \\ y \\ 1}
        \end{equation*}
    (These are 
     \definend{homogeneous coordinates}\index{homogeneous coordinates};
     see the Topic on Projective Geometry).
     \begin{answer}
       The first two components of each are $ax+cy+e$ and $bx+dy+f$.
     \end{answer}
  \item \label{exer:GeomInvDistPre}
    \begin{exparts}
     \partsitem Verify that the properties described 
       in the second paragraph of this Topic as invariant
       under distance-preserving maps are indeed so.
     \partsitem Give two more properties that are of interest 
       in Euclidean geometry from your experience in studying that
       subject that are also invariant under distance-preserving maps.
     \partsitem Give a property that is not of interest in Euclidean 
       geometry and is not invariant under distance-preserving maps.
    \end{exparts}
    \begin{answer}
      \begin{exparts}
        \partsitem The Pythagorean Theorem gives that
          three points are 
          colinear if and only if
          (for some ordering of them into $P_1$, $P_2$, and $P_3$),
          $\dist(P_1,P_2)+\dist(P_2,P_3)=\dist(P_1,P_3)$.
          Of course, where $f$ is distance-preserving, this holds
          if and only if
          $\dist(f(P_1),f(P_2))+\dist(f(P_2),f(P_3))=\dist(f(P_1),f(P_3))$,
          which, again by Pythagoras, is true if and only if
          $f(P_1)$, $f(P_2)$, and $f(P_3)$ are colinear.
        
          The argument for betweeness is similar (above, $P_2$ is 
          between $P_1$ and $P_3$).

          If the figure $F$ is a triangle then it is the union of three
          line segments $P_1P_2$, $P_2P_3$, and $P_1P_3$.
          The prior two paragraphs together show that the property of
          being a line segment is invariant.
          So $f(F)$ is the union of three line segments, and so is a 
          triangle.

          A circle $C$ centered at $P$ and of radius $r$ is the set of
          all points $Q$ such that $\dist(P,Q)=r$.
          Applying the distance-preserving map $f$ gives that the image
          $f(C)$ is the set of all $f(Q)$ subject to the condition that
          $\dist(P,Q)=r$.
          Since $\dist(P,Q)=\dist(f(P),f(Q))$, the set $f(C)$ is also
          a circle, with center $f(P)$ and radius $r$.
        \partsitem Here are two that are easy to verify: (i)~the
          property of being a right triangle, and (ii)~the property of
          two lines being parallel.
        \partsitem One that was mentioned in the section is the `sense' of
          a figure.
          A triangle whose vertices read clockwise as $P_1$, $P_2$, $P_3$
          may, under a distance-preserving map, be sent to a triangle
          read $P_1$, $P_2$, $P_3$ counterclockwise.
      \end{exparts}
    \end{answer}
\end{exercises}
\index{matrix!orthonormal|)}
\index{orthonormal matrix|)}
\endinput

