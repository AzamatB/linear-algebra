% Chapter 4, Topic _Linear Algebra_ Jim Hefferon
%  http://joshua.smcvt.edu/linalg.html
%  2001-Jun-12
\topic{Cramer's Rule}
\index{Cramer's rule|(}
% We have introduced determinant functions algebraically by looking
% for a formula to decide whether a matrix is nonsingular.
% After that introduction we saw a geometric interpretation, 
% that the determinant function
% gives the size of the box with sides formed by the columns of the matrix.
% Here we make a connection between the two views.

We have seen that a linear system
\begin{equation*}
  \begin{linsys}{2}
     x_1  &+  &2x_2  &=  &6  \\
    3x_1  &+  &x_2   &=  &8 
  \end{linsys}
\end{equation*}
is equivalent to a linear relationship among vectors. 
\begin{equation*}
  x_1\cdot\colvec{1 \\ 3}+x_2\cdot\colvec{2 \\ 1}=\colvec{6 \\ 8}
\end{equation*}
This pictures that vector equation.
A parallelogram with sides formed from
$\binom{1}{3}$ and $\binom{2}{1}$ is nested inside a parallelogram 
with sides formed from $x_1\binom{1}{3}$ and $x_2\binom{2}{1}$.
\begin{center}
  \includegraphics{ch4.1}
\end{center}
That is,
we can restate the algebraic question of
finding the solution of a linear system
in geometric terms:~by 
what factors $x_1$ and $x_2$ must we dilate the vectors to expand the small 
parallelogram so that it will fill the larger one?

We can apply the geometric significance of determinants
to that picture to get a new formula.
Compare the sizes of these shaded boxes.
\begin{center}
   \includegraphics{ch4.2}
   \hfil
   \includegraphics{ch4.3}
   \hfil
   \includegraphics{ch4.4}
\end{center}
The second is defined by the vectors $x_1\binom{1}{3}$ and $\binom{2}{1}$, and
one of the properties of the size function\Dash the determinant\Dash is 
that therefore the size of the second box is \( x_1 \) times the size of the 
first box.
Since the third box is defined by the vector
$x_1\binom{1}{3}+x_2\binom{2}{1}=\binom{6}{8}$ 
and the vector $\binom{2}{1}$,
and since the determinant does not change when we  add $x_2$ 
times the second column to the first column,
the size of the third box equals that of the second.
\begin{equation*}
  \begin{vmat}[r]
     6  &2  \\
     8  &1
  \end{vmat}
  =
  \begin{vmat}
     x_1\cdot 1  &2  \\
     x_1\cdot 3  &1
  \end{vmat}
  =
  x_1\cdot \begin{vmat}[r]
     1  &2  \\
     3  &1
  \end{vmat}
\end{equation*}
Solving gives the value of one of the variables.
\begin{equation*}
  x_1=
  \frac{\begin{vmat}[r]
     6  &2  \\
     8  &1
  \end{vmat} }{
  \begin{vmat}[r]
     1  &2  \\
     3  &1
  \end{vmat}  }
  =\frac{-10}{-5}=2
\end{equation*}

The generalization of this example is \definend{Cramer's Rule}:%
\index{determinant!Cramer's rule}%
\index{linear equation!solutions of!Cramer's rule}
if \( \deter{A}\neq 0 \) then the system \( A\vec{x}=\vec{b} \) has the
unique solution
$
   x_i=\deter{B_i}/\deter{A}
$
where the matrix $B_i$ is formed from $A$ by replacing column~$i$ 
with the vector \( \vec{b} \).
The proof is \nearbyexercise{ex:CramerRule}.

For instance, to solve this system for \( x_2 \)
\begin{equation*}
  \begin{mat}[r]
    1  &0  &4  \\
    2  &1  &-1 \\
    1  &0  &1
  \end{mat}
  \colvec{x_1 \\ x_2 \\ x_3}
  =\colvec[r]{2 \\ 1 \\ -1}
\end{equation*}
we do this computation.
\begin{equation*}
  x_2=
  \frac{ \begin{vmat}[r]
           1  &2  &4  \\
           2  &1  &-1 \\
           1  &-1 &1
         \end{vmat}  }{
         \begin{vmat}[r]
           1  &0  &4  \\
           2  &1  &-1 \\
           1  &0  &1
         \end{vmat}  }
  =\frac{-18}{-3}
\end{equation*}

Cramer's Rule allows us to solve 
simple two equations/two unknowns systems by eye
(they must be simple in that we can mentally compute with the numbers
in the system).
With practice a person can also do simple three equations/three unknowns 
systems.
But computing large determinants takes a long time so solving
large systems by Cramer's Rule is not practical.

\begin{exercises}
  \item 
    Use Cramer's Rule to solve each for each of the variables.
    \begin{exparts*}
      \partsitem $\begin{linsys}{2}
                    x  &- &y  &=  &4  \\
                   -x  &+ &2y &=  &-7
                  \end{linsys}$
      \partsitem $\begin{linsys}{2}
                    -2x  &+  &y  &=  &-2 \\
                      x  &-  &2y &=  &-2  
                  \end{linsys}$
    \end{exparts*}
    \begin{answer}
      \begin{exparts*}
        \partsitem 
          \begin{equation*}
            x=
             \frac{ \begin{vmat}[r]
                       4  &-1  \\
                      -7  &2   
                    \end{vmat}  }{
                    \begin{vmat}[r]
                       1  &-1  \\
                      -1  &2 
                    \end{vmat}  }
            =\frac{1}{1}=1
            \qquad
            y=
             \frac{ \begin{vmat}[r]
                       1  &4  \\
                      -1  &-7   
                    \end{vmat}  }{
                    \begin{vmat}[r]
                       1  &-1  \\
                      -1  &2 
                    \end{vmat}  }
            =\frac{-3}{1}=-3
          \end{equation*}
        \partsitem $x=2$, $y=2$
      \end{exparts*} 
    \end{answer}
  \item 
    Use Cramer's Rule to solve this system for \( z \).
    \begin{equation*}
      \begin{linsys}{4}
        2x  &+  &y  &+  &z  &=  &1 \\
        3x  &   &   &+  &z  &=  &4 \\
         x  &-  &y  &-  &z  &=  &2
      \end{linsys}
    \end{equation*}
    \begin{answer}
      \( z=1 \)
    \end{answer}
  \item \label{ex:CramerRule}
    Prove Cramer's Rule.
    \begin{answer}
      Determinants are unchanged by combinations, 
      including column combinations, so
      \( \det(B_i)=\det(\vec{a}_1,\dots,
      x_1\vec{a}_1+\dots+x_i\vec{a}_i+\dots+x_n\vec{a}_n,\dots,\vec{a}_n) \)
      is equal to 
      $\det(\vec{a}_1,\dots,x_i\vec{a}_i,\dots,\vec{a}_n)$
      (use the operation of taking $-x_1$ times the first column and adding 
      it to the $i$-th column, etc.).  
      That is equal to 
      $x_i\cdot\det(\vec{a}_1,\dots,\vec{a}_i,\dots,\vec{a}_n)
         =x_i\cdot\det(A)$,
      as required.
    \end{answer}
  \item 
    Suppose that a linear system has as many equations as unknowns,
    that all of its coefficients and constants are integers, and that 
    its matrix
    of coefficients has determinant~\( 1 \).
    Prove that the entries in the solution are all integers.
    (\textit{Remark.}  
     This is often used to invent linear systems for exercises.
     If an instructor makes the linear system with this property
     then the solution is not some disagreeable fraction.)
    \begin{answer}
      Because the determinant of $A$ is nonzero, Cramer's Rule applies and
      shows that $x_i=\deter{B_i}/1$.  
      Since $B_i$ is a matrix of integers, its determinant is an integer.     
    \end{answer}
  \item 
    Use Cramer's Rule to give a formula for the solution of a
    two equations/two unknowns linear system.
    \begin{answer}
      The solution of
      \begin{equation*}
        \begin{linsys}{2}
           ax  &+  by  &=  &e  \\
           cx  &+  dy  &=  &f  
        \end{linsys}
      \end{equation*}
      is
      \begin{equation*}
        x=\frac{ed-fb}{ad-bc}
        \qquad
        y=\frac{af-ec}{ad-bc}
      \end{equation*}
      provided of course that the denominators are not zero.  
    \end{answer}
  \item
    Can Cramer's Rule tell the difference between a system with no
    solutions and one with infinitely many?
    \begin{answer}
      Of course, singular systems have \( \deter{A} \) equal to zero, but
      we can characterize 
      the infinitely many solutions case is by the fact that
      all of the \( \deter{B_i} \) are zero as well.  
    \end{answer}
  \item 
    The first picture in this Topic (the one that doesn't use determinants)
    shows a unique solution case.
    Produce a similar picture for the case of infinitely many solutions,
    and the case of no solutions.
    \begin{answer}
      We can consider the two nonsingular cases together with this
      system
      \begin{equation*}
        \begin{linsys}{2}
           x_1  &+  &2x_2  &=  &6  \\
           x_1  &+  &2x_2  &=  &c 
        \end{linsys}
      \end{equation*}
      where $c=6$ of course yields infinitely many solutions, and any other
      value for $c$ yields no solutions.  
      The corresponding vector equation
      \begin{equation*}
        x_1\cdot\colvec[r]{1 \\ 1}+x_2\cdot\colvec[r]{2 \\ 2}=\colvec[r]{6 \\ c}
      \end{equation*}
      gives a picture of two overlapping vectors.
      Both lie on the line $y=x$.
      In the $c=6$ case the vector on the right side also lies on
      the line $y=x$ but in any other case it does not.
    \end{answer}  
\end{exercises}
\index{Cramer's rule|)}
\endinput
