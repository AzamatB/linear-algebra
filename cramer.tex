% Chapter 4, Topic _Linear Algebra_ Jim Hefferon
%  http://joshua.smcvt.edu/linearalgebra
%  2001-Jun-12
\topic{Cramer's Rule}
\index{Cramer's rule|(}
% We have introduced determinant functions algebraically by looking
% for a formula to decide whether a matrix is nonsingular.
% After that introduction we saw a geometric interpretation, 
% that the determinant function
% gives the size of the box with sides formed by the columns of the matrix.
% Here we make a connection between the two views.

%<*CramersRuleExample0>
A linear system is equivalent to a linear relationship among vectors. 
\begin{equation*}
  \begin{linsys}{2}
     x_1  &+  &2x_2  &=  &6  \\
    3x_1  &+  &x_2   &=  &8 
  \end{linsys}
% \end{equation*}
% \begin{equation*}
  \qquad\Longleftrightarrow\qquad
  x_1\cdot\colvec{1 \\ 3}+x_2\cdot\colvec{2 \\ 1}=\colvec{6 \\ 8}
\end{equation*}
In the picture below
the small parallelogram is formed from sides that are the vectors 
$\binom{1}{3}$ and $\binom{2}{1}$.
It is nested inside a parallelogram 
with sides $x_1\binom{1}{3}$ and~$x_2\binom{2}{1}$. 
By the vector equation, the far corner of the larger 
parallelogram is $\binom{6}{8}$.
\begin{center}
  \includegraphics{ch4.1}
\end{center}
%</CramersRuleExample0>
%<*CramersRuleExample1>
This drawing restates the algebraic question of
finding the solution of a linear system
into geometric terms:~by 
what factors $x_1$ and~$x_2$ must we dilate the sides of the starting 
parallelogram so that it will fill the other one?
%</CramersRuleExample1>

We can use this picture, and our geometric understanding of determinants,
to get a new formula for solving linear systems.
Compare the sizes of these shaded boxes.
\begin{center}
   \includegraphics{ch4.2}
   \hfil
   \includegraphics{ch4.3}
   \hfil
   \includegraphics{ch4.4}
\end{center}
%<*CramersRuleExample2>
The second is defined by the vectors $x_1\binom{1}{3}$ and $\binom{2}{1}$ and
one of the properties of the size function\Dash the determinant\Dash is 
that therefore the size of the second box is \( x_1 \) times the size of the 
first.
The third box is derived from the second by shearing, adding
$x_2\binom{2}{1}$ to $x_1\binom{1}{3}$ to get
$x_1\binom{1}{3}+x_2\binom{2}{1}=\binom{6}{8}$,
along with $\binom{2}{1}$.
The determinant is not affected by shearing so
the size of the third box equals that of the second.

Taken together, we have this.
\begin{equation*}
  \begin{vmat}[r]
     6  &2  \\
     8  &1
  \end{vmat}
  =
  \begin{vmat}
     x_1\cdot 1  &2  \\
     x_1\cdot 3  &1
  \end{vmat}
  =
  x_1\cdot \begin{vmat}[r]
     1  &2  \\
     3  &1
  \end{vmat}
\end{equation*}
%</CramersRuleExample2>
%<*CramersRuleExample3>
Solving gives the value of one of the variables.
\begin{equation*}
  x_1=
  \frac{\begin{vmat}[r]
     6  &2  \\
     8  &1
  \end{vmat} }{
  \begin{vmat}[r]
     1  &2  \\
     3  &1
  \end{vmat}  }
  =\frac{-10}{-5}=2
\end{equation*}
%</CramersRuleExample3>

The generalization of this example is \definend{Cramer's Rule}:%
\index{determinant!Cramer's rule}%
\index{linear equation!solution of!Cramer's rule}
if \( \deter{A}\neq 0 \) then the system \( A\vec{x}=\vec{b} \) has the
unique solution
$
   x_i=\deter{B_i}/\deter{A}
$
where the matrix $B_i$ is formed from $A$ by replacing column~$i$ 
with the vector \( \vec{b} \).
The proof is \nearbyexercise{ex:CramerRule}.

For instance, to solve this system for \( x_2 \)
\begin{equation*}
  \begin{mat}[r]
    1  &0  &4  \\
    2  &1  &-1 \\
    1  &0  &1
  \end{mat}
  \colvec{x_1 \\ x_2 \\ x_3}
  =\colvec[r]{2 \\ 1 \\ -1}
\end{equation*}
we do this computation.
\begin{equation*}
  x_2=
  \frac{ \begin{vmat}[r]
           1  &2  &4  \\
           2  &1  &-1 \\
           1  &-1 &1
         \end{vmat}  }{
         \begin{vmat}[r]
           1  &0  &4  \\
           2  &1  &-1 \\
           1  &0  &1
         \end{vmat}  }
  =\frac{-18}{-3}
\end{equation*}

Cramer's Rule lets us by-eye solve systems that are small and simple. 
For example, we can solve systems with two equations and two unknowns,
or three equations and three unknowns,
where the numbers are small integers.
Such cases appear often enough that many people find this formula handy.
 
But using it to solving large or complex systems is not practical,
either by hand or by a computer.
A Gauss's Method-based approach is faster.

\begin{exercises}
  \item 
    Use Cramer's Rule to solve each for each of the variables.
    \begin{exparts*}
      \partsitem $\begin{linsys}{2}
                    x  &- &y  &=  &4  \\
                   -x  &+ &2y &=  &-7
                  \end{linsys}$
      \partsitem $\begin{linsys}{2}
                    -2x  &+  &y  &=  &-2 \\
                      x  &-  &2y &=  &-2  
                  \end{linsys}$
    \end{exparts*}
    \begin{answer}
      \begin{exparts*}
        \partsitem 
          \begin{equation*}
            x=
             \frac{ \begin{vmat}[r]
                       4  &-1  \\
                      -7  &2   
                    \end{vmat}  }{
                    \begin{vmat}[r]
                       1  &-1  \\
                      -1  &2 
                    \end{vmat}  }
            =\frac{1}{1}=1
            \qquad
            y=
             \frac{ \begin{vmat}[r]
                       1  &4  \\
                      -1  &-7   
                    \end{vmat}  }{
                    \begin{vmat}[r]
                       1  &-1  \\
                      -1  &2 
                    \end{vmat}  }
            =\frac{-3}{1}=-3
          \end{equation*}
        \partsitem $x=2$, $y=2$
      \end{exparts*} 
    \end{answer}
  \item 
    Use Cramer's Rule to solve this system for \( z \).
    \begin{equation*}
      \begin{linsys}{4}
        2x  &+  &y  &+  &z  &=  &1 \\
        3x  &   &   &+  &z  &=  &4 \\
         x  &-  &y  &-  &z  &=  &2
      \end{linsys}
    \end{equation*}
    \begin{answer}
      \( z=1 \)
    \end{answer}
  \item \label{ex:CramerRule}
    Prove Cramer's Rule.
    \begin{answer}
      Determinants are unchanged by combinations, 
      including column combinations, so
      \( \det(B_i)=\det(\vec{a}_1,\dots,
      x_1\vec{a}_1+\dots+x_i\vec{a}_i+\dots+x_n\vec{a}_n,\dots,\vec{a}_n) \)
      is equal to 
      $\det(\vec{a}_1,\dots,x_i\vec{a}_i,\dots,\vec{a}_n)$
      (use the operation of taking $-x_1$ times the first column and adding 
      it to the $i$-th column, etc.).  
      That is equal to 
      $x_i\cdot\det(\vec{a}_1,\dots,\vec{a}_i,\dots,\vec{a}_n)
         =x_i\cdot\det(A)$,
      as required.
    \end{answer}
  \item
   Here is an alternative proof of Cramer's Rule that doesn't 
   overtly contain any geometry.
   Write $X_i$ for the identity matrix with 
   column~$i$ replaced by the vector~$\vec{x}$ of unknowns 
   $x_1$, \ldots,~$x_n$.
   \begin{exparts}
     \partsitem Observe that $AX_i=B_i$.
     \partsitem Take the determinant of both sides.
   \end{exparts}
   \begin{answer}
     \begin{exparts*}
       \partsitem
         Here is the case of a $\nbyn{2}$ system with $i=2$.
         \begin{equation*}
           \begin{linsys}{2}
             a_{1,1}x_1 &+ &a_{1,2}x_2 &= &b_1 \\
             a_{2,1}x_1 &+ &a_{2,2}x_2 &= &b_2
           \end{linsys}
           \quad\Longleftrightarrow\quad
           \begin{mat}
             a_{1,1}  &a_{1,2}  \\
             a_{2,1}  &a_{2,2}
           \end{mat}
           \begin{mat}
             1  &x_1 \\
             0  &x_2
           \end{mat}
           =
           \begin{mat}
             a_{1,1}  &b_1 \\
             a_{2,1}  &b_2
           \end{mat}
         \end{equation*}
      \partsitem The determinant function is multiplicative
        $\det(B_i)=\det(AX_i)=\det(A)\cdot\det(X_i)$.
        The Laplace expansion shows that $\det(X_i)=x_i$,
        and solving for $x_i$ gives Cramer's Rule.
     \end{exparts*}
   \end{answer}

  \item 
    Suppose that a linear system has as many equations as unknowns,
    that all of its coefficients and constants are integers, and that 
    its matrix
    of coefficients has determinant~\( 1 \).
    Prove that the entries in the solution are all integers.
    (\textit{Remark.}  
     This is often used to invent linear systems for exercises.)
    \begin{answer}
      Because the determinant of $A$ is nonzero, Cramer's Rule applies and
      shows that $x_i=\deter{B_i}/1$.  
      Since $B_i$ is a matrix of integers, its determinant is an integer.     
    \end{answer}
  \item 
    Use Cramer's Rule to give a formula for the solution of a
    two equations/two unknowns linear system.
    \begin{answer}
      The solution of
      \begin{equation*}
        \begin{linsys}{2}
           ax  &+  by  &=  &e  \\
           cx  &+  dy  &=  &f  
        \end{linsys}
      \end{equation*}
      is
      \begin{equation*}
        x=\frac{ed-fb}{ad-bc}
        \qquad
        y=\frac{af-ec}{ad-bc}
      \end{equation*}
      provided of course that the denominators are not zero.  
    \end{answer}
  \item
    Can Cramer's Rule tell the difference between a system with no
    solutions and one with infinitely many?
    \begin{answer}
      Of course, singular systems have \( \deter{A} \) equal to zero, but
      we can characterize 
      the infinitely many solutions case is by the fact that
      all of the \( \deter{B_i} \) are zero as well.  
    \end{answer}
  \item 
    The first picture in this Topic (the one that doesn't use determinants)
    shows a unique solution case.
    Produce a similar picture for the case of infinitely many solutions,
    and the case of no solutions.
    \begin{answer}
      We can consider the two nonsingular cases together with this
      system
      \begin{equation*}
        \begin{linsys}{2}
           x_1  &+  &2x_2  &=  &6  \\
           x_1  &+  &2x_2  &=  &c 
        \end{linsys}
      \end{equation*}
      where $c=6$ of course yields infinitely many solutions, and any other
      value for $c$ yields no solutions.  
      The corresponding vector equation
      \begin{equation*}
        x_1\cdot\colvec[r]{1 \\ 1}+x_2\cdot\colvec[r]{2 \\ 2}=\colvec[r]{6 \\ c}
      \end{equation*}
      gives a picture of two overlapping vectors.
      Both lie on the line $y=x$.
      In the $c=6$ case the vector on the right side also lies on
      the line $y=x$ but in any other case it does not.
    \end{answer}  
\end{exercises}
\index{Cramer's rule|)}
\endinput
