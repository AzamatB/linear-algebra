% Chapter 4, Topic _Linear Algebra_ Jim Hefferon
%  http://joshua.smcvt.edu/linalg.html
%  2001-Jun-12
\topic{Cramer's Rule}
\index{Cramer's rule|(}
We have introduced determinant functions algebraically by looking
for a formula to decide whether a matrix is nonsingular.
After that introduction we saw a geometric interpretation, 
that the determinant function
gives the size of the box with sides formed by the columns of the matrix.
This Topic makes a connection between the two views.

First, a linear system
\begin{equation*}
  \begin{linsys}{2}
     x_1  &+  &2x_2  &=  &6  \\
    3x_1  &+  &x_2   &=  &8 
  \end{linsys}
\end{equation*}
is equivalent to a linear relationship among vectors. 
\begin{equation*}
  x_1\cdot\colvec{1 \\ 3}+x_2\cdot\colvec{2 \\ 1}=\colvec{6 \\ 8}
\end{equation*}
The picture below shows a parallelogram with sides formed from
$\binom{1}{3}$ and $\binom{2}{1}$ nested inside a parallelogram 
with sides formed from $x_1\binom{1}{3}$ and $x_2\binom{2}{1}$.
\begin{center}
  \includegraphics{ch4.1}
\end{center}
So even without determinants
we can state the algebraic issue that opened this book,
finding the solution of a linear system,
in geometric terms:~by 
what factors $x_1$ and $x_2$ must we dilate the vectors to expand the small 
parallegram to fill the larger one?

However, by employing the geometric significance of determinants
we can get something that is not just a restatement, but also
gives us a new insight and sometimes allows us to compute 
answers quickly.
Compare the sizes of these shaded boxes.
\begin{center}
   \includegraphics{ch4.2}
   \hfil
   \includegraphics{ch4.3}
   \hfil
   \includegraphics{ch4.4}
\end{center}
The second is formed from $x_1\binom{1}{3}$ and $\binom{2}{1}$, and
one of the properties of the size function\Dash the determinant\Dash is 
that its size is therefore \( x_1 \) times the size of the 
first box.
Since the third box is formed from 
$x_1\binom{1}{3}+x_2\binom{2}{1}=\binom{6}{8}$ 
and $\binom{2}{1}$,
and the determinant is unchanged by adding $x_2$ 
times the second column to the first column,
the size of the third box equals that of the second.
We have this.
\begin{equation*}
  \begin{vmatrix}
     6  &2  \\
     8  &1
  \end{vmatrix}
  =
  \begin{vmatrix}
     x_1\cdot 1  &2  \\
     x_1\cdot 3  &1
  \end{vmatrix}
  =
  x_1\cdot \begin{vmatrix}
     1  &2  \\
     3  &1
  \end{vmatrix}
\end{equation*}
Solving gives the value of one of the variables.
\begin{equation*}
  x_1=
  \frac{\begin{vmatrix}
     6  &2  \\
     8  &1
  \end{vmatrix} }{
  \begin{vmatrix}
     1  &2  \\
     3  &1
  \end{vmatrix}  }
  =\frac{-10}{-5}=2
\end{equation*}

The theorem that generalizes this example, \definend{Cramer's Rule}%
\index{determinant!Cramer's rule}%
\index{linear equation!solutions of!Cramer's rule},
is:~if \( \deter{A}\neq 0 \) then the system \( A\vec{x}=\vec{b} \) has the
unique solution
$
   x_i=\deter{B_i}/\deter{A}
$
where the matrix $B_i$ is formed from $A$ by replacing column~$i$ 
with the vector \( \vec{b} \).
\nearbyexercise{ex:CramerRule} asks for a proof.

For instance, to solve this system for \( x_2 \)
\begin{equation*}
  \begin{pmatrix}
    1  &0  &4  \\
    2  &1  &-1 \\
    1  &0  &1
  \end{pmatrix}
  \colvec{x_1 \\ x_2 \\ x_3}
  =\colvec{2 \\ 1 \\ -1}
\end{equation*}
we do this computation.
\begin{equation*}
  x_2=
  \frac{ \begin{vmatrix}
           1  &2  &4  \\
           2  &1  &-1 \\
           1  &-1 &1
         \end{vmatrix}  }{
         \begin{vmatrix}
           1  &0  &4  \\
           2  &1  &-1 \\
           1  &0  &1
         \end{vmatrix}  }
  =\frac{-18}{-3}
\end{equation*}

Cramer's Rule allows us to solve 
many two equations/two unknowns systems by eye.
It is also sometimes used for three equations/three unknowns systems.
But computing large determinants takes a long time, so solving
large systems by Cramer's Rule is not practical.

\begin{exercises}
  \item 
    Use Cramer's Rule to solve each for each of the variables.
    \begin{exparts*}
      \partsitem $\begin{linsys}{2}
                    x  &- &y  &=  &4  \\
                   -x  &+ &2y &=  &-7
                  \end{linsys}$
      \partsitem $\begin{linsys}{2}
                    -2x  &+  &y  &=  &-2 \\
                      x  &-  &2y &=  &-2  
                  \end{linsys}$
    \end{exparts*}
    \begin{answer}
      \begin{exparts*}
        \partsitem 
          \begin{equation*}
            x=
             \frac{ \begin{vmatrix}
                       4  &-1  \\
                      -7  &2   
                    \end{vmatrix}  }{
                    \begin{vmatrix}
                       1  &-1  \\
                      -1  &2 
                    \end{vmatrix}  }
            =\frac{1}{1}=1
            \qquad
            y=
             \frac{ \begin{vmatrix}
                       1  &4  \\
                      -1  &-7   
                    \end{vmatrix}  }{
                    \begin{vmatrix}
                       1  &-1  \\
                      -1  &2 
                    \end{vmatrix}  }
            =\frac{-3}{1}=-3
          \end{equation*}
        \partsitem $x=2$, $y=2$
      \end{exparts*} 
    \end{answer}
  \item 
    Use Cramer's Rule to solve this system for \( z \).
    \begin{equation*}
      \begin{linsys}{4}
        2x  &+  &y  &+  &z  &=  &1 \\
        3x  &   &   &+  &z  &=  &4 \\
         x  &-  &y  &-  &z  &=  &2
      \end{linsys}
    \end{equation*}
    \begin{answer}
      \( z=1 \)
    \end{answer}
  \item \label{ex:CramerRule}
    Prove Cramer's Rule.
    \begin{answer}
      Determinants are unchanged by combinations, 
      including column combinations, so
      \( \det(B_i)=\det(\vec{a}_1,\dots,
      x_1\vec{a}_1+\dots+x_i\vec{a}_i+\dots+x_n\vec{a}_n,\dots,\vec{a}_n) \)
      is equal to 
      $\det(\vec{a}_1,\dots,x_i\vec{a}_i,\dots,\vec{a}_n)$
      (use the operation of taking $-x_1$ times the first column and adding 
      it to the $i$-th column, etc.).  
      That is equal to 
      $x_i\cdot\det(\vec{a}_1,\dots,\vec{a}_i,\dots,\vec{a}_n)
         =x_i\cdot\det(A)$,
      as required.
    \end{answer}
  \item 
    Suppose that a linear system has as many equations as unknowns,
    that all of its coefficients and constants are integers, and that 
    its matrix
    of coefficients has determinant~\( 1 \).
    Prove that the entries in the solution are all integers.
    (\textit{Remark.}  
     This is often used to invent linear systems for exercises.
     If an instructor makes the linear system with this property
     then the solution is not some disagreeable fraction.)
    \begin{answer}
      Because the determinant of $A$ is nonzero, Cramer's Rule applies and
      shows that $x_i=\deter{B_i}/1$.  
      Since $B_i$ is a matrix of integers, its determinant is an integer.     
    \end{answer}
  \item 
    Use Cramer's Rule to give a formula for the solution of a
    two equations/two unknowns linear system.
    \begin{answer}
      The solution of
      \begin{equation*}
        \begin{linsys}{2}
           ax  &+  by  &=  &e  \\
           cx  &+  dy  &=  &f  
        \end{linsys}
      \end{equation*}
      is
      \begin{equation*}
        x=\frac{ed-fb}{ad-bc}
        \qquad
        y=\frac{af-ec}{ad-bc}
      \end{equation*}
      provided of course that the denominators are not zero.  
    \end{answer}
  \item
    Can Cramer's Rule tell the difference between a system with no
    solutions and one with infinitely many?
    \begin{answer}
      Of course, singular systems have \( \deter{A} \) equal to zero, but
      the infinitely many solutions case is characterized by the fact that
      all of the \( \deter{B_i} \) are zero as well.  
    \end{answer}
  \item 
    The first picture in this Topic (the one that doesn't use determinants)
    shows a unique solution case.
    Produce a similar picture for the case of infintely many solutions,
    and the case of no solutions.
    \begin{answer}
      We can consider the two nonsingular cases together with this
      system
      \begin{equation*}
        \begin{linsys}{2}
           x_1  &+  &2x_2  &=  &6  \\
           x_1  &+  &2x_2  &=  &c 
        \end{linsys}
      \end{equation*}
      where $c=6$ of course yields infinitely many solutions, and any other
      value for $c$ yields no solutions.  
      The corresponding vector equation
      \begin{equation*}
        x_1\cdot\colvec{1 \\ 1}+x_2\cdot\colvec{2 \\ 2}=\colvec{6 \\ c}
      \end{equation*}
      gives a picture of two overlapping vectors.
      Both lie on the line $y=x$.
      In the $c=6$ case the vector on the right side also lies on
      the line $y=x$ but in any other case it does not.
    \end{answer}  
\end{exercises}
\index{Cramer's rule|)}
\endinput
