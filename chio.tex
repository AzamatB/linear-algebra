% Chapter 4, Topic _Linear Algebra_ Jim Hefferon
%  http://joshua.smcvt.edu/linalg.html
%  2001-Jun-12
\topic{Chi\`o's Method}
\index{Chi\`o's method|(}

When doing Gauss's Method on a matrix that contains only integers
\begin{equation*}
  A=
  \begin{mat}
    2 &1 &1 \\
    3 &4 &-1 \\
    1 &5 &1 
  \end{mat}
\end{equation*}
people often prefer to keep it that way.
Instead of the row operations $-(3/2)\rho_1+\rho_2$ and
$-(1/2)\rho_1+\rho_3$ they may start by multiplying the rows below the first
by~$2$
\begin{equation*}
  \grstep[2\rho_3]{2\rho_2}\;
  \begin{mat}
    2 &1 &1 \\
    6 &8 &-2 \\
    2 &10 &2 
  \end{mat}
  \tag{$*$}
\end{equation*}
and then the elimination in the first column goes this way.
\begin{equation*}
  \grstep[-\rho_1+\rho_3]{-3\rho_1+\rho_2}\;
  \begin{mat}
    2 &1 &1 \\
    0 &5 &-5 \\
    0 &8 &0 
  \end{mat}
  \tag{$**$}
\end{equation*}
An all-integer approach is easier for mental calculations. 
And, using integer arithmetic on a computer
avoids some sticky issues involving floating point calculations \cite{Kahan}.
So there are sound reasons to study this approach.

Another reason comes from
observing that we can easily apply Laplace's expansion to the 
first column of~($**$), and then we get the determinant of $A$ by 
remembering to divide by $4$ because of~($*$).

Here is the general $\nbyn{3}$ case of this approach to finding the determinant.
Rescale all rows except the first.
\begin{equation*}
  A=
  \begin{mat}
    a_{1,1} &a_{1,2} &a_{1,3}  \\
    a_{2,1} &a_{2,2} &a_{2,3}  \\
    a_{3,1} &a_{3,2} &a_{3,3}  
  \end{mat}
  \;\grstep[a_{1,1}\rho_3]{a_{1,1}\rho_2}\;
  \begin{mat}
    a_{1,1}       &a_{1,2}       &a_{1,3}        \\
    a_{2,1}a_{1,1} &a_{2,2}a_{1,1} &a_{2,3}a_{1,1}   \\
    a_{3,1}a_{1,1} &a_{3,2}a_{1,1} &a_{3,3}a_{1,1}   
  \end{mat}
\end{equation*}
This rescales the determinant by $a_{1,1}^2$.
Now eliminate down the first column.
\begin{equation*}
   \grstep[-a_{3,1}\rho_1+\rho_3]{-a_{2,1}\rho_1+\rho_2}\;
  \begin{mat}
    a_{1,1}       &a_{1,2}       &a_{1,3}       \\
    0 
       &a_{2,2}a_{1,1}-a_{2,1}a_{1,2} 
       &a_{2,3}a_{1,1}-a_{2,1}a_{1,3}               \\
    0 
       &a_{3,2}a_{1,1}-a_{3,1}a_{1,2} 
       &a_{3,3}a_{1,1}-a_{3,1}a_{1,3}                   
  \end{mat}
\end{equation*}
Let the $1,1$ minor be $C$.
By Laplace the determinant of the above matrix is $a_{1,1}\det(C)$.
We have $a_{1,1}^2\det(A)=a_{1,1}\det(C)$ and if $a_{1,1}\neq 0$
then this $\nbyn{3}$ case gives $det(A)=\det(C)/a_{1,1}$.

To expand this approach to $\nbyn{n}$ matrices with $n>3$ 
we must see how to find the entries of the minor.
The pattern is: each element of the minor is a 
$\nbyn{2}$~determinant.
For instance, the entry in the minor's upper left
$a_{2,2}a_{1,1}-a_{2,1}a_{1,2}$, which is the $2,2$~entry in the above matrix,
is the determinant of the matrix of these
four elements of $A$.
\begin{equation*}
  \begin{mat}
    \highlight{$a_{1,1}$} &\highlight{$a_{1,2}$} &a_{1,3}  \\
    \highlight{$a_{2,1}$} &\highlight{$a_{2,2}$} &a_{2,3}  \\
    a_{3,1}               &a_{3,2}               &a_{3,3}  
  \end{mat}
\end{equation*}
And the minor's lower left, the $3,2$ entry from above, 
is the determinant of the matrix 
of these four.
\begin{equation*}
  \begin{mat}
    \highlight{$a_{1,1}$} &\highlight{$a_{1,2}$} &a_{1,3}  \\
    a_{2,1}               &a_{2,2}               &a_{2,3}  \\
    \highlight{$a_{3,1}$} &\highlight{$a_{3,2}$} &a_{3,3}  
  \end{mat}
\end{equation*}

So, where $A$ is~$\nbyn{n}$ for $n\geq 3$, we let Chi\`o's matrix $C$ be the
$\nbyn{(n-1)}$ matrix whose $i,j$ entry is the determinant
\begin{equation*}
  \begin{vmat}
    a_{1,1}  &a_{1,j+1} \\
    a_{i+1,1}    &a_{i+1,j+1}
  \end{vmat}
\end{equation*}
where $1<i,j\leq n$.
\definend{Chi\`o's method}\index{Chi\`o's formula} 
for finding the determinant of $A$ is that
if $a_{1,1}\neq 0$ then
$\det(A)=\det(C)/a_{1,1}^{n-2}$.

By the way,
nothing in Chi\`o's formula requires that the numbers be integers; it applies
to reals as well. 

We illustrate by finding the determinant of this $\nbyn{3}$ matrix.
\begin{equation*} 
  A=
  \begin{mat}
    2 &1 &1 \\
    3 &4 &-1 \\
    1 &5 &1 
  \end{mat}
\end{equation*}
This is Chi\`o's matrix.
\begin{equation*}
  C=
  \begin{mat}
    \begin{vmat}
      2 &1 \\
      3 &4
    \end{vmat}
   &
   \begin{vmat}
     2 &1 \\
     3 &-1
   \end{vmat}        \\[3ex]      
   \begin{vmat}
     2 &1 \\
     1 &5
   \end{vmat}
   &
   \begin{vmat}
     2 &1 \\
     1 &1
   \end{vmat}
  \end{mat}
  =
  \begin{mat}
    5  &-5  \\
    9  &1
  \end{mat}
\end{equation*}
Chi\`o's formula for $\nbyn{3}$ matrices
$det(A)=\det(C)/a_{1,1}$ gives $\det(A)=(50/2)=25$.

For a larger determinant we must do multiple steps
but each involves only $\nbyn{2}$ determinants and so 
we can often only write down some
intermediate information.
For instance, given this $\nbyn{4}$ matrix
\begin{equation*}
  A=
  \begin{mat}
    3  &0  &1  &1  \\
    1  &2  &0  &1  \\
    2  &-1 &0  &3  \\
    1  &0  &0  &1
  \end{mat} 
\end{equation*}
we can find Chi\`o's matrix by mentally doing each of the 
$\nbyn{2}$ calculations and only noting the
$\nbyn{3}$ result. 
\begin{equation*}
  C_3=
  \begin{mat}
    \begin{vmat}
      3 &0 \\
      1 &2
    \end{vmat}
    &\begin{vmat}
     3 &1 \\
     1 &0 
    \end{vmat}
    &\begin{vmat}
     3 &1 \\
     1 &1
    \end{vmat}                \\[3ex]
    \begin{vmat}
     3 &0 \\
     2 &-1
    \end{vmat}
    &\begin{vmat}
     3 &1 \\
     2 &0
    \end{vmat}
    &\begin{vmat}
     3 &1 \\
     2 &3
    \end{vmat}             \\[3ex]
    \begin{vmat}
     3 &0 \\
     1 &0
    \end{vmat}
    &\begin{vmat}
     3 &1 \\
     1 &0
    \end{vmat}
    &\begin{vmat}
     3 &1 \\
     1 &1
    \end{vmat}
  \end{mat}
  =
  \begin{mat}
    6  &-1  &2 \\
   -3 &-2  &7 \\
    0  &-1  &2
  \end{mat}
\end{equation*}
We should also note that the determinant of this is 
$a_{1,1}^{4-2}=3^2$ times the determinant of the $\nbyn{4}$ matrix~$A$.

We can finish by repeating the process with the $\nbyn{3}$ matrix.
This is Chi\`o's matrix of it; note that the determinant of this matrix
is $6$~times the determinant of~$C_3$.
\begin{equation*}
  C_2=
  \begin{mat}    
    \begin{vmat}
      6 &-1 \\
     -3 &-2 
    \end{vmat}
    &\begin{vmat}
      6 &2 \\
     -3 &7
    \end{vmat}           \\[3ex]
    \begin{vmat}
      6 &-1 \\
      0 &-1
    \end{vmat}
    &\begin{vmat}
      6 &2 \\
      0 &2
    \end{vmat}          
  \end{mat}
  =
  \begin{mat}
    -15 &48 \\
    -6 &12
  \end{mat}
\end{equation*}
The determinant of $C_2$ is $108$.
Dividing to get back to the original matrix gives
$\det(A)=108/(3^2\cdot 6)=2$.

Laplace's expansion formula for evaluating determinants is recursive because
it reduces the calculation of an~$\nbyn{n}$ determinant to the evaluation
of a number of $\nbyn{(n-1)}$~ones.
Chi\`o's formula is also recursive, so it is similar in spirit, 
but it reduces an~$\nbyn{n}$
determinant to a single~$\nbyn{(n-1)}$ determinant, the calculation
of which requires a number of $\nbyn{2}$ determinants.
However, for large matrices Gauss's method is better than either; for instance,
it takes roughly half as many operations as Chi\`o's 
method~\cite{FullerLogan}.


\begin{exercises}
  \item 
    Use Chi\`o's Method to find each determinant.
    \begin{exparts*}
      \partsitem 
        $
        \begin{vmat}
          1  &2  &3  \\
          4  &5  &6  \\
          7  &8  &9  
        \end{vmat}
        $
      \partsitem 
        $
        \begin{vmat}
          2  &1  &4  &0  \\
          0  &1  &4  &0  \\
          1  &1  &1  &1  \\
          0  &2  &1  &1  
        \end{vmat}
        $
    \end{exparts*}
    \begin{answer}
      \begin{exparts*}
        \partsitem 
        \partsitem 
      \end{exparts*} 
    \end{answer}
  \item What if $a_{1,1}$ is zero?
  \item count the operations involved in Sarrus's formula and 
   Chi\`o's formula for the $\nbyn{3}$ case.
  \item Prove Chi\`o's Formula.
    \begin{answer}
Consider an $\nbyn{n}$ matrix.
\begin{equation*}
  A=
  \begin{mat}
    a_{1,1}  &a_{1,2}   &\cdots &a_{1,n-1}  &a_{1,n} \\
    a_{2,1}  &a_{2,2}   &\cdots &a_{2,n-1}  &a_{2,n} \\
            &\vdots                         \\
    a_{n-1,1} &a_{n-1,2} &\cdots &a_{n-1,n-1} &a_{n-1,n}  \\ 
    a_{n,1}  &a_{n,2}   &\cdots &a_{n,n-1}  &a_{n,n} 
  \end{mat}
\end{equation*}
Consider the element in the lower right~$a_{n,n}$.

First rescale every row but the last by~$a_{n,n}$.
\begin{equation*}
  % \grstep[a_{n,n}\rho_2 \\ \vdots \\ a_{n,n}\rho_{n-1}]{a_{n,n}\rho_1}
  \begin{mat}
    a_{1,1}a_{n,n}   &a_{1,2}a_{n,n}  &\cdots &a_{1,n-1}a_{n,n}  &a_{1,n}a_{n,n} \\
    a_{2,1}a_{n,n}   &a_{2,2}a_{n,n}  &\cdots &a_{2,n-1}a_{n,n}  &a_{2,n}a_{n,n} \\
                  &\vdots                         \\
    a_{n-1,1}a_{n,n} &a_{n-1,2}a_{n,n} &\cdots &a_{n-1,n-1}a_{n,n} &a_{n-1,n}a_{n,n}  \\ 
    a_{n,1}        &a_{n,2}        &\cdots &a_{n,n-1}         &a_{n,n} 
  \end{mat}
  % \tag{*}
\end{equation*}
That rescales the determinant by a factor of~$a_{n,n}^{n-1}$. 

Next use $a_{n,n}$ to eliminate the other entries 
in the final column by performing the row operation 
$-a_{i,n}\rho_n+\rho_i$ on each row~$i\neq n$.
These row operations don't change the determinant. 
\begin{equation*}
  % \grstep[-a_{2,n}\rho_n+\rho_2 \\ \vdots \\ -a_{n-1,n}\rho_n+\rho_{n-1}]{-a_{1,n}\rho_n+\rho_1}
  \begin{mat}
    a_{1,1}a_{n,n}-a_{1,n}a_{n,1}   
        % &a_{1,2}a_{n,n}-a_{1,n}a_{n,2}  
        &\cdots 
        % &a_{1,n-1}a_{n,n}-a_{1,n}a_{n,n-1}  
        &a_{1,n}a_{n,n}-a_{1,n}a_{n,n} 
        \\
    a_{2,1}a_{n,n}-a_{2,n}a_{n,1}   
        % &a_{2,2}a_{n,n}-a_{2,n}a_{n,2}  
        &\cdots 
        % &a_{2,n-1}a_{n,n}-a_{2,n}a_{n,n-1}  
        &a_{2,n}a_{n,n}-a_{2,n}a_{n,n}    
        \\
        \vdots                         \\
    a_{n-1,1}a_{n,n}-a_{n-1,n}a_{n,1} 
       % &a_{n-1,2}a_{n,n}-a_{n-1,n}a_{n,2} 
       &\cdots 
       % &a_{n-1,n-1}a_{n,n}-a_{n-1,n}a_{n,n-1} 
       &a_{n-1,n}a_{n,n}-a_{n-1,n}a_{n,n}  
       \\ 
    a_{n,1}        
       % &a_{n,2}        
       &\cdots 
       % &a_{n,n-1}         
       &a_{n,n} 
  \end{mat}
  % \tag{**}
\end{equation*}
The matrix is too wide to show more but entries but for instance 
the first row's second entry is $a_{1,2}a_{n,n}-a_{1,n}a_{n,2}$, 
its third entry is $a_{1,3}a_{n,n}-a_{1,n}a_{n,3}$, \ldots, and 
entry~$n-1$ is $a_{1,n-1}a_{n,n}-a_{1,n}a_{n,n-1}$.

The final column's non-bottom entries $a_{1,n}$,\ldots, $a_{n-1,n}$ are all zeros. 
\begin{equation*}
  % \grstep[-a_{2,n}\rho_n+\rho_2 \\ \vdots \\ -a_{n-1,n}\rho_n+\rho_{n-1}]{-a_{1,n}\rho_n+\rho_1}
  \begin{mat}
    a_{1,1}a_{n,n}-a_{1,n}a_{n,1}   
        % &a_{1,2}a_{n,n}-a_{1,n}a_{n,2}  
        &\cdots 
        &a_{1,n-1}a_{n,n}-a_{1,n}a_{n,n-1}  
        &0                              \\
    a_{2,1}a_{n,n}-a_{2,n}a_{n,1}   
        % &a_{2,2}a_{n,n}-a_{2,n}a_{n,2}  
        &\cdots 
        &a_{2,n-1}a_{n,n}-a_{2,n}a_{n,n-1}  
        &0                              \\
        \vdots                         \\
    a_{n-1,1}a_{n,n}-a_{n-1,n}a_{n,1} 
       % &a_{n-1,2}a_{n,n}-a_{n-1,n}a_{n,2} 
       &\cdots 
       &a_{n-1,n-1}a_{n,n}-a_{n-1,n}a_{n,n-1} 
       &0                                 \\ 
    a_{n,1}        
       % &a_{n,2}        
       &\cdots 
       &a_{n,n-1}         
       &a_{n,n} 
  \end{mat}
  %\tag{*}
\end{equation*}
And, we've kept track so we know that the determinant of this matrix is 
$a_{n,n}^{n-1}$~times the determinant of~$A$.

Denote by~$C$ the $n,n$ minor of the matrix,
that is, the submatrix consisting of the first $n-1$ rows and columns.
The Laplace expansion down the final column of the above matrix  
gives that its determinant is $(-1)^{n+n}a_{n,n}\det(C)$.

Setting the two equal and
cancelling gives $a_{n,n}^{n-2}\det(A)=\det(C)$.       
    \end{answer}
\end{exercises}

\announcecomputercode
This implements Chi\`o's method \cite{ChessMaster}.
It is in the computer language Python. 
% http://stackoverflow.com/a/10037087/238366
\begin{lstlisting}
from itertools import product, islice

def det(M,prod=1):
    dim = len(M)
    if dim == 1:
        return prod * M.pop().pop()
    it = product(xrange(1,dim),repeat=2)
    prod *= M[0][0]
    return det([[M[x][y]-M[x][0]*(M[0][y]/M[0][0]) for x,y in islice(it,dim-1)] for i in xrange(dim-1)],prod)
\end{lstlisting}



\index{Chi\`o's method|)}
\endinput
