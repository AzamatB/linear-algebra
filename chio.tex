% Chapter 4, Topic _Linear Algebra_ Jim Hefferon
%  http://joshua.smcvt.edu/linearalgebra
%  2012-Jun-18
\topic{Chi\`o's Method}
\index{Chi\`o's method|(}

When doing Gauss's Method on a matrix that contains only integers
people often like to keep it that way.
To avoid fractions in the reduction of this matrix
\begin{equation*}
  A=
  \begin{mat}
    2 &1 &1 \\
    3 &4 &-1 \\
    1 &5 &1 
  \end{mat}
\end{equation*}
they may start by multiplying the lower rows
by~$2$
\begin{equation*}
  \grstep[2\rho_3]{2\rho_2}
  \begin{mat}
    2 &1 &1 \\
    6 &8 &-2 \\
    2 &10 &2 
  \end{mat}
  \tag{$*$}
\end{equation*}
so that elimination in the first column goes like this.
\begin{equation*}
  \grstep[-\rho_1+\rho_3]{-3\rho_1+\rho_2}
  \begin{mat}
    2 &1 &1 \\
    0 &5 &-5 \\
    0 &8 &0 
  \end{mat}
  \tag{$**$}
\end{equation*}
This all-integer approach is easier for mental calculations. 
And, using integer arithmetic on a computer
avoids some sticky issues involving floating point calculations \cite{Kahan}.
So there are sound reasons for this approach.

Another advantage of this approach is that
we can easily apply Laplace's expansion to the 
first column of~($**$) and then get the determinant by 
remembering to divide by $4$ because of~($*$).

Here is the general $\nbyn{3}$ case of this approach to finding the determinant.
First, assuming $a_{1,1}\neq 0$, we can rescale the lower rows.
\begin{equation*}
  A=
  \begin{mat}
    a_{1,1} &a_{1,2} &a_{1,3}  \\
    a_{2,1} &a_{2,2} &a_{2,3}  \\
    a_{3,1} &a_{3,2} &a_{3,3}  
  \end{mat}
  \grstep[a_{1,1}\rho_3]{a_{1,1}\rho_2}
  \begin{mat}
    a_{1,1}       &a_{1,2}       &a_{1,3}        \\
    a_{2,1}a_{1,1} &a_{2,2}a_{1,1} &a_{2,3}a_{1,1}   \\
    a_{3,1}a_{1,1} &a_{3,2}a_{1,1} &a_{3,3}a_{1,1}   
  \end{mat}
\end{equation*}
This rescales the determinant by $a_{1,1}^2$.
Now eliminate down the first column.
\begin{equation*}
   \grstep[-a_{3,1}\rho_1+\rho_3]{-a_{2,1}\rho_1+\rho_2}
  \begin{mat}
    a_{1,1}       &a_{1,2}       &a_{1,3}       \\
    0 
       &a_{2,2}a_{1,1}-a_{2,1}a_{1,2} 
       &a_{2,3}a_{1,1}-a_{2,1}a_{1,3}               \\
    0 
       &a_{3,2}a_{1,1}-a_{3,1}a_{1,2} 
       &a_{3,3}a_{1,1}-a_{3,1}a_{1,3}                   
  \end{mat}
\end{equation*}
Let $C$ be the $1,1$ minor.
By Laplace the determinant of the above matrix is $a_{1,1}\det(C)$.
We thus have $a_{1,1}^2\det(A)=a_{1,1}\det(C)$ and since $a_{1,1}\neq 0$
this gives $\det(A)=\det(C)/a_{1,1}$.

To do larger matrices
we must see how to compute the minor's entries.
The pattern above is that each element of the minor is a 
$\nbyn{2}$~determinant.
For instance, the entry in the minor's upper left
$a_{2,2}a_{1,1}-a_{2,1}a_{1,2}$, which is the $2,2$~entry in the above matrix,
is the determinant of the matrix of these
four elements of $A$.
\begin{equation*}
  \begin{mat}
    \highlight{$a_{1,1}$} &\highlight{$a_{1,2}$} &a_{1,3}  \\
    \highlight{$a_{2,1}$} &\highlight{$a_{2,2}$} &a_{2,3}  \\
    a_{3,1}               &a_{3,2}               &a_{3,3}  
  \end{mat}
\end{equation*}
And the minor's lower left, the $3,2$ entry from above, 
is the determinant of the matrix 
of these four.
\begin{equation*}
  \begin{mat}
    \highlight{$a_{1,1}$} &\highlight{$a_{1,2}$} &a_{1,3}  \\
    a_{2,1}               &a_{2,2}               &a_{2,3}  \\
    \highlight{$a_{3,1}$} &\highlight{$a_{3,2}$} &a_{3,3}  
  \end{mat}
\end{equation*}

So, where $A$ is~$\nbyn{n}$ for $n\geq 3$, we let Chi\`o's matrix $C$ be the
$\nbyn{(n-1)}$ matrix whose $i,j$ entry is the determinant
\begin{equation*}
  \begin{vmat}
    a_{1,1}  &a_{1,j+1} \\
    a_{i+1,1}    &a_{i+1,j+1}
  \end{vmat}
\end{equation*}
where $1<i,j\leq n$.
\definend{Chi\`o's method}\index{Chi\`o's method} 
for finding the determinant of $A$ is that
if $a_{1,1}\neq 0$ then
$\det(A)=\det(C)/a_{1,1}^{n-2}$.
(By the way,
nothing in Chi\`o's formula requires that the numbers be integers; it applies
to reals as well.)

To illustrate we  find the determinant of this $\nbyn{3}$ matrix.
\begin{equation*} 
  A=
  \begin{mat}
    2 &1 &1 \\
    3 &4 &-1 \\
    1 &5 &1 
  \end{mat}
\end{equation*}
This is Chi\`o's matrix.
\begin{equation*}
  C=
  \begin{mat}
    \begin{vmat}
      2 &1 \\
      3 &4
    \end{vmat}
   &
   \begin{vmat}
     2 &1 \\
     3 &-1
   \end{vmat}        \\[3ex]      
   \begin{vmat}
     2 &1 \\
     1 &5
   \end{vmat}
   &
   \begin{vmat}
     2 &1 \\
     1 &1
   \end{vmat}
  \end{mat}
  =
  \begin{mat}
    5  &-5  \\
    9  &1
  \end{mat}
\end{equation*}
The formula for $\nbyn{3}$ matrices
$det(A)=\det(C)/a_{1,1}$ gives $\det(A)=(50/2)=25$.

For a larger determinant we must do multiple steps
but each involves only $\nbyn{2}$ determinants. 
So we can often calculate the determinant just by writing down a bit of
intermediate information.
For instance, with this $\nbyn{4}$ matrix
\begin{equation*}
  A=
  \begin{mat}
    3  &0  &1  &1  \\
    1  &2  &0  &1  \\
    2  &-1 &0  &3  \\
    1  &0  &0  &1
  \end{mat} 
\end{equation*}
we can mentally doing each of the 
$\nbyn{2}$ calculations and only write down the
$\nbyn{3}$ result. 
\begin{equation*}
  C_3=
  \begin{mat}
    \begin{vmat}
      3 &0 \\
      1 &2
    \end{vmat}
    &\begin{vmat}
     3 &1 \\
     1 &0 
    \end{vmat}
    &\begin{vmat}
     3 &1 \\
     1 &1
    \end{vmat}                \\[3ex]
    \begin{vmat}
     3 &0 \\
     2 &-1
    \end{vmat}
    &\begin{vmat}
     3 &1 \\
     2 &0
    \end{vmat}
    &\begin{vmat}
     3 &1 \\
     2 &3
    \end{vmat}             \\[3ex]
    \begin{vmat}
     3 &0 \\
     1 &0
    \end{vmat}
    &\begin{vmat}
     3 &1 \\
     1 &0
    \end{vmat}
    &\begin{vmat}
     3 &1 \\
     1 &1
    \end{vmat}
  \end{mat}
  =
  \begin{mat}
    6  &-1  &2 \\
   -3 &-2  &7 \\
    0  &-1  &2
  \end{mat}
\end{equation*}
Note that the determinant of this is 
$a_{1,1}^{4-2}=3^2$ times the determinant of~$A$.

To finish, iterate.
Here is Chi\`o's matrix of $C_3$.
\begin{equation*}
  C_2=
  \begin{mat}    
    \begin{vmat}
      6 &-1 \\
     -3 &-2 
    \end{vmat}
    &\begin{vmat}
      6 &2 \\
     -3 &7
    \end{vmat}           \\[3ex]
    \begin{vmat}
      6 &-1 \\
      0 &-1
    \end{vmat}
    &\begin{vmat}
      6 &2 \\
      0 &2
    \end{vmat}          
  \end{mat}
  =
  \begin{mat}
    -15 &48 \\
    -6 &12
  \end{mat}
\end{equation*}
The determinant of this matrix
is $6$~times the determinant of~$C_3$.
The determinant of $C_2$ is $108$.
So
$\det(A)=108/(3^2\cdot 6)=2$.

Laplace's expansion formula  
reduces the calculation of an~$\nbyn{n}$ determinant to the evaluation
of a number of $\nbyn{(n-1)}$~ones.
Chi\`o's formula is also recursive 
but it reduces an~$\nbyn{n}$
determinant to a single~$\nbyn{(n-1)}$ determinant, calculated
from a number of $\nbyn{2}$ determinants.
However, for large matrices Gauss's Method is better than either of these; 
for instance,
it takes roughly half as many operations as Chi\`o's 
Method~\cite{FullerLogan}.


\begin{exercises}
  \item 
    Use Chi\`o's Method to find each determinant.
    \begin{exparts*}
      \partsitem 
        $
        \begin{vmat}
          1  &2  &3  \\
          4  &5  &6  \\
          7  &8  &9  
        \end{vmat}
        $
      \partsitem 
        $
        \begin{vmat}
          2  &1  &4  &0  \\
          0  &1  &4  &0  \\
          1  &1  &1  &1  \\
          0  &2  &1  &1  
        \end{vmat}
        $
    \end{exparts*}
    \begin{answer}
      \begin{exparts*}
        \partsitem Chi\`o's matrix is
        \begin{equation*}
          C=
          \begin{mat}
            -3 &-6 \\
            -6 &-12
          \end{mat}
        \end{equation*}
        and its determinant is $0$
        \partsitem Start with
        \begin{equation*}
          C_3=
          \begin{mat}
            2 &8  &0 \\
            1 &-2 &2 \\
            4 &2  &2
          \end{mat}
        \end{equation*}
        and then the next step
        \begin{equation*}
          C_2=
          \begin{mat}
            -12 &4 \\
            -28 &4
          \end{mat}
        \end{equation*}
        with determinant $\det(C_2)=64$. 
        The determinant of the original matrix is thus $64/(2^2\cdot 2^1)=8$
      \end{exparts*} 
    \end{answer}
  \item What if $a_{1,1}$ is zero?
    \begin{answer}
      The same construction as was used for the $\nbyn{3}$~case above shows
      that in place of $a_{1,1}$ we can select any nonzero entry~$a_{i,j}$.
      Entry $c_{p,q}$ of Chi\`o's matrix is the value of this determinant
      \begin{equation*}
        \begin{vmat}
          a_{1,1}    &a_{1,q+1} \\
          a_{p+1,1}  &a_{p+1,q+1}
        \end{vmat}
      \end{equation*}
      where $p+1\neq i$ and $q+1\neq j$.
    \end{answer}
  \item The \definend{Rule of Sarrus}\index{Sarrus, Rule of}\index{Rule of Sarrus}
    is a mnemonic that many people learn 
    for the $\nbyn{3}$ determinant formula.
    To the right of the matrix, copy the first two columns. 
    \begin{equation*}
      \begin{array}{ccc|cc}
        a &b &c &a &b \\
        d &e &f &d &e \\
        g &h &i &g &h
      \end{array}
    \end{equation*}
    Then the determinant is the sum of the
    three upper-left to lower-right diagonals minus the
    three lower-left to upper-right diagonals
    $aei+bfg+cdh-gec-hfa-idb$.
   Count the operations involved in Sarrus's formula and 
   Chi\`o's formula for the $\nbyn{3}$ case and see which uses fewer.
   \begin{answer}
     Sarrus's formula uses $12$~multiplications and $5$ additions (including
     the subtractions in with the additions).
     Chi\`o's formula uses two multiplications and an addition
     (which is actually a subtraction)
     for each of the four $\nbyn{2}$~determinants, and another two
     multiplications and an addition for the $\nbyn{2}$~Chi\'o's determinant, as
     well as a final division by $a_{1,1}$.
     That's eleven multiplication/divisions and five addition/subtractions.
     So Chi\`o is the winner.
   \end{answer}
  \item Prove Chi\`o's formula.
    \begin{answer}
      Consider an $\nbyn{n}$ matrix.
      \begin{equation*}
        A=
        \begin{mat}
           a_{1,1}  &a_{1,2}   &\cdots &a_{1,n-1}  &a_{1,n} \\
           a_{2,1}  &a_{2,2}   &\cdots &a_{2,n-1}  &a_{2,n} \\
                  &\vdots                         \\
           a_{n-1,1} &a_{n-1,2} &\cdots &a_{n-1,n-1} &a_{n-1,n}  \\ 
           a_{n,1}  &a_{n,2}   &\cdots &a_{n,n-1}  &a_{n,n} 
        \end{mat}
      \end{equation*}
      Rescale every row but the first by~$a_{1,1}$.
      \begin{equation*}
        \grstep[a_{1,1}\rho_3 \\ \vdots \\ a_{1,1}\rho_{n}]{a_{1,1}\rho_2}  
        \begin{mat}
          a_{1,1}         &a_{1,2}        &\cdots &a_{1,n-1}        &a_{1,n} \\
          a_{2,1}a_{1,1}   &a_{2,2}a_{1,1}  &\cdots &a_{2,n-1}a_{1,1}  &a_{2,n}a_{1,1} \\
                        &\vdots                         \\
          a_{n-1,1}a_{1,1} &a_{n-1,2}a_{1,1} &\cdots &a_{n-1,n-1}a_{1,1} &a_{n-1,n}a_{1,1}\\
          a_{n,1}a_{1,1}  &a_{n,2}a_{1,1}   &\cdots &a_{n,n-1}a_{1,1}  &a_{n,n}a_{1,1} 
        \end{mat}
      \end{equation*}
      That rescales the determinant by a factor of~$a_{1,1}^{n-1}$. 

      Next perform the row operation 
      $-a_{i,1}\rho_1+\rho_i$ on each row~$i>1$.
      These row operations don't change the determinant. 
      \begin{equation*}
        \grstep[-a_{3,1}\rho_1+\rho_3 \\ \vdotswithin{-a_{3,1}\rho_1+\rho_3} \\ -a_{n,1}\rho_1+\rho_n]{-a_{2,1}\rho_1+\rho_2}         
      \end{equation*}
      The result is a matrix whose first row is unchanged, whose 
      first column is all zero's (except for
      the $1,1$ entry of $a_{1,1}$), and whose remaining entries are these. 
      \begin{equation*}
        \begin{mat}
              a_{1,2}  
              &\cdots 
              &a_{1,n-1}  
              &a_{1,n} 
              \\
              a_{2,2}a_{1,1}-a_{2,1}a_{1,2}  
              &\cdots 
              &a_{2,n-1}a_{n,n}-a_{2,n-1}a_{1,n-1}  
              &a_{2,n}a_{n,n}-a_{2,n}a_{1,n}    
              \\
              \vdots                         \\
             a_{n,2}a_{1,1}-a_{n,1}a_{1,2} 
             &\cdots 
             &a_{n,n-1}a_{1,1}-a_{n,1}a_{1,n-1} 
             &a_{n,n}a_{1,1}-a_{n,1}a_{1,n}  
        \end{mat}
      \end{equation*}
      The determinant of this matrix is 
      $a_{1,1}^{n-1}$~times the determinant of~$A$.

      Denote by~$C$ the $1,1$ minor of the matrix,
      that is, the submatrix consisting of the first $n-1$ rows and columns.
      The Laplace expansion down the final column of the above matrix  
      gives that its determinant is $(-1)^{1+1}a_{1,1}\det(C)$.

      If $a_{1,1}\neq 0$ then setting the two equal and 
      canceling gives $\det(A)=\det(C)/a_{n,n}^{n-2}$.       
    \end{answer}
\end{exercises}

\announcecomputercode
This implements Chi\`o's Method.
It is in the computer language Python (to make it more readable 
the code avoids some Python facilities).
Note the recursive call in the final line of 
\lstinline!chio_det!. 
\begin{lstlisting}
#!/usr/bin/python
# chio.py
#  Calculate a determinant using Chio's method.
# Jim Hefferon; Public Domain
# For demonstration only; for instance, does not handle the M[0][0]=0 case

def det_two(a,b,c,d):
    """Return the determinant of the 2x2 matrix [[a,b], [c,d]]"""
    return a*d-b*c

def chio_mat(M):
    """Return the Chio matrix as a list of the rows
        M  nxn matrix, list of rows"""
    dim=len(M)
    C=[]
    for row in range(1,dim):
        C.append([])
        for col in range(1,dim):  
            C[-1].append(det_two(M[0][0], M[0][col], M[row][0], M[row][col]))
    return C

def chio_det(M,show=None):
    """Find the determinant of M by Chio's method
        M  mxm matrix, list of rows"""
    dim=len(M)
    key_elet=M[0][0]
    if dim==1:
        return key_elet
    return chio_det(chio_mat(M))/(key_elet**(dim-2))


if __name__=='__main__':
    M=[[2,1,1], [3,4,-1], [1,5,1]]
    print "M=",M
    print "Det is", chio_det(M)
\end{lstlisting}
This is the result of calling the program from my command line.
\begin{lstlisting}
$ python chio.py
M=[[2, 1, 1], [3, 4, -1], [1, 5, 1]]
Det is 25
\end{lstlisting}
\index{Chi\`o's method|)}
\endinput
