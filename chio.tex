% Chapter 4, Topic _Linear Algebra_ Jim Hefferon
%  http://joshua.smcvt.edu/linalg.html
%  2001-Jun-12
\topic{Chi\`o's Method}
\index{Chi\`o's method|(}

When doing Gauss's Method on a matrix that contains only integers
\begin{equation*}
  A=
  \begin{mat}
    2 &1 &1 \\
    3 &4 &-1 \\
    1 &5 &1 
  \end{mat}
\end{equation*}
people often prefer to keep it that way.
Instead of the row operations $-(3/2)\rho_1+\rho_2$ and
$-(1/2)\rho_1+\rho_3$ they may start by multiplying the rows below the first
by~$2$
\begin{equation*}
  \grstep[2\rho_3]{2\rho_2}
  \begin{mat}
    2 &1 &1 \\
    6 &8 &-2 \\
    2 &10 &2 
  \end{mat}
  \tag{*}
\end{equation*}
and then complete the eliminatation in the first column.
\begin{equation*}
  \grstep[-\rho_1+\rho_3]{-3\rho_1+\rho_2}
  \begin{mat}
    2 &1 &1 \\
    0 &5 &-5 \\
    0 &8 &0 
  \end{mat}
  \tag{**}
\end{equation*}
(This is easier for mental calculations but it also has the advantage
that on a computer integer arithmetic 
avoids issues from floating point calculations \cite{Kahan}.)

Observe that we can easily apply Laplace's expansion to the first column of that
matrix, and then we get the determinant of $A$ by 
remembering to divide the determinant in ($**$) by $4$ because of~($*$).

The issue with this way of computing determinants is knowing 
the entries of the minor.
This $\nbyn{3}$ case shows the pattern.
\begin{equation*}
  A=
  \begin{mat}
    a_{1,1} &a_{1,2} &a_{1,3}  \\
    a_{2,1} &a_{2,2} &a_{2,3}  \\
    a_{3,1} &a_{3,2} &a_{3,3}  
  \end{mat}
\end{equation*}
Rescale all rows except the first.
\begin{equation*}
  \grstep[a_{1,1}\rho_3]{a_{1,1}\rho_2}
  \begin{mat}
    a_{1,1}       &a_{1,2}       &a_{1,3}        \\
    a_{2,1}a_{1,1} &a_{2,2}a_{1,1} &a_{2,3}a_{1,1}   \\
    a_{3,1}a_{1,1} &a_{3,2}a_{1,1} &a_{3,3}a_{1,1}   
  \end{mat}
\end{equation*}
Eliminate down the first column.
\begin{equation*}
  \grstep[-a_{3,1}\rho_1+\rho_3]{-a_{2,1}\rho_1+\rho_2}
  \begin{mat}
    a_{1,1}       &a_{1,2}       &a_{1,3}       \\
    a_{2,1}a_{1,1}-a_{2,1}a_{1,1} 
       &a_{2,2}a_{1,1}-a_{2,1}a_{1,2} 
       &a_{2,3}a_{1,1}-a_{2,1}a_{1,3}               \\
    a_{3,1}a_{1,1}-a_{3,1}a_{1,1} 
       &a_{3,2}a_{1,1}-a_{3,1}a_{1,2} 
       &a_{3,3}a_{1,1}-a_{3,1}a_{1,3}                   
  \end{mat}
\end{equation*}
This is the result.
\begin{equation*}
  \grstep[-a_{3,1}\rho_1+\rho_3]{-a_{2,1}\rho_1+\rho_2}
  \begin{mat}
    a_{1,1}       &a_{1,2}       &a_{1,3}       \\
    0 
       &a_{2,2}a_{1,1}-a_{2,1}a_{1,2} 
       &a_{2,3}a_{1,1}-a_{2,1}a_{1,3}               \\
    0 
       &a_{3,2}a_{1,1}-a_{3,1}a_{1,2} 
       &a_{3,3}a_{1,1}-a_{3,1}a_{1,3}                   
  \end{mat}
\end{equation*}
Its determinant is 
is $a_{1,1}$ times the determinant of the $1,1$ minor.

We can remember the elements of that minor because each is a determinant.
For instance, the entry in the upper left of that minor, the
$2,2$~entry above, is the determinant of the matrix of these
four elements of $A$.
\begin{equation*}
  \begin{mat}
    \highlight{$a_{1,1}$} &\highlight{$a_{1,2}$} &a_{1,3}  \\
    \highlight{$a_{2,1}$} &\highlight{$a_{2,2}$} &a_{2,3}  \\
    a_{3,1}               &a_{3,2}               &a_{3,3}  
  \end{mat}
\end{equation*}
And the $3,2$ entry from~($**$) if the determinant of the matrix 
of these elements of $A$.
\begin{equation*}
  \begin{mat}
    \highlight{$a_{1,1}$} &\highlight{$a_{1,2}$} &a_{1,3}  \\
    a_{2,1}               &a_{2,2}               &a_{2,3}  \\
    \highlight{$a_{3,1}$} &\highlight{$a_{3,2}$} &a_{3,3}  
  \end{mat}
\end{equation*}

So where $A$ is~$\nbyn{n}$, we define Chi\`o's matrix $C_A$ to be the
$\nbyn{(n-1)}$ matrix whose $i,j$ entry is the determinant
\begin{equation*}
  \begin{vmat}
    a_{1,1}  &a_{1,j} \\
    a_{i,1}    &a_{i,j}
  \end{vmat}
\end{equation*}
where $1<i,j\leq n$.
\definend{Chi\`o's method} for finding the determinant of $A$ is that
$a_{1,1}^{n-2}\det(A)=\det(C_A)$.

We will compute the determinant of this $\nbyn{3}$ matrix.
\begin{equation*}
  A=
  \begin{mat}
    2 &1 &1 \\
    3 &4 &-1 \\
    1 &5 &1 
  \end{mat}
\end{equation*}
This is Chi\`o's matrix.
\begin{equation*}
  C=
  \begin{mat}
    \begin{vmat}
      2 &1 \\
      3 &4
    \end{vmat}
   &
   \begin{vmat}
     2 &1 \\
     3 &-1
   \end{vmat}        \\
   \begin{vmat}
     2 &1 \\
     1 &5
   \end{vmat}
   &
   \begin{vmat}
     2 &1 \\
     1 &1
   \end{vmat}
  \end{mat}
  =
  \begin{mat}
    5  &-5  \\
    9  &1
  \end{mat}
\end{equation*}
Chi\`o's formula 
$a_{1,1}^{n-2}\det(A)=\det(C)$ gives $2\cdot\det(A)=50$
so $\det(A)=25$.

For a larger determinant we must do multiple steps.
But each step involves only $\nbyn{2}$ determinants, 
and so it may be done mentally.
\begin{equation*}
  A=
  \begin{mat}
    3  &0  &1  &1  \\
    1  &2  &0  &1  \\
    2  &-1 &0  &3  \\
    1  &0  &0  &1
  \end{mat} 
\end{equation*}
By Chi\`o, $3^2\det(A)=det(C)$ where $C$ is this.
\begin{equation*}
  \begin{mat}
    \begin{vmat}
      3 &0 \\
      1 &2
    \end{vmat}
    &\begin{vmat}
     3 &1 \\
     1 &0 
    \end{vmat}
    &\begin{vmat}
     3 &1 \\
     1 &1
    \end{vmat}                \\
    \begin{vmat}
     3 &0 \\
     2 &-1
    \end{vmat}
    &\begin{vmat}
     3 &1 \\
     2 &0
    \end{vmat}
    &\begin{vmat}
     3 &1 \\
     2 &3
    \end{vmat}             \\
    \begin{vmat}
     3 &0 \\
     1 &0
    \end{vmat}
    &\begin{vmat}
     3 &1 \\
     1 &0
    \end{vmat}
    &\begin{vmat}
     3 &1 \\
     1 &1
    \end{vmat}
  \end{mat}
  =
  \begin{mat}
    6  &-1  &2 \\
   -3 &-2  &7 \\
    0  &-1  &2
  \end{mat}
\end{equation*}

Laplace's expansion formula for evaluating determinants is recursive because
it reduces the calculation of an~$\nbyn{n}$ determinant to the evaluation
of a number of $\nbyn{(n-1)}$~ones.

The formula we see here is similar in spirit but it reduces an~$\nbyn{n}$
determinant to a single~$\nbyn{(n-1)}$ determinant, the calculation
of which requires a number of $\nbyn{2}$ determinants.

Consider an $\nbyn{n}$ matrix.
\begin{equation*}
  A=
  \begin{mat}
    a_{1,1}  &a_{1,2}   &\cdots &a_{1,n-1}  &a_{1,n} \\
    a_{2,1}  &a_{2,2}   &\cdots &a_{2,n-1}  &a_{2,n} \\
            &\vdots                         \\
    a_{n-1,1} &a_{n-1,2} &\cdots &a_{n-1,n-1} &a_{n-1,n}  \\ 
    a_{n,1}  &a_{n,2}   &\cdots &a_{n,n-1}  &a_{n,n} 
  \end{mat}
\end{equation*}
Consider the element in the lower right~$a_{n,n}$.

First rescale every row but the last by~$a_{n,n}$.
\begin{equation*}
  % \grstep[a_{n,n}\rho_2 \\ \vdots \\ a_{n,n}\rho_{n-1}]{a_{n,n}\rho_1}
  \begin{mat}
    a_{1,1}a_{n,n}   &a_{1,2}a_{n,n}  &\cdots &a_{1,n-1}a_{n,n}  &a_{1,n}a_{n,n} \\
    a_{2,1}a_{n,n}   &a_{2,2}a_{n,n}  &\cdots &a_{2,n-1}a_{n,n}  &a_{2,n}a_{n,n} \\
                  &\vdots                         \\
    a_{n-1,1}a_{n,n} &a_{n-1,2}a_{n,n} &\cdots &a_{n-1,n-1}a_{n,n} &a_{n-1,n}a_{n,n}  \\ 
    a_{n,1}        &a_{n,2}        &\cdots &a_{n,n-1}         &a_{n,n} 
  \end{mat}
  % \tag{*}
\end{equation*}
That rescales the determinant by a factor of~$a_{n,n}^{n-1}$. 

Next use $a_{n,n}$ to eliminate the other entries 
in the final column by performing the row operation 
$-a_{i,n}\rho_n+\rho_i$ on each row~$i\neq n$.
These row operations don't change the determinant. 
\begin{equation*}
  % \grstep[-a_{2,n}\rho_n+\rho_2 \\ \vdots \\ -a_{n-1,n}\rho_n+\rho_{n-1}]{-a_{1,n}\rho_n+\rho_1}
  \begin{mat}
    a_{1,1}a_{n,n}-a_{1,n}a_{n,1}   
        % &a_{1,2}a_{n,n}-a_{1,n}a_{n,2}  
        &\cdots 
        % &a_{1,n-1}a_{n,n}-a_{1,n}a_{n,n-1}  
        &a_{1,n}a_{n,n}-a_{1,n}a_{n,n} 
        \\
    a_{2,1}a_{n,n}-a_{2,n}a_{n,1}   
        % &a_{2,2}a_{n,n}-a_{2,n}a_{n,2}  
        &\cdots 
        % &a_{2,n-1}a_{n,n}-a_{2,n}a_{n,n-1}  
        &a_{2,n}a_{n,n}-a_{2,n}a_{n,n}    
        \\
        \vdots                         \\
    a_{n-1,1}a_{n,n}-a_{n-1,n}a_{n,1} 
       % &a_{n-1,2}a_{n,n}-a_{n-1,n}a_{n,2} 
       &\cdots 
       % &a_{n-1,n-1}a_{n,n}-a_{n-1,n}a_{n,n-1} 
       &a_{n-1,n}a_{n,n}-a_{n-1,n}a_{n,n}  
       \\ 
    a_{n,1}        
       % &a_{n,2}        
       &\cdots 
       % &a_{n,n-1}         
       &a_{n,n} 
  \end{mat}
  % \tag{**}
\end{equation*}
The matrix is too wide to show more but entries but for instance 
the first row's second entry is $a_{1,2}a_{n,n}-a_{1,n}a_{n,2}$, 
its third entry is $a_{1,3}a_{n,n}-a_{1,n}a_{n,3}$, \ldots, and 
entry~$n-1$ is $a_{1,n-1}a_{n,n}-a_{1,n}a_{n,n-1}$.

The final column's non-bottom entries $a_{1,n}$,\ldots, $a_{n-1,n}$ are all zeros. 
\begin{equation*}
  % \grstep[-a_{2,n}\rho_n+\rho_2 \\ \vdots \\ -a_{n-1,n}\rho_n+\rho_{n-1}]{-a_{1,n}\rho_n+\rho_1}
  \begin{mat}
    a_{1,1}a_{n,n}-a_{1,n}a_{n,1}   
        % &a_{1,2}a_{n,n}-a_{1,n}a_{n,2}  
        &\cdots 
        &a_{1,n-1}a_{n,n}-a_{1,n}a_{n,n-1}  
        &0                              \\
    a_{2,1}a_{n,n}-a_{2,n}a_{n,1}   
        % &a_{2,2}a_{n,n}-a_{2,n}a_{n,2}  
        &\cdots 
        &a_{2,n-1}a_{n,n}-a_{2,n}a_{n,n-1}  
        &0                              \\
        \vdots                         \\
    a_{n-1,1}a_{n,n}-a_{n-1,n}a_{n,1} 
       % &a_{n-1,2}a_{n,n}-a_{n-1,n}a_{n,2} 
       &\cdots 
       &a_{n-1,n-1}a_{n,n}-a_{n-1,n}a_{n,n-1} 
       &0                                 \\ 
    a_{n,1}        
       % &a_{n,2}        
       &\cdots 
       &a_{n,n-1}         
       &a_{n,n} 
  \end{mat}
  %\tag{*}
\end{equation*}
And, we've kept track so we know that the determinant of this matrix is 
$a_{n,n}^{n-1}$~times the determinant of~$A$.

Denote by~$C$ the $n,n$ minor of the matrix,
that is, the submatrix consisting of the first $n-1$ rows and columns.
The Laplace expansion down the final column of the above matrix  
gives that its determinant is $(-1)^{n+n}a_{n,n}\det(C)$.

Setting the two equal and
cancelling gives $a_{n,n}^{n-2}\det(A)=\det(C)$. 

The terms in Chi\`o's matrix~$C$ are easy to remember 
as determinants of $\nbyn{2}$~matrices.
For instance
in this $\nbyn{4}$ matrix the $c_{1,2}$ entry $a_{1,2}a_{4,4}-a_{1,4}a_{4,2}$
comes from
taking the determinant of the $\nbyn{2}$ matrix of shaded terms.
\begin{equation*}
  \begin{mat}
    a_{1,1} &\highlight{$a_{1,2}$} &a_{1,3} &\highlight{$a_{1,4}$}  \\
    a_{2,1} &a_{2,2}               &a_{2,3} &a_{2,4}  \\
    a_{3,1} &a_{3,2}               &a_{3,3} &a_{3,4}  \\
    a_{4,1} &\highlight{$a_{4,2}$} &a_{4,3} &\highlight{$a_{4,4}$}  
  \end{mat}
\end{equation*}

We will compute the determinant of this $\nbyn{3}$ matrix by this
method.
\begin{equation*}
  A=
  \begin{mat}
    2 &1 &1 \\
    3 &4 &-1 \\
    1 &5 &1 
  \end{mat}
\end{equation*}
Note that $a_{3,3}=1$, which is an especially convenient term for 
this method.
We form the $3,3$ minor.
\begin{equation*}
  C=
  \begin{mat}
    \begin{vmat}
      2 &1 \\
      1 &1
    \end{vmat}
   &
   \begin{vmat}
     1 &1 \\
     5 &1
   \end{vmat}        \\
   \begin{vmat}
     3 &-1 \\
     1 &1
   \end{vmat}
   &
   \begin{vmat}
     4 &-1 \\
     5 &1
   \end{vmat}
  \end{mat}
  =
  \begin{mat}
    1  &-4  \\
    4  &9
  \end{mat}
\end{equation*}
Here Chi\`o's formula is 
$a_{n,n}^{n-2}\det(A)=\det(C)=1\cdot\det(A)=25$
so $\det(A)=25$.

Because this method leads to $\nbyn{2}$ determinants, it makes feasible
mentally doing some determinants that would be unwieldy with Gauss's 
method or the Laplace expansion.
But for large matrices doing computations with this method is 
known to take twice as many operations as Gauss's method.

Note that this method will, if $A$ has integer entries, 
return a integer determinant for $C$.  
This is useful for mental calculations, and also useful for 
avoiding some stick issues with computer calculations using
floating point numbers.
\url{http://www.cs.berkeley.edu/~wkahan/MathH110/chio.pdf} 

This material is derived from \cite{FullerLogan}.

% http://stackoverflow.com/a/10037087/238366
\begin{lstlisting}
from itertools import product, islice

def det(M,prod=1):
    dim = len(M)
    if dim == 1:
        return prod * M.pop().pop()
    it = product(xrange(1,dim),repeat=2)
    prod *= M[0][0]
    return det([[M[x][y]-M[x][0]*(M[0][y]/M[0][0]) for x,y in islice(it,dim-1)] for i in xrange(dim-1)],prod)
\end{lstlisting}

See also http://www.cs.berkeley.edu/~wkahan/MathH110/chio.pdf
\begin{exercises}
  \item 
    Use Cramer's Rule to solve each for each of the variables.
    \begin{exparts*}
      \partsitem $\begin{linsys}{2}
                    x  &- &y  &=  &4  \\
                   -x  &+ &2y &=  &-7
                  \end{linsys}$
      \partsitem $\begin{linsys}{2}
                    -2x  &+  &y  &=  &-2 \\
                      x  &-  &2y &=  &-2  
                  \end{linsys}$
    \end{exparts*}
    \begin{answer}
      \begin{exparts*}
        \partsitem 
          \begin{equation*}
            x=
             \frac{ \begin{vmat}[r]
                       4  &-1  \\
                      -7  &2   
                    \end{vmat}  }{
                    \begin{vmat}[r]
                       1  &-1  \\
                      -1  &2 
                    \end{vmat}  }
            =\frac{1}{1}=1
            \qquad
            y=
             \frac{ \begin{vmat}[r]
                       1  &4  \\
                      -1  &-7   
                    \end{vmat}  }{
                    \begin{vmat}[r]
                       1  &-1  \\
                      -1  &2 
                    \end{vmat}  }
            =\frac{-3}{1}=-3
          \end{equation*}
        \partsitem $x=2$, $y=2$
      \end{exparts*} 
    \end{answer}
\end{exercises}
\index{Chi\`o's method|)}
\endinput
