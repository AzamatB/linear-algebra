% Chapter 1, Topic from _Linear Algebra_ Jim Hefferon
%  http://joshua.smcvt.edu/linalg.html
%  2001-Jun-09
\topic{Accuracy of Computations}
\index{accuracy! of Gauss' method|(}
\index{Gauss' method! accuracy|(}
Gauss' method lends itself nicely to computerization.
The code below illustrates.
It operates on an $\nbyn{n}$ matrix \texttt{a}, 
doing row combinations using the first row, then
the second row, etc.
\begin{lstlisting}
for(row=1;row<=n-1;row++){
  for(row_below=row+1;row_below<=n;row_below++){
    multiplier=a[row_below,row]/a[row,row];
    for(col=row; col<=n; col++){
      a[row_below,col]-=multiplier*a[row,col];
    }
  }
}
\end{lstlisting}
This is in the C~language.\index{C language}
The loop 
\lstinline!for(row=1;row<=n-1;row++){ .. }!
initializes \texttt{row} at $1$ and then iterates while
\texttt{row} is less than or equal to $n-1$, each time through
incrementing \lstinline!row! by one with the 
\lstinline!++! operation.
The other non-obvious language 
construct is that the `\texttt{-=}' in the innermost
loop amounts to the
$\text{\lstinline!a[row\_below,col]!}=
           -\text{\lstinline!multiplier!}\cdot\text{\lstinline!a[row,col]!}
           +\text{\lstinline!a[row\_below,col]!}$
operation.

This code provides a quick take on how mechanizing Gauss' method.
But it is naive in many ways.
For one thing, it assumes that 
the entry in the
$\text{\lstinline!row!},\text{\lstinline!row!}$ position is nonzero.
One way that the code needs additional development to make it practical
is to cover
the case where finding a zero in that location leads to a row swap or to the
conclusion that the matrix is singular.

Adding some \lstinline!if! statements to cover those cases is not hard,
but we will instead consider some other ways in which the code is naive.
It is prone to pitfalls arising from the computer's reliance on 
finite-precision floating point arithmetic.

For example, we have seen above that we must handle as a separate case a
system that is singular.
But systems that are nearly singular also require care.
Consider this one.
\begin{equation*}
   \begin{linsys}{2}
                   x &+ &2y &= &3\hfill\hbox{}  \\
     1.000\,000\,01x &+ &2y &= &3.000\,000\,01
   \end{linsys}
\end{equation*}
We can easily spot the solution $x=1$, $y=1$.
But a computer has more trouble.
If it represents real numbers to eight significant places, called 
\definend{single precision}\index{single precision}, 
then it will represent the second
equation internally as $1.000\,000\,0x+2y=3.000\,000\,0$, losing the
digits in the ninth place.
Instead of reporting the correct solution, this computer will report something
that is not even close\Dash this computer thinks that the system is singular
because the two equations are represented internally as equal.

For some intuition about how the computer could come up with 
something that far
off, we graph the system.
\begin{center}
  \includegraphics{ch1.33}
  %\includegraphics{ppivot1.eps}
\end{center}
At the scale that we have drawn this graph we cannot tell the two lines apart.
This system is nearly singular in the sense that
the two lines are nearly the same line.
Near-singularity gives this system the property that a small change in the
system can cause a large change in its solution; for instance, changing the 
$3.000\,000\,01$ to $3.000\,000\,03$ changes the intersection point
from $(1,1)$ to $(3,0)$.
This system changes radically depending on a ninth digit, which explains why
an eight-place computer has trouble.
A problem that is very sensitive to inaccuracy or uncertainties in
the input values is \definend{ill-conditioned}.\index{ill-conditioned}

The above example gives one way in which a system can be
difficult to solve on a computer and
it has the advantage that the picture of nearly-equal lines gives a memorable 
insight into one way that numerical difficulties can arise.
Unfortunately this insight isn't useful when we wish
to solve some large system.
We cannot, typically, hope to understand the geometry of an arbitrary large
system.
In addition, there are ways that a computer's results may be
unreliable other than that the angle between some
of the linear surfaces is small.

For an example, consider the system below, from \cite{Hamming}.
\begin{equation*}
  \begin{linsys}{2}
     0.001x  &+  &y  &=  &1  \\
          x  &-  &y  &=  &0
  \end{linsys}
\tag*{($*$)}\end{equation*}
The second equation
gives $x=y$, so $x=y=1/1.001$ and 
thus both variables have values that are just less than $1$.
A computer using two digits represents the system internally in this way
(we will do this example in two-digit floating point 
arithmetic, but inventing a similar one with eight digits is easy).
\begin{equation*}
  \begin{linsys}{2}
    (1.0\times 10^{-2})x  &+  &(1.0\times 10^{0})y  &=  &1.0\times 10^{0}  \\
    (1.0\times 10^{0})x   &-  &(1.0\times 10^{0})y  &=  &0.0\times 10^{0}
  \end{linsys}
\end{equation*}
The computer's row reduction step $-1000\rho_1+\rho_2$ produces 
a second equation $-1001y=-999$, which the computer rounds to two places as 
$(-1.0\times 10^{3})y=-1.0\times 10^{3}$.
Then the computer decides from the second equation that $y=1$ 
and from the first equation that $x=0$.
This $y$ value is fairly good, but the $x$ is very 
bad.\index{accuracy!rounding error}
Thus, another cause of 
unreliable output is a mixture of floating point arithmetic
and a reliance on using leading entries that are small. 

An experienced programmer may respond by going to
\definend{double precision}\index{double precision} %
that retains sixteen significant digits.
This will indeed solve many problems.
However, double precision has twice
the memory requirements
and besides, we can obviously tweak the above systems to give the
same trouble in the seventeenth digit, so double precision
isn't a panacea.
We need is a strategy to minimize the numerical
trouble arising from solving systems on a computer
as well as some guidance as to how far we can trust the reported 
solutions. 

A basic improvement on the naive code above 
is to not simply take the entry
in the $\text{\textit{row}},\text{\textit{row}}$
position to determine the factor to use for the row combination,
but rather to look at all of the entries in the \textit{row}
column below the $\text{\textit{row}},\text{\textit{row}}$ entry
and take the one that is most likely to give reliable results
(e.g., take one that is not too small).
This is \definend{partial pivoting}.%
\index{partial pivoting}\index{pivoting!full} 

For example, to solve the troublesome system ($*$) above,
we start by looking at both equations for a best entry to use, 
and taking the $1$ in
the second equation as more likely to give good results.
Then, the combination step of $-.001\rho_2+\rho_1$ gives a first equation of 
$1.001y=1$, which the computer will represent as 
$(1.0\times 10^{0})y=1.0\times 10^{0}$, leading to the conclusion that 
$y=1$ and, after back-substitution, $x=1$, 
both of which are close to right.  
We can adapt the code from above to this purpose.
\begin{lstlisting}
for(row=1;row<=n-1;row++){
/* find the largest entry in this column (in row max) */
  max=row;
  for(row_below=row+1;row_below<=n;row_below++){
    if (abs(a[row_below,row]) > abs(a[max,row]));
      max = row_below;
  }
/* swap rows to move that best entry up */
  for(col=row;col<=n;col++){
    temp=a[row,col];
    a[row,col]=a[max,col];
    a[max,col]=temp;
  }
/* proceed as before */
  for(row_below=row+1;row_below<=n;row_below++){
    multiplier=a[row_below,row]/a[row,row];
     for(col=row;col<=n;col++){
       a[row_below,col]-=multiplier*a[row,col];
    }
  }
}
\end{lstlisting}

A full analysis of the best way to implement Gauss' method 
is outside the scope of the book (see \cite{Wilkinson65}),
but the method recommended by most experts 
first finds the best entry
among the candidates and then scales it to a number that is less
likely to give trouble.
This is 
\definend{scaled partial pivoting}.\index{scaled partial pivoting}\index{pivoting!partial!scaled}

In addition to returning a result that is likely to be reliable,
most well-done code will return a number, the 
\definend{conditioning number}%
\index{conditioning number}\index{matrix!conditioning number}
that describes the factor by which uncertainties in the input numbers
could be magnified to become inaccuracies in the results returned 
(see \cite{Rice}).

The lesson is that,
just because Gauss' method always works in theory, and just
because computer code correctly implements that method,
doesn't mean that the answer is reliable.
In practice, always use a package
where experts have worked hard to counter what can go wrong.

\begin{exercises}
  \item 
    Using two decimal places, add $253$ and $2/3$.
    \begin{answer}
      Scientific notation is convenient to express the two-place restriction.
      We have $.25\times 10^{2}+.67\times 10^{0}=.25\times 10^{2}$.
      The $2/3$ has no apparent effect.
    \end{answer}
  \item 
    This intersect-the-lines problem contrasts with the example
    discussed above.
    \begin{center}
      \begin{tabular}{@{}c@{}}
        %\includegraphics{ppivot2.eps}
        \includegraphics{ch1.34}
      \end{tabular}
      \qquad
      $\displaystyle \begin{linsys}{2}
            x &+ &2y &= &3  \\
            3x &- &2y &= &1
      \end{linsys}$
    \end{center}
    Illustrate that in this system 
    some small change in the numbers will produce only a
    small change in the solution by changing the constant in the
    bottom equation to $1.008$ and solving.
    Compare it to the solution of the unchanged system.
    \begin{answer}
      The reduction
      \begin{equation*}
        \grstep{-3\rho_1+\rho_2}
        \begin{linsys}{2}
          x  &+  &2y  &=  &3  \\
             &   &-8  &=  &-7.992
        \end{linsys}
      \end{equation*}
      gives a solution of \( (x,y)=(1.002,0.999) \).
    \end{answer}
  \item 
    Solve this system by hand (\cite{Rice}).
    \begin{equation*}
      \begin{linsys}{2}
        0.000\,3x  &+  &1.556y  &=  &1.569 \\
        0.345\,4x  &-  &2.346y  &=  &1.018
      \end{linsys}
    \end{equation*}
    \begin{exparts*}
      \partsitem Solve it accurately, by hand.
      \partsitem Solve it by
         rounding at each step to four significant digits. 
    \end{exparts*}
    \begin{answer}
      \begin{exparts}
        \partsitem The fully accurate solution is that $x=10$ and $y=0$.
        \partsitem The four-digit conclusion is quite different.
          \begin{equation*}
            \grstep{-(.3454/.0003)\rho_1+\rho_2}
            \begin{amat}{2}
              .0003  &1.556  &1.569  \\
              0      &1789   &-1805
            \end{amat}
            \Longrightarrow
            x=10460,\,y=-1.009
          \end{equation*}
      \end{exparts}
    \end{answer}
  \item 
    Rounding inside the computer often has an effect on the result.
    Assume that your machine has eight significant digits.
    \begin{exparts}
      \partsitem Show that the machine will compute 
         $(2/3)+((2/3)-(1/3))$ as unequal to $((2/3)+(2/3))-(1/3)$.
         Thus, computer arithmetic is not associative.
      \partsitem Compare the computer's version of $(1/3)x+y=0$
        and $(2/3)x+2y=0$.
        Is twice the first equation the same as the second?
    \end{exparts}
    \begin{answer}
      \begin{exparts}
        \partsitem For the first one, first, $(2/3)-(1/3)$ is 
          $.666\,666\,67-.333\,333\,33=.333\,333\,34$
          and so 
          $(2/3)+((2/3)-(1/3))=.666\,666\,67+.333\,333\,34=1.000\,000\,0$.
          For the other one, first 
          $((2/3)+(2/3))=.666\,666\,67+.666\,666\,67=1.333\,333\,3$
          and so 
          $((2/3)+(2/3))-(1/3)=1.333\,333\,3-.333\,333\,33=.999\,999\,97$.
        \partsitem The first equation is 
          $.333\,333\,33\cdot x+1.000\,000\,0\cdot y=0$
          while the second is 
          $.666\,666\,67\cdot x+2.000\,000\,0\cdot y=0$.
      \end{exparts}
    \end{answer}
  \item 
    Ill-conditioning is not only dependent on the matrix of
    coefficients.
    This example \cite{Hamming} shows that it can arise from an
    interaction between the left and right sides of the system.
    Let $\varepsilon$ be a small real.
    \begin{equation*}
      \begin{linsys}{3}
        3x  &+  &2y           &+  &z            &=  &6\hfill   \\
        2x  &+  &2\varepsilon y  &+  &2\varepsilon z  
                                          &=  &2+4\varepsilon\hfill \\
         x  &+  &2\varepsilon y  &-  &\varepsilon z   
                                          &=  &1+\varepsilon\hfill
      \end{linsys}
    \end{equation*}
    \begin{exparts}
      \partsitem Solve the system by hand.
        Notice that the $\varepsilon$'s divide out only because there is
        an exact cancellation of the integer parts on the right side
        as well as on the left. 
      \partsitem Solve the system by hand, rounding to two decimal
        places, and with $\varepsilon=0.001$.
%      \partsitem If you have access to a computer package for solving
%        linear systems, see if you can find an $\varepsilon$ small enough
%        to get your package to give incorrect results.
    \end{exparts}
    \begin{answer}
      \begin{exparts}
        \partsitem This calculation
          \begin{eqnarray*}
            &\grstep[-(1/3)\rho_1+\rho_3]{-(2/3)\rho_1+\rho_2}\;
            &\begin{amat}{3}
              3  &2                   &1                   &6               \\
              0  &-(4/3)+2\varepsilon &-(2/3)+2\varepsilon &-2+4\varepsilon \\
              0  &-(2/3)+2\varepsilon &-(1/3)-\varepsilon  &-1+\varepsilon 
            \end{amat}                                                    \\
            &\grstep{-(1/2)\rho_2+\rho_3}\;
            &\begin{amat}{3}
              3  &2                   &1                   &6               \\
              0  &-(4/3)+2\varepsilon &-(2/3)+2\varepsilon &-2+4\varepsilon \\
              0  &\varepsilon         &-2\varepsilon       &-\varepsilon 
            \end{amat}
          \end{eqnarray*}
          gives a third equation of $y-2z=-1$.
          Substituting into the second equation gives 
          $((-10/3)+6\varepsilon)\cdot z=(-10/3)+6\varepsilon$ 
          so $z=1$ and thus $y=1$.
          With those, the first equation says that $x=1$. 
        \partsitem The solution with two digits kept 
          \begin{multline*}
            \begin{amat}{3}
              .30\times 10^{1}  &.20\times 10^{1}  &.10\times 10^{1} 
                 &.60\times 10^{1}        \\
              .10\times 10^{1}  &.20\times 10^{-3} &.20\times 10^{-3} 
                 &.20\times 10^{1}        \\
              .30\times 10^{1}  &.20\times 10^{-3}  &-.10\times 10^{-3} 
                 &.10\times 10^{1}        
            \end{amat}                                           \\
            \begin{aligned}
            &\grstep[-(1/3)\rho_1+\rho_3]{-(2/3)\rho_1+\rho_2}\;
            \begin{amat}{3}
              .30\times 10^{1}  &.20\times 10^{1}  &.10\times 10^{1} 
                 &.60\times 10^{1}        \\
              0                 &-.13\times 10^{1} &-.67\times 10^{0} 
                 &-.20\times 10^{1}        \\
              0                 &-.67\times 10^{0}  &-.33\times 10^{0} 
                 &-.10\times 10^{1}        
            \end{amat}                                          \\
            &\grstep{-(.67/1.3)\rho_2+\rho_3}\;
            \begin{amat}{3}
              .30\times 10^{1}  &.20\times 10^{1}  &.10\times 10^{1} 
                 &.60\times 10^{1}        \\
              0                 &-.13\times 10^{1} &-.67\times 10^{0} 
                 &-.20\times 10^{1}        \\
              0                 &0                  &.15\times 10^{-2} 
                 &.31\times 10^{-2}        
            \end{amat}
            \end{aligned}
          \end{multline*}
          comes out to be $z=2.1$, $y=2.6$, and $x=-.43$.
      \end{exparts}
    \end{answer}
\end{exercises}
\index{accuracy! of Gauss' method|)}
\index{Gauss' method! accuracy|)}

\endinput




