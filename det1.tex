% Chapter 4, Section 1 _Linear Algebra_ Jim Hefferon
%  http://joshua.smcvt.edu/linalg.html
%  2001-Jun-12
\chapter{Determinants} 
In the first chapter  
we highlighted the special case of linear systems
with the same number of equations as unknowns, 
those of the form \( T\vec{x}=\vec{b} \) where $T$ is a square matrix.
We noted a distinction between two classes of $T$'s.
If $T$ is associated with a unique solution
for any vector $\vec{b}$ of constants, 
such as for the homogeneous system $T\vec{x}=\zero$, then
$T$ is associated with a unique solution for every vector $\vec{b}$.
We call such a matrix of coefficients nonsingular.
The other kind of $T$, where every linear system for which it is the 
matrix of coefficients has either no solution or infinitely many solutions,
is singular.  

In our work since then the value of this distinction has been a theme.
For instance, we now know that an 
\( \nbyn{n} \) matrix \( T \) is nonsingular if and only if
each of these holds:
%<*EquivalentOfNonsingular>
\begin{itemize}
   \item any system \( T\vec{x}=\vec{b} \) has a solution 
          and that solution is unique;
   \item Gauss-Jordan reduction of $T$ yields an identity matrix;
   \item the rows of $T$ form a linearly independent set;
   \item the columns of \( T \) form a linearly independent set, 
         and a basis for \( \Re^n \);
   \item any map that \( T \) represents is an isomorphism;
   \item an inverse matrix \( T^{-1} \) exists.
\end{itemize}
%</EquivalentOfNonsingular>
So when we look at a particular square matrix, one of the first things
that we ask is whether it is  
nonsingular.

Naturally there is a formula that determines whether $T$ is nonsingular.
This chapter develops that formula. 
More precisely, we will develop infinitely many formulas, 
one for $\nbyn{1}$~matrices, one for $\nbyn{2}$~matrices, etc.
These formulas are related,  that is, 
we will develop a family of
formulas, a scheme that describes the formula for each size.

Since we will restrict the discussion to square matrices, in this chapter
we will often simply say `matrix' in place of `square matrix'.





\section{Def{}inition}
%<*DeterminantIntro>
Determining nonsingularity
is trivial for \( \nbyn{1} \) matrices.
\begin{equation*}
  \begin{mat}
       a
  \end{mat} 
  \quad\text{is nonsingular iff}\quad
  a \neq 0
\end{equation*} 
Corollary~Three.IV.\ref{cor:TwoByTwoInv} gives the formula for 
the inverse of a $\nbyn{2}$ matrix. 
\begin{equation*}
     \begin{mat}
           a  &b  \\
           c  &d
      \end{mat}  
   \quad\text{is nonsingular iff}\quad
   ad-bc \neq 0
\end{equation*}
We can produce the $\nbyn{3}$ formula as we did the prior one,  
although the computation is intricate
% \typeout{DET1 ThreeByThreeDetForm cleveref expmt!}
% (see \cref{exer:ThreeByThreeDetForm}).
(see \nearbyexercise{exer:ThreeByThreeDetForm}).
\begin{equation*}
     \begin{mat}
           a  &b  &c  \\
           d  &e  &f  \\
           g  &h  &i
     \end{mat}
  \quad\text{is nonsingular iff}\quad
    aei+bfg+cdh-hfa-idb-gec \neq 0
\end{equation*}
With these cases in mind, we posit a family of 
formulas: $a$, $ad-bc$, etc. 
For each $n$ the formula gives rise to a 
\definend{determinant}\index{determinant}\index{matrix!determinant} 
function
$\map{\det_{\nbyn{n}}}{\matspace_{\nbyn{n}}}{\Re}$ 
such that an $\nbyn{n}$ matrix $T$ is nonsingular if and
only if $\det_{\nbyn{n}}(T)\neq 0$.
%</DeterminantIntro>
(We usually omit the subscript \( \nbyn{n} \) because 
the size of \( T \) tells us which determinant function we mean.)






\subsectionoptional{Exploration}
\textit{This is an optional
motivation of the general definition, suggesting how a person might
develop that formula.
The definition is in the next 
subsection.}

Above, in each case the matrix is nonsingular if and only 
if some formula is nonzero.
But the three cases don't 
show an obvious pattern for the formula. 
We may spot that the \(\nbyn{1}\) term
\( a \) has one letter, that the \(\nbyn{2}\) terms
\(ad\) and \(bc\) have two letters, and that the \(\nbyn{3}\)
terms \(aei\), etc., have three letters.
We may also spot that in those terms
there is a letter from each row and column of the matrix, e.g., 
in the \(cdh\) term one letter 
comes from each row and from each column.
\begin{equation*}
   \begin{mat}
          &    &c \\
      d           \\
          &h
   \end{mat} 
\end{equation*}
But these observations are perhaps more puzzling than
enlightening.
For instance, we might wonder why 
we add some of the terms while we subtract others.

A good problem solving strategy is to
see what properties a solution must have and
then search for something with those properties.
So we shall start by asking what properties we require of the formulas.

At this point, our
main way to decide whether a matrix is singular or nonsingular
is to do Gaussian 
reduction and then check whether 
the diagonal of resulting echelon form matrix has any zeroes
(that is, to check whether the product down the diagonal is~zero).
So we could guess that whatever formula we find, the proof that 
it is right may involve applying Gauss' method to the matrix
to show that in the end the product down the diagonal is zero if and only if
our formula gives zero. 

This suggests a plan:~we will look for a family of determinant
formulas that are 
unaffected by row operations and such that the determinant of an
echelon form matrix is the product of its diagonal entries.
% Under this plan, a proof that the functions determine singularity would go, 
% ``Where $T\rightarrow\cdots\rightarrow\hat{T}$ is the Gaussian
% reduction, the determinant of $T$ equals the
% determinant of $\hat{T}$ (because the determinant is unchanged by row
% operations), which is the product down the diagonal, which is
% zero if and only if the matrix is singular''.
In the rest of this subsection we will test this plan against the 
$\nbyn{2}$ and $\nbyn{3}$ formulas.
In the end we will have to modify the ``unaffected by row operations'' 
part, but not by much.

The first step in testing this plan is to see whether   
the $\nbyn{2}$ and $\nbyn{3}$ formulas are unaffected by the 
row operation of combining:~if 
\begin{equation*}
   T \grstep{k\rho_i+\rho_j} \hat{T}
\end{equation*}
then is \( \det(\hat{T})=\det(T) \)? 
This check of the $\nbyn{2}$ determinant after the $k\rho_1+\rho_2$ operation
\begin{equation*}
   \det(
       \begin{mat}
         a     &b       \\
         ka+c  &kb+d    \\
      \end{mat}
   )
   = a(kb+d)-(ka+c)b = ad-bc
\end{equation*}
shows that it is indeed unchanged, and
the other $\nbyn{2}$ combination $k\rho_2+\rho_1$ gives the same result.
The $\nbyn{3}$ combination $k\rho_3+\rho_2$ leaves the determinant unchanged
  \begin{align*}
    \det(
      \begin{mat}
         a    &b    &c    \\
         kg+d &kh+e &ki+f \\
         g    &h    &i
      \end{mat}
    )
    &=\begin{array}[t]{@{}l@{}}
         a(kh+e)i+b(ki+f)g+c(kg+d)h \\
         \ \hbox{}-h(ki+f)a-i(kg+d)b-g(kh+e)c  
    \end{array}                                 \\
     &=aei + bfg + cdh - hfa - idb - gec
  \end{align*}
as do the other $\nbyn{3}$ row combination operations.

So there seems to be promise in the plan.
Of course, perhaps if we had worked out 
the $\nbyn{4}$ determinant formula and tested it then we might have found
that it is affected by row combinations.
This is an exploration and we do not yet have all the facts.
Nonetheless, so far, so good.

Next is to compare \( \det(\hat{T}) \) with 
\( \det(T) \) for row swaps.
% \begin{equation*}
%    T \grstep{ {\rho}_i \leftrightarrow {\rho}_j } \hat{T}
% \end{equation*}
We now hit a snag:
the \(\nbyn{2}\) row swap $\rho_1\leftrightarrow\rho_2$
does not yield \( ad-bc \).
  \begin{equation*}
     \det(
       \begin{mat}
         c  &d \\
         a  &b
       \end{mat}
     )
     = cb - ad
  \end{equation*}
And this $\rho_1\leftrightarrow\rho_3$ swap inside of a \(\nbyn{3}\) matrix
\begin{equation*}
   \det(
     \begin{mat}
        g  &h  &i \\
        d  &e  &f \\
        a  &b  &c
     \end{mat}
   )
   = gec + hfa + idb - bfg - cdh - aei
\end{equation*}
also does not give the same determinant as before the swap; again 
there is a sign change.
Trying a different \(\nbyn{3}\) swap $\rho_1\leftrightarrow\rho_2$ 
\begin{equation*}
   \det(
     \begin{mat}
        d  &e  &f \\
        a  &b  &c \\
        g  &h  &i
     \end{mat}
   )
   = dbi + ecg + fah - hcd - iae - gbf
\end{equation*}
also gives a change of sign.

So row swaps seem 
to change the sign of a determinant formula.
This does not wreck our plan entirely.
We intend to decide nonsingularity by considering
only whether the formula gives zero, not by considering its sign.
Therefore, instead of expecting determinant formulas to be
entirely unaffected by row operations, we modify our plan to 
have them to change sign on a swap.

To finish
we compare \( \det(\hat{T}) \) to \( \det(T) \) for the operation
% \begin{equation*}
%    T \grstep{ k{\rho}_i } \hat{T}
% \end{equation*}
of multiplying a row by a scalar $k\neq 0$. 
This
\begin{equation*}
  \det(
    \begin{mat}
      a   &b   \\
      kc  &kd
    \end{mat}
  )
  = a(kd) - (kc)b
  =k\cdot (ad-bc)
\end{equation*}
ends with the determinant multiplied by~$k$,
and the other $\nbyn{2}$ case has the same result.
This \(\nbyn{3}\) case ends the same way
\begin{align*}
  \det(
  \begin{mat}
    a    &b    &c   \\
    d    &e    &f   \\
    kg   &kh   &ki
  \end{mat}
  )
   &= \begin{array}[t]{@{}l@{}}
         ae(ki) + bf(kg) + cd(kh)                \\
         \>- (kh)fa - (ki)db - (kg)ec  
      \end{array}                                      \\
   &= k\cdot(aei + bfg + cdh - hfa - idb - gec)
\end{align*}
and the other two are similar.
These make us suspect that multiplying a row by~$k$
multiplies the determinant by~$k$.
As before, this modifies our plan but does not wreck it.
We are asking only that the
zeroness of the determinant formula be unchanged and we are not focusing on the
its sign or magnitude.

So, our modified plan is to look for determinants that remain unchanged
under the operation of row combination, that change sign on
a row swap, and that rescale on the rescaling of a row.
In the next two subsections we will find that for each~$n$
there is such a function, and is unique.

For the next subsection, note that scalars factor out
of a row without affecting other rows:
here
\begin{equation*}
    \det(
      \begin{mat}[r]
        3  &3  &9  \\
        2  &1  &1  \\
        5  &10 &-5
     \end{mat}
    )
    =3 \cdot \det(
               \begin{mat}[r]
                 1  &1  &3  \\
                 2  &1  &1  \\
                 5  &10 &-5
               \end{mat}
             )                                  
\end{equation*}
the $3$ comes only out of the top row only, leaving the other rows unchanged.
So in the definition of determinant we will
write it as a function of the rows
\( \det (\vec{\rho}_1,\vec{\rho}_2,\dots\vec{\rho}_n) \), not as 
\( \det(T) \) or as a function of the entries
\( \det(t_{1,1},\dots,t_{n,n}) \).

\begin{exercises}
  \recommended \item 
     Evaluate the determinant of each.
     \begin{exparts*}
       \partsitem \(
            \begin{mat}[r]
                3    &1   \\
               -1    &1
            \end{mat}    \)
       \partsitem \(
            \begin{mat}[r]
                2    &0   &1  \\
                3    &1   &1 \\
               -1    &0   &1
            \end{mat}    \)
       \partsitem \(
            \begin{mat}[r]
                4    &0   &1  \\
                0    &0   &1 \\
                1    &3   &-1
            \end{mat}    \)
     \end{exparts*}
     \begin{answer}
       \begin{exparts*}
         \partsitem \( 4 \)
         \partsitem \( 3 \)
         \partsitem \( -12 \)
       \end{exparts*}  
     \end{answer}
  \item 
     Evaluate the determinant of each.
     \begin{exparts*}
        \partsitem \( \begin{mat}[r]
                    2  &0  \\
                   -1  &3
                 \end{mat} \)
        \partsitem \( \begin{mat}[r]
                    2  &1  &1  \\
                    0  &5  &-2 \\
                    1  &-3 &4
                 \end{mat} \)
        \partsitem \( \begin{mat}[r]
                    2  &3  &4  \\
                    5  &6  &7  \\
                    8  &9  &1
                 \end{mat} \)
     \end{exparts*}
    \begin{answer}
      \begin{exparts*}
        \partsitem \( 6 \)
        \partsitem \( 21 \)
        \partsitem \( 27 \)
      \end{exparts*}  
    \end{answer}
  \recommended \item  
    Verify that the determinant of an upper-triangular
    $\nbyn{3}$ matrix is the product down the diagonal. 
    \begin{equation*}
       \det(
       \begin{mat}
           a    &b   &c    \\
           0    &e   &f    \\
           0    &0   &i
       \end{mat}
       )
       =aei
    \end{equation*}
    Do lower-triangular matrices work the same way?
    \begin{answer}
      For the first, apply the formula in this section, note that any
      term with a \( d \), \( g \), or \( h \) is zero, and simplify.
      Lower-triangular matrices work the same way.  
    \end{answer}
  \recommended \item 
     Use the determinant to decide if each is singular or
     nonsingular.
     \begin{exparts*}
       \partsitem \(
            \begin{mat}[r]
                2    &1   \\
                3    &1
            \end{mat}    \)
       \partsitem \(
            \begin{mat}[r]
                0    &1   \\
                1    &-1
            \end{mat}    \)
       \partsitem \(
            \begin{mat}[r]
                4    &2   \\
                2    &1
            \end{mat}    \)
     \end{exparts*}
     \begin{answer}
       \begin{exparts}
         \partsitem Nonsingular, the determinant is \( -1 \).
         \partsitem Nonsingular, the determinant is \( -1 \).
         \partsitem Singular, the determinant is \( 0 \).
       \end{exparts}  
     \end{answer}
  \item 
     Singular or nonsingular?
     Use the determinant to decide.
     \begin{exparts*}
       \partsitem \(
            \begin{mat}[r]
                2    &1   &1  \\
                3    &2   &2 \\
                0    &1   &4
            \end{mat}    \)
       \partsitem \(
            \begin{mat}[r]
                1    &0   &1  \\
                2    &1   &1 \\
                4    &1   &3
            \end{mat}    \)
       \partsitem \(
            \begin{mat}[r]
                2    &1   &0  \\
                3    &-2  &0 \\
                1    &0   &0
            \end{mat}    \)
     \end{exparts*}
     \begin{answer}
      \begin{exparts}
         \partsitem Nonsingular, the determinant is \( 3 \).
         \partsitem Singular, the determinant is \( 0 \).
         \partsitem Singular, the determinant is \( 0 \).
       \end{exparts}  
     \end{answer}
  \recommended \item
    Each pair of matrices differ by one row operation.
    Use this operation to
    compare \( \det(A) \) with \( \det(B) \).
     \begin{exparts}
        \partsitem \( A=\begin{mat}[r]
                      1  &2  \\
                      2  &3
                   \end{mat} \)
               \(  B=\begin{mat}[r]
                      1  &2  \\
                      0  &-1
                   \end{mat}  \)
        \partsitem \( A=\begin{mat}[r]
                      3  &1  &0  \\
                      0  &0  &1  \\
                      0  &1  &2
                   \end{mat} \)
              \(   B=\begin{mat}[r]
                      3  &1  &0  \\
                      0  &1  &2  \\
                      0  &0  &1
                   \end{mat}  \)
        \partsitem \( A=\begin{mat}[r]
                      1  &-1 &3  \\
                      2  &2  &-6 \\
                      1  &0  &4
                   \end{mat} \)
              \(   B=\begin{mat}[r]
                      1  &-1 &3  \\
                      1  &1  &-3 \\
                      1  &0  &4
                   \end{mat}  \)
     \end{exparts}
     \begin{answer}
       \begin{exparts}
         \partsitem \( \det(B)=\det(A) \) via \( -2\rho_1+\rho_2 \)
         \partsitem \( \det(B)=-\det(A) \) via 
             \( \rho_2\leftrightarrow\rho_3 \)
         \partsitem \( \det(B)=(1/2)\cdot \det(A) \) via \( (1/2)\rho_2 \)
       \end{exparts}   
     \end{answer}
  \item 
     Show this.
      \begin{equation*}
         \det(
         \begin{mat}
             1    &1   &1    \\
             a    &b   &c    \\
             a^2  &b^2 &c^2
         \end{mat}
         )
         =(b-a)(c-a)(c-b)
      \end{equation*}
     \begin{answer}
       Using the formula for the determinant of a $\nbyn{3}$ matrix
       we expand the left side
       \begin{equation*}
         1\cdot b\cdot c^2+1\cdot c\cdot a^2+1\cdot a\cdot b^2
          -b^2\cdot c\cdot 1 -c^2\cdot a\cdot 1-a^2\cdot b\cdot 1
       \end{equation*}
       and by distributing we expand the right side.
       \begin{equation*}
         (bc-ba-ac+a^2)\cdot(c-b)
         =c^2b-b^2c-bac+b^2a-ac^2+acb+a^2c-a^2b
       \end{equation*}
       Now we can just check that the two are equal.
       (\textit{Remark}.
       This is the \( \nbyn{3} \) case of
       \definend{Vandermonde's determinant}\index{Vandermonde!determinant}%
       \index{determinant!Vandermonde} which arises in applications).
     \end{answer}
   \recommended \item 
      Which real numbers \( x \) make this matrix singular?
      \begin{equation*}
         \begin{mat}
            12-x  &4  \\
            8    &8-x
         \end{mat}
      \end{equation*}
      \begin{answer}
         This equation
         \begin{equation*}
           0=
           \det(
             \begin{mat}
                12-x  &4  \\
                8    &8-x
             \end{mat}
           )
           =64-20x+x^2
           =(x-16)(x-4) 
       \end{equation*}
       has roots \( x=16 \) and \( x=4 \).  
     \end{answer}
  \item \label{exer:ThreeByThreeDetForm} 
    Do the Gaussian reduction to check
    the formula for $\nbyn{3}$ matrices stated in the preamble to
    this section.
    \begin{center}
      \( \begin{mat}
               a  &b  &c  \\
               d  &e  &f  \\
               g  &h  &i
         \end{mat} \)
      is nonsingular iff
      \( aei+bfg+cdh-hfa-idb-gec \neq 0 \)
    \end{center}
    \begin{answer}
      We first reduce the matrix to echelon form.
      To begin, assume that \( a\neq 0 \) and that \( ae-bd\neq 0 \).
      \begin{eqnarray*}
        \grstep{(1/a)\rho_1}\;
        \begin{mat}
           1   &b/a   &c/a   \\
           d   &e     &f     \\
           g   &h     &i
         \end{mat}                                           
        &\grstep[-g\rho_1+\rho_3]{-d\rho_1+\rho_2}
        &\begin{mat}
           1   &b/a           &c/a           \\
           0   &(ae-bd)/a     &(af-cd)/a     \\
           0   &(ah-bg)/a     &(ai-cg)/a
         \end{mat}                                            \\
        &\grstep{(a/(ae-bd))\rho_2}
        &\begin{mat}
           1   &b/a           &c/a             \\
           0   &1             &(af-cd)/(ae-bd) \\
           0   &(ah-bg)/a     &(ai-cg)/a
         \end{mat}
      \end{eqnarray*}
      This step finishes the calculation.
      \begin{equation*}
        \grstep{((ah-bg)/a)\rho_2+\rho_3}
        \begin{mat}
           1   &b/a    &c/a             \\
           0   &1      &(af-cd)/(ae-bd)      \\
           0   &0      &(aei+bgf+cdh-hfa-idb-gec)/(ae-bd)
         \end{mat}
      \end{equation*}
      Now assuming that $a\neq 0$ and \( ae-bd\neq 0 \), 
      the original matrix is nonsingular
      if and only if the \( 3,3 \) entry above is nonzero.
      That is, under the assumptions, the original matrix is
      nonsingular if and only if $aei+bgf+cdh-hfa-idb-gec\neq 0$,
      as required.

      We finish by running down what happens if the assumptions that were
      taken for convenience in the prior paragraph do not hold.
      First, if \( a\neq 0 \) but \( ae-bd=0 \) then we can swap
      \begin{equation*}
        \begin{mat}
           1   &b/a           &c/a           \\
           0   &0             &(af-cd)/a     \\
           0   &(ah-bg)/a     &(ai-cg)/a
         \end{mat}                                  
        \grstep{\rho_2\leftrightarrow\rho_3}
        \begin{mat}
           1   &b/a           &c/a           \\
           0   &(ah-bg)/a     &(ai-cg)/a     \\
           0   &0             &(af-cd)/a
         \end{mat}
      \end{equation*}
      and conclude that the matrix is nonsingular if and only if either
      \( ah-bg=0 \) or \( af-cd=0 \).
      The condition `\( ah-bg=0 \) or \( af-cd=0 \)' is equivalent to
      the condition `\( (ah-bg)(af-cd)=0 \)'.
      Multiplying out and using the case assumption that $ae-bd=0$
      to substitute $ae$ for $bd$ gives this.
      \begin{equation*}
         0=ahaf-ahcd-bgaf+bgcd
          =ahaf-ahcd-bgaf+aegc
          =a(haf-hcd-bgf+egc)
      \end{equation*}
      Since \( a\neq 0 \), we have that the matrix
      is nonsingular if and only if \( haf-hcd-bgf+egc=0 \).
      Therefore, in this \( a\neq 0 \) and \( ae-bd=0 \) case, 
      the matrix is nonsingular when
      \( haf-hcd-bgf+egc-i(ae-bd)=0 \).

      The remaining cases are routine.
      Do the \( a=0 \) but \( d\neq 0 \) case and the \( a=0 \) and \( d=0 \)
      but \( g\neq 0 \) case by first swapping rows and then going on as
      above.
      The \( a=0 \), \( d=0 \), and \( g=0 \) case is easy\Dash that matrix is
      singular since the columns form a linearly dependent set, and the
      determinant comes out to be zero.  
    \end{answer}
  \item 
     Show that the equation of a line in \( \Re^2 \) thru \( (x_1,y_1) \)
     and \( (x_2,y_2) \) is given by this determinant.
     \begin{equation*}
        \det(
        \begin{mat}
           x   &y   &1  \\
           x_1 &y_1 &1  \\
           x_2 &y_2 &1
        \end{mat})=0 \qquad x_1\neq x_2
     \end{equation*}
     \begin{answer}
       Figuring the determinant and doing some algebra gives this.
       \begin{align*}
          0
          &=y_1x+x_2y+x_1y_2-y_2x-x_1y-x_2y_1     \\
          (x_2-x_1)\cdot y
          &=(y_2-y_1)\cdot x+x_2y_1-x_1y_2              \\
          y
          &=\frac{y_2-y_1}{x_2-x_1}\cdot x+\frac{x_2y_1-x_1y_2}{x_2-x_1}
       \end{align*}
       Note that this is the equation of a line (in particular,
       in contains the familiar expression for the slope), 
       and note that \( (x_1,y_1) \)  and \( (x_2,y_2) \) satisfy it. 
     \end{answer}
  \recommended \item
    Many people know this mnemonic for the determinant of a \( \nbyn{3} \)
    matrix: first repeat the first two columns and then sum the products on the
    forward diagonals and subtract the products on the backward diagonals.
    That is, first write
    \begin{equation*}
        \begin{pmat}{ccc|cc}
          h_{1,1} &h_{1,2} &h_{1,3} &h_{1,1} &h_{1,2} \\
          h_{2,1} &h_{2,2} &h_{2,3} &h_{2,1} &h_{2,2} \\
          h_{3,1} &h_{3,2} &h_{3,3} &h_{3,1} &h_{3,2}
        \end{pmat}
     \end{equation*}
     and then calculate this.
     \begin{equation*}
      \begin{array}{l}
      h_{1,1}h_{2,2}h_{3,3}+h_{1,2}h_{2,3}h_{3,1}+h_{1,3}h_{2,1}h_{3,2} \\
      \>-h_{3,1}h_{2,2}h_{1,3}-h_{3,2}h_{2,3}h_{1,1}
        -h_{3,3}h_{2,1}h_{1,2}
      \end{array}
    \end{equation*}
    \begin{exparts}
      \partsitem Check that this agrees with the formula given in the 
        preamble to this section.
      \partsitem Does it extend to other-sized determinants?
    \end{exparts}
    \begin{answer}
      \begin{exparts}
        \partsitem The comparison with the formula given in the preamble to
          this section is easy.
        \partsitem While it holds for \( \nbyn{2} \) matrices
          \begin{align*}
            \begin{pmat}{cc|c}
              h_{1,1} &h_{1,2} &h_{1,1} \\
              h_{2,1} &h_{2,2} &h_{2,1}
            \end{pmat}
            &=\begin{array}[t]{@{}l@{}}
              h_{1,1}h_{2,2}+h_{1,2}h_{2,1}  \\
              \>-h_{2,1}h_{1,2}-h_{2,2}h_{1,1}
            \end{array}                                     \\
            &=h_{1,1}h_{2,2}-h_{1,2}h_{2,1}
          \end{align*}
          it does not hold for \( \nbyn{4} \) matrices.
          An example is that this matrix is
          singular because the second and third rows are equal 
          \begin{equation*}
            \begin{mat}[r]  
              1  &0  &0  &1   \\
              0  &1  &1  &0   \\
              0  &1  &1  &0   \\
             -1  &0  &0  &1  
            \end{mat}
          \end{equation*}
          but following the scheme of the mnemonic does not give zero.
          \begin{equation*}
            \begin{pmat}{rrrr|rrr}
                       1  &0  &0  &1  &1  &0  &0  \\
                       0  &1  &1  &0  &0  &1  &1  \\
                       0  &1  &1  &0  &0  &1  &1  \\
                      -1  &0  &0  &1  &-1 &0  &0
            \end{pmat}
            =\begin{array}[t]{@{}l@{}}
               1+0+0+0 \\
               \>-(-1)-0-0-0
              \end{array}
           \end{equation*}
      \end{exparts}  
     \end{answer}
  \item 
    The 
    \definend{cross product}\index{cross product}\index{vector!cross product} 
    of the vectors
    \begin{equation*}
      \vec{x}=\colvec{x_1 \\ x_2 \\ x_3}
      \qquad
      \vec{y}=\colvec{y_1 \\ y_2 \\ y_3}
    \end{equation*}
    is the vector computed as this determinant.
    \begin{equation*}
      \vec{x}\times\vec{y}=
      \det(\begin{mat}
        \vec{e}_1  &\vec{e}_2  &\vec{e}_3  \\
        x_1        &x_2        &x_3        \\
        y_1        &y_2        &y_3
      \end{mat})
    \end{equation*}
    Note that the first row's entries are vectors, the vectors from the
    standard basis for $\Re^3$.
    Show that the cross product of two vectors is perpendicular to each vector.
    \begin{answer}
      The determinant is 
      $
        (x_2y_3-x_3y_2)\vec{e}_1
          +(x_3y_1-x_1y_3)\vec{e}_2
          +(x_1y_2-x_2y_1)\vec{e}_3
      $.
      To check perpendicularity, we check that the dot product
      with the first vector is zero
      \begin{equation*}
        \colvec{x_1 \\ x_2 \\ x_3}
        \dotprod
        \colvec{x_2y_3-x_3y_2 \\ x_3y_1-x_1y_3 \\ x_1y_2-x_2y_1}
        =x_1x_2y_3-x_1x_3y_2+x_2x_3y_1-x_1x_2y_3+x_1x_3y_2-x_2x_3y_1=0
      \end{equation*}
      and the dot product with the second vector is also zero.
      \begin{equation*}
        \colvec{y_1 \\ y_2 \\ y_3}
        \dotprod
        \colvec{x_2y_3-x_3y_2 \\ x_3y_1-x_1y_3 \\ x_1y_2-x_2y_1}
        =x_2y_1y_3-x_3y_1y_2+x_3y_1y_2-x_1y_2y_3+x_1y_2y_3-x_2y_1y_3=0
      \end{equation*}  
    \end{answer}
  \item 
    Prove that each statement holds for $\nbyn{2}$ matrices.  
    \begin{exparts}
      \partsitem The determinant of a
        product is the product of the determinants
        $\det(ST)=\det(S)\cdot\det(T)$.
      \partsitem If \( T \) is invertible then
        the determinant of the inverse is the inverse of the determinant
        \( \det(T^{-1})=(\,\det(T)\,)^{-1} \).
    \end{exparts}
    Matrices $T$ and $T^\prime$ are 
    \definend{similar}\index{similar} if there is a 
    nonsingular matrix $P$ such that $T^\prime=PTP^{-1}$.
    (This definition is in Chapter Five.)
    Show that similar \( \nbyn{2} \) matrices have the same
    determinant.
    \begin{answer}
       \begin{exparts}
        \partsitem Plug and chug:
          the determinant of the product is this
          \begin{align*}
             \det(\begin{mat}
                       a  &b  \\
                       c  &d
                    \end{mat}
                    \begin{mat}
                       w  &x  \\
                       y  &z
                    \end{mat}  )
             &=
             \det(\begin{mat}
                 aw+by  &ax+bz  \\
                 cw+dy  &cx+dz
              \end{mat} )                 \\
             &=
             \begin{array}[t]{@{}l@{}} 
                 acwx+adwz+bcxy+bdyz  \\
                 \> -acwx-bcwz-adxy-bdyz
              \end{array}
          \end{align*}
          while the product of the determinants is this.
          \begin{equation*}
             \det(\begin{mat}
                a  &b  \\
                c  &d
             \end{mat})
             \cdot\det(\begin{mat}
                w  &x  \\
                y  &z
             \end{mat})
             =
             (ad-bc)\cdot (wz-xy)
          \end{equation*}
          Verification that they are equal is easy.
        \partsitem Use the prior item.
       \end{exparts}  
       \noindent That similar matrices have the same determinant 
       is immediate from the above two:
       $
          \det(PTP^{-1})=\det(P)\cdot\det(T)\cdot\det(P^{-1})
       $.
     \end{answer}
  \recommended \item
    Prove that the area of this region in the plane
    \begin{center}
      \includegraphics{ch4.30}
    \end{center}
    is equal to the value of this determinant.
    \begin{equation*}
       \det(
       \begin{mat}
          x_1  &x_2  \\
          y_1  &y_2
       \end{mat})
    \end{equation*}
    Compare with this.
    \begin{equation*}
       \det(
       \begin{mat}
          x_2  &x_1  \\
          y_2  &y_1
       \end{mat})
    \end{equation*}
    \begin{answer}
      One way is to count these areas
      \begin{center}
        \includegraphics{ch4.31}
      \end{center}
      by taking the area of the entire rectangle and subtracting the area of
      $A$ the upper-left rectangle, $B$ the upper-middle triangle,
      $D$ the upper-right triangle, $C$ the lower-left triangle, 
      $E$ the lower-middle triangle, and $F$ the lower-right rectangle       
      \( (x_1+x_2)(y_1+y_2)-x_2y_1-(1/2)x_1y_1-(1/2)x_2y_2
              -(1/2)x_2y_2-(1/2)x_1y_1-x_2y_1 \).
      Simplification gives the determinant formula.

      This determinant is the negative of the one above; the formula
      distinguishes whether the second column is counterclockwise from
      the first.  
     \end{answer}
  \item 
    Prove that for \( \nbyn{2} \) matrices, the determinant of a matrix
    equals the determinant of its transpose.
    Does that also hold for \( \nbyn{3} \) matrices?
    \begin{answer}
      The computation for \( \nbyn{2} \) matrices, using the
      formula quoted in the preamble, is easy.
      It does also hold for \( \nbyn{3} \) matrices; the 
      computation is routine.  
    \end{answer}
  \recommended \item 
    Is the determinant function linear \Dash  is
    \( \det(x\cdot T+y\cdot S)=x\cdot \det(T)+y\cdot \det(S) \)?
    \begin{answer}
      No.
      Recall that constants come out one row at a time.
      \begin{equation*}
         \det(
         \begin{mat}[r]
            2  &4  \\
            2  &6  \\
         \end{mat})
         =
         2\cdot\det(\begin{mat}[r]
            1  &2  \\
            2  &6  \\
         \end{mat})
         =
         2\cdot 2\cdot \det(\begin{mat}[r]
            1  &2  \\
            1  &3  \\
         \end{mat})
      \end{equation*}
      This contradicts linearity (here we didn't need \( S \), i.e., we can 
      take $S$ to be the matrix of zeros).  
    \end{answer}
  \item 
    Show that if \( A \) is \( \nbyn{3} \) then
    \( \det(c\cdot A)=c^3\cdot \det(A) \) for any scalar \( c \).
    \begin{answer}
       Bring out the \( c \)'s one row at a time.  
    \end{answer}
  \item 
    Which real numbers \( \theta \) make
    \begin{equation*}
       \begin{mat}
          \cos\theta  &-\sin\theta  \\
          \sin\theta  &\cos\theta
       \end{mat}
    \end{equation*}
    singular?
    Explain geometrically.
    \begin{answer}
      There are no real numbers \( \theta \) that make the matrix singular 
      because the determinant of the matrix
      \( \cos^2\theta+\sin^2\theta \) is never $0$, it equals $1$
      for all $\theta$.
      Geometrically, with respect to the standard basis,
      this matrix represents
      a rotation of the plane through an angle of \( \theta \).
      Each such map is one-to-one \Dash  for one thing, it is invertible.  
    \end{answer}
  \puzzle \item  
    \cite{Monthly55p257}
    If a third order determinant has elements
    \( 1 \), \( 2 \), \ldots, \( 9 \), what is the maximum value it may
    have?
    \begin{answer}
      \answerasgiven
      Let \( P \) be the sum of the three positive terms of the determinant
      and \( -N \) the sum of the three negative terms.
      The maximum value of \( P \) is
      \begin{equation*}
        9\cdot 8\cdot 7 +6\cdot 5\cdot 4 +3\cdot 2\cdot 1=630.
      \end{equation*}
      The minimum value of \( N \) consistent with \( P \) is
      \begin{equation*}
        9\cdot 6\cdot 1 +8\cdot 5\cdot 2 +7\cdot 4\cdot 3=218.
      \end{equation*}
      Any change in \( P \) would result in lowering that sum by more than
      \( 4 \).
      Therefore \( 412 \) the maximum value for the determinant and one form
      for the determinant is
      \begin{equation*}
         \begin{vmat}[r]
            9  &4  &2  \\
            3  &8  &6  \\
            5  &1  &7
         \end{vmat}.
      \end{equation*}  
    \end{answer}
\end{exercises}




















\subsection{Properties of Determinants}
\index{determinant|(}
We want a formula
to determine whether an $\nbyn{n}$  matrix is nonsingular.
We will not begin by stating such a formula.
Instead, we will begin by considering the function that such a formula
calculates.
We will define this function by its properties,
then prove that the function with these properties exist and is unique,
and also describe formulas that compute this function.
(Because we will eventually show that the function exists and is unique, 
from the start we will say `\( \det(T) \)' instead of
`if there is a unique determinant function then \( \det(T) \)'.)

\begin{definition}  \label{def:Det}
%<*df:Det>
A \( \nbyn{n} \) \definend{determinant\/}\index{determinant!definition}%
\index{matrix!determinant} is a function
\( \map{\det}{\matspace_{\nbyn{n}}}{\Re} \) such that
\begin{enumerate}
  \item
    $
       \det (\vec{\rho}_1,\dots,k\cdot\vec{\rho}_i 
                                    + \vec{\rho}_j,\dots,\vec{\rho}_n)
       =\det (\vec{\rho}_1,\dots,\vec{\rho}_j,\dots,\vec{\rho}_n)
    $ for \( i\ne j \)
  \item 
    $
       \det (\vec{\rho}_1,\ldots,\vec{\rho}_j,
               \dots,\vec{\rho}_i,\dots,\vec{\rho}_n)
       = -\det (\vec{\rho}_1,\dots,\vec{\rho}_i,\dots,\vec{\rho}_j,
                \dots,\vec{\rho}_n)
    $
    for \( i\ne j\)
  \item
    $
       \det (\vec{\rho}_1,\dots,k\vec{\rho}_i,\dots,\vec{\rho}_n)
       = k\cdot \det (\vec{\rho}_1,\dots,\vec{\rho}_i,\dots,\vec{\rho}_n)
    $
    for any scalar~$k$
    % for \( k\ne 0\)
  \item 
     $
       \det(I)=1
     $
     where \( I \) is an identity matrix
\end{enumerate}
(the $\vec{\rho}\,$'s are the rows of the matrix).
We often write \( \deter{T} \) for \( \det (T) \).
%</df:Det>
\end{definition}

\begin{remark}  \label{rem:SwapRowsRedun}
%<*re:SwapRowsRedun>
Property~(2) is redundant since
\begin{equation*}
   T\;\grstep{\rho_i+\rho_j}
    \;\grstep{-\rho_j+\rho_i}
    \;\grstep{\rho_i+\rho_j}
    \;\grstep{-\rho_i}
    \;\hat{T}
\end{equation*}
swaps rows \( i \) and~\( j \).
We have listed it only for convenience.
%</re:SwapRowsRedun>
\end{remark}

\begin{remark}
In Gauss' method the operation of multiplying a row by a constant~$k$
had a restriction that $k\neq 0$. 
Property~(3) does not have the restriction because the next result
% \nearbylemma{le:IdenRowsDetZero} below 
shows that we do not need it here.
\end{remark}

% The first result shows that a function satisfying these conditions
% gives a criterion for nonsingularity.
% (Its last sentence is that, in the context of the first
% three conditions, (4) is equivalent to the condition
% that the determinant of an echelon form matrix is the product down the 
% diagonal.)

\begin{lemma}   \label{le:IdenRowsDetZero}
%<*lm:IdenRowsDetZero>
A matrix with two identical rows has a determinant of zero.
A matrix with a zero row has a determinant of zero.
A matrix is nonsingular if and only if its determinant is nonzero.
The determinant of an echelon form matrix is the product down its diagonal.
%</lm:IdenRowsDetZero>
\end{lemma}

\begin{proof}
%<*pf:IdenRowsDetZero0>
To verify the first sentence, swap the two equal rows.
The sign of the determinant changes but the matrix is the same
and so its determinant is the same.
Thus the determinant is zero.
%</pf:IdenRowsDetZero0>

%<*pf:IdenRowsDetZero1>
The second sentence follows from property~(3).
Multiply the zero row by two.
That doubles the determinant but it also
leaves the row unchanged and hence 
leaves the determinant unchanged. 
Thus the determinant must be zero.
%</pf:IdenRowsDetZero1>

%<*pf:IdenRowsDetZero2>
For the third sentence, 
where $T \rightarrow\cdots\rightarrow\hat{T}$ is the
Gauss-Jordan reduction, by the definition
the determinant of $T$ is zero if and only if
the determinant of $\hat{T}$ is zero
(although the two could differ in sign or magnitude).
A nonsingular $T$ Gauss-Jordan reduces to an identity matrix
and so has a nonzero determinant.
A singular $T$ reduces to a $\hat{T}$ with a zero row;
by the second sentence of this lemma its determinant is zero.
%</pf:IdenRowsDetZero2>

%<*pf:IdenRowsDetZero3>
The fourth sentence has two cases.
If the echelon form matrix is singular then
it has a zero row.
Thus it has a zero on its diagonal, so
the product down its diagonal is zero.
By the third sentence the determinant is zero and
therefore this matrix's determinant equals the
product down its diagonal.
%</pf:IdenRowsDetZero3>

%<*pf:IdenRowsDetZero4>
If the echelon form matrix is nonsingular then none of its diagonal entries
is zero so we can use property~(3) to get $1$'s on the diagonal
(again, the vertical bars \( \deter{\cdots} \) indicate the determinant
operation).
\begin{equation*}
  \begin{vmat}
    t_{1,1}  &t_{1,2}  &     &t_{1,n}  \\
    0        &t_{2,2}  &     &t_{2,n}  \\
             &         &\ddots         \\
    0        &         &     &t_{n,n}
  \end{vmat}
  =
  t_{1,1}\cdot t_{2,2}\cdots t_{n,n}\cdot
  \begin{vmat}
    1        &t_{1,2}/t_{1,1}  &     &t_{1,n}/t_{1,1}  \\
    0        &1                &     &t_{2,n}/t_{2,2}  \\
             &                 &\ddots         \\
    0        &                 &     &1
  \end{vmat}
\end{equation*}
Then the Jordan half of Gauss-Jordan elimination,
using property~(1) of the definition, leaves the identity matrix.
\begin{equation*}
  =
  t_{1,1}\cdot t_{2,2}\cdots t_{n,n}\cdot
  \begin{vmat}
    1        &0                &     &0                \\
    0        &1                &     &0                \\
             &                 &\ddots         \\
    0        &                 &     &1
  \end{vmat}
  =
  t_{1,1}\cdot t_{2,2}\cdots t_{n,n}\cdot 1
\end{equation*}
So in this case also, the determinant
is the product down the diagonal. 
%</pf:IdenRowsDetZero4>
\end{proof}

That gives us a way to compute the value of a determinant
function on a matrix:
do Gaussian reduction, keeping track of any changes of  
sign caused by row swaps and any scalars that we factor out, 
and finish by multiplying
down the diagonal of the echelon form result.
This algorithm is just as fast as Gauss' method and so
practical on all of the matrices that we 
will see.

\begin{example}
Doing \( \nbyn{2} \) determinants
with Gauss' method 
\begin{equation*}
   \begin{vmat}[r]
      2  &4  \\
      -1 &3
   \end{vmat}
   =
   \begin{vmat}[r]
      2  &4  \\
      0  &5
   \end{vmat}
   =10
\end{equation*}
doesn't give a big time savings
because the $\nbyn{2}$ determinant formula is easy.
However, a \( \nbyn{3} \) determinant is often easier to calculate
with Gauss' method than with the formula given earlier.
\begin{equation*}
   \begin{vmat}[r]
     2  &2  &6  \\
     4  &4  &3  \\
     0  &-3 &5
   \end{vmat}
   =
   \begin{vmat}[r]
     2  &2  &6  \\
     0  &0  &-9 \\
     0  &-3 &5
   \end{vmat}
   =
   -\begin{vmat}[r]
     2  &2  &6  \\
     0  &-3 &5  \\
     0  &0  &-9
   \end{vmat}
   =-54
\end{equation*}
\end{example}

\begin{example}
Determinants bigger than $\nbyn{3}$ go 
quickly with the Gauss' method procedure.
\begin{equation*}
   \begin{vmat}[r]
      1  &0  &1  &3  \\
      0  &1  &1  &4  \\
      0  &0  &0  &5  \\
      0  &1  &0  &1
   \end{vmat}
   =
   \begin{vmat}[r]
      1  &0  &1  &3  \\
      0  &1  &1  &4  \\
      0  &0  &0  &5  \\
      0  &0  &-1 &-3
   \end{vmat}
   =
   -\begin{vmat}[r]
      1  &0  &1  &3  \\
      0  &1  &1  &4  \\
      0  &0  &-1 &-3 \\
      0  &0  &0  &5
   \end{vmat}
   =-(-5)=5
\end{equation*}
\end{example}

The prior example illustrates an important point.
Although we have not yet found a $\nbyn{4}$ determinant formula,
if one exists then we know what value it gives to the matrix \Dash  
if there is a function with properties (1)-(4) then on the above 
matrix the function must return $5$.

\begin{lemma} \label{lm:DetFcnIsUnique}
%<*lm:DetFcnIsUnique>
For each $n$, 
if there is an $\nbyn{n}$ determinant function then it is unique.
%</lm:DetFcnIsUnique>
\end{lemma}

\begin{proof}
%<*pf:DetFcnIsUnique>
Perform Gauss' method on the 
matrix, keeping track of how the sign alternates on row swaps, and then 
get the value by
multiplying down the diagonal of the echelon form result.
By the definition and the lemma,
all $\nbyn{n}$ determinant functions must return this value on the matrix.
%</pf:DetFcnIsUnique>
\end{proof}

The `if there is an $\nbyn{n}$ determinant function' 
emphasizes that
although we can
use Gauss' method to compute the only value that a determinant function 
could possibly return, 
we haven't yet shown that such a function exists for all $n$.
In the rest of the section we will do that.

\begin{exercises}
  \item[{\em For these, assume that an $\nbyn{n}$ determinant
      function exists for all $n$.}]
  \recommended \item 
    Use Gauss' method to find each determinant.
    \begin{exparts*}
      \partsitem \( \begin{vmat}[r]
                 3  &1  &2  \\
                 3  &1  &0  \\
                 0  &1  &4
               \end{vmat} \)
      \partsitem \( \begin{vmat}[r]
                 1  &0  &0  &1 \\
                 2  &1  &1  &0 \\
                -1  &0  &1  &0 \\
                 1  &1  &1  &0
               \end{vmat} \)
    \end{exparts*}
    \begin{answer}
      \begin{exparts}
      \partsitem \( \begin{vmat}[r]
                 3  &1  &2  \\
                 3  &1  &0  \\
                 0  &1  &4
               \end{vmat}
               =
               \begin{vmat}[r]
                 3  &1  &2  \\
                 0  &0  &-2 \\
                 0  &1  &4
               \end{vmat}
               =
               -\begin{vmat}[r]
                 3  &1  &2  \\
                 0  &1  &4  \\
                 0  &0  &-2
               \end{vmat}
               =6  \)
      \partsitem \( \begin{vmat}[r]
                 1  &0  &0  &1 \\
                 2  &1  &1  &0 \\
                -1  &0  &1  &0 \\
                 1  &1  &1  &0
               \end{vmat}
               =
               \begin{vmat}[r]
                 1  &0  &0  &1 \\
                 0  &1  &1  &-2\\
                 0  &0  &1  &1 \\
                 0  &1  &1  &-1
               \end{vmat}
               =
               \begin{vmat}[r]
                 1  &0  &0  &1 \\
                 0  &1  &1  &-2\\
                 0  &0  &1  &1 \\
                 0  &0  &0  &1
               \end{vmat}
               =1    \)
      \end{exparts} 
    \end{answer}
  \item Use Gauss' method to find each.
    \begin{exparts*}
      \partsitem \( \begin{vmat}[r]
                 2  &-1  \\
                 -1 &-1
               \end{vmat}  \)
      \partsitem \( \begin{vmat}[r]
                 1  &1  &0  \\
                 3  &0  &2  \\
                 5  &2  &2
               \end{vmat} \)
    \end{exparts*}
    \begin{answer}
     \begin{exparts*}
      \partsitem \(
        \begin{vmat}[r]
                 2  &-1  \\
                 -1 &-1
               \end{vmat}
             =\begin{vmat}[r]
                 2  &-1  \\
                 0  &-3/2
               \end{vmat}=-3  \);
      \partsitem \( \begin{vmat}[r]
                 1  &1  &0  \\
                 3  &0  &2  \\
                 5  &2  &2
               \end{vmat}
              =\begin{vmat}[r]
                 1  &1  &0  \\
                 0  &-3 &2  \\
                 0  &-3 &2
               \end{vmat}
              =\begin{vmat}[r]
                 1  &1  &0  \\
                 0  &-3 &2  \\
                 0  &0  &0
               \end{vmat}
              =0 \)
     \end{exparts*}  
    \end{answer}
  \item 
    For which values of \( k \) does this system 
    have a unique solution?
    \begin{equation*}
       \begin{linsys}{4}
          x  &  &  &+ &z  &-  &w  &=  &2  \\
             &  &y &- &2z &   &   &=  &3  \\
          x  &  &  &+ &kz &   &   &=  &4  \\
             &  &  &  &z  &-  &w  &=  &2
        \end{linsys}
    \end{equation*}
    \begin{answer}
     When is the determinant not zero?
      \begin{equation*}
         \begin{vmat}
           1  &0  &1  &-1  \\
           0  &1  &-2 &0   \\
           1  &0  &k  &0   \\
           0  &0  &1  &-1
         \end{vmat}
         =
         \begin{vmat}
           1  &0  &1  &-1  \\
           0  &1  &-2 &0   \\
           0  &0  &k-1&1   \\
           0  &0  &1  &-1
         \end{vmat}           
      \end{equation*}
      Obviously, 
      $k=1$ gives nonsingularity and hence a nonzero determinant. 
      If $k\neq 1$ then 
      we get echelon form with a $(-1/k-1)\rho_3+\rho_4$ combination.
      \begin{equation*}
         =
         \begin{vmat}
           1  &0  &1  &-1  \\
           0  &1  &-2 &0   \\
           0  &0  &k-1&1   \\
           0  &0  &0  &-1-(1/k-1)
         \end{vmat}
      \end{equation*}
      Multiplying down the diagonal gives $(k-1)(-1-(1/k-1))=-(k-1)-1=-k$.
      Thus the matrix has a nonzero determinant, and so the system
      has a unique solution, if and only if \( k\neq 0 \).  
    \end{answer}
  \recommended \item 
    Express each of these in terms of \( \deter{H} \).
    \begin{exparts}
      \partsitem \( \begin{vmat}
            h_{3,1}  &h_{3,2} &h_{3,3} \\
            h_{2,1}  &h_{2,2} &h_{2,3} \\
            h_{1,1}  &h_{1,2} &h_{1,3}
          \end{vmat} \)
      \partsitem \( \begin{vmat}
           -h_{1,1}   &-h_{1,2}  &-h_{1,3} \\
           -2h_{2,1}  &-2h_{2,2} &-2h_{2,3} \\
           -3h_{3,1}  &-3h_{3,2} &-3h_{3,3}
          \end{vmat} \)
      \partsitem \( \begin{vmat}
            h_{1,1}+h_{3,1}  &h_{1,2}+h_{3,2} &h_{1,3}+h_{3,3} \\
            h_{2,1}          &h_{2,2}         &h_{2,3} \\
            5h_{3,1}         &5h_{3,2}        &5h_{3,3}
          \end{vmat} \)
    \end{exparts}
    \begin{answer}
      \begin{exparts}
        \partsitem Property~(2) of the definition of determinants
          applies via the swap $\rho_1\leftrightarrow\rho_3$.
          \begin{equation*}
            \begin{vmat}
              h_{3,1}  &h_{3,2} &h_{3,3} \\
              h_{2,1}  &h_{2,2} &h_{2,3} \\
              h_{1,1}  &h_{1,2} &h_{1,3}
            \end{vmat}
            =
            -\begin{vmat}
              h_{1,1}  &h_{1,2} &h_{1,3} \\
              h_{2,1}  &h_{2,2} &h_{2,3} \\
              h_{3,1}  &h_{3,2} &h_{3,3} 
            \end{vmat}
          \end{equation*}
        \partsitem Property~(3) applies.
          \begin{equation*}
            \begin{vmat}
             -h_{1,1}   &-h_{1,2}  &-h_{1,3} \\
             -2h_{2,1}  &-2h_{2,2} &-2h_{2,3} \\
             -3h_{3,1}  &-3h_{3,2} &-3h_{3,3}
            \end{vmat}
            =
            (-1)\cdot(-2)\cdot(-3)\cdot
            \begin{vmat}
             h_{1,1}   &h_{1,2}  &h_{1,3} \\
             h_{2,1}   &h_{2,2}  &h_{2,3} \\
             h_{3,1}   &h_{3,2}  &h_{3,3}
            \end{vmat}
            =
            (-6)\cdot
            \begin{vmat}
             h_{1,1}   &h_{1,2}  &h_{1,3} \\
             h_{2,1}   &h_{2,2}  &h_{2,3} \\
             h_{3,1}   &h_{3,2}  &h_{3,3}
            \end{vmat}
          \end{equation*}
        \partsitem 
          \begin{align*}
          \begin{vmat}
            h_{1,1}+h_{3,1}  &h_{1,2}+h_{3,2} &h_{1,3}+h_{3,3} \\
            h_{2,1}          &h_{2,2}         &h_{2,3} \\
            5h_{3,1}         &5h_{3,2}        &5h_{3,3}
          \end{vmat}
          &=
          5\cdot     
          \begin{vmat}
            h_{1,1}+h_{3,1}  &h_{1,2}+h_{3,2} &h_{1,3}+h_{3,3} \\
            h_{2,1}          &h_{2,2}         &h_{2,3} \\
            h_{3,1}          &h_{3,2}         &h_{3,3}
          \end{vmat}                                     \\
          &=5\cdot
          \begin{vmat}
            h_{1,1}          &h_{1,2}         &h_{1,3} \\
            h_{2,1}          &h_{2,2}         &h_{2,3} \\
            h_{3,1}          &h_{3,2}         &h_{3,3}
          \end{vmat}
          \end{align*}
      \end{exparts}  
     \end{answer}
  \recommended \item 
    Find the determinant of a diagonal matrix.
    \begin{answer}
       A diagonal matrix is in echelon form, so
       the determinant is the product down the diagonal.
    \end{answer}
  \item 
    Describe the solution set of a homogeneous linear system if the
    determinant of the matrix of coefficients is nonzero.
    \begin{answer}
      It is the trivial subspace.  
    \end{answer}
  \recommended \item 
    Show that this determinant is zero.
    \begin{equation*}
      \begin{vmat}
        y+z  &x+z  &x+y  \\
        x    &y    &z    \\
        1    &1    &1
      \end{vmat}
    \end{equation*}
    \begin{answer} 
      Adding the second row to the first gives a matrix whose first row
      is $x+y+z$ times its third row.  
    \end{answer}
  \item 
    \begin{exparts}
      \partsitem Find the $\nbyn{1}$, $\nbyn{2}$, and $\nbyn{3}$ matrices
        with $i,j$ entry given by $(-1)^{i+j}$.  
      \partsitem Find the determinant of the square matrix with \( i,j \)
        entry \( (-1)^{i+j} \).
    \end{exparts}
    \begin{answer}
      \begin{exparts}
        \partsitem 
          $\begin{mat}[r]
            1
          \end{mat}$,
          $\begin{mat}[r]
            1  &-1  \\
            -1  &1
          \end{mat}$,
          $\begin{mat}[r]
            1   &-1  &1   \\
            -1  &1   &-1  \\
            1   &-1  &1
          \end{mat}$
        \partsitem The determinant in the $\nbyn{1}$ case is $1$.
          In every other case the second row is the negative of the first,
          and so matrix is singular and the determinant is zero. 
      \end{exparts} 
    \end{answer}
  \item 
    \begin{exparts}
      \partsitem Find the $\nbyn{1}$, $\nbyn{2}$, and $\nbyn{3}$ matrices
        with $i,j$ entry given by $i+j$.  
      \partsitem Find the determinant of the square matrix with 
        $i,j$ entry $i+j$.
    \end{exparts}
    \begin{answer}
      \begin{exparts}
        \partsitem 
          $\begin{mat}[r]
            2
          \end{mat}$,
          $\begin{mat}[r]
            2  &3  \\
            3  &4
          \end{mat}$,
          $\begin{mat}[r]
            2  &3  &4  \\
            3  &4  &5  \\
            4  &5  &6
          \end{mat}$
        \partsitem The $\nbyn{1}$ and $\nbyn{2}$ cases yield these.
          \begin{equation*}
            \begin{vmat}[r]
              2
            \end{vmat}
            =2
            \qquad
            \begin{vmat}[r]
              2  &3  \\
              3  &4
            \end{vmat}=-1
          \end{equation*}
          And $\nbyn{n}$ matrices with $n\geq 3$ are singular, e.g.,
          \begin{equation*}
            \begin{vmat}[r]
              2  &3  &4  \\
              3  &4  &5  \\
              4  &5  &6
            \end{vmat}=0
          \end{equation*}
          because twice the second row minus the first row 
          equals the third row.
          Checking this is routine.
      \end{exparts} 
    \end{answer}
  \recommended \item 
    Show that determinant functions are not linear by  
    giving a case where \( \deter{A+B}\neq\deter{A}+\deter{B} \).
    \begin{answer}
      This one 
      \begin{equation*}
         A=B=
         \begin{mat}[r]
           1  &2  \\
           3  &4
         \end{mat}
      \end{equation*}
      is easy to check.
      \begin{equation*}
         \deter{A+B}
         =
         \begin{vmat}[r]
           2  &4  \\
           6  &8
         \end{vmat}
         =-8
         \qquad
         \deter{A}+\deter{B}
         =
         -2-2
         =-4
      \end{equation*}
      By the way, this also gives an example where scalar multiplication 
      is not preserved 
      $\deter{2\cdot A}\neq 2\cdot\deter{A}$.  
     \end{answer}
  \item 
    The second condition in the definition, that row swaps change the 
    sign of a determinant, is somewhat annoying.
    It means we have to keep track of the number of swaps, to compute how
    the sign alternates.
    Can we get rid of it?
    Can we replace it with the condition that row swaps leave the determinant
    unchanged?
    (If so then we would need new $\nbyn{1}$,
     $\nbyn{2}$, and $\nbyn{3}$ formulas, but that would be a minor matter.)
     \begin{answer}
       No, we cannot replace it.
       \nearbyremark{rem:SwapRowsRedun}
       shows that the four conditions after the replacement would  
       conflict \Dash  no function satisfies all four.
     \end{answer}
  \item 
    Prove that the determinant of any triangular matrix, upper or lower,
    is the product down its diagonal.
    \begin{answer}
      A upper-triangular matrix is in echelon form.

      A lower-triangular matrix is either singular or nonsingular.
      If it is singular then it has a zero on its diagonal and so its 
      determinant (namely, zero) is indeed the product down its diagonal. 
      If it is nonsingular then it has no zeroes on its diagonal, and
      we can reduce it by Gauss' method to echelon
      form without changing the diagonal.  
    \end{answer}
  \item
    Refer to the definition of elementary matrices in the Mechanics
    of Matrix Multiplication subsection.
    \begin{exparts}
      \partsitem What is the determinant of each kind of elementary matrix?
      \partsitem Prove that if \( E \) is any elementary matrix then
        \( \deter{ES}=\deter{E}\deter{S} \) for any appropriately sized
        \( S \).
      \partsitem \textit{(This question doesn't involve determinants.)}
        Prove that if \( T \) is singular then a product \( TS \) is
        also singular.
      \partsitem Show that \( \deter{TS}=\deter{T}\deter{S} \).
      \partsitem Show that if \( T \) is nonsingular then
          \( \deter{T^{-1}}=\deter{T}^{-1} \).
    \end{exparts}
    \begin{answer}
      \begin{exparts}
        \partsitem The properties in the definition of determinant 
          show that 
          \( \deter{M_i(k)}=k \),
          \( \deter{P_{i,j}}=-1 \),
          and
          \( \deter{C_{i,j}(k)}=1 \).
        \partsitem The three cases are easy to check by recalling the action
          of left multiplication by each type of matrix.
        \partsitem If \( TS \) is invertible \( (TS)M=I \) then
          the associative property of matrix multiplication 
          \( T(SM)=I \) shows that \( T \) is invertible.
          So if \( T \) is not invertible then neither is \( TS \).
        \partsitem If \( T \) is singular then apply the prior answer:
          \( \deter{TS}=0 \) and
          \( \deter{T}\cdot\deter{S}=0\cdot\deter{S}=0 \).
          If \( T \) is not singular then we can write it as a product of
          elementary matrices
          $
            \deter{TS}=\deter{E_r\cdots E_1S}=
            \deter{E_r}\cdots\deter{E_1}\cdot\deter{S}=
            \deter{E_r\cdots E_1}\deter{S}=\deter{T}\deter{S}
          $.
        \partsitem \( 1=\deter{I}=\deter{T\cdot T^{-1}}
                     =\deter{T}\deter{T^{-1}} \)
      \end{exparts}  
     \end{answer} 
  \item 
    Prove that the determinant of a product is the product of the
    determinants \( \deter{TS}=\deter{T}\,\deter{S} \) in this way.
    Fix the \( \nbyn{n} \) matrix \( S \) and consider the function
    \( \map{d}{\matspace_{\nbyn{n}}}{\Re} \) given by
    \( T\mapsto \deter{TS}/\deter{S} \).
    \begin{exparts}
      \partsitem Check that \( d \) satisfies property~(1) in the definition of
        a determinant function.
      \partsitem Check property~(2).
      \partsitem Check property~(3).
      \partsitem Check property~(4).
      \partsitem Conclude the determinant of a product is the product of the
        determinants.
    \end{exparts}
    \begin{answer}
      \begin{exparts}
        \partsitem We must show that if
          \begin{equation*}
            T\grstep{k\rho_i+\rho_j}\hat{T}
          \end{equation*}
          then $d(T)=\deter{TS}/\deter{S}
          =\deter{\hat{T}S}/\deter{S}=d(\hat{T})$.
          We will be done if we show that combining rows first and
          then multiplying to get \( \hat{T}S \) gives the same result as
          multiplying first to get \( TS \) and then combining
          (because the determinant \( \deter{TS} \) is unaffected by the
          combination so we'll then have \( \deter{\hat{T}S}=\deter{TS} \), and
          hence \( d(\hat{T})=d(T) \)).
          That argument runs:~after adding 
          \( k \) times row~\( i \) of \( TS \) to
          row~$j$ of \( TS \), the \( j,p \) entry is
          \( (kt_{i,1}+t_{j,1})s_{1,p}+\dots+(kt_{i,r}+t_{j,r})s_{r,p} \),
          which is the \( j,p \) entry of \( \hat{T}S \).
        \partsitem We need only show that swapping
          $
            T\smash[b]{\grstep{\rho_i\swap\rho_j}}\hat{T}
          $
          and then multiplying to get \( \hat{T}S \) gives the same result as
          multiplying \( T \) by \( S \) and then swapping (because,
          as the determinant \( \deter{TS} \) changes sign on
          the row swap, we'll then have \( \deter{\hat{T}S}=-\deter{TS} \),
          and so \( d(\hat{T})=-d(T) \)).
          That argument runs just like the prior one.
        \partsitem Not surprisingly by now, we need only show that 
          multiplying a row by a scalar
          $
            T\smash[b]{\grstep{k\rho_i}}\hat{T}
          $
          and then computing \( \hat{T}S \) gives the same result as
          first computing \( TS \) and then multiplying the row by \( k \)
          (as the determinant \( \deter{TS} \) is rescaled by \( k \)
          the multiplication, we'll have \( \deter{\hat{T}S}=k\deter{TS} \),
          so \( d(\hat{T})=k\,d(T) \)).
          The argument runs just as above.
        \partsitem Clear.
        \partsitem Because we've shown that \( d(T) \) is a determinant
          and that determinant functions (if they exist) are
          unique, we have that
          so \( \deter{T}=d(T)=\deter{TS}/\deter{S} \).
      \end{exparts}  
    \end{answer}
  \item 
    A \definend{submatrix}\index{submatrix}\index{matrix!submatrix} 
    of a given matrix $A$ is one that we get by deleting 
    some of the rows and columns of $A$.
    Thus, the first matrix here is a submatrix of the second.
    \begin{equation*}
      \begin{mat}[r]
        3  &1  \\
        2  &5
      \end{mat}
      \qquad
      \begin{mat}[r]
        3  &4  &1  \\
        0  &9  &-2 \\
        2  &-1 &5
      \end{mat}
    \end{equation*}
    Prove that for any square matrix,
    the rank of the matrix is $r$ if and only if
    \( r \) is the largest
    integer such that there is an \( \nbyn{r} \) submatrix with a nonzero
    determinant.
    \begin{answer}
      We will first argue that a rank \( r \) matrix has a \( \nbyn{r} \)
      submatrix with nonzero determinant.
      A rank \( r \) matrix has a linearly independent set of \( r \) rows.
      A matrix made from those rows will have row rank \( r \) and thus has
      column rank \( r \).
      Conclusion: from those \( r \) rows we can extract a linearly
      independent set of \( r \) columns, and so the original matrix has a
      \( \nbyn{r} \) submatrix of rank \( r \).

      We finish by showing that if \( r \) is the largest such integer then
      the rank of the matrix is \( r \).
      We need only show, by the maximality of \( r \),
      that if a matrix has a \( \nbyn{k} \) submatrix of
      nonzero determinant then the rank of the matrix is at least \( k \).
      Consider such a \( \nbyn{k} \) submatrix.
      Its rows are parts of the rows of the original matrix, clearly the
      set of whole rows is linearly independent.
      Thus the row rank of the original matrix is at least \( k \), and the row
      rank of a matrix equals its rank.  
   \end{answer}
  \recommended \item
    Prove that a matrix with rational entries has a rational determinant.
    \begin{answer}
      A matrix with only rational entries reduces with Gauss'
      method to an echelon form matrix using only rational arithmetic.
      Thus the entries on the diagonal must be rationals, and so the product
      down the diagonal is rational.  
    \end{answer}
  \puzzle \item 
     \cite{Monthly53p115}
     Find the element of likeness in (a)~simplifying a fraction, (b)~powdering
     the nose, (c)~building new steps on the church, (d)~keeping emeritus
     professors on campus, (e)~putting \( B \), \( C \), \( D \) in the
     determinant
     \begin{equation*}
       \begin{vmat}
         1   &a   &a^2  &a^3  \\
         a^3 &1   &a    &a^2  \\
         B   &a^3 &1    &a    \\
         C   &D   &a^3  &1
       \end{vmat}.
     \end{equation*}
     \begin{answer}
       \answerasgiven
       The value \( (1-a^4)^3 \) of the determinant is independent of the
       values \( B \), \( C \), \( D \).
       Hence operation~(e) does not change the value of the determinant
       but merely changes its appearance.
       Thus the element of likeness in (a), (b), (c), (d), and (e) is
       only that the appearance of the principle entity is changed.
       The same element appears in (f)~changing the name-label of a rose,
       (g)~writing a decimal integer in the scale of \( 12 \), (h)~gilding
       the lily, (i)~whitewashing a politician, and (j)~granting an honorary
       degree.  
    \end{answer}
\end{exercises}



















\subsection{The Permutation Expansion}
%<*DifferentGaussMethodReductions>
The prior subsection defines a function to be a determinant if it
satisfies four conditions and
shows that there is at most one $\nbyn{n}$ determinant function for
each $n$.
What is left is to show that for each $n$ such a function exists.

But, we easily compute determinants: 
use Gauss' method, keeping track of the sign changes from row swaps, 
and end by multiplying down the diagonal.
So how could such a function not exist?

The difficulty is to show that 
the computation gives a well-defined\Dash that is, unique\Dash result.
Consider these two Gauss' method reductions
of the same matrix, the first without any row swap
\begin{equation*}
  \begin{mat}[r]
    1  &2  \\
    3  &4
  \end{mat}
  \grstep{-3\rho_1+\rho_2}
  \begin{mat}[r]
    1  &2  \\
    0  &-2
  \end{mat}
\end{equation*}
and the second with one.
\begin{equation*}
  \begin{mat}[r]
    1  &2  \\
    3  &4
  \end{mat}
  \grstep{\rho_1\leftrightarrow\rho_2}
  \begin{mat}[r]
    3  &4  \\
    1  &2
  \end{mat}
  \grstep{-(1/3)\rho_1+\rho_2}
  \begin{mat}[r]
    3  &4  \\
    0  &2/3
  \end{mat}
\end{equation*}
Both yield the determinant $-2$ 
since in the second one
we note that the row swap changes the sign of the result
we get by multiplying down the diagonal.
%</DifferentGaussMethodReductions>
To illustrate how a computation that is like the ones that we are doing could 
fail to be well-defined, 
suppose that \nearbydefinition{def:Det} did not include condition~(2). 
That is, suppose that we instead tried to define determinants so that
the value would not change on a row swap.
Then first reduction above would
yield $-2$ while the second would yield $+2$.
We could still do computations but they wouldn't give consistent outcomes\Dash
there is no 
function that satisfies conditions (1), (3), (4), and also this
altered second condition. 

Of course, observing that \nearbydefinition{def:Det} does the right thing
with these two reductions of the above matrix is not enough.
That is, the way that we have given to compute determinant values does not
plainly eliminate the possibility that there might be, say,
two reductions of some $\nbyn{7}$~matrix that lead to different 
determinant value outputs.
In that case we would not have a function, since the definition of a
function is that for each input there must be exactly one output. 
In the rest of this 
section we will show that there is never a conflict.

% \begin{remark}
% The natural way approach to this would be 
% to define the determinant function with:~``The value of the
% function is the result of doing Gauss' method,
% keeping track of row swaps, and finishing by multiplying down the diagonal''. 
% (Since Gauss' method allows for some variation, as we saw above, 
% we would have to fix an explicit algorithm.)
% Then we would be done if we verified that 
% this way algorithm satisfies the four properties.
% However, how to verify this is not evident.
% So the development below will not proceed in this way.
% \end{remark}

To do this we will define an alternative way to find
the value of a determinant.
(This new way is less useful in practice since it 
makes the computations awkward and slow, which is why we didn't start with it.
But it is useful for theory and it makes the proof that we need easier.)
The key idea is in   
property~(3) of \nearbydefinition{def:Det}.
It shows that the determinant function is not linear.

\begin{example}
For this matrix
\begin{equation*}
   A=\begin{mat}[r]
       2  &1  \\
      -1  &3
     \end{mat}
\end{equation*} 
\( \det(2A)\neq 2\cdot\det(A) \).
Instead, scalars come out of each of the rows separately.
\begin{equation*}
   \begin{vmat}[r]
       4  &2  \\
      -2  &6
     \end{vmat}
  =2\cdot\begin{vmat}[r]
       2  &1  \\
      -2  &6
     \end{vmat}
  =4\cdot\begin{vmat}[r]
       2  &1  \\
      -1  &3
     \end{vmat}
\end{equation*}
\end{example}

Since scalars come out a row at a time, we might guess that
determinants are linear a row at a time.

\begin{definition} \label{def:multilinear}
Let \( V \) be a vector space.
A map \( \map{f}{V^n}{\Re} \) is
\definend{multilinear}\index{function!multilinear}\index{multilinear}
if  
\begin{enumerate}
  \item 
    $
      f(\vec{\rho}_1,\dots,\vec{v}+\vec{w},
      \ldots,\vec{\rho}_n)
      =f(\vec{\rho}_1,\dots,\vec{v},\dots,\vec{\rho}_n)
      +f(\vec{\rho}_1,\dots,\vec{w},\dots,\vec{\rho}_n)
    $
  \item 
    $
      f(\vec{\rho}_1,\dots,k\vec{v},\dots,\vec{\rho}_n)
      =k\cdot f(\vec{\rho}_1,\dots,\vec{v},\dots,\vec{\rho}_n)
    $
\end{enumerate}
for \( \vec{v}, \vec{w}\in V \) and \( k\in\Re \).
\end{definition}

\begin{lemma}  \label{lem:DetsMultilinear}
%<*lm:DetsMultilinear>
Determinants are multilinear.
%</lm:DetsMultilinear>
\end{lemma}

\begin{proof}
%<*pf:DetsMultilinear0>
Property~(2) here is just condition~(3) in \nearbydefinition{def:Det}
so we need only verify property~(1).
%</pf:DetsMultilinear0>
% \begin{equation*}
%   \det(\vec{\rho}_1,\dots,\vec{v}+\vec{w},
%       \dots,\vec{\rho}_n)
%   =\det(\vec{\rho}_1,\dots,\vec{v},\dots,\vec{\rho}_n)
%   +\det(\vec{\rho}_1,\dots,\vec{w},\dots,\vec{\rho}_n)
% \end{equation*}

%<*pf:DetsMultilinear1>
There are two cases.
If the set of other rows
\(
  \set{\vec{\rho}_1,\dots,\vec{\rho}_{i-1},\vec{\rho}_{i+1},
       \dots,\vec{\rho}_n} 
\)
is linearly dependent then all three matrices are singular and so all
three determinants are zero and the equality is trivial.
%</pf:DetsMultilinear1>

%<*pf:DetsMultilinear2>
Therefore assume that the set of other rows is linearly independent.
This set has $n-1$ members so we can make a basis by 
adding one more vector
$\sequence{\vec{\rho}_1,\dots,\vec{\rho}_{i-1},\vec{\beta},
               \vec{\rho}_{i+1},\dots,\vec{\rho}_n}$.
Express $\vec{v}$ and $\vec{w}$ with respect to this basis
\begin{align*}
  \vec{v} &=v_1\vec{\rho}_1+\dots+v_{i-1}\vec{\rho}_{i-1}+v_i\vec{\beta}
            +v_{i+1}\vec{\rho}_{i+1}+\dots+v_n\vec{\rho}_n                \\
  \vec{w} &= w_1\vec{\rho}_1+\dots+w_{i-1}\vec{\rho}_{i-1}+w_i\vec{\beta}
            +w_{i+1}\vec{\rho}_{i+1}+\dots+w_n\vec{\rho}_n
\end{align*}
and add.
\begin{equation*}
  \vec{v}+\vec{w}
  =
  (v_1+w_1)\vec{\rho}_1+\dots+(v_i+w_i)\vec{\beta}
            +\dots+(v_n+w_n)\vec{\rho}_n 
\end{equation*}
Consider the left side of property~(1) and expand $\vec{v}+\vec{w}$.
% $\det (\vec{\rho}_1,\dots,\vec{v}+\vec{w},\dots,\vec{\rho}_n)$.
\begin{equation*}
    \det (\vec{\rho}_1,\dots,\,(v_1+w_1)\vec{\rho}_1+\dots+(v_i+w_i)\vec{\beta}
            +\dots+(v_n+w_n)\vec{\rho}_n,\,\dots,\vec{\rho}_n) 
   \tag{*}
\end{equation*}
% Gauss' method enables us to eliminate the terms for $\vec{\rho}_1$,
% $\vec{\rho}_2$, etc., from the expanded expression.
% the $(v_1+w_1)\vec{\rho}_1+\dots+(v_i+w_i)\vec{\beta}
%             +\dots+(v_n+w_n)\vec{\rho}_n$ expression.
By the definition of determinant's condition~(1), the value of 
($*$) % the determinant
% $\det(\vec{\rho}_1,\dots,\vec{v}+\vec{w},\dots,\vec{\rho}_n)$
is unchanged by the operation of 
adding $-(v_1+w_1)\vec{\rho}_1$ to the $i$-th row $\vec{v}+\vec{w}$.
The $i$-th row becomes this.
\begin{equation*}
  \vec{v}+\vec{w}-(v_1+w_1)\vec{\rho}_1
  =
  (v_2+w_2)\vec{\rho}_2+\cdots+(v_i+w_i)\vec{\beta}
            +\dots+(v_n+w_n)\vec{\rho}_n 
\end{equation*}
%</pf:DetsMultilinear2>
%<*pf:DetsMultilinear3>
Next add $-(v_2+w_2)\vec{\rho}_2$, etc., to eliminate 
all of the terms from the other rows.
Apply the definition of determinant's condition~(3).
\begin{multline*}
  \det (\vec{\rho}_1,\dots,\vec{v}+\vec{w},\dots,\vec{\rho}_n)          \\
  \begin{aligned}
    &=\det (\vec{\rho}_1,\dots,(v_i+w_i)\cdot\vec{\beta},\dots,\vec{\rho}_n) \\
    &=(v_i+w_i)\cdot\det (\vec{\rho}_1,\dots,\vec{\beta},\dots,\vec{\rho}_n) \\
    &=v_i\cdot \det (\vec{\rho}_1,\dots,\vec{\beta},\dots,\vec{\rho}_n)  
     +w_i\cdot \det (\vec{\rho}_1,\dots,\vec{\beta},\dots,\vec{\rho}_n)
  \end{aligned}
\end{multline*}
Now this is a sum of two determinants.
To finish, bring $v_i$ and $w_i$ back inside in front of the 
$\vec{\beta}$'s
and use row combinations again, 
this time to reconstruct the expressions of $\vec{v}$
and $\vec{w}$ in terms of the basis.
That is, start with the operations
of adding $v_1\vec{\rho}_1$ to $v_i\vec{\beta}$ 
and $w_1\vec{\rho}_1$ to $w_i\vec{\rho}_1$, etc.,
to get the expansions of $\vec{v}$ and $\vec{w}$.
%</pf:DetsMultilinear3>
\end{proof}

Multilinearity allows us to expand a determinant into a sum of
determinants, each of which involves a simple matrix.

\begin{example}
Use property~(1) of multilinearity to
break up the first row
\begin{equation*}
  \begin{vmat}[r]
     2  &1  \\
     4  &3
  \end{vmat}
  =
  \begin{vmat}[r]
     2  &0  \\
     4  &3
  \end{vmat}
  +
  \begin{vmat}[r]
     0  &1  \\
     4  &3
  \end{vmat}
\end{equation*}
and then break each of those two along the second row.
\begin{equation*}
  =\begin{vmat}[r]
     2  &0  \\
     4  &0
  \end{vmat}
  +
  \begin{vmat}[r]
     2  &0  \\
     0  &3
  \end{vmat}
  +
  \begin{vmat}[r]
     0  &1  \\
     4  &0
  \end{vmat}
  +
  \begin{vmat}[r]
     0  &1  \\
     0  &3
  \end{vmat}
\end{equation*}
We are left with four determinants such that 
in each row of each of the four there is
a single entry from the original matrix.
\end{example}

\begin{example}
In the same way,
a \( \nbyn{3} \) determinant separates into a sum of 
many simpler determinants.
Splitting along the first row produces three determinants
(we have highlighted the zero in the $1,3$ position to set it off visually
from the zeroes that appear as part of the splitting).
\begin{equation*}
  \begin{vmat}[r]
     2              &1  &-1  \\
     4              &3  &\highlight{0}  \\
     2              &1  &5
  \end{vmat}
  =
  \begin{vmat}[r]
     2              &0  &0   \\
     4              &3  &\highlight{0}  \\
     2              &1  &5
  \end{vmat}
  +
  \begin{vmat}[r]
     0              &1  &0   \\
     4              &3  &\highlight{0}   \\
     2              &1  &5
  \end{vmat}
  +
  \begin{vmat}[r]
     0              &0  &-1  \\
     4              &3  &\highlight{0}  \\
     2  &1  &5
  \end{vmat}                       
\end{equation*}
Each of these splits in three along the second row.
Each of the nine splits in three along the third row, resulting
in twenty seven determinants
such that each row contains a single entry from the starting matrix.
\begin{equation*}
  =
  \begin{vmat}[r]
     2              &0  &0   \\
     4              &0  &0   \\
     2              &0  &0
  \end{vmat}
  +
  \begin{vmat}[r]
     2  &0  &0   \\
     4  &0  &0   \\
     0  &1  &0
  \end{vmat}
  +
  \begin{vmat}[r]
     2  &0  &0   \\
     4  &0  &0   \\
     0  &0  &5
  \end{vmat}
  +
  \begin{vmat}[r]
     2  &0  &0   \\
     0  &3  &0   \\
     2  &0  &0
  \end{vmat}
  +\dots+
  \begin{vmat}[r]
     0  &0  &-1  \\
     0  &0  &\highlight{0}  \\
     0  &0  &5
  \end{vmat}
\end{equation*}
\end{example}

So with multilinearity, an \( \nbyn{n} \) determinant expands into a
sum of \( n^n \) determinants where
each row of each summand contains a single entry from the
starting matrix.
However, many of these summand determinants are zero.

\begin{example} \label{ex:SamplePermExp}
In each of these examples from the prior expansion,
two of the entries from the original matrix are in the same column.
\begin{equation*}
  \begin{vmat}[r]
     2               &0  &0   \\
     4               &0  &0  \\
     0               &1  &0
  \end{vmat}
  \qquad
  \begin{vmat}[r]
     0               &0  &-1  \\
     0               &3  &0  \\
     0               &0  &5
  \end{vmat}
  \qquad\
  \begin{vmat}[r]
     0               &1  &0   \\
     0               &0  &\highlight{0}  \\
     0               &0  &5
  \end{vmat}
\end{equation*}
For instance
in the first matrix, the $2$ and the $4$ both come from the first column
of the original matrix.
Any such matrix is singular because 
one row is a multiple of the other.
Thus, any such determinant is zero, by \nearbylemma{le:IdenRowsDetZero}.

With that observation 
the above expansion of the \( \nbyn{3} \) determinant into
the sum of the twenty seven determinants simplifies to the sum of these six,
the ones where the entries from the original matrix come not just one per row
but also one per column.
\begin{align*}
  \begin{vmat}[r]
     2  &1  &-1  \\
     4  &3  &\highlight{0}  \\
     2  &1  &5
  \end{vmat}
  &=\begin{vmat}[r]
     2  &0  &0   \\
     0  &3  &0   \\
     0  &0  &5
  \end{vmat}
  +
   \begin{vmat}[r]
     2  &0  &0   \\
     0  &0  &\highlight{0}   \\
     0  &1  &0
  \end{vmat}                      \\
  &\quad\hbox{}+\begin{vmat}[r]
     0  &1  &0   \\
     4  &0  &0   \\
     0  &0  &5
  \end{vmat}
  +
   \begin{vmat}[r]
     0  &1  &0   \\
     0  &0  &\highlight{0}   \\
     2  &0  &0
  \end{vmat}                      \\
  &\quad\hbox{}+\begin{vmat}[r]
     0  &0  &-1  \\
     4  &0  &0   \\
     0  &1  &0
  \end{vmat}
  +
   \begin{vmat}[r]
     0  &0  &-1  \\
     0  &3  &0    \\
     2  &0  &0
  \end{vmat}                      
\end{align*}
We can bring out the scalars.
\begin{align*}
  &=(2)(3)(5)\begin{vmat}[r]
               1  &0  &0  \\
               0  &1  &0  \\
               0  &0  &1
            \end{vmat}
  +(2)(\highlight{0})(1)\begin{vmat}[r]
               1  &0  &0  \\
               0  &0  &1  \\
               0  &1  &0
            \end{vmat}                     \\
  &\quad\hbox{}+(1)(4)(5)\begin{vmat}[r]
               0  &1  &0  \\
               1  &0  &0  \\
               0  &0  &1
            \end{vmat}
  +(1)(\highlight{0})(2)\begin{vmat}[r]
               0  &1  &0  \\
               0  &0  &1  \\
               1  &0  &0
            \end{vmat}                       \\
  &\quad\hbox{}+(-1)(4)(1)\begin{vmat}[r]
               0  &0  &1  \\
               1  &0  &0  \\
               0  &1  &0
            \end{vmat}
  +(-1)(3)(2)\begin{vmat}[r]
               0  &0  &1  \\
               0  &1  &0  \\
               1  &0  &0
            \end{vmat}                       
\end{align*}
To finish, we evaluate those six determinants by row-swapping them 
to the identity matrix,
keeping track of the sign changes.
\begin{align*}
  &=30\cdot (+1)+0\cdot (-1)  \\
  &\quad\hbox{}+20\cdot (-1)+0\cdot (+1) \\
  &\quad \hbox{}-4\cdot (+1)-6\cdot (-1)=12
\end{align*}
\end{example}

That example captures the new calculation scheme.
Multilinearity gives us 
many separate determinants, each with one entry per row from the 
original matrix.
Most of these have
one row that is a multiple of another
so we can omit them.
We are left with those determinants that have one entry per row and
column from the original matrix.
By factoring out the scalars we can further reduce the
determinants that we must compute to those
one-entry-per-row-and-column matrices where all the entries are $1$'s.

Recall Definition~Three.IV.\ref{df:PermutationMatrix}, that
a square matrix whose entries are~$0$'s except for
a single~$1$ in each row and column is a
\definend{permutation matrix}.\index{permutation matrix}%
\index{matrix!permutation}
%<*NotationForPermutationMatrices>
We now introduce a notation for permutation matrices.
Let $\iota_j$ be the row vector that is all $0$'s except for a $1$ in its 
$j$-th entry, so that the four-wide $\iota_2$ is $\rowvec{0 &1 &0 &0}$.
We can construct permutation matrices by 
permuting, that is, scrambling, the numbers $1$, $2$, \ldots, $n$,
and using them as indices on the $\iota$'s.
For instance, to get a \( \nbyn{4} \) permutation
matrix, we can scramble the numbers from $1$ to $4$ into 
this sequence \( \sequence{3,2,1,4} \) and take the corresponding 
$\iota$'s.
\begin{equation*}
   \begin{mat}
      \iota_{3} \\
      \iota_{2} \\
      \iota_{1} \\
      \iota_{4} 
   \end{mat}=
  \begin{mat}[r]
     0  &0  &1  &0  \\
     0  &1  &0  &0  \\
     1  &0  &0  &0  \\
     0  &0  &0  &1
  \end{mat}
\end{equation*}
%</NotationForPermutationMatrices>

\begin{definition}  \label{df:permutation}
%<*df:permutation>
An \definend{\( n \)-permutation}\index{permutation}
is a sequence consisting of the numbers 
$1$, \ldots, $n$.
%</df:permutation>
\end{definition}

\begin{example} \label{ex:AllTwoThreePerms}
The $2$-permutations are
\( \phi_1=\sequence{1,2} \) and \( \phi_2=\sequence{2,1} \).
These are the associated permutation matrices.
\begin{equation*}
  P_{\phi_1}
  =\begin{mat}
      \iota_1 \\
      \iota_2 
   \end{mat}
  =\begin{mat}[r]
    1  &0         \\
    0  &1   
  \end{mat}
  \qquad
  P_{\phi_2}
   =\begin{mat}
      \iota_2 \\
      \iota_1 
   \end{mat}
   =\begin{mat}[r]
    0  &1         \\
    1  &0   
  \end{mat}
\end{equation*}
We sometimes write permutations as functions, e.g., \( \phi_2(1)=2 \),
and \( \phi_2(2)=1 \).
Then $P_{\phi_2}$'s first row is 
$\iota_{\phi_2(1)}=\iota_2$ and its second is $\iota_{\phi_2(2)}=\iota_1$.

The $3$-permutations are
\( \phi_1=\sequence{1,2,3} \),
\( \phi_2=\sequence{1,3,2} \),
\( \phi_3=\sequence{2,1,3} \),
\( \phi_4=\sequence{2,3,1} \),
\( \phi_5=\sequence{3,1,2} \), and
\( \phi_6=\sequence{3,2,1} \).
One example permutation matrix is 
$P_{\phi_5}$ with rows $\iota_{\phi_5(1)}=\iota_3$, 
$\iota_{\phi_5(2)}=\iota_1$, and $\iota_{\phi_5(3)}=\iota_2$.\begin{equation*}
  % P_{\phi_2}
  %  =\begin{mat}
  %     \iota_1 \\
  %     \iota_3 \\
  %     \iota_2 
  %  \end{mat}
  % =\begin{mat}[r]
  %    1      &0        &0        \\
  %    0      &0        &1        \\
  %    0      &1        &0
  % \end{mat}
  % \qquad
  P_{\phi_5}
   =\begin{mat}
      \iota_3 \\
      \iota_1 \\
      \iota_2 
   \end{mat}
  =\begin{mat}[r]
     0      &0        &1        \\
     1      &0        &0        \\
     0      &1        &0
  \end{mat}
\end{equation*}
\end{example}

\begin{definition} \label{df:PermutationExpansion}
%<*df:PermutationExpansion>
The \definend{permutation expansion}\index{determinant!permutation expansion}%
\index{permutation expansion}
for determinants is
\begin{equation*}
   \begin{vmat}
      t_{1,1}  &t_{1,2}  &\ldots  &t_{1,n}  \\
      t_{2,1}  &t_{2,2}  &\ldots  &t_{2,n}  \\
               &\vdots                      \\
      t_{n,1}  &t_{n,2}  &\ldots  &t_{n,n}
   \end{vmat}
   =
   \begin{array}[t]{l}
      t_{1,\phi_1(1)}t_{2,\phi_1(2)}\cdots
           t_{n,\phi_1(n)}\deter{P_{\phi_1}}       \\[.5ex]
      \>\hbox{}+t_{1,\phi_2(1)}t_{2,\phi_2(2)}\cdots
           t_{n,\phi_2(n)}\deter{P_{\phi_2}}       \\
      \>\vdotswithin{+}                              \\
      \>\hbox{}+t_{1,\phi_k(1)}t_{2,\phi_k(2)}\cdots
           t_{n,\phi_k(n)}\deter{P_{\phi_k}} 
   \end{array}
\end{equation*}
where \( \phi_1,\ldots,\phi_k \) are all of the \( n \)-permutations.
%</df:PermutationExpansion>
\end{definition}
%<*SummationForPermutationExpansion>
This formula is often written in 
\definend{summation 
notation}\index{summation notation!for permutation expansion}
\begin{equation*}
  \deter{T}=\!\!
  \sum_{\text{permutations\ }\phi}\!\!\!\!
     t_{1,\phi(1)}t_{2,\phi(2)}\cdots t_{n,\phi(n)}
                                 \deter{P_{\phi}}
\end{equation*}
read aloud as,
``the sum, over all permutations \( \phi \), of terms having the form
\( t_{1,\phi(1)}t_{2,\phi(2)}\cdots t_{n,\phi(n)} \deter{P_{\phi}} \).''
%</SummationForPermutationExpansion>


\begin{example}
The familiar $\nbyn{2}$ determinant formula follows from the above.
\begin{align*}
  \begin{vmat}
    t_{1,1}  &t_{1,2} \\
    t_{2,1}  &t_{2,2}
  \end{vmat}
  &=
  t_{1,1}t_{2,2}\cdot\deter{P_{\phi_1}}
  +
  t_{1,2}t_{2,1}\cdot\deter{P_{\phi_2}}      \\     
  &=
  t_{1,1}t_{2,2}\cdot\begin{vmat}[r]
           1  &0 \\
           0  &1
         \end{vmat}
  +
  t_{1,2}t_{2,1}\cdot\begin{vmat}[r]
            0  &1 \\
            1  &0
          \end{vmat}               \\
  &=t_{1,1}t_{2,2}-t_{1,2}t_{2,1}
\end{align*}
So does the $\nbyn{3}$ determinant formula.
\begin{align*}
  \begin{vmat}
    t_{1,1}  &t_{1,2}  &t_{1,3} \\
    t_{2,1}  &t_{2,2}  &t_{2,3} \\
    t_{3,1}  &t_{3,2}  &t_{3,3} 
  \end{vmat}
  &=
  \begin{aligned}[t]
    &t_{1,1}t_{2,2}t_{3,3}\deter{P_{\phi_1}}
     +t_{1,1}t_{2,3}t_{3,2}\deter{P_{\phi_2}}
     +t_{1,2}t_{2,1}t_{3,3}\deter{P_{\phi_3}} \\
    &\hspace*{0em}
     +t_{1,2}t_{2,3}t_{3,1}\deter{P_{\phi_4}}
     +t_{1,3}t_{2,1}t_{3,2}\deter{P_{\phi_5}}
     +t_{1,3}t_{2,2}t_{3,1}\deter{P_{\phi_6}}
  \end{aligned}                                      \\
  &=
  \begin{aligned}[t]
    &t_{1,1}t_{2,2}t_{3,3}
     -t_{1,1}t_{2,3}t_{3,2}
     -t_{1,2}t_{2,1}t_{3,3}  \\
    &\hspace*{0em}
     +t_{1,2}t_{2,3}t_{3,1}
     +t_{1,3}t_{2,1}t_{3,2}
     -t_{1,3}t_{2,2}t_{3,1}
  \end{aligned}
\end{align*}
\end{example}

Computing a determinant by permutation expansion usually takes longer than
Gauss' method.
However, 
while it is not often used in practice, 
we use it for the theory, to prove that 
the determinant function is well-defined.

We will just state the result here and 
defer its proof to the following subsection.


\begin{theorem} \label{th:DetsExist}
\index{determinant!exists} 
%<*th:DetsExist>
For each $n$ there is an $\nbyn{n}$ determinant function.
%</th:DetsExist>
\end{theorem}

Also in the next subsection is the proof of this
result (these two proofs share some features).

\begin{theorem}  \label{th:DeterminantOfAMatrixEqualsDeterminantOfTranspose}
\index{transpose!determinant}
%<*th:DeterminantOfAMatrixEqualsDeterminantOfTranspose>
The determinant of a matrix equals the determinant of its transpose.
%</th:DeterminantOfAMatrixEqualsDeterminantOfTranspose>
\end{theorem}

Because of this theorem,
while we have so far stated determinant results in terms of rows
(e.g., determinants are multilinear in their rows, row swaps change the
sign, etc.),
all of the results also hold in terms of columns.

\begin{corollary} \label{cor:ColSwapChgSign} \label{cor:DetsMultiInCols}
%<*co:ColSwapChgSign>
  A matrix with two equal columns is singular.
  Column swaps change the sign of a determinant.
  Determinants are multilinear in their columns.
%</co:ColSwapChgSign>
\end{corollary}

\begin{proof}
%<*pf:ColSwapChgSign>
For the first statement, 
transposing the matrix results in a matrix with the same determinant,
and with two equal rows, and hence a determinant of zero.
Prove the other two in the same way.
%</pf:ColSwapChgSign>
\end{proof}

We finish this subsection with a summary:
determinant functions exist, are unique, and we know how to compute them.
As for what determinants are about, perhaps these lines
\cite{Kemp} help make it memorable.
\begin{verse} \small
Determinant none,         \\*
\ Solution: lots or none.   \\*
Determinant some,         \\*
\ Solution: just one.
\end{verse}




\begin{exercises}
  \item[]\textit{This summarizes the notation that we use 
    for the $2$-\hbox{} and $3$-\hbox{} permutations.
    \begin{center}
      \begin{tabular}[t]{c|cc}
        $i$          &$1$      &$2$    \\
        \hline
        $\phi_1(i)$  &$1$      &$2$     \\
        $\phi_2(i)$  &$2$      &$1$     
      \end{tabular}
      \qquad
      \begin{tabular}[t]{c|ccc}
        $i$          &$1$     &$2$   &$3$    \\
        \hline
        $\phi_1(i)$  &$1$     &$2$   &$3$    \\
        $\phi_2(i)$  &$1$     &$3$   &$2$    \\
        $\phi_3(i)$  &$2$     &$1$   &$3$    \\
        $\phi_4(i)$  &$2$     &$3$   &$1$    \\
        $\phi_5(i)$  &$3$     &$1$   &$2$    \\
        $\phi_6(i)$  &$3$     &$2$   &$1$    
      \end{tabular}
    \end{center}  }
  \recommended \item 
    Compute the determinant by using the permutation expansion.
    \begin{exparts*}
      \partsitem 
        $\begin{vmat}[r]
          1  &2  &3  \\
          4  &5  &6  \\
          7  &8  &9
        \end{vmat}$
      \partsitem 
        $\begin{vmat}[r]
          2  &2  &1  \\
          3  &-1 &0  \\
          -2 &0  &5
        \end{vmat}$
    \end{exparts*}
    \begin{answer}
      \begin{exparts}
        \partsitem This matrix is singular. 
          \begin{align*}
            \begin{vmat}[r]
              1  &2  &3  \\
              4  &5  &6  \\
              7  &8  &9
            \end{vmat}
            &=
            \begin{aligned}[t]
              &(1)(5)(9)\deter{P_{\phi_1}}
              +(1)(6)(8)\deter{P_{\phi_2}}
              +(2)(4)(9)\deter{P_{\phi_3}}  \\
              &\hbox{}\quad\hbox{}
              +(2)(6)(7)\deter{P_{\phi_4}}
              +(3)(4)(8)\deter{P_{\phi_5}}
              +(7)(5)(3)\deter{P_{\phi_6}}
            \end{aligned}                        \\
            &=0
          \end{align*}
        \partsitem This matrix is nonsingular. 
          \begin{align*}
            \begin{vmat}[r]
              2  &2  &1  \\
              3  &-1 &0  \\
              -2 &0  &5
            \end{vmat}
            &=
            \begin{aligned}[t]
              &(2)(-1)(5)\deter{P_{\phi_1}}
              +(2)(0)(0)\deter{P_{\phi_2}}
              +(2)(3)(5)\deter{P_{\phi_3}} \\
              &\hbox{}\quad\hbox{} 
              +(2)(0)(-2)\deter{P_{\phi_4}}
              +(1)(3)(0)\deter{P_{\phi_5}}
              +(-2)(-1)(1)\deter{P_{\phi_6}}
            \end{aligned}                      \\
            &=-42
        \end{align*}
      \end{exparts}
    \end{answer}
  \recommended \item
    Compute these both with Gauss' method and the permutation
    expansion formula.
    \begin{exparts*}
      \partsitem \( \begin{vmat}[r]
                 2  &1  \\
                 3  &1
               \end{vmat}   \)
      \partsitem \( \begin{vmat}[r]
                  0  &1  &4  \\
                  0  &2  &3  \\
                  1  &5  &1
               \end{vmat}  \)
    \end{exparts*}
    \begin{answer}
      \begin{exparts}
        \partsitem Gauss' method gives this
          \begin{equation*}
             \begin{vmat}[r]
                 2  &1  \\
                 3  &1
             \end{vmat}
             =
             \begin{vmat}[r]
                 2  &1  \\
                 0  &-1/2
             \end{vmat}
             =-1
          \end{equation*}
          and permutation expansion gives this.
          \begin{equation*}
             \begin{vmat}[r]
                 2  &1  \\
                 3  &1
             \end{vmat}
             =
             \begin{vmat}[r]
                 2  &0  \\
                 0  &1
             \end{vmat} +
             \begin{vmat}[r]
                 0  &1  \\
                 3  &0
             \end{vmat}             
             =
             (2)(1)\begin{vmat}[r]
                 1  &0  \\
                 0  &1
             \end{vmat} +
             (1)(3)\begin{vmat}[r]
                 0  &1  \\
                 1  &0
             \end{vmat}             
             =
             -1
          \end{equation*}
        \partsitem Gauss' method gives this
          \begin{equation*}
            \begin{vmat}[r]
              0  &1  &4  \\
              0  &2  &3  \\
              1  &5  &1
            \end{vmat}
            =
            -\begin{vmat}[r]
              1  &5  &1  \\
              0  &2  &3  \\
              0  &1  &4  
            \end{vmat}
            =
            -\begin{vmat}[r]
              1  &5  &1  \\
              0  &2  &3  \\
              0  &0  &5/2  
            \end{vmat}
            =-5
          \end{equation*}
          and the permutation expansion gives this.
          \begin{align*}
            \begin{vmat}[r]
              0  &1  &4  \\
              0  &2  &3  \\
              1  &5  &1
            \end{vmat}
            &=
            \begin{aligned}[t]
            &(0)(2)(1)\deter{P_{\phi_1}}
            +(0)(3)(5)\deter{P_{\phi_2}}
            +(1)(0)(1)\deter{P_{\phi_3}}  \\
            &\hbox{}\quad\hbox{}
            +(1)(3)(1)\deter{P_{\phi_4}}
            +(4)(0)(5)\deter{P_{\phi_5}}
            +(1)(2)(0)\deter{P_{\phi_6}}
            \end{aligned}                         \\
            &=-5
          \end{align*}
      \end{exparts}  
    \end{answer}
  \recommended \item 
    Use the permutation expansion formula to derive
    the formula for \( \nbyn{3} \) determinants.
    \begin{answer}
        Following \nearbyexample{ex:SamplePermExp} gives this.
          \begin{align*}
             \begin{vmat}
                t_{1,1}  &t_{1,2}  &t_{1,3}  \\
                t_{2,1}  &t_{2,2}  &t_{2,3}  \\
                t_{3,1}  &t_{3,2}  &t_{3,3}
             \end{vmat}
             &=\begin{aligned}[t]
                 &t_{1,1}t_{2,2}t_{3,3}\deter{P_{\phi_1}}
                   +t_{1,1}t_{2,3}t_{3,2}\deter{P_{\phi_2}} \\
                 &\hbox{}\quad\hbox{}
                  +t_{1,2}t_{2,1}t_{3,3}\deter{P_{\phi_3}}
                             +t_{1,2}t_{2,3}t_{3,1}\deter{P_{\phi_4}} \\
                 &\hbox{}\quad\hbox{}
                  +t_{1,3}t_{2,1}t_{3,2}\deter{P_{\phi_5}}
                             +t_{1,3}t_{2,2}t_{3,1}\deter{P_{\phi_6}}
               \end{aligned}                                               \\
             &=\begin{aligned}[t]
                 & t_{1,1}t_{2,2}t_{3,3}(+1)
                   +t_{1,1}t_{2,3}t_{3,2}(-1)  \\
                 &\hbox{}\quad\hbox{}
                   +t_{1,2}t_{2,1}t_{3,3}(-1)
                    +t_{1,2}t_{2,3}t_{3,1}(+1) \\
                 &\hbox{}\quad\hbox{}
                   +t_{1,3}t_{2,1}t_{3,2}(+1)
                         +t_{1,3}t_{2,2}t_{3,1}(-1)
               \end{aligned}                                               
          \end{align*}  
     \end{answer}
  \item 
    List all of the $4$-permutations.
    \begin{answer}
      This is all of the permutations where $\phi(1)=1$
      \begin{align*}
        &\phi_1=\sequence{1,2,3,4}
        \quad
        \phi_2=\sequence{1,2,4,3}
        \quad
        \phi_3=\sequence{1,3,2,4}  \\
        &\quad
        \phi_4=\sequence{1,3,4,2}
        \quad
        \phi_5=\sequence{1,4,2,3}
        \quad
        \phi_6=\sequence{1,4,3,2}
      \end{align*}
      the ones where $\phi(1)=1$
      \begin{align*}
        &\phi_7=\sequence{2,1,3,4}
        \quad
        \phi_8=\sequence{2,1,4,3}
        \quad
        \phi_9=\sequence{2,3,1,4}    \\
        &\quad
        \phi_{10}=\sequence{2,3,4,1}
        \quad
        \phi_{11}=\sequence{2,4,1,3}
        \quad
        \phi_{12}=\sequence{2,4,3,1}
      \end{align*}
      the ones where $\phi(1)=3$
      \begin{align*}
        &\phi_{13}=\sequence{3,1,2,4}
        \quad
        \phi_{14}=\sequence{3,1,4,2}
        \quad
        \phi_{15}=\sequence{3,2,1,4}  \\
        &\quad
        \phi_{16}=\sequence{3,2,4,1}
        \quad
        \phi_{17}=\sequence{3,4,1,2}
        \quad
        \phi_{18}=\sequence{3,4,2,1}
      \end{align*}
      and the ones where $\phi(1)=4$.
      \begin{align*}
        &\phi_{19}=\sequence{4,1,2,3}
        \quad
        \phi_{20}=\sequence{4,1,3,2}
        \quad
        \phi_{21}=\sequence{4,2,1,3}  \\
        &\quad
        \phi_{22}=\sequence{4,2,3,1}
        \quad
        \phi_{23}=\sequence{4,3,1,2}
        \quad
        \phi_{24}=\sequence{4,3,2,1}
      \end{align*}
    \end{answer}
  \item 
    A permutation, regarded as a function from the set
    $\set{1,..,n}$ to itself, is one-to-one and onto.
    Therefore, each permutation has an inverse. 
    \begin{exparts}
      \partsitem Find the inverse of each $2$-permutation.
      \partsitem Find the inverse of each $3$-permutation.
    \end{exparts}
    \begin{answer}
      Each of these is easy to check.
      \begin{exparts*}
        \partsitem 
          \begin{tabular}[t]{r|cc}
            \textit{permutation} &$\phi_1$  &$\phi_2$ \\
             \hline
            \textit{inverse}     &$\phi_1$  &$\phi_2$ 
          \end{tabular}
        \partsitem 
          \begin{tabular}[t]{r|cccccc}
            \textit{permutation} 
              &$\phi_1$ &$\phi_2$ &$\phi_3$ &$\phi_4$ &$\phi_5$ &$\phi_6$ \\
            \hline
            \textit{inverse}
              &$\phi_1$ &$\phi_2$ &$\phi_3$ &$\phi_5$ &$\phi_4$ &$\phi_6$ 
          \end{tabular}
      \end{exparts*}
    \end{answer}
  \item 
     Prove that \( f \) is multilinear if and only if for all
     \( \vec{v},\vec{w}\in V \) and \( k_1,k_2\in\Re \), this holds.
      \begin{equation*}
         f(\vec{\rho}_1,\dots,k_1\vec{v}_1+k_2\vec{v}_2,
           \dots,\vec{\rho}_n)
         =
         k_1f(\vec{\rho}_1,\dots,\vec{v}_1,\dots,\vec{\rho}_n)+
         k_2f(\vec{\rho}_1,\dots,\vec{v}_2,\dots,\vec{\rho}_n)
       \end{equation*}
       \begin{answer}
         For the `if' half, the first condition of 
         \nearbydefinition{def:multilinear} follows from taking $k_1=k_2=1$
         and the second condition follows from taking $k_2=0$.

         The `only if' half also routine.
         From
         $
           f(\vec{\rho}_1,\dots,k_1\vec{v}_1+k_2\vec{v}_2,
             \dots,\vec{\rho}_n)
         $
         the first condition of \nearbydefinition{def:multilinear} gives
         $
           =
           f(\vec{\rho}_1,\dots,k_1\vec{v}_1,\dots,\vec{\rho}_n)+
           f(\vec{\rho}_1,\dots,k_2\vec{v}_2,\dots,\vec{\rho}_n)
         $
         and the second condition, applied twice, gives the 
         result. 
       \end{answer}
  \item 
    How would determinants change if we changed property~(4) of the
    definition to read that \( \deter{I}=2 \)?
    \begin{answer}
       They would all double.
    \end{answer}
  \item 
    Verify the second and third 
    statements in \nearbycorollary{cor:ColSwapChgSign}.
    \begin{answer}
      For the second statement, 
      given a matrix, transpose it, swap rows, and transpose back.
      The result is swapped columns, and the determinant changes by a factor
      of \( -1 \).
      The third statement is similar:~given 
      a matrix, transpose it, apply multilinearity to what are now
      rows, and then transpose back the resulting matrices.
    \end{answer}
  \recommended \item
    Show that if an \( \nbyn{n} \) matrix has a nonzero determinant
    then we can express any column vector
    \( \vec{v}\in\Re^n \) as a linear combination
    of the columns of the matrix.
    \begin{answer}
      An \( \nbyn{n} \) matrix with a nonzero determinant has rank
      \( n \) so its columns form a basis for \( \Re^n \).  
    \end{answer}
  \item 
    \cite{Strang}
    True or false: a matrix whose entries are only zeros
    or ones has a determinant equal to zero, one, or negative one.
    \begin{answer}
      False.
      \begin{equation*}
        \begin{vmat}[r]
          1  &-1  \\
          1  &1
        \end{vmat}
        =2
      \end{equation*} 
     \end{answer}
  \item 
    \begin{exparts}
      \partsitem Show that there are $120$ terms in the permutation
         expansion formula of a \( \nbyn{5} \) matrix.
      \partsitem 
         How many are sure to be zero if the \( 1,2 \) entry is zero?
    \end{exparts}
    \begin{answer}
      \begin{exparts}
        \partsitem For the column index of the entry in the first row there are
          five choices.
          Then, for the column index of the entry in the second row there
          are four choices.
          Continuing, we get $5\cdot 4\cdot 3\cdot 2\cdot 1=120$.
          (See also the next question.)
        \partsitem Once we choose the second column in the first row, 
          we can choose the other entries in \( 4\cdot 3\cdot 2\cdot 1=24 \) 
          ways.
      \end{exparts}  
    \end{answer}
  \item 
    How many \( n \)-permutations are there?
    \begin{answer}
       \( n\cdot(n-1)\cdots 2\cdot 1=n! \) 
    \end{answer}
  \item Show that the inverse of a permutation matrix is its transpose.
    \begin{answer}
      \cite{SchmidtSO}
      We will show that $P\trans{P}=I$; the $\trans{P}P=I$ argument is similar.
      The $i,j$ entry of $P\trans{P}$ is the sum of terms of the form
      $p_{i,k}q_{k,j}$ where the entries of $\trans{P}$ are denoted with $q$'s,
      that is, $q_{k,j}=p_{j,k}$.
      Thus the $i,j$ entry of $P\trans{P}$ is the sum
      \(
        \sum_{k=1}^n p_{i,k}p_{j,k}
      \).
      But $p_{i,k}$ is usually $0$, and so $P_{i,k}P_{j,k}$ is usually $0$. 
      The only time $P_{i,k}$ is nonzero is when it is $1$, 
      but then there are no other $i^\prime\neq i$ such that $P_{i^\prime,k}$ 
      is nonzero ($i$ is the only row with a $1$ in column~$k$). 
      In other words,
      \begin{equation*}
        \sum_{k=1}^n p_{i,k}p_{j,k}=
        \begin{cases}
          1  &i=j  \\
          0  &\text{otherwise}
        \end{cases}
      \end{equation*}
      and this is exactly the formula for the entries of the identity matrix.
    \end{answer}
  \item 
    A matrix \( A \) is 
    \definend{skew-symmetric}\index{matrix!skew-symmetric}%
    \index{skew-symmetric} if \( \trans{A}=-A \),
    as in this matrix.
    \begin{equation*}
      A=\begin{mat}[r]
          0  &3  \\
         -3  &0
        \end{mat}
    \end{equation*}
    Show that \( \nbyn{n} \) skew-symmetric matrices with nonzero
    determinants exist only for even \( n \).
    \begin{answer}
      In 
      \( \deter{A}=\deter{\trans{A}}=\deter{-A}=(-1)^n\deter{A} \)
      the exponent $n$ must be even. 
    \end{answer}
  \recommended \item
    What is the smallest number of zeros, and the placement of
    those zeros, needed to ensure that a \( \nbyn{4} \) matrix has a
    determinant of zero?
    \begin{answer}
      Showing that no placement of three zeros suffices is routine.
      Four zeroes does suffice; put them all in the same  
      row or column.
    \end{answer}
  \recommended \item
    If we have \( n \) data points
    \( (x_1,y_1),(x_2,y_2),\dots\,,(x_n,y_n) \) and want to find a
    polynomial \( p(x)=a_{n-1}x^{n-1}+a_{n-2}x^{n-2}+\dots+a_1x+a_0 \)
    passing through those points
    then we can plug in the points to get an \( n \)~equation/\( n \)~unknown
    linear system.
    The matrix of coefficients for that system is the
    \definend{Vandermonde matrix}.\index{matrix!Vandermonde}%
    \index{Vandermonde matrix}
    Prove that the determinant of the transpose of that matrix of coefficients
    \begin{equation*}
      \begin{vmat}
         1       &1       &\ldots   &1       \\
         x_1     &x_2     &\ldots   &x_n     \\
         {x_1}^2 &{x_2}^2 &\ldots   &{x_n}^2 \\
                 &\vdots                     \\
         {x_1}^{n-1} &{x_2}^{n-1}   &\ldots   &{x_n}^{n-1}
      \end{vmat}
    \end{equation*}
    equals the product, over all indices \( i,j\in\set{1,\dots,n} \) with
    \( i<j \), of terms of the form \( x_j-x_i \).
    (This shows that 
    the determinant is zero, and the linear system has no solution,
    if and only if the \( x_i \)'s in the data are not distinct.)
    \begin{answer}
      The $n=3$ case shows what to do.
      The row combination operations of 
      $-x_1\rho_2+\rho_3$ and $-x_1\rho_1+\rho_2$
      give this.
      \begin{equation*}
        \begin{vmat}
          1     &1     &1     \\
          x_1   &x_2   &x_3   \\
          x_1^2 &x_2^2 &x_3^2 
        \end{vmat}
        =
        \begin{vmat}
          1     &1             &1             \\
          x_1   &x_2           &x_3           \\
          0     &(-x_1+x_2)x_2 &(-x_1+x_3)x_3 
        \end{vmat}
        =
        \begin{vmat}
          1     &1             &1             \\
          0     &-x_1+x_2      &-x_1+x_3      \\
          0     &(-x_1+x_2)x_2 &(-x_1+x_3)x_3 
        \end{vmat}
      \end{equation*}
      Then the row combination operation of $x_2\rho_2+\rho_3$ gives
      the desired result.
      \begin{equation*}
        =
        \begin{vmat}
          1     &1             &1                     \\
          0     &-x_1+x_2      &-x_1+x_3              \\
          0     &0             &(-x_1+x_3)(-x_2+x_3) 
        \end{vmat}
        =(x_2-x_1)(x_3-x_1)(x_3-x_2)
      \end{equation*}
    \end{answer}
  \item 
    We can divide a matrix into  
    \definend{blocks},\index{block matrix}\index{matrix!block} %
    as here,
    \begin{equation*} 
      \begin{pmat}{rr|r}
          1  &2   &0  \\
          3  &4   &0  \\  \cline{1-3}
          0  &0   &-2 
       \end{pmat}
    \end{equation*}
    which shows four blocks, the square $\nbyn{2}$ and $\nbyn{1}$ ones
    in the upper left and lower right, and the zero blocks in the
    upper right and lower left.
    Show that if a matrix is such that we can partition it as
    \begin{equation*}
      T=
      \begin{pmat}{c|c}
          J   &Z_2  \\  \cline{1-2}
          Z_1 &K
       \end{pmat}
    \end{equation*}
    where $J$ and $K$ are square, and $Z_1$ and $Z_2$ are all zeroes,
    then \( \deter{T}=\deter{J}\cdot\deter{K} \).
    \begin{answer}
      Let \( T \) be \( \nbyn{n} \),
      let \( J \) be \( \nbyn{p} \),
      and let \( K \) be \( \nbyn{q} \).
      Apply the permutation expansion formula
      \begin{equation*}
        \deter{T}=
        \sum_{\text{\scriptsize permutations }\phi}
                t_{1,\phi(1)}t_{2,\phi(2)}\dots t_{n,\phi(n)}
                \deter{P_{\phi}}
      \end{equation*}
      Because the upper right of \( T \) is all zeroes, if a
      \( \phi \) has at least one of \( p+1,\dots,n \) among its first
      \( p \) column numbers \( \phi(1),\dots,\phi(p) \) then the term arising
      from \( \phi \) is \( 0 \)
      (e.g., if \( \phi(1)=n \) then
      \( t_{1,\phi(1)}t_{2,\phi(2)}\dots t_{n,\phi(n)} \)
      is \( 0 \)).
      So the above formula reduces to a sum over all permutations
      with two halves:
      first rearrange \( 1,\dots,p \) and after that comes
      a permutation of
      \( p+1,\dots,p+q \).
      To see this gives \( \deter{J}\cdot\deter{K}  \), distribute.
      \begin{equation*}
         \bigg[\sum_{\substack{\text{\scriptsize perms }\phi_1 \\
                              \text{\scriptsize of } 1,\dots,p}}
               t_{1,\phi_1(1)}\cdots t_{p,\phi_1(p)} 
               \deter{P_{\phi_1}}                            \bigg]\cdot
         \bigg[\sum_{\substack{\text{\scriptsize perms }\phi_2 \\
                              \text{\scriptsize of } p+1,\dots,p+q}}
               t_{p+1,\phi_2(p+1)}\cdots t_{p+q,\phi_2(p+q)} 
               \deter{P_{\phi_2}}                          \bigg]
      \end{equation*}  
    \end{answer}
  \recommended \item
    Prove that for any \( \nbyn{n} \) matrix \( T \) there are at most
    \( n \) distinct reals \( r \) such that the matrix \( T-rI \) has
    determinant zero
    (we shall use this result in Chapter Five).
    \begin{answer}
      The $n=3$ case shows what happens.
      \begin{equation*}
        \deter{T-rI}
        =
        \begin{vmat}
          t_{1,1}-x  &t_{1,2}   &t_{1,3}  \\ 
          t_{2,1}    &t_{2,2}-x &t_{2,3}  \\ 
          t_{3,1}    &t_{3,2}   &t_{3,3}-x  
        \end{vmat}
      \end{equation*}
      Each term in the permutation expansion has three factors drawn from 
      entries in the matrix (e.g., $(t_{1,1}-x)(t_{2,2}-x)(t_{3,3}-x)$
      and $(t_{1,1}-x)(t_{2,3})(t_{3,2})$), and so the determinant is
      expressible as a polynomial in $x$ of degree $3$.
      Such a polynomial has at most $3$ roots.

      In general, the permutation expansion shows that
      the determinant is a sum of terms, each
      with \( n \) factors, giving a polynomial of degree $n$.
      A polynomial of degree \( n \) has at most \( n \) roots. 
    \end{answer}
  \puzzle \item 
    \cite{MathMag63Q307}
    The nine positive digits can be arranged into \( \nbyn{3} \) arrays
    in \( 9! \) ways.
    Find the sum of the determinants of these arrays.
    \begin{answer}
      \answerasgiven
      When two rows of a determinant are interchanged, the sign of the
      determinant is changed.
      When the rows of a three-by-three determinant are permuted, \( 3 \)
      positive and \( 3 \) negative determinants equal in absolute value
      are obtained.
      Hence the \( 9! \) determinants fall into \( 9!/6 \) groups, each of
      which sums to zero.  
    \end{answer}
  \item 
    \cite{MathMag63Q237}
    Show that
    \begin{equation*}
      \begin{vmat}
        x-2  &x-3  &x-4  \\
        x+1  &x-1  &x-3  \\
        x-4  &x-7  &x-10
      \end{vmat}=0.
    \end{equation*}
    \begin{answer}
      \answerasgiven
      When the elements of any column are subtracted from the elements of
      each of the other two, the elements in two of the columns of the derived
      determinant are proportional, so the determinant vanishes.
      That is,
      \begin{equation*}
        \begin{vmat}
          2  &1    &x-4  \\
          4  &2    &x-3  \\
          6  &3    &x-10
        \end{vmat}=
        \begin{vmat}
          1    &x-3  &-1   \\
          2    &x-1  &-2   \\
          3    &x-7  &-3
        \end{vmat}=
        \begin{vmat}
          x-2  &-1   &-2   \\
          x+1  &-2   &-4   \\
          x-4  &-3   &-6
        \end{vmat}=0.
      \end{equation*}  
    \end{answer}
  \puzzle \item 
    \cite{Monthly49p33}
    Let \( S \) be the sum of the integer elements of a magic square of  order
    three and let \( D \) be the value of the square considered as a
    determinant.
    Show that \( D/S \) is an integer.
    \begin{answer}
      \answerasgiven
      Let
      \begin{equation*}
        \begin{array}{ccc}
          a  &b  &c  \\
          d  &e  &f  \\
          g  &h  &i
        \end{array}
      \end{equation*}
      have magic sum \( N=S/3 \).
      Then
      \begin{align*}
        N
        &=(a+e+i)+(d+e+f)+(g+e+c)   \\
        &\quad\text{}-(a+d+g)-(c+f+i)=3e
      \end{align*}
      and \( S=9e \).
      Hence, adding rows and columns,
      \begin{equation*}
        D=
        \begin{vmat}
          a  &b  &c  \\
          d  &e  &f  \\
          g  &h  &i
        \end{vmat}
        =\begin{vmat}
          a  &b  &c  \\
          d  &e  &f  \\
         3e  &3e &3e
        \end{vmat}             
        =\begin{vmat}
          a  &b  &3e \\
          d  &e  &3e \\
         3e  &3e &9e
        \end{vmat}             
        =\begin{vmat}
          a  &b  &e  \\
          d  &e  &e  \\
          1  &1  &1
        \end{vmat}S.
      \end{equation*}  
    \end{answer}
  \puzzle \item 
    \cite{Monthly31p355}
    Show that the determinant of the \( n^2 \) elements in the upper left
    corner of the Pascal triangle
    \begin{equation*}
      \begin{array}{cccccc}
        1  &1  &1  &1  &.  &.  \\
        1  &2  &3  &.  &.      \\
        1  &3  &.  &.  &   &   \\
        1  &.  &.  &   &   &   \\
        .                      \\
        .
      \end{array}
    \end{equation*}
    has the value unity.
    \begin{answer}
      \answerasgiven
      Denote by \( D_n \) the determinant in question and by \( a_{i,j} \)
      the element in the \( i \)-th row and \( j \)-th column.
      Then from the law of formation of the elements we have
      \begin{equation*}
        a_{i,j}=a_{i,j-1}+a_{i-1,j},
        \qquad a_{1,j}=a_{i,1}=1.
      \end{equation*}
      Subtract each row of \( D_n \) from the row following it, beginning the
      process with the last pair of rows.
      After the \( n-1 \) subtractions the above equality shows that
      the element
      \( a_{i,j} \) is replaced by the element \( a_{i,j-1} \), and all the
      elements in the first column, except \( a_{1,1}=1 \), become zeroes.
      Now subtract each column from the one following it, beginning with the
      last pair.
      After this process the element \( a_{i,j-1} \) is replaced by
      \( a_{i-1,j-1} \), as shown in the above relation.
      The result of the two operations is to replace \( a_{i,j} \) by
      \( a_{i-1,j-1} \), and to reduce each element in the first row and in
      the first column to zero.
      Hence \( D_n=D_{n+i} \) and consequently
      \begin{equation*}
        D_n=D_{n-1}=D_{n-2}=\dots=D_2=1.
      \end{equation*}  
    \end{answer}
\end{exercises}














\subsectionoptional{Determinants Exist}
\textit{This subsection is optional.
It proves two results 
from the prior subsection.
These proofs involve the properties of
permutations, which will use again only in the
optional Jordan Canonical Form subsection.}

The prior subsection develops the 
permutation expansion formula for 
determinants.\index{determinant!permutation expansion}%
\index{permutation expansion}
\begin{align*}
   %\deter{T}
   %=
   \begin{vmat}
      t_{1,1}  &t_{1,2}  &\ldots  &t_{1,n}  \\
      t_{2,1}  &t_{2,2}  &\ldots  &t_{2,n}  \\
               &\vdots                      \\
      t_{n,1}  &t_{n,2}  &\ldots  &t_{n,n}
   \end{vmat}
   &=
   \begin{array}[t]{l}
      t_{1,\phi_1(1)}t_{2,\phi_1(2)}\cdots
           t_{n,\phi_1(n)}\deter{P_{\phi_1}}       \\[.75ex]
      \>\hbox{}+t_{1,\phi_2(1)}t_{2,\phi_2(2)}\cdots
           t_{n,\phi_2(n)}\deter{P_{\phi_2}}       \\[.75ex]
      \>\alignedvdots                              \\
      \>\hbox{}+t_{1,\phi_k(1)}t_{2,\phi_k(2)}\cdots
           t_{n,\phi_k(n)}\deter{P_{\phi_k}} 
   \end{array}                                                 \\[1ex]
   &=\!\!\sum_{\text{permutations\ }\phi}\!\!\!\!
     t_{1,\phi(1)}t_{2,\phi(2)}\cdots t_{n,\phi(n)}
                                 \deter{P_{\phi}}
\end{align*}
This reduces the problem of showing that for any size~$n$ the
determinant function on all $\nbyn{n}$ matrices is well-defined 
to only showing that the determinant is 
well-defined on the set of permutation matrices of that size.

A permutation matrix can be row-swapped to the identity matrix
and so we can 
calculate its determinant by keeping track of the number of swaps.
However,
we still must show that the result is well-defined.
Recall what the difficulty is:~the determinant of
\begin{equation*}
   P_{\phi}=
   \begin{mat}[r]
      0  &1  &0  &0 \\
      1  &0  &0  &0 \\
      0  &0  &1  &0 \\
      0  &0  &0  &1
   \end{mat}
\end{equation*}
could be computed with one swap
\begin{equation*}
   P_{\phi}
   \grstep{\rho_1\leftrightarrow\rho_2}
   \begin{mat}[r]
      1  &0  &0  &0 \\
      0  &1  &0  &0 \\
      0  &0  &1  &0 \\
      0  &0  &0  &1
   \end{mat}
\end{equation*}
or with three.
\begin{equation*}
   P_{\phi}
   \grstep{\rho_3\leftrightarrow\rho_1}
   \begin{mat}[r]
      0  &0  &1  &0 \\
      1  &0  &0  &0 \\
      0  &1  &0  &0 \\
      0  &0  &0  &1
   \end{mat}
   \grstep{\rho_2\leftrightarrow\rho_3}
   \begin{mat}[r]
      0  &0  &1  &0 \\
      0  &1  &0  &0 \\
      1  &0  &0  &0 \\
      0  &0  &0  &1
   \end{mat}
   \grstep{\rho_1\leftrightarrow\rho_3}
   \begin{mat}[r]
      1  &0  &0  &0 \\
      0  &1  &0  &0 \\
      0  &0  &1  &0 \\
      0  &0  &0  &1
   \end{mat}
\end{equation*}
Both reductions
have an odd number of swaps so we figure that \( \deter{P_{\phi}}=-1 \)
but if there were some way to do it with an even number of
swaps then we would have two different answers to one question.
Below, \nearbycorollary{cor:ParityInversEqParitySwaps} proves 
that this cannot happen\Dash
there is no permutation matrix
that can be row-swapped to an identity matrix in two ways, one with 
an even number of swaps and the other with an odd number of swaps.

So the critical step will be a way to calculate whether the 
number of swaps that it takes could be even or odd.

\begin{definition} \label{df:Inversion}
%<*df:Inversion>
Two rows of a permutation matrix
\begin{equation*}
  \begin{mat}
    \vdots          \\
    \iota_{k} \\
    \vdots          \\
    \iota_{j} \\
    \vdots
  \end{mat}
\end{equation*}
such that \( k>j \) are in an \definend{inversion}\index{inversion}%
\index{permutation!inversions}
of their natural order.
%</df:Inversion>
\end{definition}

\begin{example}
This permutation matrix
\begin{equation*}
  \begin{mat}
    \iota_3  \\
    \iota_2  \\
    \iota_1
  \end{mat}
  =
  \begin{mat}[r]
    0  &0  &1  \\
    0  &1  &0  \\
    1  &0  &0
  \end{mat}
\end{equation*}
has three inversions: \( \iota_3 \) precedes \( \iota_1 \),
\( \iota_3 \) precedes \( \iota_2 \), and \( \iota_2 \) precedes
\( \iota_1 \).
\end{example}

\begin{lemma} \label{le:SwapsChangeSgn}
%<*lm:SwapsChangeSgn>
A row-swap in a permutation matrix changes the number of 
inversions from even to odd, or from odd to even.
%</lm:SwapsChangeSgn>
\end{lemma}

\begin{proof}
%<*pf:SwapsChangeSgn0>
Consider a swap of rows $j$ and $k$, where $k>j$.
If the two rows are adjacent
\begin{equation*}
   P_{\phi}=
   \begin{mat}
     \vdots           \\
     \iota_{\phi(j)}  \\
     \iota_{\phi(k)}  \\
     \vdots
   \end{mat}
 \grstep{\rho_k\leftrightarrow\rho_j}
   \begin{mat}
     \vdots           \\
     \iota_{\phi(k)}  \\
     \iota_{\phi(j)}  \\
     \vdots
   \end{mat}
\end{equation*}
then since inversions involving rows not in this pair
are not affected,
the swap changes the total number of inversions by one, 
either removing or producing one inversion depending on whether
\( \phi(j)>\phi(k) \) or not.
Consequently, the total number of inversions changes from odd to even 
or from even to odd.
%</pf:SwapsChangeSgn0>

%<*pf:SwapsChangeSgn1>
If the rows are not adjacent then we can swap them 
via a sequence of adjacent swaps, first bringing row~$k$ up
\begin{equation*}
   \begin{mat}
     \vdots             \\
     \iota_{\phi(j)}    \\
     \iota_{\phi(j+1)}  \\
     \iota_{\phi(j+2)}  \\
     \vdots             \\
     \iota_{\phi(k)}    \\
     \vdots
   \end{mat}
  \grstep{\rho_k\swap\rho_{k-1}}\;\;
  \grstep{\rho_{k-1}\swap\rho_{k-2}}
  \cdots
  \grstep{\rho_{j+1}\swap\rho_j}
   \begin{mat}
     \vdots             \\
     \iota_{\phi(k)}    \\
     \iota_{\phi(j)}    \\
     \iota_{\phi(j+1)}  \\
     \vdots             \\
     \iota_{\phi(k-1)}  \\
     \vdots
   \end{mat}
\end{equation*}
%</pf:SwapsChangeSgn1>
%<*pf:SwapsChangeSgn2>
and then bringing row~$j$ down.
\begin{equation*}
  \grstep{\rho_{j+1}\swap\rho_{j+2}}\;\;
  \grstep{\rho_{j+2}\swap\rho_{j+3}}
  \cdots
  \grstep{\rho_{k-1}\swap\rho_k}
   \begin{mat}
     \vdots             \\
     \iota_{\phi(k)}    \\
     \iota_{\phi(j+1)}  \\
     \iota_{\phi(j+2)}  \\
     \vdots             \\
     \iota_{\phi(j)}    \\
     \vdots
   \end{mat}
\end{equation*}
Each of these adjacent swaps changes the number of 
inversions from odd to even or from even to odd.
There are an odd number \( (k-j)+(k-j-1) \) of them.
The total change in the number of inversions 
is from even to odd or from odd to even.
%</pf:SwapsChangeSgn2>
\end{proof}

\begin{corollary}  \label{cor:ParityInversEqParitySwaps}
%<*co:ParityInversEqParitySwaps>
If a permutation matrix has an odd number of inversions then swapping
it to the identity takes an odd number of swaps.
If it has an even number of inversions then swapping to the
identity takes an even number of swaps.
%</co:ParityInversEqParitySwaps>
\end{corollary}

\begin{proof}
%<*pf:ParityInversEqParitySwaps>
The identity matrix has zero inversions.
To change an odd number to zero requires an odd number of swaps,
and to change an even number to zero requires an even number of swaps.
%</pf:ParityInversEqParitySwaps>
\end{proof}

\begin{definition} \label{df:Signum}
%<*df:Signum>
The \definend{signum}\index{permutation!signum}\index{signum}%
\index{sgn!see{signum}}
of a permutation \( \sgn(\phi) \)
is \( -1 \) if the number of inversions in \( P_\phi \) is odd and is
\( +1 \) if the number of inversions is even.
%</df:Signum>
\end{definition}

\begin{example}
In the notation for the $3$-permutations
from \nearbyexample{ex:AllTwoThreePerms} we have
\begin{equation*}
  P_{\phi_1}=
  \begin{mat}
    1  &0  &0  \\
    0  &1  &0  \\
    0  &0  &1
  \end{mat}
  \quad\text{and}\quad
  P_{\phi_2}=
  \begin{mat}
    1  &0  &0  \\
    0  &0  &1  \\
    0  &1  &0  
  \end{mat}
\end{equation*}
so
\( \sgn(\phi_1)=1 \) because there are no inversions, 
while \( \sgn(\phi_2)=-1 \) because there is one.
\end{example}

%<*DefiningDFunction>
We still have not shown that the determinant function is
well-defined because we have not considered row operations 
on permutation matrices other than row swaps.
We will finesse this issue.
We will define a function
\( \map{d}{\matspace_{\nbyn{n}}}{\Re} \)
by altering the permutation expansion formula, replacing
$\deter{P_\phi}$ with $\sgn(\phi)$.
\begin{equation*}
  d(T)=
  \sum_{\text{permutations\ }\phi}t_{1,\phi(1)}t_{2,\phi(2)}\cdots t_{n,\phi(n)}
                                 \sgn(\phi)
\end{equation*}
This gives the same value as the permutation expansion
because the corollary shows that $\det(P_\phi)=\sgn(\phi)$.
The advantage of this formula is that the number of inversions 
is clearly well-defined\Dash  just count them.
Therefore, we will finish showing that an~$\nbyn{n}$ determinant function 
exists by showing  that this $d$
satisfies the conditions in the determinant's definition.
%</DefiningDFunction>

\begin{lemma} \label{lm:DeterminantsExist}
\index{determinant!exists}
%<*lm:DeterminantsExist>
The function \( d \) above is a determinant.
Hence determinants exist for every $n$.
%</lm:DeterminantsExist>
\end{lemma}

\begin{proof}
%<*pf:DeterminantsExist0>
We must check that it has the four properties
from the definition.
%</pf:DeterminantsExist0>

%<*pf:DeterminantsExist1>
Property~(4) is easy; where $I$ is the $\nbyn{n}$ identity, in
\begin{equation*}
  d(I)=
  \sum_{\text{perm\ }\phi}
    \iota_{1,\phi(1)}\iota_{2,\phi(2)}\cdots \iota_{n,\phi(n)}
                                 \sgn(\phi)
\end{equation*}
all of the terms in the summation are zero
except for the product down the diagonal,
which is one.
%</pf:DeterminantsExist1>

%<*pf:DeterminantsExist2>
For property~(3) consider $d(\hat{T})$ where 
\( T\smash[b]{\grstep{k\rho_i}}\hat{T} \).
\begin{multline*}
  \sum_{\text{perm\ }\phi}\!\!
    \hat{t}_{1,\phi(1)}
     \cdots\hat{t}_{i,\phi(i)}\cdots\hat{t}_{n,\phi(n)}
                                 \sgn(\phi)   \\
  =\sum_{\phi}
     t_{1,\phi(1)}\cdots kt_{i,\phi(i)}\cdots t_{n,\phi(n)}
                                 \sgn(\phi)
\end{multline*}
Factor out the~\( k \) to get the desired equality.
\begin{equation*}
  =k\cdot\sum_{\phi}
     t_{1,\phi(1)}\cdots t_{i,\phi(i)}\cdots t_{n,\phi(n)}
                                 \sgn(\phi)                 
  =k\cdot d(T)
\end{equation*}
%</pf:DeterminantsExist2>

%<*pf:DeterminantsExist3>
For~(2) suppose that 
\( T\;\smash[b]{\grstep{\rho_i\swap\rho_j}}\;\hat{T} \).
We must show this is the negative of $d(T)$.
\begin{equation*}
  d(\hat{T})=
  \sum_{\text{perm\ }\phi}\!\!
    \hat{t}_{1,\phi(1)}
    \cdots\hat{t}_{i,\phi(i)}
    \cdots\hat{t}_{j,\phi(j)}
    \cdots \hat{t}_{n,\phi(n)}
                                 \sgn(\phi)
    \tag{*}
\end{equation*}
We will show that each term in ($*$) is associated with a term in $d(t)$, 
and that the two terms are negatives of each other.
Consider the matrix from the multilinear expansion 
of $d(\hat{T})$ giving the term  
    $\hat{t}_{1,\phi(1)}
    \cdots\hat{t}_{i,\phi(i)}
    \cdots\hat{t}_{j,\phi(j)}
    \cdots \hat{t}_{n,\phi(n)}
                                 \sgn(\phi)$.
\begin{equation*}
  \begin{mat}
     \ &                  &\vdots &               &     \\
       &\hat{t}_{i,\phi(i)}  &       &               &    \\
       &                &\vdots &                 &  \\
       &                &       &\hat{t}_{j,\phi(j)} &  \\
       &                &\vdots                   &
  \end{mat}
\end{equation*}
%</pf:DeterminantsExist3>
%<*pf:DeterminantsExist4>
It is the result of the $\rho_i\swap\rho_j$ operation performed on this
matrix.
\begin{equation*}
  \begin{mat}
       \ &              &\vdots &           &        \\
         &              &       &t_{i,\phi(j)} &  \\
         &              &\vdots &           &       \\
         &t_{j,\phi(i)}    &       &          &        \\
         &              &\vdots &          &
  \end{mat}
\end{equation*}
That is, the term with hatted $t$'s is associated with 
this term from the $d(T)$ expansion:
 \(    t_{1,\sigma(1)}
       \cdots t_{j,\sigma(j)}
       \cdots t_{i,\sigma(i)}
       \cdots t_{n,\sigma(n)} \sgn(\sigma)
       \), where the permutation
\( \sigma \) 
equals \( \phi \) but with the
\( i \)-th and \( j \)-th numbers interchanged,
\( \sigma(i)=\phi(j) \) and \( \sigma(j)=\phi(i) \).
The two terms have the same multiplicands
$\hat{t}_{1,\phi(1)}=t_{1,\sigma(1)}$, \ldots, 
including the entries from the swapped rows
$\hat{t}_{i,\phi(i)}=t_{j,\phi(i)}=t_{j,\sigma(j)}$
and $\hat{t}_{j,\phi(j)}=t_{i,\phi(j)}=t_{i,\sigma(i)}$. 
But the two terms are negatives of each other
since \( \sgn(\phi)=-\sgn(\sigma) \) by \nearbylemma{le:SwapsChangeSgn}.
%</pf:DeterminantsExist4>

%<*pf:DeterminantsExist5>
Now, any permutation $\phi$ can be derived from some 
other permutation $\sigma$ by such a swap,
in one and only one way.
Therefore the summation in ($*$) is in fact a sum over all permutations,
taken once and only once.
\begin{align*}
  d(\hat{T})
  &=
  \sum_{\text{perm\ }\phi}\!\!
    \hat{t}_{1,\phi(1)}
    \cdots\hat{t}_{i,\phi(i)}
    \cdots\hat{t}_{j,\phi(j)}
    \cdots \hat{t}_{n,\phi(n)}
                                 \sgn(\phi)   \\
  &=\sum_{\text{perm\ }\sigma}\!\!
     t_{1,\sigma(1)}
    \cdots t_{j,\sigma(j)}
    \cdots t_{i,\sigma(i)}
    \cdots t_{n,\sigma(n)}
                                 \cdot\bigl(-\sgn(\sigma)\bigr)        
\end{align*}
Thus \( d(\hat{T})=-d(T) \).
%</pf:DeterminantsExist5>

%<*pf:DeterminantsExist6>
For property~(1) suppose that
\( T\;\smash[b]{\grstep{k\rho_i+\rho_j}}\;\hat{T} \)
and consider the effect of a row combination.
\begin{align*}
  d(\hat{T})
  &=\sum_{\text{perm\ }\phi}
    \hat{t}_{1,\phi(1)}\cdots\hat{t}_{i,\phi(i)}
    \cdots\hat{t}_{j,\phi(j)}\cdots\hat{t}_{n,\phi(n)}
                                 \sgn(\phi)                \\
  &=\sum_{\phi}
    t_{1,\phi(1)}\cdots t_{i,\phi(i)}
    \cdots (kt_{i,\phi(j)}+t_{j,\phi(j)})\cdots t_{n,\phi(n)}
                                 \sgn(\phi)
\end{align*}
% (notice:~that's 
% \( kt_{i,\phi(j)} \), not \( kt_{j,\phi(j)} \)).
Do the algebra.
\begin{align*}
  &= 
      \begin{aligned}[t]
        &\smash[b]{\sum_{\smash{\phi}}}\big[t_{1,\phi(1)}\cdots t_{i,\phi(i)}
          \cdots kt_{i,\phi(j)}\cdots t_{n,\phi(n)}
                                 \sgn(\phi)         \\ % [-1ex]
        &\hbox{}\qquad\quad\hbox{}
           +t_{1,\phi(1)}\cdots t_{i,\phi(i)}
           \cdots t_{j,\phi(j)}\cdots t_{n,\phi(n)}
                                 \sgn(\phi)\big]
      \end{aligned}                                               \\[.5ex]
  &=\begin{aligned}[t] 
         &\smash[b]{\sum_{\smash{\phi}}} \:t_{1,\phi(1)}\cdots t_{i,\phi(i)}
            \cdots kt_{i,\phi(j)}\cdots t_{n,\phi(n)}
                                 \sgn(\phi)           \\ % [-1ex]               
         &\hbox{}\qquad\quad\hbox{}
            +\smash[b]{\sum_{\phi}}\:
            t_{1,\phi(1)}\cdots t_{i,\phi(i)}
            \cdots t_{j,\phi(j)}\cdots t_{n,\phi(n)}
                                 \sgn(\phi)        
     \end{aligned}                                               \\[1.25ex]
  &=\begin{aligned}[t]
       &k\cdot \smash[b]{\sum_{\smash{\phi}}}\:
            t_{1,\phi(1)}\cdots t_{i,\phi(i)}
            \cdots t_{i,\phi(j)}\cdots t_{n,\phi(n)}
                                 \sgn(\phi)          \\  % [-.75ex]
       &\hbox{}\qquad\quad\hbox{}
             +d(T)          
     \end{aligned}
\end{align*}
%</pf:DeterminantsExist6>
%<*pf:DeterminantsExist7>
Finish by observing that the terms
\( t_{1,\phi(1)}\cdots t_{i,\phi(i)} \cdots t_{i,\phi(j)}\cdots t_{n,\phi(n)}
   \sgn(\phi)  \) add to zero:
this sum represents \( d(S) \) where \( S \) is a matrix equal to  \( T \)
except that row~$j$ of $S$ is a copy of row~$i$ of $T$ 
(because the factor is \( t_{i,\phi(j)} \), not \( t_{j,\phi(j)} \))
and so $S$ has two equal rows, rows~$i$ and~$j$.
Since we have already shown that $d$ changes sign on row swaps,
as in \nearbylemma{le:IdenRowsDetZero} we conclude that $d(S)=0$.
%</pf:DeterminantsExist7>
\end{proof}

We have now shown that determinant functions exist for each size.
We already know that for each size there is at most one
determinant.
Therefore, the permutation expansion
computes the one and only determinant.

We end this subsection by proving the other result 
remaining from the prior subsection.

% \begin{example}
% Writing out the permutation expansion of the general
% $\nbyn{3}$ matrix and of its transpose, and
% comparing corresponding terms
% \begin{equation*}
%   \begin{vmat}
%     a  &b  &c  \\
%     d  &e  &f  \\
%     g  &h  &i
%   \end{vmat}
%   = \cdots\,+
%   cdh\cdot\begin{vmat}[r]
%     0  &0  &1  \\
%     1  &0  &0  \\
%     0  &1  &0
%   \end{vmat}
%    +\,\cdots
% \end{equation*}
% (terms with the same letters)
% \begin{equation*}
%   \begin{vmat}
%     a  &d  &g  \\
%     b  &e  &h  \\
%     c  &f  &i
%   \end{vmat}
%   = \cdots\,+
%  dhc\cdot\begin{vmat}[r]
%     0  &1  &0  \\
%     0  &0  &1  \\
%     1  &0  &0
%   \end{vmat}
%    +\,\cdots\
% \end{equation*}
% shows that the corresponding permutation matrices are transposes.
% That is, there is a relationship between these corresponding permutations.
% \nearbyexercise{exer:PermInvIffMatTrans} 
% shows that they are inverses. 
% \end{example}

\begin{theorem} \label{th:DetMatrixEqualsDetTrans}
\index{transpose!determinant}
%<*th:DetMatrixEqualsDetTrans>
The determinant of a matrix equals the determinant of its transpose.
%</th:DetMatrixEqualsDetTrans>
\end{theorem}

\begin{proof}
%<*pf:DetMatrixEqualsDetTrans0>
Call the matrix \( T \) and denote the entries of \( \trans{T} \) with
\( s \)'s so that \( t_{i,j}=s_{j,i} \).
Substitution gives this
\begin{equation*}
  \deter{T}
  =\sum_{\text{perms }\phi} t_{1,\phi(1)}\cdots t_{n,\phi(n)}
          \sgn(\phi)    
  =\sum_{\phi} s_{\phi(1),1}\cdots s_{\phi(n),n}
          \sgn(\phi)
\end{equation*}
and we will finish the argument by manipulating the expression on the right
to be recognizable as the determinant of the transpose.
We have written all permutation expansions 
% (as in the middle expression above) 
with the row indices ascending.
To rewrite the expression on the right in this way, note that 
because $\phi$ is a permutation 
the row indices $\phi(1)$, \ldots, $\phi(n)$
are just the numbers $1$, \ldots, $n$, rearranged. 
Apply commutativity to have these ascend, giving
$s_{1,\phi^{-1}(1)}\cdots s_{n,\phi^{-1}(n)}$.
% (if the column index is $j$ and the row index is $\phi(j)$ then, 
% where the row index is $i$, the column index is $\phi^{-1}(i)$). 
% The right side becomes this.
\begin{equation*}
  =\sum_{\phi^{-1}}
     s_{1,\phi^{-1}(1)}\cdots s_{n,\phi^{-1}(n)} \sgn(\phi^{-1})
\end{equation*}
%</pf:DetMatrixEqualsDetTrans0>
%<*pf:DetMatrixEqualsDetTrans1>
\nearbyexercise{exer:PermInvFacts}
shows that \( \sgn(\phi^{-1})=\sgn(\phi) \).
Since every permutation is the inverse of another,
a sum over all inverses $\phi^{-1}$ is a sum over all permutations
\begin{equation*}
  =\sum_{\text{perms }\sigma}
     s_{1,\sigma^(1)}\dots s_{n,\sigma(n)} \sgn(\sigma)  
  =\deter{\trans{T}}
\end{equation*}
as required.
%</pf:DetMatrixEqualsDetTrans1>
\end{proof}


\begin{exercises}
  \item[]\textit{These summarize the notation used in this book
    for the $2$-\hbox{} and $3$-\hbox{} permutations.
    \begin{center}
      \begin{tabular}[t]{c|cc}
        $i$          &$1$      &$2$    \\
        \hline
        $\phi_1(i)$  &$1$      &$2$     \\
        $\phi_2(i)$  &$2$      &$1$     
      \end{tabular}
      \qquad
      \begin{tabular}[t]{c|ccc}
        $i$          &$1$     &$2$   &$3$    \\
        \hline
        $\phi_1(i)$  &$1$     &$2$   &$3$    \\
        $\phi_2(i)$  &$1$     &$3$   &$2$    \\
        $\phi_3(i)$  &$2$     &$1$   &$3$    \\
        $\phi_4(i)$  &$2$     &$3$   &$1$    \\
        $\phi_5(i)$  &$3$     &$1$   &$2$    \\
        $\phi_6(i)$  &$3$     &$2$   &$1$    
      \end{tabular}
    \end{center}  }
  \item 
    Give the permutation expansion of a general $\nbyn{2}$ matrix
    and its transpose.
    \begin{answer}
      This is the permutation expansion of the determinant of a 
      $\nbyn{2}$ matrix
      \begin{equation*}
        \begin{vmat}
          a  &b  \\
          c  &d
        \end{vmat}
         =ad\cdot \begin{vmat}[r]
                    1  &0  \\
                    0  &1
                  \end{vmat}
         +bc\cdot \begin{vmat}[r]
                    0  &1 \\
                    1  &0
                  \end{vmat}
      \end{equation*}
      and the permutation expansion of the determinant of its transpose.
      \begin{equation*}
         \begin{vmat}
            a  &c  \\
            b  &d
         \end{vmat}
         =ad\cdot\begin{vmat}[r]
                    1  &0  \\
                    0  &1
                 \end{vmat}
         +cb\cdot\begin{vmat}[r]
                   0  &1 \\
                   1  &0
                 \end{vmat}
       \end{equation*}
       As with the $\nbyn{3}$ expansions described in the subsection,
       the permutation matrices from corresponding terms are
       transposes (although this is disguised by the fact that each
       is self-transpose).
    \end{answer}
  \recommended \item 
    \textit{This problem appears also in the prior subsection.}
    \begin{exparts}
      \partsitem Find the inverse of each $2$-permutation.
      \partsitem Find the inverse of each $3$-permutation.
    \end{exparts}
    \begin{answer}
      Each of these is easy to check.
      \begin{exparts*}
        \partsitem 
          \begin{tabular}[t]{r|cc}
            \textit{permutation} &$\phi_1$  &$\phi_2$ \\
             \hline
            \textit{inverse}     &$\phi_1$  &$\phi_2$ 
          \end{tabular}
        \partsitem 
          \begin{tabular}[t]{r|cccccc}
            \textit{permutation} 
              &$\phi_1$ &$\phi_2$ &$\phi_3$ &$\phi_4$ &$\phi_5$ &$\phi_6$ \\
            \hline
            \textit{inverse}
              &$\phi_1$ &$\phi_2$ &$\phi_3$ &$\phi_5$ &$\phi_4$ &$\phi_6$ 
          \end{tabular}
      \end{exparts*}
    \end{answer}
  \recommended \item
    \begin{exparts}
      \partsitem Find the signum of each $2$-permutation.
      \partsitem Find the signum of each $3$-permutation.
    \end{exparts}
    \begin{answer}
      \begin{exparts}
        \partsitem \( \sgn(\phi_1)=+1 \), \( \sgn(\phi_2)=-1 \)
        \partsitem \( \sgn(\phi_1)=+1 \), \( \sgn(\phi_2)=-1 \),
              \( \sgn(\phi_3)=-1 \), \( \sgn(\phi_4)=+1 \),
              \( \sgn(\phi_5)=+1 \), \( \sgn(\phi_6)=-1 \)
      \end{exparts}  
     \end{answer}
  \item 
    Find the only nonzero term in the permutation expansion of
    this matrix.
    \begin{equation*}
      \begin{vmat}[r]
        0  &1  &0  &0  \\
        1  &0  &1  &0  \\
        0  &1  &0  &1  \\
        0  &0  &1  &0
      \end{vmat}
    \end{equation*}
    Compute that determinant by finding the signum of the associated
    permutation.
    \begin{answer}
      To get a nonzero term in the permutation expansion we must use
      the \( 1,2 \) entry and the \( 4,3 \) entry.
      Having fixed on those two we must also use the \( 2,1 \) entry and
      the \( 3,4 \) entry.
      The signum of \( \sequence{2,1,4,3} \) is \( +1 \) because from
      \begin{equation*}
        \begin{mat}[r]
          0  &1  &0  &0  \\
          1  &0  &0  &0  \\
          0  &0  &0  &1  \\
          0  &0  &1  &0            
        \end{mat}
      \end{equation*}
      the two row swaps $\rho_1\leftrightarrow\rho_2$ and 
      $\rho_3\leftrightarrow\rho_4$ will produce the identity matrix.  
    \end{answer}
  \item  
    \cite{Strang}
    What is the signum of the $n$-permutation
    \( \phi=\sequence{n,n-1,\dots,2,1} \)?
    \begin{answer}
      The pattern is this.
      \begin{center}
         \begin{tabular}[b]{c|ccccccc}
           \( i \) &\( 1 \) &\( 2 \) &\( 3 \) &\( 4 \) &\( 5 \)
              &\( 6 \) &\ldots \\
           \hline
           \( \sgn(\phi_i) \) &\( +1 \) &\( -1 \) &\( -1 \) &\( +1 \)
              &\( +1 \) &\( -1 \) &\ldots
         \end{tabular}
      \end{center}  
      So to find the signum of $\phi_{n!}$, we subtract one $n!-1$ and
      look at the remainder on division by four.
      If the remainder is $1$ or $2$ then the signum is $-1$, otherwise it
      is $+1$.
      For $n>4$, the number $n!$ is divisible by four, so $n!-1$ leaves a 
      remainder of $-1$ on division by four (more properly said, a remainder
      or $3$), and so the signum is $+1$.
      The $n=1$ case has a signum of $+1$, the $n=2$ case has a signum of 
      $-1$ and the $n=3$ case has a signum of $-1$.
    \end{answer}
  \item \label{exer:PermInvFacts}
     Prove these.   
     \begin{exparts}
        \partsitem Every permutation has an inverse.
        \partsitem \( \sgn(\phi^{-1})=\sgn(\phi) \)
        \partsitem Every permutation is the inverse of another.
     \end{exparts}
    \begin{answer}
      \begin{exparts}
       \partsitem We can view permutations as one-one and onto maps
         \( \map{\phi}{\set{1,\ldots,n}}{\set{1,\ldots,n}} \).
         Any one-one and onto map has an inverse.
       \partsitem If it always takes an odd number of swaps to get 
         from \( P_\phi \)
         to the identity, then it always takes an odd number of swaps
         to get from the identity to \( P_\phi \) (any swap is reversible).
       \partsitem This is the first question again.
     \end{exparts}  
    \end{answer}
  \item \label{exer:PermInvIffMatTrans}
      Prove that the matrix of the permutation inverse
      is the transpose of the matrix of the permutation 
      $P_{\phi^{-1}}=\trans{P_{\phi}}$, 
      for any permutation $\phi$.
      \begin{answer}
        If $\phi(i)=j$ then $\phi^{-1}(j)=i$.
        The result now follows on the observation that $P_{\phi}$ has a 
        $1$ in entry $i,j$ if and only if $\phi(i)=j$,
        and $P_{\phi^{-1}}$ has a 
        $1$ in entry $j,i$ if and only if $\phi^{-1}(j)=i$,
      \end{answer}
  \recommended \item 
    Show that a permutation matrix with \( m \) inversions can be
    row swapped to the identity in \( m \) steps.
    Contrast this with \nearbycorollary{cor:ParityInversEqParitySwaps}.
    \begin{answer}
      This does not say that \( m \) is the least number of swaps to produce
      an identity, nor does it say that \( m \) is the most.
      It instead says that 
      there is a way to swap to the identity in exactly \( m \) steps.

      Let \( \iota_j \) be the first row that is inverted with respect
      to a prior row
      and let \( \iota_k \) be the first row giving that inversion.
      We have this interval of rows.
      \begin{equation*}
         \begin{mat}
           \vdots      \\
           \iota_k     \\
           \iota_{r_1} \\
           \vdots      \\
           \iota_{r_s} \\
           \iota_j     \\
           \vdots
         \end{mat}
         \qquad j<k<r_1<\dots <r_s
      \end{equation*}
      Swap.
      \begin{equation*}
         \begin{mat}
           \vdots      \\
           \iota_j     \\
           \iota_{r_1} \\
           \vdots      \\
           \iota_{r_s} \\
           \iota_k     \\
           \vdots
         \end{mat}
      \end{equation*}
      The second matrix has one fewer inversion because there is one
      fewer inversion
      in the interval (\( s \) vs.\ \( s+1 \)) and inversions involving
      rows outside the interval are not affected.

      Proceed in this way, at each step reducing the number of inversions
      by one
      with each row swap.
      When no inversions remain the result is the identity. 

      The contrast with \nearbycorollary{cor:ParityInversEqParitySwaps}
      is that the statement of this exercise is a `there exists' statement:
      there exists a way to swap to the identity in exactly~$m$ steps.
      But the corollary is a `for all' statement: for all ways to swap to the
      identity, the parity (evenness or oddness) is the same. 
    \end{answer}
  \recommended \item 
    For any permutation $\phi$ let $g(\phi)$ be the integer 
    defined in this way.
    \begin{equation*}
       g(\phi)=\prod_{i<j} [\phi(j)-\phi(i)]
    \end{equation*}
    (This is the product, over all indices \( i \) and \( j \) with \( i<j \),
    of terms of the given form.)
    \begin{exparts}
      \partsitem Compute the value of $g$ on all $2$-permutations.
      \partsitem Compute the value of $g$ on all $3$-permutations.
      \partsitem Prove that $g(\phi)$ is not $0$.
      \partsitem Prove this.
         \begin{equation*}
            \sgn(\phi)=\frac{g(\phi)}{\absval{g(\phi)}}
         \end{equation*}
    \end{exparts}
    Many authors give this formula as the definition of the signum function.
    \begin{answer}
      \begin{exparts}
        \partsitem First, $g(\phi_1)$ is the product of the single
          factor $2-1$ and so $g(\phi_1)=1$.
          Second, $g(\phi_2)$ is the product of the single
          factor $1-2$ and so $g(\phi_2)=-1$.
        \partsitem 
            \begin{tabular}[t]{r|cccccc}
               \textit{permutation $\phi$} 
                 &$\phi_{1}$ &$\phi_{2}$ &$\phi_{3}$ 
                    &$\phi_{4}$ &$\phi_{5}$ &$\phi_{6}$ \\
               \hline
               $g(\phi)$ &$2$ &$-2$ &$-2$ &$2$ &$2$ &$-2$ 
            \end{tabular}
        \partsitem It is a product of nonzero terms.
        \partsitem Note that \( \phi(j)-\phi(i) \) is negative if and only if
           \( \iota_{\phi(j)} \) and \( \iota_{\phi(i)} \) are in an inversion
           of their usual order.
      \end{exparts}  
     \end{answer}
%  \item Prove that the signum function on permutations is unique in that
%    if \( f \) is a map from the permutations of \( \set{1,\ldots,n} \)
%    to the integers that is multiplicative:
%    \( f(\sigma\phi)=f(\sigma)f(\phi) \) then \( f \) is either the
%    function that always gives zero, or the function that always gives
%    one, or is the signum.
%    \begin{answer}
%      Consider the identity permutation.
%      \begin{equation*}
%        \map{\identity}{\set{1,2,\ldots,n}}{\set{1,2,\ldots,n}}    
%      \end{equation*}
%      Note that $f(\identity)$ equals $0$ or $1$ because
%      the multiplicative property gives 
%      $f(\identity)
%       =f(\composed{\identity}{\identity})
%       =f(\identity)\cdot f(\identity)
%       =f(\identity)^2$.
%      If $f(\identity)=0$ then $f$ is the constant function 
%      $f(\phi)=0$ because 
%      $f(\phi)=f(\composed{\identity}{\phi})=0\cdot f(\phi)=0$.
%
%      If $f(\identity)=1$ then consider the permutation
%      $\tau$ that just swaps $i$ and $j$
%      \begin{equation*}
%        1\mapsto 1
%        \quad 2\mapsto 2
%        \quad \cdots 
%        \quad i\mapsto j
%        \quad \cdots
%        \quad j\mapsto i
%        \quad \cdots
%        \quad n\mapsto n
%      \end{equation*}
%      (the $n=1$ special case is easy).
%      Note that $\composed{\tau}{\tau}=\identity$ so 
%      $1=f(\composed{\tau}{\tau})=f(\tau)\cdot f(\tau)$,
%      and $f(\tau)$ is either $+1$ or $-1$.
%    \end{answer}
\end{exercises}



\index{determinant|)}
