\documentclass{article}

\RequirePackage{amsmath}

\usepackage[framemethod=tikz]{mdframed}
\definecolor{shadethmcolor}{rgb}{0.9412,.9412,1.0000} % 
\definecolor{shaderulecolor}{rgb}{0.1529,0.2510,0.5451} % RoyalBlue
\mdfdefinestyle{thmstyle}{linewidth=0.4pt,
          leftmargin=0em,innerleftmargin=0.2em,
          rightmargin=0em,innerrightmargin=0.2em,
          innertopmargin=0em,
          innerbottommargin=0em,
          backgroundcolor=shadethmcolor,
          linecolor=shaderulecolor}

\newtheorem{definition}{Definition}
  \surroundwithmdframed[style=thmstyle]{definition}


\begin{document}

\begin{definition}
A \emph{linear combination} of
\( x_1 \), \ldots, \( x_n \) has the form
\begin{equation*}
   a_1x_1+a_2x_2+a_3x_3+\cdots+a_nx_n
\end{equation*}
where the numbers \( a_1, \ldots ,a_n\in\Re \) are the combination's
\emph{coefficients}.
A \emph{linear equation} has the form
$a_1x_1+a_2x_2+a_3x_3+\cdots+a_nx_n=d$
where
\( d\in\Re \) is the \emph{constant}.

An \( n \)-tuple \( (s_1,s_2,\ldots ,s_n)\in\Re^n \) is a 
\emph{solution} 
of, or \emph{satisfies}, that equation if substituting the numbers
$s_1$, \ldots, $s_n$ for the variables $x_1$, \ldots, $x_n$
gives a true statement:
$a_1s_1+a_2s_2+\ldots+a_ns_n=d$.

A \emph{system of linear equations}
\begin{gather*}
    a_{1,1}x_1 + a_{1,2}x_2  +  \cdots + a_{1,n}x_n = d_1  \\
    a_{2,1}x_1 + a_{2,2}x_2  +  \cdots + a_{2,n}x_n =  d_2  \\
                \qquad         \vdots   \\
    a_{m,1}x_1 + a_{m,2}x_2  + \cdots + a_{m,n}x_n =  d_m
\end{gather*}
has the solution
\( (s_1,s_2,\ldots ,s_n) \) if that $n$-tuple is a solution of all
of the equations in the system.
\end{definition}

\end{document}