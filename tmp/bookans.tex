\chapter{Chapter One: Linear Systems}
\subsection{Subsection One.I.1: Linear Systems}
\begin{ans}{One.I.1.17}
      Gauss' method can be performed in different ways, so these simply
      exhibit one possible way to get the answer.
      \begin{exparts}
        \partsitem Gauss' method
          \begin{equation*}
            \grstep{-(1/2)\rho_1+\rho_2}\;
            \begin{linsys}{2}
               2x  &+  &3y      &=  &13  \\
                   &-  &(5/2)y  &=  &-15/2
            \end{linsys}
          \end{equation*}
          gives that the solution is $y=3$ and $x=2$.
        \partsitem Gauss' method here
          \begin{equation*}
            \grstep[\rho_1+\rho_3]{-3\rho_1+\rho_2}\;
            \begin{linsys}{3}
              x   &  &  &-  &z  &=  &0  \\
                  &  &y &+  &3z &=  &1  \\
                  &  &y &   &   &=  &4
            \end{linsys}
            \;\grstep{-\rho_2+\rho_3}\;
            \begin{linsys}{3}
              x   &  &  &-  &z    &=  &0  \\
                  &  &y &+  &3z   &=  &1  \\
                  &  &  &   &-3z  &=  &3
            \end{linsys}
          \end{equation*}
          gives $x=-1$, $y=4$, and $z=-1$.
      \end{exparts}
    
\end{ans}
\begin{ans}{One.I.1.18}
      \begin{exparts}
       \partsitem Gaussian reduction
        \begin{eqnarray*}
          &\grstep{-(1/2)\rho_1+\rho_2}
          &\begin{linsys}{2}
             2x  &+  &2y  &=  &5  \\
                 &   &-5y &=  &-5/2
          \end{linsys}
        \end{eqnarray*}
        shows that \( y=1/2 \) and \( x=2 \) is the unique solution.
      \partsitem Gauss' method
        \begin{eqnarray*}
          &\grstep{\rho_1+\rho_2}
          &\begin{linsys}{2}
             -x  &+  &y   &=  &1  \\
                 &   &2y  &=  &3
           \end{linsys}
        \end{eqnarray*}
        gives \( y=3/2 \) and \( x=1/2 \) as the only solution.
      \partsitem Row reduction
        \begin{eqnarray*}
            &\grstep{-\rho_1+\rho_2}
            &\begin{linsys}{3}
                x  &-  &3y  &+  &z  &=  &1  \\
                   &   &4y  &+  &z  &=  &13
             \end{linsys}
        \end{eqnarray*}
        shows, because the variable $z$ is not a leading variable in any
        row, that there are many solutions.
      \partsitem Row reduction
        \begin{eqnarray*}
          &\grstep{-3\rho_1+\rho_2}
          &\begin{linsys}{2}
             -x  &-  &y   &=  &1  \\
                 &   &0   &=  &-1
           \end{linsys}
        \end{eqnarray*}
        shows that there is no solution.
      \partsitem Gauss' method
        \begin{equation*}
            \grstep{\rho_1\leftrightarrow\rho_4}\;
            \begin{linsys}{3}
                x  &+  &y   &-  &z  &=  &10 \\
               2x  &-  &2y  &+  &z  &=  &0  \\
                x  &   &    &+  &z  &=  &5  \\
                   &   &4y  &+  &z  &=  &20
             \end{linsys}
            \;\grstep[-\rho_1+\rho_3]{-2\rho_1+\rho_2}\;
            \begin{linsys}{3}
                x  &+  &y   &-  &z  &=  &10 \\
                   &   &-4y &+  &3z &=  &-20\\
                   &   &-y  &+  &2z &=  &-5 \\
                   &   &4y  &+  &z  &=  &20
             \end{linsys}
            \;\grstep[\rho_2+\rho_4]{-(1/4)\rho_2+\rho_3}\;
            \begin{linsys}{3}
                x  &+  &y   &-  &z      &=  &10 \\
                   &   &-4y &+  &3z     &=  &-20\\
                   &   &    &   &(5/4)z &=  &0  \\
                   &   &    &   &4z     &=  &0
             \end{linsys}
        \end{equation*}
        gives the unique solution \( (x,y,z)=(5,5,0) \).
      \partsitem Here Gauss' method gives
         \begin{equation*}
            \grstep[-2\rho_1+\rho_4]{-(3/2)\rho_1+\rho_3}\;
            \begin{linsys}{4}
               2x  &   &   &+  &z       &+  &w       &=  &5  \\
                   &   &y  &   &        &-  &w       &=  &-1 \\
                   &   &   &-  &(5/2)z  &-  &(5/2)w  &=  &-15/2  \\
                   &   &y  &   &        &-  &w       &=  &-1
             \end{linsys}
            \;\grstep{-\rho_2+\rho_4}\;
            \begin{linsys}{4}
               2x  &   &   &+  &z       &+  &w       &=  &5  \\
                   &   &y  &   &        &-  &w       &=  &-1 \\
                   &   &   &-  &(5/2)z  &-  &(5/2)w  &=  &-15/2  \\
                   &   &   &   &        &   &0       &=  &0
             \end{linsys}
         \end{equation*}
         which shows that there are many solutions.
      \end{exparts}
    
\end{ans}
\begin{ans}{One.I.1.19}
      \begin{exparts}
        \partsitem From $x=1-3y$ we get that $2(1-3y)+y=-3$, giving $y=1$.
        \partsitem From $x=1-3y$ we get that $2(1-3y)+2y=0$, leading to
           the conclusion that $y=1/2$.
      \end{exparts}
      Users of this method must check any potential solutions by
      substituting back into all the equations.
    
\end{ans}
\begin{ans}{One.I.1.20}
      Do the reduction
      \begin{eqnarray*}
       &\grstep{-3\rho_1+\rho_2}
       &\begin{linsys}{2}
          x  &-  &y  &=  &1\hfill  \\
             &   &0  &=  &-3+k\hfill
        \end{linsys}
      \end{eqnarray*}
      to conclude this system has no solutions if \( k\neq 3 \) and if
      \( k=3 \) then it has infinitely many solutions.
      It never has a unique solution.
    
\end{ans}
\begin{ans}{One.I.1.21}
      Let \( x=\sin\alpha \), \( y=\cos\beta \), and \( z=\tan\gamma \):
      \begin{eqnarray*}
        \begin{linsys}{3}
           2x  &-  &y  &+  &3z  &=  &3  \\
           4x  &+  &2y &-  &2z  &=  &10  \\
           6x  &-  &3y &+  &z   &=  &9
        \end{linsys}
        &\grstep[-3\rho_1+\rho_3]{-2\rho_1+\rho_2}
        &\begin{linsys}{3}
           2x  &-  &y  &+  &3z  &=  &3  \\
               &   &4y &-  &8z  &=  &4   \\
               &   &   &   &-8z &=  &0
         \end{linsys}
      \end{eqnarray*}
      gives \( z=0 \), \( y=1 \), and \( x=2 \).
      Note that no \( \alpha \) satisfies that requirement.
      
\end{ans}
\begin{ans}{One.I.1.22}
      \begin{exparts}
       \partsitem Gauss' method
         \begin{equation*}
           \grstep[-\rho_1+\rho_3 \\ -2\rho_1+\rho_4]{-3\rho_1+\rho_2}\;
           \begin{linsys}{2}
              x  &-  &3y  &=  &b_1\hfill \\
                 &   &10y &=  &-3b_1+b_2\hfill \\
                 &   &10y &=  &-b_1+b_3\hfill \\
                 &   &10y &=  &-2b_1+b_4\hfill
            \end{linsys}
           \;\grstep[-\rho_2+\rho_4]{-\rho_2+\rho_3}\;
           \begin{linsys}{2}
              x  &-  &3y  &=  &b_1\hfill \\
                 &   &10y &=  &-3b_1+b_2\hfill \\
                 &   &0   &=  &2b_1-b_2+b_3\hfill \\
                 &   &0   &=  &b_1-b_2+b_4\hfill
            \end{linsys}
         \end{equation*}
         shows that this system is consistent if and only if both
         \( b_3=-2b_1+b_2 \) and \( b_4=-b_1+b_2 \).
       \partsitem Reduction
         \begin{equation*}
            \grstep[-\rho_1+\rho_3]{-2\rho_1+\rho_2}\;
            \begin{linsys}{3}
              x_1  &+  &2x_2  &+  &3x_3  &=  &b_1\hfill  \\
                   &   &x_2   &-  &3x_3  &=  &-2b_1+b_2\hfill  \\
                   &   &-2x_2 &+  &5x_3  &=  &-b_1+b_3\hfill
             \end{linsys}
            \;\grstep{2\rho_2+\rho_3}\;
            \begin{linsys}{3}
              x_1  &+  &2x_2  &+  &3x_3  &=  &b_1\hfill  \\
                   &   &x_2   &-  &3x_3  &=  &-2b_1+b_2\hfill  \\
                   &   &      &   &-x_3  &=  &-5b_1+2b_2+b_3\hfill
             \end{linsys}
         \end{equation*}
         shows that each of \( b_1 \), \( b_2 \), and \( b_3 \) can be any
         real number\Dash this system always has a unique solution.
      \end{exparts}
      
\end{ans}
\begin{ans}{One.I.1.23}
      This system with more unknowns than equations
      \begin{equation*}
        \begin{linsys}{3}
          x  &+  &y  &+  &z  &=  &0  \\
          x  &+  &y  &+  &z  &=  &1
        \end{linsys}
      \end{equation*}
      has no solution.
      
\end{ans}
\begin{ans}{One.I.1.24}
      Yes.
      For example, the fact that the same reaction can be performed
      in two different flasks shows that twice any solution is another,
      different, solution (if a physical reaction occurs then there must be
      at least one nonzero solution).
    
\end{ans}
\begin{ans}{One.I.1.25}
      Because \( f(1)=2 \), \( f(-1)=6 \), and \( f(2)=3 \) we get
      a linear system.
      \begin{equation*}
        \begin{linsys}{3}
          1a  &+  &1b  &+  &c  &=  &2  \\
          1a  &-  &1b  &+  &c  &=  &6  \\
          4a  &+  &2b  &+  &c  &=  &3
         \end{linsys}
      \end{equation*}
      Gauss' method
      \begin{eqnarray*}
         \grstep[-4\rho_1+\rho_3]{-\rho_1+\rho_2}\;
         \begin{linsys}{3}
            a  &+  &b  &+  &c  &=  &2  \\
               &   &-2b&   &   &=  &4  \\
               &   &-2b&-  &3c &=  &-5
          \end{linsys}
         &\grstep{-\rho_2+\rho_3}
         &\begin{linsys}{3}
            a  &+  &b  &+  &c  &=  &2  \\
               &   &-2b&   &   &=  &4  \\
               &   &   &   &-3c&=  &-9
           \end{linsys}
      \end{eqnarray*}
      shows that the solution is \( f(x)=1x^2-2x+3 \).
      
\end{ans}
\begin{ans}{One.I.1.26}
     Here $S_0=\set{(1,1)}$
     \begin{equation*}
         \begin{linsys}{2}
            x  &+  &y  &=  &2  \\
            x  &-  &y  &=  &0
          \end{linsys}
         \;\grstep{0\rho_2}\;
         \begin{linsys}{2}
            x  &+  &y  &=  &2  \\
               &   &0  &=  &0
          \end{linsys}
     \end{equation*}
     while $S_1$ is a proper superset because it
     contains at least two points: $(1,1)$ and~$(2,0)$.
     In this example the solution set does not change.
     \begin{equation*}
         \begin{linsys}{2}
            x  &+  &y  &=  &2  \\
           2x  &+  &2y &=  &4
          \end{linsys}
         \;\grstep{0\rho_2}\;
         \begin{linsys}{2}
            x  &+  &y  &=  &2  \\
               &   &0  &=  &0
          \end{linsys}
     \end{equation*}
   
\end{ans}
\begin{ans}{One.I.1.27}
       \begin{exparts}
        \partsitem Yes, by inspection the given equation results from
          \( -\rho_1+\rho_2 \).
        \partsitem No.
          The given equation is satisfied by the pair \( (1,1) \).
          However, that pair
          does not satisfy the first equation in the system.
        \partsitem Yes.
          To see if the given row is \( c_1\rho_1+c_2\rho_2 \), solve
          the system of equations relating the coefficients of $x$, $y$,
          $z$, and the constants:
          \begin{equation*}
            \begin{linsys}{2}
               2c_1  &+  &6c_2  &=  &6  \\
                c_1  &-  &3c_2  &=  &-9 \\
               -c_1  &+  &c_2   &=  &5  \\
               4c_1  &+  &5c_2  &=  &-2
            \end{linsys}
          \end{equation*}
          and get $c_1=-3$ and $c_2=2$, so the given row is
          \( -3\rho_1+2\rho_2 \).
      \end{exparts}
     
\end{ans}
\begin{ans}{One.I.1.28}
      If \( a\neq 0 \) then the solution set of the first equation is
      \( \set{(x,y)\suchthat x=(c-by)/a} \).
      Taking $y=0$ gives the solution $(c/a,0)$, and since the second
      equation is supposed to have the same solution set, substituting into
      it gives that $a(c/a)+d\cdot 0=e$, so $c=e$.
      Then taking $y=1$ in $x=(c-by)/a$ gives that $a((c-b)/a)+d\cdot 1=e$,
      which gives that $b=d$.
      Hence they are the same equation.

      When \( a=0 \) the equations can be different and still have the
      same solution set:~e.g.,
      \( 0x+3y=6 \) and \( 0x+6y=12 \).
     
\end{ans}
\begin{ans}{One.I.1.29}
      We take three cases: that $a\neq 0$, that $a=0$ and
      $c\neq 0$, and that both $a=0$ and $c=0$.

      For the first, we assume that \( a\neq 0 \).
      Then the reduction
      \begin{eqnarray*}
        &\grstep{-(c/a)\rho_1+\rho_2}
        &\begin{linsys}{2}
          ax  &+  &by                  &=  &j \hfill \\
              &   &(-\frac{cb}{a}+d)y  &=  &-\frac{cj}{a}+k \hfill
         \end{linsys}
      \end{eqnarray*}
      shows that this system has a unique solution if and only if
      \( -(cb/a)+d\neq 0   \); remember that \( a\neq 0 \) so
      that back substitution yields a unique \( x \)
      (observe, by the way, that \( j \) and \( k \) play no role in the
      conclusion that there is a unique solution, although if there is a
      unique solution then they contribute to its value).
      But \( -(cb/a)+d = (ad-bc)/a \) and a fraction is not equal to \( 0 \)
      if and only if its numerator is not equal to \( 0 \).
      Thus, in this first case, there is a unique solution if and only if
      $ad-bc\neq 0$.

      In the second case, if \( a=0 \) but \( c\neq 0 \), then we swap
      \begin{equation*}
        \begin{linsys}{2}
          cx  &+  &dy  &=  &k  \\
              &   &by  &=  &j
        \end{linsys}
      \end{equation*}
      to conclude that the system has a unique solution if and only if
      \( b\neq 0 \)
      (we use the case assumption that \( c\neq 0 \) to get a unique
      \( x \) in back substitution).
      But\Dash where \( a=0 \) and \( c\neq 0 \)\Dash
      the condition ``\( b\neq 0 \)''
      is equivalent to the condition ``\( ad-bc\neq 0 \)''.
      That finishes the second case.

      Finally, for the third case,
      if both \( a \) and \( c \) are \( 0 \) then the system
      \begin{equation*}
        \begin{linsys}{2}
          0x  &+  &by  &=  &j  \\
          0x  &+  &dy  &=  &k
        \end{linsys}
      \end{equation*}
      might have no solutions (if the second equation is not a multiple of the
      first) or it might have infinitely many solutions (if the second
      equation is a multiple of the first then for each \( y \) satisfying
      both equations, any pair \( (x,y) \) will do), but it never has a unique
      solution.
      Note that \( a=0 \) and \( c=0 \) gives that \( ad-bc=0 \).
    
\end{ans}
\begin{ans}{One.I.1.30}
      Recall that if a pair of lines share two distinct points then
      they are the same line.
      That's because two points determine a line, so these
      two points determine each of the two lines,
      and so they are the same line.

      Thus the lines can share one point (giving a unique solution),
      share no points (giving no solutions), or
      share at least two points (which makes them the same line).
    
\end{ans}
\begin{ans}{One.I.1.31}
     For the reduction operation of multiplying $\rho_i$ by a nonzero
     real number $k$, we have that \( (s_1,\ldots,s_n) \) satisfies
     this system
     \begin{equation*}
       \begin{linsys}{4}
         a_{1,1}x_1  &+  &a_{1,2}x_2 &+  &\cdots  &+  &a_{1,n}x_n  &=  &d_1  \\
                     &   &           &   &        &   &            &\vdots   \\
        ka_{i,1}x_1  &+  &ka_{i,2}x_2 &+  &\cdots  &+  &ka_{i,n}x_n
            &=  &kd_i  \\
                     &   &           &   &        &   &            &\vdots   \\
         a_{m,1}x_1  &+  &a_{m,2}x_2 &+  &\cdots  &+  &a_{m,n}x_n  &=  &d_m
       \end{linsys}
     \end{equation*}
     if and only if
     \begin{align*}
        a_{1,1}s_1+a_{1,2}s_2+\cdots+a_{1,n}s_n
        &=d_1                                              \\
        &\alignedvdots                                     \\
        \text{and\ } ka_{i,1}s_1+ka_{i,2}s_2+\cdots+ka_{i,n}s_n
        &=kd_i                                              \\
        &\alignedvdots                                      \\
        \text{and\ } a_{m,1}s_1+a_{m,2}s_2+\cdots+a_{m,n}s_n
        &=d_m
     \end{align*}
     by the definition of `satisfies'.
     But, because \( k\neq 0 \), that's true if and only if
     \begin{align*}
        a_{1,1}s_1+a_{1,2}s_2+\cdots+a_{1,n}s_n
        &=d_1                                              \\
        &\alignedvdots                                     \\
        \text{and\ } a_{i,1}s_1+a_{i,2}s_2+\cdots+a_{i,n}s_n
        &=d_i                                              \\
        &\alignedvdots                                      \\
        \text{and\ } a_{m,1}s_1+a_{m,2}s_2+\cdots+a_{m,n}s_n
        &=d_m
     \end{align*}
     (this is straightforward canceling on both sides of the $i$-th equation),
     which says that \( (s_1,\ldots,s_n) \) solves
     \begin{equation*}
       \begin{linsys}{4}
         a_{1,1}x_1  &+  &a_{1,2}x_2 &+  &\cdots  &+  &a_{1,n}x_n  &=  &d_1  \\
                     &   &           &   &        &   &            &\vdots   \\
         a_{i,1}x_1  &+  &a_{i,2}x_2 &+  &\cdots  &+  &a_{i,n}x_n  &=  &d_i  \\
                     &   &           &   &        &   &            &\vdots   \\
         a_{m,1}x_1  &+  &a_{m,2}x_2 &+  &\cdots  &+  &a_{m,n}x_n  &=
              &d_m
         \end{linsys}
     \end{equation*}
     as required.

     For the combination operation $k\rho_i+\rho_j$,
     we have that \( (s_1,\ldots,s_n) \) satisfies
     \begin{equation*}
       \begin{linsys}{4}
         a_{1,1}x_1             &+  &\cdots  &+  &a_{1,n}x_n  &=  &d_1\hfill \\
                                &   &        &   &            &\vdots   \\
         a_{i,1}x_1             &+  &\cdots  &+  &a_{i,n}x_n  &=  &d_i\hfill \\
                                &   &        &   &            &\vdots   \\
         (ka_{i,1}+a_{j,1})x_1  &+  &\cdots  &+  &(ka_{i,n}+a_{j,n})x_n
               &=  &kd_i+d_j \hfill \\
                                &   &        &   &            &\vdots   \\
         a_{m,1}x_1             &+   &\cdots  &+  &a_{m,n}x_n  &=
          &d_m\hfill\hbox{}
        \end{linsys}
     \end{equation*}
     if and only if
     \begin{align*}
        a_{1,1}s_1+\cdots+a_{1,n}s_n
        &=d_1                                              \\
        &\alignedvdots                                     \\
        \text{and\ } a_{i,1}s_1+\cdots+a_{i,n}s_n
        &=d_i                                              \\
        &\alignedvdots                                      \\
        \text{and\ } (ka_{i,1}+a_{j,1})s_1+\cdots+(ka_{i,n}+a_{j,n})s_n
        &=kd_i+d_j                                              \\
        &\alignedvdots                                      \\
        \text{and\ } a_{m,1}s_1+a_{m,2}s_2+\cdots+a_{m,n}s_n
        &=d_m
     \end{align*}
     again by the definition of `satisfies'.
     Subtract \( k \) times the \( i \)-th equation from the \( j \)-th
     equation
     (remark:~here is where \( i\neq j \) is needed; if \( i=j \) then the two
     \( d_i \)'s above are not equal) to
     get that the previous compound statement holds if and only if
     \begin{align*}
        a_{1,1}s_1+\cdots+a_{1,n}s_n
        &=d_1                                              \\
        &\alignedvdots                                     \\
        \text{and\ } a_{i,1}s_1+\cdots+a_{i,n}s_n
        &=d_i                                              \\
        &\alignedvdots                                      \\
        \text{and\ } (ka_{i,1}+a_{j,1})s_1+\cdots+(ka_{i,n}+a_{j,n})s_n \\
        \quad\hbox{}-(ka_{i,1}s_1+\cdots+ka_{i,n}s_n)
        &=kd_i+d_j-kd_i                                    \\
        &\alignedvdots                                      \\
        \text{and\ } a_{m,1}s_1+\cdots+a_{m,n}s_n
        &=d_m
     \end{align*}
     which, after cancellation, says that \( (s_1,\ldots,s_n) \) solves
     \begin{equation*}
       \begin{linsys}{4}
         a_{1,1}x_1  &+   &\cdots  &+  &a_{1,n}x_n  &=  &d_1  \\
                     &    &        &   &            &\vdots   \\
         a_{i,1}x_1  &+   &\cdots  &+  &a_{i,n}x_n  &=  &d_i  \\
                     &    &        &   &            &\vdots   \\
         a_{j,1}x_1  &+  &\cdots  &+  &a_{j,n}x_n  &=  &d_j  \\
                     &   &        &   &            &\vdots   \\
         a_{m,1}x_1  &+  &\cdots  &+  &a_{m,n}x_n  &=
              &d_m\hfill\hbox{}
       \end{linsys}
     \end{equation*}
     as required.
   
\end{ans}
\begin{ans}{One.I.1.32}
      Yes, this one-equation system:
      \begin{equation*}
         0x+0y=0
      \end{equation*}
      is satisfied by every \( (x,y)\in\Re^2 \).
    
\end{ans}
\begin{ans}{One.I.1.33}
      Yes.
      This sequence of operations swaps rows \( i \) and \( j \)
      \begin{equation*}
         \grstep{\rho_i+\rho_j}\quad
         \grstep{-\rho_j+\rho_i}\quad
         \grstep{\rho_i+\rho_j}\quad
         \grstep{-1\rho_i}
      \end{equation*}
      so the row-swap operation is redundant in the presence of the other two.
     
\end{ans}
\begin{ans}{One.I.1.34}
      Swapping rows is reversed by swapping back.
      \begin{eqnarray*}
         \begin{linsys}{3}
           a_{1,1}x_1  &+  &\cdots  &+  &a_{1,n}x_n  &=  &d_1  \\
                       &   &        &   &            &\vdots   \\
           a_{m,1}x_1  &+  &\cdots  &+  &a_{m,n}x_n  &=  &d_m
         \end{linsys}
        &\grstep{\rho_i\leftrightarrow\rho_j}\;
        \grstep{\rho_j\leftrightarrow\rho_i}
        &\begin{linsys}{3}
           a_{1,1}x_1  &+  &\cdots  &+  &a_{1,n}x_n  &=  &d_1  \\
                       &   &        &   &            &\vdots   \\
           a_{m,1}x_1  &+  &\cdots  &+  &a_{m,n}x_n  &=  &d_m
         \end{linsys}
      \end{eqnarray*}
      Multiplying both sides of a row by \( k\neq 0  \) is reversed by
      dividing by \( k \).
      \begin{eqnarray*}
         \begin{linsys}{3}
           a_{1,1}x_1  &+  &\cdots  &+  &a_{1,n}x_n  &=  &d_1  \\
                       &   &        &   &            &\vdots   \\
           a_{m,1}x_1  &+  &\cdots  &+  &a_{m,n}x_n  &=  &d_m
         \end{linsys}
        &\grstep{k\rho_i}\;
        \grstep{(1/k)\rho_i}
        &\begin{linsys}{3}
           a_{1,1}x_1  &+  &\cdots  &+  &a_{1,n}x_n  &=  &d_1  \\
                       &   &        &   &            &\vdots   \\
           a_{m,1}x_1  &+  &\cdots  &+  &a_{m,n}x_n  &=  &d_m
         \end{linsys}
      \end{eqnarray*}
      Adding \( k \) times a row to another is reversed by adding \( -k \)
      times that row.
      \begin{eqnarray*}
         \begin{linsys}{3}
           a_{1,1}x_1  &+  &\cdots  &+  &a_{1,n}x_n  &=  &d_1  \\
                       &   &        &   &            &\vdots   \\
           a_{m,1}x_1  &+  &\cdots  &+  &a_{m,n}x_n  &=  &d_m
          \end{linsys}
        &\grstep{k\rho_i+\rho_j}\;
        \grstep{-k\rho_i+\rho_j}
        &\begin{linsys}{3}
           a_{1,1}x_1  &+  &\cdots  &+  &a_{1,n}x_n  &=  &d_1  \\
                       &   &        &   &            &\vdots   \\
           a_{m,1}x_1  &+  &\cdots  &+  &a_{m,n}x_n  &=  &d_m
        \end{linsys}
      \end{eqnarray*}

       Remark:~observe for the third case that if we were to allow
       \( i=j \) then the result wouldn't hold.
       \begin{equation*}
         \begin{linsys}{2}
           3x  &+  &2y  &=  &7
         \end{linsys}
         \;\grstep{2\rho_1+\rho_1}\;
         \begin{linsys}{2}
           9x  &+  &6y  &=  &21
          \end{linsys}
         \;\grstep{-2\rho_1+\rho_1}\;
         \begin{linsys}{2}
          -9x  &-  &6y  &=  &-21
         \end{linsys}
       \end{equation*}
    
\end{ans}
\begin{ans}{One.I.1.35}
      Let \( p \), \( n \), and \( d \) be the number of
      pennies, nickels, and dimes.
      For variables that are real numbers, this system
      \begin{eqnarray*}
         \begin{linsys}{3}
              p  &+ &n   &+  &d   &=  &13   \\
              p  &+ &5n  &+  &10d &=  &83
         \end{linsys}
         &\grstep{-\rho_1+\rho_2}
         &\begin{linsys}{3}
              p  &+ &n   &+  &d   &=  &13   \\
                 &  &4n  &+  &9d  &=  &70
          \end{linsys}
      \end{eqnarray*}
      has more than one solution, in fact, infinitely many of them.
      However, it has a limited number of solutions in which \( p \), \( n \),
      and \( d \) are non-negative integers.
      Running through \( d=0 \), \ldots, \( d=8 \) shows that
      \( (p,n,d)=(3,4,6) \)
      is the only solution using natural numbers.
    
\end{ans}
\begin{ans}{One.I.1.36}
      Solving the system
      \begin{equation*}
        \begin{linsys}{2}
        (1/3)(a+b+c)  &+  &d  &=  &29  \\
        (1/3)(b+c+d)  &+  &a  &=  &23  \\
        (1/3)(c+d+a)  &+  &b  &=  &21  \\
        (1/3)(d+a+b)  &+  &c  &=  &17
        \end{linsys}
      \end{equation*}
      we obtain $a=12$, $b=9$, $c=3$, $d=21$.
      Thus the second item, 21, is the correct answer.
     
\end{ans}
\begin{ans}{One.I.1.37}
        \answerasgiven
        A comparison of the units and hundreds columns of this
        addition shows that there must be a carry from the tens column.
        The tens column then tells us that \( A<H \), so there
        can be no carry from the units or hundreds columns.
        The five columns then give the following five equations.
        \begin{align*}
          A+E  &=  W  \\
          2H   &=  A+10  \\
          H    &=  W+1  \\
          H+T  &=  E+10  \\
          A+1  &=  T
        \end{align*}
        The five linear equations in five unknowns, if solved simultaneously,
        produce the unique solution: \( A=4 \), \( T=5 \), \( H=7 \),
        \( W=6 \) and \( E=2 \), so that the original example in addition
        was \( 47474+5272=52746 \).
      
\end{ans}
\begin{ans}{One.I.1.38}
       \answerasgiven
       Eight commissioners voted for $B$.
       To see this, we will use the given information to study how many voters
       chose each order of $A$, $B$, $C$.

       The six orders of preference are $ABC$, $ACB$, $BAC$, $BCA$, $CAB$,
       $CBA$; assume they receive $a$, $b$, $c$, $d$, $e$, $f$ votes
       respectively.
       We know that
       \begin{equation*}
         \begin{linsys}{3}
           a  &+  &b  &+  &e  &=  &11  \\
           d  &+  &e  &+  &f  &=  &12  \\
           a  &+  &c  &+  &d  &=  &14
         \end{linsys}
       \end{equation*}
       from the number preferring $A$ over $B$, the number preferring
       $C$ over $A$, and the number preferring $B$ over $C$.
       Because 20 votes were cast, we also know that
       \begin{equation*}
         \begin{linsys}{3}
           c  &+  &d  &+  &f  &=  &9  \\
           a  &+  &b  &+  &c  &=  &8  \\
           b  &+  &e  &+  &f  &=  &6
         \end{linsys}
       \end{equation*}
       from the preferences for $B$ over $A$, for $A$ over $C$, and for
       $C$ over $B$.

       The solution is $a=6$, $b=1$, $c=1$, $d=7$, $e=4$, and $f=1$.
       The number of commissioners voting for $B$ as their first choice
       is therefore $c+d=1+7=8$.

       \par\noindent {\em Comments.}
       The answer to this question would have been the same had we known only
       that {\em at least\/} 14 commissioners preferred $B$ over $C$.

       The seemingly paradoxical nature of the commissioner's preferences
       ($A$ is preferred to $B$, and $B$ is preferred to $C$, and $C$ is
       preferred to $A$), an example of ``non-transitive dominance'', is not
       uncommon when individual choices are pooled.
     
\end{ans}
\begin{ans}{One.I.1.39}
       \answerasgiven
       \textit{We have not used ``dependent'' yet;
       it means here that Gauss'
       method shows that there is not a unique solution.}
       If \( n\geq 3 \) the system is dependent and the solution is not
       unique.
       Hence \( n<3 \).
       But the term ``system'' implies \( n>1 \).
       Hence \( n=2 \).
       If the equations are
       \begin{equation*}
         \begin{linsys}{2}
              ax  &+ &(a+d)y  &=  &a+2d  \\
         (a+3d)x  &+ &(a+4d)y &=  &a+5d
         \end{linsys}
       \end{equation*}
       then \( x=-1 \), \( y=2 \).
    
\end{ans}
\subsection{Subsection One.I.2: Linear Systems}
\begin{ans}{One.I.2.15}
      \begin{exparts*}
        \partsitem \( 2 \)
        \partsitem \( 3 \)
        \partsitem \(-1 \)
        \partsitem Not defined.
      \end{exparts*}
    
\end{ans}
\begin{ans}{One.I.2.16}
      \begin{exparts*}
        \partsitem \( \nbym{2}{3} \)
        \partsitem \( \nbym{3}{2} \)
        \partsitem \( \nbym{2}{2} \)
      \end{exparts*}
    
\end{ans}
\begin{ans}{One.I.2.17}
      \begin{exparts*}
        \partsitem \( \colvec[r]{5 \\ 1 \\ 5} \)
        \partsitem \( \colvec[r]{20 \\ -5} \)
        \partsitem \( \colvec[r]{-2 \\ 4 \\ 0} \)
        \partsitem \( \colvec[r]{41 \\ 52} \)
        \partsitem Not defined.
        \partsitem \( \colvec[r]{12 \\ 8 \\ 4} \)
      \end{exparts*}
     
\end{ans}
\begin{ans}{One.I.2.18}
      \begin{exparts}
        \partsitem This reduction
          \begin{eqnarray*}
            \begin{amat}[r]{2}
              3  &6  &18 \\
              1  &2  &6
            \end{amat}
            &\grstep{(-1/3)\rho_1+\rho_2}
            &\begin{amat}[r]{2}
              3  &6  &18 \\
              0  &0  &0
            \end{amat}
          \end{eqnarray*}
          leaves \( x \) leading and \( y \) free.
          Making \( y \) the parameter, we have \( x=6-2y \) so the solution
          set is
          \begin{equation*}
            \set{\colvec[r]{6 \\ 0}+\colvec[r]{-2 \\ 1}y
              \suchthat y\in\Re}.
          \end{equation*}
        \partsitem This reduction
          \begin{eqnarray*}
            \begin{amat}[r]{2}
              1  &1  &1  \\
              1  &-1 &-1
            \end{amat}
            &\grstep{-\rho_1+\rho_2}
            &\begin{amat}[r]{2}
              1  &1  &1  \\
              0  &-2 &-2
            \end{amat}
          \end{eqnarray*}
          gives the unique solution \( y=1 \), \( x=0 \).
          The solution set is
          \begin{equation*}
            \set{\colvec[r]{0 \\ 1} }.
          \end{equation*}
        \partsitem This use of Gauss' method
          \begin{equation*}
            \begin{amat}[r]{3}
              1  &0  &1  &4  \\
              1  &-1 &2  &5  \\
              4  &-1 &5  &17
            \end{amat}
            \;\grstep[-4\rho_1+\rho_3]{-\rho_1+\rho_2}\;
            \begin{amat}[r]{3}
              1  &0  &1  &4  \\
              0  &-1 &1  &1  \\
              0  &-1 &1  &1
            \end{amat}
            \;\grstep{-\rho_2+\rho_3}\;
            \begin{amat}[r]{3}
              1  &0  &1  &4  \\
              0  &-1 &1  &1  \\
              0  &0  &0  &0
            \end{amat}
          \end{equation*}
          leaves \( x_1 \) and \( x_2 \) leading with \( x_3 \) free.
          The solution set is
          \begin{equation*}
            \set{\colvec[r]{4 \\ -1 \\ 0}+\colvec[r]{-1 \\ 1 \\ 1}x_3
              \suchthat x_3\in\Re}.
          \end{equation*}
        \partsitem This reduction
          \begin{equation*}
            \begin{amat}[r]{3}
              2  &1  &-1 &2  \\
              2  &0  &1  &3  \\
              1  &-1 &0  &0
            \end{amat}
            \;\grstep[-(1/2)\rho_1+\rho_3]{-\rho_1+\rho_2}\;
            \begin{amat}[r]{3}
              2  &1    &-1   &2  \\
              0  &-1   &2    &1  \\
              0  &-3/2 &1/2  &-1
            \end{amat}
            \;\grstep{(-3/2)\rho_2+\rho_3}\;
            \begin{amat}[r]{3}
              2  &1  &-1   &2  \\
              0  &-1 &2    &1  \\
              0  &0  &-5/2 &-5/2
            \end{amat}
          \end{equation*}
          shows that the solution set is a singleton set.
          \begin{equation*}
            \set{\colvec[r]{1 \\ 1 \\ 1}}
          \end{equation*}
        \partsitem This reduction is easy
          \begin{equation*}
            \begin{amat}[r]{4}
              1  &2  &-1 &0  &3 \\
              2  &1  &0  &1  &4 \\
              1  &-1 &1  &1  &1
            \end{amat}
            \;\grstep[-\rho_1+\rho_3]{-2\rho_1+\rho_2}\;
            \begin{amat}[r]{4}
              1  &2  &-1 &0  &3  \\
              0  &-3 &2  &1  &-2 \\
              0  &-3 &2  &1  &-2
            \end{amat}
            \;\grstep{-\rho_2+\rho_3}\;
            \begin{amat}[r]{4}
              1  &2  &-1 &0  &3  \\
              0  &-3 &2  &1  &-2 \\
              0  &0  &0  &0  &0
            \end{amat}
          \end{equation*}
          and ends with \( x \) and $y$ leading, while \( z \) and \( w \) are
          free.
          Solving for \( y \) gives \( y=(2+2z+w)/3 \) and substitution shows
          that \( x+2(2+2z+w)/3-z=3 \) so \( x=(5/3)-(1/3)z-(2/3)w \),
          making the solution set
          \begin{equation*}
            \set{\colvec[r]{5/3 \\ 2/3 \\ 0 \\ 0}
                 +\colvec[r]{-1/3 \\ 2/3 \\ 1 \\ 0}z
                 +\colvec[r]{-2/3 \\ 1/3 \\ 0 \\ 1}w
                 \suchthat z,w\in\Re}.
          \end{equation*}
        \partsitem The reduction
          \begin{equation*}
            \begin{amat}[r]{4}
              1  &0  &1  &1  &4 \\
              2  &1  &0  &-1 &2 \\
              3  &1  &1  &0  &7
            \end{amat}
            \;\grstep[-3\rho_1+\rho_3]{-2\rho_1+\rho_2}\;
            \begin{amat}[r]{4}
              1  &0  &1  &1  &4 \\
              0  &1  &-2 &-3 &-6\\
              0  &1  &-2 &-3 &-5
            \end{amat}
            \;\grstep{-\rho_2+\rho_3}\;
            \begin{amat}[r]{4}
              1  &0  &1  &1  &4 \\
              0  &1  &-2 &-3 &-6\\
              0  &0  &0  &0  &1
            \end{amat}
          \end{equation*}
          shows that there is no solution\Dash the solution set is empty.
      \end{exparts}
     
\end{ans}
\begin{ans}{One.I.2.19}
      \begin{exparts}
      \partsitem This reduction
        \begin{eqnarray*}
          \begin{amat}[r]{3}
            2  &1  &-1  &1  \\
            4  &-1 &0   &3
          \end{amat}
          &\grstep{-2\rho_1+\rho_2}
          &\begin{amat}[r]{3}
            2  &1  &-1  &1  \\
            0  &-3 &2   &1
          \end{amat}
        \end{eqnarray*}
        ends with \( x \) and \( y \) leading while \( z \) is free.
        Solving for \( y \) gives \( y=(1-2z)/(-3) \), and then substitution
        \( 2x+(1-2z)/(-3)-z=1 \) shows that \( x=((4/3)+(1/3)z)/2 \).
        Hence the solution set is
        \begin{equation*}
          \set{\colvec[r]{2/3 \\ -1/3 \\ 0}
               +\colvec[r]{1/6 \\ 2/3 \\ 1}z
              \suchthat z\in\Re}.
        \end{equation*}
      \partsitem This application of Gauss' method
        \begin{equation*}
          \begin{amat}[r]{4}
            1  &0  &-1  &0  &1 \\
            0  &1  &2   &-1 &3 \\
            1  &2  &3   &-1 &7
          \end{amat}
          \;\grstep{-\rho_1+\rho_3}\;
          \begin{amat}[r]{4}
            1  &0  &-1  &0  &1 \\
            0  &1  &2   &-1 &3 \\
            0  &2  &4   &-1 &6
          \end{amat}
          \;\grstep{-2\rho_2+\rho_3}\;
          \begin{amat}[r]{4}
            1  &0  &-1  &0  &1 \\
            0  &1  &2   &-1 &3 \\
            0  &0  &0   &1  &0
          \end{amat}
        \end{equation*}
        leaves  \( x \), \( y \), and \( w \)  leading.
        The solution set is
        \begin{equation*}
          \set{\colvec[r]{1 \\ 3 \\ 0 \\ 0}
               +\colvec[r]{1 \\ -2 \\ 1 \\ 0}z
              \suchthat z\in\Re}.
        \end{equation*}
      \partsitem This row reduction
        \begin{equation*}
          \begin{amat}[r]{4}
            1  &-1 &1   &0  &0 \\
            0  &1  &0   &1  &0 \\
            3  &-2 &3   &1  &0 \\
            0  &-1 &0   &-1 &0
          \end{amat}
          \;\grstep{-3\rho_1+\rho_3}\;
          \begin{amat}[r]{4}
            1  &-1 &1   &0  &0 \\
            0  &1  &0   &1  &0 \\
            0  &1  &0   &1  &0 \\
            0  &-1 &0   &-1 &0
          \end{amat}
          \;\grstep[\rho_2+\rho_4]{-\rho_2+\rho_3}\;
          \begin{amat}[r]{4}
            1  &-1 &1   &0  &0 \\
            0  &1  &0   &1  &0 \\
            0  &0  &0   &0  &0 \\
            0  &0  &0   &0  &0
          \end{amat}
        \end{equation*}
        ends with \( z \) and \( w \) free.
        The solution set is
        \begin{equation*}
          \set{\colvec[r]{0 \\ 0 \\ 0 \\ 0}
               +\colvec[r]{-1 \\ 0 \\ 1 \\ 0}z
               +\colvec[r]{-1 \\ -1 \\ 0 \\ 1}w
              \suchthat z,w\in\Re}.
        \end{equation*}
      \partsitem Gauss' method done in this way
        \begin{eqnarray*}
          \begin{amat}[r]{5}
            1  &2  &3   &1  &-1 &1  \\
            3  &-1 &1   &1  &1  &3
          \end{amat}
          &\grstep{-3\rho_1+\rho_2}
          &\begin{amat}[r]{5}
            1  &2  &3   &1  &-1 &1  \\
            0  &-7 &-8  &-2 &4  &0
          \end{amat}
        \end{eqnarray*}
        ends with \( c \), \( d \), and \( e \) free.
        Solving for \( b \) shows that \( b=(8c+2d-4e)/(-7) \) and then
        substitution
        \( a+2(8c+2d-4e)/(-7)+3c+1d-1e=1 \) shows that
        \( a=1-(5/7)c-(3/7)d-(1/7)e \) and so the solution set is
        \begin{equation*}
          \set{\colvec[r]{1 \\ 0 \\ 0 \\ 0 \\ 0}
               +\colvec[r]{-5/7 \\ -8/7 \\ 1 \\ 0 \\ 0}c
               +\colvec[r]{-3/7 \\ -2/7 \\ 0 \\ 1 \\ 0}d
               +\colvec[r]{-1/7 \\ 4/7 \\ 0 \\ 0 \\ 1}e
              \suchthat c,d,e\in\Re}.
        \end{equation*}
    \end{exparts}
   
\end{ans}
\begin{ans}{One.I.2.20}
      For each problem we get a system of linear equations by looking at the
      equations of components.
      \begin{exparts}
       \partsitem $k=5$
       \partsitem The second components show that $i=2$, the third
       components show that $j=1$.
       \partsitem $m=-4$, $n=2$
      \end{exparts}
    
\end{ans}
\begin{ans}{One.I.2.21}
      For each problem we get a system of linear equations by looking at the
      equations of components.
      \begin{exparts}
        \partsitem Yes; take $k=-1/2$.
        \partsitem No; the system with equations $5=5\cdot j$ and
            $4=-4\cdot j$ has no solution.
        \partsitem Yes; take $r=2$.
        \partsitem No.
           The second components give $k=0$.
           Then the third components give $j=1$.
           But the first components don't check.
      \end{exparts}
     
\end{ans}
\begin{ans}{One.I.2.22}
      \begin{exparts}
        \item Let $c$ be the number of acres of corn, $s$ be the number of
          acres of soy, and $a$ be the number of acres of oats.
          \begin{eqnarray*}
            \begin{linsys}{3}
              c   &+   &s   &+   &a   &=   &1200 \\
            20c   &+   &50s &+   &12a &=   &40\,000
            \end{linsys}
            &\grstep{-20\rho_1+\rho_2}
            &\begin{linsys}{3}
              c   &+   &s   &+   &a   &=   &1200 \\
                  &    &30s &-   &8a  &=   &16\,000
            \end{linsys}
          \end{eqnarray*}
          To describe the solution set we can parametrize using $a$.
          \begin{equation*}
            \set{\colvec{c \\ s \\ a}
                 =\colvec{20\,000/30 \\ 16\,000/30 \\ 0}
                  +\colvec{8/30 \\ -38/30 \\ 1}a
                 \suchthat a\in\Re}
          \end{equation*}
        \item There are many answers possible here.
          For instance we can take $a=0$ to get $c=20\,000/30\approx 666.66$ and
          $s=16000/30\approx 533.33$.
          Another example is to take $a=20\,000/38\approx 526.32$, giving
          $c=0$ and $s=7360/38\approx 193.68$.
        \item Plug your answers from the prior part into
          $100c+300s+80a$.
      \end{exparts}
    
\end{ans}
\begin{ans}{One.I.2.23}
      This system has \( 1 \) equation.
      The leading variable is \( x_1 \), the other variables are free.
      \begin{equation*}
        \set{\colvec[r]{-1 \\ 1 \\ \vdotswithin{-1} \\ 0}x_2
             +\cdots+
             \colvec[r]{-1 \\ 0 \\ \vdotswithin{-1} \\ 1}x_n
             \suchthat x_2,\ldots,x_n\in\Re}
      \end{equation*}
     
\end{ans}
\begin{ans}{One.I.2.24}
      \begin{exparts}
        \partsitem Gauss' method here gives
          \begin{eqnarray*}
            \begin{amat}{4}
              1  &2  &0  &-1  &a  \\
              2  &0  &1  &0   &b  \\
              1  &1  &0  &2   &c
            \end{amat}
            &\grstep[-\rho_1+\rho_3]{-2\rho_1+\rho_2}
            &\begin{amat}{4}
              1  &2  &0  &-1  &a  \\
              0  &-4 &1  &2   &-2a+b  \\
              0  &-1 &0  &3   &-a+c
            \end{amat}                                  \\
            &\grstep{-(1/4)\rho_2+\rho_3}
            &\begin{amat}{4}
              1  &2  &0    &-1  &a  \\
              0  &-4 &1    &2   &-2a+b  \\
              0  &0  &-1/4 &5/2 &-(1/2)a-(1/4)b+c
            \end{amat},
          \end{eqnarray*}
          leaving \( w \) free.
          Solve: \(  z=2a+b-4c+10w \),
          and \( -4y=-2a+b-(2a+b-4c+10w)-2w \) so
          \( y=a-c+3w \), and
          \( x=a-2(a-c+3w)+w=-a+2c-5w. \)
          Therefore the solution set is this.
          \begin{equation*}
             \set{\colvec{-a+2c \\ a-c \\ 2a+b-4c \\ 0}
                  +\colvec{-5 \\ 3 \\ 10 \\ 1}w
                  \suchthat w\in\Re}
          \end{equation*}
        \partsitem Plug in with \( a=3 \), \( b=1 \), and \( c=-2 \).
          \begin{equation*}
             \set{\colvec[r]{-7 \\ 5 \\ 15 \\ 0}
                  +\colvec[r]{-5 \\ 3 \\ 10 \\ 1}w
                  \suchthat w\in\Re}
          \end{equation*}
      \end{exparts}
     
\end{ans}
\begin{ans}{One.I.2.25}
       Leaving the comma out, say by writing \( a_{123} \),
       is ambiguous because it could mean $a_{1,23}$ or $a_{12,3}$.
    
\end{ans}
\begin{ans}{One.I.2.26}
      \begin{exparts*}
        \partsitem \(
           \begin{mat}[r]
             2  &3  &4  &5  \\
             3  &4  &5  &6  \\
             4  &5  &6  &7  \\
             5  &6  &7  &8
           \end{mat} \)
        \partsitem \(
           \begin{mat}[r]
             1  &-1  &1   &-1  \\
            -1  &1   &-1  &1  \\
             1  &-1  &1   &-1  \\
            -1  &1   &-1  &1
           \end{mat} \)
      \end{exparts*}
    
\end{ans}
\begin{ans}{One.I.2.27}
      \begin{exparts*}
        \partsitem \( \begin{mat}[r]
                   1  &4  \\
                   2  &5  \\
                   3  &6
                 \end{mat}  \)
        \partsitem \( \begin{mat}[r]
                   2  &1  \\
                  -3  &1
                 \end{mat}  \)
        \partsitem \( \begin{mat}[r]
                   5  &10 \\
                  10  &5
                 \end{mat}  \)
        \partsitem \( \rowvec{1 &1 &0}  \)
      \end{exparts*}
     
\end{ans}
\begin{ans}{One.I.2.28}
      \begin{exparts}
        \partsitem Plugging in \( x=1 \) and \( x=-1 \) gives
          \begin{eqnarray*}
            \begin{linsys}{3}
              a  &+  &b   &+  &c  &=  &2  \\
              a  &-  &b   &+  &c  &=  &6
            \end{linsys}
            &\grstep{-\rho_1+\rho_2}
            &\begin{linsys}{3}
              a  &+  &b   &+  &c  &=  &2  \\
                 &   &-2b &   &   &=  &4
              \end{linsys}
          \end{eqnarray*}
          so the set of functions is
          \( \set{f(x)=(4-c)x^2-2x+c\suchthat c\in\Re} \).
        \partsitem Putting in \( x=1 \) gives
          \begin{equation*}
            \begin{linsys}{3}
              a  &+  &b   &+  &c  &=  &2
            \end{linsys}
          \end{equation*}
          so the set of functions is
          \( \set{f(x)=(2-b-c)x^2+bx+c\suchthat b,c\in\Re} \).
      \end{exparts}
    
\end{ans}
\begin{ans}{One.I.2.29}
      On plugging in the five pairs $(x,y)$ we get a system with the
      five equations and six unknowns $a$, \ldots, $f$.
      Because there are more unknowns than equations, if no inconsistency
      exists among the equations then there are infinitely many solutions
      (at least one variable will end up free).

      But no inconsistency can exist because $a=0$, \ldots, $f=0$ is a
      solution (we are only using this zero solution to show that the system
      is consistent\Dash the prior paragraph shows that
      there are nonzero solutions).
    
\end{ans}
\begin{ans}{One.I.2.30}
      \begin{exparts}
      \partsitem Here is one\Dash the fourth equation is redundant
        but still OK.
        \begin{equation*}
          \begin{linsys}{4}
             x  &+  &y  &-  &z  &+  &w  &=  &0  \\
                &   &y  &-  &z  &   &   &=  &0  \\
                &   &   &   &2z &+  &2w &=  &0  \\
                &   &   &   &z  &+  &w  &=  &0
          \end{linsys}
        \end{equation*}
      \partsitem Here is one.
        \begin{equation*}
          \begin{linsys}{4}
             x  &+  &y  &-  &z  &+  &w  &=  &0  \\
                &   &   &   &   &   &w  &=  &0  \\
                &   &   &   &   &   &w  &=  &0  \\
                &   &   &   &   &   &w  &=  &0
          \end{linsys}
        \end{equation*}
      \partsitem This is one.
        \begin{equation*}
          \begin{linsys}{4}
             x  &+  &y  &-  &z  &+  &w  &=  &0  \\
             x  &+  &y  &-  &z  &+  &w  &=  &0  \\
             x  &+  &y  &-  &z  &+  &w  &=  &0  \\
             x  &+  &y  &-  &z  &+  &w  &=  &0
          \end{linsys}
        \end{equation*}
    \end{exparts}
   
\end{ans}
\begin{ans}{One.I.2.31}
       \answerasgiven
       My solution was to define the numbers  of  arbuzoids
       as $3$-dimensional vectors, and express all possible
       elementary transitions as such vectors, too:
       \begin{center}
         \begin{tabular}{rr}
           R: &$13$  \\
           G: &$15$  \\
           B: &$17$
         \end{tabular}
         \qquad
         Operations:
         $\colvec[r]{-1 \\ -1 \\ 2}$,
         $\colvec[r]{-1 \\ 2 \\ -1}$,
         and
         $\colvec[r]{2 \\ -1 \\ -1}$
       \end{center}
       Now, it is enough to check whether the  solution  to
       one  of  the  following  systems of linear equations
       exists:
       \begin{equation*}
         \colvec[r]{13 \\ 15 \\ 17}
         +x\colvec[r]{-1 \\ -1  \\ 2}
         +y\colvec[r]{-1 \\ 2 \\ -1}
         +\colvec[r]{2 \\ -1 \\ -1}
         =\colvec[r]{0 \\ 0 \\ 45}
         \qquad
         \text{(or $\colvec[r]{0 \\ 45 \\ 0}$ or $\colvec[r]{45 \\ 0 \\ 0}$)}
       \end{equation*}
       Solving
       \begin{eqnarray*}
         \begin{amat}[r]{3}
          -1  &-1 &2  &-13  \\
          -1  &2  &-1 &-15  \\
           2  &-1 &-1 &28
         \end{amat}
         &\grstep[2\rho_1+\rho_3]{-\rho_1+\rho_2}
         \;\grstep{\rho_2+\rho_3}
         &\begin{amat}[r]{3}
          -1  &-1 &2  &-13  \\
           0  &3  &-3 &-2  \\
           0  &0  &0  &0
         \end{amat}
       \end{eqnarray*}
       gives $y+2/3=z$ so if the number of transformations $z$ is an integer
       then $y$ is not.
       The other two systems give similar conclusions so there is no
       solution.
    
\end{ans}
\begin{ans}{One.I.2.32}
       \answerasgiven
       \begin{exparts}
        \partsitem Formal solution of the system yields
          \begin{equation*}
            x=\frac{a^3-1}{a^2-1}
            \qquad
            y=\frac{-a^2+a}{a^2-1}.
          \end{equation*}
          If $a+1\neq 0$ and $a-1\neq 0$, then the system has the single
          solution
          \begin{equation*}
            x=\frac{a^2+a+1}{a+1}
            \qquad
            y=\frac{-a}{a+1}.
          \end{equation*}
          If $a=-1$, or if $a=+1$, then the formulas are meaningless; in the
          first instance we arrive at the system
          \begin{equation*}
            \left\{
            \begin{linsys}{2}
              -x &+  &y  &=  &1 \\
               x &-  &y  &=  &1
            \end{linsys}\right.
          \end{equation*}
          which is a contradictory system.
          In the second instance we have
          \begin{equation*}
            \left\{
            \begin{linsys}{2}
               x &+  &y  &=  &1 \\
               x &+  &y  &=  &1
            \end{linsys}\right.
          \end{equation*}
          which has an infinite number of solutions (for example, for
          $x$ arbitrary, $y=1-x$).
        \partsitem Solution of the system yields
          \begin{equation*}
            x=\frac{a^4-1}{a^2-1}
            \qquad
            y=\frac{-a^3+a}{a^2-1}.
          \end{equation*}
          Here, is $a^2-1\neq 0$, the system has the single solution
          $x=a^2+1$, $y=-a$.
          For $a=-1$ and $a=1$, we obtain the systems
          \begin{equation*}
            \left\{
            \begin{linsys}{2}
              -x &+  &y  &=  &-1 \\
               x &-  &y  &=  &1
            \end{linsys}\right.
            \qquad
            \left\{
            \begin{linsys}{2}
               x &+  &y  &=  &1 \\
               x &+  &y  &=  &1
            \end{linsys}\right.
         \end{equation*}
         both of which have an infinite number of solutions.
      \end{exparts}
    
\end{ans}
\begin{ans}{One.I.2.33}
      \answerasgiven
      Let \( u \), \( v \), \( x \), \( y \), \( z \) be the volumes in
      \( {\rm cm}^3 \) of Al, Cu, Pb, Ag, and Au, respectively, contained in
      the sphere, which we assume to be not hollow.
      Since the loss of weight in water (specific gravity \( 1.00 \)) is
      \( 1000 \) grams, the volume of the sphere is \( 1000\mbox{ cm}^3 \).
      Then the data, some of which is superfluous, though consistent, leads to
      only \( 2 \) independent equations, one relating volumes and the
      other, weights.
      \begin{equation*}
        \begin{linsys}{5}
           u  &+  &v    &+  &x     &+  &y     &+  &z     &=  &1000  \\
        2.7u  &+  &8.9v &+  &11.3x &+  &10.5y &+  &19.3z &=  &7558
        \end{linsys}
      \end{equation*}
      Clearly the sphere must contain some aluminum to bring its mean specific
      gravity below the specific gravities of all the other metals.
      There is no unique result to this part of the problem, for the amounts
      of three metals may be chosen arbitrarily, provided that the choices
      will not result in negative amounts of any metal.

      If the ball contains only aluminum and gold, there are
      \( 294.5\mbox{ cm}^3 \) of gold and \( 705.5\mbox{ cm}^3 \) of aluminum.
      Another possibility is \( 124.7\mbox{ cm}^3 \) each of Cu, Au, Pb, and
      Ag and \( 501.2\mbox{ cm}^3  \) of Al.
    
\end{ans}
\subsection{Subsection One.I.3: Linear Systems}
\begin{ans}{One.I.3.15}
      For the arithmetic to these, see the answers from the prior
      subsection.
      \begin{exparts}
        \partsitem
          The solution set is
          \begin{equation*}
            \set{\colvec[r]{6 \\ 0}+\colvec[r]{-2 \\ 1}y
              \suchthat y\in\Re}.
          \end{equation*}
          Here the particular solution and the solution set for the associated
          homogeneous system are
          \begin{equation*}
            \colvec[r]{6 \\ 0}
              \quad\text{and}\quad
            \set{\colvec[r]{-2 \\ 1}y
              \suchthat y\in\Re}.
          \end{equation*}
        \partsitem
          The solution set is
          \begin{equation*}
            \set{\colvec[r]{0 \\ 1} }.
          \end{equation*}
          The particular solution and the solution set for the associated
          homogeneous system are
          \begin{equation*}
            \colvec[r]{0 \\ 1}
              \quad\text{and}\quad
            \set{\colvec[r]{0 \\ 0} }
          \end{equation*}
        \partsitem
          The solution set is
          \begin{equation*}
            \set{\colvec[r]{4 \\ -1 \\ 0}+\colvec[r]{-1 \\ 1 \\ 1}x_3
              \suchthat x_3\in\Re}.
          \end{equation*}
          A particular solution and the solution set for the associated
          homogeneous system are
          \begin{equation*}
            \colvec[r]{4 \\ -1 \\ 0}
              \quad\text{and}\quad
            \set{\colvec[r]{-1 \\ 1 \\ 1}x_3
              \suchthat x_3\in\Re}.
          \end{equation*}
        \partsitem
          The solution set is a singleton
          \begin{equation*}
            \set{\colvec[r]{1 \\ 1 \\ 1}}.
          \end{equation*}
          A particular solution and the solution set for the associated
          homogeneous system are
          \begin{equation*}
            \colvec[r]{1 \\ 1 \\ 1}
              \quad\text{and}\quad
            \set{\colvec[r]{0 \\ 0 \\ 0}t
              \suchthat t\in\Re}.
          \end{equation*}
        \partsitem
          The solution set is
          \begin{equation*}
            \set{\colvec[r]{5/3 \\ 2/3 \\ 0 \\ 0}
                 +\colvec[r]{-1/3 \\ 2/3 \\ 1 \\ 0}z
                 +\colvec[r]{-2/3 \\ 1/3 \\ 0 \\ 1}w
                 \suchthat z,w\in\Re}.
          \end{equation*}
          A particular solution and the solution set for the associated
          homogeneous system are
          \begin{equation*}
            \colvec[r]{5/3 \\ 2/3 \\ 0 \\ 0}
              \quad\text{and}\quad
            \set{\colvec[r]{-1/3 \\ 2/3 \\ 1 \\ 0}z
                 +\colvec[r]{-2/3 \\ 1/3 \\ 0 \\ 1}w
                 \suchthat z,w\in\Re}.
          \end{equation*}
        \partsitem This system's solution set is empty.
          Thus, there is no particular solution.
          The solution set of the associated homogeneous system is
          \begin{equation*}
            \set{\colvec[r]{-1 \\ 2 \\ 1 \\ 0}z
                 +\colvec[r]{-1 \\ 3 \\ 0 \\ 1}w
                 \suchthat z,w\in\Re}.
          \end{equation*}
      \end{exparts}
    
\end{ans}
\begin{ans}{One.I.3.16}
      The answers from the prior subsection show the row operations.
    \begin{exparts}
      \partsitem
        The solution set is
        \begin{equation*}
          \set{\colvec[r]{2/3 \\ -1/3 \\ 0}
               +\colvec[r]{1/6 \\ 2/3 \\ 1}z
              \suchthat z\in\Re}.
        \end{equation*}
        A particular solution and the solution set for the associated
        homogeneous system are
        \begin{equation*}
          \colvec[r]{2/3 \\ -1/3 \\ 0}
            \quad\text{and}\quad
        \set{\colvec[r]{1/6 \\ 2/3 \\ 1}z
            \suchthat z\in\Re}.
        \end{equation*}
      \partsitem
        The solution set is
        \begin{equation*}
          \set{\colvec[r]{1 \\ 3 \\ 0 \\ 0}
               +\colvec[r]{1 \\-2 \\ 1 \\ 0}z
              \suchthat z\in\Re}.
        \end{equation*}
        A particular solution and the solution set for the associated
        homogeneous system are
        \begin{equation*}
          \colvec[r]{1 \\ 3 \\ 0 \\ 0}
            \quad\text{and}\quad
          \set{\colvec[r]{1 \\ -2 \\ 1 \\ 0}z
              \suchthat z\in\Re}.
        \end{equation*}
      \partsitem
        The solution set is
        \begin{equation*}
          \set{\colvec[r]{0 \\ 0 \\ 0 \\ 0}
               +\colvec[r]{-1 \\ 0 \\ 1 \\ 0}z
               +\colvec[r]{-1 \\ -1 \\ 0 \\ 1}w
              \suchthat z,w\in\Re}.
        \end{equation*}
        A particular solution and the solution set for the associated
        homogeneous system are
        \begin{equation*}
          \colvec[r]{0 \\ 0 \\ 0 \\ 0}
            \quad\text{and}\quad
          \set{\colvec[r]{-1 \\ 0 \\ 1 \\ 0}z
               +\colvec[r]{-1 \\ -1 \\ 0 \\ 1}w
              \suchthat z,w\in\Re}.
        \end{equation*}
      \partsitem
        The solution set is
        \begin{equation*}
          \set{\colvec[r]{1 \\ 0 \\ 0 \\ 0 \\ 0}
               +\colvec[r]{-5/7 \\ -8/7 \\ 1 \\ 0 \\ 0}c
               +\colvec[r]{-3/7 \\ -2/7 \\ 0 \\ 1 \\ 0}d
               +\colvec[r]{-1/7 \\ 4/7 \\ 0 \\ 0 \\ 1}e
              \suchthat c,d,e\in\Re}.
        \end{equation*}
        A particular solution and the solution set for the associated
        homogeneous system are
        \begin{equation*}
          \colvec[r]{1 \\ 0 \\ 0 \\ 0 \\ 0}
            \quad\text{and}\quad
          \set{\colvec[r]{-5/7 \\ -8/7 \\ 1 \\ 0 \\ 0}c
               +\colvec[r]{-3/7 \\ -2/7 \\ 0 \\ 1 \\ 0}d
               +\colvec[r]{-1/7 \\ 4/7 \\ 0 \\ 0 \\ 1}e
              \suchthat c,d,e\in\Re}.
        \end{equation*}
    \end{exparts}
   
\end{ans}
\begin{ans}{One.I.3.17}
      Just plug them in and see if they satisfy all three equations.
      \begin{exparts}
        \partsitem No.
        \partsitem Yes.
        \partsitem Yes.
      \end{exparts}
    
\end{ans}
\begin{ans}{One.I.3.18}
      Gauss' method on the associated homogeneous system gives
      \begin{equation*}
        \begin{amat}[r]{4}
           1  &-1  &0  &1  &0  \\
           2  &3   &-1 &0  &0  \\
           0  &1   &1  &1  &0
        \end{amat}
        \;\grstep{-2\rho_1+\rho_2}\;
        \begin{amat}[r]{4}
           1  &-1  &0  &1  &0  \\
           0  &5   &-1 &-2 &0  \\
           0  &1   &1  &1  &0
        \end{amat}
        \;\grstep{-(1/5)\rho_2+\rho_3}\;
        \begin{amat}[r]{4}
           1  &-1  &0  &1  &0  \\
           0  &5   &-1 &-2 &0  \\
           0  &0   &6/5&7/5&0
        \end{amat}
      \end{equation*}
      so this is the solution to the homogeneous problem:
      \begin{equation*}
        \set{\colvec[r]{-5/6 \\ 1/6 \\ -7/6 \\ 1}w\suchthat w\in\Re}.
      \end{equation*}
      \begin{exparts}
        \partsitem That vector is indeed a particular solution, so the required
          general solution is
          \begin{equation*}
            \set{\colvec[r]{0 \\ 0 \\ 0 \\ 4}+
                 \colvec[r]{-5/6 \\ 1/6 \\ -7/6 \\ 1}w\suchthat w\in\Re}.
          \end{equation*}
        \partsitem That vector is a particular solution so the required
          general solution is
          \begin{equation*}
            \set{\colvec[r]{-5 \\ 1 \\ -7 \\ 10}+
                 \colvec[r]{-5/6 \\ 1/6 \\ -7/6 \\ 1}w\suchthat w\in\Re}.
          \end{equation*}
        \partsitem That vector is not a solution of the system since
          it does not satisfy the third equation.
          No such general solution exists.
      \end{exparts}
    
\end{ans}
\begin{ans}{One.I.3.19}
       The first is nonsingular while the second is singular.
       Just do Gauss' method and see if the echelon form result has
       non-$0$ numbers in each entry on the diagonal.
     
\end{ans}
\begin{ans}{One.I.3.20}
      \begin{exparts}
      \partsitem Nonsingular:
        \begin{eqnarray*}
          &\grstep{-\rho_1+\rho_2}
          &\begin{mat}[r]
            1  &2  \\
            0  &1
          \end{mat}
        \end{eqnarray*}
        ends with each row containing a leading entry.
      \partsitem Singular:
        \begin{eqnarray*}
          &\grstep{3\rho_1+\rho_2}
          &\begin{mat}[r]
            1  &2  \\
            0  &0
          \end{mat}
        \end{eqnarray*}
        ends with row \( 2 \) without a leading entry.
      \partsitem Neither.
        A matrix must be square for either word to apply.
      \partsitem Singular.
      \partsitem Nonsingular.
     \end{exparts}
    
\end{ans}
\begin{ans}{One.I.3.21}
        In each case we must decide if the vector is a linear combination
        of the vectors in the set.
        \begin{exparts}
          \partsitem Yes.
            Solve
            \begin{equation*}
              c_1\colvec[r]{1 \\ 4}+c_2\colvec[r]{1 \\ 5}=\colvec[r]{2 \\ 3}
            \end{equation*}
            with
            \begin{eqnarray*}
              \begin{amat}[r]{2}
                1  &1  &2  \\
                4  &5  &3
              \end{amat}
              &\grstep{-4\rho_1+\rho_2}
              &\begin{amat}[r]{2}
                1  &1  &2  \\
                0  &1  &-5
              \end{amat}
            \end{eqnarray*}
            to conclude that there are $c_1$ and $c_2$ giving the combination.
          \partsitem No.
            The reduction
            \begin{equation*}
              \begin{amat}[r]{2}
                2  &1  &-1 \\
                1  &0  &0  \\
                0  &1  &1
              \end{amat}
              \;\grstep{-(1/2)\rho_1+\rho_2}\;
              \begin{amat}[r]{2}
                2  &1     &-1 \\
                0  &-1/2  &1/2  \\
                0  &1     &1
              \end{amat}
              \;\grstep{2\rho_2+\rho_3}\;
              \begin{amat}[r]{2}
                2  &1     &-1 \\
                0  &-1/2  &1/2  \\
                0  &0     &2
              \end{amat}
            \end{equation*}
            shows that
            \begin{equation*}
              c_1\colvec[r]{2 \\ 1 \\ 0}+c_2\colvec[r]{1 \\ 0 \\ 1}
                =\colvec[r]{-1 \\ 0 \\ 1}
            \end{equation*}
            has no solution.
          \partsitem Yes.
            The reduction
            \begin{equation*}
              \begin{amat}[r]{4}
                1  &2  &3  &4  &1  \\
                0  &1  &3  &2  &3  \\
                4  &5  &0  &1  &0
              \end{amat}
              \;\grstep{-4\rho_1+\rho_3}\;
              \begin{amat}[r]{4}
                1  &2  &3  &4  &1  \\
                0  &1  &3  &2  &3  \\
                0  &-3 &-12&-15&-4
              \end{amat}
              \;\grstep{3\rho_2+\rho_3}\;
              \begin{amat}[r]{4}
                1  &2  &3  &4  &1  \\
                0  &1  &3  &2  &3  \\
                0  &0  &-3 &-9 &5
              \end{amat}
            \end{equation*}
            shows that there are infinitely many ways
            \begin{equation*}
              \set{\colvec[r]{c_1 \\ c_2 \\ c_3 \\ c_4}=
                   \colvec[r]{-10 \\ 8 \\ -5/3 \\ 0}+
                   \colvec[r]{-9 \\ 7 \\ -3 \\ 1}c_4
                    \suchthat c_4\in\Re}
            \end{equation*}
            to write
            \begin{equation*}
              \colvec[r]{1 \\ 3 \\ 0}=
              c_1\colvec[r]{1 \\ 0 \\ 4}+
              c_2\colvec[r]{2 \\ 1 \\ 5}+
              c_3\colvec[r]{3 \\ 3 \\ 0}+
              c_4\colvec[r]{4 \\ 2 \\ 1}.
            \end{equation*}
          \partsitem No.
            Look at the third components.
        \end{exparts}
      
\end{ans}
\begin{ans}{One.I.3.22}
        Because the matrix of coefficients is nonsingular, Gauss' method
        ends with an echelon form where each variable leads an equation.
        Back substitution gives a unique solution.

      (Another way to see the solution is unique is to note that
      with a nonsingular matrix of coefficients the associated
      homogeneous system has a unique solution, by definition.
      Since the general solution is the sum of a particular solution with
      each homogeneous solution, the general solution has
      (at most) one element.)
     
\end{ans}
\begin{ans}{One.I.3.23}
      In this case the solution set is all of \( \Re^n \) and can be
      expressed in the required form
      \begin{equation*}
        \set{c_1\colvec[r]{1 \\ 0 \\ \vdotswithin{1} \\ 0}
             +c_2\colvec[r]{0 \\ 1 \\ \vdotswithin{0} \\ 0}
             +\cdots
             +c_n\colvec[r]{0 \\ 0 \\ \vdotswithin{0} \\ 1}
             \suchthat c_1,\ldots,c_n\in\Re}.
      \end{equation*}
     
\end{ans}
\begin{ans}{One.I.3.24}
      Assume \( \vec{s},\vec{t}\in\Re^n \) and write
      \begin{equation*}
        \vec{s}=\colvec{s_1 \\ \vdotswithin{s_1} \\ s_n}
          \quad\mbox{and}\quad
        \vec{t}=\colvec{t_1 \\ \vdotswithin{t_1} \\ t_n}.
      \end{equation*}
      Also let \( a_{i,1}x_1+\cdots+a_{i,n}x_n=0 \) be the \( i \)-th equation
      in the homogeneous system.
      \begin{exparts}
        \partsitem The check is easy:
          \begin{eqnarray*}
            a_{i,1}(s_1+t_1)+\cdots+a_{i,n}(s_n+t_n)
            &=
            &(a_{i,1}s_1+\cdots+a_{i,n}s_n)
            +(a_{i,1}t_1+\cdots+a_{i,n}t_n)         \\
            &=
            &0+0.
          \end{eqnarray*}
        \partsitem This one is similar:
          \begin{equation*}
            a_{i,1}(3s_1)+\cdots+a_{i,n}(3s_n)
            =3(a_{i,1}s_1+\cdots+a_{i,n}s_n)
            =3\cdot 0=0.
          \end{equation*}
        \partsitem This one is not much harder:
          \begin{eqnarray*}
            a_{i,1}(ks_1+mt_1)+\cdots+a_{i,n}(ks_n+mt_n)
            &=
            &k(a_{i,1}s_1+\cdots+a_{i,n}s_n)
            +m(a_{i,1}t_1+\cdots+a_{i,n}t_n)         \\
            &=
            &k\cdot 0+m\cdot 0.
          \end{eqnarray*}
      \end{exparts}
     What is wrong with that argument is that any linear combination of the
     zero vector yields the zero vector again.
   
\end{ans}
\begin{ans}{One.I.3.25}
      First the proof.

      Gauss' method will use only rationals (e.g.,
      \( -(m/n)\rho_i+\rho_j \)).
      Thus the solution set can be expressed using only rational numbers as
      the components of each vector.
      Now the particular solution is all rational.

      There are infinitely many (rational vector) solutions if and only if the
      associated homogeneous system has infinitely many
      (real vector) solutions.
      That's because setting any parameters to be rationals will produce an
      all-rational solution.
   
\end{ans}
