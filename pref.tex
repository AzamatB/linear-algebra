%\documentstyle{mybook}
%\input{latexmac}
%
%\setcounter{chapter}{0}
%\setcounter{section}{0}
%\setcounter{subsection}{0}
%
\thispagestyle{plain}
%\setcounter{page}{1}\pagenumbering{roman}
%\begin{document}
\bigskip
\vspace*{1in}
\noindent{\Huge\bf Preface}
\bigskip
\par\noindent
This book helps students to master the material of a standard 
US undergraduate linear algebra course.

The material is standard in that the topics covered are
Gaussian reduction, 
vector spaces, linear maps,
determinants, and eigenvalues and eigenvectors.
Another standard is book's audience:
sophomores or juniors, usually with a background 
of at least one semester of calculus. 
The help that it gives to students comes from taking a developmental 
approach\Dash 
this book's presentation emphasizes motivation and naturalness, 
driven home by a wide variety of
examples and by extensive and careful exercises.

The developmental approach is the feature that most recommends this book 
so I will say more.
Courses in the beginning of most mathematics programs
focus less on understanding theory and more on correctly
applying formulas and algorithms.
Later courses
ask for mathematical maturity:~the ability to follow different 
types of arguments, 
a familiarity with
the themes that underlie many mathematical investigations such as
elementary set and function facts,
and a capacity for some independent reading and thinking. 
Linear algebra is an ideal spot to work on the transition.
It comes early in a program so that progress made here pays off later,
but also comes late enough that
students are serious about mathematics, often majors and minors.
The material is accessible, coherent, and elegant.
There are a variety of argument styles, including proofs by
contradiction, if and only if statements, and proofs by induction.
And, examples are plentiful.

Helping readers start the transition to being serious students of 
the subject of mathematics itself
means taking the mathematics seriously, so
all of the results in this book are proved.
On the other hand, we cannot
assume that students have already arrived
and so 
in contrast with more abstract texts, 
we give many examples
and they are often quite detailed.

Some linear algebra books
begin with extensive computations of linear systems, 
matrix multiplications, 
and determinants.
Then, when the concepts\Dash 
vector spaces and linear maps\Dash finally appear, 
and definitions and proofs start, often the abrupt change
brings students to a stop.
In this book, while we start with
a computational topic, linear reduction, from the first
we do more than compute.
We do linear systems quickly but completely,
including the proofs needed to justify what we are computing.
Then, with the linear systems work as motivation
and at a point where the study of linear combinations seems natural,
the second chapter starts with the definition of a real vector space.
In the schedule below, this occurs by the end of the third week.

Another example of our emphasis on motivation and naturalness
is that the third chapter on linear maps
does not begin with the definition of homomorphism, 
but with isomorphism.
The definition of isomorphism is easily motivated
by the observation that some spaces are ``just like'' others.
After that,
the next section takes the reasonable step of defining homomorphism by
isolating the operation-preservation idea.
This approach loses mathematical slickness, 
but it is a good trade because it gives to students
a large gain in sensibility.

One aim of our developmental approach is 
to present the material in such a way that students 
can see how the ideas arise, and perhaps can picture themselves
doing the same type of work.

The clearest example of the developmental approach is the
exercises.
A student progresses most while doing the exercises, so the ones
included here have
been selected with great care.
Each problem set ranges from
simple checks to reasonably involved proofs.
Since an instructor usually assigns about a dozen exercises
after each lecture,
each section ends with about twice that many, 
thereby providing a selection.
There are even a few problems that are challenging puzzles
taken from various journals, competitions, or
problems collections. 
(These are marked with a
`\puzzlemark' and 
as part of the fun, the original wording
has been retained as much as possible.)
In total, the exercises are aimed to both build an ability at,
and help students experience the pleasure of, 
\emph{doing} mathematics.


%\vspace*{.5in}
\medskip
\noindent{\bf Applications and computers.}
%\smallskip
The point of view taken here, that students should think of 
linear algebra as about vector spaces
and linear maps, is not taken to the complete exclusion of others.
Applications and computing are important and vital aspects 
of the subject.
Consequently, each of this book's chapters closes with a few 
application or computer-related topics.
Some are:~network flows, the speed and accuracy of
computer linear reductions, Leontief Input/Output analysis,
dimensional analysis, Markov chains, voting paradoxes,
analytic projective geometry, and difference equations.

These topics are brief enough to be done in a day's class 
or to be given as independent projects.
Most simply give a reader
a taste of the subject, discuss how linear algebra comes in,
point to some further reading, and give a few exercises. 
In short, these topics invite 
readers to see for themselves that linear algebra is a tool
that a professional must have. 




\medskip
\noindent{\bf The license.}
This book is freely available.
You can download and read it without restriction.
Class instructors can print copies for
students and charge for those.
See 
\url{http://joshua.smcvt.edu/linearalgebra}
for more license information.

That page also contains the latest version of this book, and the
latest version of the worked answers to every exercise.
Also there, I provide the \LaTeX\ source of the text and some instructors
may wish to add their own material.
If you like, you can send such additions to me and I may possibly incorporate
them into future editions.

I am very glad for bug reports.
I save them and periodically issue updates; people who
contribute in this way are acknowledged in the text's source files.



%\vspace*{.5in}
\medskip
\noindent{\bf For people reading this book on their own.}
%\smallskip
%
\newcommand{\classday}[1]{\textsc{#1}}
\newcommand{\colwidth}{1.35in}
This book's emphasis on motivation and development make it
a good choice for self-study.
But while a professional instructor can judge what pace and topics suit a
class, if you are an independent student 
then you may find some advice helpful.

Here are two timetables for a semester.
The first focuses on core material.
\begin{center}
   \begin{tabular}{r|*{2}{p{\colwidth}}l}
      \textit{week}  
       &\textit{Monday}          
       &\textit{Wednesday}            
       &\textit{Friday}        \\ \hline
       1    &One.I.1         &One.I.1, 2        &One.I.2, 3         \\
       2    &One.I.3         &One.II.1          &One.II.2         \\
       3    &One.III.1, 2    &One.III.2         &Two.I.1         \\
       4    &Two.I.2         &Two.II            &Two.III.1         \\
       5    &Two.III.1, 2    &Two.III.2         &\classday{exam}          \\
       6    &Two.III.2, 3    &Two.III.3         &Three.I.1        \\
       7    &Three.I.2         &Three.II.1          &Three.II.2         \\
       8    &Three.II.2        &Three.II.2          &Three.III.1          \\
       9    &Three.III.1       &Three.III.2         &Three.IV.1, 2       \\
      10    &Three.IV.2, 3, 4  &Three.IV.4          &\classday{exam}          \\
      11    &Three.IV.4, Three.V.1 &Three.V.1, 2        &Four.I.1, 2         \\
      12    &Four.I.3         &Four.II            &Four.II       \\
      13    &Four.III.1       &Five.I             &Five.II.1         \\
      14    &Five.II.2        &Five.II.3          &\classday{review}        
   \end{tabular}
\end{center}
The second timetable is more ambitious.
It supposes that you know One.II, the elements of vectors, 
usually covered in third semester calculus.
\begin{center}
   \begin{tabular}{r|*{2}{p{\colwidth}}l}
      \textit{week}  
         &\textit{Monday}          
         &\textit{Wednesday}            
         &\textit{Friday}        \\ \hline
       1    &One.I.1         &One.I.2           &One.I.3         \\
       2    &One.I.3         &One.III.1, 2      &One.III.2         \\
       3    &Two.I.1         &Two.I.2           &Two.II           \\
       4    &Two.III.1       &Two.III.2         &Two.III.3         \\
       5    &Two.III.4       &Three.I.1           &\classday{exam}          \\
       6    &Three.I.2         &Three.II.1          &Three.II.2         \\
       7    &Three.III.1       &Three.III.2         &Three.IV.1, 2      \\
       8    &Three.IV.2        &Three.IV.3          &Three.IV.4         \\
       9    &Three.V.1         &Three.V.2           &Three.VI.1         \\
      10    &Three.VI.2        &Four.I.1           &\classday{exam}          \\
      11    &Four.I.2         &Four.I.3           &Four.I.4         \\
      12    &Four.II          &Four.II, Four.III.1   &Four.III.2, 3      \\
      13    &Five.II.1, 2     &Five.II.3          &Five.III.1         \\
      14    &Five.III.2       &Five.IV.1, 2       &Five.IV.2         
   \end{tabular}
\end{center} 
In the table of contents
I have marked subsections as optional if
some instructors will pass over them in favor of spending more time elsewhere. 

You might pick one or two topics that appeal to you 
from the end of each chapter.
You'll get more from these
if you have access to computer software that can do any
big calculations.
I recommend \textit{Sage}, freely available 
from \url{http://sagemath.org}.

My main advice is: do many exercises.
I have marked a good sample with \recommendationmark's in the margin.
For all of them, you must justify your answer either with a computation
or with a proof.
Be aware 
that few inexperienced people can write correct proofs.
Try to find someone with training to work with
you on this.

\bigskip
Finally, if I may, a caution for all students, independent or not:~I 
cannot overemphasize how much the statement
that I sometimes hear, ``I understand the material, but it's only 
that I have trouble with the problems''\spacefactor=1000\ %
is mistaken.
Being able to do things with the ideas is their entire point.
The quotes below express this sentiment admirably.
They state what I believe is the key to both the beauty and the power
of mathematics and the sciences in general, 
and of linear algebra in particular; I took the liberty of formatting
them as verse.

\bigskip
\par\noindent\begin{tabular}[t]{@{}l@{}}
  \textit{I know of no better tactic}                     \\
  \textit{\ than the illustration of exciting principles} \\
  \textit{by well-chosen particulars.}                    \\
  \hspace*{1in}\textit{--Stephen Jay Gould}
\end{tabular}

\bigskip
\par\noindent\begin{tabular}[t]{@{}l@{}}   
\textit{If you really wish to learn}                     \\
   \textit{\ then you must mount the machine}  \\ 
   \textit{\ and become acquainted with its tricks} \\
   \textit{by actual trial.}                    \\
   \hspace*{1in}\textit{--Wilbur Wright}
\end{tabular}

\par\ \hfill\begin{tabular}[t]{@{}l@{}}
                       Jim Hef{}feron            \\
                       Mathematics, Saint Michael's College \\ 
                       Colchester, Vermont\ USA 05439  \\     
                       \texttt{http://joshua.smcvt.edu} \\
                       2011-Jan-01
                    \end{tabular}

\vfill
\par\noindent\textit{Author's Note.}
Inventing a good exercise, one that enlightens as well as tests, 
is a creative act, and hard work.
%At least half of the the effort on this text has gone into exercises 
%and solutions.
The inventor deserves recognition.
But for some reason texts have traditionally not given attributions for
questions.
I have changed that here where I was sure of the source.
I would be glad to hear from anyone who can help me to correctly
attribute others of the questions.   
%\end{document}
