% pref.tex  See http://joshua.smcvt.edu/linearalgebra
{\setlength{\parskip}{.7ex}  % note the group-starting open curly
% \bigskip
% \vspace*{1.25in plus .2in minus .1in}
% \noindent{\Huge\bf Preface}
% \vspace*{.4in plus .1in minus .05in}
% \par\noindent
\chapter*{Preface}
This book helps students to master the material of a standard 
US undergraduate first course in Linear Algebra.

The material is standard in that the subjects covered are
Gaussian reduction, 
vector spaces, linear maps,
determinants, and eigenvalues and eigenvectors.
Another standard is book's audience:
sophomores or juniors, usually with a background 
of at least one semester of calculus. 
The help that it gives to students comes from taking a developmental 
approach\Dash 
this book's presentation emphasizes motivation and naturalness, 
using many examples as well as extensive and careful exercises.

The developmental approach is what most recommends this book
so I will elaborate.
Courses at the beginning of a mathematics program
focus less on theory and more on calculating.
Later courses
ask for mathematical maturity:~the ability to follow different 
types of arguments, 
a familiarity with
the themes that underlie many mathematical investigations such as
elementary set and function facts,
and a capacity for some independent reading and thinking.
Some programs have a separate course devoted to developing maturity and
some do not. 
In either case, a Linear Algebra course 
is an ideal spot to work on this transition.
It comes early in a program so that progress made here pays off later
but also comes late enough that
students are serious about mathematics.
The material is accessible, coherent, and elegant.
There are a variety of argument styles, including 
direct proofs, proofs by
contradiction, and proofs by induction.
And, examples are plentiful.

Helping readers start the transition to being serious students of 
mathematics
requires taking the mathematics seriously so
all of the results here are proved.
On the other hand, we cannot
assume that students have already arrived
and so 
in contrast with more advanced texts 
this book is filled with examples,
often quite detailed.

Some books that assume a not-yet-sophisticated reader
begin with extensive computations of linear systems, 
matrix multiplications, 
and determinants.
Then, when 
vector spaces and linear maps finally appear
and definitions and proofs start, the abrupt change
can bring students to an abrupt stop.
While this book begins with
linear reduction, from the start
we do more than compute.
The first chapter
includes proofs showing that linear reduction gives a correct and
complete solution set.
Then, with the linear systems work as motivation
so that the study of linear combinations is natural,
the second chapter starts with the definition of a real vector space.
In the schedule below this happens before the third week.

Another example of this book's emphasis on motivation and naturalness
is that the third chapter on linear maps
does not begin with the definition of homomorphism.
Instead it begins with the definition of isomorphism, which
is natural: students themselves
observe that some spaces are ``the same'' as others.
After that,
the next section takes the reasonable step of 
isolating the operation-preservation idea
to define homomorphism.
This approach loses mathematical slickness 
but it is a good trade because it gives to students
a large gain in sensibility.

A student progresses most in mathematics while doing exercises. 
% The ones here have gotten close attention.
In this book problem sets start with 
simple checks and range up to reasonably involved proofs.
Since instructors usually assign about a dozen exercises
I have tried to put two dozen in each set, 
thereby giving a selection.
There are even a few that are puzzles
taken from various journals, competitions, or
problems collections. 
These are marked with a
`\puzzlemark' and 
as part of the fun I have retained the original wording
as much as possible.

That is, as with the rest of the book 
the exercises are aimed to both build an ability at,
and help students experience the pleasure of, 
\emph{doing} mathematics.
Students should see how the ideas arise and should be able to 
picture themselves doing the same type of work.


%\vspace*{.5in}
\medskip
\noindent{\bf Applications and computers.}
%\smallskip
The point of view taken here, that students should think of 
Linear Algebra as about vector spaces
and linear maps, is not taken to the complete exclusion of others.
Applications and computing are interesting and vital aspects 
of the subject.
Consequently each of this book's chapters closes with a few 
topics in those areas.
They are brief enough that an instructor can do one
in a day's class 
or can assign them as independent or small-group projects.
Most simply give a reader
a taste of the subject, discuss how Linear Algebra comes in,
point to some further reading, and give a few exercises. 
Whether they figure formally in a course or not these help
readers see for themselves that Linear Algebra is a tool
that a professional must master. 




\medskip
\noindent{\bf Availability.}
This book is freely available.
In particular, instructors can print copies for students 
and sell them out of a college bookstore.
See 
\url{http://joshua.smcvt.edu/linearalgebra}
for the license details.
That page also has the latest version, 
exercise answers, beamer slides,
and \LaTeX\ source.

A text is a large and complex project. 
One of the lessons of software
development is that such a project will have errors.
I welcome bug reports and 
% (contributions are acknowledged in the source).
I periodically issue revisions.
My contact information is on the web page.


\newcommand{\classday}[1]{\textsc{#1}}
\newcommand{\colwidth}{1.25in}

%\vspace*{.5in}
\medskip
\noindent{\bf If you are reading this on your own.}
%\smallskip
%
This book's emphasis on motivation and development,
and its availability, make it widely used for self-study.
If you are an independent student then good for you; I admire your industry.
However, you may find some advice helpful.

While an experienced instructor knows what subjects and
pace suit their class, you may find useful a
timetable for a semester.
(This is adapted from one contributed by George Ashline.)
% \begin{center} \small
%    \begin{tabular}{r|*{2}{p{\colwidth}}l}
%       \textit{week}  
%        &\textit{Monday}          
%        &\textit{Wednesday}            
%        &\textit{Friday}        \\ \hline
%        1    &One.I.1         &One.I.1, 2        &One.I.2, 3         \\
%        2    &One.I.3         &One.II.1          &One.II.2         \\
%        3    &One.III.1, 2    &One.III.2         &Two.I.1         \\
%        4    &Two.I.2         &Two.II            &Two.III.1         \\
%        5    &Two.III.1, 2    &Two.III.2         &\classday{exam}          \\
%        6    &Two.III.2, 3    &Two.III.3         &Three.I.1        \\
%        7    &Three.I.2         &Three.II.1          &Three.II.2         \\
%        8    &Three.II.2        &Three.II.2          &Three.III.1          \\
%        9    &Three.III.1       &Three.III.2         &Three.IV.1, 2       \\
%       10    &Three.IV.2, 3, 4  &Three.IV.4          &\classday{exam}        \\
%       11    &Three.IV.4, Three.V.1 &Three.V.1, 2        &Four.I.1, 2       \\
%       12    &Four.I.3         &Four.II            &Four.II       \\
%       13    &Four.III.1       &Five.I             &Five.II.1         \\
%       14    &Five.II.2        &Five.II.3          &\classday{review}        
%    \end{tabular}
% \end{center}
\begin{center}   % George Ashline's
   \begin{tabular}{r|*{2}{p{\colwidth}}l}
      \textit{week}  
       &\textit{Monday}          
       &\textit{Wednesday}            
       &\textit{Friday}        \\ \hline
       1    &One.I.1         &One.I.1, 2        &One.I.2, 3         \\
       2    &One.I.3         &One.III.1          &One.III.2         \\
       3    &Two.I.1         &Two.I.1, 2         &Two.I.2         \\
       4    &Two.II.1         &Two.III.1         &Two.III.2         \\
       5    &Two.III.2        &Two.III.2, 3         &Two.III.3        \\
       6    &\classday{exam}   &Three.I.1         &Three.I.1       \\
       7    &Three.I.2         &Three.I.2          &Three.II.1         \\
       8    &Three.II.1        &Three.II.2          &Three.II.2          \\
       9    &Three.III.1       &Three.III.2         &Three.IV.1, 2       \\
      10    &Three.IV.2, 3   &Three.IV.4          &Three.V.1          \\
      11    &Three.V.1       &Three.V.2            &Four.I.1         \\
      12    &\classday{exam}  &Four.I.2            &Four.III.1       \\
      13    &Five.II.1    &\multicolumn{2}{c}{\classday{--Thanksgiving break--}} \\
      14    &Five.II.1, 2     &Five.II.2          &Five.II.3        
   \end{tabular}
\end{center}
This timetable 
supposes that you already know Section~One.II, the elements of vectors.
Note that in addition to the exams and the final exam that is not shown,
an important part of the above course is that there are required 
take-home problem sets that include proofs.
The computations are important in this course but so are the proofs.

% The second is more ambitious.
% \begin{center} \small
%    \begin{tabular}{r|*{2}{p{\colwidth}}l}
%       \textit{week}  
%          &\textit{Monday}          
%          &\textit{Wednesday}            
%          &\textit{Friday}        \\ \hline
%        1    &One.I.1         &One.I.2           &One.I.3         \\
%        2    &One.I.3         &One.III.1, 2      &One.III.2         \\
%        3    &Two.I.1         &Two.I.2           &Two.II           \\
%        4    &Two.III.1       &Two.III.2         &Two.III.3         \\
%        5    &Two.III.4       &Three.I.1           &\classday{exam}          \\
%        6    &Three.I.2         &Three.II.1          &Three.II.2         \\
%        7    &Three.III.1       &Three.III.2         &Three.IV.1, 2      \\
%        8    &Three.IV.2        &Three.IV.3          &Three.IV.4         \\
%        9    &Three.V.1         &Three.V.2           &Three.VI.1         \\
%       10    &Three.VI.2        &Four.I.1           &\classday{exam}          \\
%       11    &Four.I.2         &Four.I.3           &Four.I.4         \\
%       12    &Four.II          &Four.II, Four.III.1   &Four.III.2, 3      \\
%       13    &Five.II.1, 2     &Five.II.3          &Five.III.1         \\
%       14    &Five.III.2       &Five.IV.1, 2       &Five.IV.2         
%    \end{tabular}
% \end{center} 
In the table of contents
I have marked subsections as optional if
some instructors will pass over them in favor of spending more time elsewhere. 

You might pick one or two topics that appeal to you 
from the end of each chapter.
You'll get more from these
if you have access to software for calculations.
I recommend \textit{Sage}, freely available 
from \url{http://sagemath.org}.

My main advice is: do many exercises.
I have marked a good sample with \recommendationmark's in the margin.
For all of them, you must justify your answer either with a computation
or with a proof.
Be aware that few inexperienced people can write correct proofs;
try to find a knowledgeable person to work with you on these.

\bigskip
Finally, a caution for all students, independent or not:~I 
cannot overemphasize that the 
statement, ``I understand the material but it's only 
that I have trouble with the problems''\spacefactor=1000\ %
shows a misconception.
Being able to do things with the ideas is their entire point.
The quotes below express this sentiment admirably.
They capture the essence of both the beauty and the power
of mathematics and science in general, 
and of Linear Algebra in particular.
(I took the liberty of formatting them as poetry).

\bigskip
\par\noindent\begin{tabular}[t]{@{}l@{}}
  \textit{I know of no better tactic}                     \\
  \textit{\ than the illustration of exciting principles} \\
  \textit{by well-chosen particulars.}                    \\
  \hspace*{1in}\textit{--Stephen Jay Gould}
\end{tabular}

\bigskip
\par\noindent
\begin{tabular}[t]{@{}l@{}}   
\textit{If you really wish to learn}                     \\
   \textit{\ then you must mount the machine}  \\ 
   \textit{\ and become acquainted with its tricks} \\
   \textit{by actual trial.}                    \\
   \hspace*{1in}\textit{--Wilbur Wright}
\end{tabular}

\vspace{3ex}
\par\ \hfill\begin{tabular}[t]{@{}l@{}}
                       Jim Hef{}feron            \\
                       Mathematics, Saint Michael's College \\ 
                       Colchester, Vermont\ USA 05439  \\     
                       \url{http://joshua.smcvt.edu/linearalgebra} \\
                       2012-Feb-29
                    \end{tabular}

\vfill
\par\noindent\textit{Author's Note.}
Inventing a good exercise, one that enlightens as well as tests, 
is a creative act and hard work.
The inventor deserves recognition.
But texts have traditionally not given attributions for
questions.
I have changed that here where I was sure of the source.
I would be glad to hear from anyone who can help me to correctly
attribute others of the questions.   
} % ends the open curly for the parskip from the top of this file
