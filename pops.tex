% Chapter 4, Topic _Linear Algebra_ Jim Hefferon
%  http://joshua.smcvt.edu/linalg.html
%  2001-Jun-12
\topic{Stable Populations}
\index{stable populations|(}
\index{populations, stable|(}
Imagine a reserve park with animals from a species that we are 
trying to protect.
The park doesn't have a fence and so animals cross the boundary, 
both from the inside out and in the other direction.
Every year, $10\%$ of the animals from inside of the park leave, and  
$1\%$ of the animals from the outside 
find their way in.
We can ask if we can find a stable level of population for this park:
is there a population that, once established, will stay constant over time,
with the number of animals leaving equal to  the number of animals entering?

To answer that question, we must first establish the equations.
Let the year $n$ population in the park be $p_n$ and 
in the rest of the world be $r_n$.
\begin{align*}
  p_{n+1} 
  &=.90p_n+.01r_n    \\
  r_{n+1}
  &=.10p_n+.99r_n 
\end{align*}
We can  set this system up as a matrix equation (see the Markov Chain topic).
\begin{equation*}
  \colvec{p_{n+1} \\ r_{n+1}}
  =\begin{pmatrix}
    .90  &.01  \\
    .10  &.99
  \end{pmatrix}
  \colvec{p_{n} \\ r_{n}}
\end{equation*}
Now, ``stable level'' means that $p_{n+1}=p_n$ and $r_{n+1}=r_n$, so that the 
matrix equation $\vec{v}_{n+1}=T\vec{v}_{n}$  becomes $\vec{v}=T\vec{v}$.
We are therefore looking for eigenvectors for $T$ that are associated with
the eigenvalue $\lambda=1$.
The  equation $\zero=(\lambda I-T)\vec{v}=(I-T)\vec{v}$ is
\begin{equation*}
  \begin{pmatrix}
      0.10  &-0.01  \\
      -0.10  &0.01
  \end{pmatrix}
  \colvec{p \\ r}=
  \colvec{0 \\ 0}
\end{equation*}
which gives the eigenspace:~vectors with the restriction that $p=.1r$.
For example,
if we start with a park population $p=10,000$~animals,
so that the rest of the world has $r=100\,000$~animals then every year 
ten percent (a thousand animals) of those inside will leave the park, 
and every year one percent (a thousand) of those from the rest of
the world will enter the park.
It is stable, self-sustaining. 

Now imagine that we are trying to gradually build 
up the total world population of this species.
We can try, for instance, to have the world population grow at a rate
of $1\%$ per year.
This would make the population level over time ``stable'' in a way,
although it is a dynamic stability, in contrast to the static 
population of the $\lambda=1$ case.
The equation $\vec{v}_{n+1}=1.01\cdot\vec{v}_n=T\vec{v}_{n}$ leads to 
$((1.01 I-T)\vec{v}=\zero$, which gives this system. 
\begin{equation*}
  \begin{pmatrix}
     0.11   &-0.01  \\
     -0.10  &0.02
  \end{pmatrix}
  \colvec{p \\ r}=
  \colvec{0 \\ 0}
\end{equation*}
This matrix is nonsingular and so the only solution is $p=0$,
$r=0$.
Thus, there is no nontrivial initial population that 
we can establish at the park 
and expect that it will grow at the same rate as the rest of the world.

Knowing that an annual world population growth rate of $1\%$ forces an
unstable park population,
we can ask which growth rates there are that would 
allow an initial population for
the park that will be self-sustaining.
We consider $\lambda\vec{v}=T\vec{v}$ and solve for $\lambda$.
\begin{equation*}
  0=\begin{vmatrix}
    \lambda-.9  &.01  \\
    .10         &\lambda-.99
  \end{vmatrix}
  =(\lambda-.9)(\lambda-.99)-(.10)(.01)
  =\lambda^2-1.89\lambda+.89
\end{equation*}
A shortcut to factoring that quadratic is our knowledge that $\lambda=1$
is an eigenvalue of $T$, so the other eigenvalue is $.89$.
Thus there are two ways to have a stable park population (a population that
grows at the same rate as the population of the rest of the world, despite
the leaky park boundaries):~have a world population that is does not 
grow or shrink, and have a world population that shrinks by $11\%$ every year.
 
So this is one meaning of eigenvalues and eigenvectors\Dash they give a 
stable state for a system.
If the eigenvalue is $1$ then the system is static.
If the eigenvalue isn't $1$ then the system is either growing or
shrinking, but in a dynamically-stable way.





\begin{exercises}
  \item 
    What initial population for the park discussed above
    should be set up in the
    case where world populations are allowed to decline by $11\%$ every year?
  \item 
    What will happen to the population of the park in the event of
    a growth in world population of $1\%$ per year?
    Will it lag the world growth, or lead it?
    Assume that the inital park population is ten thousand, and the
    world population is one hunderd thousand, 
    and calculate over a ten year span.
  \item 
    The park discussed above is partially fenced so that now,
    every year, only $5\%$ of the animals from inside of the park leave (still,
    about $1\%$ of the animals from the outside 
    find their way in).
    Under what conditions can the park maintain a stable population now?
  \item 
    Suppose that a species of bird only lives in Canada, the United States,
    or in Mexico.
    Every year, $4\%$ of the Canadian birds travel to the US, and $1\%$ of them
    travel to Mexico.
    Every year, $6\%$ of the US birds travel to Canada, and $4\%$ go to Mexico.
    From Mexico, every year $10\%$ travel to the US, and $0\%$ go to Canada.
    \begin{exparts}
      \partsitem Give the transition matrix.
      \partsitem Is there a way for the three countries to have constant
         populations?
      \partsitem Find all stable situations.   
    \end{exparts}
\end{exercises}
\index{populations, stable|)}
\index{stable populations|)}
\endinput






