% Topic from _Linear Algebra_ Jim Hefferon
%  http://joshua.smcvt.edu/linalg.html
%  2012-Feb-12
\topic{Page Ranking}
\index{page ranking|(}

Imagine that you are trying to find the best book
on Linear Algebra.
You probably would try a web search engine such as Google.
These lists pages ranked by importance.
The ranking is, as Google's founders said, 
that a page is important if other important
pages link to it:
``a page can have a high PageRank if there are many pages that point
to it, or if there are some pages that point to it and have a high PageRank''
\cite{BrinPage}.
But isn't that circular\Dash how can they tell whether a page is
important without first deciding on the important pages?
With eigenvalues and eigenvectors.

We will present a simplified version of the Page Rank algorithm. 
We model the World Wide Web as a collection of pages connected by
links.
This diagram from \cite{Wills}
shows the pages as circles, and the links as arrows;
for instance, page~$p_1$ has a link to page~$p_2$. 
\begin{center}  % add a little vertical spacing; looked tight to me.
  \shortstack{\rule{0em}{1.5ex} \\ \includegraphics{ch5.9} \\ \rule{0em}{1.5ex}}
\end{center}

The key idea is that pages that should be highly ranked if they are
often cited by other pages.
% (In practice people have tried to game the ranking system by
% setting up link farms of many pages that all point to each other.
% But our model will be simple.)
That is, we raise the importance of a page~$p_i$
if it is linked-to from page~$p_j$.
The increment is the importance of the linking page~$p_j$
divided by how many out-links $a_j$ are on that page.
\begin{equation*}
  \mathcal{I}(p_i)=\sum_{\text{in-linking pages $p_j$}}  \frac{\mathcal{I}(p_j)}{a_j}
\end{equation*}
This matrix stores the information.
\begin{equation*}
  \begin{mat}
    0   &0  &1/3  &0   \\
    1   &0  &1/3  &0   \\
    0   &1  &0    &0 \\
    0   &0  &1/3  &0
  \end{mat}
\end{equation*}

The algorithm's inventors describe a way to think about the matrix~$H$. 
\begin{quotation}
PageRank can be thought of as a model of user behavior. 
We assume there is a `random surfer' who is
given a web page at random and keeps clicking on links, 
never hitting ``back'' \ldots 
% but eventually gets bored
% and starts on another random page. 
The probability that the random surfer visits a page is its PageRank.
\cite{BrinPage}
\end{quotation}
In the diagram, a surfer on page~$p_3$ has a probability $1/3$ of going 
next to each of the other pages. 

We will find vector $\vec{\mathcal{I}}$ whose components are the
importance rankings of each page $\mathcal{I}(p_i)$.
With this notation, 
our requirements for the page rank are: $H\vec{\mathcal{I}}=\vec{\mathcal{I}}$.
That is, we want an eigenvector of the matrix associated with the
eigenvalue~$\lambda=1$.

To that basic strategy we will add two refinements.
The first is the problem of page~$p_4$.
Many targets of links are 
\definend{dangling} or \definend{sink links},
without any outbound links.
The simplest thing is to imagine that when the surfer gets to a page like this
then they go to a next page entirely at random.
\begin{equation*}
  H=\begin{mat}
    0   &0  &1/3  &1/4   \\
    1   &0  &1/3  &1/4   \\
    0   &1  &0    &1/4 \\
    0   &0  &1/3  &1/4
  \end{mat}
\end{equation*}

This is \textit{Sage}'s calculation of the eigenvectors 
(slightly edited to fit on the page).
\begin{lstlisting}
sage: H=matrix([[0,0,1/3,1/4], [1,0,1/3,1/4], [0,1,0,1/4], [0,0,1/3,1/4]])   
sage: H.eigenvectors_right()                                                 
[(1, [
(1, 2, 9/4, 1)
], 1), (0, [
(0, 1, 3, -4)
], 1), (-0.3750000000000000? - 0.4389855730355308?*I, 
  [(1, -0.1250000000000000? + 1.316956719106593?*I, 
   -1.875000000000000? - 1.316956719106593?*I, 1)], 1), 
  (-0.3750000000000000? + 0.4389855730355308?*I, 
  [(1, -0.1250000000000000? - 1.316956719106593?*I, 
   -1.875000000000000? + 1.316956719106593?*I, 1)], 1)]
\end{lstlisting}
The eigenvector that \textit{Sage} gives 
associated with the eigenvalue~$\lambda=1$ is this.
\begin{equation*}
  \colvec{1 \\ 2 \\ 9/4 \\ 1}
\end{equation*}
Of course, there are many vectors in that eigenspace.
To get a page rank number we normalize to length one.
\begin{lstlisting}
sage: v=vector([1, 2, 9/4, 1])
sage: v/v.norm()
(4/177*sqrt(177), 8/177*sqrt(177), 3/59*sqrt(177), 4/177*sqrt(177))
sage: w=v/v.norm()
sage: w.n()
(0.300658411201132, 0.601316822402263, 0.676481425202546, 0.300658411201132)
\end{lstlisting}
So we rank the first and fourth pages as of equal 
importance.
We rank the second and third pages as much more important than those, and 
about equal in importance as each other. 

We'll add one more refinement.
We will allow the surfer to pick a new page at random 
even if they are not on a dangling page.
Let this happen with probability~$\alpha$.
\begin{equation*}
  G=\alpha\cdot\begin{mat}
    0   &0  &1/3  &1/4   \\
    1   &0  &1/3  &1/4   \\
    0   &1  &0    &1/4 \\
    0   &0  &1/3  &1/4
  \end{mat}
  +(1-\alpha)\cdot\begin{mat}
    1/4   &1/4  &1/4  &1/4   \\
    1/4   &1/4  &1/4  &1/4   \\
    1/4   &1/4  &1/4  &1/4  \\
    1/4   &1/4  &1/4  &1/4
  \end{mat}
\end{equation*}
This is the \definend{Google matrix}\index{Google matrix}\index{matrix!Google}.
In practice $\alpha$ is often set in between $0.85$ and~$0.99$.

Here are the ranks for the four pages with a couple of different values
of $\alpha$.
\begin{center}
  \begin{tabular}{r|cccc}
    \multicolumn{1}{c}{$\alpha$}
          &$0.85$  &$0.90$  &$0.95$ &$0.99$ \\ \hline
    $p_1$ &$0.325$ &$0.317$ &$0.309$  &$0.302$  \\
    $p_2$ &$0.602$ &$0.602$ &$0.602$  &$0.601$  \\
    $p_3$ &$0.652$ &$0.661$ &$0.669$  &$0.675$  \\
    $p_4$ &$0.325$ &$0.317$ &$0.309$  &$0.302$  
  \end{tabular}
\end{center}
% Numbers computed as:
% sage: H=matrix(QQ,[[0,0,1/3,1/4], [1,0,1/3,1/4], [0,1,0,1/4], [0,0,1/3,1/4]])
% sage: S=matrix(QQ,[[1/4,1/4,1/4,1/4], [1/4,1/4,1/4,1/4], [1/4,1/4,1/4,1/4], [1/4,1/4,1/4,1/4]]) 
% sage: alpha=0.95                       
% sage: G=alpha*H+(1-alpha)*S    
% sage: N=G-I  
% sage: 1200*N         
% [-1185.00000000000  15.0000000000000  395.000000000000  300.000000000000]
% [ 1155.00000000000 -1185.00000000000  395.000000000000  300.000000000000]
% [ 15.0000000000000  1155.00000000000 -1185.00000000000  300.000000000000]
% [ 15.0000000000000  15.0000000000000  395.000000000000 -900.000000000000]
% sage: M=matrix(QQ,[[-1185,15,395,300], [1155,-1185,395,300], [15,1155,-1185,300], [15,15,395,-900]])
% sage: M.echelon_form()         
% [         1          0          0         -1]
% [         0          1          0     -39/20]
% [         0          0          1 -3423/1580]
% [         0          0          0          0]
% sage: v=vector([1,39/20,3423/1580,1]) 
% sage: v/v.norm().n()   
% (0.308665053665036, 0.601896854646821, 0.668709163731278, 0.308665053665036)


\medskip
The details of the algorithms used by commercial search engines are 
secret, and no doubt have many refinements, and also change frequently.
But the inventors of Google were gracious enough to outline the basis for 
their work in \cite{BrinPage}.
Two additional excellent expositions are 
\cite{Wills} and
\cite{Austin}.

\begin{exercises}
  \item A square matrix is \definend{stochastic}\index{stochastic matrix}\index{matrix!stochastic}
    if the sum of the entries in each column is one.
    The Google matrix is computed by taking a combination
    $G=\alpha*H+(1-\alpha)*S$ of two stochastic matrices.
    Show that $G$ must be stochastic.
    \begin{answer}
      The sum of the entries in column~$j$ is 
      $\sum_i \alpha h_{i,j}+(1-\alpha)s_{i,j}
       =\sum_i \alpha h_{i,j}+\sum_i (1-\alpha)s_{i,j}
       =\alpha\sum_i \alpha h_{i,j} +(1-\alpha)\sum_i s_{i,j}
       =\alpha\cdot 1+(1-\alpha)\cdot 1$,
      which is one.
    \end{answer}
\end{exercises}
\index{page ranking|)}
\endinput


