\documentclass[11pt]{article}
\usepackage[margin=1in]{geometry}
\usepackage{../linalgjh}

\setlength{\parindent}{0em}
\begin{document}
\makebox[\linewidth]{\textbf{Homework, MA~213}\hspace*{4in}\textbf{Due 2014-Sep-22}}

\vspace*{3ex}
\textit{You may work with others to figure out how to do questions, 
and you are welcome to look for answers in the book, online, by talking
to someone who had the course before, etc.
However, you must write 
the answers on your own.
You must also show your work (you may, of course, 
quote any result from the book).}

\begin{enumerate}
\item Verify that each is a vector space by checking the conditions.
  \begin{enumerate}
  \item The collection of $\nbyn{2}$ matrices with $0$'s in the 
    upper right and lower left entries.
    \begin{equation*}
      \set{
        \begin{mat}
          a  &0  \\
          0  &b  
        \end{mat}
        \suchthat a,b\in\Re}
    \end{equation*}
  Condition~(1), closure under addition follows from this sum
  (take all scalars $a_i,b_j\in\Re$)
  \begin{equation*}
    \begin{mat}
      a_1  &0  \\
      0    &b_1
    \end{mat}
    +
    \begin{mat}
      a_2  &0  \\
      0    &b_2
    \end{mat}
    =
    \begin{mat}
      a_1+a_2  &0  \\
      0        &b_1+b_2
    \end{mat}
  \end{equation*}
  and the observation that the matrix on the right has $0$'s in the upper right 
  and the lower left.

  For the other closure condition, (6), closure under scalar multiplication,
  \begin{equation*}
    r\cdot
    \begin{mat}
      a  &0  \\
      0    &b
    \end{mat}
    =
    \begin{mat}
      ra  &0  \\
      0   &rb
    \end{mat}
  \end{equation*}
  observe that the matrix on the right has $0$'s in the upper right and 
  lower left.

  The other conditions are routine and won't be shown here.
  For example, commutativity of addition goes just as in the argument for
  general $\nbyn{2}$~matrices.
  \begin{equation*}
    \begin{mat}
      a_1  &0  \\
      0    &b_1
    \end{mat}
    +
    \begin{mat}
      a_2  &0  \\
      0    &b_2
    \end{mat}
    =
    \begin{mat}
      a_1+a_2  &0  \\
      0        &b_1+b_2
    \end{mat}
    =
    \begin{mat}
      a_2  &0  \\
      0    &b_2
    \end{mat}
    +
    \begin{mat}
      a_1  &0  \\
      0    &b_1
    \end{mat}
  \end{equation*}
  (In addition, we had a result that said that for subspaces, which this
  space is, we need only check closure.)

  \item The collection  $\set{a_0+a_1x+a_3x^3 \suchthat a_0,a_1,a_3\in\Re}$
     of cubic polynomials with no quadratic term.
  
  As with the prior item we focus on the two closure conditions.
  For closure under addition, where each $a_i,,b_j\in\Re$,
  the sum
  $(a_0+a_1x+a_3x^3)+(b_0+b_1x+b_3x^3)
  =(a_0+b_0)+(a_1+b_1)x+(a_2+b_2)x^3$ 
  gives a cubic polynomial with no quadratic term.

  For closure under scalar multiplication,
  $r\cdot(a_0+a_1x+a_3x^3)=(ra_0)+(ra_1)x+(ra_2)x^3$
  is a cubic polynomial with no quadratic term.

  As above, the other conditions are routine.
  For instance, here is associtivity under addition
  $[(a_0+a_1x+a_3x^3)+(b_0+b_1x+b_3x^3)]+(c_0+c_1x+c_3x^3)
   =(a_0+b_0+c_0)+(a_1+b_1+c_1)x+(a_3+b_3+c_3)x^3
   =(a_0+a_1x+a_3x^3)+[(b_0+b_1x+b_3x^3)+(c_0+c_1x+c_3x^3)]$.
  \end{enumerate}



\item Determine if each set is linearly independent (in the natural vector 
space).
  \begin{enumerate}
  \item $\set{\colvec{1 \\ 2 \\ 0}, 
              \colvec{-1 \\ 1 \\ 0}}$

  The natural vector space is $\Re^3$.
  We set up the equation
  \begin{equation*}
    \colvec{0 \\ 0 \\ 0}=c_1\colvec{1 \\ 2 \\ 0}
                         +c_2\colvec{-1 \\ 1 \\ 0}
  \end{equation*}
  and consider the resulting homogeneous system.
  \begin{equation*}
    \begin{amat}{2}
      1  &-1 &0  \\
      2  &1  &0  \\
      0  &0  &0
    \end{amat}
    \grstep{-2\rho_1+\rho_2}
    \begin{amat}{2}
      1  &-1 &0  \\
      0  &3  &0  \\
      0  &0  &0
    \end{amat} 
  \end{equation*}
  This has the unique solution, that $c_1=0$, $c_2=0$.
  So it is linearly independent.


  \item $\set{\rowvec{1 &3 &1}, \rowvec{-1 &4 &3}, \rowvec{-1 &11 &7}}$

  The natural vector space is the set of three-wide row vectors.
  The equation
  \begin{equation*}
    \rowvec{0 &0 &0}=c_1\rowvec{1 &3 &1}
                     +c_2\rowvec{-1 &4 &3}
                     +c_3\rowvec{-1 &11 &7}
  \end{equation*}
  gives rise to a linear system
  \begin{equation*}
    \begin{amat}{3}
      1  &-1  &-1  &0  \\
      3  &4   &11  &0  \\
      1  &3   &7   &0
    \end{amat}
    \grstep[-\rho_1+\rho_3]{-3\rho_1+\rho_2}
    \begin{amat}{3}
      1  &-1  &-1  &0  \\
      0  &7   &14  &0  \\
      0  &4   &8   &0
    \end{amat}
    \grstep{(4/7)\rho_2+\rho_3}
    \begin{amat}{3}
      1  &-1  &-1  &0  \\
      0  &7   &14  &0  \\
      0  &0   &0   &0
    \end{amat}
  \end{equation*}
  with infinitely many solutions, that is, more than just the trivial solution.
  \begin{equation*}
    \set{\colvec{c_1 \\ c_2 \\ c_3}
          =\colvec{-1 \\ -2 \\ 1}c_3
          \suchthat c_3\in\Re}
  \end{equation*}
  So the set is linearly dependent.
  One dependence comes from setting $c_3=2$, giving 
  $c_1=-2$ and $c_2=-4$.

  \item $\set{\begin{mat}
                5 &4 \\
                1 &2 
              \end{mat},
              \begin{mat}
                0 &0 \\
                0 &0
              \end{mat},
              \begin{mat}
                1 &0 \\
               -1 &4
              \end{mat}
           }$

  Without having to set up a system we can see that the second element
  of the set is a multiple of the first (namely, $0$ times the first). 
  \end{enumerate}

\item Is the vector in the span of the set?
  \begin{equation*}
    \colvec{1 \\ 0 \\ 3}
    \quad
    \set{\colvec{2 \\ 1 \\ -1},
         \colvec{1 \\ -1 \\ 1}}
  \end{equation*}

  The equation
  \begin{equation*}
    \colvec{1 \\ 0 \\ 3} = 
             c_1\colvec{2 \\ 1 \\ -1}
             +c_2\colvec{1 \\ -1 \\ 1}
  \end{equation*}
  gives rise to a linear system
  \begin{equation*}
    \begin{amat}{2}
      2  &1  &1  \\
      1  &-1 &0  \\
      -1 &1  &3
    \end{amat}
    \grstep[(1/2)\rho_1+\rho_3]{(-1/2)\rho_1+\rho_2}
    \begin{amat}{2}
      2  &1    &1  \\
      0  &-3/2 &-1/2  \\
      0  &0  &3
    \end{amat}
  \end{equation*}
  that has no solution, so the vector is not in the span.
\end{enumerate}
\end{document}


sage: load "../lab/gauss_method.sage"
sage: M = matrix(QQ, [[1,-1], [2,1], [0,0] ])
sage: v = vector(QQ, [0,0,0])
sage: M_prime = M.augment(v, subdivide=True)
sage: gauss_method(M_prime)
[ 1 -1| 0]
[ 2  1| 0]
[ 0  0| 0]
 take -2 times row 1 plus row 2
[ 1 -1| 0]
[ 0  3| 0]
[ 0  0| 0]

sage: M = matrix(QQ, [[1,-1,-1], [3,4,11], [1,3,7] ])
sage: v = vector(QQ, [0,0,0])
sage: M_prime = M.augment(v, subdivide=True)
sage: gauss_method(M_prime)
[ 1 -1 -1| 0]
[ 3  4 11| 0]
[ 1  3  7| 0]
 take -3 times row 1 plus row 2
 take -1 times row 1 plus row 3
[ 1 -1 -1| 0]
[ 0  7 14| 0]
[ 0  4  8| 0]
 take -4/7 times row 2 plus row 3
[ 1 -1 -1| 0]
[ 0  7 14| 0]
[ 0  0  0| 0]
age: eqns = [c1-c2-c3==0, 3*c1+4*c2+11*c3==0, c1+3*c2+7*c3==0]
sage: solve(eqns, c1, c2, c3)
[[c1 == -r1, c2 == -2*r1, c3 == r1]]

sage: M = matrix(QQ, [[2,1], [1,-1], [-1,1] ])
sage: var('c1,c2,c3,c4')
(c1, c2, c3, c4)
sage: v = vector(QQ, [1,0,3])
sage: M_prime = M.augment(v, subdivide=True)
sage: gauss_method(M_prime)
[ 2  1| 1]
[ 1 -1| 0]
[-1  1| 3]
 take -1/2 times row 1 plus row 2
 take 1/2 times row 1 plus row 3
[   2    1|   1]
[   0 -3/2|-1/2]
[   0  3/2| 7/2]
 take 1 times row 2 plus row 3
[   2    1|   1]
[   0 -3/2|-1/2]
[   0    0|   3]




