% Chapter 1, Topic from _Linear Algebra_ Jim Hefferon
%  http://joshua.smcvt.edu/linalg.html
%  2001-Jun-09
\topic{Analyzing Networks}
\index{networks|(}
The diagram below shows some of a car's electrical network.
The battery is on the left, drawn as stacked line segments. 
The wires are drawn as lines, shown straight and with sharp right angles
for neatness.
Each light is a circle enclosing a loop.
\begin{center}
  \includegraphics{ch1.46}
\end{center}
The designer of such a network needs to answer questions like: 
How much electricity flows 
when both the hi-beam headlights and the brake lights are on?
Below, we will use linear systems to analyze simpler versions of
electrical networks. 

For the analysis we need two facts about electricity 
and two facts about electrical networks.

The first fact about electricity is that a battery is like a pump:~it 
provides a force impelling the electricity to flow through the 
circuits connecting the battery's ends, if there are any such circuits.  
We say that the battery provides a \definend{potential}\index{potential}
to flow.
Of course, this network accomplishes its function when, as the electricity  
flows through a circuit, it goes through a light.
For instance, when the driver steps on the brake then the switch makes contact
and a circuit is formed on the left side of the
diagram, and the electrical current flowing through that circuit will
make the brake lights go on, warning drivers behind.

The second electrical fact is that in some 
kinds of network components 
the amount of flow is proportional to the force provided by the battery.
That is, for each such component there is a number,  
it's \definend{resistance},\index{resistance}
such that the potential is equal to the flow times the resistance.
The units of measurement are: potential is described in \definend{volts},
the rate of flow is in \definend{amperes},
and resistance to the flow is in \definend{ohms}.
These units are defined so that
$\mbox{volts}=\mbox{amperes}\cdot\mbox{ohms}$.

Components with this property,
that the voltage-amperage response curve is a line through the origin,
are called \definend{resistors}.\index{resistor}
(Light bulbs such as the ones shown above are not this kind of component,
because their ohmage changes as they heat up.)
For example, if a resistor measures $2$~ohms 
then wiring it to a $12$~volt battery 
results in a flow of $6$~amperes.
Conversely, if we have flow of electrical current of
$2$~amperes through it then there must be 
a $4$~volt potential difference  
between it's ends. 
This is the \definend{voltage drop}\index{voltage drop} across the 
resistor.
One way to think of a electrical circuits like the one above
is that the battery provides a voltage rise while the other components 
are voltage drops.

The two facts that we need about networks are Kirchhoff's
Laws.\index{Kirchhoff's Laws}\index{networks!Kirchhoff's Laws} 
\begin{itemize}
  \item[] \textit{Current Law.} For any point in a network, the flow in
     equals the flow out.
  \item[] \textit{Voltage Law.} Around any circuit the total drop equals 
     the total rise.
\end{itemize}
In the above network there is only one voltage rise, at the battery, but
some networks have more than one.

For a start we can consider the network below.
It has a battery that provides the potential to flow 
and three resistors
(resistors are drawn as zig-zags).
When components are wired one after another, as here,
they are said to be in \definend{series}.\index{circuits!series}
\begin{center}
  \includegraphics{ch1.35}
\end{center}
By Kirchhoff's Voltage Law, because the voltage rise is
$20$~volts, the total voltage drop must also be $20$~volts.
Since the resistance from start to finish is
$10$~ohms (the resistance of the wires is negligible),
we get that the current is $(20/10)=2$~amperes. 
Now, by Kirchhoff's Current Law, there are $2$~amperes through
each resistor.
(And therefore the voltage drops are: 
$4$~volts across the $2$~oh m resistor,
$10$~volts across the $5$~ohm resistor, 
and $6$~volts across the $3$~ohm resistor.)

The prior network is so simple that we didn't use a linear system, but
the next network is more complicated.
In this one, the resistors are in \definend{parallel}.\index{circuits!parallel}
This network is more like the car lighting diagram shown earlier.
\begin{center}
  \includegraphics{ch1.36}
\end{center}
We begin by labeling the branches, shown below.
Let the current through the left branch of the parallel portion
be $i_1$ and that through
the right branch be $i_2$, and also
let the current through the battery be $i_0$.
(We are following Kirchoff's Current Law; for instance, all points in the 
right branch have the same current, which we call $i_2$.
Note that we don't need to know 
the actual direction of flow\Dash if current flows in the direction 
opposite to our arrow 
then we will simply get a negative number in the solution.)
\begin{center}
  \includegraphics{ch1.37}
\end{center}
The Current Law, applied to the point in the upper right
where the flow $i_0$ meets $i_1$ and $i_2$, gives that $i_0=i_1+i_2$. 
Applied to the lower right it gives $i_1+i_2=i_0$.  
In the circuit that loops out of the top of the battery, 
down the left branch of the
parallel portion, and back into the bottom of the battery, the voltage
rise is $20$ while the voltage drop is $i_1\cdot 12$, so
the Voltage Law gives that $12i_1=20$.
Similarly, the circuit from the battery to the right branch and back to the
battery gives that $8i_2=20$.
And, in the circuit that simply loops around in 
the left and right branches of the parallel portion 
(arbitrarily taken clockwise), 
there is a voltage rise of $0$ and a voltage drop of $8i_2-12i_1$
so the Voltage Law gives that $8i_2-12i_1=0$.
\begin{displaymath}
  \begin{linsys}{3}
   i_0&- &i_1    &-  &i_2   &=  &0 \\
  -i_0&+ &i_1    &+  &i_2   &=  &0  \\
      &  &12i_1  &   &      &=  &20  \\
      &  &       &   &8i_2  &=  &20  \\
      &  &-12i_1 &+  &8i_2  &=  &0  
  \end{linsys}
\end{displaymath}
The solution
is $i_0=25/6$, $i_1=5/3$, and $i_2=5/2$, all in amperes.
(Incidentally, this illustrates that redundant equations do indeed arise in
practice.) 

Kirchhoff's laws can be used to
establish the electrical properties of networks of great complexity.
The next diagram shows
five resistors, wired in 
a \definend{series-parallel}\index{circuits!series-parallel} 
way.
\begin{center}
  \includegraphics{ch1.38}
\end{center}
This network is a \definend{Wheatstone bridge}\index{Wheatstone bridge} 
(see \nearbyexercise{exer:WheatstoneBr}).
To analyze it, we can place the arrows in this way.
\begin{center}
  \includegraphics{ch1.39}
\end{center}
Kirchoff's Current Law, applied to the
top node, the left node, the right node, and the bottom node gives
these.
\begin{align*}
   i_0     &=  i_1+i_2  \\
   i_1     &=  i_3+i_5  \\
   i_2+i_5 &=  i_4      \\
   i_3+i_4 &=  i_0
\end{align*} 
Kirchhoff's Voltage Law,
applied to the inside loop (the $i_0$ to~$i_1$ to~$i_3$ to~$i_0$ loop), 
the outside loop, 
and the upper loop not involving the battery, gives these.  
\begin{align*}
      5i_1+10i_3  &= 10   \\
      2i_2+4i_4   &= 10   \\
      5i_1+50i_5-2i_2  &= 0     
\end{align*} 
Those suffice to determine the solution 
$i_0=7/3$, $i_1=2/3$, $i_2=5/3$, 
$i_3=2/3$, $i_4=5/3$, and $i_5=0$. 

Networks of other kinds, not just electrical ones, can also be
analyzed in this way.
For instance, networks of streets are given in the exercises.
 
\begin{exercises}
  \item[] \textit{Many of the systems for these problems
      are mostly easily solved on a computer.}
  \item 
    Calculate the amperages in each part of each network.
    \begin{exparts}
      \partsitem This is a simple network.
          \begin{center}
             \includegraphics{ch1.40}
        \end{center}
      \partsitem Compare this one with the parallel case discussed above.
          \begin{center}
             \includegraphics{ch1.41}
        \end{center}
      \partsitem This is a reasonably complicated network.
          \begin{center}
           \includegraphics{ch1.42}
        \end{center}
    \end{exparts}
    \begin{answer}
      \begin{exparts}
        \partsitem The total resistance is $7$~ohms.
          With a $9$~volt potential, the flow will be $9/7$~amperes.
          Incidentally, the voltage drops will then be:~$27/7$~volts
          across the $3$~ohm resistor, and $18/7$~volts across each of
          the two $2$~ohm resistors.        
        \partsitem One way to do this network is to note that the $2$~ohm
          resistor on the left has a voltage drop across it of $9$~volts
          (and hence the flow through it is $9/2$~amperes), and the 
          remaining portion on the right also has a voltage drop of 
          $9$~volts, and so is analyzed as in the prior item.

          We can also use linear systems.
          \begin{center}
            \includegraphics{ch1.48}
          \end{center}
          Using the variables from the diagram we get a linear system
          \begin{equation*}
            \begin{linsys}{4}
              i_0  &- &i_1  &- &i_2  &  &    &= &0  \\
                   &  &i_1  &+ &i_2  &- &i_3 &= &0  \\
                   &  &2i_1 &  &     &  &    &= &9  \\  
                   &  &     &  &7i_2 &  &    &= &9       
            \end{linsys}
          \end{equation*}
          which yields the unique solution $i_1=81/14$, $i_1=9/2$, $i_2=9/7$,
          and $i_3=81/14$.

          Of course, the first and second paragraphs yield the same answer.
          Esentially, in the first paragraph we solved the linear system 
          by a method less systematic than Gauss' method, solving for some
          of the variables and then substituting.  
        \partsitem
          Using these variables
          \begin{center}
            \includegraphics{ch1.49}
          \end{center}
          one linear system that suffices to yield a unique solution is this.
          \begin{equation*}
            \begin{linsys}{7}
              i_0  &- &i_1  &- &i_2  &  &    &  &    & &    & &    &= &0  \\
                   &  &     &  &i_2  &- &i_3 &- &i_4 & &    & &    &= &0  \\
                   &  &     &  &     &  &i_3 &+ &i_4 &-&i_5 & &    &= &0  \\  
                   &  &i_1  &  &     &  &    &  &    &+&i_5 &-&i_6 &= &0  \\  
                   &  &3i_1 &  &     &  &    &  &    & &    & &    &= &9  \\  
                   &  &     &  &3i_2 &  &    &+ &2i_4&+&2i_5& &    &= &9  \\  
                   &  &     &  &3i_2 &+ &9i_3&  &    &+&2i_5& &    &= &9  
            \end{linsys}
          \end{equation*}
          (The last three equations come from the circuit involving
            $i_0$-$i_1$-$i_6$, 
            the circuit involving $i_0$-$i_2$-$i_4$-$i_5$-$i_6$, 
            and the circuit with $i_0$-$i_2$-$i_3$-$i_5$-$i_6$.)
           Octave gives
            $i_0=4.35616$, $i_1=3.00000$, $i_2=1.35616$,
            $i_3=0.24658$, $i_4=1.10959$, $i_5=1.35616$, $i_6=4.35616$.
      \end{exparts}
    \end{answer}
  \item 
    In the first network that we analyzed, with the three resistors  
    in series, we just added to get
    that they acted together like a single resistor of $10$~ohms.
    We can do a similar thing for parallel circuits. 
    In the second circuit analyzed,
    \begin{center}
      \includegraphics{ch1.36}
    \end{center}
    the electric current through the battery is $25/6$~amperes.
    Thus, the parallel portion is 
    \definend{equivalent}\index{resistance:equivalent}
    to a single resistor of 
    $20/(25/6)=4.8$~ohms. 
    \begin{exparts}
      \partsitem What is the equivalent resistance if we change
         the $12$~ohm resistor to $5$~ohms?
      \partsitem What is the equivalent resistance if the two are each
         $8$~ohms?
      \partsitem Find the formula for the equivalent resistance if
         the two resistors in parallel are $r_1$~ohms and $r_2$~ohms.
    \end{exparts}
    \begin{answer}
      \begin{exparts}
        \partsitem 
          Using the variables from the earlier analysis,
          \begin{displaymath}
            \begin{linsys}{3}
              i_0&- &i_1    &-  &i_2   &=  &0 \\
             -i_0&+ &i_1    &+  &i_2   &=  &0  \\
                 &  &5i_1   &   &      &=  &20  \\
                 &  &       &   &8i_2  &=  &20  \\
                 &  &-5i_1  &+  &8i_2  &=  &0  
            \end{linsys}
          \end{displaymath}
          The current flowing in each branch is then
          is $i_2=20/8=2.5$, $i_1=20/5=4$, and $i_0=13/2=6.5$, all in amperes.
          Thus the parallel portion is acting like a single resistor
          of size $20/(13/2)\approx 3.08$~ohms.
        \partsitem 
          A similar analysis gives that
          is $i_2=i_1=20/8=4$ and $i_0=40/8=5$~ amperes.
          The equivalent resistance is $20/5=4$~ohms.
        \partsitem 
          Another analysis like the prior ones gives 
          is $i_2=20/r_2$, $i_1=20/r_1$, 
          and $i_0=20(r_1+r_2)/(r_1r_2)$, all in amperes.
          So the parallel portion is acting like a single resistor of
          size $20/i_1=r_1r_2/(r_1+r_2)$~ohms.
          (This equation is often stated as:~the equivalent 
          resistance~$r$ satisfies $1/r=(1/r_1)+(1/r_2)$.)
      \end{exparts}
    \end{answer}
  \item 
    For the car dashboard example that opens this Topic, solve
    for these amperages
    (assume that all resistances are $2$~ohms).
    \begin{exparts}
     \partsitem If the driver is stepping on the brakes, so the
       brake lights are on, and no other circuit is closed.
     \partsitem If the hi-beam headlights and the brake lights are on.
%      \partsitem If the hi-beam headlights, brake lights, 
%        and dome light are all on.
%        (Perhaps the driver, realizing that the dome light is on and so
%        that the door must be open, has stepped on the brake!)
    \end{exparts}
    \begin{answer}
      \begin{exparts}
        \partsitem The circuit looks like this.
          \begin{center}
            \includegraphics{ch1.52}
          \end{center}
        \partsitem The circuit looks like this.
          \begin{center}
            \includegraphics{ch1.53}
          \end{center}
      \end{exparts}
    \end{answer}
\item \label{exer:WheatstoneBr} 
   Show that, in this Wheatstone Bridge, 
   \begin{center}
     \includegraphics{ch1.47}
   \end{center}
   $r_2/r_1$ equals $r_4/r_3$ if and only if the current
   flowing through $r_g$ is zero.
   (The way that this device is used in practice is that an unknown
   resistance at $r_4$ is compared to the other three 
   $r_1$, $r_2$, and $r_3$.
   At $r_g$ is placed a meter that shows the current.
   The three resistances $r_1$, $r_2$, and $r_3$ are varied\Dash typically
   they each have a calibrated knob\Dash until the 
   current in the middle reads $0$,
   and then the above equation gives the value of $r_4$.)
   \begin{answer}
     Not yet done.                             
   \end{answer}
   %  \item Prove Th\`evenin's Theorem.
\item[]\textit{There are networks other than electrical ones, and 
              we can ask how well Kirchoff's laws apply to them.
              The remaining questions consider an extension to 
              networks of streets.}
\item 
    Consider this traffic circle.
    \begin{center}
      \includegraphics{ch1.43}
     \end{center}
     This is the traffic volume, in units of cars per five minutes.    
     \begin{center}
       \begin{tabular}{r|ccc}
        \multicolumn{1}{c}{\ } % get rid of vert bar
                            &\textit{North}  
                            &\textit{Pier}  
                            &\textit{Main}  \\
         \cline{2-4}
         \begin{tabular}{@{}r@{}} \textit{into} \\ \textit{out of} \end{tabular}
         &\begin{tabular}{@{}r@{}} $100$ \\ $75$ \end{tabular}
         &\begin{tabular}{@{}r@{}} $150$ \\ $150$ \end{tabular}
         &\begin{tabular}{@{}r@{}} $25$ \\  $50$ \end{tabular}
         % rows:
         % \textit{into}      &$100$           &$150$          &$25$      \\
         % \textit{out of}    &$75$            &$150$          &$50$       
       \end{tabular}
    \end{center}
    We can set up equations to model how the traffic flows.   
    \begin{exparts}
      \partsitem 
         Adapt Kirchoff's Current Law to this circumstance.
         Is it a reasonable modelling assumption?
      \partsitem 
         Label the three between-road arcs in the circle with a variable.
         Using the (adapted) Current Law,
         for each of the three in-out intersections state an equation
         describing the traffic flow at that node.
      \partsitem 
         Solve that system.
      \partsitem
         Interpret your solution.
      \partsitem
         Restate the Voltage Law for this circumstance.
         How reasonable is it?
    \end{exparts}
    \begin{answer}
      \begin{exparts}
        \partsitem
           An adaptation is:~in any intersection the flow in equals the 
           flow out.
           It does seem reasonable in this case, unless cars are stuck at
           an intersection for a long time.
        \partsitem
           We can label the flow in this way.
           \begin{center}
             \includegraphics{ch1.44}
           \end{center}
           Because $50$ cars leave via Main while $25$~cars enter, 
           $i_1-25=i_2$.
           Similarly Pier's in/out balance means that $i_2=i_3$ and
           North gives $i_3+25=i_1$. 
           We have this system.
           \begin{equation*}
             \begin{linsys}{3}
               i_1  &-  &i_2  &   &     &=  &25  \\
                        &i_2  &-  &i_3  &=  &0   \\
              -i_1  &   &     &+  &i_3  &=  &-25
             \end{linsys}
           \end{equation*}
        \partsitem 
           The row operations $\rho_1+\rho_2$ and $rho_2+\rho_3$ lead
           to the conclusion that there are infinitely many solutions.
           With $i_3$ as the parameter,
           \begin{equation*}
             \set{\colvec[c]{25+i_3  \\ i_3  \\i_3} \suchthat i_3\in\Re}
           \end{equation*}
           of course, since the problem is stated in number of cars, we
           might restrict $i_3$ to be a natural number.
        \partsitem
           If we picture an initially-empty circle with the given input/output
           behavior, we can superimpose a $z_3$-many cars circling endlessly
           to get a new solution.
        \partsitem
           A suitable restatement might be:~the number of cars entering the 
           circle must equal the number of cars leaving.
           The reasonableness of this one is not as clear.
           Over the five minute time period it could easily work out that
           a half dozen more cars entered than left, 
           although the into/out of table in the problem statement 
           does have that this property is satisfied.
           In any event it is of no help in getting a unique solution
           since for that we need to know the number of cars circling
           endlessly.
      \end{exparts}
    \end{answer}
  \item 
     This is a network of streets. 
          \begin{center}
            \includegraphics{ch1.44}
             \end{center}
         The hourly flow of cars into this network's
         entrances, and out of its exits can be observed.
          \begin{center}
            \begin{tabular}{r|ccccc}
               \multicolumn{1}{c}{}  
                 &\textit{east Winooski}  
                 &\textit{west Winooski}  
                 &\textit{Willow}  
                 &\textit{Jay} 
                 &\textit{Shelburne} \\ 
                 \cline{2-6}
              \begin{tabular}{@{}r@{}} \textit{into} \\ \textit{out of} \end{tabular} 
              &\begin{tabular}{@{}r@{}} $80$ \\ $30$ \end{tabular} 
              &\begin{tabular}{@{}r@{}} $50$ \\ $5$ \end{tabular} 
              &\begin{tabular}{@{}r@{}} $65$ \\ $70$ \end{tabular} 
              &\begin{tabular}{@{}r@{}} -- \\ $55$ \end{tabular} 
              &\begin{tabular}{@{}r@{}} $40$ \\ $75$ \end{tabular} 
              % rows:
              % \textit{into}      &80    &50    &65     &--    &40      \\
              % \textit{out of}    &30    &5     &70     &55    &75      
            \end{tabular}
          \end{center}
          (Note that to reach Jay a
          car must enter the network via some other road first, which is why
          there is no `into Jay' entry in the table.
          Note also that over a long period of time, 
          the total in must approximately equal the total 
          out, which is why both rows add to $235$~cars.)
          Once inside the network, the traffic may flow in different
          ways, perhaps filling Willow and leaving Jay 
          mostly empty, or perhaps flowing in some other way.
          Kirchhoff's Laws give the limits on that freedom.
    \begin{exparts}
       \partsitem Determine the restrictions on the flow inside this network 
          of streets by setting
          up a variable for each block, establishing the equations, 
          and solving them.
          Notice that some streets are one-way only.
          (\textit{Hint:}~this will not yield a unique solution, since traffic
          can flow through this network in various ways;
          you should get at least one free variable.)
       \partsitem Suppose that some construction is proposed for
         Winooski Avenue East between Willow and Jay, 
         so traffic on that block will be reduced.
         What is the least amount of traffic flow that can be
         allowed on that block without disrupting the  
         hourly flow into and out of the network?
    \end{exparts}
    \begin{answer}
      \begin{exparts}
        \partsitem Here is a variable for each unknown block; each known
          block has the flow shown.
          \begin{center}
            \includegraphics{ch1.51}          
          \end{center}
          We apply Kirchoff's principle that the flow into the intersection
          of Willow and Shelburne must equal the flow out to get
          $i_1+25=i_2+125$.
          Doing the intersections from right to left and top to bottom
          gives these equations.
          \begin{equation*}
            \begin{linsys}{7}
              i_1 &- &i_2 &  &    &  &    &  &    &  &    &  &    &= &10  \\
             -i_1 &  &    &+ &i_3 &  &    &  &    &  &    &  &    &= &15   \\
                  &  &i_2 &  &    &+ &i_4 &  &    &  &    &  &    &= &5   \\
                  &  &    &  &-i_3&- &i_4 &  &    &+ &i_6 &  &    &= &-50 \\
                  &  &    &  &    &  &    &  &i_5 &  &    &- &i_7 &= &-10 \\
                  &  &    &  &    &  &    &  &    &  &-i_6&+ &i_7 &= &30 
            \end{linsys}
          \end{equation*}
          The row operation $\rho_1+\rho_2$ followed by $\rho_2+\rho_3$
          then $\rho_3+\rho_4$ and $\rho_4+\rho_5$ and finally $\rho_5+\rho_6$
          result in this system.
          \begin{equation*}
            \begin{linsys}{7}
              i_1 &- &i_2 &  &    &  &    &  &    &  &    &  &    &= &10  \\
                  &  &-i_2&+ &i_3 &  &    &  &    &  &    &  &    &= &25   \\
                  &  &    &  &i_3 &+ &i_4 &- &i_5 &  &    &  &    &= &30  \\
                  &  &    &  &    &  &    &  &-i_5&+ &i_6 &  &    &= &-20  \\
                  &  &    &  &    &  &    &  &    &  &-i_6&+ &i_7 &= &-30 \\
                  &  &    &  &    &  &    &  &    &  &    &  &0   &= &0   
            \end{linsys}
          \end{equation*}
          Since the free variables are $i_4$ and $i_7$ we take them as 
          parameters.
          \begin{equation*}
          \begin{split}
            i_6  &=  i_7-30  \\
            i_5  &=  i_6+20=(i_7-30)+20=i_7-10 \\
            i_3  &=  -i_4+i_5+30=-i_4+(i_7-10)+30=-i_4+i_7+20 \\
            i_2  &=  i_3-25=(-i_4+i_7+20)-25=-i_4+i_7-5 \\
            i_1  &=  i_2+10=(-i_4+i_7-5)+10=-i_4+i_7+5
          \end{split}
          \tag{}\end{equation*}
          Obviously $i_4$ and $i_7$ have to be positive, and in fact
          the first equation shows that $i_7$ must be at least $30$.
          If we start with $i_7$, then the $i_2$~equation shows that
          $0\leq i_4\leq i_7-5$.
        \partsitem We cannot take $i_7$ to be zero or else $i_6$ will
          be negative (this would mean cars going the wrong way on the
          one-way street Jay).
          We can, however, take $i_7$ to be as small as $30$, and then 
          there are many suitable $i_4$'s.
          For instance, the solution
          \begin{equation*}
            (i_1,i_2,i_3,i_4,i_5,i_6,i_7)
            =
            (35,25,50,0,20,0,30)
          \end{equation*}
          results from choosing $i_4=0$.
      \end{exparts}
    \end{answer}
\end{exercises}
\index{networks|)}
\endinput
